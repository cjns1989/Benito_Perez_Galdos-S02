\PassOptionsToPackage{unicode=true}{hyperref} % options for packages loaded elsewhere
\PassOptionsToPackage{hyphens}{url}
%
\documentclass[oneside,8pt,spanish,]{extbook} % cjns1989 - 27112019 - added the oneside option: so that the text jumps left & right when reading on a tablet/ereader
\usepackage{lmodern}
\usepackage{amssymb,amsmath}
\usepackage{ifxetex,ifluatex}
\usepackage{fixltx2e} % provides \textsubscript
\ifnum 0\ifxetex 1\fi\ifluatex 1\fi=0 % if pdftex
  \usepackage[T1]{fontenc}
  \usepackage[utf8]{inputenc}
  \usepackage{textcomp} % provides euro and other symbols
\else % if luatex or xelatex
  \usepackage{unicode-math}
  \defaultfontfeatures{Ligatures=TeX,Scale=MatchLowercase}
%   \setmainfont[]{EBGaramond-Regular}
    \setmainfont[Numbers={OldStyle,Proportional}]{EBGaramond-Regular}      % cjns1989 - 20191129 - old style numbers 
\fi
% use upquote if available, for straight quotes in verbatim environments
\IfFileExists{upquote.sty}{\usepackage{upquote}}{}
% use microtype if available
\IfFileExists{microtype.sty}{%
\usepackage[]{microtype}
\UseMicrotypeSet[protrusion]{basicmath} % disable protrusion for tt fonts
}{}
\usepackage{hyperref}
\hypersetup{
            pdftitle={Los Apostólicos},
            pdfauthor={Benito Pérez Galdós},
            pdfborder={0 0 0},
            breaklinks=true}
\urlstyle{same}  % don't use monospace font for urls
\usepackage[papersize={4.80 in, 6.40  in},left=.5 in,right=.5 in]{geometry}
\setlength{\emergencystretch}{3em}  % prevent overfull lines
\providecommand{\tightlist}{%
  \setlength{\itemsep}{0pt}\setlength{\parskip}{0pt}}
\setcounter{secnumdepth}{0}

% set default figure placement to htbp
\makeatletter
\def\fps@figure{htbp}
\makeatother

\usepackage{ragged2e}
\usepackage{epigraph}
\renewcommand{\textflush}{flushepinormal}

\usepackage{indentfirst}

\usepackage{fancyhdr}
\pagestyle{fancy}
\fancyhf{}
\fancyhead[R]{\thepage}
\renewcommand{\headrulewidth}{0pt}
\usepackage{quoting}
\usepackage{ragged2e}

\newlength\mylen
\settowidth\mylen{...................}

\usepackage{stackengine}
\usepackage{graphicx}
\def\asterism{\par\vspace{1em}{\centering\scalebox{.9}{%
  \stackon[-0.6pt]{\bfseries*~*}{\bfseries*}}\par}\vspace{.8em}\par}

 \usepackage{titlesec}
 \titleformat{\chapter}[display]
  {\normalfont\bfseries\filcenter}{}{0pt}{\Large}
 \titleformat{\section}[display]
  {\normalfont\bfseries\filcenter}{}{0pt}{\Large}
 \titleformat{\subsection}[display]
  {\normalfont\bfseries\filcenter}{}{0pt}{\Large}

\setcounter{secnumdepth}{1}
\ifnum 0\ifxetex 1\fi\ifluatex 1\fi=0 % if pdftex
  \usepackage[shorthands=off,main=spanish]{babel}
\else
  % load polyglossia as late as possible as it *could* call bidi if RTL lang (e.g. Hebrew or Arabic)
%   \usepackage{polyglossia}
%   \setmainlanguage[]{spanish}
%   \usepackage[french]{babel} % cjns1989 - 1.43 version of polyglossia on this system does not allow disabling the autospacing feature
\fi

\title{Los Apostólicos}
\author{Benito Pérez Galdós}
\date{}

\begin{document}
\maketitle

\hypertarget{i}{%
\chapter{I}\label{i}}

Tradiciones fielmente conservadas y ciertos documentos comerciales, que
podrían llamarse el Archivo Histórico de la familia de Cordero,
convienen en que Doña Robustiana de los Toros de Guisando, esposa del
héroe de Boteros, falleció el 11 de Diciembre de 1826. ¿Fue peritonitis,
pulmonía matritense o tabardillo pintado lo que arrancó del seno de su
amante familia y de las delicias de este valle de lágrimas a tan digna y
ejemplar señora? Este es un terreno oscuro en el cual no ha podido
penetrar nuestra investigación ni aun acompañada de todas las luces de
la crítica.

Esa pícara Historia, que en tratándose de los reyes y príncipes, no hay
cosa trivial ni hecho insignificante que no saque a relucir, no ha
tenido una palabra sola para la estupenda hazaña de Boteros, ni tampoco
para la ocasión lastimosa en que el héroe se quedó viudo con cinco
hijos, de los cuales los dos más pequeñuelos vinieron al mundo después
que el giro de los acontecimientos nos obligó a perder de vista a la
familia Cordero.

Cuando murió la señora, Juanito Jacobo (a quien se dio este nombre en
memoria de cierto filósofo que no es necesario nombrar) tenía dos meses
no bien cumplidos, y por su insaciable apetito así como su berrear
constante declaraba la raza y poderoso abolengo de Toros de Guisando.
Sus bruscas manotadas y la fiereza con que se llevaba los puños a la
boca, ávido de mamarse a sí mismo por no poder secar un par de amas cada
mes, señales eran de vigor e independencia, por lo que D. Benigno, sin
dejar de agradecer a Dios las buenas dotes vitales que había dado a su
criatura, pasaba la pena negra en su triste papel de viudo, y ora
valiéndose de cabras y biberones, cuando faltaban las nodrizas, ora
buscando por Puerta Cerrada y ambas Cavas lo mejor que viniera de
Asturias y la Alcarria en el maleado género de \emph{amas para casa de
los padres}; ya desechando a esta por enferma y a aquella por desabrida,
taimada y ladrona, ya suplicando a tal cual señora de su conocimiento
que diera una mamada al muchacho cuando le faltaba el pecho mercenario,
era un infeliz esclavo de los deberes paternales y perdía el seso, el
humor, la salud, el sueño, si bien jamás perdía la paciencia.

En las frías y largas noches ¿quién sino él habría podido echarse en
brazos la infantil carga y acallar los berridos con paseos, arrullos,
halagos y cantorrios? ¿Quién sino él habría soportado las largas
vigilias y el cuneo incesante y otros muchos menesteres que no son para
contados? Pero D. Benigno tenía un axioma que en todas estas ocasiones
penosas le servía de grandísimo consuelo, y recordándolo en los momentos
de mayor sofoco, decía:

---El cumplimiento estricto del deber en las diferentes circunstancias
de la existencia es lo que hace al hombre buen cristiano, buen
ciudadano, buen padre de familia. El rodar de la vida nos pone en
situaciones muy diversas exigiéndonos ahora esta virtud, más tarde
aquella. Es preciso que nos adaptemos hasta donde sea posible a esas
situaciones y casos distintos, respondiendo según podamos a lo que la
Sociedad y el Autor de todas las cosas exigen de nosotros. A veces nos
piden heroísmo que es la virtud reconcentrada en un punto y momento; a
veces paciencia que es el heroísmo diluido en larga serie de instantes.

Después solía recordar que Catón el Censor abandonaba los negocios más
arduos del gobierno de Roma para presenciar y dirigir la lactancia, el
lavatorio y los cambios de vestido de su hijo, y que el mismo Augusto,
señor y amo del mundo, hacía otro tanto con sus nietecillos. Con esto
recibía D. Benigno gran alivio, y después de leer de cabo a rabo el
libro del \emph{Emilio} que trata de las nodrizas, de la buena leche, de
los gorritos y de todo lo concerniente a la primera crianza, contemplaba
lleno de orgullo a su querido retoño, repitiendo las palabras del gran
ginebrino: «así como hay hombres que no salen jamás de la infancia, hay
otros de quienes se puede decir que nunca han entrado en ella y son
hombres desde que nacen».

Con estos trabajos, que hacía más llevaderos la satisfacción de un noble
deber cumplido, iba pasando el tiempo. El primer aniversario del
fallecimiento de su mujer renovó en Cordero todas las hondas tristezas
de aquel luctuoso día, y negándose al trivial consuelo de la tertulia de
amigos y parroquianos, cerró la tienda y se retiró a su alcoba, donde
las memorias de la difunta parecían tomar realidad y figura sensible
para acompañarle. El segundo aniversario halló bastante cambiadas
personas y cosas: la tienda había crecido, los niños también. Juanito
Jacobo, ni un ápice mermado en su constitución becerril, atronaba la
casa con sus gritos y daba buena cuenta de todo objeto frágil que en su
mano caía. En el alma de D. Benigno iba declinando mansamente el dolor
cual noche que se recoge expulsada poco a poco por la claridad del nuevo
día.

En el tercer aniversario (11 de Diciembre de 1829) el cambio era mucho
mayor y D. Benigno, restablecido en la majestad de su carácter ameno,
sencillo, bondadoso y lleno de discreción y prudencia, parecía un
soberano que torna al solio heredado después de lastimosos destierros y
trapisondas. No dejaron, sin embargo, de asaltarle en la mañanita de
aquel día pensamientos tristes; pero al volver de la misa conmemorativa
que había encargado, según costumbre de todo aniversario, y oído
devotamente en Santa Cruz, viósele en su natural humor cotidiano,
llenando la tienda con su activa mirada y su atención diligente. Después
de cerrar la vidriera para que no se enfriara el local, palpó con cierta
suavidad cariñosa las cajas que contenían el \emph{género}; hojeó el
libro de cuentas, pasó la vista por el \emph{Diario} que acababan de
traer; dio órdenes al mancebo para llevar a dos o tres casas algunas
compras hechas la noche anterior; cortó un par de plumas con el
minucioso esmero que la gente de los buenos tiempos ponía en operación
tan delicada, y habría puesto sobre el papel algunos renglones de
aquella hermosa letra redonda que ya sólo se ve en los archivos, si no
le sorprendieran de súbito sus niños, que salieron de la trastienda
cartera en cinto, los libros en correa, la pizarra a la espalda y el
gorrete en la mano para pedir a Padre la bendición.

---¡Cómo!---exclamó D. Benigno, entregando su mano a los labios y a los
húmedos hociquillos de los Corderos.---¿No os he dicho que hoy no hay
escuela?\ldots{} Es verdad que no me había acordado de decíroslo; pues
ya había pensado que en este día, que para nosotros no es alegre y para
toda España será, según dicen, un día felicísimo, todos los buenos
madrileños deben ir a batir palmas delante de ese astro que nos traen de
Nápoles, de esa reina tan ponderada, tan trompeteada y puesta en los
mismos cuernos de la luna, como si con ella nos vinieran acá mil dichas
y tesoros\ldots{} hablo también con usted, apreciable \emph{Hormiga},
pase usted\ldots{} no me molesta ahora ni en ningún momento.

Dirigíase Don Benigno a una mujer que se había presentado en la puerta
de la trastienda, deteniéndose en ella con timidez. Los chicos, luego
que oyeron el anuncio feliz de que no había escuela, no quisieron
esperar a conocer las razones de aquel sapientísimo acuerdo, y
despojándose velozmente de los arreos estudiantiles, se lanzaron a la
calle en busca de otros caballeritos de la vecindad.

---Tome usted asiento---añadió Cordero, dejando su silla, que era la más
cómoda de la tienda, para ofrecérsela a la joven.---Ayude usted mi flaca
memoria. ¿Qué nombre tiene nuestra nueva reina?

---María Cristina.

---Eso es\ldots{} María Cristina\ldots{} ¡Cómo se me olvidan los
nombres!\ldots{} Dícese que este casamiento nos va a traer grandes
felicidades, porque la napolitana\ldots{} pásmese usted\ldots{}

El héroe, después de mirar a la puerta para estar seguro de que nadie le
oía, añadió en voz baja:

---Pásmese usted\ldots{} es una francmasona, una insurgente, mejor
dicho, una real dama en quien los principios liberales y filosóficos se
unen a los sentimientos más humanitarios. Es decir, que tendremos una
Reina domesticadora de las fierezas que se usan por acá.

---A mí me han dicho, que ha puesto por condición para casarse que el
rey levante el destierro a todos los emigrados.

---A mí me han dicho algo más---añadió Cordero, dando una importancia
extraordinaria a su revelación,---a mí me han dicho que en Nápoles bordó
secretamente una bandera para los insurrectos de\ldots{} de no sé qué
insurrección. ¿Qué cree usted? La mandan aquí porque si se queda en
Italia da la niña al traste con todas las tiranías\ldots{} Que ella es
de lo fino en materia de liberalismo ilustrado y filosófico, me lo
prueba más que el bordar pendones el odio que le tiene toda la
turbamulta inquisidora y apostólica de España y Europa y de las cinco
partes del globo terráqueo. ¿Estaba usted anoche aquí cuando el Sr.~de
Pipaón leyó un papel francés que llaman la \emph{Quotidienne}?
¡Barástolis! ¡Y qué herejías le dicen! Ya se sabe que esa gente cuando
no puede atacar nuestro sistema gloriosísimo a tiros y puñaladas lo
ataca con embustes y calumnias. Bendita sea la princesa ilustre que ya
trae el diploma de su liberalismo en las injurias de los realistas. Nada
le falta, ni aun la hermosura, y para juzgar si es tan acabada como
dicen los papeles extranjeros, vamos usted y yo a darnos el gustazo de
verla entrar.

La persona a quien de este modo hablaba el tendero de encajes no tenía
un interés muy vivo en aquellas graves cosas de que pendía quizás el
porvenir de la patria; pero llevada de su respeto a D. Benigno, le
miraba mucho y pronunciaba un sí al fin de cada parrafillo. Conocida de
nuestros lectores desde 1821\footnote{Véase \emph{El Grande Oriente}.},
esta discreta joven había pasado por no pocas vicisitudes y conflictos
durante los ocho años transcurridos desde aquella fecha liberalesca
hasta el año quinto de Calomarde en que la volvemos a encontrar. Su
carácter, altamente dotado de cualidades de resistencia y energía que
son como el antemural que defiende al alma de los embates de la
desesperación, era la causa principal de que las desgracias frecuentes
no desmejorasen su persona. Por el contrario, la vida activa del
corazón, determinando actividades no menos grandes en el orden físico,
le había traído un desarrollo felicísimo, no sólo por lo que con él
ganaba su salud sino por el provecho que de él sacaba su belleza. Esta
no era brillante ni mucho menos, como ya se sabe, y más que belleza en
el concepto plástico era un conjunto de gracias accesorias realzando y
como adornando el principal encanto de su fisonomía que era la expresión
de una bondad superior.

La madurez de juicio y la rectitud en el pensar; el don singularísimo de
convertir en fáciles los quehaceres más enojosos, la disposición para el
gobierno doméstico, la fuerza moral que tenía de sobra para poder darla
a los demás en días de infortunio, la perfecta igualdad del ánimo en
todas las ocasiones, y finalmente aquella manera de hacer frente a todas
las cosas de la vida con serenidad digna, cristiana y sin afán, como
quien la mira más bien por el lado de los deberes que por el de los
derechos, hacían de ella la más hermosa figura de un tipo social que no
escasea ciertamente en España, para gloria de nuestra cultura.

---Los que no la ven a usted desde el año 24---le dijo aquel mismo día
D. Benigno observándola con tanta atención como complacencia,---no la
conocerán ahora. Me tengo por muy feliz al considerar que en mi casa ha
sido donde ha ganado usted esos frescos colores de su cara, y que bajo
este techo humilde ha engrosado usted considerablemente\ldots{} digo
mal, porque no está usted como mi pobre Robustiana ni mucho
menos\ldots{} quiero decir, proporcionadamente, de un modo adecuado a su
estatura mediana, a su talle gracioso, a su cuerpo esbelto. Beneficios
de la vida tranquila, de la virtud, del trabajo, ¿no es verdad?\ldots{}
Todos los que la vieron a usted en aquellos tristes días, cuando a
entrambos nos pusieron a la sombra y colgaron al pobre Sarmiento\ldots{}

Este recuerdo entristeció mucho a la joven, impidiendo que su amor
propio se vanagloriase con los elogios galantes que acababa de oír. Eran
ya las once de la mañana, y vestida como en día de fiesta para acompañar
a D.~Benigno, esperaba en la tienda la señal de partida.

---Aguarde usted: voy a hacer un par de asientos en el libro---dijo este
sentándose en su escritorio.---Todavía tenemos tiempo de sobra. Iremos a
la casa de D. Francisco Bringas, de cuyos balcones se ha de ver muy
requetebién toda la comitiva. Los pequeños se quedarán con mi hermana y
llevaremos a Primitivo y a Segundo. ¿Están vestidos?

Los dos muchachos, de doce y diez años respectivamente, no tenían la
soltura que a tal edad es común en los polluelos de nuestros días; antes
bien encogidos y temerosos, vestidos poco menos que a mujeriegas,
representaban aquella deliciosa perpetuidad de la niñez que era el
encanto de la generación pasada. Despabilados y libertinos en las
travesuras de la calle, eran dentro de casa humildes, taciturnos y
frecuentemente hipócritas.

Gozosos de salir con su padre a ver la entrada de la cuarta reina,
esperaban impacientes la hora y formando alrededor de la joven grupo
semejante al que emplean los artistas para representar a la Caridad, la
manoseaban so pretexto de acariciarla, le estrujaban la mantilla,
arrugándole las mangas y curioseando dentro del ridículo. La joven tenía
que acudir a cada instante a remediar los desperfectos que los dos
inquietos y pegajosos muchachos se hacían en su propio vestido, y ya
atando al uno la cinta de la gorra o cachucha, o abotonándole el
casaquín, ya asegurando al otro con alfileres la corbata, no daba reposo
a sus manos ni tenía ocasión para quitárseles de encima.

---No seáis pesados---les dijo con enfado su padre,---y no sobéis tanto
a nuestra querida \emph{Hormiguita}. Para verla, para darle a entender
que la queréis mucho, no es preciso que le pongáis encima esas
manazas\ldots{} que sabe Dios cómo estarán de limpias: ni hace falta que
la llenéis de saliva besuqueándola\ldots{}

Esta reprimenda les alejó un poco del objeto de su adoración; pero
siguieron contemplándola como bobos, cortados y ruborosos, mientras
ella, con la sonrisa en los labios, reparaba tranquilamente las
chafaduras de su vestido y las arrugas del encaje, para abrir luego su
abanico y darse aire con aquel ademán ceremonioso y acompasado, propio
de la mujer española.

Entretanto, allá arriba, en el piso donde vivía la familia oíanse
batahola y patadillas con llanto y becerreo, señal del pronunciamiento
de los dos Corderos menores, Rafaelito y Juan Jacobo, rebelándose contra
la tiranía que les dejaba encerrados en casa en la fastidiosa compañía
de la tía Crucita.

---Ya escampa---dijo Cordero señalando al techo con el rabo de la
pluma,---oiga usted al pueblo soberano que aborrece las cadenas\ldots{}
Verdad que mi hermana no es de aquellas personas organizadas por la
Naturaleza para hacer llevadero y hasta simpático el despotismo.

Y dejando por un momento la escritura entró en la trastienda dirigiendo
hacia arriba por el hueco de la tortuosa escalerilla estas palabras:

---Cruz y Calvario, no les pegues, que harta desazón tienen con quedarse
en casa en día de tanto festejo.

---Idos de una vez a la calle y dejadme en paz---contestó de arriba una
voz nada armoniosa ni afable,---que yo me entenderé con los enemigos. Ya
sé cómo les he de tratar\ldots{} Eso es, marchaos vosotros, marchaos al
paseíto tú y la linda Marizápalos, que aquí se queda esta pobre mártir
para cuidar serpentones y aguantar porrazos, siempre sacrificada entre
estos dos cachidiablos\ldots{} Idos enhorabuena\ldots{} a bien que en la
otra vida le darán a cada cual su merecido.

Violento golpe de una puerta fue punto final de este agrio discurso, y
en seguida se oyeron más fuertes las patadillas infantiles de los
corderos y el sermoneo de la pastora.

---Siempre regañando---dijo D. Benigno con jovialidad,---y arrojando
venablos por esa bendita boca, que con ser casi tan atronadora como la
de un cañón de a ocho, no trae su charla insufrible de malas entrañas ni
de un corazón perverso. Mil veces lo he dicho de mi inaguantable hermana
y ahora lo repito: «es la paloma que ladra».

Esto lo dijo Cordero guardando en su lugar las plumas con el libro de
cuentas y todos los trebejos de escribir, y tomó después con una mano el
sombrero para llevarlo a la cabeza, mientras la otra mano trasportaba el
gorro carmesí de la cabeza a la espetera en que el sombrero estuvo.

---Vámonos ya, que si no llegamos pronto encontraremos ocupados los
balcones de Bringas.

La joven alzaba la tabla del mostrador para salir con los chicos, cuando
la tienda se oscureció por la aparición de un rechoncho pedazo de
humanidad que casi llenaba el marco de la puerta con su bordada casaca,
sus tiesos encajes, su espadín, su sombrero, sus brazos que no sabían
cómo ponerse para dar a la persona un aspecto pomposo en que la
rotundidad se uniera con la soltura. ---Felices, Sr.~D. Juan de
Pipaón---dijo don Benigno observando de pies a cabeza al
personaje.---Pues no viene usted poco majo\ldots{} Así me gusta a mí la
gente de corte\ldots{} Eso es vestirse con gana y paramentarse de veras.
A ver, vuélvase usted de espaldas\ldots{} ¡Magnífico! ¡qué
faldones!\ldots{} A ver de frente\ldots{} ¡qué pechera! Alce usted el
brazo: muy bien. ¡Cómo se conoce la tijera de Rouget! De mis encajes
nada tengo que decir\ldots{} ¡qué saldrá de esta casa que no sea la
bondad misma! Póngase usted el sombrero a ver qué tal cae\ldots{}

Superlative\ldots{} ¡Con qué gracia está puesta la llave dorada sobre la
cadera!\ldots{} ¿Estas medias son de casa de Bárcenas?\ldots{} ¡Qué bien
hacen las cruces sobre el paño oscuro!\ldots{} una, dos, tres, cuatro
veneras\ldots{} Bien ganaditas todas, ¿no es verdad, ilustrísimo señor
D. Juan?\ldots{} ¡Barástolis! parece usted un patriarca griego, un
sultán, un califa, el Rey que rabió o el mismísimo mágico de Astracán.

Conforme lo decía iba examinando pieza por pieza, haciendo dar vueltas
al personaje como si este fuera un maniquí giratorio. Don Benigno y la
joven, no menos admirada que él, ponderaban con grandes exclamaciones la
belleza y lujo de todas las partes del vestido, mientras el cortesano se
dejaba mirar y asentía en silencio, con un palmo de boca abierta, todo
satisfecho y embobado de gozo, a los encarecimientos que de su persona
se hacían.

---Todo es nuevo---dijo la dama.

---Todo---repitió Pipaón mirándose a sí mismo en redondo como un pavo
real.---Mi destino de la secretaría de S. M. ha exigido estos
dispendios.

En seguida fue enumerando lo que le había costado cada pieza de aquel
torreón de seda, galones, plumas, plata, encajes, piedras y ballenas,
rematado en su cúspide por la carátula más redonda, más alborozada, más
contenta de sí misma que se ha visto jamás sobre un montón de carne
humana.

---Pero no nos detengamos---dijo al fin,---ustedes salían\ldots{}

---Vamos a casa de Bringas. ¿Va usted también allá?

---¿Yo?, no, hombre de Dios. Mi cargo me obliga a estar en palacio con
los señores ministros y los señores del Consejo para recibir allí
a\ldots{}

Acercó su boca al oído de D. Benigno y protegiéndola con la palma de la
mano, dijo en voz baja:

---A la francmasona\ldots{}

Ambos se echaron a reír y D. Benigno se envolvió en su capa diciendo:

---¡Pues viva la reina francmasona! El desfrancmasonizador que la
desfrancmasonice buen desfrancmasonizador será.

---Eso no lo dice Rousseau.

---Pero lo digo yo\ldots{} Y andando que es tarde.

---Andandito\ldots---murmuró Pipaón incrustando su persona toda en el
hueco de la puerta para ofrecerla a la admiración de los
transeúntes.---Pero se me olvidaba el objeto de mi visita.

---¿Pues no ha venido usted a que le viéramos?

---Sí, y también a otra cosa. Tengo que dar una noticia a la señora doña
Sola.

La joven se puso pálida primero, después como la grana, siguiendo con
los ojos el movimiento de la mano de Pipaón que sacaba unos papeles del
bolsillo del pecho.

---¿Noticias? Siempre que sean buenas---dijo Cordero cerrando y
asegurando una de las hojas de la puerta.

---Buenas son\ldots{} Al fin nuestro hombre da señales de vida. Me ha
escrito y en la mía incluye esta carta para usted.

Soledad tomó la carta, y en su turbación la dejó caer, y la recogió y
quiso leerla y tras un rato de vacilación y aturdimiento, guardola para
leerla después.

---Y no me detengo más---dijo Pipaón,---que voy a llegar tarde a
palacio. Hablaremos esta noche, Sr.~D. Benigno, señora doña
\emph{Hormiga}. Abur.

Se eclipsó aquel astro. Por la calle abajo iba como si rodara, semejante
a un globo de luz, deslumbrando los ojos de los transeúntes con los mil
reflejos de sus entorchados y cruces, y siendo pasmo de los chicos,
admiración de las mujeres, envidia de los ambiciosos, y orgullo de sí
mismo.

Cuando el héroe de Boteros, dada la última vuelta a la llave de la
puerta y embozado en su pañosa, se puso en marcha, habló de este modo a
su compañera:

---¿Noticias de aquel hombre?\ldots{} Bien. ¿Cartas venidas por conducto
de Pipaón?\ldots{} \emph{malum signum}. No tenemos propiamente
correo\ldots{} Querida \emph{Hormiga}, es preciso desconfiar en todo y
por todo de este tunante de Bragas y de sus melosas afabilidades y
cortesanías. Mil veces le he definido y ahora le vuelvo a definir: «es
el cocodrilo que besa».

\hypertarget{ii}{%
\chapter{II}\label{ii}}

¿Por qué vivía en casa de Cordero la hija de Gil de la Cuadra? ¿Desde
cuándo estaba allí? Es urgente aclarar esto.

Cuando pasó a mejor vida del modo lamentable e inicuo que todos sabemos
D. Patricio Sarmiento, Soledad siguió viviendo sola en la casa de la
calle de Coloreros. D. Benigno y su familia continuaron también en el
piso principal de la misma casa. La vecindad continuada y más aún la
comunidad de desgracias y de peligros en que se habían visto, aumentaron
la afición de Sola a los Corderos y el cariño de los Corderos a Sola,
hasta el punto de que todos se consideraban como de una misma familia, y
llegó el caso de que en la vecindad llamaran a la huérfana \emph{Doña
Sola Cordero}.

A poco de nacer Rafaelito trasladose don Benigno a la subida de Santa
Cruz, y al principal de la casa donde estaba su tienda, y como allí el
local era espacioso, instaron a su amiga para que viviera con ellos.
Después de muchos ruegos y excusas quedó concertado el plan de
residencia. En aquellos días se casó Elena con el jovenzuelo Angelito
Seudoquis, el cual, destinado a Filipinas cuatro meses después de la
boda, emprendió con su muñeca el viaje por el Cabo, y a los catorce
meses los señores de Cordero recibieron en una misma carta dos noticias
interesantes; que sus hijos habían llegado a Manila y que antes de
llegar les habían dado un nietecillo. Lo mismo D. Benigno que su esposa
veían que la amiga huérfana iba llenando poco a poco el hueco que en la
familia y en la casa había dejado la hija ausente. Pruebas dio aquella
bien pronto de ser merecedora del afecto paternal que marido y mujer le
mostraban. Asistió a doña Robustiana en su larga y penosa enfermedad con
tanta solicitud y abnegación tan grande que no lo haría mejor una santa.
Nadie, ni aun ella misma, hizo la observación de que había pasado su
juventud toda asistiendo enfermos. Gil de la Cuadra, doña Fermina,
Sarmiento, doña Robustiana marcaban las fechas culminantes y sucesivas
de una existencia consagrada al alivio de los males ajenos, siempre con
absoluto desconocimiento del bien propio.

Doña Robustiana sucumbió. Las buenas costumbres y el respeto a las
apariencias morales, que no sin razón auxilian a la moral verdadera, no
permitían que una joven soltera viviese en compañía de un señor viudo.
Fue necesario separarse. D.~Benigno tenía una hermana vieja y solterona,
avecindada en Madrid, medianamente rica, y de cuya suavidad, semejante a
la de un puerco-espín, tiene el lector noticia. Poseía doña Cruz Cordero
un carácter espinoso, insufrible, inexpugnable como una ruda fortaleza
natural de displicencia, artillada con los cañones de las palabras
agrias y duras. No se llegaba al interior de tal plaza ni por la
violencia ni por el cariño. No se rendía a los ataques ni se dejaba
sorprender por la zapa. El pobre D. Benigno apuró todos los medios para
conseguir que su hermana se fuera a vivir con él, a fin de constituir la
casa en pie mujeril y poder retener a su lado a Sola sin miedo a
contravenir las prácticas sociales. Pero Doña Cruz hacía tan poco caso
de la voz de la razón como de las voces del cariño y se fortalecía más
cada vez en el baluarte de su egoísmo. Todo provenía de su odio a los
muchachos, ya fueran de pecho, ya pollancones o barbiponientes. En esto
no había diferencias: aborrecía la flor de la humanidad cualquiera que
fuese su estado, y seguramente se dudara de la aptitud de su corazón
para toda clase de amor si no existiesen gatos y perros y aun mirlos
para probar lo contrario.

Si no pudo conseguir D. Benigno que Doña Cruz fuese a vivir con él,
logró que admitiese en su compañía a Sola, no sin que pusiera mil
enojosas condiciones la vieja. A aquella época pertenecen los apuros de
D. Benigno, su soledad de padre viudo entre biberones y amas de cría y
los otros ruines trabajos que hemos descrito al principio de esta
narración. La de Gil de la Cuadra ayudábale un poco durante el día, pero
no en las noches, porque doña Cruz había hecho la gracia de irse a vivir
al extremo de la Villa, lindando con el Seminario de Nobles, y rarísima
vez visitaba a su hermano en horas incómodas.

Llegó un día en que la paciencia de Don Benigno, como todo aquello que
ha tenido largo y abundante uso, tocó a su límite. Ya no había más
paciencia en aquella alma tan generosamente dotada de nobles prendas por
Dios. Pero aún había, en dosis no pequeña, la decisión para acometer
grandes cosas, aquella bravura de la acción unida a la audacia del
pensamiento que en una fecha memorable le pusieron al nivel de los más
grandes héroes.

So pretexto de una enfermedad grave, Cordero hizo venir a Doña Crucita a
su casa, y luego que la tuvo allí, le endilgó este discurso,
amenazándola con una gruesa llave que en la mano tenía:

---Sepa usted, señora Doña Basilisco, que de aquí no saldrá si no es
para el cementerio, siempre que no se conforme a vivir en compañía de su
hermano. Solo estoy y viudo, con hijos pequeños y uno todavía mamón.
Dígame si es propio que yo abandone los quehaceres de mi comercio para
arrullar muchachos, teniendo, como tengo, dos mujeres en mi familia que
lo harán mejor que yo\ldots{} ¡Silencio, porque pego!\ldots{} De aquí no
se sale.

Doña Crucita alborotó la casa, y aun quiso llamar a la justicia; pero D.
Benigno, Sola y el padre Alelí que era muy amigo de ambos hermanos
lograron calmarla, para lo cual fue preciso anteponer a todas las
razones la traslación de todos los bichos que en su morada tenía la
señora, añadiendo a la colección nuevos ejemplares que Cordero compró
para acabar de conquistar la voluntad de la \emph{paloma ladrante}. Al
digno señor no le importaba ver su casa convertida en un arca de Noé,
con tal de tener en ella la compañía que deseaba.

Desde entonces varió la existencia de Cordero, así como la de Sola.
Aquel volvió a sus quehaceres naturales. Los chicos tuvieron quien les
cuidara bien y todo marchó a pedir de boca. Crucita, sin dejar de
renegar de su hermano, de los endiablados borregos y del insoportable
ruido de la calle, se fue conformando poco a poco.

Pronto se conoció que el gobierno de la casa estaba en buenas manos.
Sola la encontró como una leonera y la puso en un pie de orden, limpieza
y arreglo que inundaba de gozo el corazón de D. Benigno. Ni aun en
tiempo de su Robustiana había él visto cosa semejante. Ya no se volvió a
ver ninguna pieza descosida sobre el cuerpo de los corderillos, ni se
echó de menos botón, faja ni cinta. Ninguna prenda ni objeto se vio
fuera de su sitio, ni rodaba la loza por el suelo, ni subía el polvo a
los vasares, ni estaban las sillas patas arriba y las lámparas boca
abajo. Todo mueble ocupó su lugar conveniente, y toda ocupación tuvo su
hora fija e inalterable. No se buscaba cosa alguna que al punto no se
encontrara, ni se hacía esperar la comida ni la cena. Los objetos
preciosos no podían confundirse con los últimos cachivaches, porque
había sido inaugurado el reinado de las distancias. El latón brillaba
como la plata y el cerezo tenía el lustre de la caoba. D. Benigno estaba
embelesado, y repetía aquel pasaje de su autor favorito: «Sofía conoce
maravillosamente todos los detalles del gobierno de la casa, entiende de
cocina, sabe el precio de los comestibles y lleva muy bien las cuentas.
Tiene un talento agradable sin ser brillante, y sólido sin ser
profundo\ldots{} La felicidad de una joven de esta clase consiste en
labrar la de un hombre honrado».

La casa era grande, tortuosa y oscura como un laberinto. Había que
conocerla bien para andar sin tropiezo por sus negros pasillos y
aposentos, construidos a estilo de rompe-cabezas. Sólo dos piezas tenían
ambiente y luz, y en una de ellas, la mejor de la casa, fue preciso
instalar a Crucita con las doce jaulas de pájaros que eran su delicia.
No faltaba en el estrado ningún objeto de los que entonces constituían
el lujo, pues a D. Benigno se le había despertado el amor de las cosas
elegantes, cómodas y decentes, y como no carecía de dinero, cada día
daba permiso a su diligente \emph{Hormiga} para introducir alguna
novedad. Con las onzas de Cordero y el buen gusto de Sola viose pronto
la casa en un pie de elegancia que era el asombro de la vecindad. Fue
vestida la sala de hermoso papel imitando mármol, y una batería de
sillas de caoba sustituyó a las antiguas de nogal y cerezo. El brasero
era como un gran artesón de cobre, sustentado sobre cuatro garras
leoninas, y con la badila y reja no pesaba menos de medio quintal. El
sofá y los dos sillones, que hoy nos parecerían potros de suplicio, eran
de lo más selecto. Las cortinas de percal blanco con franjas de tafetán
encarnado, tenían aspecto risueño y se conceptuaban entonces como cosa
de gran lujo y elegancia. No faltaban las mesillas de juego con sus
indispensables candeleros de plata, ni las célebres y ya olvidadas
rinconeras llenas de baratijas y objetos de arte y ciencia, tales como
cajas, caracoles, figurillas de yeso, algún jarro, libros y un par de
pajaritos disecados. En el marco del espejo apaisado veíanse algunas
plumas de pavo real puestas con arte y simetría, como las pintan en las
cabezas de los salvajes. En cuestión de láminas, habíanse conservado las
antiguas que eran \emph{el León de Florencia devorando a un niño, la
Desgraciada muerte de Luis XVI} y la \emph{Caída de Ícaro}.

Vistos de la calle los balcones presentaban el aspecto más alegre que
puede imaginarse. Los tiestos, con ser tantos, no eran bastantes para
quitar sitio a las jaulas colgadas unas sobre otras. Interiormente no
cesaba la algarabía formada por el piar de algunos pájaros, el canto de
otros, el ladrido de los falderillos, el mayido de los gatos y los
roncos discursos de la cotorra. El esmero con que Crucita atendía al
cuidado y a las necesidades todas de su riqueza zoológica hacía que la
existencia de tanto y tanto bicho no fuera incompatible con el perfecto
aseo de la casa.

D.~Benigno estaba contentísimo del buen arreglo que Sola había puesto en
el gabinete donde él vivía. Sus ropas abundantes y tan bien dispuestas
que jamás notó en ellas rotura de más ni botón de menos, le recreaban la
vista, así como la limpieza de su variada colección de sombreros. No le
cautivaba menos el ver libres siempre de polvo sus adminículos de caza
(diversión a que era muy aficionado), ni la buena colocación que se
había dado a las estampas de Santa Leocadia y la Virgen del Sagrario
(ambas proclamando el abolengo toledano del propietario), ni lo bien
puestos que estaban los libros. Estos no eran muchos, pero sí escogidos,
y sólo formaban dos obras: las de Rousseau, edición de 1827, en
veinticinco tomitos, y el \emph{Año Cristiano} en doce. Aunque alineados
en dos grupos distintos, no por eso dejaban de andar a cabezadas, dentro
de un mismo estante, el \emph{Vicario Saboyano} y San Agustín.

Con el orden perfecto en la disposición de todo lo de la casa corría
parejas la buena concordia entre sus habitantes, si se exceptúan las
genialidades de Crucita, que fueron menos molestas desde que Sola adoptó
el sistema de hacerle poco caso sin aparentar contrariarla.

Desapacible y brusca con los chicos, no consentía que se le acercaran a
dos varas a la redonda. No obstante, el frecuente trato con ellos y la
dulzura de su hermano y de la \emph{Hormiga} fueron poco a poco
arrancando las espinas de aquel carácter endiablado, y al fin sin dejar
de hablarles en el lenguaje más duro y desabrido que se puede imaginar,
manifestaba algún interés por los cuatro \emph{enemigos}, ayudaba a
cuidarles, y aun se permitía contarles algún trasnochado y soso cuento.

Los muchachos, a excepción del más pequeño, eran pacíficos. Primitivo y
Segundo adelantaban regularmente en sus estudios, y en cuanto a
vocaciones, el tono especial de la época y los personajes de aquel
tiempo despertaban en ellos ambiciones varias. El mayor quería ser Padre
Guardián, para tomar mucho chocolate, dar a besar su mano a los
transeúntes y salir a paseo entre un par de duques o marqueses. El
segundo, que era vanidosillo y fachendoso, quería ser tambor mayor de la
Guardia Real, porque eso de ir delante de un regimiento haciendo gestos
y espantando moscas con un bastón de porra, le parecía el colmo de la
dicha. Rafaelito era más modesto. No le hablaran a él de figuraciones ni
altas dignidades: él no quería ser sino confitero, para poder atracarse
de dulces desde la mañana a la noche y hacer bonitas velas para los
santos. En cuanto a Juanito Jacobo, aunque no hablaba, bien se le
conocía que su vocación era la de gigante Goliat o Hércules, según lo
que destrozaba, berreaba y las diabluras que hacía andando a gatas, sin
dejarse amedrentar por cocos ni espantajos.

Tranquilo, feliz, gozoso del orden en que vivía y que amaba por
naturaleza y costumbre, Cordero veía pasar suavemente los días. El
método en la existencia le encantaba, y la semejanza entre el hoy y el
ayer era su principal delicia.

Hombre laborioso, de sentimientos dulces y prácticas sencillas;
aborrecedor de las impresiones fuertes y de las mudanzas bruscas, D.
Benigno amaba la vida monótona y regular, que es la verdaderamente
fecunda. Compartiendo su espíritu entre los gratos afanes de su comercio
y los puros goces de la familia; libre de ansiedad política; amante de
la paz en la casa, en la ciudad y en el estado; respetuoso con las
instituciones que protegían aquella paz; amigo de sus amigos; amparador
de los menesterosos; implacable con los pillos, fuesen grandes o
pequeños; sabiendo conciliar el decoro con la modestia y conociendo el
justo medio entre lo distinguido y lo popular, era acabado tipo del
\emph{burgués} español que se formaba del antiguo pechero fundido con el
hijodalgo, y que más tarde había de tomar gran vuelo con las compras de
bienes nacionales y la creación de las carreras facultativas hasta
llegar al punto culminante en que ahora se encuentra.

La formidable clase media que hoy es el poder omnímodo que todo lo hace
y deshace, llamándose política, magistratura, administración, ciencia,
ejército, nació en Cádiz entre el estruendo de las bombas francesas y
las peroratas de un congreso híbrido, inocente, extranjerizado si se
quiere, pero que había brotado como un sentimiento o como un instinto
ciego e incontrastable del espíritu nacional. El tercer estado creció,
abriéndose paso entre frailes y nobles, y echando a un lado con
desprecio estas dos fuerzas atrofiadas y sin savia, llegó a imperar en
absoluto, formando, con sus grandezas y sus defectos una España nueva.

Perdónesenos la digresión, y volvamos a Cordero, del cual nos falta
decir que en los últimos años había prosperado grandemente en su
comercio. Pocas noches antes de aquel día en que suponemos comenzada
esta narración, el héroe estaba en su gabinete contando el dinero de la
semana. Después que tomó nota de las cantidades y distribuyó estas
cariñosamente en las cestillas de paja que servían para el caso, llamó a
Sola, y haciéndola sentar frente a él, le dijo así:

---Si no comunico a alguien lo que pienso en este instante, apreciable
\emph{Hormiguita}, reviento de seguro.

Sola sonreía, dando más luz al \emph{quinqué} que sobre la mesa colocado
repartía en porción igual su resplandor a los dos personajes. Don
Benigno se reía también, y ya se acariciaba la barba redondita y
arrebolada, como una manzana recién cogida, ya se arreglaba las gafas de
oro, cuya tendencia a resbalar sobre la nariz picuda y fina iba en
aumento cada día.

---Pues lo que pienso---añadió,---es que sin saber cómo, me encuentro
rico\ldots{} es decir, no muy rico, entendámonos, sino simplemente en
ese estado de buen acomodo que me permitiría, si quisiera, renunciar al
comercio y retirarme a vivir tranquilo en mis queridos Cigarrales, donde
no me ocuparía más que en labrar el campo y criar a mis hijos.

Sola le respondió a estas palabras con otras de felicitación, y el
héroe, que se sentía aquella noche con muchas ganas de charlar, continuó
así:

---Con usted no hay secretos. Sepa usted que ayer he pagado el último
plazo de esta casa en que vivimos; de modo que es mía, tan mía como mis
anteojos y mi corbata de suela. En los Cigarrales he comprado ya más de
cien fanegadas para agregarlas a las que heredé de mis padres, y pienso
comprar las del tío \emph{Rezaquedito}, que saldrán a la venta muy
pronto. De modo que ya estamos libres de perder el sueño por cavilar en
el día de mañana, y si por acaso me da un torozón (que no me dará) no
estaré afligido en mi última hora con la idea de que mis hijos tengan
que vivir a expensas de parientes y amigos. Vea usted por dónde la
Divina Providencia ha premiado mi laboriosidad, y nada más que mi
laboriosidad, pues talentos no los tengo, y en cuanto a picardías, ya se
sabe que esa moneda no corre dentro de mi casa.

---Dios ha querido que un hombre tan bueno y tan cabal en todo---le dijo
Sola,---tenga su merecido en el mundo, porque si al bueno no le da Dios
los medios de ser caritativo y generoso ¿qué sería de los pobres, de los
abandonados, de los huérfanos?

---No, no\ldots---replicó Cordero un si es no es conmovido,---no hay
aquí generosidades que alabar ni virtudes que enaltecer. Algo he hecho
por los menesterosos, y si alguna persona ha recibido especialmente de
mí ciertos beneficios, estos han sido menores de los que ella se merece.
Dios no puede estar satisfecho de mí en esta parte\ldots{} Que se han
sucedido buenos años para el género; que los cambios políticos,
improvisando posiciones han desarrollado el lujo; que las modas han
favorecido grandemente el comercio de blondas y puntillas; que la paz de
estos años de despotismo ha traído muchos bailes y saraos, equivalentes
a gran despilfarro de Valenciennes, Flandes y Malinas; que el
restablecimiento del culto y clero después de los tres años trajo la
renovación de toda la ropa de altar y mucho consumo de encajería
religiosa; que mi puntualidad y honradez me dieron la preferencia entre
las damas; que la corte misma, a pesar de que son bien notorias mis
ideas contrarias a la tiranía, no quiere ver entrar por las puertas de
palacio ni media vara de Almagro que no sea de casa de Cordero, y en
fin, que Dios lo ha querido y con esto se dice todo. Bendigámosle y
pidámosle luces para acertar a hacer el bien que aún no hemos hecho, y
que es a manera de una sagrada deuda pendiente con la sociedad, con la
conciencia\ldots{}

El héroe se atascó en su propia retórica, como le pasaba siempre que
quería expresar una idea no bien determinada aún en su espíritu, y un
sentimiento oprimido en las fuertes redes de la timidez y la delicadeza.

---Acabe usted que me da gusto oírle---le dijo Sola sonriendo,---pero
prontito, que hay mucho que hacer esta noche.

---Descanse usted un momento, por amor de Dios. ¿Siempre hemos de estar
sobre un pie?\ldots{} ¡Oh!, por mi parte, apreciable \emph{Hormiga},
estoy decidido a descansar. Verdad es que no soy un niño. Tengo
cincuenta y dos años.

Dicho esto, D. Benigno miró como extasiado a su protegida, que a su vez
contemplaba fijamente la luz a riesgo de quedarse deslumbrada toda la
noche.

---Cincuenta y dos años, que es mucho y es poco, según se
considere---añadió el héroe con cierta turbación.---Todo es relativo,
hasta los años, y yo con mi constitución recia y firme, mis acerados
músculos, mi desconocimiento absoluto de lo que son médicos y boticas,
no me cambio por esos pisaverdes de color de cera de muerto, que se
llaman muchachos por una equivocación del tiempo.

---Es usted rico; goza de perfecta salud---murmuró Sola, cuyas miradas,
como mariposas, gustaban de recrearse en la llama;---es además bueno
como el buen pan, tiene buen nombre y fama limpia, ¿qué más puede
desear?

Don Benigno dio un suspiro y mirando al tapete, dijo así:

---Es verdad: nada puedo desear. Temeridad e impertinencia sería pedir
más.

Ambos callaron.

---¿Tiene usted algo más que decirme?---preguntó Sola levantándose.

---Nada, nada, apreciable \emph{Hormiga}---dijo D. Benigno irradiando
bondad y sentimientos puros de su cara de rosa.---Nada más sino
que\ldots{} Dios sobre todo.

Después que la joven se fue, Cordero tomó a Rousseau como se toma el
brazo de un amigo para apoyarse en él, y abriendo el libro por donde
estaba la marca, indicando sin duda capítulo, párrafo o renglón de gran
interés, se quedó un buen rato meditando en la extraordinaria
profundidad, intención y filosofía de la sentencia con que el ginebrino
encabeza el libro Quinto del \emph{Emilio}.

Dice así: \emph{No es bueno que el hombre esté solo}.

\hypertarget{iii}{%
\chapter{III}\label{iii}}

El día era de los mejores que suele tener Madrid en invierno, con cielo
limpio y espléndido sol. Los madrileños, que por su índole castiza, no
necesitaban entonces ni ahora de grandes atractivos para echarse en
tropel a la calle, invadieron aquel día la carrera de las procesiones
regias que va desde Atocha a Palacio, vía ciertamente histórica y muy
interesante, por la cual han pasado tantos monarcas felices o
desgraciados, y no pocos ídolos populares. Si fuera posible reproducir
la serie de comitivas diversas que han recorrido ese camino del
entusiasmo desde la primera entrada de Fernando VII en Mayo de 1808,
tendríamos una galería curiosa en la cual muy pocas pinceladas tendría
que añadir la historia para hacer el cuadro completo de las sucesivas
idolatrías españolas. El quemar de los ídolos, cuando estamos cansados
de adorarlos, se verifica en otra parte.

Estas grandiosas comparsas tienen una monotonía que desespera; pero el
pueblo no se cansa de ver los mismos lacayos con las mismas pelucas, los
mismos penachos en la frente de los mismos caballos, y el inacabable
desfilar de uniformes abigarrados, de coches enormes más ricos que
elegantes, de generales en número infinito, y el trompeteo, la bulla, el
oscilar mareante de plumachos mil, el fulgor de bayonetas, y por último
el revoloteo de palomitas y de hojas de papel conteniendo los peores
sonetos y madrigales que pueden imaginarse.

Aquel día de Diciembre de 1829 el pueblo de Madrid admiró principalmente
la hermosura de la nueva reina, la cual era, según la expresión que
corría de boca en boca, una \emph{divinidad}. Su cara incomparablemente
graciosa y dulce tenía un sonreír constante que se entraba, como decían
entonces, hasta el corazón de todo el pueblo, despertando las más
ardientes simpatías. Bastaba verla para conocer su agudo talento, que
tanto había de brillar en las lides cortesanas, y para prever las nobles
conquistas que la gracia y la confianza habían de hacer prontamente en
el terreno de la brutalidad y del recelo. Jamás paloma alguna entró con
más valentía que aquella en el negro nidal de los búhos, y aunque no
pudo hacerles amar la luz, consiguió someterles a su talante y albedrío
consiguiendo de este modo que pareciesen menos malos de lo que eran. Fue
mirada su belleza como un sol de piedad que venía, si bien un poco
tarde, a iluminar los antros de venganza y barbarie en que vivía como un
criminal aherrojado, el sentimiento nacional.

No ha existido persona alguna a quien se hayan dedicado más versos. Por
ella sola se han fatigado más \emph{las deidades de Hipócrene} y ha
hecho más corvetas el buen Pegaso que por todas las demás reinas juntas.
A ella se le dijo que si el Vesubio la había despedido con
\emph{sombríos fulgores}, el Manzanares la recibió \emph{vestido de
flores}; se le dijo que Pirene había inclinado la \emph{erguida espalda}
para dejarla pasar y que en los \emph{vergeles de Aretusa} tocaba la
lira el \emph{virginal concilio} celebrando a la \emph{ninfa bella de
Parténope}.

La hermosa reina fue también cantada por los grandes poetas; que no todo
había de ser ruido en las diversas cataratas de versos que celebraron su
casamiento, su entrada, su embarazo, sus dos alumbramientos, sus días,
sus actos políticos más notables, y en particular el glorioso hecho de
la amnistía. D.~Juan Bautista Arriaza, que desde el año 8 venía haciendo
todos los versos decorativos y de circunstancias, la letra de todos los
himnos y las inscripciones de todos los arcos triunfales, echó el resto,
como decirse suele, en las fiestas del año 29. Quintana dedicó al
\emph{feliz enlace de Fernando VII} una canción epitalámica que no quiso
incluir en las ediciones de sus obras, y otros insignes vates de la
época la ensalzaron en aquellas odas resonantes y tiesas, algo parecidas
al parche duro y ruidoso de una caja de guerra, y cuya lectura deja en
los oídos impresión semejante a la que produciría una banda de tambores
en día de parada. Con todo, en la corona poética de esta insigne reina
se encuentran altos pensamientos y graciosas imágenes, principalmente en
todo aquello que aparece inspirado por la seductora sonrisa,

~

\begin{center}
\textit{que cuanto más se ve más enamora.}
\end{center}

Entró Cristina en coche acompañada de sus padres los reyes de Nápoles.
Al estribo derecho venía el esposo y tío, rigiendo magistralmente su
hermoso caballo. Era, según dicen, el primer jinete de su época;
verdaderamente nuestro Rey tenía un aspecto tan majestuoso como gallardo
cuando montaba en uno de aquellos apopléticos corceles cuya pesadez y
arrogancia nos han trasmitido Velázquez y Goya. La alzada del animal, el
corpulento busto del monarca, su rico uniforme, su alto sombrero de tres
picos, muy parecido, según la absurda moda de la época, a las mitras o
tinajones que llevan en su cabeza los bueyes de la arquitectura asiria,
daban a la colosal figura no sé qué apariencia babilónica que infundía
respeto y algo de supersticioso miedo.

Pero la arrogancia de la majestad ecuestre, la misma riqueza abigarrada
de su traje de gala no disimulaban en Fernando aquella decadencia precoz
que le hacía viejo a los cuarenta y cinco años. En su rostro duro y de
poco a propósito para ganar simpatías (por lo que se acomodaba
perfectamente al carácter) parecía que la nariz se había agrandado,
impaciente de juntarse al labio belfo, el que por su parte se estiraba a
más no poder, como si quisiera echarse fuera de tal cara. Su color, que
era una mezcla enfermiza del verdoso y del amoratado, extendía por sus
mejillas como una sombra lúgubre, en la cual lucían mejor sus ojos
grandes y negros, por donde en ciertos momentos se asomaban, con el
instantáneo fulgor del relámpago, sus alborotadas pasiones.

Pasaron. Aquel río de morriones, pelucas, sables desnudos, entorchados,
pompones y cabezas mil que se movían al compás de la marcha de tanto
caballo festoneado y lleno de garambainas; la sucesión de tanto y tanto
coche, semejante a canastillas hechas con todos los materiales posibles
desde la concha y el marfil hasta el cobre y la madera; el estruendo
solemne de la marcha real y todo lo demás que realza estas procesiones
tenían tan absorto y embobado al pueblo madrileño, amante de estas cosas
como ningún otro pueblo del mundo, que si la Corte hubiera estado
pasando y repasando de aquella manera por espacio de tres meses
seguidos, no faltarían ni un momento las grandes líneas de gente con la
boca abierta a un lado y otro de la carrera.

Por la multitud de caras bonitas y la variedad de colores que en ellos
había, parecían babilónicos jardines los balcones de las casas. En los
de la de Bringas que daban a la calle Mayor, estaba D. Benigno con Sola
y los chicos, amén de otras familias amigas del rico comerciante, que
dio su nombre a los soportales cercanos a Platerías. Quiso la
desgraciada suerte de Sola que le tocase salir al mismo balcón donde
estaba una señora a quien ciertamente no gustaba de ver en parte alguna,
y no porque la dama fuese de mal aspecto, sino por otros motivos muy
poderosos. Era de tal manera hermosa que cautivaba los ojos y el corazón
de cuantos la miraban. Por singular capricho de la Naturaleza, el tiempo
que de ordinario es enemigo y destructor de la hermosura, allí era su
cultivador y como su custodio, pues la conservaba fielmente y aun
parecía aumentarla cada año. De esta galantería del tiempo unida a los
adornos escogidos y a un esmero constante y casi religioso en la
persona, resultaba el \emph{boccato di cardinale} más rico que podría
imaginarse. Para mayor gracia, había tenido el buen acuerdo de vestirse
de maja, lo mismo que otras muchas damas que en aquel día clásico
adoptaron el traje nacional. Llevaba, pues, falda de alepín inglés color
de amaranto con abalorios negros, chaquetilla de terciopelo con muchos
botoncitos de filigrana de oro, mantilla de casco de tafetán con gran
velo de blonda, y peineta de pico de pato, todo puesto con
extraordinaria bizarría.

\hypertarget{iv}{%
\chapter{IV}\label{iv}}

Cuando Sola se vio junto a ella tuvo que disimular su espanto, viéndose
obligada a recibir el saludo de la dama y a devolverlo cortésmente.
Después hablaron las dos de lo bonita que estaba la carrera, de la
hermosura del tiempo, de los dichos y hechos que se contaban de la reina
Cristina y del excesivo número de personas que había en casa de Bringas,
las cuales rebosaban por los balcones como guindas en cesta.

Ocupada la mejor parte de los balcones por las señoras, los hombres poco
o casi nada podían ver. Cordero paseaba de largo a largo por la sala,
charlando con su amigo D. Francisco Bringas de cosas sustanciosas y muy
importantes, como la paz entre Rusia y Turquía, la cuestión de Grecia,
que pronto iba a ser reino independiente, y las tristes nuevas que
habían llegado de la expedición americana, deshecha y rota en Tampico,
con lo que parecía terminada nuestra dominación en aquel continente. D.
Benigno, que leía diariamente la \emph{Gaceta} y \emph{Diario}, estaba
al tanto de todo y sobre cada asunto daba juiciosos dictámenes. Los
impronunciables nombres de los puntos donde se batían turcos y rusos
salían de la boca de nuestro héroe con no poca dificultad, y Bringas,
que seguía con grandísimo ahínco el negocio de la nueva Grecia, barajaba
los nombres gatunos de los personajes de aquel país, y así no se oía
otra cosa que Miaulis, Mauromichalis y también Kalocotroni, Maurocordato
y Capodistria.

Pronto tomó la conversación otro rumbo con la llegada de cierto joven de
arrogante presencia, alto de cuerpo, agraciadísimo de rostro, con el
pelo en rizos, las mejillas rosadas, el color blanco, los ojos garzos,
los ademanes desenvueltos, el vestir elegante. Respondía al nombre de
Salustiano Olózaga y era un abogado de veinticuatro años, medio célebre
ya por sus brillantes alegatos forenses, y mayormente por la defensa que
había hecho ante el Consejo y Cámara de Castilla de un pobre albañil
inclusero, condenado a muerte por el robo de dos libras de tocino. La
Milicia Nacional, cuando había Milicia, el foro cuando había foro y la
política siempre consumían todo el ardor de su existencia.

Era el campeón juvenil de la idea naciente, y la Providencia habíale
dado, entre otras notables prendas, elocuencia, si no brillante, varonil
y sobria, con una lógica irresistible.

Los jóvenes de hoy, alumnos aprovechados del eclecticismo y del justo
medio, no comprenderán quizás el entusiasmo y valentía de aquellos
muchachos que sintiendo en su mente, por la natural índole de los
tiempos, una especie de inspiración sacerdotal, hablaban de los déspotas
y de la libertad como hablaría un romano de la primera república. Y no
se paraban en barras, y aun deseaban martirios heroicos, y se metían en
las conspiraciones más absurdas e inocentes, y osaban decir en pleno
foro, delante de los consejeros, cosas que pasman por lo valerosas e
intencionadas.

Desde que entró Salustiano no se habló más de Miaulis ni del bueno de
Kalocotroni. Alejados un tanto del salón principal y reforzado el grupo
con otras personas, el librero Miyar, el ingeniero Marcoartú y un
comerciante de la calle de Postas, llamado Bárcenas, se despacharon
todos a su gusto, siendo Olózaga tan hablador y contundente que no se
paraba en pelillos y con su lengua que más bien era un hacha iba dejando
muy mal parada a lo que todavía no se llamaba \emph{la situación}.

D.~Benigno que no gustaba de engolfarse mucho en política por los
peligros que pudiera traer, dejó a sus amigos para buscar en los
balcones la tertulia más grata y segura de las damas. La que vestía de
maja se había puesto a bromear con el marqués de Falfán de los Godos, el
hombre más mujeriego de aquel tiempo y también el más fino y galante, si
bien su persona, hecha ya ruina lastimosa, no le ayudaba nada en lo que
él quisiera que le ayudase. A Sola, en tanto, le daba conversación una
señora muy impertinente llamada doña Salomé Porreño, y a cada rato ponía
los ojos en blanco y echaba suspiros, cual si no tuviera en el mundo
otra misión ni empleo que estarse lamentando a todas horas de una cosa
perdida. Al lado de ella estaba una joven muy bonita, casada y por
añadidura en aquel interesante estado que anuncia la maternidad. La de
Presentacioncita, que así se llamaba, debía estar ya muy próxima, según
se echaba de ver al primer examen. Era su marido un tal D. Gaspar de
Grijalva con más riqueza que buen seso, y muy aficionado a meterse en
trapisondas políticas, por lo que Presentación se afligía mucho y estaba
siempre sobre ascuas temiendo que le ahorcasen. Esta señora, lo mismo
que Sola, parecían tener muy pocas ganas de conversación; pero doña
Salomé que estaba entre ellas como una especie de mediador parlante,
suplía la desgana de ellas con un insaciable apetito de palique, y así
no cesaba de hacer preguntas y observaciones poniendo en el discurso,
como se pone la sal en la comida, los suspiros y el incesante revolver
de los ojos.

Jenara, que era la maja, volví hacia atrás la cara a cada instante para
responder a Falfán de los Godos, y en uno de estos dimes y diretes habló
así:

---Sí, hoy mismo he tenido noticias suyas. Pipaón me entregó esta mañana
una carta que es de perlas, por las muchas cosas ingeniosas que me dice.
Creo que en mucho tiempo no le veremos por acá. Me anuncia que piensa
casarse.

Jenara hablaba en voz muy alta; pero como Falfán de los Godos era algo
teniente, es decir, sordo, nadie lo extrañaba. Al mismo tiempo la de
Porreño daba con el codo a Sola y le decía:

---¿Pero no me oye usted lo que le pregunto? Tres veces he preguntado a
usted que si conoce a aquel comandante que pasa, y no me ha dado
contestación\ldots{} Por lo visto aquí todos son sordos\ldots{} Se ha
quedado usted lela; ¿en qué piensa usted que está tan pálida?\ldots{}
¿no oye usted?\ldots{}

---Sí, sí---replicó Sola, como se replicaría a las avispas, si la picada
de estas alimañas fuera, en vez de picada, pregunta.---He oído
perfectamente.

La de Porreño, al ver que por aquella banda no sacaba nada de provecho,
se volvió a la otra y a Presentación. Después que la oyó, Presentación,
que era muy maligna, dijo así:

---Aguarde usted. Mandaré a casa por la \emph{Guía de Forasteros}, y con
ella en la mano le diré a usted los nombres de todos los comandantes,
capitanes y coroneles que hay en España.

La de Porreño miró al cielo, como si quisiera ponerle por testimonio de
tanta injusticia. Bueno es decir que no vestía de maja ni de cosa que lo
pareciera, sino a la moda pura y neta de 1822, con dulleta que ella
misma había trocado en pelliza, aplicándole los restos de un capisayo
antiguo. Su tocado era el llamado de turbante, guarnecido de cordones
que fueron de oro y unas plumas que más parecían de escribano que de
avestruz, como no pudiera aplicarse a uno y otro.

---También a mí me han dicho que piensa casarse---manifestó Falfán de
los Godos.

Entonces se oyó un murmullo, una voz sorda y general que sin decir nada,
claramente decía: «Ya viene, ya viene, ya, ya\ldots» La multitud se
agitó cual una gran culebra que pone en movimiento todas sus vértebras,
y en los balcones hubo un hondo suspiro de ansiedad que corrió de un
cabo a otro de la calle. Todos los ojos miraban a la Puerta del Sol, por
donde sonaba como el mugido de un mar, y al poco rato se vio que se
agitaba la superficie de cabezas y que brincaban saltando por encima de
la gente penachos de caballos, plumas de morriones y espadas desnudas.
El murmullo creció, estalló la marcha real como un trueno, y empezó a
pasar la corte.

Sola no veía nada, sino una confusa corriente de colorines y formas,
caballos que parecían hombres, hombres que trotaban, y un rodar continuo
de formas y magnificencias, todo en tropel y borrosamente al modo de
nube formada de la disolución de todas las visiones humanas. Un cerebro
que desfallece, permitiendo la alteración de las sensaciones ópticas
suele producir desvanecimiento y síncope; pero Sola hizo un esfuerzo,
cerró los ojos, dejando pasar la mareante comparsa, y así resistió,
fuertemente asida a los hierros del balcón. Cuando, pasada la corriente
de abigarrados coches, sólo quedaban los escuadrones de escolta,
principió a serenarse; pero todavía su visión estaba perturbada, y las
casas y balcones cuajados de damas seguían corriendo juntamente con la
caballería. Principiado el desfile por delante de Palacio, los
regimientos de infantería pasaban por la calle.

---Ese, ese coronel, ¿quién es?---preguntó súbitamente la de Porreño.

---Si no me engaño, es el moro Muza---replicó Presentación.

Diciéndolo, el caballo que montaba el teniente coronel señalado por
Salomé resbaló, y sin que el jinete pudiera sujetarlo, cayó pesadamente,
arrastrando a este. La caída fue tremenda. Oyose inmensa gritería
mujeril. Detúvose la gente, arremolinose el regimiento, acudieron
soldados y paisanos al infeliz jinete, magullado y aturdido por la
fuerza del golpe, y alzándole del suelo le entraron en una tienda para
darle algún socorro. Era un hombre de cuerpo largo y flaco, cara morena
y varonil. Al ser levantado del suelo hacía recordar involuntariamente
la figura de D. Quijote tendido en tierra después de cualquiera de sus
desventuradas aventuras.

En los balcones de Bringas agolpáronse todos para ver al caído.

---¡Pobre hombre!---exclamó Cordero.

---¡Y qué bien iba en el caballo!---dijo la de Porreño.

---Se parece al de la Triste Figura---indicó Bringas.---Es el mismísimo
D. Quijote ---observó Olózaga.

Jenara volviose prontamente, y con cierto tonillo de enfado dijo así:

---Pues no es D. Quijote, señor discursista, sino D. Tomás
Zumalacárregui, apostólico neto y con un corazón mayor que esta casa.

Cuando poco o nada había que ver en los balcones, Bringas obsequió a sus
amigos con algunas golosinas, acompañadas de licores y agua fresca, y
unos hartos de dulces, otros sin probarlos, empezó a desfilar. D.
Benigno con Sola y sus hijos fue a recorrer las calles para ver los
preparativos de las grandes fiestas que empezaban aquel día, y
principalmente para contemplar y admirar por sus cuatro costados
\emph{el templete}, monumento de lienzo pintado de que se hablaba mucho
y que con grandes dispendios se construyó en la Puerta del Sol sobre la
misma Mariblanca. Era la máquina más bonita que habían visto los
madrileños hasta entonces. Millares de personas la admiraban a todas
horas formando un círculo de papamoscas, y a la verdad, las columnas
pintadas, las cuatro estatuas y el globo terráqueo que lo tapaba todo
como un bonete harían caer de espaldas a Miguel Ángel, Herrera y a todos
los arquitectos habidos y por haber. Todo lo fue examinando Cordero, y
sobre todos los preparativos dio opiniones muy discretas. En los días y
noches siguientes llevó a su familia a ver las comparsas e iluminaciones
y a admirar la gran novedad del carro triunfal alegórico mitológico
manolesco, dispuesto por el corregidor Barrajón, y en el cual iban
haciendo de ninfas varias bellezas de Madrid, entre ellas \emph{Pepa la
Naranjera} que subida en el escabel más alto representaba a la Diosa
Venus.

La gente decía que iba \emph{vestida de Venus}, de lo que resultaba un
contrasentido; pero el decoro de nuestras costumbres y la santidad de
los tiempos no habrían consentido que las diosas salieran a la calle
como andaban por el Olimpo.

\hypertarget{v}{%
\chapter{V}\label{v}}

Entre las muchas sociedades más o menos secretas que amenazaron el poder
de Calomarde, hubo una que no precisamente por lo temible sino por otras
razones merece las simpatías de la posteridad. Llamose de los
\emph{Numantinos} y componíase de mucha y diversa gente. Entre los
atrevidos fundadores de ella hubo tres cuyos ilustres nombres conserva y
conservará siempre la historia patria: llamábanse Veguita, Pepe y
Patricio.

El objeto de los \emph{Numantinos} era, como quien no dice nada,
\emph{derrocar la tiranía}. Los medios para conseguir este fin no podían
ser más sencillos. Todo se haría bonitamente por medio de la siguiente
receta: \emph{matar al tirano} y \emph{fundar una república a estilo
griego}.

Retratemos a los tres audaces patriotas, ante cuya grandeza heroica
palidecerían los Gracos, Brutos y Aristogitones.

El primero, \emph{Veguita}, tenía diez y ocho años y era de la piel de
Barrabás, inquieto, vivo, saltón, con la más grande inventiva que se ha
visto para idear travesuras, bien fueran una voladura de pólvora, un
escalamiento de tapias, una paliza dada a tiempo o cualquier otro
desafuero. Su casta americana se revelaba en el brillo de sus negros
ojos, en su palidez y en sus extremadas alternativas de agitación e
indolencia. Vino de América casi a la ventura. Su madre le envió a
Europa para educarse y para heredar. Si esto último no fue logrado, en
cambio su nueva patria heredó de él abundantes bienes de la mejor
calidad. Pertenecía a la célebre empolladura del colegio de San Mateo,
donde dos retóricos eminentes sacaron una robusta generación de poetas.
Antes de ser derrocador de tiranos fundó la academia del \emph{Mirto},
cuyo objeto era hacer versos, y allí entre sáficos y espondeos nació el
complot \emph{numantino}; que en España, ya es sabido, se pasa
fácilmente de las musas a la política.

El segundo, Pepe, tenía quince años. Nació en un camino, entre el
estruendo de un ejército en marcha; arrullaron su primer sueño los
cañones de la guerra de la Independencia. Creció en medio de soldados y
cureñas, y a los cinco años montaba a caballo. Sus juguetes fueron
balas. Ya mozo, era mediano de cuerpo, y agraciado de rostro, en lo
moral generoso, arrojado hasta la temeridad, ardiente en sus deseos,
pobre en caudales, rico en palabra, cuando triste tétrico, cuando alegre
casi loco. Educose también en San Mateo con los retóricos y desde
aquella primera campaña con los libros, le atormentaba el anhelo de
cosas grandes, bien fueran hechas o sentidas. Los embriones de su genio,
brotando y creciendo antes de tiempo con fuerza impetuosa, le exigieron
acción, y de esta necesidad precoz salió la sociedad \emph{numantina}.
También le exigían arte, y por eso en las sesiones de la asamblea
infantil, a Pepe le salía del cuerpo y del alma, en borbotones, una
elocuencia inocentemente heroica que entusiasmaba a todo el concurso. Él
no pedía niñerías, ni aspiraba a nada menos que a \emph{quebrantar las
cadenas que oprimían a la patria}, empresa en verdad muy humanitaria y
que iba a ser realizada en un periquete.

El tercero, Patricio, tenía como \emph{Veguita} diez y ocho años. Se le
contaba por lo tanto entre los respetables.

Era formalillo, atildado, de buena presencia, palabra fácil y fantasía
levantisca y alborotada. Sentía vocación por las armas y por las letras,
y lo mismo despachaba un madrigal que dirigía un formidable ejército de
estudiantes en los claustros de doña María de Aragón. También era
orador, que es casi lo mismo que ser español y español poeta. En los
\emph{Numantinos} asombraba por su energía y el aborrecimiento que tenía
a todos los tiranos del mundo. Insistía mucho en lo de hacer trizas a
Calomarde, medio excelente para llegar después a la pulverización
completa de la tiranía.

Las reuniones se celebraban en una botica de la calle de Hortaleza las
más de las veces, otras en una imprenta, y cuando había olores de
persecución toda \emph{Numancia} se refugiaba en una cueva de las que
había en la parte inculta del Retiro no lejos del Observatorio. Los
mayores de la cuadrilla no pasaban de veinte abriles: estos eran los
ancianos, \emph{expertos}, o \emph{maestros sublimes perfectos}; que, a
decir verdad, la pandilla gustaba de darse ciertos aires masónicos, sin
lo cual todo habría sido muy soso y descolorido.

Si aquello no era inocente lo parecía, porque a lo mejor, los enemigos
del Tirano, bien se hallaran en la botica, bien en la novelesca cueva
del Retiro, se distraían sin saber cómo de su misión heroica y se ponían
a acertar charadas y a representar comedias. Otras veces, cuando alguno
de ellos tenía dineros, cosa muy extraordinaria y fuera de lo natural,
alquilaban borricos y se iban en escuadrón por las afueras, dando
costaladas y buscando aventuras que siempre concluían con alguna pesada
chanza de Pepe.

Fuera o no pueril la sociedad \emph{Numantinos}, lo cierto es que
Calomarde la descubrió y puso la mano en ella, dando con todos los
chicos en la cárcel de corte, y metiendo más ruido que si cada uno de
ellos fuese un Catilina y todos juntos el mismo Averno. La importancia
que dio aquel gobierno menguado y cobarde a la conspiración infantil
puso en gran zozobra a las familias. Se creyó que los más traviesos iban
a ser ahorcados, y había razón para temerlo, pues quien supo ahorcar a
hombres y mujeres, bien podía hacer lo mismo con los muchachos, que era
el mejor medio para extirpar el liberalismo futuro.

Mas por fortuna Calomarde no gustó de hacer el papel de Herodes, y
después de tener algunos meses en la cárcel a los que no se salvaron
huyendo, les repartió por los conventos \emph{para que aprendieran la
doctrina}.

\emph{Patricio} se escapó a Francia. A Pepe me le enviaron al convento
de franciscanos de Guadalajara, y a \emph{Veguita} le tuvieron recluso
en la Trinidad de Madrid. Esta prisión eclesiástica fue muy provechosa a
los dos, porque los frailes les tomaron cariño, les perfeccionaron en el
latín y en la filosofía, y les quitaron de la cabeza todo aquel fárrago
masónico numantino y el derribo de tiranías para edificar repúblicas
griegas.

\hypertarget{vi}{%
\chapter{VI}\label{vi}}

Lo azaroso de los tiempos traía entonces mudanzas muy bruscas en todo, y
las pandillas variaban a menudo, modificadas por las muertes y
destierros. En 1827 echábase de menos a \emph{Patricio}, que estaba en
París, y a \emph{Pepe} que perseguido nuevamente por sus calaveradas se
había marchado a Lisboa con muchas ilusiones y pocas pesetas, que por
cierto arrojó al mar en la boca del Tajo. Quedaba \emph{Veguita}, a
quien hallamos siendo núcleo de una nueva cuadrilla. Ya no se ocupaba de
política inocente. La juventud abría los ojos, columbrando la grandeza
lejana de sus destinos. ¡Generación valiente, en buen hora naciste!

Junto a \emph{Veguita} hallamos a un joven riojano y por añadidura
tuerto que hacía ya las comedias más saladas que podrían imaginarse.
Había sido primero soldado raso y después empleado en los tres años, con
su impurificación correspondiente el 24. Tenía las chuscadas más
ingeniosas y las ocurrencias más felices. Hablaba mejor en verso que en
prosa y montaba mejor en el Pegaso que en un burro alquilón, pues
restablecido en la partida el uso de las expediciones asnales, nuestro
soldado poeta apenas sabía tenerse sobre la albarda. Era el mismo
demonio para contar cuentos y para buscar consonantes, siendo tal en
esto su destreza que no le arredraban los más difíciles y enrevesados.

El más notable, después de estos, era un muchacho que hacía muy malos
versos y no muy buena prosa, medio traductor de Homero, casi abogado,
casi empleado, casi médico, que había empezado varias carreras sin
concluir ninguna. Sabía lenguas extranjeras. Tenía veinte años, y en tan
corta edad había pasado de una infancia alegre a una juventud taciturna.
Tan bruscas eran a veces las oscilaciones de su ánimo arrebatado en un
vértigo de afectos vehementes, que no se podía distinguir en él la risa
del llanto, ni el dudoso equívoco de la expresión sincera. Había en su
tono y en su lenguaje un doble sentido que aterraba y un epigramático
gracejo que seducía. Era pequeño de cuerpo y bien proporcionado de
miembros. A su pelo muy negro acompañaban bigote y barba precoces, y su
color era malo, bilioso, y sus ojos grandes y tristes. Tenía mala boca y
peores dientes, lo cual le afeaba bastante. Fumaba sin descanso, como si
padeciera una sed de humo, que jamás podía aplacarse, y era en su vestir
pulcro, elegante y casi lechuguino.

Educado en Francia, afectaba a veces desprecio de su nación y la
censuraba con acritud, quejándose de ella como el prisionero que se
queja de la estrechez incómoda de su jaula. Frecuentemente, después de
alborotar en el grupo de un café con palabras impetuosas o mordaces, se
retiraba a un rincón rechazando toda compañía, o despidiéndose a la
francesa, huía. Después de largas ausencias tornaba a la pandilla con
humor hipocondríaco.

Daba su opinión sobre poesía y literatura con un aplomo y una
originalidad de juicios que pasmaba a todos. Ni \emph{Veguita} ni el
tuerto autor de comedias tenían conocimiento, por lo que sus maestros de
aquí les enseñaban, de aquel nuevo y peregrino modo de juzgar, buscando
el fondo más bien que la forma de las obras. Pero cuando nuestro
atrabiliario quería echarse a poeta, los mismos que le admiraban como
juez, se reían en sus barbas diciéndole que \emph{una cosa es predicar y
otra dar trigo}. Por mucho tiempo fue objeto de risa y chacota su oda a
los Terremotos de Murcia, que es de lo peor que en nuestra lengua se ha
escrito. Cuando se anunció que la reina Cristina estaba en cinta, todos
los poetas echaron otra vez mano a la lira, y el hipocondriaco endilgó
su soneto

\small
\newlength\mlenb
\settowidth\mlenb{Guarda ya el seno de Cristina hermosa}
\begin{center}
\parbox{\mlenb}{\textit{Guarda ya el seno de Cristina hermosa     \\
                vástago incierto de alta dinastía...}}            \\
\end{center}
\normalsize

Verdad es que no eran mucho mejores los que al mismo asunto compusieron
\emph{Veguita} y el autor de comedias.

Había en la pandilla otros muchos chicos. De ellos algunos no serán
mencionados en razón de la oscuridad en que siempre han vivido, otros lo
serán más tarde cuando las necesidades de esta verídica historia lo
reclamen.

Reuníanse primero en el café de Venecia y después en el del Príncipe,
que desde entonces sacó el nombre de \emph{Parnasillo}. Entonces la
juventud no tenía más que dos medios para dar desahogo a su ardor y eran
hacer versos o hacer diabluras. Los estudios estaban muertos, la prensa
no existía, las letras mismas y el teatro principalmente yacían
encadenados por una censura bestial y vergonzosa, el conspirar olía a
cáñamo, la política era patrimonio de las camarillas, las bellas artes,
música y pintura estaban en su primera alborada. Los muchachos que no
sentían gusto por los soeces ejercicios de la tauromaquia se entretenían
en trepar por las asperezas del Olimpo, y como la mayor parte carecían
de estro, no tenían más recurso que la murmuración y las travesuras. De
todas las musas, la que más andaba entre los de la pandilla, tratándoles
de tú, era la \emph{Décima}, por otro nombre \emph{el hambre}, a quien
\emph{Veguita} dedicó una composición muy chusca. Sin dinero, sin
ocupación, sin estímulo, aquellos insignes poetas o prosistas o simples
mortales vivían de la poderosa fuerza íntima que en unos era la
fantasía, en otros la conciencia de un gran valer y en todos el presagio
de que habían de ser principio y fundamento de una generación fecunda.

Todo cansa en el mundo, hasta el hacer versos. Así es que no podían
satisfacer al bullidor espíritu de tales muchachos las sesiones del
\emph{Parnasillo} y el ardiente disputar sobre odas, comedias y poemas.
La juventud necesita acción, necesita el elemento dramático de la vida,
sin el cual esta no es más que un soliloquio de dolor o un quietismo
morboso. La juventud de aquel tiempo, la más ilustre que había tenido
España desde que envejeció la gran pléyade del siglo {\textsc{xvii}}, no
sabía vivir sin drama. Es verdad que había amores y de lo fino, pero las
aventuras galantes no podían satisfacer completamente a aquella juventud
que era la empolladura de una gran época. Si la hubiesen dejado, ella
habría hecho revoluciones, derribado gobiernos, aplastado ídolos entre
el tumulto estrepitoso de millares de discursos. Sentía en sí, mezclado
con la facultad y con la facilidad versificante, el germen de la
gloriosa oratoria parlamentaria, que en nuestra tierra y en nuestro
genio es una especie de poesía combatiente. En España es común que el
fuego de las ambiciones rompa las liras para forjar con ellas las
espadas.

La acción, que era una necesidad, un apetito irresistible de la insigne
pandilla, estaba circunscrita por Calomarde a la esfera del
\emph{Parnasillo}. La policía no estorbaba que allí dentro se dispararan
ovillejos, quintillas y décimas, llenas de pimienta como los antiguos
vejámenes; pero el libro, el drama, el periódico, todas las grandes
armas del pensamiento, les estaban vedadas. No se les permitía más que
los alfileres.

Su instinto de grandes empresas con la palabra o con la acción les
llevaba derechamente a las travesuras, y aquellos rapaces inspirados se
ocupaban de noche en salir por ahí a romper faroles y a dar bromazos a
los vecinos pacíficos. ¡Romper un farol! ¡Cuántas delicias, cuánto
ingenio, cuánta charla preparatoria y cuántos trámites para obra tan
divertida! Escogida por el día la inocente víctima, bien por la
diafanidad relativa de sus vidrios, bien por hallarse próxima a
cualquier casa de habitantes pusilánimes, se le formaba causa criminal.
Uno defendía en toda regla al farol, alegando sus buenos servicios, otro
le acusaba probando su complicidad en las tinieblas de la calle, o por
el contrario el robo que había hecho de los rayos del sol. Después de
consultar toda la jurisprudencia farolística recaía sentencia en verso,
y se nombraba la comisión ejecutiva. Por la noche un repentino estruendo
y el salpicar de los vidrios rotos anunciaba el terrible cumplimiento de
la justicia, y con la oscuridad, la alarma de los vecinos y la
intromisión de algunos de estos en la gresca, venían nuevas trapisondas
y al cabo palos y carreras.

Otras veces se entretenían en llamar con fuertes aldabonazos a las
puertas, y daban aviso a media docena de médicos, diciéndoles con mucho
apuro que tal o cual enfermo se hallaba en crisis. Enviaban la partera a
casa de quien menos la necesitaba y la caja de muerto a quien gozaba de
excelente salud.

Desde Santa Catalina hasta la Cuaresma, menudeaban entonces las
reuniones de máscaras, diversión que prevalece en épocas de poca
libertad. Eran célebres y vistosas las de Aristizábal, Commoto y
Mariátegui, familias ricas y que recibían y obsequiaban en el tono y
forma de la urbanidad moderna. Pero el españolismo rancio tenía tantas
raíces que las tertulias de aquella especie eran señaladas y aun puestas
en ridículo por los enemigos de los cumplimientos, partidarios de la
antigua llaneza ramplona, de quien eran secuaces la incomodidad, el
desaseo, los modales burdos y la grosería.

Entre las pocas tertulias donde no imperaba el españolismo rancio, había
una, que era sin duda la más agradable de todas. No ha llegado su fama
hasta nuestros días; pero esto no importa ni hace al caso, toda vez que
apenas hemos tenido, como los tuvo Francia, \emph{salones} célebres que
fueran centro de hábiles tramas políticas. La tertulia o salón de Doña
Jenara, que tal nombre se le daba, no tuvo importancia mayor como centro
político ni podía tenerla en aquellos días; no era tampoco de primer
orden por la riqueza de su dueña, y sus únicas preeminencias consistían
en el buen gusto, en el trato amable, festivo, ligero y exquisitamente
urbano, tan distante de la afectada etiqueta como de la llaneza, en lo
exquisito de los manjares, en la comodidad del servicio de estos, en la
libertad un tanto excesiva de los juegos de azar, y principalmente en la
chispa inagotable de la charla ingeniosa, rica en intención y en
travesura. Era opinión común que allí no entraban los tontos. Concurrían
a la tertulia menos mujeres que hombres. De los poetas nuevos no faltaba
uno, y de la gente antigua y machucha iba toda la turbamulta volteriana.

No quiere decir esto que la tertulia fuese un centro liberalesco, ni el
volterianismo significaba de modo alguno entonces ideas avanzadas en
política; por el contrario los más heterodoxos eran comúnmente los más
\emph{cangrejos}, como solía decirse. Si algún color político dominaba
en las reuniones era el absolutista tolerante o ilustrado, el ideal
monárquico con Carta a lo Luis XVIII, habilidosa componenda de donde en
tiempos más próximos había de salir el Estatuto, y luego los moderados,
doctrinarios, etc.

La dueña de la casa parecía complacerse en sostener equilibrio perfecto
entre el elemento apostólico y el reformista, pues ambos tenían algún
adalid en sus tertulias. Pero no todo era política. Casi casi las tres
cuartas partes del tiempo se invertían en leer versos y hablar de
comedias, y la música no ocupaba el último lugar. Después que algún
aficionado tocaba al clave una sonatina de Haydn o gorjeaba un aria de
la \emph{Zelmira} cualquier italiano de los de la compañía de ópera,
solía el ama de la casa tomar la guitarra, y entonces\ldots{} No hay
otra manera de expresar la gracia de su persona y de su canto sino
diciendo que era la misma Euterpe, bajada del Parnaso para proclamar el
descrédito del plectro y hacer de nuestro grave instrumento nacional la
verdadera lira de los dioses.

Era hermosa sobre toda ponderación y mujer de historia. Estaba separada
de su esposo y no se le conocían desvaríos. Si alguien se aventuraba a
hablar de cosas que ofendieran su buen nombre, era tan por lo bajo que
aquellos vientecillos de murmuración apenas salían de un pequeño
círculo. Había viajado mucho y hablaba el francés con perfección, cosa
que ya era de grandísimo valor entre los elegantes. Existían en su vida
pasajes misteriosos que nadie acertaba a explicar bien, y que, por el
mismo misterio, se trocaban en dramáticos; y finalmente, mariposeaban en
torno a ella muchos individuos con pretensiones de cortejos; pero aunque
a todas horas le echaban memoriales de suspiros o de galanterías, no dio
ocasión a ninguno para que se creyera favorecido.

La danza no podía faltar en las tertulias. ¡Ah!, entonces el baile era
baile, un verdadero arte con todos los elementos plásticos que le
hicieron eminente en Oriente y Grecia, por donde parece natural mirarle
como antecesor de la escultura. Entonces había caderas, piernas,
cinturas, agilidad, pies y brazos; hoy no hay más que armazones
desgarbadas dentro de la funda negra del traje moderno.

Al ver en estos últimos años a ciertos hombres eminentes que han sido (y
los que viven lo son todavía) el \emph{summum} de la gravedad en la
magistratura, en la política y en el ejército, y al mirarles, repetimos,
ora en el sillón presidencial del Senado, ora en el banco azul, ya
vestidos con la toga de la justicia, ya con el respetabilísimo uniforme
de generales, no hemos podido tener la risa considerando que vimos a
esos mismos señores dando brincos y haciendo trenzados en el salón de
doña Jenara con el más loco entusiasmo.

La política se trataba en aquella casa con toda la discreción que la
época exigía. Ninguno de los sucesos que ocuparon la atención pública
desde 1829 a 1831 dejó de tratarse allí, mezclándose los exteriores con
los de casa, según los traía la revuelta corriente del tiempo. Allí se
dijo cuanto podía decirse de la trascendentalísima Pragmática Sanción
del 29 de Marzo del 30, origen inmediato de varias guerras crueles,
pretexto de esa horrible contienda histórica, secular, característica
del genio español del siglo {\textsc{xix}} y que no ha concluido, no,
aunque así lo indiquen las treguas en que el pérfido monstruo toma
aliento.

Esa batalla grandiosa en que han peleado con saña los ideales más
hermosos y las tradiciones poéticas, los entusiasmos más firmes y las
ranciedades más respetables, los intereses más nobles y los más
bastardos, mezclándose en una y otra parte el legítimo anhelo de la
reforma con la terquedad de la costumbre, el generoso vuelo del
pensamiento con la noble exaltación de la fe; esa batalla, digo, estaba
trabada hace tiempo en el corazón y en el pensar de España y tarde o
temprano había de venir al terreno de las armas. Así tenía que ser por
ley ineludible. Quiso el cielo que nuestra revolución fuera larga,
sangrienta, toda compuesta de fieros encuentros, heroísmos, infamias y
martirios, como una gran prueba; quiso que se desataran las pasiones en
una guerra sin fin, empezada, concluida y vuelta a empezar y concluir en
larga serie de años de zozobra.

Hay pueblos que se transforman en sosiego, charlando y discutiendo con
algaradas sangrientas de tres, cuatro o cinco años, pero más bien
turbados por las lenguas que por las espadas. El nuestro ha de seguir su
camino con saltos y caídas, tumultos y atropellos. Nuestro mapa no es
una carta geográfica sino el plano estratégico de una batalla sin fin.
Nuestro pueblo no es pueblo sino un ejército. Nuestro gobierno no
gobierna: se defiende. Nuestros partidos no son partidos mientras no
tienen generales. Nuestros montes son trincheras, por lo cual están
sabiamente desprovistos de árboles. Nuestros campos no se cultivan, para
que pueda correr por ellos la artillería. En nuestro comercio se
advierte una timidez secular originada por la idea fija de que
\emph{mañana} habrá jaleo. Lo que llamamos paz es entre nosotros como la
frialdad en física, un estado negativo, la ausencia de calor, la tregua
de la guerra. La paz es aquí un prepararse para la lucha, y un ponerse
vendas y limpiar armas para empezar de nuevo.

Pues esta guerra, esta inquietud que ha llegado a ser en la madre patria
como un crónico mal de San Vito, se declaró abiertamente, después de
ciertos amagos, cuando se quiso averiguar quién sucedería en el trono a
nuestro amado soberano, toda vez que era creencia general que se nos
moriría pronto. Felipe V establece la ley Sálica y Carlos IV la deroga
en secreto. Fernando VII quiere hacerlo en público y lo hace. El
problema terrible, o sea la rivalidad de las dos ideas cardinales,
encuentra al fin un hecho en que encarnarse, la sucesión. Tradición y
libertad se miran y aguardan con mano armada y corazón palpitante lo que
dirá la esfinge. La esfinge en aquellos críticos días es una reina en
cinta.

¿Varón o hembra? He aquí la duda, la pregunta general, la esperanza y el
temor juntos, la cifra misteriosa. Cuando llegó el día 10 de Octubre de
1830, día culminante en nuestra historia, y retumbó el cañón llevando la
alegría o el miedo a todos los habitantes de la Villa, el ingenioso
cortesano de 1815, D. Juan de Pipaón, entró sofocado y sudoroso en casa
de Jenara. Venía sin aliento, echando los bofes, con la cara como un
tomate, por la violencia del correr y de las emociones.

---¿Qué?\ldots{} ¿qué es?---preguntó Jenara con calma.

Pipaón se dejó caer en un sofá y dándose aire con el pañuelo exclamó:

---¡Hembra!\ldots{} España es nuestra.

---¡Hembra!---repitió Jenara.---¡Pobre España!

\hypertarget{vii}{%
\chapter{VII}\label{vii}}

Excusado es decir que las fiestas sucedieron a las fiestas, que a la
alegría oficial correspondió la del inocente pueblo y que la inmensa
mayoría de este no comprendió la importancia extraordinaria del suceso,
origen de tanto cañoneo y regocijos tantos. Se había arrojado la moneda
al juego de \emph{cara o cruz} y había salido \emph{cara}. Los de la
\emph{cruz} estaban como es fácil suponer. Había que oírles en sus
camarillas, conventículos y madrigueras oscuras. No se hablaba más que
de las Partidas, del Auto acordado y de la Pragmática Sanción, y la
palabra \emph{legitimidad} se escribió en la oculta bandera.

Luego que Jenara y Pipaón dijeron lo que escrito queda, empezaron a
llegar a la casa los amigos, unos contentos, otros reservados. Aquella
misma noche leyeron algunos poetas los versos en que celebraban el feliz
alumbramiento de la hermosa reina, y la señora de la casa obsequió a
todos con espléndido \emph{ambigú}, en el cual hubo tanta alegría y
abundancia tal de exquisitos vinos, que algunos salieron a la calle con
más soltura de lengua y más flaqueza de piernas de lo que fuera
menester.

Por mucho tiempo los temas de política extranjera cedieron en la
tertulia ante el grave tema de nuestros negocios. Ya no se habló más de
la revolución de Julio en Francia, asunto socorridísimo que dio para
todo el verano y otoño, ni del nuevo reinillo de Grecia, ni del
reconocimiento de Luis Felipe, ni de Polonia, ni aun siquiera del famoso
decreto de 1º de Octubre, en el cual, para acabar más pronto con los
llamados \emph{negros}, se condenaba a muerte a todo el género humano o
poco menos. Y la causa de esta barrabasada draconiana fue que el buenazo
de Luis Felipe, viendo que aquí no le querían reconocer como Rey de los
franceses, abrió la frontera a los emigrados y aun dícese que les dio
auxilio y adelantó algunos dineros. Ellos que necesitaban poco para
armarla, cuando se vieron protegidos por el francés, asomaron impávidos
por diversas partes del Pirineo. Mina Valdés y Chapalangarra,
acompañados de López Baños, Jáuregui Sancho y otros andantescos de la
revolución aparecieron por Navarra. Cataluña vio en sus riscos a Milans
y a Brunet, y por Roncesvalles vinieron Gurrea y Plasencia. En Gibraltar
los más temibles aguardaban coyuntura para hacer un desembarco. Pero
todos estos amagos no pasaron adelante. El gobierno acabó pronto con
todas las partidas, y habiendo caído en la cuenta de que debía reconocer
a Luis Felipe, hízolo así, y Francia cerró la frontera. De este modo ha
jugado siempre la buena vecina con nuestras discordias, y lo mismo será
mientras haya discordias, emigrados y fronteras.

Muchas particularidades desconocidas del público y aun del gobierno en
las frustradas intentonas, fueron sabidas de los tertulios de Jenara. En
la casa de esta había un grupo que solía reunirse a solas presidido por
la señora, y en él la confianza y la amistad habían apretado sus dulces
lazos. Allí solían leerse algunas cartas venidas de Francia, no
ciertamente con intento de conspirar, sino como mensajes de cariño. Vega
(a quien ya no es conveniente llamar \emph{Veguita}) contaba que Pepe
Espronceda había estado en la frontera batiéndose al lado del bravo y
desgraciado Chapalangarra. Todo lo sabía Ventura por una carta que
recibió en Noviembre y en la cual se referían las aventuras que le
salieron a Espronceda desde que entró en Lisboa hasta que pasó el
Pirineo, las cuales eran tantas y tan maravillosas que bastaran a
componer la más entretenida novela de amores y batallas.

En Lisboa le metieron en un pontón donde se enamoró de la hija de cierto
militar compañero de encierro. Este le parecía ya más que cárcel un
paraíso, cuando me le cogieron y embarcándole en un pesado buque, me le
zamparon en Londres. Allí vivió, mejor dicho, murió algún tiempo de
tristeza y desesperación, cuando cierto día en que acertó a pasar por el
Támesis vio que desembarcaba su amada. Días felices siguieron a aquel
encuentro; pero cuáles serían las aventuras del poeta que tuvo que salir
a toda prisa de Inglaterra y huir a Francia, donde encontró a muchos
emigrados, y juntándose con ellos y con estudiantes y periodistas,
empezó a alborotar en los clubs. Vinieron las célebres ordenanzas de
Polignac contra los periódicos. Ya se sabe que de las ruinas de la
prensa nacen las barricadas. Espronceda se batió en ellas bravamente, y
sucio de pólvora y fango respiró con delicia y gritó con entusiasmo
viendo por el suelo la más venerada monarquía del mundo, que con toda su
veneración había caído ya tres veces con estruendo y pavor de toda
Europa.

Espronceda no se contentaba con libertar a Francia. Era preciso libertar
también a Polonia. Entonces era casi una moda el compadecer al pueblo
mártir, al pueblo amarrado, desnacionalizado, cesante de su soberanía.
La cuestión polaca fue llevada al sentimentalismo, y al paso que se
hicieron innumerables versos y cantatas con el título de \emph{Lágrimas
de Polonia}, se formaban ejércitos de patriotas para restablecer en su
trono a la nación destituida. El que cantó al Cosaco se alistó en uno de
aquellos ejércitos, que en honor de la verdad más tenían de
sentimentales que de aguerridos. Pero afortunadamente para el poeta,
Luis Felipe que como Rey nuevecito quería estar bien con todo el mundo,
incluso con los rusos, prohibió el alistamiento. A la sazón el banquero
Lafitte daba (con mucho sigilo se entiende), dinero y armas a los
emigrados españoles para que vinieran a meter cizaña a la frontera. En
esto era correveidile del francés que deseaba probar a España los
inconvenientes de no reconocer a los reyes nuevos. Espronceda, que se
ilusionaba fácilmente como buen poeta, al ver los aprestos de la
emigración creyó que ya no había más que entrar, combatir, avanzar,
ganar a Madrid, repetir en él las jornadas de Julio y quitar a Fernando
el dictado de rey de España para llamarle \emph{de los españoles},
trocándolo de absoluto y neto en soberano popular, \emph{bourgeois},
\emph{bonnet de coton} o como quisiera llamársele. Ya se sabe el término
que tuvieron estas ilusiones. Después de las escaramuzas quedamos, con
el sanguinario decreto de Octubre, más absolutos, más netos, más
apostólicos, más \emph{narizotas} y más \emph{calomardizados} que antes.

Si Vega y otros de los tertulios recibían de peras a higos alguna carta,
Jenara las tenía constantemente y con puntualidad, cosa notable en un
tiempo en que la correspondencia o no circulaba o circulaba después que
la paternal policía se enteraba bien de su contenido para evitar
camorras. La correspondencia de Jenara se salvaba por mediación del gran
Bragas, que la sacaba incólume del correo, y al mismo tiempo recibía de
él numerosas confidencias de sucesos más o menos misteriosos. De estas
confidencias muchas no le servían para nada, otras las utilizaba para
favorecer a los amigos que caían en desgracia del gobierno, y de todas
tomaba pie para burlarse a la calladita de Calomarde, personaje a quien
estimaba lo menos posible.

Habían pasado muchos días desde el nacimiento de la princesa de
Asturias, esperanza de la patria, cuando Pipaón fue a ver a Jenara y le
anunció con misterio que tenía que comunicarle cosas de importancia.

---O yo no soy quien soy---dijo sentándose junto a ella en el
gabinete,---o yo he perdido el olfato, o nuestro endemoniado amigo está
en Madrid.

---¿Será posible? ¡En Madrid!\ldots{} ¡qué locura!, ¡y sin ponerse bajo
nuestra protección!---exclamó la dama palideciendo un poco.

---Yo no le he visto; pero hay en Gracia y Justicia algunos datos que
permiten creer que está aquí\ldots{} Y no habrá venido seguramente a
matar moscas. Algún jaleo lindísimo traen entre manos esos bribones, que
no quieren dejarnos en paz. El Gobierno teme algo en Andalucía, por lo
cual no hay carta que no se abra ni vivienda que no se registre.
Manzanares, Torrijos y Flores Calderón andan por allá preparando algo, y
al fin, tanto va a la fuente el cántaro de la represión que en una de
estas se rompe\ldots{}

---¡Sangre\ldots{} horca!---dijo maquinalmente Jenara mirando al suelo.

---D. Tadeo pierde cada día su fuerza, y el Rey se está haciendo todo
mantecas a medida que la gente de orden y el respetabilísimo clero ponen
los ojos en el Infante, única esperanza de esta nación francmasonizada y
hecha trizas por el ateísmo. Ya no es nuestro Rey aquel hombre que se
ponía verde siempre que le hablaban de liberalismo. Con los achaques y
el mal de ojo que le ha hecho la Reina, pues el amor que le tiene parece
maleficio, está más embobado que novio en vísperas. Doña Cristina sabe a
dónde va y dulcifica que te dulcificarás, está haciendo la cama al
democratismo. Ya se habla de amnistía, de abrir la puerta a los lobos,
señora, y traernos otros tres añitos como los de marras.

Al decir esto, el ilustre D. Juan, inflamado en patriótica ira, dio un
porrazo en el suelo con la contera de su bastón, añadiendo luego:

---Pero no será, no será; que antes que doblar el cuello a las
melifluidades pérfidas de la napolitana, antes que dejarnos llevar por
ella a la ratonera liberalesca, echaremos a rodar Pragmática y Reina y
la \emph{áurea cuna de la angélica Isabel}, como dicen esos menguados
poetastros, y habrá aquí un Vesubio, señora, un Etna\ldots{}

La señora no le hizo caso y seguía meditando.

---Se levantará la nación---dijo el cortesano levantándose de la silla
para expresar emblemáticamente su idea,---y veremos cuántas son cinco.
Tenemos un príncipe varón, sabio, religioso, honesto; tenemos doscientos
mil voluntarios realistas que se beberán el ejército como un vaso de
agua, tenemos el reverendo clero con los reverendísimos obispos a su
cabeza; tenemos el apoyo de la Europa, que, fuera de la nación francesa,
marcha por las vías apostólicas. ¡Viva el señor Don\ldots!

---¡Silencio!---indicó la dama.---No me atormente usted con su
entusiasmo. Estoy de apostólicos hasta la corona y deseo que los
\emph{kirie-eleysones} del cuarto de D.~Carlos no lleguen hasta mi casa
trayéndome el olorcillo a sacristía que tanto me enfada\ldots{} Pasando
a otra cosa, ¿sabe usted que es temeridad venir a Madrid sin ponerse
bajo nuestro amparo?\ldots{} Yo le ofrecí mi protección para que
viniera\ldots{} Sin ella está en grandísimo peligro y tan bien ahorca a
Juan como a Pedro.

---Exactamente. ¿Pero le ha visto usted hacer cosa alguna que no fuera
temeridad, locura y disparate?

---Trabajo le doy a quien intente averiguar dónde está escondido---dijo
la dama sin cuidarse de disimular su inquietud.---¿Será posible
averiguarlo?

---Muy posible---repuso Pipaón soplando fuerte; que era en él signo
claro de legítimo orgullo.---Como que ya tengo si no averiguado, casi
casi\ldots{}

---¿De veras? Estará en casa de algún amigo.

---Que te quemas\ldots{} digo, que se quema usted.

---¿En casa de Bringas?

---No.

---¿En casa de Olózaga?

---Nones.

---¿En casa de Marcoartú?

---Requetenones\ldots{} En suma, señora mía, yo no sé fijamente dónde
está; pero tengo una presunción, una sospecha\ldots{}

---Venga\ldots{} Si no me lo dice usted pronto, le contaré a Calomarde
sus picardías.

---No por la amenaza de usted sino por mi cortesía y deseo de
complacerla le diré que me tendré por el más bobo, por el más torpe de
los cortesanos de este planeta si no resultase que nuestro temerario
trapisondista está en casa de Cordero.

---¡En casa de Cordero!

La dama pronunció estas palabras con asombro y quedó luego sumergida en
el mar de sus pensamientos, sin que los comentarios de Pipaón lograran
sacarla a la superficie.

---¿Estorbo?---dijo al fin el cortesano advirtiendo que la dama no le
hacía más caso que a un mueble.

---Sí---repuso ella con la franqueza que tanta gracia le daba en
ocasiones.

---¿Va usted de paseo?

---No\ldots{} me duele la cabeza\ldots{} Abur, Pipaón, no olvide usted
mis recomendaciones, a saber: la canonjía, la canonjía, Santo Dios, que
esos benditos primos me tienen loca\ldots{} la bandolera para el sobrino
del canónigo; que su familia no me deja respirar\ldots{} el pronto
despacho en la censura de teatros de ese nuevo drama traducido por el
busca-ruidos\ldots{} en fin, no sé qué más. Esto no es casa, es una
agencia.

Despidiose Pipaón después de prometer activar aquellos asuntos, y la
dama, al punto que se vio sola, empezó a vestirse con gran prisa y
turbación. Le había ocurrido que aquel día necesitaba de ciertos encajes
y no quería dilatar un minuto en ir a comprarlos.

\hypertarget{viii}{%
\chapter{VIII}\label{viii}}

A pesar de su amor a la vida inalterable y metódica, D. Benigno no veía
con gusto que transcurriese el tiempo sin traer cambios o novedades en
su existencia. Es que se había amparado del alma del héroe cierto
desasosiego o comezoncilla que le sacaba a veces de su natural índole
reposada. A menudo se ponía triste, cosa también muy fuera de su
condición, y sufría grandes distracciones, de lo que se asombraban los
parroquianos, los amigos y el mancebo.

En la casa no había más variaciones que las que trae consigo el tiempo:
los muchachos crecían, los pájaros se multiplicaban, los gatos y perros
se rodeaban de numerosa y agraciada prole, Crucita gruñía un poco menos
y Sola había engrosado un poco más.

De todos los amigos de Cordero el más querido era el buen padre Alelí,
de la orden de la Merced, viejísimo, bondadoso, campechano. Era de
Toledo como D. Benigno y aun medio pariente suyo. Le ganaba en edad por
valor de unos treinta años, y acostumbrado a tratarle como un chico
desde que Cordero andaba a gatas por los cerros de Polán, seguía
llamándole, por inveterado uso, \emph{chicuelo}, \emph{Don Piojo},
\emph{harto de bazofia}, \emph{el de las bragas cortas}. Cordero, por su
parte, trataba a su amigo con mucho desenfado y libertad, y como las
ideas políticas de uno y otro eran diametralmente opuestas y Alelí no
disimulaba su absolutismo neto ni Cordero sus aficiones liberalescas, se
armaba entre los dos cada zaragata que la trastienda parecía un
Congreso. Felizmentetoda esta bulla acababa en apretones de manos, risas
y platos de migas al uso de la tierra, rociadas con vino de Yepes o
Esquivias.

He aquí un modelo de conversación Alelí-Corderesca:

---Buenos días, Benignillo. ¿Cómo vas de \emph{régimen nefando}?

---Padre Monumento, vamos tal cual. Los del régimen se entretienen en
tirarse coces unos a otros y no se acuerdan de perseguirnos.

---Don Fulastre, don Piojo, el asno será él. ¿Sabes algo del nuevo Papa
que tenemos, Gregorio XVI, el cual, o no será tal Papa o no dejará un
Rey liberal en toda la Europa?

---¡Barástolis! No sé más sino que allá me las den todas y que le beso
las manos a mi señor Don Gregorio como católico que soy.

---¿Católico y jacobista? Átame esa mosca. Oye, tú, \emph{el de las
bragas cortas}; ¿qué pasaje leíste anoche?

---Tío Latinajo, leí el pasaje que dice: \emph{He visto en la religión
la misma falsedad que en la política. No hay religión, por buena que
sea, que no haya derramado sangre inocente.}

---Sigue, que me muero de risa. Eres un filósofo de agua y lana. Cuando
acabes de volverte loco con tu \emph{Emilio} saldremos a enseñarte en
las ferias a dos cuartos por barba. Ven acá, almacén de sandeces y
tienda de majaderías, ¿qué sabes tú lo que es religión?

---Me lo enseñan los de sayo y sandalia, a quienes se puede
decir\ldots{} «\emph{Je, je, son tontos y piden para las ánimas}».

---Cuando tú y tus amigos los liberales herejes os desocupéis de la
paliza que os están dando en toda la Europa, y soltéis el ronzal para
formar Congreso y decir: «señor presidente, pido el rebuzno», no faltará
quien os enseñe a hablar con respeto de las cosas sagradas.

---Día vendrá en que rompamos el ronzal, padre difinidor, y entonces
difiniremos la \emph{conventualla}, diciendo: \emph{Al fraile hueco,
soga verde y almendro seco}.

---También se dijo: \emph{Donde las dan las toman}.

---Y también: \emph{Cuentas de beato y uñas de gato}.

---¡Ah!, mercachifle, si fueras bueno no serías rico. Esas sí que son
uñas de gato, que es como decir de filósofo.

---No sé si se dijo por mí aquello de \emph{A la puerta del rezador
nunca eches tu trigo al sol}.

---Ladrón y rapante tú; mas no nosotros, que de limosna vivimos.

---¿De limosna, eh? ¡Ah!, señor \emph{D. Cepillo de Ánimas}, qué bien
dijo el que dijo: \emph{Reniego de sermón que acaba en daca}.

---Yo he oído que tienes la cabeza a pájaros.

---A propósito de pájaros. Yo he oído que el \emph{abad y el gorrión dos
malas aves son}.

---Mira, Benigno---dijo Alelí cuando el tiroteo llegaba a este
punto,---vete al mismo cuerno, y echa acá un cigarrillo.

Cordero alargó su petaca al fraile, diciéndole:

---A la paz de Dios. Viva mil años mi fraile.

---¿Cómo están hoy tus nenes?---preguntó Alelí encendiendo su
cigarro.---Lo de Rafaelillo resultó indigestión como te dije, ¿no es
verdad? Dale hojas de Sen y créeme.

---No sólo de Sen sino de Can y Jafet se las ha dado Cruz, que tiene en
casa el herbolario más completo de Madrid.

---¿Ha parido la podenca?

---Todavía, no; pero parirá su merced. Para ser un Retiro a esto no le
falta más que el estanque; que de animales y hierbas tenemos cuanto Dios
crió, sin que falte el león, que es mi hermana\ldots{} ¡Ah!, me
olvidaba: las perdices que traje ayer las están aderezando a la
toledana, a lo Castañar puro. Si viene usted tendremos para diez
perdices cuatro.

---¿Pues no he de venir, hombre de Dios? Sr.~D. \emph{Ladrón de
encajes}. No faltaba más sino desairar a la tierra\ldots{} ¿Hoy?

---Hoy. Además yo tengo que hablar con usted de un asunto grave.

Al decir esto, Cordero tomó un aire de seriedad y de temor, que puso en
gran curiosidad al padre Alelí.

---¿Un asunto grave? No será el primero que me consultas.

---Pero es seguramente el más delicado, el más peliagudo. Necesito
consejo y ayuda.

---Para eso estoy yo. Vengan esos cinco.

Se estrecharon las manos, y Cordero besó las flacas y temblorosas del
anciano fraile con mucho cariño.

---El mal camino andarlo pronto, y pues esto urge, tratémoslo ahora.

---Cuando quieras hijo. A bien que ambos somos toledanos y parientes.

---¡Viva la Virgen del Sagrario!---dijo Cordero con emoción.---Es
temprano: ahora viene poca gente. El chico se quedará en la tienda.
Subamos a mi cuarto y hablaremos.

---¿Es cosa larga?

---Primero una confesión, un secreto, que si no lo suelto pronto, creo
que me hará daño; después un consejo sobre lo que se ha de hacer, y por
último\ldots{} a ver si se luce el buen Padre \emph{Engarza-credos} con
una comisión delicada.

---Vamos, por el hábito que visto, que estoy curioso.

Salieron. Media hora después, D. Benigno y su amigo reaparecieron en la
trastienda. El comerciante traía el semblante alegre y las mejillas más
que de ordinario encendidas. Alelí movía su cabeza con más nerviosidad y
temblor que de ordinario, y al despedirse de su paisano, le dijo:

---Me parece muy bien, Benigno de mi corazón. Yo quedo encargado de
arreglarlo.

\hypertarget{ix}{%
\chapter{IX}\label{ix}}

Dulce melancolía inundaba el alma pura del buen Cordero. Parecíale que
todo lo de la tienda, incluso el feo hortera, concordaba con el estado
de su espíritu, tiñéndose de inexplicable color lisonjero, y que había
una sonrisa general en todo lo externo, como si cada objeto fuera espejo
en que a sí propio se miraba. Para más dicha, hasta hubo muchas ventas
aquel día, que fue, si no estamos mal informados, uno de los de Febrero
del año de 1831, al cual se podría llamar, como se verá más adelante, el
año sangriento.

Serían las once cuando entró en la tienda una dama y tomó asiento. Era
parroquiana y amiga. D. Benigno la saludó y al punto empezó a sacar
género y más género, blondas de Almagro, Valenciennes, Bruselas,
Cambray, Malinas, en tal abundancia y variedad que no parecía sino que
la señora iba a llevarse todo Flandes a su casa.

---¡Qué carero se ha vuelto usted!\ldots{} Ya no vuelvo más acá\ldots{}
Me voy a casa de Capistrana\ldots{} ¿Cincuenta y seis reales?, ¡qué
herejía!\ldots{} Esto no vale nada\ldots{} Es imitación\ldots{} Vaya una
carestía\ldots{} No doy más que tres onzas por todo.

---No es sino muy barato\ldots{} Por ser usted lo llevará en cincuenta
duros todo\ldots{} ¿Capistrana? No hay allí más que maulas,
señora\ldots{} Volverá usted por más\ldots{} Es legítimo de
Malinas\ldots{} lo recibí la semana pasada. Este encaje de Inglaterra me
cuesta a mí veinticuatro. Pierdo el dinero.

---Lo que pierde usted es la caridad\ldots{} ¡Santo Dios, cómo nos
desuella! Así está más rico que un perulero\ldots{} Con estos precios
que aquí usan, ¡ya se ve!, no es extraño que se compren casas y más
casas.

Tantos dimes y diretes concluyeron con que la dama pagó en buenas onzas
y doblones. Mientras Cordero empaquetaba las compras para mandarlas a la
casa de la señora, esta le preguntó si era cierto que se había hecho
propietario de la finca donde estaba la tienda, y como el encajero le
contestara que sí, la parroquiana aparentó alegrarse mucho diciendo:

---Precisamente estoy muy descontenta del cuarto en que vivo y deseo
mudarme. ¿No viven en este principal los de Muñoz? ¿No se van de Madrid?
Pues si dejan la casa yo la tomo.

---Mucho me alegraré---replicó el héroe.---Pero me figuro que mi
principal será pequeño para quien tanto lujo tiene y a tanta gente
recibe en sus tertulias.

---¡Oh!, no\ldots{} pienso reducirme mucho y vivir más para mí que para
los otros---dijo la dama con mucha gracia.---Estoy cansada de poetas, de
mazurcas y de chismes políticos. El Gobierno ha principiado a mirar con
malos ojos mis reuniones, a pesar de que mi absolutismo pasa por
artículo de fe. Ya sabe usted lo que es Calomarde y toda esa gente: van
de exageración en exageración\ldots{} están ciegos. El poder absoluto es
como el vino, una cosa muy buena y un vicio, según el uso que de él se
haga. No lo dude usted, esa gente está borracha, y mientras más bebe y
más se turba más quiere beber. El año comienza mal, y según dicen, las
conspiraciones arrecian y el Gobierno no se para en pelillos para
ahorcar.

---No faltará tampoco quien amanse y dulcifique---dijo Cordero apoyando
sus codos en el mostrador para atender mejor a un tema tan de su
gusto.---La Reina\ldots{}

---¡Oh!, sí, la Reina\ldots---exclamó la dama con ironía.---Sus
dulcificaciones, de que tanto se ha hablado, son pura música. Ya lo ve
usted, ha fundado un Conservatorio por aquello de que \emph{el arte a
las fieras domestica}. Me hace reír esto de querer arreglar a España con
músicas. Al menos el Rey es consecuente, y al fundar su escuela de
Tauromaquia, cerrando antes con cien llaves las Universidades, ha
querido probar que aquí no hay más doctor que Pedro Romero. Eso es,
dedíquese la juventud a las dos únicas carreras posibles hoy, que son
las de músico y torero, y el Rey barbarizando y la Reina dulcificando
nos darán una nación bonita\ldots{} ¡Ah!, me olvidaba de otra de las
principales dulcificaciones de Cristina. Por intercesión de ella ¡oh
alma generosa!, se va a suprimir la horca para sustituirla ¡enternézcase
usted, amigo Cordero!\ldots{} para sustituirla con el garrote\ldots{} No
sé si en el Conservatorio se creará también una cátedra de dar
garrote\ldots{} con acompañamiento de arpa.

D.~Benigno se rió de estas despiadadas burlas; mas lo hizo por pura
galantería, pues siendo entusiasta admirador de la joven y generosa
Reina, no admitía las interpretaciones malignas de su parroquiana.

---Ello es, querido D. Benigno ---añadió esta,---que yo he determinado
quitarme de en medio. Presiento no sé qué desgracias y persecuciones.
Deseo una vida retirada y oscura. No más tertulias, no más versos
dedicados a bodas reales, embarazos de reinas y nacimientos de
princesas, no más murmuración ni secreto sobre lo que no me importa. Si
su casa de usted me gusta, a ella me vengo y en ella me encierro\ldots{}
Decidido, señor de Cordero.

---Como buena y cómoda no hay otra en Madrid.

---Yo quisiera verla.

---Lo haré presente al señor de Muñoz y de seguro me dará permiso para
que usted la vea.

---No, no se moleste usted---dijo la dama, observando con atención el
rostro de Cordero, por ver si se turbaba.---¿No son iguales todos los
pisos?

---Todos enteramente iguales.

---Pues enséñeme usted el entresuelo donde usted vive\ldots{} Pero ahora
mismo. Tengo prisa. Quiero decidir de una vez.

Levantose resueltamente dirigiéndose a alzar la tabla del mostrador para
pasar a la trastienda. De aquel modo brusco y ejecutivo hacía ella todas
sus cosas.

---No hay inconveniente, señora---dijo Cordero manifestando más bien
agrado que contrariedad.---Pero la señora me permitirá que no la
acompañe, porque tendría que dejar la tienda sola. El chico no está.

---No faltaba más sino que también conmigo gastara usted cumplidos.
Quédese usted\ldots{} subiré sola, ya sé el camino\ldots{} por esta
escalerilla\ldots{}

---¡Sola!\ldots{} ¡Cruz!\ldots---gritó D. Benigno desde el primer
peldaño.

La dama subió con ágil pie por la escalera, la cual era tan estrecha que
en la angostura de las paredes se le chafaron a la señora las huecas
mangas de jamón, y el chal de cachemira se le resbaló de los hombros.

En aquel mismo momento Crucita estaba limpiando jaulas y soplando la
paja del alpiste, sin parar un momento en su conversación con todos los
pájaros, la cual era un lenguaje compuesto de suavísimas interjecciones
cariñosas, de voces incomprensibles, cuyas variadas inflexiones no
expresaban ideas, sino un vago sentimiento de arrullo o los apetitos y
anhelos del instinto. Era aquella charla como los rudimentos o albores
de la palabra humana cuando el hombre pegado aún a la Naturaleza por el
cordón umbilical de la barbarie, desconocía las relaciones sociales.
¡Oh!, ¡qué dato para aquel filósofo que tenía en D. Benigno el más
entusiasta de sus admiradores! Oyendo hablar a doña Crucita con los
habitantes enjaulados de su selva de balcón, Rousseau habría comprendido
mejor el estado feliz y perfecto del hombre, y su amigo Voltaire se
habría puesto de cuatro pies para practicar, no de burlas, sino de puras
veras, las teorías del autor del \emph{Contrato}.

Doña Cruz era una mujercita seca y bastante vieja, muy limpia, fuerte y
dispuesta como una muchacha, lista de pies y manos, con la cabeza medio
escondida dentro de una escofieta que parecía alzarse y bajarse con el
mover de la cabeza, como las moñas o tocas de ciertas aves. Para mirar
daba a la cara un brusco movimiento lateral, lo mismo que los pájaros
cuando están azorados o en acecho. Fuera por la asociación de ideas o
por verdadera semejanza, ello es que al verla daban ganas de echarle
alpiste.

Interrumpida en lo mejor de su faena, doña Cruz se escandalizó, se
asustó, aleteó un tanto con los bracitos flacos, miró de lado, graznó un
poquillo. Al mismo tiempo dos, tres o quizás cuatro perrillos se
abalanzaron a la dama, ladrando y chillando, rodeándola de tal modo que
si fueran mastines en vez de falderos, la dejarían malparada. La cotorra
y el loro ponían en aquel desacorde tumulto algunos comentarios roncos
que aumentaban la confusión. La dama expresó el objeto de su subida al
entresuelo, mas como Crucita no podía oírla, fuele preciso alzar la voz,
y con esto alzaron la suya los perros, mayaron los gatos, se enfadaron
cotorra y loro y los pájaros prorrumpieron en una carcajada estrepitosa
de cantos y píos. Mientras más gritaba la turba animalesca más se
desgañitaba doña Cruz diciendo: «¿Qué se le ofrece a usted? ¿Por quién
pregunta usted?» Y a cada subida del diapasón de la vieja más elevaba el
suyo la señora, mientras D. Benigno desde la escalera gritaba sin que le
escucharan: «¡Cruz! ¡Sola!» armándose tal laberinto que sin duda hubiera
parado en algo desagradable si no se presentara afortunadamente la
\emph{Hormiga} a desvanecer aquella confusión, imponiendo silencio y
enterándose de lo que la dama quería.

Sorprendida y algo cortada estaba Sola ante aquel brusco modo de ver
casas, y pasado el asombro primero dio en sospechar que otra intención
distinta de la manifestada tenía la dama. Aunque esta le inspiraba
miedo, por figurársele que su presencia le anunciaba alguna trapisonda,
quiso disimular su temor. Tan bien lo consiguió, que la señora empezó a
sorprenderse a su vez de hallar en la protegida de Cordero un semblante
tan festivo, un ánimo tan sereno y tal disposición a la complacencia,
que dijo para sí con despecho y tristeza:---O esta disimula mejor que
yo, o no hay aquí hombre escondido ni cosa que lo valga.

\hypertarget{x}{%
\chapter{X}\label{x}}

Vieron la casa toda, que la señora encontró más pequeña de lo que creía
y bastante oscura en lo interior. Después Sola, que no había tenido
tiempo de echarse un mantón por los hombros, ni aun de quitarse el
delantal, que era su librea de gala por las mañanas, acompañó a la
señora a la sala para que descansase y le pidió indulgencia por el mal
pergenio con que la recibía. Considerándose ella como una especie de ama
de gobierno más bien que como dueña de la casa, su posición frente a la
otra era, en verdad, un poco desairada. Pero no le importaba nada ser
allí un poco más o menos señora, y sentándose a cierta distancia de la
visitante, esperó a que Crucita o el mismo D.~Benigno vinieran a
relevarla de su señorío provisional. Crucita se había encerrado en el
gabinete para colgar las jaulas y echar agua a los tiestos, y no se
cuidaba de que hubiese o no en el estrado una persona extraña. Cordero
estaba vendiendo, y tampoco podía subir.

En cambio, Juanito Jacobo se adelantaba lentamente pegado a la pared y
rozándose con las sillas, como una babosa que marcha pegada a las
piedras de una tapia. Con el ceño fruncido, un dedo en la boca y ambas
manos teñidas con la pintura de un caballejo de palo, a quien acababa de
dar un baño en la cocina, miraba a Sola y a la otra señora, esperando
que cualquiera de ellas le llamase.

---¿Es este el niño más pequeño de D. Benigno?---preguntó la dama.

---Sí, señora\ldots{} ¡y es tan malo!\ldots{} Ven acá, chico, ven;
saluda a esta señora.

El muchacho no se hizo de rogar y vino con ademán de recelo y
azoramiento, metiéndose, no ya el dedo, sino toda la mano dentro de la
boca. La abundante pintura negra y roja que en los dedos tenía se le
pasó a los labios y carrillos.

---Estás bonito por cierto\ldots{} pareces un salvaje---le dijo
Sola.---¿No te da vergüenza de que te vean así, grandísimo tunante?

---No le riña usted.

---¡Eh!\ldots{} no te acerques a la señora con esas manazas
puercas\ldots{} Tira ese caballo, que está chorreando pintura. Le ha
dado ahora por lavar todo lo que encuentra, y el otro día metió en la
tinaja las gafas de su padre.

---Es un fenómeno de robustez esta criatura---afirmó la señora
acariciándole.

---Eso sí; está más sano que una manzana y come más que un
sabañón---dijo Sola apretándole una nalga y dándole un palmetazo en el
cogote para que por el chasquido de las carnazas del chiquillo juzgase
la señora de su robustez.

Parecía una madre en plena manifestación de su orgullo de tal.

Juan Jacobo miró a la señora con expresión de desvergüenza, la cual se
aumentaba con los manchurrones de su cara.

---¿Quieres mucho a esta señorita?---le preguntó la dama, dándole un
golpe con su abanico.

El muchacho, que apoyaba sus codos en la rodilla de Sola, alzó la pierna
para montarse arriba.

---No, no, fuera, fuera\ldots---dijo Sola quitándose de encima la
preciosa carga.---No faltaba más\ldots{} A fe que es chiquito el
elefante para llevarlo en brazos\ldots{} Quita allá, mostrenco.

---¿Un hombre como tú no tiene vergüenza de que le coja en brazos una
mujer?---le dijo la señora riendo.---¡Le tenemos tan
mimoso\ldots!---dijo Sola con naturalidad.---Como es el más
pequeño\ldots{} Su padre está medio bobo con él, y yo\ldots{}

No pudo seguir porque el muchacho, que era tan ágil como fuerte, saltó
de un brinco sobre las rodillas de Sola y echándola los brazos al cuello
la apretó fuertemente.

---Ya ve usted\ldots---dijo ella,---me tiene crucificada este
sayón\ldots{} Si le dejaran estaría así todo el día\ldots{} Vaya, vaya,
basta de fiestas\ldots{} Sí, sí, ya sé que me quieres mucho. Haz el
favor de no quererme tanto\ldots{} Abajo, abajo\ldots{} ¡Qué pensará de
ti esta señora! Dirá que eres un mal criado, un niño feo\ldots{}

---No extraño que los hijos de Cordero la quieran a usted
tanto\ldots---manifestó la dama.---Es usted tan buena, y les ha criado
con tanto esmero\ldots{} Así está D. Benigno tan orgulloso de usted, y
así no concluye nunca cuando empieza a elogiarla. ¡Cómo la pone en las
nubes!\ldots{} Y verdaderamente el amigo Cordero ha encontrado una joya
de inestimable precio para su casa. Yo creo que en el caso presente el
agradecimiento le corresponde a él más bien que a usted.

Sola protestó de esta idea con exclamaciones y también con movimientos
negativos de cabeza.

---¿Pues qué ha hecho usted sino sacrificarse?---añadió la dama.---Bien
podría vivir hoy, si lo hubiera querido, en otra posición, en otro
estado, que de seguro sería más independiente\ldots{} pero dudo que
fuera más tranquilo y feliz.

---No creo que para mí pudieran existir posición ni estado mejores que
los que ahora tengo---repuso la \emph{Hormiga} con sequedad.

---Verdaderamente así es, porque, si no recuerdo mal, usted se encontró
después de la muerte de su señor padre, sola y abandonada en el mundo.
Me parece haber oído que alguien la protegió a usted en aquellos días;
pero como andando el tiempo, ese alguien o se murió o desapareció o no
quiso acordarse más de usted, el resultado es, hija mía, que su orfandad
no ha tenido verdadero y seguro amparo hasta que este angelical D.
Benigno la trajo a su casa. En él tiene usted un padre cariñoso\ldots{}
¡Oh!, páguele usted con un cariño de hija y no busque fuera de esta casa
otros afectos ni otro estado de mejor apariencia. Cuidado con casarse;
no cambie usted el arrimo de este santo varón por el de cualquier
hombrecillo que no sepa comprender su mérito.

Siguió apurando el tema la señora y vino a parar en una filípica contra
los hombres, sin especificar si la merecían en el concepto de maridos o
en el de novios o cortejos; pero deteniéndose de repente, se echó a
reír.

---Mas usted dirá que le doy consejos sin que me los pida y que hablo de
lo que no me importa.

---No, señora; todo lo que usted dice me parece muy puesto en razón, y
es natural que dé el consejo quien tiene la experiencia\ldots{} Estate
quieto, por amor de Dios, chiquillo\ldots{}

---Bien, bien---dijo la dama riendo otra vez.---En fin, señora, yo estoy
molestando a usted y quitándole el tiempo\ldots{}

---De ningún modo.

Levantáronse ambas.

---Tiene una hermosa sala el amigo Cordero---indicó la señora alargando
la mano a Sola, y observando al mismo tiempo las cortinas blancas, las
rinconeras, los candeleros de plata y las plumas de pavo real.---La
parte de la casa que da a la calle me parece muy bonita\ldots{} En fin,
en mí tiene usted una servidora\ldots{} Adiós, hermoso; dame un
beso\ldots{} ¡Ah!, ¿no sabe usted lo que me ocurre en este momento?

La señora que ya iba en camino de la puerta, se detuvo, retrocedió
algunos pasos y mirando a Sola fijamente, le dijo así:

---Me olvidaba de hacer a usted una pregunta.

Sola esperó, palideciendo un poco, por sentir corazonada de que la tal
pregunta iba a ser de cosa triste. Su instinto zahorí lo adivinaba y
parecía leer en los ojos de la hermosa dama la pregunta misma con todas
sus palabras antes de que la primera de estas fuese pronunciada.

---Dígame usted---preguntó la señora, afectando poco interés,---aquel
caballero, aquel joven, aquel, en fin, a quien usted llamaba su hermano,
¿dónde está?

---No lo sé, señora---replicó Sola pasando bruscamente de la palidez al
rubor.---Hace tiempo que no sé nada.

---¿Vive, o qué es de él?

---No sé una palabra. Hace dos años que no me escribe\ldots{} ¿Usted
sabe algo?

El rubor desapareció en ella dejándola en su natural color y aspecto
tranquilo.

---Dos años justos hace que tampoco sé nada\ldots{} Es muy
particular\ldots{}

Para la astuta dama no pasó inadvertida la circunstancia de que si la
joven se turbó al recibir la primera impresión de la pregunta, supo
contestar con serenidad a ella. Ya fuese por disimulo, ya porque
realmente se interesaba poco por el personaje recordado tan bruscamente,
no se afectó como la otra creía.

---O está aquí---pensó la dama,---y la muy pícara lo oculta con
admirable disimulo, o si no está, ella no se cuida ya de él para maldita
la cosa.

---Quiero ser franca con usted---dijo después de ligera pausa, en que la
miró a los ojos como se miraría en un espejo.---Me dijeron hace días que
estuvo en Madrid y que D. Benigno le había ocultado en su casa.

---¡Aquí!\ldots{} ¡señora!---exclamó Sola echando sorpresa por sus ojos
con tanta naturalidad que la dama no pudo menos de sorprenderse
también.---La han engañado a usted\ldots{} Apuesto a que Pipaón\ldots{}
¡Ah!, ese buen don Juan miente más que habla\ldots{} Todos los días
viene contando unas patrañas que nos hacen reír. En cuanto a ese
desgraciado, yo creo que no puede ocultarse aquí ni en ninguna
parte\ldots{}

---¿Por qué?

---Yo tengo mis razones para creer\ldots{} Sí, bien lo puedo asegurar
casi sin temor de equivocarme: mi hermano ha muerto.

Parecía que iba a llorar un poco; pero no lloró ni poco ni mucho. La
dama vaciló un momento entre la emoción y la incredulidad. Llevose el
pañuelo a la boca como si quisiera poner a raya los suspiros que contra
todas las leyes del disimulo querían echarse fuera, y dijo esto:

---¡Válganos Dios, y cómo mata usted a la gente!\ldots{} Con permiso de
usted no creo\ldots{}

¡Horrible y nunca oída algazara! Quiso el Demonio, o por mejor hablar,
doña Crucita, que en el momento de decir la señora no creo, se abriese
la puerta del gabinete y diera salida a dos falderillos, un doguito y un
pachón que soltando a un tiempo el ladrido atronaron la sala; y como por
la misma puerta venía el chillar de los pájaros, y como de añadidura
subían por la angosta escalera los tres chicos de Cordero, procedentes
de la escuela, se armó un estrépito tal que no lo hiciera mayor la diosa
misma de la jaqueca, caso de que pueda haber tal diosa. Los perros se
tiraban a acariciar a los Corderillos, los Corderillos a los perros y en
medio del tumulto se oyó la pacífica voz de D. Benigno que también por
la escalera subía diciendo: «orden, silencio, compostura, que hay visita
en casa».

Detrás de D. Benigno apareció la figura de Zurbarán a quien llamaban
padre Alelí, y con el furor que los chicos ponían en besar la mano del
padre y la correa del amigo, se aumentó el estruendo, porque los perros
también querían dar pruebas de su veneración con ladridos. Al fin, para
que nada faltara, apareció doña Crucita echando toda la culpa de la
bulla a los muchachos, y les llamó \emph{perros}, y a los perros
\emph{nenes} y a su hermano \emph{Borrego de Cristo} y a Sola \emph{Doña
Aquí me estoy}, y al buen fraile el \emph{Zancarrón de Mahoma}.

---Cállate, \emph{Cruz del Mal Ladrón}---dijo Alelí riendo,---y guarda
adentro toda esta jauría del Infierno\ldots{} ¡Oh! Cuánto bueno por
aquí. Sí, ya me ha dicho Benigno que había subido usted a ver la casa.
¿Y qué tal? Tiene magníficas vistas nocturnas el patio, y en jardines
colgantes no le ganaría Babilonia, así como en diversidad de alimañas no
le ganaría el África entera.

La dama habló un momento de las condiciones de la casa; después se
despidió para marcharse, porque era la una, hora sacramental de la
comida.

---Un momento, señora---dijo D. Benigno, ahuyentando a sus hijos y a los
perros.---Aquí tiene usted al buen Alelí con más miedo que un masón
delante de las comisiones militares. Usted que tiene valimiento puede
sacarle de este apuro. Figúrese usted\ldots{}

---Nada, nada, señora---dijo Alelí nerviosamente, con extraordinaria
recrudescencia en el temblor de su cabeza sobre el cuello que parecía de
alambre.---No es más sino que hace un rato se ha metido por la puerta de
mi celda un emigrado, un terrible \emph{democracio} que se ha colado en
España sin pedir permiso a Dios ni al Diablo, y con palabras angustiosas
me ha rogado que le ampare y le esconda allí\ldots{}

---¿Y qué es un \emph{democracio}?---preguntó la dama riendo.---Un
perdis, un masón, un liberalote, un conspirador, un \emph{democracio},
así les llamamos.

---¿Y cuál es su nombre?

---Eso, señora---dijo Alelí con gravedad,---no lo revelaré, pues aunque
estoy decidido a no tenerle oculto más que el tiempo necesario para que
reciba contestación escrita de los que puedan o quieran protegerle
mejor, no cantaré quién es, aunque me ahorquen. Confío en la discreción
de todos los presentes. Bien saben que no amparo conspiradores contra mi
rey y la religión que profeso, y si a este he amparado, hícelo porque me
juró que no venía acá para armar camorra, sino para corregirse y vivir
pacíficamente, confiado en el perdón que espera alcanzar de Su Majestad.

---Sabe Dios a qué vendrá mi hombre---dijo Cordero, gozándose en
aumentar el susto de su amigo.---Me parece que de la Trinidad Calzada
van a salir sapos y culebras si Calomarde no da una vuelta por allí.

---Yo me lavo las manos\ldots{} y callandito, que estamos hablando más
de la cuenta. Benigno, a comer se ha dicho. Esta señora nos va a
acompañar a hacer penitencia.

Rehusando los obsequios e invitaciones de aquella buena gente retirose
la dama con harto dolor suyo, por no poder alcanzar el fin de la
interesante noticia que el fraile traía del convento. Por la calle iba
pensando en el desconocido que se acogía al amparo de la celda de Alelí.
Al llegar a su casa encontró a Pipaón que la aguardaba.

---¡Necio!---exclamó, sentándose muy fatigada.---En casa de Cordero no
hay nada\ldots{} Como siga usted rastreando de este modo, pronto le
dedicará Calomarde a coger moscas\ldots{} Pero una feliz
casualidad\ldots{}

---¿Ha descubierto usted\ldots?

---Sí, hombre ¿qué cosa habrá que yo no descubra? Vea usted por
dónde\ldots{} Déjeme usted que descanse.

---En Gracia y Justicia se sabe que continúa funcionando en Francia, más
envalentonado que nunca, el famoso \emph{Directorio provisional del
levantamiento de España contra la tiranía}.

---Noticia fresca.

---Se sabe---añadió Pipaón dándose mucha importancia,---que constituyen
el tal \emph{Directorio} los patriotas, o dígase perdularios, Valdés,
Sancho, Calatrava, Istúriz y Vadillo.

---Que Mendizábal es el depositario de los fondos.

---Que Lafayette les protege ocultamente y les busca dinero, y
finalmente que han enviado a Madrid a cierto individuo con nombre
supuesto\ldots{}

---El cual, o yo soy incapaz de sacramento, o está en la Trinidad
Calzada.

Pipaón abrió su boca todo lo que su boca podía abrirse y después de
permanecer buen rato haciendo competencia a las carátulas de mármol que
de antiguo existen en los buzones del correo, repitió con asombro:

---¡En la Trinidad Calzada!

\hypertarget{xi}{%
\chapter{XI}\label{xi}}

El padre Alelí amenizó la comida con su charla, que habría sido la más
sabrosa del mundo, si por efecto de los muchos años no tuviera la cabeza
tan desvanecida y descuadernada que todo era desorden y divagaciones en
sus discursos. Sucedía que el buen señor empezaba a contar una cosa, y
sin saber cómo se escurría fuera del tema principal y pasando de un
incidente a otro hallábase a lo mejor a cien leguas del punto adonde
quería ir. Era hombre que antes de llegar a la decrepitud, tuvo una
memoria fresquísima y una chispa especial para contar cosas pasadas y
presentes; pero estaba ya tan débil de cascos que de aquel recordar
prodigioso y de aquel arte admirable para la narración ya no quedaba más
que una facundia deshilvanada, un chorrear de ideas y palabras, y un
grandísimo enfado si alguien le interrumpía o intentaba llamarle al
orden.

---Puesto que queréis conocer el caso del \emph{democracio} que se ha
metido por las puertas de mi celda---dijo al principiar la comida,---os
lo voy a contar como se deben contar las cosas, con todos sus pelos y
señales. Empecemos por donde debe empezarse. Pues señor\ldots{} iba yo
por la calle de Carretas arriba, y al llegar a la esquina de Majaderitos
veo que viene hacia mí un elefante con los brazos abiertos. Era para
causar espanto a cualquiera la acometida de aquel monstruo con sotana y
manteo; pero yo que conozco a mis fieras me dejé abrazar y le abracé
también con mucho gozo. «¿Cómo va? Bien, ¿y tú, gigantón\ldots?» En fin,
para no cansar, era Juan Nicasio Gallego. Ya sabéis que fue discípulo
mío en Salamanca donde leí sagrados cánones por los años de 792 a 794.
Era entonces Nicasio el jayán más guapote que había salido de la tierra
del garbanzo; sus disposiciones eran grandes, tan grandes como su
pereza, y hubiéramos tenido en él un acabado canonista si no cayera en
la tentación de enamorarse de Horacio y Virgilio, fomentadores de la
holgazanería. El bribón de Meléndez le tomó mucho cariño, y lo mismo el
calzonazos de Iglesias que fabricó su reputación con
chascarrillos\ldots{} Yo digo que si Iglesias no se llega a morir a los
treinta y ocho años hubiera puesto el Breviario en epigramas\ldots{}
Pero sigo contando con orden. Quedamos en que una tarde paseábamos por
el Zurguén el maestro Peláez, Meléndez, Gallego y yo. Por aquellos días
había venido la noticia de la degollación de Luis XVI, y estábamos
consternados, muy consternados, atrozmente consternados. A mí no me
digan, ¿hay en la historia antigua ni moderna un crimen tan
atroz?\ldots{}

---Por vida de Sancho Panza---dijo D. Benigno riendo,---que eso se
parece al cuento del hidalgo y el labrador\ldots{} ¿A dónde va usted a
parar con sus divagaciones, ni qué tiene que ver Luis XVI con el poeta
zamorano?\ldots{}

---Allá voy, hombre, allá voy---replicó Alelí muy amostazado.---Yo sé lo
que cuento y no necesito de apuntadores.

---Sepamos ante todo lo que le dijo Gallego en la esquina de
Majaderitos, si es que esto tiene algo que ver con el cuento del
\emph{democracio}.

---Seguramente tiene que ver. Gallego es también un grande y descomedido
\emph{democracio}, y a eso iba\ldots{} Pues me contó Juan Nicasio cómo
le está engañando Calomarde, fingiéndole protección, y cómo el Rey le ha
prometido no sé cuántas prebendas sin darle ninguna. Además, el hombre
está temblando porque le han delatado por franc-masón, y bien sabemos
todos que el año 8 fue empleado de los liberales en Cádiz, y el año 10
diputado en las pestíferas Cortes.

---Eso de pestíferas no pasa---exclamó Cordero, dando un golpe en la
mesa con el mango del tenedor.---Repórtese el fraile o se sabrá quién es
Calleja.

---Vete con dos mil demonios.

---Siga el cuento.

---Sigo, y no interrumpirme.

---Pero cuidado con echar por los cerros de Úbeda.

---Que diga Sola si voy mal.

---Va admirablemente---replicó ella sonriendo.---Eso se llama contar
bien, y no falta sino saber lo que dijo ese señor \emph{gallego} o
asturiano.

---Pues dijo que está empleado en la biblioteca del duque de Frías y que
hace poco le fueron a prender por revoltoso, y equivocándose los de
policía, en vez de cogerle a él cogieron al archivero y le plantaron en
la cárcel. Cuando el Rey lo supo se rió mucho, y dijo a Calomarde:
\emph{«Tan malos sois como tontos».} Después, Gallego fue a ver al Rey,
y como este tiene debilidad por los poetas\ldots{} Ya sabéis cuánto se
entusiasma con Moratín. ¡Ah!, hace dos años que murió ese buen hombre y
yo me acuerdo, como si fuera de ayer, de haberle visto trabajando en la
platería de su tío el joyero del Rey. Creo haberos contado que Moratín
tuvo una novia, una tal doña Paquita, hija de la dueña de la casa donde
vivía \emph{Mustafá}. Ya sabéis que así llamábamos al pobre Juan Antonio
Conde por ser escritor de cosas de moros.

---Nos lo ha contado unas doscientas veces---dijo Cordero al oído de
Sola.

---No sabíamos eso---añadió esta en voz alta, para no desanimar al
bondadoso fraile.---¿Con que Moratín\ldots?

---Sí, hija mía, estuvo enamorado de esa doña Paquita, habitante en la
calle de Valverde con su madre, la señora doña María Ortiz, que fue el
pintiparado modelo de la saladísima doña Irene de \emph{El sí de las
niñas}. Moratín ya no era mozo y doña Paquita apenas tendría los
dieciocho años, es decir, que con veinte de por medio entre los dos,
¡qué había de suceder\ldots! Leandro, enamorado como suelen estarlo los
machuchos que se reverdecen, la niña afectando acceder por timidez, por
hipocresía o por agradecimiento, hasta que vino el desengaño, un
desengaño cruel, horrible\ldots{}

---¡Barástolis!\ldots{} señor don Plomo---exclamó Cordero con repentino
enfado,---que estamos hartos de oírle contar lo de Moratín y doña
Paquita. ¿Qué tiene eso que ver ni con el amigo que encontró en
Majaderitos, ni menos con el \emph{democracio} que está escondido en la
Trinidad?

---A ello voy, a ello voy, señor don Azogue---replicó Alelí enojándose
también.---Pues qué ¿no se han de contar los antecedentes de los
sucesos? Precisamente iba a decir que en el momento de despedirme de
Gallego acertó a pasar ese muchacho americano, Veguita, un
enredadorzuelo que dio que hablar cuando aquella barrabasada de los
\emph{Numantinos} y fue castigado con dos meses de encierro en nuestra
casa para que le enseñáramos la doctrina. El tal es de buena pasta.
Pronto le tomamos afición. Cantaba con nosotros en el coro y rezaba las
horas. Yo le daba golosinas y le hacía leer y traducir autores latinos,
y él me leía sus versos o me representaba trozos de comedias. Esto lo
hace tan perfectamente que si mucho tiene de poeta, más tiene de cómico.
Yo le animaba para que abandonase el mundo y entrase en la Orden\ldots{}
¡Oh, amigos míos!\ldots{} Cuando uno considera que en nuestra Orden
vivió y murió el primero de los predicadores del mundo Fray Hortensio
Paravicino, cuya celda ocupo en la actualidad\ldots!

---Que te descarrías, que te pierdes---dijo riendo D. Benigno.---Por
Dios, querido padre mío, ya está usted otra vez a setecientas leguas de
su cuento.

---Iba diciendo que Ventura me besó las manos y después se las besó al
\emph{padre de la Constitución}, que así llama a Gallego la gente
apostólica, y de esta manera le calificó en su infame delación el
religioso agonizante Fray José María Díaz y Jiménez, a quien nuestro
soberano llama el \emph{número uno de los podencos} por lo bien que
huele, rastrea, señala y acusa toda conspiración y astucia de esos
tontainas de liberales. No sé si os he dicho que, según confesión del
buen elefante zamorano, Calomarde le odia más que a un tabardillo
pintado, y si no fuera porque D. Miguel Grijalva, amigo mío y de
Nicasio, vio a Su Majestad y le llevó aquel famoso soneto que hizo
Gallego cuando la Reina estaba de parto\ldots{}

---Al grano, al grano, que eso más que referir sucedidos es marear a
Cristo.

---Un poquitín de paciencia, señores. Yo decía que se llegó a nosotros
Veguita, a quien, después del encarcelamiento en nuestra casa yo no
había visto más que dos veces, una en casa de Norzagaray cuando él y sus
amigos ensayaban la comedia de Zabala \emph{Faustina y Gerwal}, y otra
en la Puerta del Sol cuando le llevaban preso por tener la audacia de
dejarse las melenas largas, al uso masónico. Por cierto que ese
atrevidillo se ha dejado crecer un bigote que no hay más que ver, y con
aquellos precoces pelos insulta públicamente a la gente que manda, y
hace descarado alarde de liberalismo\ldots{} En una palabra, queridos,
Venturilla y Gallego empezaron a hablar del censor de teatros Reverendo
padre Carrillo, y excuso deciros que le pusieron como siete caños porque
no deja resollar a los autores. Después\ldots{} y aquí entra lo
principal de mi cuento\ldots{}

---Gracias a Dios\ldots{} Aleluya.

---Pues Veguita dijo una cosa al oído de Gallego\ldots{} y después
acercose a mí poniéndose de puntillas, porque él es muy pequeño y yo más
que regularmente alto, y me dijo también cuatro palabras al oído.

---¿Qué?---preguntó con mucha curiosidad Cordero.

---Pues no faltaba más sino que os fuera a revelar lo que se me confió
como un secreto.

\hypertarget{xii}{%
\chapter{XII}\label{xii}}

---¡Barástolis!, que estamos enterados---dijo Cordero comiéndose las
últimas almendras del postre.

Pero el famoso Alelí no paró mientes en estas palabras, y empezó a rezar
en acción de gracias por la comida. Poco después se habían levantado los
manteles, y los muchachos, bien fregoteadas las manos y la boca,
tornaron a la escuela. D.~Benigno, que acostumbraba dormir muy breve
siesta, la suprimió aquel día y bajó sin demora a la tienda porque la
comida había sido aquel día más larga que de ordinario. Doña Crucita que
no podía pasarse sin su regalado sueño de dos o tres horas, se fue a su
cuarto, llevando en un plato las golosinas con que solía obsequiar en
tal hora a sus queridas alimañas, y tras ella se fue Juan Jacobo, con el
sombrerón del padre Alelí encajado en la cabeza hasta tocar los hombros,
y en la mano un látigo que él mismo había hecho con una orilla de paño
amarrada al mango roto de un molinillo de chocolate. Alelí buscó el
blando acomodo de un sillón que en el testero del comedor estaba, y que
parecía decir \emph{dormid en mí} con la suave hondura de su asiento, la
inclinación de su viejo respaldo gordinflón y la curva de sus cariñosos
brazos. Allí dormía antaño la siesta doña Robustiana, y allá solía hacer
sus digestiones el buen Alelí, las cuales no eran difíciles, por ser él
la sobriedad misma.

Para mayor comodidad Sola le ponía delante una silla para que estirase
las piernas, y tras de la cabeza una mofletuda almohada de su propia
cama, con lo que el padre estaba tan bien, que ni en la misma gloria.
Aquella tarde, cuando Sola trajo silla y almohada, el fraile le tomó una
mano, y mirándola con sus ojos soñolientos, le dijo:---Cordera\ldots{}

Sonriendo como la misma bondad sonreiría, Sola acomodó en la almohada la
venerable cabeza que parecía la de un santo, y dijo así:

---¿Qué me quiere Su Reverencia?

---Cordera---murmuró el fraile sonriendo también como un
bienaventurado,---vete al cuarto de Benigno, y en el chaquetón, bolsillo
de la izquierda\ldots{} ¿entiendes?

---Sí, un cigarrito.

---Se me olvidó pedírselo antes que bajara\ldots{}

Ni medio minuto tardó la joven en traer el cigarrito, y con él la lumbre
para encenderlo.---Es que quiero echar una fumada para despabilarme,
porque desearía no dormir siesta\ldots{} ¿entiendes, paloma?

Como el fraile estaba con la cabeza echada atrás, en la más blanda y
cómoda postura que pueden apetecer humanos huesos, Sola no quiso que se
incorporase y ella misma le encendió el cigarro en el braserillo, no
siendo aquella la primera vez que tal cosa hacía. Chupó un poco con la
inhabilidad que en tal caso es propia de mujeres (como no sean
hombrunas), y cuando logró hacer ascua de tabaco, no sin perder mucha
saliva, presentó el cigarro a su amigo, cerrando los ojos por el picor
que el humo le causaba en ellos.

---Gracias, gracias, serafín de esta casa. Comprendo muy bien que ese
santo varón\ldots{} Pues, hija de mi alma, quiero despabilarme con este
cigarrito, porque necesito hablarte de una cosa grave, delicada, digo
mal, archi-delicadísima.

A Sola le pasó una nube por la frente, quiero decir, que se puso seria y
pensativa.

---Tiempo hay de hablar todo lo que se quiera---dijo, inclinada sobre
uno de los brazos del sillón en que el religioso estaba.---Duerma su
Reverencia.

---Bueno, hijita, con tal que me llames a las tres y media\ldots---Eso
es poco. A las cinco.

---No, no. Si me duerno, no podré hablarte del susodicho negocio, y lo
he prometido, cordera, he prometido que esta tarde misma\ldots{}

Esto decía cuando llegó un corpulento y bellísimo gato, que solía echar
sus dormidas en el mismo sillón donde estaba Alelí, y viendo ocupado
aquel lugar delicioso, dio algunas vueltas por delante con rostro
lastimero. Al fin, discurriendo que había sitio para todos, subió al
regazo del fraile y como encontrara agasajo, se enroscó y se echó a
dormir cual un bendito.

A poco de esto oyose un ruido estrepitoso, y fue que Juanito Jacobo
había cogido una bandeja de latón vieja, que olvidada estaba en la
despensa, y venía batiendo generala sobre ella con el palo del
molinillo, tan fuertemente que habría puesto en pie, con el estrépito
que hacía, a los siete durmientes. Acudió Sola y le trajo prisionero por
un brazo.

---¡Condenado chico! ¿No sabes que está tu tía durmiendo la
siesta?\ldots{} Ven acá: suelta eso\ldots{} Ya, ya es tiempo de que tu
padre te mande a la amiga\ldots{} Ríñale, Padre Alelí. No se le puede
aguantar. Cuando el señorito está de vena, parece que hay un ejército en
la casa.

Diciendo esto, Sola le iba quitando sombrero, bandeja y palo, y después
de sentarse le acercó a sí y le acarició pasando suavemente su mano por
los hermosos cabellos del niño.

---Si mete bulla---dijo Alelí acariciando también con su mano los
rizos,---no le traeré a mi señor don Juan Jacobo las hostias que le
prometí, ni las velitas de cera, ni el San Miguel de alcorza\ldots{}
Pues te decía, hija, que ahora vamos a hablar los dos de un asunto
superlativamente delicado\ldots{} Mira, vuelve al chaquetón de Benigno y
tráeme otro cigarrito, o mejor dos.

Sola hizo lo que le mandaba el reverendo y se volvió a sentar aguardando
aquello tan delicado que manifestarle quería. Durante un rato no
pequeño, los dos estuvieron callados, y Alelí fijaba sus ojos en el
reloj, que era de los antiguos con las pesas colgando al descubierto. La
péndola se paseaba lenta y solemnemente en el breve espacio que las
leyes de la gravedad y las de la mecánica le señalan, y así marcaba con
el tono más severo el compás de la vida. Sola, por mirar algo, que es
acto preciso a las meditaciones, miraba a \emph{La Creación}, gran
lámina que con otra representando el monumento de la catedral de Toledo,
decoraba artísticamente el comedor. En la primera estaban nuestros
primeros padres en el traje que es de suponer, en medio de un fértil
país poblado de todas suertes de animales, recibiendo la bendición del
Padre Eterno, que muy barbado y envuelto en una especie de capote se
asomaba por un balcón de nubes.

---¡Qué buenos cigarros tiene Benigno!---dijo Alelí, que al fin había
encontrado la fórmula del exordio.---Pero mejor que sus cigarros es él
mismo. Te digo con toda verdad que yo he visto muchos hombres buenos,
pero ninguno como nuestro Benigno. Es el corazón más puro y la voluntad
más cristiana que he conocido en mi larga vida; es incapaz de hacer nada
malo y capaz de las bondades más extraordinarias. Su razón es firme, sus
sentimientos generosos, su vida la carrera del bien. No aborrece a
nadie, y cuando quiere, quiere con toda su alma. Tiene un carácter
entero para hacer frente a las adversidades, y en las bienandanzas no
puede vivir contento si no distribuye su ventura entre los que le
rodean, quedándose él con la absolutamente precisa para no ser
desventurado. Si tú nos oyes diciéndonos majaderías, es por lo mucho que
nos queremos. Él me llama \emph{Tío Engarza-Credos}, y yo le llamo
\emph{Don Leño} o \emph{Chiribitas}, y así nos reímos. Eso sí, en ideas
políticas somos, como quien dice, el \emph{toma} y el \emph{daca}, lo
más opuesto que puede existir; pero estos arrumacos de la política no
han de tocar, no, a las cosas del alma ni a la amistad\ldots{} Porque yo
digo, ¿qué me importa que Benigno tenga la manía de leer a ese perdido
hereje de Rousseau, si por eso no deja de ser buen cristiano y de
obedecer a la Iglesia en todo?\ldots{} Viva Benigno, y viva con su
pepita, es decir, con su \emph{Emilio} y su \emph{Contrato social}, que
así me cuido yo de estas cosas como de los que ahora se están afeitando
en la luna\ldots{} No creas tú, los padres del convento me critican por
esta tolerancia mía, y yo les contesto: «vale más un amigo en la mano
que cien teorías volando». Mi carácter es así; en burlas disputo y
machaco como todos los españoles; pero antes que tronos y repúblicas,
antes que congresos y horcas está el corazón\ldots{} ¡Cómo me reí una
tarde hablando de esto! Paseaba yo a eso de las cinco por Atocha con dos
hombres de ideas contrarias, D. José Somoza, liberal, poeta, hombre
ameno y dulce y cabal si los hay, y D Juan Bautista Erro, absolutista
siempre, ahora apostólico vergonzante. Pues señor\ldots{}

---Paréceme---dijo Sola, cortando la digresión, que le parecía muy
importuna,---que se resbala usted, como dice D. Benigno. Ya está sabe
Dios a cuántas leguas de lo que me estaba contando\ldots{}

---¡Ah! Sí, perdona, hija\ldots{} me distraje. Te decía que ese bendito
amigo juan-jacobesco es el mejor tragador de pan y garbanzos que he
conocido, y que ahora ha dado en la flor de querer casarse\ldots{}

---¡Casarse!---exclamó Sola poniéndose encarnada.

---¿Te asombras, hija?\ldots{} Más me asombré yo\ldots{} No, no, no me
asombré; al contrario, me pareció muy natural. Le conviene por mil
razones; y ahora te pregunto yo: cuando Benigno tome estado ¿no será
para ti un gran motivo de amargura el salir de esta casa, donde has sido
tan amada, y separarte de estos chicos que has criado y que como a madre
te miran?\ldots{}

El padre Alelí fijó en ella sus ojos, ávidos de leer en los de la joven
lo que de su alma saliese al rostro, si es que algo salía. El buen
fraile, que a pesar de su decrepitud llena de perturbaciones mentales,
conservaba algo de su antigua penetración, creyó ver en Sola una pena
muy viva. Esto le hacía sonreír, diciendo para su sayo: «mujercita
tenemos».

---D. Benigno no se casará---dijo ella.---¿Será posible que caiga en tan
mala tentación? Yo de mí sé decir que si salgo de esta casa me moriré de
pena; tan tranquila, tan considerada y tan feliz he vivido en ella. Y
luego, estos diablillos del cielo, como yo les llamo; estos muchachos, a
quienes quiero tanto sin ser míos, y no tengo mejor gusto que ocuparme
de ellos\ldots{} No, digo que D. Benigno no se casará. Sería un
disparate; ya no está en edad para eso.

---¿Qué dices ahí, tontuela?---exclamó Alelí incorporándose con
enojo,---¿con que mi amigo no está en edad de casarse? ¿Es acaso algún
viejo chocho, está por ventura enfermo? No, más sana y limpia está su
persona y su sangre noble que la de todos esos mozuelos del día.

Esto decía cuando Juan Jacobo, cansado de estarse quieto tanto tiempo y
no teniendo interés en la conversación, empezó a tirarle de los bigotes
al gato que dormido estaba en la falda del fraile. Sentirse el animal
tan malamente interrumpido en su sueño de canónigo y empezar a dar
bufidos y a sacar las uñas fue todo uno. Alborotose el fraile con los
rasguños, y dio un coscorrón al chico, Sola le aplicó dos nalgadas y
todo concluyó con enfadarse el muchacho y coger al gato en brazos y
marcharse con él a un rincón donde le puso el sombrero del mercenario
para que durmiera.

---Eso es, sí, está mi sombrero para cama de gatos---refunfuñó Alelí.

---¡Jesús qué criatura!\ldots{} le voy a matar---dijo Sola amenazándole
con la mano.---Trae acá el sombrero.

Juan trajo el sombrero, y aprovechándose del interés que en la
conversación tenían el fraile y la joven, rescató su molinillo y su
bandeja y bajó a la tienda para escaparse a la calle.

---Vaya con la tonta---dijo Alelí continuando su interrumpido tema.---Si
Benigno es un muchacho, un chiquillo\ldots{} Si me parece que fue ayer
cuando le vi arrastrándose a gatas por un cerrillo que hay delante de su
casa\ldots{} ¡Qué piernazas aquellas, qué brazos y qué manotas tenía! ¡Y
cómo se agarraba al pecho de su madre, y qué mordidas le daba el muy
antropófago! Yo le cogía en brazos y le daba unos palmetazos en los
muslos\ldots{} Sabrás que fui al pueblo a restablecerme de unas
intermitentes que cogí en Madrid cuando vine a las elecciones de la
Orden. Entonces conocí al bueno de Jovellanos, un Voltaire encubierto,
dígase lo que se quiera, y al conde de Aranda, que era un Pombal
español, y a mi señor D. Carlos III, que era un Federico de Prusia
españolizado\ldots{}

---Al grano, al grano.

---Justo es que al grano vayamos. Cuando Nicolás Moratín y yo
disputábamos\ldots{}

---Al grano.

---Pues digo, que Benigno es un mozalbete. ¿No ves su arrogancia, su
buen color, sus bríos? Bah, bah\ldots{} Oye una cosa, hijita: Benigno se
casará, tú te quedarás sola, y entonces será bien añadir a tu nombre
otra palabra, llamándote \emph{Sola y monda} en vez de Sola a secas.
Pero aquí viene bien darte un consejo\ldots{} ¿Sabes, hija mía, que me
está entrando un sueño tal, que la cabeza me parece de plomo?

---Pues deme Su Reverencia el consejo y duérmase después---repuso ella
con impaciencia.

---El consejo es que te cases tú también, y así del matrimonio de
Benigno no podrá resultar ninguna desgracia\ldots{} ¡Qué sueño, santo
Dios!

Sola se echó a reír.

---¡Casarme yo!\ldots{} Qué bromas gasta el padrito.

---Hija, el sueño me rinde\ldots{} no puedo más---dijo Alelí luchando
con su propia cabeza que sobre el pecho se caía, y tirando de sus
propios párpados con nervioso esfuerzo para impedir que se cerraran cual
pesadas compuertas.

---Otro cigarrito.

---Sí\ldots{} chupetón\ldots{} humo---murmuró Alelí, cuya flaca
naturaleza era bruscamente vencida por la necesidad del reposo.

\hypertarget{xiii}{%
\chapter{XIII}\label{xiii}}

Sola corrió a buscar el despertador y a su vuelta encontró al pobre
religioso más que medianamente dormido, la cabeza inclinada a un lado,
la boca entreabierta, roncando como un viejo y sonriendo como un niño.
No quiso despertarle, aunque estaba curiosa por saber en qué pararía
aquel asunto del casamiento de su protector. Sospechaba la intención del
fraile y todo el intríngulis de aquella conferencia cortada por el
sueño, y gozaba interiormente considerando los rodeos y la timidez de su
protector.

Acomodó la cabeza del anciano en la almohada, le puso una manta en las
piernas para que no se enfriase y le dejó dormir. Sentada en una silla
al pie de \emph{La Creación} le miró mucho, cual si en el semblante
frailesco estuvieran estampadas y legibles las palabras que Alelí había
dicho y las que no había tenido tiempo de decir. Profundo silencio
reinaba en el comedor. Oíase, sin embargo, el paseo igual y sereno de la
péndola y un roncar lejano, profundo, que tenía algo de la trompa épica,
y era la melopea del sueño de doña Crucita cantada en tonante estilo por
sus órganos respiratorios. Los del reverendo Alelí no tardaron en unir
su autorizada voz a la que de la alcoba venía, y sonando primero en
aflautados preludios, después en períodos rotundos, llegaron a
concertarse tan bien con la otra música que no parecía sino que el mismo
Haydn había andado en ello.

Entre las dos ventanas de la pieza, que recibían de un patio la poca luz
de que este podía disponer, había un armario lleno de loza fina, tan
bien dispuesta que bastaba una ojeada para enterarse de las distintas
piezas allí guardadas. Las copas puestas en fila y boca abajo,
sustentando cada cual una naranja, parecían enanos con turbantes
amarillos. En todas las tablas las cenefas de papel recortado caían
graciosamente formando picos como un encaje, y de este modo los
arabescos de la loza tenían mayor realce. Algunas cafeteras y jarros
echaban hacia fuera sus picos como aves que, después de tomar agua,
estiran el cuello para tragarla mejor, y las redondas soperas se estaban
muy quietas sobre su plato, como gallinas que sacan pollos. En el
chinesco juego de té que regalaron a D. Benigno el día de su santo, las
tacitas puestas en círculo semejando la empolladura recién salida y
piando junto a la madre. Un alto y descomedido botellón cuya boca
figuraba la de un animalejo, era el rey de toda aquella muchedumbre
porcelanesca y parecía amenazar a las piezas vasallas con cierta ley
escrita en el fondo de una fuente. Era un letrero dorado que decía:
\emph{«Me soy de Benigno Cordero de Paz. Año de 1827».}

Junto al armario había una silla de tijera en la cual estaba Sola, con
los brazos cruzados. Miraba a Alelí, a la lámpara de cuatro brazos, a
\emph{La Creación}, al monumento de Toledo y al suelo cubierto de estera
común. También fue objeto de sus miradas el aguamanil, cuya llavecita,
un poco desgastada, dejaba caer una gota de agua a cada diez
oscilaciones de la péndola. La caja de latón en que estaba el agua tenía
pintado un pajarillo picando una flor, con tan desdichado arte que más
bien parecía que la flor se comía al ave. También miraba Sola al techo
donde había cuatro ligeras manchas de humo correspondientes a los cuatro
\emph{quinquets} de cada uno de los brazos de la lámpara. Tales manchas
eran las únicas nubes que empañaban el azul de aquel cielo de yeso que
en verano se estrellaba de moscas.

La joven dirigía sus ojos a todas estas partes, cual si estuviese
buscando sus pensamientos perdidos y desparramados por la estancia.
Creeríase que habían salido a holgar volando como mariposas a distintos
parajes, y que su dueña los iba recogiendo uno a uno o dos a dos para
traerlos a casa y someterlos al yugo del raciocinio.

Y así era en efecto. Ella tenía que concertar algo en su cabeza y
discurrir. Convidábanle a ello la soledad en que estaba y la suave
sombra que empezaba a ocupar el comedor dominando primero los ángulos,
el techo, y extendiéndose poco a poco y avanzando un paso al compás de
los que daba la péndola. Las voces o díganse ronquidos se apagaron un
momento cual si los músicos que las producían descansasen para tomar más
fuerza. La de doña Crucita empezó luego a crecer, a crecer, desafiando a
la del padre Alelí. La de este sonaba entonces en el registro del
caramillo pastoril y parecía convidar a la égloga con su gorjeo
cariñoso.

Y en tanto el murmullo de Crucita se tornaba de llamativo en provocador
y de provocador en insolente como si decir quisiera: «en esta casa nadie
ronca más que yo».

Indudablemente Sola discurría con muy buen juicio en medio de estas
músicas. Pensaba que era un disparate vivir tanto tiempo en un mundo
quimérico. La edad avanzaba; la juventud, aunque todavía rozagante y
lozana en ella, había dejado ya atrás una buena parte de sí misma. Su
vida marchaba ya muy cerca de aquel límite en que están la razón y la
prudencia, las posibilidades y las prosas, de tal modo que las ilusiones
se iban quedando atrás envueltas en brumas de recuerdos, mal iluminados
por la luz vespertina de esperanzas desvanecidas. La fantasía se cansaba
de su trabajo estéril, de aquella fatigosa edificación de castillos
llevados del viento y descompuestos en aire como las bovedillas de la
espuma, que no son más que juegos del jabón transformándose por un
instante en pedrería de mil matices. Llegaba Doña \emph{Sola} y
\emph{Monda} a la edad en que parece verificarse en la mente un despejo
de todas las jugueterías y figuraciones que traemos de la niñez, y queda
aquel aposento de nuestro espíritu limpio de las telarañas que parecen
tapices por capricho de la luz filtrada.

El sentimiento de la realidad empezaba a hacer en ella su tardía y
radical conquista, y así sentía la imposición ineludible de ciertas
ideas. ¿Cómo vivir más tiempo por y para un fantasma? ¿Cómo subordinar
toda la existencia a lo que tal vez no tenía ya existencia real o si la
tenía estaba tan distante que su alejamiento equivalía al no existir?
¿No podía suceder que sin quererlo ella misma, se destruyesen en su alma
ciertos afectos, y que de las ruinas de estos nacieran otros con menos
intensidad y lozanía, pero con más condiciones de realidad y firmeza?

Tan abstraída estaba que no advirtió cuán bravamente aceptaba la voz del
padre Alelí el reto de los lejanos bramidos de doña Crucita, y dejando
el tono pastoril, iba aumentando en intensidad sonora hasta llegar a un
toque de clarines que habrían infundido ideas belicosas a todo aquel que
los oyera. Los cañones respiratorios del reverendo decían seguramente en
su enérgico lenguaje: «cuando yo ronco en esta casa, nadie me levanta el
gallo». Acobardada y humillada por tan marcial alboroto, doña Crucita se
recogió y se fue aplacando hasta que su música no fue más que un
murmullo como el de los perezosos beatos que rezan dentro de una vasta
catedral, y luego se cambió en el sollozo de las hojas de otoño
arrancadas por el viento y bailando con él.

A su vez, el victorioso ronquido de Alelí remedó el fagot de un coro de
frailes, y después dejó oír varias notas vagas, suspironas, fugitivas
como los murmullos del órgano cuando el organista pasa los dedos sobre
el teclado en tanto a que el oficiante le da con sus preces la señal de
empezar. La música roncadora se había hecho triste, coincidiendo con la
oscuridad casi completa que llenaba la pieza.

Pero el alma de doña \emph{Sola} y \emph{Monda} no estaba triste. Había
echado una mirada al porvenir y lo había visto placentero, tranquilo,
honroso y honrado. Su corazón al declararse vencido por las realidades
un poco brutales, como conquistadores que eran, no estaba vacío de
sentimientos, antes bien se llenaba de los afectos más puros, más
delicados, más profundos. La vida nueva que se le ofrecía, debía
inaugurarse, eso sí, con un poco de tristeza; pero ¡cuánta dignidad en
aquella nueva vida!, ¡qué hermoso realce en la personalidad!, ¡qué
ocasión para mostrar los más nobles sentimientos, tales como la
abnegación, la constancia, la fidelidad, el trabajo!, ¡qué ocasión para
perfeccionarse constantemente y ser cada día mejor, realizando el bien
en todas las formas posibles y gozando en el sostenimiento de esa
deliciosa carga que se llama el deber!

¿Pero qué estruendo, qué fragor temeroso era aquel que Sola sentía tan
cerca y que interrumpía sus discretos pensamientos en lo mejor de ellos?
Sonaban ya sin duda las trompetas del Juicio Final, pues no de otro modo
debían llamarse los destemplados y altísonos ronquidos de Crucita y el
Padre Alelí. Los de este se detuvieron bruscamente, cual si fuera a
despertar, y oyose su voz que entre sueños decía:

---Vete, vete de mi celda, terrible \emph{democracio}\ldots{} ¿Qué
buscas aquí?, ¿a qué vienes a España y a Madrid, si no es a que te
ahorquen?\ldots{} ¡Vuélvete a la emigración de donde jamás debiste
salir!\ldots{} ¡conspirador\ldots{} vagabundo!

\emph{Doña Sola} y \emph{Monda} se acercó al fraile para oír mejor lo
que entre dientes seguía diciendo.

Alelí extendió los brazos quedándose un buen rato como un crucifijo en
sabroso estiramiento de músculos, y con voz clara y entera dijo así:

---Esproncedilla\ldots{} busca-ruidos\ldots{} vagabundo, no me
comprometas\ldots{} vete de mi celda.

Sola se acercó y le tomó una mano.

---¿Pero qué oscuridad es esta?, ¿en dónde estoy?

---¡Vaya un modo de dormir y de disparatar!---replicó Sola riendo.

---¿Pues qué, he dormido yo?\ldots{} Si no he hecho más que aletargarme
un instante, cinco minutos todo lo más\ldots{} Vaya, que se pone pronto
el sol en esta dichosa casa\ldots{} Chiquilla, dame mi sombrero que me
voy.

---Primero voy a traer luz---dijo la \emph{Hormiga} saliendo.

Al poco rato volvió con una lámpara, cuyos rayos ofendieron la vista del
fraile.

---Yo creí que ya habían empezado a crecer los días\ldots{} ¿qué hora
es? Las cinco y media\ldots{} Lo dicho dicho, querida señorita\ldots{}
¿Reflexionarás en lo que te he dicho?

---Pues qué he de hacer sino reflexionar.

---¿Y comprenderás que se te entra por las puertas la fortuna y que vas
a ser la más dichosa de las mujeres?

---Pues claro que sí.

---¡Bendita seas tú y bendito quien te trajo a esta casa!---exclamó
Alelí con acento muy evangélico.

Abriose con no poco estrépito la puerta del comedor y apareció Crucita
de malísimo talante diciendo:

---No he podido pegar los ojos en toda la tarde con la dichosa
conversación de la niña y el fraile.

---Quita allá, \emph{Cruz del Mal Ladrón}---replicó Alelí.---Lo que ha
sido es que con la trompeta de tus roncamientos no me has dejado a mí
descabezar un mal sueño.

---Sí, porque a fe que el Padrito toca algún cascabelillo sordo cuando
duerme\ldots{} Me habéis tenido toda la tarde despabilada como un lince,
primero con la charla de sus mercedes y luego con los piporrazos de Su
Reverencia\ldots{} ¡qué importunidad, santo Dios! Busque usted un
momento de tranquilidad en esta casa.

---Cállate, \emph{serpiente del Paraíso,} que así guardas silencio
dormida como despierta, y no hables de eso, que el que más y el que
menos todos, todos repicamos, y abur.

Echáronse a reír Sola y el fraile, y al fin se rió un poco Crucita, pues
su genio arisco también tenía flores de cuando en cuando, si bien estas
eran como las plantas marinas que están en el fondo y casi siempre en el
fondo mueren.

\hypertarget{xiv}{%
\chapter{XIV}\label{xiv}}

En la tienda, D. Benigno preguntó con mucho interés a su amigo por el
resultado de la conferencia que con Sola había tenido.

---Muy bien---dijo Alelí.---Admirablemente bien.

Después se quedó perplejo, con los ojos fijos en el suelo y el dedo
sobre el labio, como revolviendo en el caótico montón de sus recuerdos;
y al cabo de tantas meditaciones, habló así:

---Pues, hijo, ahora caigo en que no llegué a decirle lo más importante,
porque me acometió un sueño tal que no hubiera podido vencer aunque me
echaran encima un jarro de agua fría\ldots{} Ya la tenía preparada; ya,
si no me engaño, había ella comprendido el objeto de mi discurso, y
manifestaba un gran contento por la felicidad que Dios le depara,
cuando\ldots{} Yo no sé sino que me desperté en la oscuridad de tu
comedor que parece la boca de un lobo\ldots{} Y qué quieres,
hijo\ldots{} lo demás puedes decírselo tú, o se lo diré yo mañana.
Quédate con Dios y con la Virgen.

Marchose Alelí y D. Benigno se quedó muy contrariado y ofendido de la
poca destreza de su amigo. Juró no volver a confiar misiones delicadas a
un viejo decrépito y medio lelo, y al mismo tiempo se sentía él muy
cobarde para desempeñar por sí mismo el papel que había confiado al
otro. Cuando subió, después de cerrar la tienda, en compañía de Juan
Jacobo que había entrado de la calle con un chichón en la frente, dijo a
Sola:

---Ya estoy convencido de que ese estafermo de Alelí es el bobo de
Coria\ldots{} Apreciabilísima \emph{Hormiga}, quisiera hablar con
usted\ldots{}

---¿Hablar conmigo?\ldots{} Ahora mismo; ya escucho---dijo ella,
sonriendo de tal modo que a Cordero se le encandilaron los ojos.

Pero en el mismo instante le acometió la timidez de tal modo, que no se
atrevió a decir lo que decir quería, y sólo balbució estas palabras:

---Es que conviene ponerle a este enemigo una venda y dos cuartos sobre
el chichón, que es el mejor medio de curar estas cosas.

Aquella noche D. Benigno estuvo muy triste y se pasó algunas horas en su
cuarto, sin leer a Rousseau, aunque bien se le acordaba aquel pasaje del
Libro quinto del \emph{Emilio}: «Emilio es hombre, Sofía es
mujer\ldots{} Sofía no enamora al primer golpe de vista, pero agrada más
cada día. Sus encantos se van manifestando por grados en la intimidad
del trato. Su educación no es ni brillante ni estrecha. Tiene gusto sin
estudio, talento sin arte y criterio sin erudición\ldots{} La
desconformidad de los matrimonios no nace de la edad, sino del
carácter\ldots» Y luego añadía, alterando un poco el texto: «Sofía había
leído el \emph{Telémaco} y estaba prendada de él; pero ya su tierno
corazón ha cambiado de objeto y palpita por el buen Mentor».

Después Cordero se reía de sí mismo y de su timidez, haciendo juramento
de vencerla al día siguiente pues lo que él sentía era un afecto
decoroso, un sentimiento de gratitud y de respeto y no pasión ni
capricho de mozalbete.

Al día siguiente Sola mostraba excelente humor que rayaba en festivo, lo
que dio muy buena espina al héroe de Boteros. Cantorreaba entre dientes,
cosa que no hacía todos los días, y su cara estaba muy animada, si bien
podía observarse que tenía los ojos algo encendidos. Sin duda había
visto y aceptado la posibilidad de un destino nuevo, honrado y honroso
en extremo, y se complacía en él, creyéndolo dispuesto por Dios con
extraordinaria sabiduría. Pero si no se entra en la vida sin llanto,
también parece natural que no se entre en las felicidades nuevas sin
algo de lágrimas. Los nuevos estados, aunque sean muy buenos y santos,
no siempre seducen tanto que hagan aborrecible la situación vieja por
detestable que haya sido. De aquí venía, sin duda, el que, estando con
tan buen humor, tuviese en lo encendido de sus ojos el testimonio de
haber lloriqueado algo.

O quizás aquella alegría que mostraba venía más bien de la voluntad que
del corazón, como si aquel espíritu, tan hecho a la observancia de los
deberes, hubiese resuelto que convenía estar alegre. La razón sin duda
lo mandaba así, y la razón iba siendo la señora de ella\ldots{} No hay
más sino que se dominaba maravillosamente y lograba alcanzar tan grande
victoria sobre sí misma, que era al fin, si es permitido decirlo así, un
producto humano de todas las ideas razonables, una conciencia puesta en
acción.

Su protector le dijo que aquella tarde se verían los dos en su cuarto
para hablar a solas. El héroe se atrevía al fin. Prometió ella ser
puntual y esperó la hora. Pero Dios que, sin duda por móviles altísimos
e inexplicables quería estorbar los honestos impulsos del héroe, dispuso
las cosas de otra manera. Ya se sabe lo que significan todas las
voluntades humanas cuando Él quiere salirse con la suya.

Sucedió que poco antes de la hora de comer, Juanito Jacobo, todavía
vendado por los chichones del día anterior, andaba enredando con una
pelota. Trabáronse las palabras de él y su hermano Rafaelito sobre a
quién pertenecía la tal pelota. Hay indicios y aun antecedentes
jurídicos para creer que el verdadero propietario era el pequeñuelo, y
así debió de sentirlo en su conciencia Rafael; que tanto imperio tiene
la justicia en la conciencia humana aunque sea conciencia en agraz.

Pero de reconocerlo en la conciencia a declararlo hay gran distancia, y
si tal distancia no existiera no habría abogados ni curiales en el
mundo. Por eso Rafael, no sintiéndose bastante egoísta para apandar la
pelota ni bastante generoso para dejársela a su rival, hizo lo que
suelen hacer los chicos en estas contiendas, es a saber: cogió la pelota
y la arrojó a lo alto del armario del comedor donde no podría ser
alcanzada ni por uno ni por otro.

¡Valiente hazaña la de Rafaelito!\ldots{} Pero el pequeño Hércules no
había nacido para retroceder ante contrariedades tan tontas. ¡Bonito
genio tenía él para acobardarse porque el techo esté más alto que el
suelo!\ldots{} Arrastró el sillón hasta acercarlo al armario; puso sobre
el sillón una silla, sobre la silla una banqueta, y ya trepaba él por
aquella frágil torre, cuando esta se vino al suelo con estruendo y rodó
el chico y se abrió la cabeza contra una de las patas de la mesa.

El laberinto que se armó en la casa no es para descrito. Salió D.
Benigno, acudió Sola, puso el grito en el cielo Crucita, ladraron todos
los perros, maldijo la criada todas las pelotas, habidas y por haber,
lloró Rafael, gimieron sus hermanos, y el herido fue alzado del suelo
sin conocimiento. Pronto volvió en sí, y la descalabradura no parecía
grave, gracias a la mucha sangre que salió de aquella cabezota. En tanto
que Sola batía aceite con vino, y la criada, partidaria de otro sistema,
mascaba romero para hacer un emplasto, doña Crucita que en todas estas
ocasiones se remontaba siempre al origen de los conflictos, repartía una
zurribanda general entre los muchachos mayores, azotándoles sin piedad
uno tras otro. Los perros seguían chillando y hasta la cotorra tuvo algo
que decir acerca de tan memorable suceso.

Toda la tarde duró la agitación y nadie tuvo ganas de comer, porque el
muchacho padecía bastante con su herida. Vino el médico y dijo que sin
ser grave, la herida era penosa, y exigía mucho cuidado. No hubo, pues,
conferencia entre Cordero y Sola, porque la ocasión no era propicia. Por
la noche Juanito Jacobo se durmió sosegadamente. Sola que en la misma
pieza puso su cama, estaba alerta vigilando al niño enfermo. Ya muy
tarde este se despertó intranquilo, calenturiento, pidiendo de beber y
quejándose de dolores en todo el cuerpo. Sola se arrojó del lecho, medio
vestida, y echándose un mantón sobre los hombros salió para llamar a la
criada. Levantose esta, y entre las dos prepararon medicinas,
encendieron la lumbre, fueron y vinieron por los helados pasillos. A la
madrugada cuando el chico se durmió al parecer sosegado y repuesto, Sola
sintió un frío intensísimo con bruscas alternativas de calor sofocante.
Arrojose en su lecho y al punto sintió una postración tan grande que su
cuerpo parecía de plomo. La respiración érale a cada instante más
difícil, y no podía resistir el agudo dolor de las sienes. La tos seca y
profunda añadía una molestia más a tantas molestias y en su costado
derecho le habían seguramente clavado un gran clavo, pues no otra cosa
parecía la insufrible punzada que la atormentaba en aquella parte.

La criada que al punto conoció lo grave de tales síntomas, quiso llamar
a D.~Benigno y a Crucita; pero Sola no consintió que se les molestara
por ella. Era la madrugada. Mientras llegaba el día la alcarreña preparó
no sé cuántos sudoríficos y emolientes, sin resultado satisfactorio. Al
fin cuando daban las siete Crucita dejó las ociosas plumas, y enterada
de lo que pasaba, reprendió a la enferma por haberse puesto mala
voluntariamente; que no otra cosa significaba el haber tomado aires
colados, hallándose, como se hallaba desde hace días, con un catarro más
que regular. La avinagrada señora echó por la boca mil prescripciones
higiénicas para evitar los enfriamientos y otros tantos anatemas contra
las personas que no se cuidaban. Cuando Cordero se levantó, Crucita, que
tenía un singular placer en anunciar los sucesos poco lisonjeros, fue a
su encuentro y le dijo:

---Ya tenemos otro enfermo en campaña. Sola se ha puesto muy mala.

---¿Qué tiene?---dijo el héroe con repentino dolor como presagiando una
gran desgracia.

---Pues una pulmonía fulminante.

Si lo partiera un rayo, no se quedara Don Benigno más tieso, más mudo,
más parado, más muerto que en aquel momento estaba. Creía ver su dicha
futura, sus risueños proyectos desplomándose como un castillo de naipes
al traidor soplo del Guadarrama.

---Veámosla---dijo recobrando la esperanza, y corrió a la alcoba.

Sola le miró con cariñosos y agradecidos ojos. Quiso hablarle y la
violenta tos se lo impedía. D. Benigno no pudo decir nada, porque
indudablemente el corazón se le había partido en dos pedazos, y uno de
estos se le había subido a la garganta. Al fin hizo un esfuerzo, quiso
llenarse de optimismo, y echó una sonrisa forzada y dijo:

---Eso no será nada. Veamos el pulso.

¡Ay!, el pulso era tal, que Cordero, en la exaltación de su miedo, creyó
que dentro de las venas de Sola había un caballo que relinchaba.

---Que venga D. Pedro Castelló, el médico de Su Majestad---exclamó sin
poder contener su alarma.---Que vengan todos los médicos de
Madrid\ldots{} Diga usted, apreciable \emph{Hormiga}, ¿desde cuándo se
sintió usted mal?

---Desde ayer tarde---pudo contestar la joven.

---¡Y no había dicho nada!\ldots{} ¡qué crueldad consigo mismo y con los
demás!---¡Ya se ve\ldots{} no dice nada!\ldots---vociferó
Crucita.---¡Bien merecido le está!\ldots{} ¿Hase visto terquedad
semejante? Esta es de las que se morirán sin quejarse\ldots{} ¿Por qué
no se acostó ayer tarde, por qué? ¡Bendito de Dios, qué mujer! Si ella
tuviese por costumbre, como es su deber, consultarme todo, yo le habría
aconsejado anoche que tomara un buen tazón de flor de malva con unas
gotas de aguardiente\ldots{} Pero ella se lo hace todo y ella se lo sabe
todo\ldots{} Silencio Otelo\ldots{} vete fuera, Mortimer\ldots{} No
ladres, Blanquillo.

Y en tanto que su hermana imponía silencio al ejército perruno, el
atribulado D.~Benigno elevaba el pensamiento a Dios Todopoderoso
pidiéndole misericordia.

Sin pérdida de tiempo hizo venir al médico de la casa, y a todos los
médicos célebres precedidos por D. Pedro Castelló, que era el más
célebre de todos.

\hypertarget{xv}{%
\chapter{XV}\label{xv}}

En tanto que esto pasaba en casa del vendedor de encajes, doña Jenara y
Pipaón andaban atortolados por el ningún éxito de sus averiguaciones, y
los días iban pasando y la sombra o fantasma que ambos perseguían se les
escapaba de las manos cuando creían tenerla segura. El terrible
\emph{democracio} albergado en la Trinidad resultó ser el más inocente y
el más calavera de todos, hombre que jamás haría nada de provecho fuera
de las hazañas en el glorioso campo del arte; gran poeta que pronto
había de señalarse cantando dolores y melancolías desgarradoras. No
sabiendo cómo lo recibiría la policía, acogiose a los frailes
Trinitarios por indicación de Vega, que en aquella casa cumplió seis
años antes su condena, cuando el desastre numantino. Los empeños de su
familia y amigos le consiguieron pronto el indulto, y decidido a ser en
lo sucesivo todo lo juicioso que su índole de poeta fuera compatible,
solicitó una plaza en la Guardia de la Real Persona que le fue concedida
más adelante.

Bretón, desesperado por las horribles trabas del teatro, marchó a
Sevilla con Grimaldi, autor de la \emph{Pata de cabra}. Vega, que
luchaba con la pobreza y era muy perezoso para escribir, quería hacerse
cómico y aun llegó a ajustarse en la compañía de Grimaldi. Considerando
esto los amigos como una deshonra, pusieron el grito en el cielo; pero
como los alimentos no podían sacar al poeta de su atolladero, fue
preciso echar un guante para rescatarle, por haber cobrado con
anticipación parte del sueldo de galán joven. Grimaldi era un empresario
hábil que sabía elegir la gente, y en su memorable excursión por Cádiz y
Sevilla, dio a conocer como actriz de grandísima precocidad a una niña
llamada Matilde, que a los doce años hacía la protagonista de \emph{La
huérfana de Bruselas} con extraordinario primor.

En Madrid, después de la marcha de Grimaldi, el teatro se alimentaba de
traducciones. Algunas de estas fueron hechas por un muchacho carpintero,
de modestia suma y apellido impronunciable. Era hijo de un alemán y
hacía sillas y dramas. Fue el primero que acometió en gran escala la
restauración del teatro nacional, para sacar al gran Lope del
polvoriento rincón en que Moratín y los clásicos le habían puesto
juntamente con los demás inmortales del siglo de oro. El infeliz
ebanista que no podía ver representadas sus obras originales, traducía a
Voltaire y a Alfieri y refundía a Rojas y al buen Moreto. Pero su
estrella era tan mala que no logró abrirse camino ni hacer resonar su
nombre en la república de las letras; y así pocos años después, la
víspera del estreno de su gran obra original que le llevó de un golpe a
las alturas de la fama, el lenguaraz satírico de la época, el mal
humorado y bilioso escritor a quien ya conocemos, decía: «Pues si el
autor es sillero, la obra debe de tener mucha paja». El enrevesado
nombre del ebanista nacido de alemán y criado en un taller fue, desde
que se conocieron \emph{Los amantes de Teruel}, uno de los más gloriosos
que España tuvo y tiene en el siglo que corre.

Y el satírico seguía satirizando en la época a que nos referimos (1831);
mas con poca fortuna todavía, y sin anunciar con sus escritos lo que más
tarde fue. Se había casado a los veinte años, y su vida no era un modelo
de arreglo, ni de paz doméstica. Recibió protección de D. Manuel
Fernández Varela, a quien se debe llamar \emph{El Magnífico} por serlo
en todas sus acciones. Su corazón generoso, su amor a la esplendidez, a
las artes, a las letras, a todo lo que fuera distinguido y antivulgar,
su trato cortesano, las cuantiosas rentas de que dispuso hacían de él un
verdadero prócer, un Mecenas, un magnate, superior por mil conceptos a
los estirados e ignorantes señorones de su época, a los rutinarios y
suspicaces ministros. Era la figura del Sr.~Varela arrogante y
simpática, su habla afabilísima y galante, sus modales muy finos. Vestía
con magnificencia y adornaba el severo vestido sacerdotal con pieles y
rasos tan artísticamente que parecía una figura de otras edades. En su
mesa se comía mejor que en ninguna otra, de lo que fueron testimonio dos
célebres gastrónomos a quienes convidó y obsequió mucho. El uno se
llamaba Aguado, marqués de las Marismas, y el otro Rossini, no ya
marqués, sino príncipe y emperador de la Música.

El Sr.~Varela protegió a mucha y diversa gente, distinguiendo
especialmente a sus paisanos los gallegos; fundó colegios, desecó
lagunas, erigió la estatua de Cervantes que está en la plazuela de las
Cortes, ayudó a Larra, a Espronceda y dio a conocer a Pastor Díez.

Cuando vino Rossini en Marzo de aquel año le encargó una misa. Rossini
no quería hacer misas\ldots{} «Pues un \emph{Stabat Mater»} le dijo
Varela. El maestro compuso en aquellos días el primer número de su gran
obra religiosa que parece dramática. El resto lo envió desde el
extranjero. Cuentan que Varela le pagó bien.

Algunos números del célebre \emph{Stabat} se estrenaron aquella Semana
Santa en San Felipe el Real, dirigidos por el mismo Rossini, y hubo
tantas apreturas en la iglesia que muchos recibieron magulladuras y
contusiones y se ahogaron dos o tres personas en medio del tumulto.
Rossini fue obsequiado, como es de suponer, atendida su gran fama. Tenía
próximamente cuarenta años, buena figura, y su hermosa cara, un poco
napoleónica, revelaba, más que el estro músico y el aire de la familia
de Orfeo, su afición al epigrama y a los buenos platos.

Habiendo recibido en un mismo día dos invitaciones a comer, una del
Sr.~Varela y otra de un grande de España, prefirió la del primero.
Preguntada la causa de esta preferencia, respondió:

---Porque en ninguna parte se come mejor que en casa de los curas.

En efecto; la mesa de este generoso y espléndido sacerdote era la mejor
de Madrid. A sus salones de la plazuela de Barajas concurría gente muy
escogida, no faltando en ellos damas elegantes y hermosas, porque, a
decir verdad, el Sr. Varela no estaba por el ascetismo en esta materia.

Pero allí la opulencia del señor y su misma gravedad de eclesiástico no
permitían la confianza y esparcimientos de otras tertulias. La de
Cambronero, por el contrario, era de las más agradables y divertidas,
dentro de los límites de la decencia más refinada.

Era el señor D. Manuel María Cambronero varón dignísimo, de altas
prendas y crédito inmenso como abogado. Durante muchos años no tuvo
rival en el foro de Madrid, y todos los grandes negocios de la
aristocracia estaban a su cargo. Fue en su época lo que posteriormente
Pérez Hernández y más tarde Cortina. Su señora era castellana vieja,
algo chapada a la antigua, y sus hijos siguieron diversos destinos y
carreras. Uno de ellos, D.~José, casó por aquellos años con Doloritas
Armijo, guapísima muchacha, cuyo nombre parece que no viene al caso en
esta relación, y sin embargo, está aquí muy en su lugar.

El primer pasante de Cambronero era un joven llamado Juan Bautista
Alonso, a quien el insigne letrado tomó gran cariño, legándole al morir
sus negocios y su rica biblioteca. Alonso, que más tarde fue también
abogado eminente, político y filósofo de nota, tuvo en su mocedad
aficiones de poeta, y por tanto, amistad con todos los poetas y
literatos jóvenes de la época. Él fue, pues, quien introdujo en las
agradabilísimas y honestas tertulias de Cambronero a Vega, Espronceda,
Felipe Pardo, Juanito Pezuela, y por último, al misántropo, al
incomprensible, al que ya se llamaba con poca fortuna \emph{Duende
Satírico}, y más tarde se había de llamar \emph{Pobrecito hablador},
\emph{Bachiller Pérez de Murguía}, \emph{Andrés Niporesas}, y finalmente
\emph{Fígaro}.

Como Pipaón había de meterse en todas partes, iba también a casa de
Cambronero. Jenara, sin que se supiese la causa, había disminuido
considerablemente sus tertulias; recibía poquísima gente, y sólo daba
convites en muy contados días. En cambio, iba a la tertulia de
Cambronero, donde hallaba casi todo el contingente de la suya, y además
otras personas que no había tratado hasta entonces, tales como D. Ángel
Iznardi, D. José Rives, D. Juan Bautista Erro y el conde de Negri.

También se veía por allí al joven Olózaga, pasante, como Alonso, en el
bufete de Cambronero, si bien menos asiduo en el trabajo. Desde los
principios del año andaba Salustiano tan distraído, que no parecía el
mismo. Iba a las reuniones como por compromiso o por temor de que al
echarse de menos su persona, se le creyese empeñado en conspiraciones
políticas. Su mismo padre, D. Celestino, se quejaba de sus frecuentes
ausencias de la casa. Tal conducta no podía atribuirse sino a dos
motivos, política o amores. La familia y los conocidos, inclinándose
siempre a lo menos peligroso, presumían que Salustiano andaba enamorado.
Su buena figura, su elocuencia, sus distinguidos modales, la misma
exaltación de sus ideas políticas y otras prendas de mucha estima,
dándole desde su tierna juventud gran favor entre las damas,
justificaban aquella idea.

De repente, Jenara dejó de asistir también con puntualidad a las
tertulias. El público, que todo lo quiere explicar según su especial
modo de ver, comentó aquellas ausencias con cierta malignidad, y hasta
hubo quien hablara de fuga al extranjero en busca de apartadas y
placenteras soledades, propicias al amor. Se daban pormenores, se
refirieron entrevistas, se repitieron frases, y sin embargo, todo esto y
lo demás que se dijo y que no es para contarlo, era un castillo aéreo
levantado por las delicadas manos de la chismografía. Pero acontece que
tales obras, con ser de aire, son más fáciles de levantar que de
destruir, y así de día en día aquella iba tomando consistencia y
alzándose más y engalanándose con torreones de epigramas y chapiteles de
calumnias.

\hypertarget{xvi}{%
\chapter{XVI}\label{xvi}}

Mediaba el mes de Marzo cuando estas hablillas llegaron a su más alto
grado de malicia. Jenara no recibía a nadie; pero no estaba enferma,
porque a menudo se la veía en la calle o paseando en coche o visitando a
personajes de alto copete.

Un día se encontraron ella y Pipaón en la antesala de la Comisión
Militar. Jenara salía, Pipaón entraba. Eran las cinco de la tarde hora
excelente para el paseo en aquella estación.

---Iba a su casa de usted---le dijo D. Juan,---para prevenirla del
peligro que corre\ldots{}

---¡Yo!---exclamó la dama con gesto de orgullo.---¿También yo corro
peligro?

---También.

---¿Y por qué?

---Salgamos de esta caverna, señora, que si en todas partes oyen las
paredes, aquí oyen las ropas que vestimos, hasta la sombra que hacemos
sobre el suelo. Vámonos.

---¿Qué hay?---dijo la señora, extraordinariamente alarmada.---Quiero
ver a Maroto.

---No recibe ahora\ldots{} Salgamos y hablaremos. Principiaré diciendo a
usted que hemos errado en todos nuestros cálculos. Buscábamos a nuestro
amigo en casa de Cordero, en el convento de la Trinidad, en la cárcel de
Corte, en el parador de Zaragoza, en el sótano de la botica de la calle
de Hortaleza, en la habitación del jefe del \emph{guardamangier} de
palacio, y ahora resulta que no estaba en ninguno de estos parajes,
sino\ldots{}

---¿En dónde, en dónde?

---Salgamos de esta casa, señora---añadió Pipaón al poner el pie en el
último peldaño.---Advierta usted que no digo está, sino estaba.

---Quiere decir que\ldots{}

---Quiere decir que le han llevado a un sitio de donde ni usted ni yo
podremos fácilmente sacarle.

---Bravo, bravísimo, señor D. Inservible\ldots---dijo la dama, toda
colérica y nerviosa, abriendo con mano firme la portezuela de su coche.

En este había una joven que acompañaba a Jenara en todas sus
excursiones, y a la cual, según las lenguas cortesanas, galanteaba el
bueno de Pipaón con más calor del que la simple urbanidad consiente.
Acomodados los tres en el coche, D. Juan dijo a la dama que, siendo
largo lo que tenía que contarle, convenía extender el paseo hasta
Atocha. Así se convino y partieron.

---Beso a usted los pies, Micaelita---dijo después el cortesano.---¿Y
cómo está el señor D. Felicísimo?

---Furioso con usted porque no ha ido a verle en tres días.

---Esta noche iremos todos allá. Con esto que pasa y el continuo trabajo
en que vivimos nos falta tiempo para dar pábulo\ldots{}

---Ahora salimos con pábulos\ldots---dijo Jenara impaciente y mal
humorada.---Basta de pesadeces y dígame usted lo que tenía que decirme.

---Pábulo sí; digo que no hay tiempo para satisfacer los puros goces de
la amistad, ni aun los del corazón.

Micaelita bajó los ojos. Pintémosla en dos palabras. Era fea. Y si no lo
fuera, ¿cómo la habría escogido Jenara para ser su inseparable compañera
y usarla cual discreta sombra de que se valía la pícara para hacer
brillar más la luz de su hermosura?

---Si empiezan las tonterías me voy a casa---dijo la dama
hermosa.---Vamos, hable usted, D. Plomo.

---Paciencia, señora, paciencia. Dígame usted, ¿se permiten las malas
noticias?

---Se permite todo lo que sea breve.

---Pues derramemos una lágrima aquí, en este sitio nefando\ldots{}

Al decir esto el coche pasaba junto al torreón del Ayuntamiento donde
estaba la Cárcel de Villa. Micaelita, que para todas las ocasiones
tristes llevaba siempre apercibido un \emph{paternoster}, lo rezó con
pausa y devoción. Jenara se puso pálida y sacó su cabeza por la
portezuela para mirar la torre.

---¡Allí!---exclamó señalando con el abanico y con sus ojos.

Vuelta a su posición primera, echó un suspiro casi tan grande como el
torreón y habló así:

---Ahora, dígame usted dónde estaba.

---Donde menos creíamos. En casa de Olózaga.

---¿En casa de D. Celestino Olózaga?

---Calle de los Preciados.

---Usted bromea: no puede ser---manifestó la dama un poco
aturdida.---Veo a Salustiano todos los días y nada me ha dicho.

---Esas cosas no se dicen.

---A mí sí\ldots{} Hoy me lo dirá.

---No dirá nada, como no hable la torre.

---¿Por qué?\ldots{} ¿También Olózaga ha sido preso?

---También está allí; ¡ay!---replicó lúgubremente Pipaón señalando la
parte de la calle que iban dejando a la zaga.

---¡Qué atrocidad! Usted me engaña\ldots{} Que pare el coche. Quiero
entrar en casa de Bringas a preguntarle\ldots{}

---Guarda, Pablo---dijo el cortesano deteniendo a la señora en su brusco
movimiento para avisar al cochero.---El Sr.~Bringas también\ldots{}

---¿Está allí, en el torreón?

---No: a ese se le ha puesto en la de Corte.

---Iznardi me dirá algo\ldots{} Cochero, a casa de Iznardi.

---¿Iznardi?\ldots{} Ya pedí permiso para dar malas noticias, señora.

---¿También él?

---Y Miyar. Y la misma suerte habría tenido Marcoartú si no hubiera
saltado por un balcón.

---Es una iniquidad. Yo hablaré a Calomarde---manifestó con soberbia la
dama, echando atrás su mantilla, como si dentro del coche reinase un
verano riguroso.

---¡Oh!, sí, hable usted a Su Excelencia---dijo el cortesano, con
aquella sonrisa traidora que ponía en su cara un brillo semejante al del
puñal asesino al salir de la vaina.---Su Excelencia desea mucho ver a
usted.

---Dios maldiga a Su Excelencia y a usted---exclamó Jenara abriendo y
cerrando su abanico con tanta fuerza y rapidez que sonaba como una
carraca.---Pero todavía no me ha dicho usted lo principal.

---A eso voy. Nuestro amigo llegó aquí, según se supone, pues de cierto
no lo sé, con recadillos de Mina, Valdés y demás brujos del aquelarre
democrático. Estuvo oculto en Madrid por algunos días; luego pasó a
Aranjuez y a Quintanar de la Orden para entenderse con ciertos militares
que a estas horas están también a la sombra; regresó después acá
concertando con Bringas, Olózaga, Miyar y compañeros mártires un plan de
revolución que si les llega a cuajar ¡ay mi Dios!, se deja atrás a la de
Francia\ldots{} Nuestro buen amiguito se pinta solo para estas cosas, y
andaba por ahí llamándose \emph{Don No sé Cuántos} Escoriaza.

---¿Y está usted seguro de que es él?

---Seguro, seguro no. Ahora será fácil saberlo, porque el Escoriaza está
en la cárcel de Villa, y en la causa ha de salir su verdadero
nombre\ldots{} Sigo mi cuento. Un hombre dignísimo, tan enemigo de
revoluciones como amante de la paz del reino, se enteró de la trama y
avisó a Su Excelencia. Yo he visto las cartas del denunciante que se
firma \emph{El de las diez de la noche}, y si he de decir verdad su
ortografía y su estilo no están a la altura de su realismo. Calomarde
recompensó al desconocido dándole fondos para que pudiera seguir la
pista a Escoriaza y los suyos, y con esto y un habilidoso examen de
todas las cartas del correo, se hizo el hallazgo completo de los nenes,
y anoche se les puso donde siempre debieran estar para escarmiento de
bobos. Anoche no nos acostamos en Gracia y Justicia hasta no saber que
los señores Alcaldes habían salido de su paso. ¡Ah!, esos señores Cavia
y Cutanda valen en oro más de lo que pesan. No sé cuál de los dos fue a
casa de Olózaga; pero un alguacil me ha contado que en el portal
encontraron a Pepe y mandándole salir entraron con él en la casa y
dieron al pobre D. Celestino un susto más que mediano. Hicieron registro
escrupuloso, encontrando, en vez de papeles de conspiración, muchas
cartas de novias y queridas. Excuso decir que las leyeron todas, porque
así cuadraba al buen servicio de Su Majestad, y cuando estaban en esta
ocupación dulcísima, ved aquí que entra Salustiano muy sereno, con
arrogancia, ya sabedor de que andaba por allí la nariz de los señores
Alcaldes. El padre gimió, desmayose la hermana, siguió el registro dando
por resultado el hallazgo de un sable, y a la media noche se llevaron a
Salustiano a la Villa, y aquí se acabó mi cuento, \emph{arre borriquito
para el convento}\ldots{} ¡Pobre Salustiano, tan joven, tan guapo, tan
listo, tan simpático! ¡Desgraciado él mil veces, y desgraciado también
ese amigo nuestro que ahora se esconde debajo del nombre de Escoriaza!
Esta vez no escapará del peligro como tantas otras en que su misma
temeridad le ha dado alas milagrosas para salir libre y
triunfante\ldots{} ¡Infelices amigos!

Micaelita, afectada por la tristeza del relato, volvió a cerrar los ojos
y a rezar para sí el \emph{Padrenuestro} que tenía dispuesto para cuando
lo melancólico de las circunstancias lo hiciera menester. Jenara seguía
imprimiendo a su abanico los movimientos de cierra y abre, cuyo ruido
semejaba ya por lo estrepitoso, más que al instrumento de Semana Santa,
al rasgar de una tela.

Durante un buen rato callaron los tres. Había entrado el coche en el
paseo de Atocha cuando vieron que por este venía a pie D. Tadeo
Calomarde, en compañía de su inseparable sombra el Colector de Espolios.
Paseaba grave y reposadamente, con casaca de galones, tricornio en
facha, bastón de porra de oro, y una vistosa comitiva de sucios
chiquillos que admirados de tanto relumbrón le seguían. El célebre
ministro, a quien Femando VII tiraba de las orejas, era todo vanidad y
finchazón en la calle; si en Palacio adquirió gran poder fomentando los
apetitos y doblegándose a las pasiones del Rey, frente a frente de los
pobres españoles parecía un ídolo asiático en cuyo pedestal debían
cortarse las cabezas humanas como si fuesen berenjenas. A su lado iba la
carroza ministerial, un armatoste del cual se puede formar idea
considerando un catafalco de funeral tirado por mulas.

---No le salude usted, ocúltese usted en el fondo del coche---dijo
Pipaón con mucho apuro.---No conviene que la vea a usted.

Mas ella sacó fuera su linda cabeza y el brazo y saludó con mucha gracia
y amabilidad al poderoso ídolo asiático.

---En estos tiempos---dijo la dama al retirarse de la
portezuela,---conviene estar bien con todos los pillos.

---Señora, que los coches oyen.

---Que oigan.

Seria, cejijunta, descolorida Jenara murmuró algunas palabras para
expresar el desprecio que le merecía el abigarrado tiranuelo a quien
poco antes saludara con tanta zalamería. En seguida dio orden al cochero
de marchar a casa.

Pasaban por el Prado cuando Pipaón dijo con cierta timidez, precedida de
su especial modo de sonreír:

---Señora, ¿se permite la verdad?

---Se permite.

---¿Aunque sea amarga?

---Aunque sea el mismo acíbar.

---Pues debo decir a usted que no puede ir a su casa.

---¡Que no puedo ir a mi casa!

---No, señora mía apreciabilísima, porque en su casa de usted encontrará
al alcalde de Casa y Corte y a los alguaciles que desde la una de la
tarde tienen la orden de prender a una de las damas más hermosas de
Madrid.

---¡A mí!---exclamó la ofendida, disparando rayos de sus ojos.

---A usted\ldots{} Triste es decirlo\ldots{} pero si yo no lo dijera,
sacrificando a la amistad el servicio del Rey, la señora tendría un
disgustillo. Ya está explicado este buen acuerdo mío de entretener a
usted toda la tarde, impidiéndole ir a su casa y facilitándole como le
facilitaré, un lugar donde se oculte.

---¡Presa yo!\ldots{} No siento ira, sino asco, asco Sr.~de
Pipaón---exclamó la dama demostrando más bien lo primero que lo
segundo.---¿Por qué me persiguen?

---No sé si será por alguna denuncia malévola o causa de los papeles
hallados en casa de Olózaga\ldots{}

---Alto ahí, señor desconsiderado. En casa de Salustiano no se han
encontrado papeles de mi letra porque no los hay.

---Perdones mil señora: no tuve intención\ldots{}

---¡Presa yo!\ldots{} será preciso que me oculte hasta ver\ldots{} ¡Y yo
saludaba a la serpiente!\ldots{}

La rabia más que el dolor sacó dos ardorosas lágrimas a sus ojos; pero
se las limpió prontamente con el pañuelo cual si tuviera vergüenza de
llorar. Después rompió en dos el abanico. Al ver estas lamentables
muestras de consternación, Micaelita se conmovió mucho, y sin pensarlo,
se le vino a la boca el \emph{Padrenuestro} que de repuesto estaba. A la
mitad lo interrumpió para decir a su amiga.

---Puedes venir a casa.

---Me parece muy bien. Nadie sospechará que el Sr.~Carnicero oculta a
los perseguidos de la justicia Calomardina\ldots{} Cochero, a casa de
Micaelita.

\hypertarget{xvii}{%
\chapter{XVII}\label{xvii}}

Hacia el promedio de la calle del Duque de Alba vivía el Sr.~D.
Felicísimo Carnicero, del cual es bien que se hable en esta ocasión, no
sólo porque se prestó a dar asilo a la afligida amiga, sino porque dicho
señor merece un párrafo entero y hasta un capítulo. Era de edad muy
avanzada, pero inapreciable, porque sus facciones habían tomado desde
muy atrás un acartonamiento o petrificación que le ponía, sin que él lo
sospechara, en los dominios de la paleontología. Su cara, donde la piel
parecía haber tomado cierta consistencia y solidez calcárea, y donde las
arrugas semejaban los hoyos y los cuarteados durísimos de un guijarro,
era de esas caras que no admiten la suposición de haber sido menos
viejas en otra época. Fuera de esta apariencia de hombre fósil, lo que
más sorprendía en la cara de don Felicísimo era lo chato de su nariz, la
cual no avanzaba fuera de la tabla del rostro más que lo necesario para
que él pudiera sonarse Y la \emph{chateza} (pase el vocablo) del señor
Carnicero era tal que no se circunscribía al reino de la nariz sino que
daba motivo a que el espectador de su merced hiciera las suposiciones
que vamos a apuntar. Todo el que por primera vez contemplaba al Sr.~D.
Felicísimo, suponía que su rostro había sido hecho de barro o pasta muy
blanda, y que en el momento en que el artista le daba la última mano, la
máscara se deslizó al suelo cayendo de golpe boca abajo, con lo que
aplastada la nariz y la región propiamente facial resultó una superficie
plana desde la raíz del cabello hasta la barba. El espectador suponía
también que el artista, viendo cómo había quedado su obra, la encontró
graciosa y echándose a reír la dejó en tal manera.

Ahora pongamos el santo en su nicho. A esta máscara chata, de color de
tierra, rugosa y dura, añadamos primero por la parte superior un gorro
negro que hasta el campo de las orejas se encaja y tiene su coronamiento
en una borlita que ora se inclina al lado derecho, ora al izquierdo.
Añadámosle por debajo un corbatín negro a quien sería mejor llamar
corbatón, tan alto que por ciertas partes se junta con el gorro, dejando
escapar algunos cabellos rucios, que a hurtadillas salen a estirarse al
aire y a la luz, recordando aún con tristeza suma las grasas olientes
que han tenido en el pasado siglo. Desde los dominios de la corbata, en
cuyas paredes metálicas parece tener cierto eco la voz de D. Felicísimo,
pongamos un revuelto oleaje de pliegues negros, el cual o no es cosa
ninguna o debe llamarse levitón, más que por la forma, por el ligero
matiz de ala de mosca que en las partes más usadas se advierte;
derivemos de este levitón dos cabos o brazos que a la mitad se enfundan
en manguitos verdes con rayas negras como los mandiles de los maragatos,
y hagamos que de las bocas de esos manguitos salgan, como vomitadas,
unas manos, de las cuales no se ven sino diez taponcillos de corcho que
parecen dedos. El resto de la persona no puede verse porque lo ponemos
detrás de la mesa, la cual está cubierta de negro hule que en ciertos
sitios pasaría por playa, a causa de la arenilla que en ella se
extiende. Es mesa de camilla, y una faldamenta verde la tapa toda
honestamente, la cual enagua no se mueve sino cuando el gato entra para
enroscarse en la banqueta junto a los pies de D. Felicísimo. Encima de
la mesa, se ve un Cristo pequeño atado a la columna, con la espalda en
pura llaga y la soga al cuello, obra de un realismo espantoso y
aterrador que se atribuye al célebre Zarcillo. La escultura está a la
derecha y vuelve su rostro dolorido y acardenalado al D. Felicísimo,
cual si le pidiera informes y cuentas, más que de los azotes que le han
dado los judíos, de los motivos porque está en aquella mesa y entre tal
balumba de legajos como allí se ven. Son papeles atados con cintas
rojas, paquetes de cartas y algunos libros de cuentas, cuyas sebosas
tapas indican los años que llevan de servicio. La escribanía es de
cobre, pues aunque D. Felicísimo posee algunas de plata, no las usa, y
en la que allí está los dos cántaros amarillos tienen tinta y arena para
seis meses. Las plumas de puro mosqueadas no tienen color, y hay un
pisa-papeles que es la pezuña de un cabrón imitada en bronce, y está tan
al vivo que no le falta más que correr.

En aquella mesa escribe casi todo el día el Sr.~Carnicero, a quien el
peso de los años no estorba para seguir trabajando; allí toma su
chocolate macho con bollo maimón; allí come su cocidito con más de vaca
que de carnero, algo de oreja cerdosa y algunas hilachas de jamón que el
vacilante tenedor busca entre los garbanzos azafranados; allí duerme la
siesta, echando la cabeza sobre las orejeras del sillón; allí se le
sirve la cena que empieza invariablemente en migas esponjosas y acaba en
guisado de ternera, todo muy especioso y aromático; allí cuenta el
dinero que es, según dicen, el más constante de sus visitadores, y se
desliza sin hacer ruido por entre sus dedos alcornoqueños, cual si por
virtud rara también el oro se sometiese a tomar las apariencias del
corcho o del pergamino en aquel imperio del silencio; allí recibe a los
que van a ocuparle, y son por lo general clérigos o frailes, y allí está
cuando entran Jenara, Pipaón y Micaelita.

Era ya de noche. Un gran candil de cuatro mecheros, de los cuales sólo
dos estaban encendidos, echaba luz no muy copiosa, que la pantalla
dirigía sobre el pupitre. Al sentir gente, D. Felicísimo alzó la
pantalla de cobre y entonces la claridad le hirió de frente en su cara
plana, que parecía un bajo-relieve gótico, roído por los siglos. Pero
esto duró poco tiempo, porque abatiendo la pantalla, volvió la luz a
caer forzosamente sobre los papeles como un estudiante desaplicado a
quien se obliga a no apartar la vista de los libros.

---¡Oh!\ldots{} \emph{gratias tibi Domine}\ldots{} Bendito Pipaón,
¿usted por aquí?---dijo D. Felicísimo con agrado.---¡Oh! ¿Es Jenarita?
La misma que viste y calza. Sea muy bien venida a esta humilde morada.
¡Cuánto bueno por aquí!

Y alzando la voz, que era chillona y desapacible, prosiguió:

---Sagrario, Sagrario, ven, mira quién está aquí. Micaelita, di a tu tía
que venga, y de paso da una voz en la cocina para que me traigan la
cena.

Mientras viene doña María del Sagrario, hija del Sr.~D. Felicísimo,
demos acerca de este señor las noticias que son necesarias. Llevaba más
de cuarenta años en la profesión de agente de negocios eclesiásticos, y
le había sido tan favorable la fortuna que, según el dicho público,
estaba \emph{podrido de dinero}. Por los rótulos de los legajos y
papeles que sobre su mesa estaban, podía venirse en conocimiento de la
multiplicidad de asuntos que bajo el dominio de sus talentos agenciales
caían. Él contemplaba con no disimulado embeleso los dichos rótulos,
asemejándose, aunque esté mal la comparación, a un borracho que antes de
beber se deleita leyendo las etiquetas de las botellas. Por un lado se
leía \emph{Subcolecturía de Espolios}, \emph{Vacantes}, \emph{Medias
Annatas y Fondo pío beneficial del obispado de León}; por otro
\emph{Santa Iglesia Metropolitana de Granada}; más allá \emph{Juzgado
ordinario de Capellanías}, \emph{Patronatos}, \emph{Visita
Eclesiástica}, etc.; junto a esto \emph{Tribunal de Cruzada}, y al lado
\emph{Racioneros medios patrimoniales de Tarazona}, \emph{Arcedianato de
Murviedro} o \emph{Señores Pabordres de Valencia;} al opuesto extremo
\emph{Agustinos Descalzos}; más lejos \emph{Reyes Nuevos de Toledo}, o
bien, \emph{Nuestra Señora del Favor de Padres Teatinos}.

Preciso es decir que D. Felicísimo se había distinguido siempre por su
celo y actividad en despachar los mil y mil asuntos que se le confiaban.
Les tomaba cariño, mirándolos como cosa propia, y ponía en ellos sus
cinco sentidos y su alma toda en tal manera que llegó a identificarse
con ellos y a asimilárselos, trayéndolos como a formar parte de su
propia sustancia. Así no había en su larga vida suceso ni accidente que
no se confundiera con cualquier negocio de su lucrativa profesión, y así
jamás contaba cosa alguna sin empezar de este o parecido modo:
\emph{Cuando el señor Vicario Foráneo de Paterna venía a esta casa}, o
bien así: \emph{Cuando me convidó a comer el Padre Prepósito de
Portaceli}\ldots{}

Otra afición también muy vehemente, aunque secundaria, reinaba en el
espíritu de nuestro insigne Carnicero; era la afición a los Toros,
fiesta que, si no existieran los negocios eclesiásticos, sería para él
cosa punto menos que sagrada. Como ya era tan viejo y no salía ya de
casa, contentábase con hablar de los Toros pretéritos, poniéndolos cien
codos más altos que los presentes y en estas conversaciones también era
común oírle decir: \emph{«Cierto día en que Sentimientos y el señor
Rector del Hospital de Convalecencia de Unciones vinieron a buscarme
para ir a ver el encierro\ldots»} u otra frase por el estilo.

La cantidad de dinero que D. Felicísimo había ganado en tantos años de
actividad, celo y honradez, no era calculable. Algunos la hacían subir a
un número grande de talegas, otros reducían un poco la cifra; pero el
vulgo y los vecinos juraban que siempre que se daba un golpe en los
tabiques de la casa de Carnicero o en el lienzo de los cuadros viejos
que allí tenía, sonaba un cierto tintineo como de monedas anacoretas que
en todos los huecos y escondrijos habitaban, huyendo del mundo y sus
pompas vanas. Él gastaba poco, tan poco que se había llegado a hacer la
ilusión de que era pobre, siendo rico. Contaban que para ilusionar a los
demás en esta materia se negaba con tenacidad heroica a dar dinero, y ya
podían irle con lamentos los menesterosos, que así les hacía caso como
si fueran predicadores moros. Únicamente se desprendía de alguna
cantidad siempre que mediaran garantías y un interés módico, así así
como de diez por ciento al mes u otra friolera semejante.

La casa en que vivía era de su propiedad y estaba toda blanqueada, sin
papeles ni pinturas, con las vigas del techo apanzadas cual toldo de
lienzo. Era de un solo piso alto, antiquísima, y en invierno tenía
condiciones inmejorables para que cuantos entraban en ella se hicieran
cargo de cómo es la Siberia. Había sido edificada en los tiempos en que
la calle del Duque de Alba se llamaba \emph{de la Emperatriz}, y ya, con
tan largos servicios, no podía disimular las ganas que tenía de
reposarse en el suelo, soltando el peso del techo, estirándose de
tabiques y paredes para sepultar su cornisa en el sótano y rascarse con
las tejas de su cabeza los entumecidos pies de sus cimientos. Pero D.
Felicísimo que no consentía que su casa viviera menos que él, la
apuntaló toda, y así desde el portal se encontraban fuertes vigas que
daban el \emph{quién vive}. La escalera, que partía de menguados arcos
de yeso, también tenía dos o tres muletas, y los escalones se echaban de
un lado como si quisieran dormir la siesta. Arriba los pisos eran tales,
que una naranja tirada en ellos hubiera estado rodando una hora antes de
encontrar sitio en que pararse, y por los pasillos era necesario ir con
tiento so pena de tropezar con algún poste, que estaba de centinela como
un suizo con orden de no permitir que el techo se cayera mientras él
estuviese allí.

D. Felicísimo era toledano, no se sabe a punto fijo si de Tembleque o de
Turleque o de Manzaneque, que los biógrafos no están acordes todavía.
Estuvo casado con doña María del Sagrario Tablajero, de la que nacieron
Mariquita del Sagrario y Leocadia. De esta, que casó pronto y mal con un
tratante en ganado de cerda, nació Micaelita, que se quedó huérfana de
padre y madre a los seis años. Esta Micaelita era, pues, heredera
universal del Sr.~D. Felicísimo, circunstancia que, a pesar de su escasa
belleza, debía hacer de ella un partido apetitoso. Sin embargo, habiendo
tenido en sus quince años ciertos devaneos precoces con un muchacho de
la vecindad, quedó muy mal parada su honra. El mancebo se fue a las
Américas, D. Felicísimo enfermó del disgusto, doña María del Sagrario,
tía de la joven, enfermó también; divulgose el caso, salió mal que bien
de su paso Micaelita, y desde entonces no hubo galán que la pretendiera.
Cuentan los cronistas toledanos que desde entonces se arraigó en
Micaelita la piadosa costumbre de reservar un Padrenuestro para todas
las ocasiones apuradas en que se encontrase.

Pasados algunos años, la situación de la joven había cambiado: su
carácter agriándose en extremo la hacía menos simpática aún de lo que
realmente era. Su abuelo, que entrañablemente la amaba, le permitía
frecuentar la sociedad y gastar algo en tocados y ropas de moda. Ella
quería borrar su mancha; pero no lo podía conseguir, careciendo de
aquellas prendas que fácilmente inspiran el perdón o el olvido. Lo
singular es que a su mal genio unía un cierto orgullito sobremanera
repulsivo y que sin duda nacía de su seguridad de enriquecer
considerablemente al fallecimiento del abuelo.

Todas las noches del año, en el de 1831, luego que D. Felicísimo con un
mediano vaso de vino echaba la rúbrica a su cena (frase de D.
Felicísimo), se levantaba de aquella especie de trono, y tomando con su
propia mano el candil de cuatro mecheros se dirigía a la sala, donde ya
doña María del Sagrario había encendido una lámpara de las llamadas de
\emph{Monsieur Quinquet}, y allí se encontraba a varios amigos que se
reunían en amena tertulia. La estancia era como una gran sala de
capítulo conventual; pero estaba blanqueada, sin más adorno que un gran
cuadro del Purgatorio donde ardían hasta diez docenas de ánimas. Dos
cortinas de sarga, cuya amarillez declaraba haber sido verde, cubrían
los balcones, y por las cuatro paredes se enfilaban en batería tres
docenas de sillas de caoba con el respaldo tieso y el asiento durísimo.
Cuatro sillones de cuero claveteado, contemporáneo del cuadro de las
Ánimas del Purgatorio, si no del Purgatorio mismo, servían para la
comodidad relativa; una urna con imagen vestida servía para la devoción,
y una mesa que parecía pila bautismal para que dieran golpes sobre ella
los de la tertulia. D. Felicísimo entraba diciendo, \emph{Pax vobis} y
después saludaba sucesivamente a sus amigos.

---Buenas noches, Elías ¿cómo te va?\ldots{} Señor conde de Negri,
buenas noches\ldots{} Buenas noches, Sr.~D. Rafael Maroto.

\hypertarget{xviii}{%
\chapter{XVIII}\label{xviii}}

Veamos ahora lo que pasó aquella noche. Jenara tomó asiento en el
despacho del señor D. Felicísimo, y Pipaón, acercándose a este, le habló
un poco al oído para contarle lo que a la dama le pasaba. A cada dos
palabras que oía, D. Felicísimo articulaba una especie de chillido, un
ji ji, que más tenía de suspiro que de interjección y que al mismo
tiempo expresaba hipo y burla.

---Bueno, bueno---murmuró el anciano moviendo la cabeza en ademán de
conciliación.---En mi casa no será molestada; yo le respondo de que no
será molestada, ji ji.

---Gracias---dijo la dama secamente tratando de darse aire con los
restos de su abanico.

---El Sr.~D. Miguel de Baraona y yo fuimos muy amigos---añadió
Carnicero, volviendo a Jenara su faz plana, fría, sin expresión de
sentimiento alguno,---pero muy amigos. Cuando aquellas cuestiones de la
Santa Iglesia Colegial de Vitoria con los \emph{Canónigos cuartos de
frutos de Calahorra}, vino aquí don José Marqués, \emph{canónigo
entero}, D. Vicente Morales, \emph{racionero medio} y D. Andrés de
Baraona, \emph{canónigo cuarto de optación}, hermano de su abuelo de
usted que también vino. Yo le conseguí el arcedianato de Berberiega para
su primo. ¡Cuántas tardes pasamos juntos en este despacho hablando de
sermones y Toros! Era en los tiempos de Pedro Romero y dicho se está que
había materia para dos buenos aficionados como nosotros. Si el señor de
Baraona viviera se acordaría de cuando vimos la cogida de Pepe-Hillo y
la célebre cornada de José Cándido, motivada por haberse \emph{escupido}
el toro, con lo que se atolondró José y quiso matarlo fuera de la
jurisdicción, recibiendo un encontronazo\ldots{}

Estas últimas frases no las dirigía D. Felicísimo a Jenara, sino a
cierto personaje, desconocido para nosotros, que a su lado estaba y
había entrado poco antes que nuestros amigos. Era un joven de aspecto
más bien ordinario que fino, de rostro tan salpicado de viruelas, que
parecía criba, de complexión sanguínea y algo gigántea; de ajustada
chaqueta vestido, con el pelo corto y la frente más corta acaso. Su
facha, su traje y cierta expresión inequívoca que impresa en su rostro
estaba como un letrero, decían que aquel hombre era del gremio de
tablajeros, cortadores o tratantes en carnes. Los tres oficios había
tenido, mas con tan poco aprovechamiento, que los cambió por una plaza
de demandadero en la cárcel de Villa. Era hijo de una antigua sirviente
de D. Felicísimo y este le había criado en su casa y le tenía bastante
cariño. Pedro López, por otro nombre \emph{Tablas} (que así le
bautizaron en el Matadero), respetaba mucho a su protector. Iba a verle
diariamente al anochecer, se sentaba a su lado, le hablaba un poco de la
cárcel, de becerros si era invierno y de Toros si era verano; después le
servía la cena, y por último le acompañaba a rezar el rosario, devoción
a que no faltó D. Felicísimo ni en un solo día de su vida.

Doña María del Sagrario no tardó en venir. Era una señora que aparentaba
más edad de la que realmente tenía, por causa de una lamentable
emigración de todos los dientes de su boca, no quedando en aquellos
reinos más que algunas muelas, que temblando habían pedido también sus
pasaportes. Ella no tenía pretensiones de belleza ni aun de buen
parecer, y así su elegancia era la sencillez, su perfumería la limpieza
y su peinado un trabajo simplicísimo. Este consistía en recoger en una
sola trenza los cabellos fieles que le quedaban y hacer con esta un moño
chiquito, el cual, atravesado de una horquilla o flecha, como corazón
simbólico, parecía una limosna de cabellos enviada por el Cielo sobre su
cráneo, que iba igualando a las encías en sus condiciones de país
desierto. Por lo demás, Doña María del Sagrario era bondadosa, de
excelente corazón y de mucho palique; pero tanto desentonaba su voz, por
causa de estar su boca tan solitaria como casa de mostrencos, que las
palabras parecían salir y entrar por aquellas cavidades jugando y
haciendo cabriolas. Cuando reía creeríase que lloraba, y cuando regañaba
a la criada parecía mandar un batallón, y el rezar era en ella como un
soplamiento de fuelles rotos.

---Mucho nos honra usted, Jenarita---le dijo besándola,---con aceptar
nuestra hospitalidad. Eso no será nada. Algún mal entendido. ¡Es tan
fácil ahora que los buenos se confundan con los pícaros! Ayer mismo ¿no
apalearon en esta calle al sacristán de la V. O. T. por confundirlo con
un pícaro zapatero que fue condenado a horca y luego indultado en el
\emph{llamado tiempo constitucional}, que ni fue tal tiempo ni cosa que
lo valga?

---Sagrario, mucha conversación es esa, ji ji---dijo a este punto D.
Felicísimo.---Jenarita no es persona con quien debemos gastar cumplidos
ni etiquetas; por tanto, tráeme mi cena, que la gusana me dice que es
hora.

Poco después el Sr.~Carnicero tenía delante la servilleta en lugar del
papel y la cuchara en vez de la pluma. Tras los primeros bocados, habló
así:

---No es extraño, Jenarita, que con la marcha que lleva este Gobierno
por el camino de la francmasonería, sean perseguidos los buenos
españoles. Ese pobre Rey se ha entregado en manos de la herejía y del
democratismo; la Reina nos quiere embobar con músicas pero no le valdrán
sus mañas para hacernos tragar la sucesión de su hija Isabelita, que así
será reina de España como yo emperador de la China, ji ji. Ellos ven
venir el nublado y se preparan, pero nosotros nos preparamos
también\ldots{} y es flojita cosa la que defendemos\ldots{} así como
quien no dice nada, la religión sacratísima, el trono español y nuestras
costumbres tradicionales, puras, nobles y sencillas. ¡Ah!, perdóneme
usted, Jenarita, me olvidé de decirle si gustaba cenar. Pero aquí no
andamos con etiquetas y en mi casa todo es llaneza y confianza.

---Gracias---repuso Jenara que solicitada de otros pensamientos no había
oído ni una sola palabra del discurso del Sr.~Carnicero.

Pipaón y Micaelita cuchicheaban en la sala inmediata y doña María del
Sagrario había ido a preparar la cena para todos, lo que requería no
poca habilidad por haber aumentado las bocas y no los manjares. Tablas
servía la cena al Sr.~D. Felicísimo, el cual le hablaba de este modo:

---Pues volviendo a lo que te decía cuando entraron estos señores, el
toreo está ahora tan por los suelos que no se puede hablar de él sin que
se le caiga a uno la cara de vergüenza. Y no me digan que se ha fundado
un Conservatorio de Tauromaquia. Tonto de capirote es el que lo inventó.
Yo admiro a Don Pedro Romero, yo le tengo por un Cid de los tiempos
modernos; por eso no quisiera verle hecho un catedrático de brega. Mira
tú, los toreros de hoy dan asco\ldots{} Si el Señor Omnipotente te
hubiera querido hacer el favor de criarte en aquel tiempo en que todo
era mejor que ahora, todo, todo; en que era más honrada la gente, más
rico el país, más barata la comida, más guapas las mujeres, más
religiosos los hombres, más valientes los militares, más benigno el
frío, más alegre el cielo, más honestas las costumbres, más bravos los
toros y más, mucho más hábiles los toreros\ldots{} ji ji\ldots{} ¿por
qué te ríes?

El hipo de D. Felicísimo arreció de tal modo que hubo de pararse un rato
para tomar aire. Después prosiguió así:

---Si hubieras vivido en aquel feliz tiempo, te habrías desbaratado de
gusto viendo en medio del redondel a Joaquín Rodríguez, por otro nombre
\emph{Costillares}, o a José Delgado, mi amigo queridísimo, por otro
nombre \emph{Pepe-Hillo}. Me parece que le estoy mirando, cuando el toro
se ceñía. Entonces tenías que ver su serenidad y destreza, ji. Él lo
llamaba de frente, tomando la rectitud de su terreno conforme las
piernas que le advertía la fiera, y luego que le partía, ji, le empezaba
a cargar y tender la suerte, ¿entiendes? Con este quiebro el toro se iba
desviando del terreno del diestro y cuando llegaba a jurisdicción, le
daba el remate seguro, ji, ji, ji.

Con las cabezadas que daba D. Felicísimo brillaban sus ojos en el
semblante plano como los agujeros de una palmeta. Al mismo tiempo su
mano armada de tenedor tomaba las actitudes toreriles amenazando el vaso
de vino, puesto en el lugar del tintero.

---Señora, usted se aburrirá con esta conversación mía---dijo el anciano
contemplando a Jenara que estaba con los ojos bajos.---Como aquí no hay
cumplimientos, que es palabra compuesta de \emph{cumplo} y
\emph{miento}, ni las pamemas que llaman etiqueta, yo hablo de lo que
más me gusta, ji. Este buen \emph{Tablas} es un chiquilicuatro que por
no tener alma no ha emprendido el oficio de mirar cara a cara a la
cuerna, y está de demandadero en la cárcel de Villa. Si no tuviera el
defecto de coger sus monas los lunes y aun los martes, sería un cumplido
muchacho, siempre que se corrigiera del vicio de sobar las cuarenta.

Tablas se ruborizó al oír su panegírico.

---Jenara, venga usted a cenar---dijo Sagrario entrando.---Deme usted su
mantilla.

Don Felicísimo había concluido.

---Hija, ¿ha venido esta tarde el padre Alelí?---preguntó.

---No ha parecido Su Reverencia.

---¿No se sabe nada de la pupila de Benigno Cordero, que está con
pulmonía?

---Iba mejor, pero ha recaído. ¡Cristo, qué desgracia!---exclamó
Sagrario en un desentono tan singular que parecía enjuagarse la boca con
las palabras.---Cruz fue esta tarde a la iglesia y me dijo que el pobre
Benigno está como alma en pena. Va a la botica por las medicinas y se
deja el sombrero sobre el mostrador, habla solo y cuando vende no cobra
y cuando cobra no da la vuelta, y cuando la da, da oro por cobre.

---Es un alma de cántaro, ji\ldots{} Tablas, ve después a preguntar por
la enferma. Benigno es loco, pero es paisano y le aprecio\ldots{}
Jenarita, ¿por qué tiene usted ese aire de tristeza y abatimiento? Aquí
no hay nada que temer. Estamos en sagrado, es decir en una casa pura y
absolutamente, ji ji\ldots{} apostólica.

~

Jenara no cenó. Había perdido el apetito, y la especial manera de guisar
que en aquella casa había no era la más a propósito para despertarlo. A
esta feliz circunstancia de la desgana de un convidado, debió Pipaón que
le tocara algo, aunque no fue mucho, según consta en las crónicas que de
aquellos acontecimientos quedaron escritas.

Levantose Jenara de la mesa antes que los demás para decir una cosa
importante al señor D. Felicísimo, que aún no había salido de su
guarida, y al llegar a la puerta de esta, oyó la voz del anciano muy
desentonada y colérica. Decía así:

---Ladrón, verdugo, borracho, no te daré un maravedí aunque te me pongas
de rodillas delante y me enciendas velas. Yo no soy bueno, yo no soy
santo; no pienses que me embobarás con tus lisonjas. ¿Tengo yo alguna
mina, ji? ¿Acuño moneda, ji? Quítateme, ji, de delante y púdrete si
quieres. No hay un cuarto; hoy no se fía aquí. Toca a otra puerta,
muérete, revienta, pégate un tiro y si no basta, ji, ji\ldots{} te pegas
dos o media docena.

Con voz humilde y ahogada por la pena, Tablas habló después para pintar
con las frases más amañadas la enormidad de su apuro, y Carnicero
redobló sus negativas, sus bufidos, sus hipos, todo en defensa de su
bolsa. Jenara no necesitó oír más, y al punto renunció a decir a D.
Felicísimo lo que había pensado. Mujer de recursos intelectuales,
improvisaba planes con la celeridad propia de todo grande y fecundo
ingenio.

La campanilla sonó y Tablas fue a abrir la puerta. Llegaron tres señores
que se dirigieron a la sala, donde Sagrario acababa de poner luz.
Entrando otra vez en el comedor la dama vio que Pipaón y Micaelita no
parecían disgustados de hallarse juntos. Sagrario andaba por la cocina
riñendo con la criada, en lenguaje discorde e inarmónico, semejando un
órgano que tuviera todos los tubos agujereados. Jenara volvió al
pasillo, que era largo, complicado, anguloso y a causa del blanqueo daba
más cuerpo a las sombras que sobre él caían. Allí vio la atlética figura
de Tablas que salía del cuarto del señor, y dirigiéndose a un ángulo
oscuro donde estaban algunos muebles viejos como en destierro, dejábase
caer sobre una silla y apoyaba la cabezota en ambas manos mirando al
cielo. Jenara se llegó a él. Era el ángel del consuelo.

\hypertarget{xix}{%
\chapter{XIX}\label{xix}}

---¿Cómo te va, Elías? Señor conde de Negri, buenas noches. Buenas
noches, Sr.~D. Rafael Maroto.

Así saludó D. Felicísimo a sus amigos, entrando en la sala, candilón en
mano. Como aún no le hemos visto andar, no hemos podido decir que andaba
a pasitos cortos, muy cortos, y así tardó una buena pieza en llegar al
centro de la estancia. Viose entonces la longitud de su levitón negro,
el cual le llegaba hasta los pies, de modo que no parecía que andaba,
sino que estaba fijo sobre una tablilla con ruedas de la cual tirara con
lentitud una invisible mano. Puso el candilón sobre la mesa, y como la
vecindad de la lámpara hacía que aquel palideciera de envidia, lo apagó.

---Usted siempre tan fuerte---dijo uno de los amigos dando un palmetazo
en la rodilla de Carnicero.

Era este amigo un señor pequeño, o por mejor decir, archipequeño,
adamado y no muy viejo.

---Defendiéndonos admirablemente---repuso Carnicero cogiéndose una
pierna con las manos y levantándola para ponerla sobre la otra.

---Un cigarrito---dijo aquel de los amigos que llamaban Maroto, y era el
más joven de los tres, de buena presencia, bigotudo y con señalado
aspecto marcial.

El conde de Negri, con el cigarrito en la boca, sacó eslabón y piedra y
empezó a echar chispas. Durilla era la faena y la mecha no quería
encenderse.

---¡Maldito pedernal!---murmuró el señor conde.

Y las chispas iban en todas direcciones menos en la que se quería. Una
fue a estrellarse en la cara plana de D. Felicísimo como un proyectil
ardiente en la muralla de un bastión formidable, otra parecía que se le
quería meter por los ojos al propio señor conde, y chispa hubo que llegó
hasta el cuadro de Ánimas dando instantáneamente un resplandor verdadero
a aquel Purgatorio figurado. Al fin prendió la mecha.

---¡Gracias a Dios que tenemos fuego!---dijo D. Felicísimo entre dos
hipos.---Con estos tubos de vidrio que han inventado ahora para encerrar
las luces, no se puede encender en las lámparas.

En tanto el tercero de los amigos, que era bastante anciano y se
distinguía por la curvatura exagerada de su nariz, había puesto unos
papeles sobre la mesa, y los miraba y revolvía atentamente. De repente
dijo así:

---No hay que contar con Zumalacárregui.

---¡Todo sea por Dios!---exclamó Carnicero.---¿Ha escrito? Pues a mi
carta no se dignó contestar. ¿Sigue en el Ferrol?

---Pues nos pasaremos sin él---indicó el conde de Negri.---La causa
revienta de partidarios, quiero decir que los tiene de sobra en todas
las clases de la sociedad, y así no es bien que solicite coroneles, como
es uso y costumbre entre liberalejos.

---Ya sabemos---dijo con tono de autoridad el llamado Elías alzando los
ojos del papel,---que la causa que defendemos es legalmente una batalla
ganada. Habiendo sucesor varón no puede suceder una hembra. Moralmente
también es cosa fuera de duda. El clero en masa apoya al partido de la
religión y con el clero la mayoría del reino, y la aristocracia.

---Y el ejército---declaró el conde pequeñito, plegando mucho los
párpados porque le ofendía la luz.

---Eso está por ver---replicó Elías Orejón.---Desde la guerra de la
Independencia, el ejército, lo mismo que la marina, están carcomidos por
la masonería. La revolución del 23 obra fue de los masones militares;
las intentonas de estos años también son cosa suya, y en estos momentos,
señores, se está formando una sociedad llamada la \emph{Confederación
Isabelina}, en la que andan muchos pajarracos de alto vuelo, y que por
el rotulillo ya da a entender a dónde va. Necesitamos\ldots{}

---¡Claro, clarísimo, indubitable!---exclamó Carnicero, que deseaba
meter baza, por no hallarse conforme con su amigo en aquel tema.

---Necesitamos---prosiguió el otro alzando la voz en señal de enojo por
verse interrumpido,---necesitamos, aunque el escrupuloso señor Infante
no lo crea así, asegurar y comprometer aquellas cabezas militares más
potentes. Ya se puede decir que son \emph{de acá} los siguientes
señores: el conde de España, capitán general del Principado; el
Sr.~González Moreno, gobernador militar de Málaga\ldots{}

---Buenos, buenos, bonísimos---dijo Carnicero, que no podía contener sus
ganas de interrumpir a cada instante.

Orejón citó otros nombres, añadiendo luego.

---En el ramo de hombres civiles o eclesiásticos de gran nota, andamos a
la conquista del Sr.~Abarca, obispo de León, y de D. Juan Bautista Erro,
consejero de Estado, a los cuales sólo les falta el canto de un duro
para caer también de la parte acá.

---Bueno es que los clérigos y hombres civiles vengan---dijo
Maroto,---pero por santa y gloriosa que sea la causa de Su Alteza, y yo
doy de barato que es la causa de Dios, no se hará nada sin tropa.

---¿Y los voluntarios realistas?

---Son buenos como auxilio; pero nada más. Denme generales aguerridos,
jefes de valor y prestigio, y el día en que D. Fernando acabe, que no
tardará, al decir de los médicos, don Carlos será Rey por encima de
todas las cosas.

---Eso, eso---afirmó Elías sentando la palma de su mano sobre los
papeles-generales aguerridos, jefes militares de valor y prestigio; al
grano, al grano.

---Todo vendrá---indicó Carnicero,---cuando el caso llegue. Cuando se
cuenta, como ahora, ji, con el santo clero en masa, capaz de alzar en
masa al reino todo, como en la guerra de la Independencia, lo demás
vendrá por sus pasos contados. En cartas y por manifestaciones verbales,
me han demostrado su conformidad las siguientes órdenes y religiones:
los Agustinos calzados de Madrid, la Congregación benedictina
Tarraconense Cesaraugustana de la corona de Aragón y de Navarra, los
Menores de San Francisco, los Agustinos Recoletos o Calzados, los
Canónigos seglares del Orden Premonstratense\ldots{}

---Espadas, espadas---dijo bruscamente Maroto,---y con espadas, no sólo
no estarán demás las correas y rosarios, sino que servirán de mucho.

---Y yo---indicó el conde de Negri dirigiéndose al balcón a punto que
sonaba en la calle el estrepitoso rodar de un coche,---me atrevo a
proponer que todas las conquistas se pospongan a la conquista del
vecino.

El coche paró junto a la casa. Era el carruaje de Calomarde, que vivía
frente por frente de Carnicero, en el palacio del duque de Alba.

---Su Excelencia ha entrado en su palacio---dijo el conde de Negri,
atisbando por los vidrios verdosos y pequeñuelos de uno de los balcones

.---Todo se andará ---manifestó D. Felicísimo.---La conversación que
tuvimos él y yo hace dos días, me hace creer que D. Tadeo tardará en ser
apostólico lo que tarde Su Majestad en tener, ji, el ataque de gota que
corresponde al otoño próximo.

---Y si no---dijo Negri tornando a su asiento,---le barrerán. Después
veremos quién toma la escoba\ldots{} ¡Cuidado con doña Cristina y qué
humos gasta! Si creerá que está en Nápoles y que aquí somos
\emph{lazzaronis}\ldots{} ¿Pues no se atrevió a pedir mi destitución del
puesto que tengo en la mayordomía del señor Infante? Gracias a que los
señores me han sostenido contra viento y marea. Aquí entre cuatro
amigos---añadió el conde bajando la voz,---puede revelarse un secreto.
He dado ayer un bromazo a nuestra soberana provisional, que va a dar
mucho que reír en la Corte. En imprenta que no necesito nombrar se están
imprimiendo unos versos de no sé qué poeta, en elogio de su majestad
napolitana. Hacia la mitad de la composición se habla de la
\emph{angélica} Isabel y de la \emph{inmortal Cristina}. Pues yo\ldots{}

El conde se detuvo, sofocado por la risa.

---¿Qué?

---Pues yo, como tengo relaciones en todas partes, me introduje en la
imprenta, y di ocho duros al corrector de pruebas para que quitara
bonitamente la t de la palabra inmortal.

---La \emph{inmoral} Cristina, ji ji\ldots{}

---Espadas, espadas---gruñó Maroto,---y no bromas de esta especie que a
nada conducen.

---Toda cooperación debe aceptarse---dijo Elías refunfuñando,---aunque
sea la cooperación de una errata de imprenta.

Cuando esto decían, la luz de la lámpara, ya fuera porque doña María del
Sagrario, firme en sus principios económicos, no le ponía todo el aceite
necesario, ya porque D. Felicísimo descompusiera a fuerza de darle
arriba y abajo el sencillo mecanismo que mueve la mecha, empezó a
decrecer, oscureciendo por grados la estancia.

---Voy a contar a ustedes, señores---dijo Elías,---la conversación que
ayer tuve con el Sr.~Abarca, obispo de León, el hombre de confianza de
Su Majestad\ldots{} Pero D. Felicísimo, esa luz\ldots{}

---Empiece usted. Es que la mecha\ldots---replicó Carnicero moviendo la
llave.

---Pues el señor Abarca me pidió informes de lo que se pensaba y se
decía en el cuarto del Infante. Yo creí que con un hombre tan sabio y
leal como el señor Abarca no debía guardar misterios\ldots{} Le dije pan
pan, vino vino\ldots{} Pero esa luz.

---No es nada; siga usted; ya arderá.

---Le expuse la situación del país, anhelante de verse gobernado por un
príncipe real y verdaderamente absoluto que no transija con masones, que
no admita principios revolucionarios, que cierre la puerta a las
novedades, que se apoye en el clero, que robustezca al clero, que dé
preeminencias al clero, que atienda al clero, que mime al clero\ldots{}
Pero esa luz, señor D. Felicísimo\ldots{}

---Verdaderamente no sé qué tiene. Siga usted.

---Él convino conmigo en que por el camino que va el Rey, marchamos
francamente y él el primero por la senda de la revolución\ldots{} ¡Que
nos quedamos a oscuras!\ldots{}

La luz decrecía tanto que los cuatro personajes principiaron a dejar de
verse con claridad. Las sombras crecían en torno suyo. Los
empingorotados respaldos de los sillones parecían extenderse por las
paredes en correcta formación, simulando un cabildo de fantasmas
congregados para deliberar sobre el destino que debía darse a las
ánimas. Las rojas llamas del cuadro se perdían en la oscuridad, y sólo
se veían los cuerpos retorcidos.

---Díjome también Su Ilustrísima que ahora se va a emprender una campaña
de exterminio contra los liberales\ldots{} ¡Por Dios, Sr.~don
Felicísimo, luz, luz!

La lámpara se debilitaba y moría derramando con esfuerzo su última
claridad por las paredes blancas, y por el techo blanco también. La
llama lanzaba a ratos un destello triste como si suspirase y después
despedía un hilo de humo negro que se enroscaba fuera del tubo. Luego se
contraía en la grasienta mecha, y burbujeando con una especie de lamento
estertoroso, se tomaba en rojiza. Las cuatro caras aparecían ora
encendidas, ora macilentas y la sombra jugaba en las paredes y subía al
techo, invadiendo a veces todo el aposento, retirándose a veces al suelo
para esconderse entre los pies y debajo de los muebles.

---Esa campaña de exterminio que se va a emprender, fíjense ustedes
bien---prosiguió Orejón,---no favorece al Rey, sino al Infante. Todo lo
que ahora sea reprimir es en ventaja de la gente apostólica. Así nos lo
darán todo hecho, y lo odioso del castigo caerá sobre ellos, mientras
que nosotros\ldots{} ¡Luz, luz!

D.~Felicísimo quiso llamar; pero en aquella casa no se conocían las
campanillas. Así es que empezó a gritar también:

---¡Luz, luz; que traigan una luz!

La lámpara se extinguió completamente y todos quedaron de un color.

---¡Luz, luz!---volvió a gritar D. Felicísimo.

Orejón, que estaba muy lleno de su asunto y no quería soltarlo de la
boca, a pesar de la oscuridad, prosiguió así:

---Que utilizando con energía la horca y los fusilamientos, limpien el
reino de esas perversas alimañas, es cosa que nos viene de molde.

---Aguarde usted, hombre\ldots{} Estamos a oscuras\ldots{}

---Ji\ldots{} se han dormido y no nos traen luz---dijo D.
Felicísimo.---Sagrario, Sagrario. Tablas\ldots{} Nada: todos dormidos.

Así era en verdad.

---¿Tiene usted avíos de encender, señor Conde? Aquí en este cajoncillo
de la mesa debe de haber, ji, ji, pajuela.

Pronto se oyó el chasquido del eslabón contra el pedernal. Las súbitas
chispas sacaban momentáneamente la estancia de la oscuridad. Se veían
como a luz de relámpago las cuatro caras apostólicas, la fúnebre fila de
sillas de caoba y el cuadro de ánimas.

---La raza liberalesca y masónica estará ya exterminada cuando llegue el
momento de la sucesión de la corona---decía Orejón
entusiasmado.---¡Admirable, señores!

D.~Felicísimo tenía la pajuela en la mano para acercarla a la mecha
luego que esta prendiese, y al brotar de la chispa, su cara plana, en
que se pintaban la ansiedad y la atención, parecía figura de pesadilla o
alma en pena.

---Trabajan para nosotros, y ahorcando a los liberales se ahorcan a sí
mismos.

---Es evidente---murmuró D. Rafael Maroto.

---¡Demonches de pedernal!

---¡Luz, luz!---volvió a decir D. Felicísimo.---Pero Sagrario\ldots{}
Nada, lo que digo: todos dormidos.

Por fin prendió la mecha y aplicada a ella la pajuela de azufre, ardió
rechinando como un condenado cuyas carnes se fríen en las ollas de Pedro
Botero. A la luz sulfúrea de la pajuela reaparecieron las cuatro caras,
bañadas de un tinte lívido, y la estancia parecía más grande, más fría,
más blanca, más sepulcral\ldots{}

---De modo---continuaba Elías, cuando D. Felicísimo encendía el candilón
de cuatro mecheros,---que en vez de apartarles de ese camino, debemos
instarles a que por él sigan.

---Sí, que limpien, que despojen\ldots{}

---Pues ahora---dijo Negri,---contaré yo la conversación que tuve con Su
Alteza la infanta doña Francisca.

---Y yo---añadió Carnicero,---referiré lo que me dijo ayer fray Cirilo
de Alameda y Brea.

\hypertarget{xx}{%
\chapter{XX}\label{xx}}

Jenara no pudo dormir en el abominable camastrón que le destinara doña
María del Sagrario, el cual estaba en un cuarto más grande que bonito,
todo blanco, todo frío, todo triste, con alto ventanillo por donde
venían mayidos y algazara de gatos. Al amanecer pudo aletargarse un
poco, y en su desvariado sueño creía ver a D. Felicísimo hecho un
demonio, ora volando, montado en su pluma, ora descuartizando gente con
la misma pluma, en cuchillo convertida. La casa se le representaba como
un lisiado que suelta sus muletas para arrojarse al suelo, y allí eran
el crujir de tabiques, el desplome de paredes, la pulverización de
techos, y las nubes de polvo, en medio del cual, como ave rapante,
revoloteaba D.~Felicísimo llorando con lúgubre graznido, mientras los
demás habitantes de la casa se asfixiaban sepultados entre cascote y
astillas.

Al despertar sin haber hallado reposo, sus ojos enrojecidos reconocieron
la estancia, que más tenía de prisión que de albergue, y acometida de
una viva aflicción lloró mucho. Después las reflexiones, los planes
habilísimos que había concebido y más que nada la valentía natural de su
espíritu la fueron serenando. Vistiose y acicalose como pudo, echando
muy de menos los primores de su tocador, y pudo presentarse a Micaelita
y a Doña Sagrario con semblante risueño.

En sus planes entraba el de amoldar su conducta y sus opiniones a las
opiniones y conducta de los dueños de la casa, y así cuando visitó al
Sr.~D. Felicísimo en su despacho y hablaron los dos, era tan apostólica
que el mismo Infante la habría juzgado digna de una cartera en su
ministerio futuro. Según ella, la perseguían por apostólica, y su
\emph{apostoliquismo} (fue su palabra) era de tal naturaleza que la
llevaría valientemente a la lucha y al martirio. Carnicero, que en su
marrullería no carecía de inocencia (virtud hasta cierto punto
apostólica), creyó cuanto la dama le dijo, y establecida entre ambos la
confianza, el anciano le contaba diariamente mil cosas de gran sustancia
y meollo, referentes a la causa. Sirvan de ejemplo las siguientes
confidencias.

«¡Bomba, señora! Direle a usted lo más importante que he sabido anoche.
Una monjita de las Agustinas Recoletas de la Encarnación soñó no hace
mucho que el Infante se ceñía la corona asistido de no sé cuántas
legiones de ángeles. Escribió su sueño en una esquelita que remitió a Su
Alteza, el cual la besó y tuvo con esto un grandísimo gozo. Me lo ha
contado Orejón».

«¡Bomba, señora! La trapisonda de Andalucía ha terminado. Los marinos
que se sublevaron en San Fernando están ya fusilados y el bribón de
Manzanares que desembarcó con unos cuantos tunantes ha perecido también.

¡Si no hay sahumerio como la pólvora para limpiar un reino! Que
desembarquen más si quieren. El Gobierno se ha preparado, arma al brazo.
Ahora, vengan pillos».

«¡Gran bomba, señora! Mañana ahorcan a Miyar, el librero de la calle del
Príncipe, por escribir cartas democráticas. Pronto le harán compañía
Olózaga, Bringas y Ángel Iznardi».

Generalmente estas noticias eran dadas al anochecer o durante la cena,
en presencia de Tablas. Después se rezaba el rosario, con asistencia de
todos los de la casa, y de Jenara que desempeñaba su parte con
extraordinario recogimiento y edificación.

Ya se habrá comprendido que la muy pícara se valió de los ahogos
pecuniarios del bueno de Perico Tablas para sobornarle y ponerle de su
parte. El demandadero de la cárcel de Villa, que no era ciertamente un
Catón, se rindió a la voluntad dispendiosa de Jenara sirviéndole como se
sirve a una dama que reúne en sí afabilidad, hermosura y dinero.

Dos días habían pasado desde la prisión de Olózaga, cuando se vio a
Tablas y a Pepe Olózaga hermano menor de Salustiano, bebiendo
\emph{medios chicos} de vino en la taberna de la calle Mayor, esquina a
la de Milaneses. Jenara no sólo supo explotar en provecho propio los
buenos servicios de Tablas, sino que los utilizó en pro de Salustiano
por quien mucho se interesaba.

Este insigne joven, que después había de alcanzar fama tan grande como
orador y hábil político, fue primero encerrado en lo que llamaban
\emph{El Infierno}, lugar tenebroso, pero más horrendo aún por sus
habitantes que por sus tinieblas, pues estaba ocupado por bandidos y
rateros, la peor y más desvergonzada canalla del mundo. No creyéndole
seguro en \emph{El Infierno,} el alcaide le trasladó a un calabozo, y de
allí a una de las altas bohardillas de la torre. Antes de que mediara
Tablas pudo Pepe Olózaga ponerse en comunicación con su hermano,
valiéndose de una fiambrera de doble fondo y del palo del molinillo de
la chocolatera.

El ingenio, la serenidad, la travesura de Salustiano eran tales, que en
pocos días se hizo querer y admirar de los presos que le rodeaban y que
allí entraron por raterías y otros desafueros. Los demás presos no se
comunicaban con él. Pepe Olózaga, después de ganar a Tablas, a quien
hizo creer que su hermano estaba encarcelado por \emph{cosas de
mujeres}, intentó ganar también a uno de los carceleros; pero no pudo
conseguirlo. Más afortunado fue Salustiano, que seduciendo dentro de la
prisión a sus guardianes con aquella sutilísima labia y trastienda que
tenía, pudo comunicarse con Bringas. Ambos sabían que si no se fugaban
serían irremisiblemente ahorcados. Discurrieron los medios de alagar los
procedimientos para ver si ganando tiempo adelantaba el negocio de su
salvación, y al cabo convinieron en que Bringas se fingiría mudo y
Olózaga loco.

Tan bien desempeñó este su papel, que por poco le cuesta la vida.
Principió por fingirse borracho; propinose después una pulmonía
acostándose desnudo sobre los ladrillos, y los carceleros le hallaron
por la mañana tieso y helado como un cadáver. Tras esto venía tan bien
la farsa de su locura, que siete médicos realistas le declararon sin
juicio. Así ganó un mes.

Miyar, que no era travieso, ni abogado, ni hombre resuelto, pereció en
la horca el 11 de Abril.

Mejor le fue a Olózaga con su locura que a Bringas con su mutismo,
porque impacientes los jueces con aquel tenaz silencio, que les impedía
despachar pronto, imaginaron darle un tormento ingenioso, el cual
consistía en clavarle en las uñas astillas o estacas de caña. Nada
consiguieron con esto; pero Bringas perdió la salud y no salió de la
cárcel sino para morirse. Es un mártir oscuro, del cual se ha hablado
poco, y que merece tanta veneración como lástima.

Pepe Olózaga y los amigos de Salustiano trabajaban sin reposo. Las
comunicaciones con el preso eran frecuentes, y no sólo recibió este
ganzúas y dinero, que son dos clases de llaves falsas, sino también el
correspondiente puñal y un poquillo de veneno para el momento
desesperado. Antes el suicidio que la horca.

Jenara, que salía de noche furtivamente de la casa de Don Felicísimo,
iba a donde se le antojaba sin que nadie la molestase, y así pudo ayudar
a la familia de Olózaga. Hízose muy amiga de la mujer del escribano
señor Raya, y también de la mujer del alcaide. A la sangre fría del
preso primeramente, a la constancia y diplomacia de su hermano Pepe, al
oro de la familia, y por último, a la compasión y buen ingenio de
algunas mujeres, debiose la atrevidísima y dramática evasión, que
referiremos más adelante en breves palabras, aunque referida está del
modo más elocuente por quien debía y sabía hacerlo mucho mejor que
nadie.

Jenara, preciso es declararlo, no tenía puestos sus ojos en la cárcel de
Villa por el solo interés de Salustiano y su apreciabilísima familia.
Allí, en la siniestra torre que modernamente han pintado de rojo para
darle cierto aire risueño, estaba un preso menos joven que Olózaga, de
gentil presencia y muchísima farándula, el cual pasaba por preso
político entre los rateros y por un ladronzuelo entre los políticos.
Era, según Tablas, hombre de grandes fingimientos y transmutaciones, al
parecer instruido y cortés. Figuraba en los registros con dos o tres
nombres, sin que se hubiera podido averiguar cuál era el suyo verdadero.
Tablas reveló a la señora que no era ella sola quien se interesaba por
aquel hombre, sino que otras muchas de la Corte le agasajaban y
atendían.

Las señas que el demandadero indicaba de la persona del preso convencían
a Jenara de que era quien ella creía, y más aún las respuestas que a sus
preguntas daba este. No obstante la dama no pudo lograr ver su letra por
más que a entablar correspondencia le instó por conducto del mandadero.
El preso pidió algunas onzas y se las mandaron con mil amores. Se
trabajó con jueces y escribanos para que le soltaran, estudiose la causa
y ¿cuál sería la sorpresa, el despecho y la vergüenza de Jenara al
descubrir que el preso misterioso no era otro que el celebérrimo
Candelas, el hombre de las múltiples personalidades y de los infinitos
nombres y disfraces, figura eminente del reinado de Fernando VII, y que
compartió con José María los laureles de la caballería ladronera, siendo
el héroe legendario de las ciudades como aquel lo fue de los campos?

Corrida y enojada la señora descargó su colera sobre Pipaón, a quien
puso cual no digan dueñas, y no le faltaba motivo para ello, porque el
astuto cortesano de 1815 la había engañado, aunque no a sabiendas,
diciéndole que el que buscaba estuvo primero en casa de Olózaga y
después preso en la Villa con los demás conjurados, noticias ambas
enteramente contrarias a la verdad.

A todas estas, Jenara no tenía valor para abandonar la hospitalidad que
le había ofrecido D. Felicísimo y continuaba embaucándole con su
entusiasmo apostólico, sabedora de que la mayor tontería que podía
hacerse en tan benditos tiempos era enemistarse con la gente de aquel
odioso partido.

~

Al anochecer de cierto día de Mayo, Jenara vio salir al padre Alelí del
cuarto de D. Felicísimo, y poco después de la casa. Hacía días que no
tenía noticias de Sola ni del estado de su peligrosa y larga enfermedad,
y así, luego que el fraile se marchó, fue derecha a la madriguera de D.
Felicísimo para saber de la protegida del Sr.~Cordero.

---¡Grande, estupenda bomba, señora!---dijo el anciano a quien
acompañaba, rosario en mano, el atlético Tablas.

---¿Se sabe algo de esa joven?\ldots{}

---Ya pasó a mejor, o peor vida, que eso Dios lo sabrá---repuso
Carnicero volviendo hacia Jenara su cara plana que iluminada de soslayo
parecía una luna en cuarto menguante.

---¡Ha muerto!---exclamó la dama con aflicción grande.

---Ya le han dado su merecido. Conozco que es algo atroz, pero no están
los tiempos para blanduras. Hazme la barba y hacerte he el copete.

---Yo pregunto por la pupila de nuestro amigo Cordero.

---Acabáramos; yo me refiero a esa joven que han a ahorcado en Granada.
¿Cómo la llamaban, Tablillas?

---Mariana Pineda.

---Eso es. Bordadme banderitas para los liberales desembarcadores. El
cabello se pone de punta al ver las iniquidades que se cometen. ¡Bordar
una bandera, servir de estafeta a los liberales!, y ¡sabe Dios las demás
picardías que los señores jueces habrán querido dejar ocultas por
miramientos al sexo femenino\ldots!

---¡Y esa señora ha sido ahorcada!---exclamó Jenara, lívida a causa de
la indignación y el susto.

---¿Que si ha sido\ldots? Y lo sería otra vez si resucitara. O hay
justicia o no hay justicia. Como el Gobierno afloje un poco, la
revolución lo arrastra todo, monarquía, religión, clases,
propiedad\ldots{} Esta doña Mariana Pineda debe de ser nieta de un D.
Cosme Pineda que vino aquí por los años de 98 a gestionar conmigo cierto
negocio de las capellanías de Guadix\ldots{} buena persona, sí, buena.
Era poseedor de una de las mejores ganaderías de Andalucía, la única que
podía competir con la de los Religiosos Dominicos de Jerez de la
Frontera, donde se crían los mejores toros del mundo.

---Y esa doña Mariana---dijo Jenara,---era, según he oído, joven,
hermosa, discreta\ldots{} ¡Bendito sea Dios que entre tantas maravillas
de hermosura, ha criado, Él sabrá por qué, tantos monstruos terribles,
los leones, las serpientes, los osos y los señores de las Comisiones
Militares\ldots!

---¿Chafalditas tenemos\ldots?---dijo don Felicísimo echando de su boca
un como triquitraque de hipos, sonrisillas y exclamaciones que no
llegaban a ser juramentos.---Mire usted que se puede decir: «al que a mí
me trasquiló, las tijeras, ji, ji, le quedaron en la mano».

La dama le miró, reconcentrada en el corazón la ira; mas no tanto que
faltase en sus ojos un destello de aquel odio intenso que tantos
estragos hacía cuando pasaba de la voluntad a los hechos. En aquel
momento Jenara hubiera dado algunos días de su vida por poder llegarse a
D. Felicísimo y retorcerle el pescuezo, como retuerce el ladrón la fruta
para arrancarla de la rama; pero excusado es decir que no sólo no puso
por obra este atrevido pensamiento homicida, sino que se guardó muy bien
de manifestarlo.

---Yo no soy tampoco de piedra---añadió Carnicero echando un
suspiro;---yo me duelo de que se ahorque a una mujer; pero ella se lo ha
guisado y ella se lo ha comido, porque ¿es o no cierto que bordó la
bandera? Cierto es. Pues la ley es ley, y el decreto de Octubre ha
proclamado el tente-tieso. Con que adóbenme esos liberales. Dicen que
fueron tigres los señores jueces de Granada. Calumnia, enredo. Yo sé de
buena tinta\ldots{} vea usted: aquí tengo la carta del Sr.~Santaella,
racionero medio y tiple de la catedral de Granada\ldots{} hombre veraz y
muy apersonado, que por no gustar del clima de Andalucía, quiere una
plaza de tiple en la Real capilla de Madrid\ldots{} pues me dice, vea
usted, me dice que cuando la delincuente subió al patíbulo, los
voluntarios realistas que formaban el cuadro se echaron a llorar\ldots{}
Un Padre nuestro, Tablas, recémosle un Padre nuestro a esa pobre señora.

Igual congoja que los voluntarios realistas sintió Jenara al oír el rezo
de Carnicero y Tablas; pero dominándose con su voluntad poderosa, varió
de conversación diciendo:

---¿Se sabe de la pupila de Cordero?

---Esa\ldots---replicó D. Felicísimo con desdén,---está fuera de
peligro. Hierba ruin no muere.

\hypertarget{xxi}{%
\chapter{XXI}\label{xxi}}

---Sí, ya está fuera de peligro, gracias al Señor y a su Santísima y
única madre, la Virgen del Sagrario. Decir lo que he padecido durante
esta larga y complicada dolencia de la apreciable Hormiga, durante estos
cuarenta y tantos días de vicisitudes, mejorías, inesperados recargos y
amenazas de muerte, fuera imposible. El corazón se me partía dentro del
pecho al ver cómo caía y se deslizaba hasta el borde del sepulcro
aquella criatura ejemplar dotada por el Cielo de tantas riquezas de
espíritu y que parece puesta adrede en el mundo para que sirva de espejo
a los que necesitamos mirarnos en un alma grande para poder engrandecer
un poquito la nuestra. Y más me angustiaba el ver cómo se moría sin
quejarse, aceptando los dolores como si fueran deberes; que su costumbre
es llevar sobre sí las pesadumbres de la vida, como llevamos todos
nuestra ropa.

»Ya está fuera de peligro, y gracias a Dios ya sigue bien. Me parece
mentira que es así, y a cada instante tiemblo, figurándome que su cara
no recobra tan prontamente como yo quisiera, los colores de la salud. Si
la oigo toser, tiemblo, si la veo triste tiemblo también. Pero D. Pedro
Castelló, que es el primer Esculapio de España, me asegura que ya no
debo temer nada. Es fabuloso lo que he gastado en médicos y botica; pero
hubiera dado hasta el último maravedí de mi fortuna por obtener una
probabilidad sola de vida. Mi conciencia está tranquila. Ni sueño ni
descanso ha habido para mí en este período terrible. He olvidado mi
tienda, mis negocios, mi persona y al fin con la ayuda de Dios he dado
un bofetón a la pícara y fea muerte. ¡Viva la Virgen del Sagrario, D.
Pedro Castelló y también Rousseau que dice aquello tan sabio y profundo:
\emph{«no conviene que el hombre esté solo»!}

Así hablaba D. Benigno Cordero en la tienda con un amigo suyo muy
estimado, el marqués de Falfán. Y era verdad lo que decía de sus
congojas y del gran peligro en que había puesto a Sola una traidora
pleuresía aguda. La naturaleza con ayuda de la ciencia y de cuidados
exquisitos triunfó al cabo; pero después recayó la enferma, hallándose
en peligro igual si no superior al primero. Cuanto humanamente puede
hacerse para disputar una víctima a la muerte, lo hizo D.~Benigno, ya
rodeándose de los facultativos más reputados ya procurando que las
medicinas fueran escogidas aunque costaran doble, y principalmente
asistiendo a la enferma con un cuidado minucioso, y con puntualidad tan
refinada que casi rayaba en la extravagancia. Digamos en honor suyo que
había hecho lo mismo por su difunta esposa.

Aunque parezca extraño, Doña Crucita manifestó en aquella ocasión
lastimosa una bondad de sentimientos y una ternura franca y solícita de
que antes no tenían noticia más que los irracionales. Sin dejar de
gruñir por motivos pueriles, atendía a la enferma con el más vivo
interés, velaba y hacía las medicinas caseras con paciencia y esmero.
Bueno es decir para que lo sepa la posteridad, que doña Crucita tenía en
su gabinete el mejor herbolario de todo Madrid.

Cuando D. Pedro Castelló dijo que la enferma no tenía remedio, D.
Benigno manifestó grandeza de ánimo y resignación. No hizo aspavientos
ni habló a lo sentimental. Solamente decía: «Dios lo quiere así, ¿qué
hemos de hacer? Cúmplase la voluntad de Dios». La \emph{Paloma
ladrante}, que tenía en su natural genio el quejarse de todo, no supo
mantenerse en aquellos límites de cristiana prudencia y dijo algunas
picardías inocentes de los santos tutelares de la casa; pero a solas
cuando nadie podía verla, se limpiaba las lágrimas que corrían de sus
ojos. La posteridad se enterará con asombro de las palizas que la buena
señora daba a sus perros para que no hicieran bulla ni salieran del
gabinete en que estaban encerrados.

Los Corderillos mayores compartían la pena de su padre y tía, y los
minúsculos, sin darse cuenta de lo que sentían, estaban taciturnos y con
poco humor para pilladas. Deportados con las cotorras en el gabinete de
su tía, jugaban en silencio, desbaratando una obra de encaje que Crucita
tenía empezada, para rehacerla después ellos a su modo. Cuando Sola
estuvo fuera de peligro y sin fiebre, lo primero que pidió fue ver a los
chicos. Radiante de alegría los llevó D. Benigno al cuarto de la enferma
diciendo: «aquí está la Guardia Real Granadera» y al mismo tiempo se le
aguaron un poco los ojos. Sola les besó uno tras otro y puso sobre su
cama a Juan Jacobo, diciendo:

---¡Cómo ha crecido este!\ldots{} y ¡qué gordo está! Bendito sea Dios
que me ha dejado vivir para que os siga viendo y queriendo a todos.

Cordero se había vuelto de espaldas y hacía como que jugaba con el gato:
después se quitó las gafas para limpiarlas. Lo que realmente hacía era
defender su emoción de las miradas de Sola y los chicos. Aun en aquel
primer día de su convalecencia, pudo Sola hacer a la \emph{Guardia Real
Granadera} un obsequio inusitado. Desde el día anterior había guardado
cuatro piedras de azúcar de pilón, y dio una a cada muchacho, destinando
la mayor a Juanito Jacobo, precisamente por ser el más chico y a la vez
el más goloso.

---Un ángel---les dijo,---que ha venido todas las noches a preguntar por
mí y a ver si se me ofrecía algo, me dio anoche estos terrones para
todos, encargándome que no se los diera si no se habían portado bien. Yo
no sé qué tal se han portado\ldots{}

---Muy mal, muy mal---dijo doña Crucita.---No merecían sino azúcar de
acebuche y miel de fresno.

---Lo pasado pasado---añadió Sola.---Ahora se portarán bien.

Esto no se había acabado de decir cuando ya se oían los fuertes
chasquidos de los dientes de Juanito Jacobo, partiendo el azúcar. Los
cuatro besaron a la que había hecho con ellos las veces de madre y se
retiraron muy contentos. D. Benigno no podía contener cierta expansión
de gozosa generosidad que naciendo en su corazón le llenaba todo entero.
Fue tras los muchachos y dio cuatro cuartos a cada uno para que
compraran chufas, triquitraques, pasteles o lo que quisieran. Después le
pareció poco y a los dos mayores les dio una peseta por barba,
advirtiéndoles que aquel dinero era para \emph{correrla} en celebración
del restablecimiento de Sola, y por tanto no debía ser metido en la
hucha. Cada uno tenía su hucha con sendos capitales.

Crucita se fue a sus quehaceres y D. Benigno se quedó solo con la
\emph{Hormiga}. En los días de gravedad, cuando le acometía fuertemente
la calentura, Sola deliraba mucho. Los individuos conservan en sus
desvaríos febriles casi todas las cualidades que les adornan hallándose
en estado de perfecta salud, y así Sola enferma era diligente, bondadosa
y afable. Agitándose en su lecho con horrible desvarío, mandaba a los
chicos a la escuela, le pasaba la lección a Rafaelito, reñía a Juanito
Jacobo por romper los figurines del \emph{Correo de las Damas}, bromeaba
con Crucita por cuestión de pájaras lluecas o de perros con moquillo,
daba órdenes a la criada sobre la comida, se afligía porque no estaban
planchadas las camisas de D. Benigno, le pedía a este cigarros para el
padre Alelí, preguntaba a los dos qué plato era el más de su gusto para
la próxima cena y hablaba con todos de los Cigarrales y de cierta
expedición que tenían proyectada; era una reproducción o un lúgubre
espejismo de su actividad y de sus pensamientos todos en la vida
ordinaria. Acontecía que después de un largo período de exaltación
febril, Sola se quedaba muda y sosegada otro largo rato sin decir más
que algunas palabras a media voz. D. Benigno que atendía a estos
monólogos con tanto dolor como interés, pudo entender algunas palabras
entre ellas: D. \emph{Jaime Servet}\footnote{Véase \emph{Un voluntario
  realista}.}.

Aquel famoso día de los terrones de azúcar, D.~Benigno, luego que con
ella se quedó solo, le preguntó quién era el tal D. Jaime Servet que en
sueños nombraba, y ella quiso explicárselo punto por punto; pero apenas
había empezado cuando entraron Primitivo y Segundo trayendo un grande,
magnífico y oloroso ramo de rosas que ofrecieron a Sola con cierto
énfasis de galantería caballaresca. Los dos muchachos tuvieron la
excelente idea de emplear las dos pesetas que les dio su padre en
comprar flores para obsequiar con ellas a su segunda madre en el fausto
día de su restablecimiento; y en verdad que era de alabar la delicadeza
exquisita con que procedían los muchachos, probando que en la edad de
las travesuras no escasea cierta inspiración precoz de acciones
generosas y de la más alta cortesía. Decir cuánto agradeció Sola la
fineza, fuera imposible, y si el fuerte olor de las flores no la marease
un poco, habría puesto el ramo sobre la almohada. Les dio besos y luego
pasó el ramo a Cordero para que aspirase la rica fragancia.

D.~Benigno no cabía en sí de satisfacción. Se puso nervioso, se le
resbalaron las gafas nariz abajo, y esta parecía hacerse más picuda,
tomando no sé qué expresión de órgano inteligente. Sonrisa de vanagloria
retozaba en sus labios, y aquel aroma parecíale que llevaba a su alma un
regalado confortamiento, una paz deleitosa, un gozo, una esperanza, una
vida nueva. Los muchachos, al ver el éxito de su hazaña, estaban
soplados de orgullo.

D.~Benigno se los llevó prontamente a su cuarto y les dijo:

---Tomad\ldots{} un duro para cada uno. Sois caballeros finos y
agradecidos. Muy bien; muy bien, señoritos: este rasgo me ha gustado
mucho. En vez de comprar golosinas que os ensucian el estómago\ldots{}
comprasteis el ramo\ldots{} pues\ldots{} Idos a paseo: no vayáis esta
tarde al colegio. Yo lo mando\ldots{} Adiós\ldots{} un duro a cada uno.

Cuando volvió al lado de Sola, Crucita había llevado, para que la
enferma los viera, los pajarillos en cría, pelados y trémulos dentro del
nido, mientras la pájara saltaba inquieta de un palo a otro, y el pájaro
ponía muy mal gesto por aquel desconsiderado trasporte de la jaula. Sola
admiró todo lo que allí había que admirar, la sabiduría y la paciencia
de aquellos menudos animalillos que así pregonaban con su manera de
criar la sabiduría maravillosa y el poder del Criador, el cual en todas
partes donde algo respira ha puesto un bosquejo de la familia humana.

---Lléveselos usted---dijo Sola,---que se asustan y se enojan, y creo
que el enojo lo van a pagar los pequeñuelos, quedándose hoy sin
almorzar.

Después cargó Crucita, no sin trabajo, con algunos tiestos de minutisa y
pensamientos para que Sola viera cómo con el calor de la estación se
cubrían de pintadas florecillas, las unas formando ramilletes o grupos,
como un canastillo de piedras preciosas, otras sueltas con diferentes
tamaños y matices; pero todas guapas y alegres. También trajo un lirio
que parecía un obispo, vestido de largas faldamentas moradas, un moco de
pavo que más bien parecía gallo de cresta roja, y otras muchas hierbas
que llevaban la alegría a la alcoba, pocos días antes tan silenciosa y
tan fúnebre. ¡Con cuánto gusto recibía Sola aquellas visitas! Era la
vida que le enviaba aquellos mensajes para cumplimentarla; era la casa
amada que la saludaba con lo más hermoso y agradable que en sí tenía.
Para que nada faltase, vino también la cotorra, a quien Sola encontró
más crecida, vino el loro que le pareció haber sufrido algún desperfecto
en su casaca verde, y por último entraron también los perros en tropel,
y se lanzaron a la cama aullando y lamiendo. En tanto D. Benigno,
después de estar un rato como en éxtasis, bajó los ojos y apoyó la barba
en su mano trémula. O rezaba o recitaba algún famoso texto de Rousseau:
en esto no parecen acordes las crónicas, y por eso ponemos las dos
versiones para que el lector elija la que más le cuadre.

Pasó un rato. Todo estaba en silencio. El héroe de Boteros saboreaba en
el pensamiento la dicha presente que no era sino anticipado anuncio de
su dicha futura.

---Pues como decía a usted\ldots---indicó Sola.

---Eso es, apreciable \emph{Hormiga}. Siga usted su cuento y dígame
quién es ese D. Jaime Servet.

Sola satisfizo cumplidamente la curiosidad de su amigo.

\hypertarget{xxii}{%
\chapter{XXII}\label{xxii}}

Habiendo ordenado los médicos que la enferma fuera a convalecer en el
campo, D. Benigno empezó a preparar el viaje a los Cigarrales de Toledo
donde él poseía extensas tierras y una casa de labranza. Extraordinario
gusto tenía el héroe en estos preparativos por ser muy aficionado a la
dulce vida del campo, al cultivo de frutales, a la caza y a la crianza
de aves y frutos domésticos. Por su desgracia él no podía abandonar su
comercio en aquella estación, y érale forzoso seguir en la tienda por lo
menos hasta que pasase el Corpus, fiesta de gran despacho de encajes
para Iglesia y \emph{modistería}. Pero resignándose a su esclavitud en
la Corte se deleitaba pensando en el dichoso verano que iba a pasar.
Amaba la Naturaleza por afición innata y por asimilación de lo que había
leído en su autor favorito y maestro. Así nada le parecía tan de perlas
como aquella frase: \emph{el campo enseña a amar a la humanidad y a
servirla.}

Su plan era llevar a Sola a últimos de Mayo acompañada de Crucita y los
niños menores. Inmediatamente regresaría él solo a Madrid y cuando
acabase Junio, volvería con los otros dos chicos a los Cigarrales donde
estarían todos hasta fin de Septiembre.

¡Los Cigarrales! ¡Cuánta poesía, cuántas amenidades, qué de inocentes
gustos y de puros amores despertaba esta palabra sola en el alma del
buen Cordero! ¡Qué meriendas de albaricoques, qué gratos paseos por
entre almendros y olivos, qué mañanitas frescas para salir con el perro
y la escopeta a levantar algún conejo entre las olorosas matas de
tomillo, romero y mejorana! ¡Qué limpieza y frescura la de las aguas,
qué color tan hermoso el de las cerezas, y qué dulzura y maravilla en
los panales fabricados por el pasmoso arte de las abejas en el tronco
hueco de añosos alcornoques o entre peñas y jaras! En los cercanos
montes el gruñido del jabalí hace temblar de ansiedad el corazón del
audaz montero, y abajo, junto a la margen del río aurífero, del río
profeta que ha visto levantarse y caer tan diferentes imperios, la peña
seca y el remanso profundo solicitan al pescador de caña flor y espejo
de la paciencia. Pensando en estos cuadros poéticos, y gozando ya con la
fantasía estos legítimos placeres, D. Benigno se sonreía solo, se
frotaba las manos y decía para sí.

---Barástolis, ¡qué bueno es Dios!

¡Y luego!\ldots{} esta reticencia le regocijaba más que aquellas
risueñas perspectivas bucólicas. Había decidido no hablar a Sola de
cierto asunto hasta que ambos estuvieran en los Cigarrales y ella
completamente restablecida.

Cordero fue una mañana a la Cava Baja en busca de arrieros y trajinantes
para arreglar con ellos su viaje. Entró en la posada de la Villa, y en
la que antiguamente se llamaba del Dragón. En esta y en uno de los
aposentos más altos encontró a un mayoral que ha tiempo conocía, y
después de concertar ambos las condiciones del viaje, siguieron en
calorosa conversación sobre el mismo asunto, porque se había despertado
en D. Benigno cierto entusiasmo pueril por la dichosa expedición. Allí
preguntó varias veces Cordero la distancia que hay desde Madrid a
Toledo, hizo comentarios sobre tal cuesta, sobre cuál mal paso, y
finalmente disertó largo rato sobre si llovería o no al día siguiente,
que era el señalado para la salida. Cordero opinaba resueltamente que no
llovería. Ya se marchaba, cuando al pasar por el corredor alto donde
había varias puertecillas numeradas vio a un hombre que tocaba en una de
estas. El hombre preguntó en voz alta:

---¿D. Jaime Servet vive aquí?

Detúvose Cordero y oyó una voz que de dentro gritaba:

---No ha llegado todavía.

El héroe no dio a lo que había oído más importancia de la que merece una
simple coincidencia de nombres.

¡Qué afán puso el buen señor en preparar su viaje, en disponer lo
referente a vestidos, provisiones y todo lo demás que se había de
llevar! Creeríase que iban a dar la vuelta al mundo, según la prolijidad
con que Cordero se proveía de todo y las infinitas precauciones que
tomaba, y las advertencias que hacía, y el itinerario escrupuloso que
trazaba, y la elección de vituallas, y el acopio de drogas por si
ocurrían descalabraduras o molimiento de huesos. Todo le parecía poco
para que a Sola no faltara ninguna comodidad, ni se privase de nada que
pudiera convenir a su espíritu y su salud. Y deseando anticipar las
delicias del viaje, aquella noche le habló de la distancia, le describió
los pueblos que habían de recorrer, pintole paisajes de ríos y montañas,
diciendo estas o parecidas cosas:---Cuando pasemos de Torrejón de la
Calzada, a Casarrubielos fíjese en aquellas lomas de viñas, que están en
fila y hacen unos bailes tan graciosos cuando pasa el coche
corriendo\ldots{} Después en tierra de la Sagra verás unos panoramas que
encantan\ldots{} Luego que se pasa de Olías te quedarás pasmada cuando
veas allá lejos la torre de la catedral que parece saluda al
viajero\ldots{} sin quitarse el sombrero, se entiende, el cual es un
capacete que está emparentando con el cielo y que trata de tú a los
rayos\ldots{}

En fin, llegó la mañana y se marcharon despedidos por Alelí que se quedó
muy triste. Cuando el coche, dejando atrás el puente de Toledo, entró en
la extensa, libre y alegre campiña inundada de luz, D. Benigno sintió
que la alegría se rebosaba del vaso de su espíritu, chorreando fuera
como las caídas de una fuente de Aranjuez, y aquel chorrear de la
alegría era en él risas, frases, exclamaciones, chascarrillos y por
último la elocuente frase: ---Barástolis, ¡qué bueno es Dios!

Aquel mismo día corrió por Madrid la noticia de haberse escapado de la
cárcel de Villa el preso que ya estaba destinado a la horca. Jenara se
alegró tanto cuando Pipaón se lo dijo que al instante salió a la calle
para felicitar a D.~Celestino. Hacía ya dos semanas que había empezado a
perder el miedo, y salía de noche a pie acompañada de Micaelita,
vestidas ambas en traje tan humilde que difícilmente podían ser
conocidas.

Después de dar la enhorabuena a D. Celestino y a su hija regresó a casa
de Carnicero y se entretuvo escribiendo algunas cartas. Pipaón la visitó
en su cuarto, donde hablaron un poco de política. Jenara fue luego a ver
cenar a D. Felicísimo, operación que le hacía gracia por las
singularidades y extravagancias de aquel santo hombre en tan solemne
instante, y le halló sumamente ocupado con un alón que por ninguna parte
quería dejarse comer, según estaba de cartilaginoso y duro.

---Bomba, señora\ldots---dijo Carnicero picoteando el hueso por aquí y
por allá de modo que unas veces se lo ponía por bigote y otras se lo
tascaba como un freno.---En Portugal el señor D. Miguel está apretando
las clavijas a aquel insubordinado reino. Ahora dicen que vendrán del
Brasil D. Pedro y doña María de la Gloria a disputar la corona a D.
Miguel\ldots{} Quisiera yo ver eso\ldots{} Sigue, querido Tablas, lo que
me estabas contando, que esta señora no puede ser insensible a las
glorias del toreo, y si es verdad, como dices, que ese muchacho
rondeño\ldots{}

Tablas aseguró que el muchacho rondeño que acababa de llegar a Madrid y
se llamaba Montes, por sobrenombre Paquiro, era un enviado de Dios para
restablecer la decaída y casi muerta orden de la tauromaquia. Dijo
también que cuando Madrid le conociera bien sería puesto por encima de
todos sus predecesores en aquel arte, incluso Pepe-Hillo y Romero, pues
tenía todas las cualidades de los antiguos y aun algunas más, siendo
autor de varias suertes y reglas, y de un toreo nuevo\ldots{}

---Por lo que deberá llamarse---dijo D. Felicísimo riendo como un
bobo,---el Moratín de la muleta.

Algo más se habló de este tema, aventurando en él Jenara algunas
observaciones; mas como esta dijera que se verificaría una
\emph{Revolución} en el toreo, se enfadó Carnicero al oír la palabra y
dijo que no habría revoluciones en nada y que bien estaba el mundo como
estaba, aunque estuviera sin toros. Jenara dio su asentimiento y
mientras el anciano tomaba sus últimos bocados, se entretuvo en observar
la habitación, pues nunca se cansaba de mirarla ni de reconocer la
extraordinaria concordancia que había entre ella y su habitador, de tal
manera que así como el capullo es molde del gusano, así parecía que D.
Felicísimo había hilado su despacho envolviéndose en él. Detrás del
sillón de la mesa había un largo estante del tamaño de la pared, cuyas
puertas tenían en vez de vidrios rejillas de alambres y por los huecos
de estas asomaban sus caras amarillentas los legajos, como enfermos que
se asoman a las rejas de un hospital. Muchos tenían cruzados de cintas
rojas y cartoncillos colgantes con rótulos. Algunos estaban tendidos
horizontalmente, semejando no ya enfermos sino verdaderos cadáveres que
no volverían a la vida aunque les royeran ratones mil; otros estaban
inclinados sobre sus compañeros, como borrachos o mal heridos, y los
menos aparecían completamente derechos y erguidos. Estos eran los que se
asían a las rejillas y aun echaban fuera sus cintas rojas cual si
meditaran una evasión arriesgada. En el más alto andamio de la sepulcral
estantería Jenara vio una colección de objetos que semejaban tinajas
negras, alternando con otros que si no eran avechuchos disecados, lo
parecían. Eran los sombreros que había usado D.~Felicísimo en su larga
vida, y que en aquel retiro estaban gozando de una pingüe jubilación de
polvo y telarañas, ilusionados aún con remozarse y pasar a cubrir las
cabezas de otra generación menos ingrata que la pasada.

Todo lo que decimos iba pasando por la fantasía de Jenara, y después
esta se fijó en la mesa, donde aquella noche había, no ya un montón,
sino una cordillera de legajos por cuya recortada cima aparecía de vez
en cuando la cara de D. Felicísimo, iluminada de lleno por la lámpara,
como luna que platea las cumbres de los montes. En aquella altura que
podría ser Calvario estaba el Cristo de la espalda en llaga y del cuello
en soga, y era de ver cómo volvía su rostro ensangrentado hacia la
pezuña de macho cabrío, pidiéndole misericordia, y cómo no hacía maldito
caso la pezuña, sólo ocupada en oprimir duramente, cual si quisiera
patearla, una carta en cuyo sobrescrito se leía:

\emph{Al Sr.~D. Jaime Servet.---Posada del Dragón}.

\hypertarget{xxiii}{%
\chapter{XXIII}\label{xxiii}}

Jenara no vio tal carta. Llamáronla a cenar y cenó. Después doña María
del Sagrario, siguiendo su tradicional costumbre, que por lo infalible
debía haberse puesto en el Almanaque, se quedó dormida en un sillón,
mientras Micaelita y Bragas, que acababa de entrar, se secreteaban de lo
lindo en el comedor. La dama huésped esperó a que Tablas y la criada
cenasen también para ir con aquel al rincón de los muebles viejos donde
solían hablar de cosas reservadas. Llegó la ocasión y Tablas, que
obedecía servilmente a la señora y era como un esclavo, por la cuenta
que le tenía, contestó a las apremiantes preguntas de esta manera:

---Fue a las dos en punto. El señorito don José, el Sr.~D. Celestino y
yo habíamos convenido en que las dos era la mejor hora. Yo di al
carcelero las onzas que me dio el Sr.~D. Celestino y el carcelero pidió
más, y le llevé más, y luego dijo que no era bastante y se le dieron
otras pocas onzas. Al preso le llevé las mangas con galones de teniente
coronel, y la gorra de cuartel, que eran el trapo para engañar a
cualquier carcelero de sentido. Ya se le había llevado puñal y pistola y
un cinto de onzas, que son la mejor brega para parar los pies a la
justicia y hacerla que obedezca al engaño. El carcelero y yo habíamos
convenido en correr el cerrojo sin echarle el gancho, y D. Salustiano
tenía ya una cuerda para descorrerle desde dentro. Para que no hiciera
ruido untamos de aceite al cerrojo. El preso salió: yo no sé cómo se las
compuso para que no ladraran los dos grandes perros que se quedan todas
las noches en el pasillo. Debió de echarles pan o hacerles maleficio,
porque aquellos animales no se empapan en el engaño. Ello es que bajó y
por la escalera se le apagó la luz y tuvo que volver a subir para
encender otra. Yo le sentía desde abajo y no me atrevía a ayudarle ni a
decir esta boca es mía, por miedo a que los carceleros se escurrieran
fuera percatándose del engaño. Todos habían recibido sus pases de dinero
para que se atontaran; pero yo no tenía confianza y estaba con el alma
en un hilo, esperando a ver qué tal se portaba la cuadrilla. Por fin,
señora, apareció el preso en la sala de guardia de la cárcel donde
estábamos varios, algunos vendidos y otros que no se habían dejado
comprar, echándoselas de bravos y boyantes. Yo les había convidado a
beber y estaban un poco fuera de la jurisdicción del tino. Al ver al
preso se quedaron pasmados. Venía con la capa terciada, enseñando la
manga derecha y los galones de oro. En aquella mano traía un puñal, y en
la otra la muleta o sea un puñado de onzas. ¡Qué momento! D. Salustiano
arrojó al suelo las onzas y amenazó con la herramienta gritando:
\emph{¡onzas y muertes reparto!}\ldots{} Allá voy.

Había sonado la campanilla, y Tablas, interrumpiendo su relación, corrió
a abrir. Aquella noche venía más gente que de ordinario a la misteriosa
tertulia de D. Felicísimo, y así la campanilla no sabía estar callada ni
un cuarto de hora.

---Pues decía---añadió Tablas,---que al ver las onzas por el suelo y el
puñal en el aire, se quedaron todos parados, ciñéndose en el engaño sin
saber si atender al oro o al hierro, al trapo o al estoque. Pero la
mayor parte se fueron al capote y anduvieron un rato a cuatro pies.
Otros quisieron cortar el terreno. Ya el preso tenía la llave en la
cerradura para abrir la puerta\ldots{} Esta llave se había hecho días
antes por moldes de cera que yo saqué\ldots{}

La campanilla volvió a sonar. Jenara hizo un gesto de impaciencia.
Cuando después de abrir volvió Tablas y dijo a la señora con mucho
misterio:

---Ahí está.---¿Quién?

---El de ahí enfrente.

---¿Pero quién es el de ahí enfrente?

---El culebrón con pintas\ldots{} Viene muy embozado en su capa y le
acompaña un cura.

---¿Pero quién?

---El que se casó con la jorobada, el degollador de España, Calomarde,
señora.

---Bien, siga usted.

---Puso la llave en la cerradura; pero en esto el bribón de Poela, que
es el que había tomado más varas, quiero decir más onzas, se fue a él
con muchos pies y le tiró a matar con un puñal. Felizmente no le hirió
porque el preso llevaba sobre el pecho la tapa de un misal. Pero con el
encontronazo se le cayó la llave de la cerradura y de la mano. Yo hice
un cuarteo, apagué la luz, recogí la llave, se la di, abrió él a fondo,
sin vacilar. En un mete y saca quedó hecho todo, y digo mete y saca
porque D. Salustiano, después de abrir, tuvo alma para sacar la llave,
salir y cerrar por fuera. Lo que pasó en la calle no lo sé, pero según
entiendo ya está ese caballero en corral seguro. En la cárcel hubo luego
porrazos, caídas, puños y varas. Yo saqué un rasguño en esta mano.
Vinieron dos alcaldes de Casa y Corte y estuvieron tomando
declaraciones\ldots{} a mí con esas. ¡Buen trasteo les dimos! Yo, aunque
me citaban sus mercedes sobre corto y sobre largo y a la derecha y a la
izquierda, no quise embestir a la palabra y me callé como un cabestro.

Apenas concluyó el atleta oyose allá en el fondo del pasillo una voz que
decía: ¡Luz, luz!

Era que aquella noche como en otra ya mencionada la lámpara que
alumbraba el congresillo furibundo resolvió apagarse y de nada valieron
contra esta determinación autocrática las exclamaciones y protestas de
D. Felicísimo. Es fama que la luz comenzó a palidecer precisamente
cuando la tertulia llegaba a su grado más alto de calor político y de
cólera apostólica; por lo que contrariados todos al ver que desaparecían
las caras, clamaban en tonos distintos: ¡luz, luz!

Allá corrió Tablas, y sacando la lámpara les dejó completamente a
oscuras, mas no callados. Salía de la sala un murmullo impaciente, del
cual Jenara no pudo entender cosa alguna. Cuando volvió Tablas llevando
en alto la lámpara encendida, como el coloso antiguo alumbrando el
puerto de Rodas, la dama pudo ver por la entornada puerta las sombras
que se movían en aquel antro blanquecino. Conoció a algunos y haciéndose
cruces se apartó de allí y dijo:

---¡También D. Juan Bautista Erro!

---Y el señor obispo de León---murmuró Tablas.---Es el que mete más
ruido y el que, cuando yo entré decía: «Para nada hace falta la luz».

---Tiene razón. Para nada les hace falta. Y si no que se lo pregunten a
los topos.

Después que supo cuanto podía saber de la evasión de Olózaga, intentó
pescar algunas frases de las que en la sala se decían. Acercose y puso
atención; pero el espesor de las antiguas puertas no permitía que se
oyeran palabras. Aburrida dio algunos paseos por el corredor blanco en
el cual los puntales interrumpían a cada instante la marcha, y los
ladrillos del piso tecleaban bajo los pies. Sobre el yeso veíanse las
correderas que de noche salían de las infinitas grietas de la casa para
hacer sus excursiones, y el gato corría cazando y trepaba por las vigas
y desaparecía por ignorados agujeros para reaparecer en la habitación
más lejana, o bien se estiraba perezoso en el rincón de los muebles
viejos, donde sus ojos brillaban como dos gotas de oro encendido. Cuando
alguien andaba por los pasillos con paso muy vivo, sentíase un
estremecimiento temeroso en la casa toda y los puntales parecían
temblar, como los músculos del atleta que hace un esfuerzo grande, y
caían algunas cascarillas de yeso de las paredes y el techo. La casa
tenía, pues, sus palpitaciones súbitas y sus corazonadas nerviosas.

Jenara se retiró a su cuarto y apagó la luz fingiendo que se acostaba.
Cuando los apostólicos se fueron, y se fue Pipaón y se encerró en su
dormitorio D. Felicísimo, la dama salió envuelta en manto negro y
andando tan quedamente que sus pasos no se sentían más que los del gato.
Vio a Tablas, le habló en secreto indicándole que deseaba salir sin que
nadie lo supiera en la casa; vaciló un momento el gigante; pero su
venalidad fue también llave de aquella evasión, no tan cara como la de
Olózaga. ¿A dónde iba la aventurera? ¿A su casa, que continuaba puesta y
servida, como si ella estuviera de viaje, o a otra parte misteriosa y no
sabida de ser alguno vendido ni por vender? Lo ignoramos. Este es un
punto en el cual todas nuestras pesquisas y diligencias han valido poco,
y al tratarlo sin conocimiento nos ocurre decir como los apostólicos:
«¡Luz luz!»

Al día siguiente muy temprano, cuando don Felicísimo y su hermana se
levantaron, Jenara estaba en casa; pero salió muy tarde de su habitación
porque había pasado, según indicó, muy mala noche. Cuando fue a saludar
a Carnicero, este le dijo:

---¡Qué mala noticia tenemos hoy! Ese bribón de Olózaga que se escapó de
la cárcel de villa no parece. Se ha revuelto todo Madrid\ldots{} ¡Ah!,
es que no se habrá revuelto bien. Si la policía supiera cumplir con su
deber\ldots{} Por cierto, señora mía, que anoche uno de los amigos que
me honran viniendo a mi tertulia me habló de usted\ldots{} Por de
contado, señora, ni las moscas saben que está usted en mi casa.

---¿Y no se puede saber por qué motivo me tomó en boca ese amigo de
usted?

---Ese amigo---dijo Carnicero,---sostiene que usted debe saber dónde se
oculta Olózaga.

---¿Yo? Su amigo de usted es tonto rematado. ¡Qué sandeces se permiten
algunas personas!

Y no dijo más porque, habiéndose acercado a la mesa de D. Felicísimo,
tenía los cinco sentidos puestos en el sobre de la carta que bajo la
pezuña estaba.

---Tablas, Tablas---gritó a la sazón el anciano.---Pero hombre, ¿que
nunca has de estar aquí cuando haces falta\ldots? Toma, ve, corre, lleva
esta carta a la posada del Dragón.

Y levantó la pezuña de macho cabrío para tomar la carta, que
violentamente oprimida por aquel pesado objeto parecía hallarse a punto
de reventar echando fuera todas sus letras.

---Pues sí, señora mía---prosiguió D. Felicísimo luego que marchó Tablas
con el recado.---Eso me decía mi amigo, y me lo repitió tres
veces\ldots{} «Ella debe saberlo, ella debe saberlo y ella debe
saberlo\ldots» Y que le apearan de esto.

---Su amigo de usted---replicó Jenara,---será un gran farsante y un
perverso calumniador, porque esto envuelve una calumnia, Sr.~Carnicero.

~

Y era verdad que la dama aventurera no sabía dónde se ocultaba el que
después fue insigne tribuno y jefe de un partido. Siendo ella una de las
personas que más ayudaron en el oscuro complot de la evasión, no fue
partícipe del secreto del escondite, el cual, por excesivamente delicado
y peligroso, no salió de la familia. Hoy se sabe que Salustiano al salir
de la cárcel, cerrando por fuera la puerta, tropezó con un nuevo
obstáculo, el centinela. Estaba concertado que un amigo, fingiéndose
asistente del supuesto teniente coronel, entretendría al centinela
contándole cuentos. Pero este amigo había faltado y el centinela se
paseaba solo a la claridad de la luna, que aquella noche brillaba de un
modo tan poético como importuno. Un \emph{buenas noches, centinela},
pronunciado con serenidad asombrosa, salvó a Salustiano de este nuevo
peligro. Avanzó tranquilamente, y en la esquina de la calle de Luzón se
le unió un amigo que le aguardaba. Por las calles menos concurridas se
apartaron a buen paso de la cárcel, dirigiéndose a la vivienda destinada
a servir de refugio al fugitivo, la cual era una sombrerería de la
Puerta del Sol. Llegaron al centro de Madrid, y vieron que en el
Principal se agolpaba la gente. Ya se tenía allí noticia de la
escapatoria. Olózaga tuvo que dar un rodeo de un cuarto de legua para
dirigirse a la sombrerería, entrando en la Puerta del Sol por la carrera
de San Jerónimo, y al fin se vio seguro en el asilo que se le había
preparado. Baráibar se llamaba el sombrerero, patriota generoso, que
guardó el secreto con fidelidad admirable y supo arrancar al absolutismo
una de sus víctimas. Escondido en el sótano de la tienda estuvo
Salustiano muchos días, mientras se preparaba el no menos difícil ardid
de ausentarle de España. Había trocado una prisión por otra; pero en
esta última la esperanza, la idea de libertad y de triunfo le
acompañaban en las solitarias horas. Por las noches, contra la opinión
de su amigo Baráibar, que temblaba con las temeridades de Olózaga, este
se disfrazaba hábilmente y se salía del sótano de la casa, no
precisamente para pasearse por Madrid, sino para correr a misteriosas
citas, en que no tenía participación la política. Como estas atrevidas
expediciones nocturnas son de un carácter reservado, debe interponerse
entre ellas y la luz de la historia la pantalla de la discreción; y así,
doblando esta página, sólo escribiremos en ella: «Oscuridad, oscuridad».

\hypertarget{xxiv}{%
\chapter{XXIV}\label{xxiv}}

«¡Barástolis, mayoral, que ya estamos en casa; pare usted, pare usted!»
Esto decía D. Benigno, y al punto el desclavijado vehículo se detuvo en
lo más fragoso de un caminejo lleno de guijarros y junto a una tapia
carcomida. Bajaron todos molidos y aporreados, y D. Benigno enderezó la
caminata hacia la casa, que distaba como dos tiros de fusil del lugar
donde había parado el coche. Cada uno de los chicos iba abrazado con su
hucha, y entre todos conducían mal que bien los cinco perros de Crucita.
Esta no había querido confiar a nadie sus dos gatos, y por el camino no
había cesado de echar maldiciones contra el mayoral, el camino y el
coche, que era una verdadera fábrica de chichones.

El panorama de la finca se presentó de un golpe a la contemplación de
los viajeros. D. Benigno no cabía en sí de gozo, y a cada paso decía a
Sola:

---Vea usted cómo están esos almendros\ldots{} ¿Quién diría que esos
olivos no tienen más que diez años?\ldots{} Aquellos otros, que aún son
estacas, los planté yo por mi mano hace tres años\ldots{} Mire usted a
la derecha; pues aquello es lo del tío Rezaquedito, tierras que vendrán
a ser mías el año que viene.

La casa era de labor, medianamente arreglada para vivienda cómoda. Tenía
una huertecilla, a la que daba frescura y sustancia el agua clara de una
noria. Más allá había un prado muy lucido, en el cual pastaban algunos
carneros, y las gallinas en bandadas, que regía un arrogante y enfatuado
gallo, recorrían libremente todo, olivar, viñas y prado, respetando la
huerta, donde les prohibía la entrada, con muy mal gesto, una cerca de
zarza erizada de púas.

El sitio no era prodigio de hermosura pero sí muy agradable y tenía los
inapreciables encantos de la soledad, del silencio campesino y del
verdor perenne aunque un poco triste de los olivos. Los horizontes eran
anchos, la luz mucha, el aire puro y sano. Todo convidaba allí a la vida
sosegada y a desencadenar de tristezas y preocupaciones el espíritu,
dejándole libre y a sus anchas.

Interiormente la casa valía poco; pero Sola, en cuanto la vio, hizo
mentalmente la reforma y compostura de toda ella, prometiéndose ponerla,
si la dejaban, en un grado tal de limpieza, comodidad y arreglo que
podrían allí vivir canónigos y aun obispos. Todo lo observaba ella, y si
al principio no decía nada, cuando Cordero le preguntó su opinión, no
pudo menos de darla, diciendo:---¡Qué bien vendría aquí un
tabique\ldots!, y abrir allá una puerta\ldots{} y extender este corredor
poniéndole escalera exterior para bajar a la huerta\ldots{} y en la
huerta yo plantaría una fila de árboles que dieran sombra a la casa por
esta parte\ldots{} y quitaría el gallinero de donde está para ponerlo
allá en el fondo del corral donde están las mulas\ldots{} Hay que cuidar
mejor de la huerta y componer esa noria que sin duda es del tiempo de
los moros.

Todo esto lo oía extasiado D. Benigno, prometiéndose formalmente hacer
las reformas indicadas por Sola y aun algunas más.

Desgraciadamente para él, no podía estar en los Cigarrales sino un par
de días, porque le precisaba volver a Madrid, pero ¡qué feliz sería
cuando volviese definitivamente a sus queridas tierras para pasar todo
el verano! Sí, sí, sí: era ya cosa decidida en el espíritu del bueno del
comerciante liquidar cuentas, traspasar la tienda, renunciar al comercio
y hacerse labrador para el resto de sus días. Estos dulces pensamientos
le hacían sonreír a solas.

La historia cuenta que D. Benigno regresó a Madrid sin que le ocurriera
nada de particular en su viaje, dejando buenos y sanos, y además muy
contentos, a los que en los Cigarrales se quedaron. También dice que
vendió muchos encajes en la temporada del Corpus, y que allá por los
últimos días de Junio el héroe hizo entrega de la tienda a un amigo de
toda su confianza, y se dispuso a partir para Toledo con sus dos hijos,
Primitivo y Segundo, que ya estaban de vacaciones, con buenas notas y
las correspondientes huchas llenas de dinero. Para colmo de dicha, el
padre Alelí, a quien los médicos de la Orden habían prescrito sosiego y
campo, se disponía a acompañarle a los Cigarrales. ¿Qué faltaba? Sólo
faltaba para poner la veleta al edificio de la felicidad Corderil que se
resolviera un asunto delicado, un asunto del alma, un problema de
corazón, del cual pendían todos los demás problemas, cuestiones y
proyectos del héroe de Boteros. Una de las dificultades más graves, que
era la de la enunciación o planteamiento verbal del problema, estaba ya
vencida, porque D. Benigno halló un medio excelente de vencer, o mejor
dicho, de esquivar su timidez, y fue escribir a Sola una larga carta
cuando ella se hallaba en los Cigarrales y él en Madrid.

La carta era tan fina, tan discreta y comedida, que no vacilamos en
reproducir algunos párrafos de ella. Decían así:

«Esto que siento no es una pasión de mozalbete, que sería impropia de mi
edad, es un afecto que empezó siendo compasión y poco a poco se fue
volviendo un tanto egoísta; luego se robusteció mucho con admiraciones
de las virtudes de usted, y más tarde se hizo fuerte con la
consideración de asociar a mi vida una vida tan útil por todos conceptos
y que me traería tan gran dote de riquezas morales y de méritos
positivos.

»Aquí, apreciabilísima \emph{Hormiga}, viene por sus pasos contados la
cuestión del agradecimiento. Usted dirá que lo tiene por mí, y yo
replico que mayor debe ser el mío porque los favores que me ha hecho son
de los que no se pagan con nada del mundo. Usted ha criado a mis hijos,
usted ha ordenado mi casa, usted ha hecho agradable, fácil y metódica la
vida. Y quien tanto ha hecho, quien tanto merece, ¿no ha de tener una
posición digna en el mundo? Sí, y mil veces sí. Huérfana y sola, pobre y
sin más tesoro que sus virtudes, su amor al trabajo, su tierna solicitud
por todas las criaturas débiles o enfermas, usted ha cautivado mi
corazón, no con afecto ardiente de esos que más bien hacen desgraciados
que felices a los hombres, sino despertando en mí un sentimiento puro,
en el cual se enlazan el amor y el respeto, la consideración y la
ternura, el deseo vivísimo de ser feliz y el más vivo aún de hacer
feliz, rica, considerada y señora a quien ya tiene en su alma todas las
señorías de Dios.

»No me conteste usted por escrito. Medite usted mi proposición, y cuando
yo vaya, que será dentro de ocho o diez días, me responderá verbalmente
con una sola palabra, en la inteligencia, apreciable \emph{Hormiga}, de
que si mi proposición mereciera una negativa, sería usted para mí lo
mismo que ahora es, la primera y más santa de las amigas, y siempre
sería yo para usted el mismo leal, admirador y ferviente amigo.

\flushright{\textit{Benigno Cordero.»}}
\justify

Muy satisfecho y descansado se encontró el hombre después de escrita la
carta. Leída y aprobada por el padre Alelí, D. Benigno la entregó por su
propia mano al ordinario de Toledo. Aquel día vendió muchos encajes.
Dios estaba de su parte.

\hypertarget{xxv}{%
\chapter{XXV}\label{xxv}}

Por fin vino el último día de Junio, y el héroe, con sus dos hijos y el
padre Alelí, se embanastó en el coche, y helos aquí en camino de los
Cigarrales. Durante el viaje el fraile hablaba por siete, siendo tan
extremado aquel día el desorden caótico de su cabeza que no hablara
mejor ni con más gracia el mismo descubridor de los \emph{cerros de
Úbeda}, o el fabricante de los \emph{pies de banco}. A cada instante
suspendía sus paliques para quedarse mirando al cielo, con el dedo en el
labio y el entrecejo lleno de pliegues y laberínticas arrugas, imagen
exacta de la confusión que dentro reinaba. Las únicas palabras que
entonces profería eran estas:---Benignillo, yo tenía que decirte una
cosa\ldots{} ¿Qué es lo que yo tenía que decirte, Benignillo?\ldots{}
Pues no me acuerdo.

El de Boteros, aunque anheloso y lleno de dudas, tenía presentimientos
felices, y el corazón le aseguraba que sería venturoso el término o
solución de sus amorosas ansiedades. Llegaron. Sola, doña Crucita y los
chicos menores con regular escolta de perrillos y perrazos salieron a
recibirles al camino. Por un rato no se oyó más que el estallido de los
besos con que se saludaban los hermanos. No poca parte del besuqueo fue
para la correa y las flacas manos de Alelí, el cual, sintiendo un gozo
superior a lo que las palabras podían expresar, echaba bendiciones a
derecha e izquierda, como sembrador que desparrama a puñados el trigo
sobre un fértil terreno. D. Benigno se encontró bastante cohibido en
presencia de Sola; y así sus frases fueron balbucientes, truncadas y
sosas. Ella estaba en su natural buen humor, alegre por la llegada de
los viajeros, y un poco más decidora que de costumbre. Crucita no
parecía la misma y andaba por el campo hecha una zagaleja, vestida con
un \emph{desavillé} extravagante y cómodo, que no era ciertamente tomado
de los figurines de la Arcadia ni del Zurguén.

Era una naturaleza constituida moralmente para la vida del campo, por su
amor a las flores y a los animales, su espíritu de independencia y su
actividad. Así cuando vio trocadas las arboledas de sus balcones por
aquel espacioso tiesto en que había olivares, viñedos, albaricoques,
establos, huerta, cerros y horizonte, enloqueció de contento y todo el
día andaba por aquellos campos con un pañuelo liado a la cabeza y un
garrote en la mano, echando de comer a las gallinas, vigilando los
carneros, expulsando a los guarros de los sitios donde no debían estar,
o bien cogiendo fruta, regando lechugas, arreglando una espaldera de
cañas para que se enredaran trepando las tiernas y vacilantes judías.
Los chicos que ya llevaban un mes en aquella vida, estaban negros como
cuervos de tanto andar por el campo, jugando a todas horas con tierra,
palitroques y guijarros. Parecían dos pintiparados paletos, y en sus
caras, color de pucheros de Alcorcón, brillaban los ojos de azabache
despidiendo centellas de picardías.

Antes de que llegara la noche, D. Benigno recorrió la casa, hallando en
ella y en la distribución de sus escasos muebles tanta novedad y arreglo
que su corazón bailó de contento. Ya se conocía bien qué manos divinas
habían andado por allí y qué instinto sublime había hecho de un caserón
un hogar y del desmantelado hueco un delicioso nido.

---¡Qué admirable, qué encantadora manera de responder a mi
proposición!---dijo Cordero para sí.---Me contesta con hechos, no con
palabras. Estas paredes y estos muebles me responden por ella
diciéndome: «Nos ha arreglado la señora de la casa».

En la huerta halló Cordero nuevos motivos de admiración. No parecía la
misma que él había dejado al regresar a Madrid. Todos los cuadros
estaban sembrados de hortaliza; las gallinas expulsadas de allí tenían
mejor acomodo en un local admirablemente elegido y dispuesto. La cerca
limpia y podada reverdecía y echaba verdadera espuma de tiernos
renuevos, como si en sus venas hirviera la savia; las callejuelas y
paseos admirablemente enarenados parecían recibir con agradecimiento la
blanda pisada del amo, cuando por aquellos frescos contornos se paseaba.
La noria estaba ya compuesta y no se desperdiciaba el agua, ni quedaba
ningún cangilón roto. Toda la máquina funcionaba dando vueltas
majestuosamente y sin chirridos, semejando una vida serena, arreglada y
prudente que iba sacando del hondo depósito del tiempo futuro los días
para vaciarlos serenamente en el manso río del pasado. A Don Benigno se
le antojaba que los árboles habían crecido mucho y era la verdad que si
no habían crecido mucho, estaban verdes y lozanos y por haber sido
limpiados de todo el ramaje viejo y seco. Extendían los morales su
fresquísimo follaje como diciendo: «hemos echado estas hojas tan grandes
y tan verdes para coronar a la señora de la casa».

---Parece mentira---dijo D. Benigno sintiendo su garganta oprimida por
un dogal de satisfacción, pues también hay dogales de gozo;---parece
mentira, apreciable Sola, que haya hecho usted tantas maravillas con el
poco dinero que le dejé. La casa está trasformada y la huerta también.
De este tugurio y de este rincón de tierra ha hecho usted con su mano de
oro un palacio y un edén.

Sola se ruborizó un poco y dijo que era preciso echar abajo dos tabiques
y plantar una nueva fila de árboles, y traer algunos muebles.

¿Muebles? ¡Ah! D. Benigno habría traído, si en su mano estuviera, el
trono de las Españas para sentar en él a la que de este modo inundaba su
alma y su vida de esperanza y alegría. Al hablar de las reformas de la
finca, Sola hablaba ingenuamente el lenguaje de la señora de la casa. Y
en esto no había afectación de ninguna clase, ni menos desenfado de
advenediza, sino que se expresaba así porque todo aquello le parecía
suyo y muy suyo de hecho, aunque no mediasen las circunstancias que se
lo iban a dar de derecho.

~

Cenaron. La cena fue alegre y opulenta. Abundante caza, sabrosos
salmorejos, perdices escabechadas, estofado de vaca que propagó por toda
la casa su exquisito olor de refectorio, legumbres fritas en menestra,
festoneada con ruedecillas de huevos duros, vino fresco de Esquivias, y
luego un bandejón de albaricoques de la finca, frescos, ruborizados, y
echando pura miel por aquella boquirrita con que se pegaban al árbol,
compusieron la colación. En la mesa se encontraron cosas de los
Cigarrales y cosas de Madrid. Llevaba en esto la palabra el fraile que
en tocando a hablar se parecía a la noria tal como estaba antes, echando
agua sin concierto ni orden. Más de una vez se quedó parado y lelo,
diciendo:---«Benignillo, yo tenía que contarte una cosilla\ldots» «¡Ah!,
ya caigo»---añadía dando un grito. Y después decía:---«Pues no: se me
fue. Me anda dando vueltas por el magín y no la puedo atrapar».

Con estas cosas se acabó la cena y el fraile rezó el rosario, contestado
por Benigno y Sola, porque Crucita y los cuatro muchachos se quedaron
dormidos teniendo entre los dientes el último hueso de albaricoque y el
primer Padrenuestro.

\emph{---Ite, mensa est}. A acostarse todo el mundo---gritó al concluir
Alelí.---Estamos muertos de cansancio.

Y se acostaron todos. D. Benigno durmió con plácido sosiego y soñó que
estaba su cabeza circundada de una aureola, de un disco de luz como el
que tienen los santos. Por la mañana cuando se levantó y salió de su
alcoba, persistía en él la ilusión de tener en su cabeza el nimbo y de
estar despidiendo de sus sienes chorros de luz. Tomó su chocolate,
encendió un cigarrillo, entró en la sala baja y vio a Sola que estaba
abriendo las maderas para que entrara el aire puro del campo, y al mismo
tiempo para atar la cuerda donde se había de colgar la ropa que se
estaba lavando. El otro extremo de la cuerda debía atarse en el moral
grande que había en medio de la huerta. Don Benigno tomó la soga y salió
muy contento a ayudar a su protegida en aquella trajín doméstico.

---Más fuerte---le dijo Sola riendo.

Si Cordero se atara la soga en el mismo cogollo de su corazón, no
sintiera este más alborotado y palpitante.

---Más flojo---dijo Sola.

---¿Así?

---No tanto. Si se tira mucho se rompe, y si se afloja mucho, el viento
se lleva la ropa. Ahora está bien.

D.~Benigno volvió a la sala. Una gran cesta de ropa blanca aguardaba a
la robusta moza que había de llevarla a la huerta. La moza salió, Sola
se quedó allí mirando a fuera. D. Benigno se acercó a ella. Ambos
hablaron un rato, diciéndose todo lo más quince palabras que nadie pudo
oír, ni aun el narrador mismo que todo lo oye. La moza y dos criados más
entraron. D. Benigno salió con la aureola de su cabeza tan crecida que
le parecía ir derramando una claridad celestial por donde quiera que
iba. Pasó a la huerta donde topó de manos a boca con un maestro de obras
que había mandado venir de Toledo para encargarle las reformas de la
casa.

D.~Benigno no le conocía, pero le dio un abrazo. Estaba muy nervioso;
pero su discreción y buen juicio pugnaban por sobreponerse a aquella
exaltación, y al fin pudo lograrlo.

---Maestro---dijo,---es preciso emprender las obras inmediatamente. Hay
que derribar dos tabiques y construir una galería exterior sobre la
huerta\ldots{} En fin, la señora le dirá a usted; póngase usted a las
órdenes de la señora. ¡Ah!\ldots{} lo principal es arreglar la pieza que
va a ser gabinete de la señora, ¿me entiende usted?, gabinete de la
señora. ¿Cuánto se tardará en las obras? Hay que concluirlas pronto;
pero muy pronto. Tienen ustedes una calma\ldots{}

---Señor\ldots{}

---Sí, mucha calma. Empiece usted pronto. ¿Ha traído las herramientas?

---Si no sabía\ldots{}

---¡Qué cachaza! Quiero que la casa sea una tacita de plata. La señora
dirigirá las obras. Pensamos vivir aquí constantemente. ¿Qué hace usted
que no toma medidas? ¡Qué cachaza! ¡Barástolis, barástolis!

El maestro se excusó de no haber empezado las obras que aún no estaban
formalmente encargadas, y D. Benigno, que en los momentos de mayor
exaltación era hombre razonable, comprendió la justicia de las excusas y
le dio otro abrazo. Juntos recorrieron la casa. Uniose a ellos Sola y
durante un rato no se habló más que de pies castellanos, de una puerta
por aquí, de cuatro vigas por allá, de las paredes que debían
empapelarse y de las que debían ser pintadas, del nuevo corredor para ir
a la cocina, del cielo raso y de otras menudencias. Sola explanaba sus
proyectos y deseos con una claridad admirable, demostrando en todo la
elevación de su genio doméstico.

Cuando el maestro se retiró, Cordero y Sola hablaron larguísimo rato.
Separáronse al fin, porque ella no podía abandonar ciertas ocupaciones
de la casa, y cuando entró Sola en el cuarto donde estaban planchando se
secó los ojos, que pestañeaban como si quisieran lloriquear un poco.
Después cantó entre dientes, apartando la ropa que iba a repasar.

D.~Benigno salió a la huerta y de la huerta al campo, porque necesitaba
dar un paseo largo que sirviera de expansión a su alma. Iba por en medio
de los olivos cuando oyó la voz de Alelí que decía:

---Benigno, ¿dónde estás? La espesura de los árboles no permitía que se
vieran.

---¿Dónde está usted, padre Monumento?

---Hijo, aquí estoy. Este enemigo malo, esta buena pieza de Jacobito me
ha traído a estos andurriales para que viera un nido y aquí estoy en una
zanja de donde no puedo salir.

Acercose Cordero a donde la voz sonaba y vio a su venerable amigo en lo
más bajo de una hondonada que el terreno hacía. Jacobito se había subido
a los hombros del fraile, montando a horcajadas sobre su cuello, y desde
aquella eminencia alargaba la mano con un palo queriendo alcanzar el
nido.

---Mírame aquí sirviendo de caballería al bergante de tu hijo\ldots{}
Lobezno, si coges el nido o lo rompes te tiro al suelo. No espolees,
verdugo, que me rompes una clavícula. Benigno, por Dios, quítame este
jinete y ayúdame a salir del hoyo.

---Abajo, abajo, atrevido, insolente chiquillo---dijo D. Benigno
riendo.---¿Pues qué, nuestro amigo es campanario?

Desmontose el muchacho y Alelí, libre de tan molesto peso y ayudado de
Cordero, salió del atolladero en que estaba. Arreglándose el hábito,
tomó de la mano a su amigo y le dijo así:

---Ya me acuerdo de lo que tenía que decirte. Vaya con mi memoria que
está dando vueltas como una veleta y tan pronto apunta al Norte como al
Sur. ¿Sabes lo que tenía que decirte? Pues era que se susurra que Su
Majestad napolitana está otra vez en cinta. Como salga varón ¡quién verá
la cara que ponen mis señores los apostólicos!

---Eso me lo ha dicho usted catorce veces durante el viaje, tío
Engarza-Credos.

---Dale bola, es verdad---repitió Alelí pegando en el suelo.---Pues no
era eso. Era que\ldots{} ¿qué era?

Después de una larga pausa diose un palmetazo en la frente y agarrando a
D.~Benigno por la solapa tiró de él y le dijo:

---Ya lo pesqué\ldots{} ya di con mi idea\ldots{} ¡Cómo se escapan las
ideas! Oye tú, \emph{D. Sábelo Todo}. ¿Quién es \emph{Monsieure Servet}?

D.~Benigno miró al cielo.

---No sé---dijo,---ni me importa.

Después estuvo un momento confuso, porque aquel nombre sonaba en sus
oídos de un modo extraño.

---Pues el día de nuestra salida, cuando tú estabas fuera de casa
arreglando las cosas del viaje y yo en tu tienda charlando con el
mancebo, llegó un caballero preguntando por ti. Preguntó por todos los
de la casa y dijo que no podía esperar porque tenía prisa. Se fue
soltándonos su nombre que era D.~\emph{Yo no sé cuántos} Servet, y como
por el modo de vestir y la arrogancia y el habla y el sonsonete del
apellido me pareció francés, lo llamo \emph{monsieure}.

Alelí pronunciaba esta palabra, así como todas las palabras francesas,
lo mismo que se escribe.

---¿Y no dejó recado?

---Que ya volvería. Pero la del humo. Y el mancebo y yo opinamos que es
un extranjero de los que vienen a enredar y hacer diabluras y
revoluciones.

D.~Benigno meditó un momento. Después desechó las ideas que le
asaltaban, diciendo:

---No sé quién es, ni me importa. Ese apellido lo han llevado otras
personas que ya no existen; con que padre Monumento, basta de sandeces y
vamos de paseo. Jacobito, ven. Corre por delante: no te alejes de
nosotros\ldots{} Reverendísimo fraile, todo va bien, muy bien.

---Gracias a Dios\ldots{} ¿Y para cuándo?

---Lo más pronto posible. Hoy mismo se pedirán los papeles.
Barástolis\ldots{}

---Sí, echa, echa de ese cuerpo dos docenas de \emph{barástolis}, y yo
te acompañaré echando cuatro\ldots{} Ya era tiempo, ya era tiempo.

\hypertarget{xxvi}{%
\chapter{XXVI}\label{xxvi}}

Deseoso de que su dicha fuera realidad dentro del más breve plazo, D.
Benigno arregló sus papeles y pidió los de Sola que estaban en un pueblo
del reino de León. Entretanto que venían aquellos malhadados documentos,
sin los cuales no es posible encender cristianamente la antorcha de
Himeneo, los futuros cónyuges vivían en intimidad honesta y dulce, en
una especie de luna de miel de la amistad, en pleno reinado de la paz
doméstica, cuyos encantos se multiplicaban con la deliciosa existencia
campesina. Los días pasaban empujándose suavemente unos a otros y cada
uno de ellos tenía sobre sus propias alegrías la esperanza de las
alegrías del siguiente. Nunca faltaba una operación de labranza, un
paseo al monte, una merienda en las praderas del río, y nunca como en
aquellas gratas ocasiones se le venían a la memoria al buen Cordero los
pensamientos del filósofo de la libertad y la naturaleza. Tan pronto
recitaba aquel pasaje en que Rousseau encomia las dulzuras de la amistad
como aquel otro en que hace el panegírico de las \emph{comidas rústicas
preparadas por el ejercicio}, \emph{sazonadas por el apetito}, \emph{la
libertad y la alegría}. El anatema de los convites urbanos no es menos
enérgico que la apología de las meriendas sobre la hierba.

Emprendiéronse las reformas de la casa con gran actividad. Cordero
encargó a Madrid los regalos con que pensaba expresar a Sola la pureza
de su afecto y la enormidad de su admiración. También ella hacía sus
preparativos, aunque en pequeña escala, pues quería que los nuevos
dominios que iba a poseer se rigieran por la ley de sus dominios
antiguos que era la modestia.

Sólo una contrariedad agriaba el ánimo de Cordero, poniéndole de mal
humor a ratos. Era que los papeles de Sola no venían. Era que en los
libros parroquiales de la Bañeza había no sabemos qué embrollo o
confusión, y quizás algo de ineptitud o mala fe en la persona
comisionada para arreglar el asunto. Llegó el mes de Agosto y los
dichosos papeles no parecían. A mediados de dicho mes, el cansancio de
Cordero no podía ser mayor, y recordando que tenía en Madrid un amigo
que era el mejor agente de negocios eclesiásticos de toda España, le
escribió una larga carta encomendándole la reclamación y pronto despacho
de aquel asunto, que era la clave de su dicha. En el sobrescrito puso:
«Sr.~D. Felicísimo Carnicero, calle del Duque de Alba en Madrid».

¿Y qué?, ¿perderemos esta ocasión de trasladarnos otra vez a la Villa y
Corte sin pagar costas de viaje? No mil veces; que estas ocasiones no se
presentan todos los días. Callandito nos deslizamos dentro de la carta,
y henos aquí en poder del ordinario de Toledo que puntualmente la
llevará a su destino, y con ella a nosotros.

Muy bien se va dentro de una carta. Además de que no hay mejor aposento
que un pedazo de papel doblado, tenemos la ventaja de conocer los
secretos que nuestras compañeras de viaje, las señoras letras, llevan
consigo. Una oblea es llave de nuestra breve cárcel y un dedo vacilante
rompiendo la frágil pared nos devuelve la libertad.

Ya estamos.

Abierto el papel, salimos un poco estropeados y entumecidos a causa de
la postura violenta que es indispensable en los viajes epistolares, y de
pronto nos hallamos frente a frente de una tabla que se esforzaba en ser
semblante humano. Era D. Felicísimo, que en aquel momento en que le
vimos decía:

---Permítame usted que lea esta carta.

Tenía visita. Miramos, y en efecto, frente a la mesa estaba un caballero
de muy buena presencia, el cual si no tenía cuarenta años andaba muy
cerca de ellos. Vestía bien. Su rostro era moreno, su frente alta y
hermosa, su complexión robusta, sin dejar de ser delicada, su modo de
mirar triste, sus ojos negros y ardientes a la vez como las noches de
verano.

Carnicero leyó la carta, y dijo entre dientes: «bueno».

Después la puso bajo el pie de cabrón y prosiguió lo que con aquel buen
señor hablaba cuando llegamos.

---Decía que el negocio de usted es de los más delicados que he visto.
Parte de la fortuna de su tío de usted el señor canónigo de la Sonora,
ha debido pasar al Monte Pío beneficial de la diócesis de Pamplona. Lo
que está en la escribanía de la Puebla de Arganzón puede ser recogido
por usted si tiene valimiento y trabaja mucho. ¿Por qué no se presentó
usted a recoger su herencia cuando tuvo noticia del depósito? Ya me ha
dicho usted que en aquellos días estaba emigrado y perseguido por las
leyes. Pero eso no es una razón. Hoy también lo está usted y si se le
deja en paz y aun se le permite abandonar la farsa del nombre supuesto
es porque ha traído recomendaciones de altos personajes legitimistas.
Yo\ldots{} puesto en lugar de usted me decidiría a perder la mitad de la
herencia del señor canónigo de la Sonora con tal de sacar libre la otra
mitad, y confiaría mi pleito a un agente hábil y astuto que supiera
mover los trastos y sacar adelante el negocio con toda prontitud.

---Ya lo he pensado---dijo el caballero,---y no tengo inconveniente en
ceder la mitad de la herencia a la persona que arregle esta cuestión
sacando del Monte Pío Beneficial de Pamplona lo que indebidamente ha
sido llevado a él. ¿Quiere usted que hagamos el convenio ahora mismo?

D.~Felicísimo pareció dudar. Su cara de fósil sufrió trasformaciones
ligerísimas en color y contextura cual si estuviera sometida en un
laboratorio a fuertes influencias químicas. Variaron sus mejillas del
gris cretáceo al rojo de cinabrio, su frente se llenó de arrugas como un
terreno que se cuartea a causa de un recalentamiento interior, y sus
ojos cambiaron un momento la trasparencia imperfecta del talco por el
brillo del feldespato.

---La mitad, la mitad y punto concluido---dijo el otro, que sin duda era
más vivo que un azogue y gustaba de las resoluciones prontas.---Hagamos
el contrato hoy mismo y fijemos seis meses para el despacho del negocio.
Si a los seis meses está resuelto, la mitad para mí, la mitad para
usted.

D.~Felicísimo empezó a balbucir excusas y a presentar sus muchos años y
su retraimiento de los negocios como un obstáculo para emprender aquel
que se le proponía. Habló mucho reconociéndose incapaz. Por los dos
ángulos de su boca salía la saliva como una erupción bituminosa que en
aquellas concreciones y repliegues de la barba rapada se dividía en
menudos arroyos. El taimado viejo ponderaba las dificultades del pleito
y su ineptitud, sin duda porque no le parecía bastante la mitad y quería
dos tercios de la herencia.

---La mitad---manifestó resueltamente el otro.---¿Quiere usted, sí o no?

---Por ser usted recomendado del señor don Alejando Aguado, marqués de
las Marismas---replicó el viejo,---acepto y tomo a mi cargo su negocio.

---La mitad\ldots{} seis meses.

---La mitad\ldots{} seis meses---repitió Carnicero, y su vocecilla salió
de la espelunca de su boca, rugiendo como el oso prehistórico.---Hagamos
hoy nuestra escritura.

Tomando el pie de cabrón con su mano de corcho dio un porrazo sobre la
mesa, que hizo temblar hasta en sus cimientos el montón de legajos.
Después rodó la conversación sobre diversos asuntos, y concluyó en
política. Acerca de ella dijo el caballero lo siguiente:

---He perdido todas las ilusiones. He vivido mucho tiempo en España en
medio de las tempestades de los partidos victoriosos, y mucho tiempo
también en el extranjero en medio del despecho de los españoles vencidos
y desterrados. La experiencia me ha hecho ver que son igualmente
estériles los Gobiernos que persiguen defendiéndose y los bandos que
atacan conspirando. Yo he conspirado también algunas veces, y en
aquellos trabajos oscuros he visto en derredor mío pocos móviles
generosos y muchas, muchísimas ambiciones locas, apetitos y rencores que
no se diferenciaban de los del despotismo más que en el nombre. La
realidad me ha ido desencantando poco a poco y llenándome de hastío, del
cual nace este mi aborrecimiento de la política, y el propósito firme de
huir de ella en lo que me quedare de vida.

---Bien, bien---dijo D. Felicísimo agitándose en su asiento y golpeando
sus manos una con otra en señal de júbilo.---Es usted un enemigo más de
esas endiabladas teorías constitucionales y de esas invenciones
satánicas llamadas partidos y del estira y afloja de Cortes que
gobiernan y rey que reina y hurga, por aquí y escarba por allá, y el
demonio que lo entienda\ldots{} De pensar así a ser apostólico
proclamando esta gloriosa monarquía del porvenir no hay más que un paso.
Le veo a usted en el buen camino y en jurisdicción apostólica.

El caballero no pudo reprimir la risa que estas palabras provocaron en
él.

---¡Yo apostólico!---dijo.---No espere tal cosa el Sr.~D. Felicísimo.
Para que eso suceda será preciso que Dios varíe mi natural ser, y
arranque de mí la memoria. Esa forma nueva del despotismo que se anuncia
ahora va a ser más brutal que cuantos despotismos se han conocido,
porque sobre todos sus inconvenientes va a tener el de ser populachero.
No es el absolutismo de Felipe II o de Luis XIV, grande, aristocrático,
batallador, adornado de mil glorias militares y artísticas, y que
disculpa sus atrocidades con grandes empresas y conquistas de mundos; va
a ser un sistema de mojigatería y desconfianza, adicionado con todas las
corruptelas de las camarillas que vienen funcionando desde los tiempos
de Godoy. Se alimentará del suelo por dos grandes raíces, una que estará
en las sacristías, claustros y locutorios de monjas, y otra que se
fijará en las tabernas donde se reúnen los voluntarios realistas. Va a
ser una tiranía ramplona que si es sufrida por nuestro país, lo que dudo
mucho, pondrá a este en un lugar que no envidiará seguramente ninguna
región del África.

Al oír esto D. Felicísimo hizo un gesto tan displicente que su cara se
arrugó toda, y desaparecían los ojos, y los pliegues de sus labios se
extendieron multiplicándose y describiendo un número infinito de rayas
hasta el último confín de las orejas.

---Según eso es usted liberal\ldots{}

---Lo soy, sí, señor; soy liberal en idea, y deploro que el país entero
no lo sea. Si no estuvieran tan arraigadas aquí las rutinas, la
ignorancia, y sobre todo, la docilidad para dejarse gobernar, otro gallo
nos cantara. El absolutismo sería imposible y no habría apostólicos más
que en el Congo o en la Hotentocia. Por desgracia nuestro país no es
liberal ni sabe lo que es la libertad, ni tiene de los nuevos modos de
gobernar más que ideas vagas. Puede asegurarse que la libertad no ha
llegado todavía a él más que como un susurro. Es algo que ha hecho
ligera impresión en sus oídos, pero que no ha penetrado en su
entendimiento ni menos en su conciencia. No se tiene idea de lo que es
el respeto mutuo, ni se comprende que para establecer la libertad
fecunda es preciso que los pueblos se acostumbren a dos esclavitudes, a
la de las leyes y a la del trabajo. A excepción de tres docenas de
personas\ldots{} no pongo sino tres docenas\ldots{} los españoles que
más gritan pidiendo libertad entienden que esta consiste en hacer cada
cual su santo gusto y en burlarse de la autoridad. En una palabra, cada
español, al pedir libertad, reclama la suya, importándole poco la del
prójimo\ldots{}

---Luego usted---dijo D. Felicísimo, que ya había recobrado la fijeza
pétrea de su rostro,---no es liberal al modo de acá.

---Lo soy al modo mío, según mi idea, y creo que estos principios,
aprendidos donde no son sólo principios sino hechos, prevalecerán en
todo el mundo y conquistarán todas las tierras incluso España; pero
cuando me detengo a calcular el tiempo que tardaremos en ser
conquistados, me confundo, me mareo, porque todos los años me parecen
pocos para tan grande obra. De aquí mi escepticismo, que no es realmente
escepticismo, sino tristeza. Creo en la libertad porque he visto sus
frutos en otras partes; pero no creo que esa misma libertad pueda darlos
allí donde hay poquísimos liberales y de estos la mayor parte lo son de
nombre. España tiene hoy la controversia en los labios, una aspiración
vaga en la mente, cierto instinto ciego de mudanza; pero el despotismo
está en su corazón y en sus venas. Es su naturaleza, es su humor, es la
herencia leprosa de los siglos que no se cura sino con medicina de
siglos. He visto hombres que han predicado con elocuencia las ideas
liberales, que con ellas han hecho revoluciones y con ellas han
gobernado. Pues bien, esos han sido en todos sus actos déspotas
insufribles. Aquí es déspota el ministro liberal, déspota el empleado,
el portero y el miliciano nacional; es tiranuelo el periodista, el
muñidor de elecciones, el juntero de pueblo y el que grita por las
calles himnos y bravatas patrióticas. La idea de libertad entrando
súbitamente aquí a principios del siglo nos dio fórmulas, discursos,
modificó algo las inteligencias; pero ¡ay!, los corazones siguen
perteneciendo al absolutismo que los crió. Mientras no se modifiquen los
sentimientos, mientras la envidia que aquí es como una segunda
naturaleza, no ceda su puesto al respeto mutuo, no habrá libertades.
Mientras el amor al trabajo no venza los bajos apetitos y el prurito de
vivir a costa ajena no habrá libertades. No habrá libertades mientras no
concluya lo que se llama sobriedad española que es la holgazanería del
cuerpo y del espíritu alimentada por la rutina; porque las pasiones
sanguinarias, la envidia, la ociosidad, el vivir de limosna, el
esperarlo todo del suelo fértil o de la piedad de los ricos, el anhelo
de someter al prójimo, la ambición de sueldo y de destinos para tener
alguien sobre quien machacar, no son más que las distintas caras que
toma el absolutismo, el cual se manifiesta según las edades, ya servil y
rastrero, ya levantisco y alborotado.

---Según eso---dijo D. Felicísimo que empezaba a estar algo
confuso,---usted considera a nuestro país inepto para las libertades.
Por consiguiente, como no puede haber más que dos clases de gobiernos y
el liberal es imposible, tenemos que aceptar el absoluto.

---No---replicó el otro,---porque una ley ineludible arrastrará, mal de
su agrado, a España por el camino que ha tomado la civilización. La
civilización ha sido en otras épocas conquista, privilegios, conventos,
fueros, obediencia ciega, y España ha marchado con ella en lugar
eminente; hoy la civilización tan constante en la mudanza de sus medios
como en la fijeza de sus fines, es trabajo, industria, investigación,
igualdad, derechos, y no hay más remedio que seguir adelante con ella,
bien a la cabeza, bien a la cola. España se pone las sandalias, toma su
palo y anda: seguramente andará a trompicones, cayendo y levantándose a
cada paso; pero andará. El absolutismo es una imposibilidad, y el
liberalismo es una dificultad. A lo difícil me atengo, rechazando lo
imposible. Hemos de pasar por un siglo de tentativas, ensayos, dolores y
convulsiones terribles.

---¡Un siglo!

---Sí, y esta es la causa de mi tristeza. Yo me encuentro en la mitad de
mi vida. He trabajado mucho por la idea salvadora; pero ya me siento
fatigado y me reconozco sin fuerzas para esta labor inmensa que será
cada día mayor. Otros vendrán que arrimen el hombro a tan terrible
carga. Yo no puedo más. Las circunstancias en que me encuentro, solo,
sin familia, lleno de tedio y viendo cuán poco hemos adelantado en la
cuarta parte de un siglo, me desaniman atrozmente. Reconozco que cuanto
de mis fuerzas dependía ya lo hice; está mi conciencia tranquila y me
retiro. Hasta ahora yo no he vivido para mí ni un solo día. Llega la
hora en que me es necesario vivir un poco para mí. No obteniendo gloria
ni siquiera éxito, el sacrificio de mi existencia a un ideal sería
estéril; pues vivamos, vivamos siquiera un poco y descansemos. Sobre las
ruinas de mis quiméricas ambiciones se levanta hoy una ambición grande,
potente, la ambición de ser feliz, tener una familia y vivir de los
afectos puros, humildes, domésticos. ¡Es tan dulce no ser nada para el
público y serlo todo para los nuestros! Apartado de todo lo que es
política, deseando el olvido, miro a todas partes buscando un rincón en
que ocultarme y a donde no llegue el fragor de la lucha.

D.~Felicísimo movía la cabeza, sonriendo. Creía firmemente que el
caballero, su amigo y cliente, tenía la cabeza vacía de lo que llaman
seso, ¿pues qué mayor locura, en aquellos agitados días, que no ser
apostólico, ni absolutista, ni siquiera liberal?

Ya iba a decir algo muy ingenioso sobre esta enfermiza manía de no ser
nada, absolutamente nada, cuando entró Pipaón y estrechando con ímpetu
amistoso la mano del caballero, le dijo:

---Enhorabuenas mil, queridísimo amigo. Vengo de ver a su Excelencia,
que ya ha leído las cartas que trajiste del Sr.~D. Alejandro Aguado,
marqués de las Marismas, y de su parte te aseguro que puedes vivir aquí
tan libremente como en el mismo París o Londres. El Sr.~Aguado es, como
soberano absoluto del dinero, una potencia de primer orden, una
autoridad indiscutible; ahora bien: considerando que el mencionado
Sr.~Aguado (Pipaón no abandonaba jamás su estilo de expediente)
garantiza bajo su palabra de oro que vienes exclusivamente con la misión
de comprarle cuadros para su rica galería, y además a asuntillos tuyos
que nada tienen que ver con la política, se ha dado cuenta a S. M. de
todo lo actuado y S. M. se ha servido disponer que no se te moleste en
lo más mínimo. Tendreislo entendido, y ahora, discreto amigo, ruégote
que adoptes tu verdadero nombre y vengas a comer conmigo a mi casa,
donde encontrarás personas que más desean verte que escribirte\ldots{}

El caballero se levantó y muy gozoso dijo:

---Confío sin vacilar en la libertad que se me ofrece y recobro mi
nombre.

\hypertarget{xxvii}{%
\chapter{XXVII}\label{xxvii}}

Tenía sus papeles en regla, pasaporte, partida de bautismo, a más de
otros documentos importantes, y aquel mismo día se celebró la escritura
para llevar adelante lo pactado con D. Felicísimo, asistiendo a este
acto solemne, como notario, el licenciado Lobo, a quien conocemos desde
hace veinticuatro años. Por la tarde Pipaón se llevó al amigo a su casa,
donde le obsequió bizarramente con suntuosa comida, cigarros exquisitos
y licores de primera.

Esta esplendidez y el lujo de la vivienda en que estaba admiraron mucho
al convidado, que no podía menos de traer a la memoria la humildad con
que el Sr. Bragas dio los primeros pasos en la carrera de covachuelista.
El medro había sido grandísimo y el aprovechamiento tan colosal, que
allí podrían tomar lecciones cuantas hormigas hay en el mundo.

Los dos camaradas charlaron de lo lindo sobre cosas diversas, pero
especialmente sobre el destino y vicisitudes del amigo que por tanto
tiempo había estado ausente de España y envuelto en misterios. Las
preguntas sucedían a las preguntas y las explicaciones a las
explicaciones, y no fue todo paz y concordia en su interesante diálogo,
porque a lo mejor de él hubo peligro de que los ánimos se soliviantaran
dando al traste con la amistad y buena armonía que son compañeras
inseparables de una serie de buenos platos. Parece ser que el amigo
había enviado a Pipaón, durante los últimos años, todas las cartas que
tenía que dirigir a Madrid. El objeto de esta mediación era que el
diestro cortesano salvara de las asechanzas de la policía en Correos una
correspondencia inocente en que nada se hablaba de política. Así lo hizo
durante algún tiempo; pero desde mediados del 29, don Juan Bragas, que
en las cosas privadas lo mismo que en las públicas había de mostrar la
doblez y bajeza de su carácter, abusó de la confianza del emigrado
dejando de entregar algunas de sus cartas a la persona a quien se
dirigían, para dárselas a otra.

La cuestión de las cartas salió, pues, a relucir en la mesa, y Pipaón
que en frescura y demás dotes para el fingimiento no tenía rival en el
mundo, se desenvolvió gallardamente de aquel compromiso. Su sofistería,
sus protestas de amistad, auxiliadas de su serenidad hacían quiebros
admirables, y no se dejaba él coger en mentira aunque la lógica misma se
encargara de acometerle.

---Puedes estar seguro, amigo Salvador---le decía,---de que desde
Octubre del 29 no he recibido ningún paquete tuyo. Si lo recibiera,
tonto, ¿para qué lo quería yo? ¿De qué podrían valerme tus cartas, no
trayendo nada de política?, y aunque trajeran algo, hombre, aunque fuera
cada letra de ellas una bomba explosiva, ¿me crees capaz de vender a un
amigo de la infancia?, ¿me crees capaz de abusar indignamente de tu
confianza?, ¿me crees capaz de violar el sacratísimo misterio de la
correspondencia\ldots? ¡Oh!, no me des a entender que hay en ti, no digo
sospecha, pero ni siquiera un átomo de sospecha, porque nace en mí
cierta indignación terrible que me hará olvidar la amistad, la
consideración; me desvanezco, me exalto, me sulfuro\ldots{} No, tú no
puedes tener de mí tan baja opinión, tú bromeas, tú has perdido la
memoria de mis buenas partes, y allá en la emigración has olvidado lo
arraigada que está la hidalguía en pechos españoles.

El amigo no se convenció con estas vehementes razones; pero no queriendo
volver sobre lo pasado, dejó aquel tema para tomar otro. Apremiado por
Bragas, contó lo más notable de su vida durante las largas ausencias,
extendiéndose mucho en los dramáticos sucesos de su expedición a
Cataluña, durante la insurrección apostólica de este país. Pasmado lo
oía todo el buen cortesano, y cuando su amigo llegaba a narrar un
peligro extraordinario o el acometimiento de alguna aventura terrible
temblaba y sudaba como si él mismo se sintiera empeñado en aquellos
grandes riesgos y compromisos; tal verdad e interés había en la
relación.

Ya estaban en los postres, cuando Pipaón, oído el relato del convidado
contó a su vez los chascos que él (Pipaón) y otra persona (Jenara) se
habían llevado en Madrid, creyendo ver al buen amigo en cada uno de los
individuos que sucesivamente iba deteniendo la policía por creerlos
emisarios de Mina o Valdés.

---Como no recibíamos cartas tuyas---dijo,---y en tanto los emigrados se
agitaban en París y en Londres, siempre que teníamos noticia de la
llegada misteriosa de algún conspirador, creíamos que eras tú. En Gracia
y Justicia me enteraba yo de los soplos de la policía, y\ldots{}
francamente, como siempre tuviste afición a zurcir voluntades de
revolucionarios y preparar sediciones\ldots{} no levantaban una pieza
los buenos podencos de la Superintendencia, sin que Jenara y yo
dijéramos: «él es». Cuando Espronceda vino y se escondió por unas horas
en la Trinidad, creímos que eras tú. ¿Llegó un tipo, un no sé quién y
estuvo tres días en la botica de la calle de Hortaleza?\ldots{} pues
eras tú. ¿Hablose de otro que se metió en el \emph{guardamangier} de
Palacio y que luego resulto ser un choricero perseguido por haber dado
unos palos?\ldots{} pues tú. ¿Súpose por los serenos que un hombre
encopetado había entrado a deshora varias noches en casa de
Olózaga?\ldots{} pues tú. Pero el más gracioso engaño de todos es el que
padeció nuestra paisanita durante la prisión de Olózaga, engaño en el
cual no he tenido parte ni responsabilidad. Ella sobornó carceleros y
compró mequetrefes de cárcel de esos que traen y llevan recados. Esta
gente sirve bien, como anden las onzas por medio, y lo prueba la evasión
de Olózaga. Pues bien. En el torreón de la Villa había un preso a quien
daban el nombre de Escoriaza, el cual unas veces atribuía su
encerramiento a cosas de mujeres, y otras a tramas políticas. Intrigando
para salvar a Olózaga, nuestra amiga, cuyo corazón es tan grande como su
entendimiento, se interesaba por el misterioso Escoriaza,
creyendo\ldots{} no podía faltar la muletilla\ldots{} creyendo que eras
tú. Él recibió recados y dineros, comprendió que había un engaño y lo
sostuvo hábilmente. En fin, querido, a la postre resultó ser ese
raterillo a quien llaman Candelas, que si Dios no lo remedia, pasará a
la posteridad por sus hazañas. Mira, Salvador, cuando lo supe, estuve
riéndome dos horas\ldots{} Por último, al cabo de tantas equivocaciones
vino la verdad, y la sin par Generosa, que te buscaba en todas partes,
te encontró de improviso en su propia casa, en casa de D. Felicísimo. Y
fue de la manera más inesperada y más teatral. Un día vio sobre la mesa
de Carnicero una carta para D. Jaime Servet, nombre que usaste en
Cataluña, según nos dijo el marqués de Falfán de los Godos, que te
encontró en Canfranc cuando volvías sano y salvo a Francia. Al punto
Jenara\ldots{} ya sabes que es un fuego vivo de actividad y de
impaciencia\ldots{} corrió a la posada del Dragón\ldots{} ¡Qué
desgracia!, no estabas\ldots{} Pasaron días. La carta para ti volvió a
la mesa de D. Felicísimo donde ha estado dos meses esperándote. Pero
ayer nuestra amiga sintió una voz en el despacho de Carnicero; ella y
Micaela se acercaron, entreabrieron la puerta, miraron\ldots{} Eras tú,
tú mismo, real, verdadero, efectivo. Jenara se desmayó en el pasillo y
Micaela y yo la llevamos a su cuarto, donde sin más medicina que un
vasito de agua, volvió en sí y de repente me dijo entre riendo y
llorando: «Ha engrosado bastante ese badulaque\ldots» Y en conclusión,
chico, esta tarde tendrás el gusto de verla, porque para eso estás aquí
y para eso te he convidado de acuerdo con ella, y ya\ldots{}

El cortesano miró el reló, añadiendo con socarronería:

---No, no es hora todavía\ldots{} ¿Llevarás a mal lo que he hecho? ¡Qué
demonios! Si supieras el interés que tiene por ti\ldots{} Te quiere como
a un hijo.

Salvador no dijo cosa alguna concreta acerca de este inopinado amor de
madre que la señora le tenía, y volviendo al tema pasado riose mucho de
los lances cómicos ocurridos con su supuesta persona, y principalmente
de haber sido confundido con dos hombres que habían de ser pronto
celebridades del siglo, si bien de orden muy distinto, Espronceda y
Candelas. Dijo luego que al volver a Francia de vuelta de Cataluña,
había seguido ayudando a Mina en sus planes; pero que, desde la
intentona del año 30, había cesado en sus trabajos, renunciando para
siempre y con decidido propósito a la política. Desde que tal resolución
tomó, habíase aplicado a buscar los medios de volver libremente a
España, donde le llamaban afectos nobles y una regular herencia por
recoger. Tuvo la suerte entonces de conocer a D. Alejandro Aguado, el
cual le empleó en diferentes comisiones en Bélgica e Inglaterra. Sirvió
con celo y habilidad al banquero, y el banquero se encargó de abrirle
las puertas de España. Quiso traerle cuando vino Rossini en Marzo del
31; pero entonces no fue posible. A la vuelta de Aguado a Francia, el
célebre contratista dio a Salvador el encargo de reunirle cuadros para
su afamada colección (que hoy puede admirarse en el Louvre), y para
esto, y para hacerle posible la residencia en España, escribió en su
obsequio cartas de recomendación de esas que todos los obstáculos
allanan y vencen dificultades que al oro mismo son rebeldes. Aguado era
el prestamista del Tesoro español y tenía en su mano la fortuna pública
y gran parte de la privada de esta nación venturosísima. Por estas
causas sus relaciones en Madrid eran sólidas y su firma como una especie
de fórmula abreviada del Evangelio.

D.~Felicísimo había tenido a principios de 1831 correspondencia con
Aguado, con motivo de ciertos negocios de los Santos Lugares que este
arregló en París y Roma. Concluidas y zanjadas las cuentas a gusto de
ambos, lo mismo el banquero que el agente eclesiástico deseaban ocasión
de servirse mutuamente, y como en poder de Carnicero obraba todavía una
cantidad, resto de la negociación realizada y de la cual debía disponer
Aguado, este suplicó a su amigo la entregase al Sr.~D. Jaime Servet, su
amigo y corresponsal que llegaría a Madrid en época concertada.
Reservadamente enteraba Aguado a Carnicero de quién era este Servet y de
su verdadero nombre y la herencia y los cuadros y los propósitos
pacíficos que llevaba a Madrid, por lo cual esperaba que le ayudase en
todo. Con esto y con las cartas que Salvador trajo para Calomarde,
Varela, Ballesteros y la Reina Cristina, no fue difícil que al llegar a
Madrid dejase su falso nombre, entrando en el pleno goce de lo que
podría llamarse derechos civiles y que era en realidad tolerancia o
benignidad del gobierno absoluto. La carta para Cristina, que entregó el
primer día, fue como es de suponer eficacísima, y todo lo demás se le
hizo fácil. Ya tenemos noticia de las buenas disposiciones de Carnicero,
el cual miraba al Sr.~Aguado como a un Dios; pues en aquel espíritu el
furor apostólico no excluía la adoración de becerros de oro con todos
los servilismos que esta religiosidad insana trae consigo.

Ya habían concluido de comer y estaban de sobremesa fumando excelentes
puros, cuando sonó la campanilla, y Pipaón dijo a su amigo:

---Me parece que ya está ahí. Es puntual como la hora triste.

Salvador hizo una pregunta interesante por demás, a la cual contestó el
tunante de Pipaón con sonrisa maliciosa y en voz tan baja que el
narrador se quedó en ayunas. Es evidente que la pregunta se refería a la
señora que en aquel momento llamaba a la puerta, y también lo es que
Pipaón contestó con un nombre. Lo único que pudimos percibir de este
oscurísimo coloquio fue la observación de Salvador, diciendo:

---Me lo figuré\ldots{} le vi en Francia\ldots{} ¡qué cosas!

Era ella en efecto. Salvador, dejando a su amigo, fue a la sala, donde
la encontró de pie, fijos los ojos en la puerta. Se saludaron con
afecto, demostrándose el uno al otro sentimientos de amistad y alegría
por verse después de tanto tiempo. En ella había cierto alborozo del
alma que luchaba por encerrarse en el círculo de lo que se llama
satisfacción en lenguaje de urbanidad, y en él había frialdades que se
mostraban de improviso, rompiendo el velo de expresiones convencionales
con que las quería cubrir. Ella estaba turbada, tan turbada que después
de los primeros saludos decía una cosa por otra; él no parecía muy
sereno, pero se recobró antes que ella y fue de los dos el primero que
rió. ¡Sabe Dios cuál sería el último!

La discreción que en el uno emanaba naturalmente del desamor y en la
otra del remordimiento, les llevó a una conversación en que ni por
incidencia se tocó ningún punto de la vida pasada de ambos. Hablaron del
tiempo y de política, los dos temas obligados en toda reunión donde no
hay nada de que hablar. Allí parecía más bien que ella y él temían
abordar otros asuntos. Lo único que se permitió Jenara fuera de los
lugares comunes de la política y el tiempo, fue algunas exhortaciones
que demostraban bastante interés por el que fue su amigo.

---No te fíes de esta gente, ni de la buena acogida que te han
hecho---le dijo.---Esta canalla es más temible cuanto más halaga, y
cuando parece que perdona es que prepara el golpe de muerte. La
protección de la Reina Cristina, que tanto considera al Sr.~Aguado, te
servirá de mucho mientras haya tal Reina; pero, hijo, aquí no hay nada
seguro; estamos sobre un abismo. Al Rey le repiten ya con más frecuencia
los ataques de gota y el mejor día nos quedamos sin él. Ya supones lo
que pasará en la botella de cerveza el día que le falte el corcho.
Muerto el Rey, adiós Reina y Roque; se armará aquí una marimorena de
todos los demonios, y el bando apostólico será dueño del reino y nos
hará gustar las delicias del gobierno de Cafrería. Como no me resigno a
que me gobiernen a la africana, tengo todo preparado para marchar en
cuanto haya síntomas; así desde que el Rey cojea del pie izquierdo, ya
me tienes haciendo las maletas. Prepárate tú también, y no te fíes de la
protección de Cristina, un ídolo a quien derribará de su pedestal el
último suspiro del Rey.

Salvador, conviniendo en muchas de estas apreciaciones respondió que por
nada del mundo volvería a la emigración, y que resuelto a huir de la
política, esperaba que nadie le molestaría. No queda duda alguna de que
la hermosa dama, al oírle hablar tenía en su alma eso que no se puede
designar sino diciendo que estaba agobiada bajo un formidable peso.
Claramente decían sus ojos que tras de la fórmula artificiosa y vana que
articulaban los labios, había una reserva de palabras verdaderas que al
menor descuido de la voluntad saldrían en torrente diciendo lo que ellas
solas sabían decir. Que se echara fuera, por capricho o audacia, una
palabra sola y las demás saldrían vibrando con el sentimiento que las
nutría. Por un instante se habría creído que el volcán (demos al
fenómeno referido su nombre platónico convencional) llegaba al momento
supino de la erupción echando fuera su lava y su humo. Salvador tembló
al ver con cuánto afán, digno de mejor motivo, contaba la señora las
varillas de su abanico, pasándolas entre los dedos cual si fueran
cuentas de rosario, y mirándolo y remirándolo como si él también
hablase. Después la dama alzó los ojos que tenía empañados, cual si
fluctuara sobre aquel cielo azul la niebla del lloriqueo, y echando
sobre su amigo una mirada que era más bien explosión de miradas,
desplegó los labios, empezó una sílaba y se la tragó en seguida
juntamente con otras muchas, que estaban entre los lindos dientes
esperando vez. La señora se sometió a sí misma con formidable tiranía y
en vez de aquello que iba a decir no dijo más que esto:

---Hoy me han regalado una cesta de albaricoques.

A esta noticia insignificante contestó Monsalud diciendo que a él le
gustaban poco los albaricoques, y que delante de un racimo de uvas no se
podía poner ninguna otra especie de fruta. Con esto se empeñó un
eruditísimo coloquio sobre cuáles eran las mejores frutas, defendiendo
la señora con argumento irrebatible el melón de Añover y los
albaricoques de Toledo, pasando la conversación a los Cigarrales, y por
último a D. Benigno Cordero, a cuya obsequiosa amistad debía Jenara la
cestilla mencionada. Entonces el otro dio en hacer pregunta tras
pregunta sobre la honrada familia del encajero, y Jenara dio en
responderle con malísima gana y con tanta avaricia de palabras como
liberalidad de movimientos para darse aire con el abanico. Creeríase que
se estaba azotando el seno para castigarle de haber engrosado más de la
cuenta, y así todos los faralanes de su vestido en aquella parte se
agitaban como flámulas y gallardetes en día de festejo y de temporal. De
repente la señora cortó la conversación diciendo:

---Son las seis y Micaelita me espera para ir al Prado. Yo estoy libre
también; ya me ha dicho hoy D. Felicísimo por encargo del \emph{esposo
de la jorobada} (Calomarde) que se acabó la tontería de mi persecución.

Salvador manifestó alegrarse mucho de aquella franquicia, y no dijo sino
palabras convencionales y frías para retener a la dama en la visita.
También habló de su próximo viaje a Toledo. Ella se levantó, y sus
bellos ojos ya no echaban de sí sentimientos amorosos sino un
chisporroteo de orgullo. Despidiose secamente diciéndole: «Nos veremos
otro día» y se retiró majestuosa, como soberana que no sabe lo que es
abdicar y antes consentirá en equivocarse mil veces que en ceder una
sola.

\hypertarget{xxviii}{%
\chapter{XXVIII}\label{xxviii}}

A principios de Setiembre todavía el benignísimo D. Benigno no había
podido allanar aquel endiablado obstáculo de los papeles. El agente no
contestaba nada de provecho, y todo era dilaciones, por lo cual Cordero,
que ya iba perdiendo la paciencia, determinó hacer un viaje a Madrid
para comunicar algo de su inquietud y de su prisa al Sr.~Carnicero. El
héroe había resuelto encontrar los papeles, aunque tuviera que ir por
ellos a la misma villa de La Bañeza o al fin del mundo. Así lo dijo al
partir, despidiéndose para poco tiempo.

Dos días después de su partida estaba Sola en una de las piezas altas,
ocupada, por más señas, en pegar botones a una camisa de su futuro
esposo, cuando recibió aviso de que un señor acababa de llegar a la
finca y deseaba hablar con la señorita. Comprendiendo al punto quién
era, Sola se quedó como estatua, sin habla, sin ideas en la cabeza, sin
sangre en las venas, sintiendo una alegría disparatada, que al mismo
tiempo era pena muy viva, y miedo y cortedad de genio. Ella sabía quién
era el visitante; se lo decía aquel mismo azoramiento súbito en que
estaba y el horrible salto de su corazón alarmado. Había tenido noticia
por D. Benigno, dos semanas antes, de la aparición de Salvador en
Madrid, padeciendo con esto un trastorno general en sus ideas. Pocos
días después había recibido una carta del mismo anunciándole visita, y
desde que recibiera la carta el barullo de sus ideas y la estupefacción
de su alma habían aumentado. Grandes cosas se preparaban sin duda,
anunciándose en la infeliz joven con sentimientos de miedo y espasmos de
alegría. Armándose de valor, se dispuso a recibir al que un tiempo se
llamó su hermano. Mientras se arreglaba un poco para presentarse a él,
miró por la ventana. Allá abajo, entre los olivos, había un caballo,
sujeto por un muchacho de la casa. Era el caballo de él. La puertecilla
de la huerta por donde se pasaba para llegar a la casa, estaba abierta.
Él la había dejado abierta al pasar. En la salita baja se sentían pasos.
Eran sus pasos.

Sola bajó, apoyándose fuertemente en el barandal para no bajar de
cabeza. Entró en la salita\ldots{} ¡Qué grueso, qué moreno!\ldots{}
¡tenía algunas canas!\ldots{} Sola no pudo decir nada y se dejó abrazar
fuertemente.

---¡Ay!---exclamó sintiéndose inerte entre los brazos de su hermano, que
parecían de hierro.

Sola no se hacía cargo de nada. Estaba pálida y con los labios secos,
muy secos. No se dio cuenta de que él se sentó en un sofá de paja, que
era el principal adorno de la salita; no se dio cuenta de que él,
tomándole las manos, la llevó al mismo sofá y la sentó allí como se
sienta una muñeca; no se dio cuenta tampoco de que Salvador dijo:

---Ya sé que no está D. Benigno; ¡cuánto lo siento!

Sola no hacía más que mirarle asombrada, encontrándole grueso, no tan
grueso que perdiera su gallardía de otros tiempos; asombrada de verle
mucho más moreno y curtido que antes y con algunas manchas de canas en
el cabello.

---¡Me miras las canas!---dijo él.---Estoy viejo, hermana, viejo del
todo. A ti te encuentro más guapa, más mujer, más saludable. Ya sé que
eres tan buena como antes o más buena aún, si cabe. El marqués de Falfán
me ha hablado mucho de ti, y me contó tu grave enfermedad. ¡Pobrecita!
También sé que no has recibido mis cartas desde hace dos años, como no
las recibió Falfán ni otros amigos míos. Es una traición de Bragas,
aunque él jura y perjura que no ha recibido paquetes míos en mucho
tiempo. La última carta que me escribiste la recibí en Inglaterra hace
dos años. Después, yo escribía, escribía, y tú no me contestabas.

Hablaron un rato de aquel singular extravío de cartas, que no podía ser
sino pillada de Pipaón, falaz intermediario; pero como ya el mal pasado
no tenía remedio, dejaron de hablar de ello para ocuparse de cosas más
vivas y más interesantes para uno y otro.

---¡Cuántos años sin verte!---dijo él, mirándola de tan buena gana que
bien se conocía el largo ayuno que de aquellas vistas habían tenido sus
ojos.

---El marqués de Falfán---repitió ella,---que iba algunas veces a la
tienda de D. Benigno y siempre me hablaba de ti, me contó que pasando él
la frontera cierto día del año 27 te encontró. Ibas a caballo disfrazado
y te habías puesto el nombre de Jaime Servet. Este nombre se me quedó
tan presente que lo dije muchas veces cuando estaba delirando. Después
de esto me escribiste desde París. Un día que fuimos a ver entrar a la
Reina Cristina a casa de Bringas, me dio Pipaón una carta tuya; fue la
última. Poco después el marqués de Falfán me dijo que tenía ciertos
indicios para creer que habías muerto.

Salvador le contó luego a grandes rasgos los principales sucesos de su
vida en el período de ausencia, y le explicó las causas de su venida a
España. Lo que más sorprendió a Sola de cuanto dijo su hermano fue aquel
aborrecimiento a la política y al conspirar. Salvador le dijo:

---Cuando el hombre se enamora desde su niñez de ciertas ideas, o sea de
lo que llamamos ideales\ldots{} no sé si me entiendes\ldots{} y se lanza
a trabajar en ellos, se crea una vida artificial. Las ambiciones, la sed
de gloria y el afán de todos los días la forman. Así pasa el tiempo y
así consume el hombre las fuerzas de su alma en un combate con
fantasmas. Cuando hay éxito, querida hermanita, cuando Dios dispone las
cosas para que determinados hombres en determinados países sean
instrumento de planes providenciales, entonces la vida que he llamado
artificial puede dejar de serlo, mudándose en realidad hermosa. Pero
cuando no hay éxito, cuando después de mucho desvarío hallamos que todo
es quimera, sea por el tiempo, por el lugar o porque realmente no
valemos para maldita de Dios la cosa, resulta uno de estos dos
fenómenos: o la desesperación o el recogimiento y el deseo de la vida
vulgar, tranquila, compartida entre los afectos comunes y los deberes
fáciles. Yo he querido optar por lo segundo, que es más natural. Un
poeta hablando de estas cosas dijo: \emph{Es como una encina plantada en
un vaso, la encina crece y el vaso se rompe}. Yo creo que en la
generalidad de los casos hay que decir: \emph{El vaso es muy duro y la
encina se seca}, y este es el caso mío, querida.

Sola dio un suspiro por único comentario.

---La encina se seca---añadió Monsalud.---En mí se empezó a secar hace
tiempo, y ya quedan en ella muy pocas ramas con vida; pero a su sombra
ha nacido un árbol modesto que vivirá más y a falta de laureles dará
frutos\ldots{} Pronto tendré cuarenta años. ¡Si vieras tú qué efecto tan
raro nos hace el vernos de cerca de esta edad y reconocer que no hemos
vivido nada en tan larga juventud! Porque un hombre puede haber
emprendido muchas cosas, haber estudiado, leído y haber querido a muchas
mujeres, y sin embargo encontrarse el mejor día con la triste seguridad
de no ser nada, ni saber nada, ni amar a nadie. Pronto empezaré a ser
viejo. ¡Qué triste cosa es la vejez sin otros goces que las memorias de
una juventud alborotada ni más compañía que el rastro que dejaron todos
aquellos fantasmas y figurillas al convertirse en humo!\ldots{} Se me
figura que comprendes esto perfectamente\ldots{} ¿Pero a que no sabes
cuál es ahora la aspiración de mi vida?

---Ya me lo has dicho, no ser nada.

---Pues aspiro a ser el vecino tal, de tal calle, de cual pueblo; nada
más que un vecino, querida. ¿Crees que esto es fácil? Mira que no lo es.
La vida errante me fatiga, la vida solitaria me entristece. Para ser
vecino de tal calle es preciso fijarse y tener compañía que nos ate con
cuerda de afectos y deberes. No hay nada que tan dulcemente abrume al
hombre como el peso de un techo propio.

Esta frase, dicha así como sentencia, conmovió a Sola hasta lo más
profundo de su alma. Por un momento creyó que todo se volvía negro en su
alrededor.

---¿Qué dices a esto?---le preguntó él.---Hace un año, hallándome en
París curado ya de la manía del vivir quimérico, y prendado de amores
por la vida posible, por la vida que no temo llamar vulgar, te escribí,
manifestándote lo que pensaba.

---¡A mí!---exclamó Sola figurándose en el acto, como por inspiración
divina, la carta que no había recibido, y viéndola toda letra por letra.

---A ti\ldots{} Ya sé que no la recibiste. Sería preciso desollar vivo a
Pipaón. En mi carta te consultaba, te pedía consejo. Fue aquel un tiempo
en que tú te realzabas a mis ojos de un modo nuevo y no iba mi
pensamiento a ninguna parte sin tropezar contigo. Siempre había admirado
yo tus virtudes, siempre había sentido por ti un afecto entrañable; pero
entonces todos los sueños de la vida posible venían a mi cerebro como
envueltos en ti, quiero decir que todas las ideas de esta nueva
existencia y las imágenes de mi reposo y de mi felicidad futura se me
presentaban como un contorno de tu cara. Esto es concluir por donde
otros han empezado, esto es cosa de mozalbetes; pero los que no han
sabido vivir la vida del corazón cuando niños, la viven cuando viejos, y
así\ldots{}

La miró un rato y viéndola perpleja, él que gustaba de expresar las
cosas con prontitud y claridad, le dijo en un galanteo máximo todo lo
que tenía que decirle. Sus palabras fueron estas.

---Y así vengo a proponerte que nos casemos.

Sola no estaba ya confusa sino espantada. Se mordía un labio y la yema
de un dedo. Se los mordía tan bien que a poco más arrojara sangre. Al
mismo tiempo miraba al suelo, temerosa de mirar a otra parte. Su alma
estaba, si es permitido decirlo así, como una grande y sólida torre que
acababa de desplomarse sacudida por terremotos. No acertaba a pensar
cosa alguna derechamente, ni a concretar sus ideas para formar un plan
de respuesta. Salvador le tomó una mano. Entonces ella, herida de súbito
por no sé qué sentimiento, por el pudor, por la dignidad tal vez o
quizás por el miedo retiró su mano y dijo:

---Soy casada.

---¡Tú!\ldots{}

---Como si lo fuera. He dado mi palabra.

---En Madrid me dijeron eso, como una sospecha. Yo creí que era falso.

---Es cierto---dijo Sola que, recobrándose con gran esfuerzo, luchaba
con sus lágrimas para que no salieran.---Si no hubieran ocurrido ciertos
entorpecimientos, ya estaría casada con el mejor de los hombres.

A Salvador tocó entonces el morderse el labio y la coyuntura del índice
de su mano derecha. Sola invocó mentalmente a Dios, tomó fuerzas de su
valeroso espíritu y de la idea del deber que era siempre su confortante
más poderoso, y quiso dominar la situación haciendo el panegírico de su
futuro esposo.

---Hay un hombre---dijo,---a quien debo la vida, de quien he sido hija
cuando no tenía padre ni hermano. Siente por mí un respeto que yo no
merezco y un cariño que no podré pagar con cien vidas mías. Cuantos
miramientos, cuantas atenciones se puedan tener con una persona amada,
ha tenido él para mí. Yo he pedido a Dios que me diera algo con que
poder pagar beneficios tan grandes, y Dios ha puesto en mi corazón lo
que me hacía falta. Ese hombre ha querido tener casas, tierras, criados
para que yo fuera señora de todo, y él mío por toda la vida.

Salvador miró por la ventana los árboles, la deliciosa paz y abundancia
que todo aquel conjunto rústico expresaba. Sintió el corazón oprimido de
pena y lleno de la noble envidia que infunde el bien no merecido. En la
ventana que frente a él estaba, un arbolillo agitado por el viento
tocaba con sus ramas los vidrios. Varias veces durante el curso del
diálogo precedente, Salvador había mirado allí creyendo que alguien
llamaba en los vidrios. Ya llegado el momento de su desengaño, miró la
rama y viendo que daba más fuerte, murmuró: «Ya me voy, ya me voy».

Volviéndose otra vez a Sola, le dijo:

---Me has hablado en un lenguaje que no admite réplica. No debo
quejarme, pues he venido tarde, y habiendo tenido el bien en mi mano
durante mucho tiempo, lo he soltado para seguir locamente un camino de
aventuras. Pero algo me disculparán mi desgracia, mi destierro y también
mi pobreza, causa de que antes no te propusiera lo que ahora te
propongo. Aquí me tienes razonable, con esperanzas de ser rico, y a
pesar de tales ventajas, más desgraciado y más solo que antes.

Animada por el pequeño triunfo que había obtenido en su espíritu, Sola
quiso ir más allá, quiso hacer un alarde de valentía diciendo a su
amigo: \emph{ya encontrarás otra con quien casarte}; pero cuando iba a
pronunciar la primera sílaba de esta frase triste no tuvo ánimos para
ello y fue vencida por su congoja. No dijo nada.

---Yo quería---dijo Salvador, no desesperanzado todavía,---que
meditaras\ldots{}

Sola que vio un abismo delante de sí, quiso hacer lo que vulgarmente se
llama \emph{cortar por lo sano}.

---No hables de eso\ldots---dijo.---No puede ser\ldots{} Figúrate que no
existo.

Sin darse cuenta de ello le miró con lágrimas. Pero sobrecogida
repentinamente de miedo, se levantó y corriendo a la ventana se puso a
mirar los morales al través de los vidrios. Allí la infeliz imaginó un
engaño o salida ingeniosa para justificar su emoción. Volviose a él
segura de salir bien de tal empeño.

---¿Sabes por qué lloro? Porque me acuerdo de tu pobre madre, que murió
en mis brazos, desconsolada por no verte\ldots{} Dejome un encargo para
ti, un paquetito donde hay una carta y varias alhajas, encargándome que
a nadie lo fiara y que te lo diera en tu propia mano. ¡Y yo tan tonta
que no te lo he dado aún, cuando no debí hacer otra cosa desde que
entraste!\ldots{} Lo que me confió tu madre no se separa nunca de
mí\ldots{} Aquí lo tengo y voy a traértelo.

Sin esperar respuesta, Sola subió a su habitación y al poco rato puso en
manos de Monsalud un paquete cuidadosamente cerrado con lacres. Salvador
lo abrió con mano trémula. Lo primero que sacó fue una carta, que besó
muchas veces. En pie al lado de su amigo, que continuaba en el sofá de
paja, Sola no podía apartar los ojos de aquellos interesantes objetos.
La carta tenía varios pliegos. Salvador pasó la vista rápidamente por
ellos antes de leer.

---¡Mira, mira lo que dice aquí!---exclamó señalando una línea.---Mi
madre me suplica que me case contigo.

---Te lo suplicaba hace mucho tiempo---dijo Sola disimulando su pena con
cierta jocosidad afectada, que si no era propia del momento venía bien
como pantalla.

---Necesito una hora para leer esto---dijo Monsalud.---¿Me permites
leerlo aquí?

Sola miró a las ventanas y por un momento pareció aturdida. Su corazón
atenazado le sugería clemencia, mientras la dignidad, el deber y otros
sentimientos muy respetables, pero un poco lúgubres, como los
magistrados que condenan a muerte con arreglo a la justicia, le
ordenaban ser cruel y despiadada con el advenedizo.

---Mucho siento decírtelo, hermano---manifestó la joven sonriendo como
se sonríe a veces el que van a ajusticiar,---lo siento muchísimo; pero
va a anochecer. Tú que estás ahora tan razonable, me dirás si es
conveniente\ldots{}

---Sí, debo marcharme---replicó Salvador levantándose.

---Debes marcharte y no volver\ldots{} y no volver---afirmó ella
marcando muy bien las últimas palabras.

---¿Y qué pensaré de ti? Sola meditó un rato y dijo:

---¡Que me he muerto!

Se apretaron las manos. Sola miraba fijamente al suelo. Fue aquella la
despedida de menos lances visibles que imaginarse puede. No pasó nada,
absolutamente nada, porque no puede llamarse acontecimiento el que
\emph{Doña Sola} y \emph{Monda} se acercase a los vidrios de la ventana
para verle salir y que le estuviese mirando hasta que desapareció entre
los olivos, caballero en el más desvencijado cuartago que han visto
cuadras toledanas. Ni es tampoco digno de mención el fenómeno (que no
sabemos si será óptico o qué será) de que Sola le siguiese viendo aun
después de que las ramas de los olivos y la creciente penumbra de la
tarde ocultaran completamente su persona.

La noche cayó sobre ella como una losa.

~

Fatigado y displicente, con los hábitos arremangados y su gran caña de
pescar al hombro, subía el padre Alelí la cuestecilla del olivar. Ya era
de noche. Los muchachos acompañaban al fraile, trayendo el uno la cesta,
el otro los aparejos y el pequeño dos ranas grandes y verdes. Esto era
lo único que el reino acuático había concedido aquella tarde a la
expedición piscatoria de que era patrón el buen Alelí. Todas nuestras
noticias están conformes en que tampoco en las tardes anteriores fueron
más provechosas la paciencia del fraile y la constancia de los muchachos
para convencer a las truchas y otras alimañas del aurífero río de la
conveniencia de tragar el anzuelo; por lo que Alelí volvía de muy mal
humor a casa echando pestes contra el Tajo y sus riberas.

Todavía distaba de la casa unas cincuenta varas cuando encontró a Sola
que lentamente bajaba como si se paseara, saliendo al encuentro de las
primeras ondas de aire fresco que de los cercanos montes venían. Los
niños menores la conocieron de lejos y volaron hacia ella saludándola
con cabriolas y gritos, o colgándose de sus manos para saltar más a
gusto.

---¿Usted por aquí a estas horas?---dijo Alelí deteniendo el paso para
descansar.---La noche está buena y fresquita. ¿Querrá usted creer que
tampoco esta tarde nos han dicho las truchas esta boca es mía? Nada,
hijita, pasan por los anzuelos y se ríen. Esos animalillos de Dios han
aprendido mucho desde mis tiempos y ya no se dejan engañar\ldots{} Hola,
hola, ¿no son estas pisadas de caballo? Por aquí ha pasado un jinete.
Dígame usted, ¿ha enviado Benigno algún propio con buenas noticias?

Sola dio un grito terrible, que dejó suspenso y azorado al bondadoso
fraile. Fue que Jacobito puso una de las ranas sobre el cuello de la
joven. Sentir aquel contacto viscoso y frío y ver casi al mismo tiempo
el salto del animalucho rozándole la cara fueron causa de su miedo
repentino; que este modo de asustarse y esta manera de gritar son cosas
propias de mujeres. Alelí esgrimió la caña, como un maestro de escuela,
y dio dos cañazos al nene.

---¡Tonto, mal criado!

---No, no han venido buenas noticias---dijo Sola temblando.

Aquella noche cenaron como siempre, en paz y en gracia de Dios, hablando
de Cordero y pronosticando su vuelta para tal o cual día. La vida feliz
de aquella buena gente no se alteró tampoco en lo más mínimo en los
siguientes días. Sola estaba triste; pero siempre en su puesto, siempre
en su deber, y todas las ocupaciones de la casa seguían su marcha
regular y ordenada. Ninguna cosa faltó de su sitio ni ningún hecho
normal se retrasó de su marcada hora. La reina y señora de la casa,
inalterable en su delicado imperio, lo regía con actitud pasmosa, cual
si ni uno solo de sus pensamientos se distrajese de las faenas
domésticas. Interiormente fortalecía su alma con la conformidad y
exteriormente con el trabajo.

Fuera de algunos breves momentos, ni el observador más perspicaz habría
notado alteración en ella. Estaba como siempre, grave sin sequedad,
amable con todos, jovial cuando el caso lo requería, enojada jamás. Sin
embargo, cuando Crucita y ella se sentaban a coser, podían oírse en boca
de la hermana de D. Benigno observaciones como esta:

---Pero mujer, está \emph{Mosquetín} haciéndote caricias y ni siquiera
le miras.

Sola se reía y acariciaba al perro.

---Hace días que estás no sé cómo\ldots---continuaba el ama de
\emph{Mosquetín}.---Nada, mujer, ya vendrán esos papeles; no te apures,
no seas tonta. Pues qué, ¿han de estar en la China esos cansados
legajos?\ldots{} ¡Vaya cómo se ponen estas niñas del día cuando les
llega el momento de casarse! Todo no puede ser a qué quieres boca. Menos
orgullito, señora, que ya que el bobalicón de mi hermano ha querido
hacerte su mujer, Dios no ha de permitir que este disparate se realice
sin que te cueste malos ratos.

Sola se volvía a reír y volvía a acariciar a \emph{Mosquetín}.

Una mañana, los chicos, que estaban en la huerta haciendo de las suyas,
empezaron a gritar: «Padre, padre». D. Benigno llegaba. Entró en la casa
sofocado, ceñudo, limpiándose con el pañuelo el copioso sudor de su
inflamado rostro, y dejándose caer en una silla con muestras de
cansancio, no decía más que esto:

---¡Los papeles!\ldots{} ¡Los papeles!\ldots{} ¡D. Felicísimo!\ldots{}

---¿Qué?\ldots{} ¿Han parecido?\ldots---le preguntó Sola con ansiedad.

---¡Qué han de aparecer!\ldots{} ¡Barástolis! No hay paciencia para
esto, no hay paciencia\ldots{}

\hypertarget{xxix}{%
\chapter{XXIX}\label{xxix}}

¿Y cómo habían de aparecer, santo Dios, si el cura de La Bañeza, a
consecuencia de una reyerta con el obispo de la diócesis había hecho la
gracia de huir del pueblo, después de arrojar a un pozo todos los libros
parroquiales? Véase aquí por dónde la tremenda y sorda lucha que entre
el régimen absolutista y el espíritu moderno estaba empeñada, había de
estorbar la felicidad de aquel candoroso Don Benigno, que, aunque
liberal, en nada se metía.

Era el obispo de León, Sr.~Abarca, absolutista furibundo de ideas y
aragonés de nacimiento, con lo que basta para pintarle. De consejero
áulico del Rey y atizador de sus pasiones pasó a la intimidad de D.
Carlos y a la dirección del partido de este, llegando a ser más tarde
ministro universal de la corte de Oñate. El cura de La Bañeza se
diferenciaba de su pastor en lo de liberal, y se le parecía en que era
aragonés. Puede suponerse lo que sería una pendencia clerical y política
entre dos aragoneses de sotana. El obispo tenía, entre otros defectos,
el de los modos ásperos, los procedimientos brutales y las palabras
destempladas; el cura, sobre todas estas máculas, tenía la de ser algo
más presbítero de Baco que sacerdote de Cristo. Resistiose el cura a
dejar la parroquia (que precisamente estaba a cuatro pasos de la
taberna); insistió el obispo, salieron a relucir mil zarandajas,
canónicas de un lado, liberalescas de otro, y al fin, vencido el
subalterno, escapó una noche antes de que le cayera encima el brazo
secular; pero como hombre de ideas filosóficas, pensó que los libros
parroquiales, por ser expresión de la verdad, debían estar, como la
verdad misma, en el fondo de un pozo, y de aquí la pérdida de los tales
libros.

De orden de Su Ilustrísima hízose una información en el pueblo para
restablecer los libros, y al cabo de algunos meses, D. Benigno supo por
Carnicero que en la partida de bautismo no había ya dificultades. Pero
el Demonio, que siempre está inventando diabluras, hizo que apareciese
nueva contrariedad. Uno de los libros del registro de matrimonios se
había conservado y en el tal libro constaba que una Soledad Gil de la
Cuadra había contraído nupcias en 1823. Indudablemente no era esta
Soledad nuestra simpática heroína; pero mientras se ponía en claro, ji,
ji, (así lo decía D. Felicísimo a su cliente Cordero) había de pasar
algún tiempo, siendo quizás preciso llevar el asunto a un tribunal
eclesiástico, pues estas delicadas cosas no son buñuelos, que se hacen
en un segundo.

Así, entre obispos y curas aragoneses, pozos llenos de libros, agentes
eclesiásticos y torna y vuelve y daca, el héroe de Boteros sufrió el
martirio de Tántalo durante un año largo, pues hasta el verano de 1832
no se allanaron las dificultades. Cuando D. Felicísimo escribió a
Cordero participándole este feliz suceso añadía que sólo faltaba una
firma del señor Obispo Abarca para que todo aquel grandísimo lío
terminase.

Durante esta larga espera la familia de Cordero continuaba sin novedad
en la salud y en las costumbres. El invierno lo pasaron en Madrid para
atender a la educación de los niños y a la tienda, que D. Benigno juró
no abandonar mientras el edificio de sus felicidades no fuese coronado
con la gallarda cúpula de su casamiento. Desde la primavera se
trasladaron todos a los Cigarrales, acompañados de Alelí que cada día
tomaba más afición a la familia y se entretenía en enseñar a
\emph{Mosquetín} a andar en dos pies.

Innecesario será decir, pero digámoslo, que D. Benigno, si bien trataba
familiarmente a Sola, no traspasó jamás, en aquella larga antesala de
las bodas, los límites del decoro y de la dignidad. Se estimaba
demasiado a sí mismo y amaba a Sola lo bastante para proceder de aquella
manera delicada y caballerosa, magnificando su ya magnífica conducta con
el mérito nuevo de la castidad. Ni siquiera se permitía tutear a su
prometida, porque el tuteo, decía, trae insensiblemente libertades
peligrosas, y porque el decoro del lenguaje es siempre una garantía del
decoro de las acciones.

En este tiempo ocurrió también la dispersión de algunos personajes muy
principales de esta historia. Salvador se fue a Andalucía donde encontró
abundancia de cuadros y antigüedades de mérito. Luego subió por
Extremadura a Salamanca, vino a Madrid, en febrero de 1832 a exigir a
Carnicero el cumplimiento del pacto, y habiendo ocurrido ciertas
dilaciones, celebraron un nuevo pacto-prórroga, que terminó cuatro meses
después con feliz éxito el asunto. El aventurero vio al fin en sus manos
la mitad de la herencia de su tío, gracias a las uñas de D. Felicísimo,
que acariciando la otra mitad, desenmarañó la madeja. Fue Salvador a
París en la primavera para rendir cuentas a Aguado, y en el verano tornó
a España y a Madrid para ultimar un asunto de vales reales que en la
Corte tenía.

Jenara pasó en Madrid el invierno de 1831 a 1832 y en primavera se
trasladó a Valencia, volviendo al poco tiempo para instalarse en San
Ildefonso. La opinión pública que, tal vez sin motivo, le tenía mala
voluntad, hacía correr acerca de su conducta rumores poco favorables,
aunque eran de esos que cualquier dama ilustre de aquellos tiempos y de
estos y todos los tiempos soporta sin detrimento alguno en el lustre de
su casa, antes bien aumentándolo y viéndose cada día más obsequiada y
enaltecida. Si en el año anterior fue tildada de aficionarse con exceso
a la oratoria forense y parlamentaria, ahora decían de ella que se
pirraba por la poesía lírica, prefiriendo sobre todos los géneros el
\emph{byroniano}, o sea de las desesperaciones y lamentos, sin admitir
consuelo alguno en este mundo ni en el otro.

Enorme escuadrón de amigos la despidió al marchar a la Granja. Adiós,
gentil Angélica, engañadora Circe. No podemos seguirte aún. Nos llaman
por algún tiempo en Madrid afecciones de literatos que nos son más caras
que las propias niñas de nuestros ojos. Y era curioso ver cómo se iba
encrespando aquel piélago de ideas, de temas literarios e imágenes
poéticas del cafetín llamado Parnasillo. Sin duda de allí había de salir
algo grande. Ya se hablaba mucho y con ardor de un drama célebre
estrenado en París el 25 de Febrero de 1830 y que tenía el privilegio de
dividir y enzarzar a todos los ingenios del mundo en atroz contienda. El
asunto, según algunos de los nuestros, no podía ser más disparatado. Un
príncipe apócrifo que se hace bandolero, una dama obsequiada por tres
pretendientes, un viejo prócer enamorado, y un emperador del mundo, son
los personajes principales. Luego hay aquello de que todos conspiran
contra todos y de que pasan cosas históricas que la historia no ha
tenido el honor de conocer jamás. Y hay un pasaje en que el prócer que
aborrece al bandido lo salva del emperador; y luego el emperador se
lleva la muchacha y el bandolero se une al prócer; y como uno de los dos
está demás porque ambos quieren a la señorita, el bandolero jura que se
matará cuando el prócer toque un cierto cuerno que aquel le da en prenda
de su palabra; y cuando todo va a acabar en bien porque el emperador ha
perdonado a chicos y grandes y viene el casorio de los amantes con
espléndida fiesta, suena el consabido cuerno: el príncipe bandolero se
acuerda de que juró matarse, y en efecto se mata.

Si a unos les parece esto el colmo del absurdo, a otros les parece de
perlas. Riñen los exaltados con los retóricos, y en medio de las
disputas sale a relucir una palabra que estos profieren con desprecio,
aquellos con orgullo. \emph{¡Románticos!}\ldots{} Aguarde un poco el
lector que ya vendrán a su tiempo la amarillez del rostro, las largas y
descuidadas melenas, las estrechas casacas. Por ahora el romanticismo no
ha pasado a las maneras ni al vestido, y se mantiene gallardo y
majestuoso en la esfera del ideal.

El drama francés es un monstruo para algunos; pero ¡qué aliento de vida,
de inspiración, de grandeza en este monstruo, pariente sin duda de las
hidras calderonianas, ante cuya indómita arrogancia, a veces sublime,
salvaje a veces, parecen gatos disecados las esfinges del clasicismo!
Contra la frialdad de un arte moribundo protesta un arte incendiario; la
corrección es atropellada por el delirio; las reglas con sus gastados
cachivaches se hunden para dar paso a la regla única y soberana de la
inspiración. Se acaba la poesía que proscribe los personajes que no sean
reyes, y se proclama la igualdad en el colosal imperio de los
protagonistas. Rómpese como un código irrisorio la jerarquía de las
palabras nobles e innobles, y el pueblo con su sencillez y crudeza
nativa habla a las musas de tú. Caen heridos de muerte todos los
monopolios: ya no hay asuntos privilegiados, y al templo del arte se le
abren unas puertas muy grandes para dar paso a la irrupción que se
prepara. Se suprimen los títulos nobiliarios de ciertas ideas, y se
ordena que el Mar, por ejemplo, que de antiguo venía metiendo bulla y
soplándose mucho con los retumbantes dictados de Nereo, Neptuno, Tetis,
Anfitrite, sea despojado de estos tratamientos y se llame simplemente
Fulano de Tal, es decir, \emph{el Mar}. Lo mismo les pasa a la Tierra,
al Viento, al Rayo.

Mucho podríamos decir sobre esta revolución que tuvimos la gloria de
presenciar; pero damos punto aquí porque no es llegada aún la sazón de
ella, y sus insignes jefes no eran todavía más que conspiradores. El
café del Príncipe era una logia literaria, donde se elaborara entre
disputas la gloriosa emancipación de la fantasía, al grito mágico de
\emph{¡España por Calderón!}

El teatro estaba aún solitario y triste; pero ya sonaban cerca las
espuelas de \emph{Don Álvaro}. \emph{Marsilla} y \emph{Manrique} estaban
más lejos, pero también se sentían sus pisadas, estremeciendo las
podridas tablas de los antiguos corrales. Comenzaba a invadir los ánimos
la fiebre del sentimiento heroico, y las amarguras y melancolías se
ponían de moda.

Las grandes obras de Espronceda no existían aún, y de él sólo se
conocían el \emph{Pelayo}, la \emph{Serenata} compuesta en Londres y
otras composiciones de calidad secundaria. Vivía sin asiento, derramando
a manos llenas los tesoros de la vida y de la inteligencia, llevando
sobre sí, como un fardo enojoso que para todo le estorbaba, su genio
potente y su corazón repleto de exaltados afectos. Unos versos
indiscretos le hicieron perder su puesto en la Guardia Real. Fue
desterrado a la villa de Cuéllar, donde se dedicó a escribir novelas.

Vega había escrito ya composiciones primorosas; pero sin entrar aún en
aquellas íntimas relaciones con Talía, que tanto dieron que hablar a la
Fama. Bretón había vuelto de Andalucía, y con sin igual ingenio
explotaba la rica hacienda heredada de Moratín. Martínez de la Rosa
trabajaba oscuramente en Granada. Gallego estaba a la sazón en Sevilla;
Gil y Zárate, perseguido siempre por la inquisitorial censura del padre
Carrillo, había abandonado el teatro por una cátedra de francés.
Caballero, Villalta, Revilla, Vedia, Segovia y otros insignes jóvenes
cultivaban con brío la lírica, la historia y la crítica.

Al propio tiempo la pintura de la vida real, es decir, del espíritu,
lenguaje y modo de la sociedad en que vivimos, era acometida por un
joven artista madrileño para quien esta grande empresa estaba guardada.

Miradle. No parece tener más de veintiséis o veintisiete años. Es
pequeño de cuerpo, usa anteojos y siempre que mira parece que se burla.
Es, más que un hombre, la observación humanada, uniéndose a la gracia y
disimulando el aguijoncillo de la curiosidad maleante con el floreo de
la discreción. De sus ojos parte un rayo de viveza que en un instante
explora toda la superficie y sin saber cómo se mete hasta el fondo,
sacando los corazones a la cara; al mismo tiempo parece que se ríe, como
dando a entender que no hará daño a nadie en sus disecciones de vivos.

Este joven a quien estaba destinado el resucitar en nuestro siglo la
muerta y casi olvidada pintura de la realidad de la vida española tal
como la practicó Cervantes, comenzó en 1832 su labor fecunda, que había
de ser principio y fundamento de una larga escuela de prosistas. Él
trajo el cuadro de costumbres, la sátira amena, la rica pintura de la
vida, elementos de que toma su sustancia y hechura la novela. Él arrojó
en esta gran alquitara, donde bulliciosa hierve nuestra cultura, un
género nuevo, despreciado de los clásicos, olvidado de los románticos, y
él solo había de darle su mayor desarrollo y toda la perfección posible.
Tuvo secuaces, como Larra, cuya originalidad consiste en la crítica
literaria y la sátira política, siendo en la pintura de costumbres
discípulo y continuador de \emph{El Curioso Parlante}; tuvo imitadores
sin cuento y tantos, tantos admiradores que en su larga vida los
españoles no han cesado de poner laureles en la frente de este valeroso
soldado de Cervantes.

En 1831 hizo el \emph{Manual de Madrid}, anunciando en él sus dotes
literarias y una pasión que le había de ocupar toda la vida, la pasión
de Madrid. En Enero del año siguiente publicó \emph{El Retrato en las
Cartas Españolas} de Carnerero, y tras \emph{El Retrato} vino sin
interrupción esa galería de deliciosos cuadros matritenses, que servirá,
el día en que la capital de España se pierda, para encontrarla aunque se
meta cien estados bajo tierra. ¡Asombroso poder del ingenio! Aquellos
revueltos tiempos en que se decidió la suerte de la nación española han
quedado más impresos en nuestra mente por su literatura que por su
historia; y antes que la Pragmática Sanción, y el Carlismo y la Amnistía
y el Auto acordado y la Corte de Oñate y el Estatuto, viven en nuestra
memoria D. Plácido Cascabelillo, D.~Pascual Bailón Corredera, D.
Solícito Ganzúa, D. Homobono Quiñones y otras dignas personas nacidas de
la realidad y lanzadas al mundo con el perdurable sello del arte.

En Agosto del mismo año de 1832 principió a salir el \emph{Pobrecito
Hablador} de Larra. De este quisiéramos hablar un poco; pero el
insoportable calor nos obliga a salir de Madrid.

Antes de partir haremos una visita a D. Felicísimo, en cuya casa
hallamos grandísima novedad, y es que al cabo de muchas dudas y
vacilaciones, el insigne Pipaón se decidió a manifestar a Micaelita su
propósito de tomarla por esposa, considerando para sí que si buenos
desperfectos tenía, con buenas talegas iban disimulados. Es opinión
admitida por todos los historiadores que Micaelita no rezó ningún
Padrenuestro al oír nueva tan lisonjera de los labios del cortesano de
1815. D. Felicísimo y doña Sagrario se regocijaron mucho, pues no podían
soñar mejor partido para aquel poco solicitado género, que un individuo
encaminado a ser, por sus prendas especiales el Calomarde de los
venideros tiempos.

Nuestra buena suerte quiso que al dar un vistazo al agente de asuntos
eclesiásticos halláramos al Sr.~de Pipaón, que también se despedía.
Deleitosa conversación se entabló entre los dos. Cuando el cortesano
estrechó entre los suyos fuertísimos los dedos de corcho del Sr.~D.
Felicísimo, este exhaló un hipo y dijo:

---Me olvidaba\ldots{} Querido Pipaón, puesto que va usted
inmediatamente para allá, hágame el favor de llevar esta carta.

Y diciéndolo, el anciano levantó el pie de cabrón con ademán que algo
tenía de ceremonioso y cabalístico, como el mágico que alza cubiletes y
descubre signos. El sobre de la carta de que se hizo cargo Pipaón,
decía:

\emph{Al Sr.~D. Carlos Navarro, en San Ildefonso.}

\hypertarget{xxx}{%
\chapter{XXX}\label{xxx}}

En los primeros días del mes de Setiembre, un viajero llegó a la posada
del Segoviano en la Granja, y pidió cuarto y comida, exigencias a que
con tanto tesón como desabrimiento se negó el fondista. Era inaudito
atrevimiento venir a pedir techo y manteles en una posada que por su
mucha fama y prez estaba llena de gente principal desde el sótano a los
desvanes. ¡Ahí era nada en gracia de Dios lo de personajes que en la
casa había! Cuatro consejeros de Estado, un fiscal de la Rota, un
administrador del Noveno y Excusado, dos brigadieres exentos, un padre
prepósito, un definidor y seis cantores de ópera sobrellevaban allí con
paciencia las incomodidades de los cuartos y compartían el ayuno de las
parcas comidas y mermadas cenas.

---Perdone por Dios, hermano---dijo a nuestro viajero el implacable
dueño del mesón, que reventaba de gordura y orgullo considerando el buen
esquilmo de aquel año, gracias al ansia de los partidos que tanta gente
llevaba a San Ildefonso.

Y el viajero redoblaba su amabilidad suplicante, en vista de la negativa
venteril. Era tímido y circunspecto, quizás en demasía para aquel caso
en que tenía que habérselas con la ralea de posaderos y fondistas.

---Deme usted un cuchitril cualquiera---dijo.---No estaré sino el tiempo
necesario para conseguir que Su Ilustrísima el Sr.~Abarca eche una firma
en cierto documento.

---¿El Sr.~Abarca?\ldots{} Buena persona\ldots{} Es muy amigo
mío---replicó el ventero.---Pero no puedo alojarle a usted\ldots{} Como
no sea en la cuadra\ldots{}

Ya se había decidido el atribulado señor a aceptar esta oferta, cuando
acertó a pasar D. Juan de Pipaón. El viajero y el cortesano se vieron,
se saludaron, se abrazaron, y\ldots{} ¿cómo había de consentir D. Juan
que un tan querido amigo suyo se albergara entre cuadrúpedos teniendo
él, como tenía, en la casa de Pajes, dos hermosísimas y holgadas
estancias, donde estaba como garbanzo en olla?

---Venga conmigo el buen Cordero---dijo con generosa bizarría,---que le
hospedaré como a un príncipe. La Granja rebosa de gente. Amigo---añadió,
hablándole al oído, cuando ambos marchaban hacia la casa de Pajes,---el
Rey se nos muere.

---De modo que sobrevendrá\ldots{}

---El diluvio universal\ldots{} Háblase de componer la cosa en familia.
Pero vamos, vamos a que descanse usted.

Cordero dio un suspiro y ambos entraron en la casa. Después de un ligero
descanso y del desayuno consiguiente, Cordero salió a ver los jardines.

~

¡La Granja! ¿Quién no ha oído hablar de sus maravillosos jardines, de
sus risueños paisajes, de la sorprendente arquitectura líquida de sus
fuentes, de sus laberintos y vergeles?\ldots{} Versalles, Aranjuez,
Fontainebleau, Caserta, Schoenbrünn, Potsdam, Windsor, sitios donde se
han labrado un nido los reyes europeos huyendo del tumulto de las
capitales y del roce del pueblo, podrán igualarle, pero no superan al
rinconcito que fundó el primer Borbón para descansar del gobierno. Y no
hay más remedio que admirar esta pasmosa obra del despotismo ilustrado,
reconociéndola conforme a la idea que la hizo nacer. El despotismo
ilustrado fomentó la riqueza en todos los órdenes, desterró abusos,
alivió contribuciones, acometió mejoras en bien del pueblo; pero todo lo
sometió a una reglamentación prolija. Hacía el bien como una merced y lo
distribuía como se distribuye la sopa a los pobres recogidos en un
asilo. Todo había de sujetarse a canon y a medida, y la nación, que nada
podía hacer por sí, lo recibía todo con arreglo a disciplina de
hospital.

El despotismo ilustrado da vida en el orden económico a los Pósitos, a
los Bancos privilegiados, a los Gremios; en el orden político crea los
pactos de familia, y en el artístico protege el clasicismo. Llega al fin
un día en que pone su mano en la Naturaleza, y entonces aparece Le
Nôtre, el arquitecto de jardines. Este hombre somete la vegetación a la
geometría y hace jardines con teodolito. A su mando inapelable los
árboles ya no pueden nacer libremente donde la tierra, el agua y Dios
quisieron que naciesen, y se ponen en filas, como soldados, o en
círculo, como bailarines. No basta esto para conseguir aquella
conformidad disciplinaria que es el mayor gusto del despotismo
ilustrado, y son escogidos los árboles como Federico de Prusia escoge a
sus granaderos. Es preciso que todos sean de un tamaño y que las ramas
crezcan por reguladas dosis. El hacha se encarga de convertir un bosque
en alameda, y surgen, como por encanto, esos bellos escuadrones de tilos
y esas compañías de olmos que parecen esperar el grito de un pino para
marchar en orden de parada.

El despotismo ilustrado y sus jardineros aspiran a más; aspiran a que la
Naturaleza no parezca Naturaleza sino un reino fiel sometido a la
voluntad de su dueño y señor. Las tijeras, que antes sólo eran arma de
los sastres, son ahora la primera herramienta de horticultura y con ella
se establece una igualdad de vasallaje que confunde en un solo tamaño al
grande y al chico. Es un instrumento de corrección como la lima de que
tanto hablaban los clásicos, y que a fuerza de pulimentar hacía que
todos los versos fueran igualmente fastidiosos. La tijera hace de los
amorosos mirtos y del espeso boj las baratijas más graciosas que puede
imaginarse. Córtalos en todas las formas, y talla guarniciones, muebles,
dibujos, casitas, arcos, escudos, trofeos. Los jardineros redondean los
árboles, dejándoles cual si salieran del torno, y las esbeltas copas se
convierten en pelotas verdes. En el bajo suelo cortan y recortan el
césped como se cortaría el paño para hacer una casaca, y luego bordan
todo esto con flores vivas que ponen donde la topografía ordena. Hacen
mil juegos y mosaicos, tapicerías y arabescos. ¡Ay de aquella florecilla
indisciplinada que se salga de su sitio! La arrancan sin piedad. La
lozanía excesiva tiene pena de muerte como la libertad entre los
hombres.

A un jardín le hacen parecer teatro, plaza, cementerio o cosa semejante.
Resulta un lugar frío, triste, desabrido, que trae al pensamiento las
tragedias en que Alejandro salía vestido de Luis XIV. Es preciso poner
algo que anime aquella soledad, algo que se mueva. ¿Quién será el juglar
de este escenario amanerado? Pues el agua. El agua que es la libertad
misma, la independencia, el perpetuo correr y la risa y la alegría del
mundo, es sacada de aquellos plácidos arroyos, de aquellas tranquilas
lagunas, de los agrestes manantiales y sujeta con presas y trasportada
en cañerías, y luego sometida al martirio inquisitorial de las fuentes
que la obligan a saltar y hacer cabriolas de un modo indecoroso. El
clasicismo hortícola quiere que en todo jardín haya mucha mitología,
faunos groseros, ninfas muy fastidiosas, dioses pedantes, geniecillos
mal criados. Pues todos estos individuos no tienen gracia si no echan un
chorro de agua, quién por la boca, quién por ánforas y caracoles, aquel
por todas las partes de su musgoso cuerpo, y diosa hay que arroja de sus
pechos cantidad bastante para abrevar toda la caballería de un ejército.

En la Granja la fuente de la Fama escupe al cielo un surtidor de 184
pies de altura y el Canastillo traza en el espacio todo un problema
geométrico con rayas de agua, mientras Neptuno, rigiendo sus caballos
pisciformes, eleva a los aires sorprendente arquitectura de movible
cristal que con los juegos de la luz embelesa y fascina. Las fuentes de
Pomona, Anfitrite y los Dragones también hacen con el agua las
prestidigitaciones más originales. Desde la plaza de las Ocho Calles se
ven, con sólo girar la mirada, todas las extravagancias de gimnástica y
coreografía con que el pobre elemento esclavizado divierte a reyes y a
pueblos. Los atónitos ojos del espectador dudan si aquello será verdad o
será sueño, inclinándose a veces a creer que es un manicomio de ríos.

Era primer domingo de mes y corrían las fuentes. Toda la sociedad del
Real Sitio estaba en los jardines disfrutando de la frescura del
ambiente y de la perspectiva de los árboles, cosa bellísima aunque
académica. Las damas de la corte y las que sin serlo habían ido a
veranear, los militares de todas graduaciones, los señores y los
consejeros, los lechuguinos y por último la gente del pueblo a quien se
permitía entrar aquel día por causa del correr de las fuentes, formaban
un conjunto tan curioso como rico en matices y animación. Por aquí
corrillos de pastoreo cortesano como el que inspiró a Watteau, por allá
rusticidades en crudo, más lejos Ariadnas que se quieren perder en
laberintillos de boj, y por todas las rectas calles grupos que se
cruzan, bandadas alegres que van y vienen. Como el agua salta risueña de
las tazas de mármol, así surge la conversación chispeante de los
movibles grupos. No se puede entender nada.

Allá va Pipaón con su amigo. Al pasar oímos que este le dijo:---Y Jenara
¿dónde está? No la he visto por ninguna parte.

---¿Qué la has de ver, si ha ido a Cuéllar?---replicó el cortesano. Y
perdiéronse entre el gentío elegante. El vestir ceremonioso era entonces
de rúbrica en los paseos, y no había las libertades que la comodidad ha
introducido después. Entonces ni el calor ni el esparcimiento estival
eran razones bastantes para prescindir de la etiqueta, y así lo mismo en
el Prado de Madrid que en los jardines de San Ildefonso, el hombre culto
tenía que encorbatinarse al uso de la época, que era una elegante
parodia de la pena de muerte en garrote vil. ¡Ay de aquel cuya cabeza no
se presentara sirviendo de cimiento a un mediano torreón de felpa negra
o blanca con pelos como de zalea, ala estrecha y figura cónico-truncada
que daba gloria verlo!

Las solapas altas, las mangas de pernil, las apretadas cinturas son
accidentes muy conocidos para que necesitemos pintarlos. El paño oscuro
lo informaba todo, y entonces no había las rabicortas americanas de
frágil tela, ni los trajes cómodos, ni sombreros de paja, ni quitasoles.

¿Pues y el vestido y los diversos atavíos de las damas? Entonces el
peinarse era peinarse; había arquitectura de cabellos y una peineta
solía tener más importancia que el Congreso de Verona. Para calle las
damas retorcían y alzaban por detrás el pelo sujetándole en la corona
con una peineta que se llamaba de \emph{teja}, de \emph{sofá} o de
\emph{pico de pato}, según su forma. ¡Qué cosa tan bonita!, ¿no es
verdad? Pues ved ahora por delante los rizos batidos, como una fila de
pequeños toneles negros o rubios suspendidos sobre la frente. Esto era
monísimo, sobre todo si se completaba tan lindo artificio con la cadena
a la \emph{Ferronière} y broche a la \emph{Sévigné} sujetando el
cabello. Esto hacía creer que las señoras llevaban el reloj en el moño,
de lo que resultaba mucho atractivo.

Tentado estoy de describiros el peinado a la \emph{jirafa} con tres
grandes lazos armados sobre un catafalco de alambre, los cuales lazos
aparecían como en un trono, rodeados de un servil ejército de rizos
huecos.

¡Cielos piadosos, quién pudiera ver ahora aquellas dulletas de inglesina
tan pomposas que parecían sacos, y aquellos abrigos de \emph{gros
tornasol} o de casimir \emph{Fernaux} o tafetán de Florencia,
guarnecidos de \emph{rulos} y trenzas, todo tan propio y rico que cada
señora era un almacén de modas! ¡Quién pudiera ver ahora resucitados y
puestos en uso aquellos vestidos de invierno, altos de talle, escurridos
de falda, y guarnecidos de marta o chinchilla! Lo más airoso de este
traje era el \emph{gato}, o sea un desmedido rollo de piel que las
señoras se envolvían en el cuello, dejando caer la punta sobre el pecho,
y así parecían víctimas de la voracidad de una cruel serpiente.

Pero estas son cosas de invierno, y volvamos a nuestro verano y a
nuestros jardines de La Granja. Todos los que esto lean, convendrán en
que no podría darse cosa más bonita que aquellas mangas de jamón,
abultadas por medio de ahuecadores de ballena, y con los cuales las
señoras parecían llevar un globo aerostático en cada brazo. ¡Y dicen que
entonces no había modas elegantes!

¿Pues, y dónde nos dejan aquel talle que por lo alto tocaba el cielo y
aquella falda que intentaba seguir el mismo camino, huyendo de los pies,
y aquel escote recto por pecho y espalda que a veces quería bajar al
encuentro del talle y que disimulaba su impudencia con hipocresía de
\emph{canesús} y sofisma de tules? Si no fuera porque las damas
ataviadas en tal guisa se asemejaban bastante a una alcazarra, este
vestido merecía haberse perpetuado. ¡Qué precioso era! Tenía la ventaja
de no alterar las formas, y entonces el pecho era pecho y las caderas
caderas.

¡Ay!, entonces también los pies eran pies, es decir que no había esas
falsificaciones de pies que se llaman botinas. Los zapateros no habían
intentado aún enmendar la plana a Dios creando extremidades
convencionales al cuerpo humano. ¿Y qué cosa más bonita que aquellas
galgas y aquel cruzado de cintas por la pierna arriba hasta perderse
donde la vista no podía penetrar? La suela casi plana, el tacón
moderado, el empeine muy bajo, eran indudablemente la última parodia de
aquellas sandalias que usaban las heroínas antiguas y que servían para
lo que no sirve ningún zapato moderno, para andar.

Ni que me maten dejaré de hablar de las mantillas, las cuales entonces
eran a propósito para echar abajo la teoría de que esta prenda no sirve
para nada. Entonces las mantillas eran mantillas; como que había unas
que se llamaban de toalla, y esto pinta su longitud. Aquellas mantillas
tapaban y tenían infinito número de pliegues, cuya disposición y
gobierno sometidos a la mano de la mujer que la llevaba, eran casi un
lenguaje. La toquilla de ahora es un adorno, la mantilla de entonces era
la persona misma. Las toquillas de hoy se \emph{llevan}; las mantillas
de entonces se \emph{ponían}. Los pliegues relumbrones de su raso
interior, el brillo severo de su terciopelo, la niebla negra de sus
encajes, hechura fantástica de hilos tejidos por moscas, y la
pasamanería de sus guarniciones reunían en derredor de una cara hermosa
no sé que misterioso cortejo de geniecillos, que ora parecían serios ora
risueños y a su modo expresaban el pudor y la provocación, la reserva o
el desenfado. El ideal se hizo trapo, y se llamó mantilla.

En cambio de otras ventajas que el vestir moderno lleva al antiguo,
aquellos tenían la de la variedad de tonos. Entonces los colores eran
colores, y no como ogaño variantes del gris, del canelo y de los tintes
metálicos. Entonces la gente se vestía de verde, de colorado, de
amarillo, y los jardines de la Granja vistos a lo lejos, eran un prado
de pintadas florecillas. El alepín, la cúbica, el tafetán de la reina,
el \emph{muaré antic}, las sargas, la inglesina, el \emph{cotepali}
ofrecían variedad de bultos y colores. Los parisienses que en esto de
hacer modas se pintan solos y cuando no pueden inventar formas y colores
nuevos les dan nombres extraños, habían lanzado al mundo el color
\emph{jirafa}, el \emph{pasa de corinto}, el no menos gracioso \emph{La
Vallière}, el \emph{azul Cristina}; pero los que verdaderamente merecen
un puesto en la historia son el color \emph{ayes de Polonia} y el
\emph{humo de Marengo}.

El cuadro de interés indumentario con fondos de verdor académico que
hemos trazado carece aún de ciertos tonos fuertes, que echará de menos
todo el que hubiera contemplado el original. Con el pincel gordo
apuntaremos en los primeros términos algunas manchas de encarnado
rabioso, amarillo y pardo que son las pintorescas sayas de las mujeres
del campo venidas de los inmediatos pueblos. La elegancia de estos
trajes se pierde en la oscuridad de los tiempos, y a nuestro siglo sólo
ha llegado una especie de alcachofa de burdos refajos, dentro de la cual
el cuerpo femenino no parece tal cuerpo, sino una peonza que da vueltas
sobre los pies, mientras los hombres, (aquí es preciso volcar sobre el
cuadro toda la pintura negra) fatigados y oprimidos dentro de las
enjutas chaquetas y los ahogados pantalones y las medias de punto,
parecen saltamontes puestos de pie, guardando la cabeza bajo anchísimo
queso negro.

El pincel más amanerado nos servirá para apuntar, oscilando sobre esta
multitud de cabezas, como las llamas de Pentecostés, los pompones de los
militares; y si hubiera tiempo y lienzo, pondríamos en último término,
con tintas graciosas, un zaguanete de alabarderos, que, semejante a un
ejército de zarzuela, pasa por el jardín precedido de su música de
tambor y pífanos. Lejos, más lejos aún que la vaporosa proyección del
agua en el aire, ponemos la fachada del palacio, rectilínea, clásica, de
formas discretas y limadas como los versos de una oda. ¡Ay!, en el
momento en que le contemplamos, gran gentío de cortesanos, militares y
personajes de todas las categorías entra y sale por las tres grandes
puertas del centro con afán y oficiosidad. De pronto el murmullo alegre
de las fuentes cesa, y todas dejan de correr. El agua vacila en los
aires, los chorros se truncan, se desmayan, descienden, caen, como
castillos fantásticos deshechos por la luz de la razón, y en estanques y
tazones se extingue el último silbido de los surtidores, que vuelven a
esconderse en sus misteriosas cañerías. En los jardines reina un estupor
lúgubre; la gente se para, pregunta, contesta, murmura, y de boca en
boca van pasando como chispazos de pólvora fugaz estas palabras: «El Rey
se muere, el Rey se muere».

Las puertas del palacio se abren de par en par. Entremos.

\hypertarget{xxxi}{%
\chapter{XXXI}\label{xxxi}}

---Se ha fijado la gota en el pecho\ldots{}

---Así parece.

---Peligro inminente\ldots{} ¡muerte!

---El Señor lo dispone así\ldots{}

El que tal dijo (y lo dijo con el aplomo del que está en los secretos de
Dios y mantiene relaciones absolutamente familiares con Él) era un
anciano corpulento, recio y hasta majestuoso, vestido de luengas ropas
moradas. Parecía la efigie de un santo doctor bajado de los altares, y
así sus palabras tenían una autoridad semi-divina. Hablaba
dogmáticamente y no admitía réplica. Era obispo y aragonés.

Su interlocutor vestía también ropas talares pero negras, sin adorno
alguno ni preciadas insignias. No parecía tener más de treinta y cinco
años y se distinguía por su hermosura como el obispo de León por su
apostólica majestad. Era el Padre Carranza, prepósito de los Jesuitas,
hombre listo si los hay, y además de cara bonita, calidad que avaloraba
su extraordinaria elocuencia, de tal modo que cuando subía al púlpito
parecía un ángel con sotana, celestial mensajero para proclamar con
encantadora voz lo pecadores que somos. Por su elocuencia y talento, (no
por otras de sus eminentes cualidades, como la malignidad ha dicho
alguna vez) ganó en absoluto la confianza de doña Francisca, a quien
conoceremos en seguida.

---Diga usted a Sus Altezas que Su Majestad me ha llamado para pedirme
consejo en estas críticas circunstancias. En este momento Su Excelencia
el Sr.~Calomarde está en la cámara de Su Majestad, el cual\ldots{} Dios
lo quiere así\ldots{} continúa en malísimo estado, en deplorable
estado\ldots{} Cúmplase la voluntad del Altísimo.

Esto se decía en lujosa antecámara de esas que abundan en nuestros
palacios reales y que en su ornato y mueblaje ofrecían mezcla confusa
del estilo Luis XV y del gusto neo-clásico puesto en moda por el imperio
francés. La tapicería era rica y graciosa; el piso, cubierto de finísimo
junco, daba carácter español al recinto, y por el techo corrían entre
nubecillas semejantes a espuma de huevo batido, varias ninfas a lo Bayeu
que parecían representaciones de la retórica de Hermosilla y de la
poesía Moratiniana, según las baratijas simbólicas que cada una llevaba
en la mano para dar a conocer su empleo en el vasto reino del ideal. La
luz que alumbraba la pieza era escasa y apenas se distinguía un Carlos
IV en traje de caza que en la pared principal estaba, escopeta en mano,
la bondadosa boca contraída por la sonrisa, y con la vista un poco
extraviada hacia el techo, cual si intentara dar un susto a las ninfas
que por él se paseaban tranquilas sin meterse con nadie.

La hermosa figura del obispo y el elegante cuerpo negro del jesuita
concordaban admirablemente con aquel fondo o decoración palatina. Ambos
dijeron algunas palabras precipitadas que no pudimos oír y salieron a
prisa por distintas puertas. Seguiremos al jesuita guapo, quien
rápidamente nos llevó a otra monumental y vistosa sala donde salieron a
recibirle dos damas más notables por su rango que por su belleza. Eran
la infanta doña Francisca y la princesa de Beira, brasileñas y
ambiciosas. La primera habría sido hermosa si no afeara sus facciones el
tinte rojizo, comúnmente llamado color de hígado. La segunda llamaba la
atención por su arremangada nariz, su boca fruncida, su entrecejo
displicente, rasgos de los cuales resultaba un conjunto orgulloso y nada
simpático, como emblema del despotismo degenerado que se usaba por
aquellos tiempos.

El padre Carranza les habló con nerviosa precipitación, y ellas le
oyeron con la complacencia, mejor dicho, con la fe que el buen padre
Carranza les inspiraba, y en el ardiente y vivísimo coloquio, semejante
a un secreto de confesonario, se destacaban estas frases: «Dios lo
dispone así\ldots{} veremos lo que resulta de ese consejo\ldots{} ¿y qué
hará esa pobre Cristina?»

Los tres pasaron luego a la pieza inmediata, sólo ocupada en aquel
momento por un hombre, en el cual conviene que nos fijemos por ser de
estos individuos que, aun careciendo de todo mérito personal y también
de maldades y vicios, dejan a su paso por el mundo más memoria y un
rastro mayor que todos los virtuosos y los malvados todos de una
generación. Estaba sentado, apoyado el codo en el pupitre y la mejilla
en la palma de la mano, serio, meditabundo, parecido por causa del lugar
y las circunstancias a un grande emperador de cuyos planes y designios
depende la suerte de toda la tierra. Y la de España dependía entonces de
aquel hombre extraordinariamente pequeño para colocado en las alturas de
la monarquía. Tenía todas las cualidades de un buen padre de familia y
de un honrado vecino de cualquier villa o aldea; pero ni una sola de las
que son necesarias al oficio de Rey verdadero. Siendo, como era, rey de
pretensiones, y por lo tanto batallador, su nulidad se manifestaba más,
y no hubo momento en su vida, desde que empezó la reclamación armada de
sus derechos, en que aquella nulidad no saliese a relucir, ya en lo
político, ya en lo marcial. Era un genio negativo, o hablando
familiarmente, no valía para maldita de Dios la cosa.

Su Alteza se parecía poco al Rey Fernando. Su mirada turbia y sin brillo
no anunciaba, como en este, pasiones violentas, sino la tranquilidad del
hombre pasivo, cuyo destino es ser juguete de los acontecimientos. Era
su cara de esas que no tienen el don de hacer amigos, y si no fuera por
los derechos que llevaba en sí como un prestigio indiscutible emanado
del Cielo, no habrían sido muchos los secuaces de aquel hombre frío de
rostro, de mirar, de palabra, de afectos y de deseos, como no fuera el
vehemente prurito de reinar. Su boca era grande y menos fea que la de
Fernando, pues su labio no iba tan afuera; pero el gran desarrollo de su
mandíbula inferior, alargando considerablemente su cara, le hacía
desmerecer mucho. El tipo austriaco se revelaba en él más que el
borbónico, y bajo sus facciones reales se veía pasar confusa la
fisonomía de aquel espectro que se llamó Carlos II el Hechizado. A pesar
del lejano parentesco, la quijada era la misma, sólo que tenía más
carne.

Cuando entraron las infantas D. Carlos levantó los ojos de su pupitre,
miró con tristeza a las damas y después a un cuadro que frente a él
estaba y era la imagen de la Purísima Concepción. El Soberano de los
apostólicos dio un suspiro como los que daba D. Quijote en la presencia
ideal de Dulcinea del Toboso, y luego se quedó mirando un rato a la
pintura cual si mentalmente rezara.

---Francisquita---dijo al concluir,---no me traigas recados, como no
sean para darme cuenta de la enfermedad de mi adorado hermano. No quiero
intrigas palaciegas, ni menos conspiraciones para sublevar tropa,
paisanos o voluntarios realistas. Mis derechos son claros y vienen de
Dios: no necesitan más que su propia fuerza divina para triunfar, y aquí
están de más las espadas y bayonetas. No se ha de derramar sangre por
mí, ni es necesario tampoco. Yo no conquisto, tomo lo mío de manos de
Altísimo que me lo ha de dar. Esa, esa augusta señora---añadió señalando
el cuadro,---es la patrona de mi causa y la generalísima de nuestros
ejércitos: ella nos dará todo hecho sin necesidad de intrigas, ni de
sangre, ni de conspiraciones y atropellos.

Doña Francisca miró a la imagen bendita, y aunque era, como su ilustre
esposo, mujer de mucha devoción, no parecía fiar mucho, en aquellos
momentos, de la excelsa patrona y generalísima. La de Beira fue la
primera que tomó la palabra para decir a Su Alteza:

---Carlitos, no podemos estar mano sobre mano ni esperar los
acontecimientos con esa santa calma tuya, cuando se van a decidir las
cosas más graves. Nosotras no intrigamos, lo que hacemos es apercibirnos
para cortar las intrigas que se traman contra ti, legítimo heredero del
trono, y contra nosotras. No conspiramos; pero estamos a la mira de la
conspiración asquerosa de los liberales, que ahora se llamarán
cristinos, para burlar tus derechos, emanados de Dios, y alterar la ley
sagrada de la sucesión a la corona. En este momento, Cristina, por
encargo del Rey, llama a Consejo al ministro Calomarde, al obispo de
León y al conde de la Alcudia. ¿Sabes para qué?

---¿Para qué?

---Para proponer un arreglo, una componenda---dijo prontamente Doña
Francisca, no menos iracunda que su hermana.---Pronto lo sabremos. Esa
pobre Cristina apelará a todos los medios para embrollar las cosas y
ganar tiempo, hasta que se desencadenen las furias de la revolución, que
es su esperanza.

---¡Un arreglo!\ldots---dijo D. Carlos con entereza.---¿Con quién y de
qué? Entre los derechos legítimos, sagrados y la usurpación ilegal no
puede haber arreglo posible.

Dijo esto con tanto aplomo que parecía un sabio. Después miró a la
Virgen como para tener la satisfacción de ver que ella opinaba lo mismo.

---Basta de cuestiones políticas---dijo Su Alteza volviendo a tomar una
actitud tranquila.---¿Sigue Fernando más aliviado del paroxismo de esta
tarde?---Hasta ahora no hay síntomas de que se repita\ldots{}

---Pero puede suceder que de un momento a otro\ldots{}

---¡Pobre Fernando!---exclamó D. Carlos dando un gran suspiro y apoyando
la barba en el pecho. Incapaz de fingimiento y de mentira, la apariencia
tétrica del Infante era fiel expresión de la vivísima pena que sentía.
Amaba entrañablemente a su hermano. Para que todo fuera en desventaja de
los españoles, Dios quiso que estos se dividieran en bandos de
aborrecimiento, mientras los hermanos que ocasionaron tantos desastres
vivieron siempre enlazados por el afecto más leal y cariñoso.

Poco más de lo transcrito hablaron el Infante y las dos damas, porque
empezó a reunirse la camarilla en el salón inmediato, y Doña Francisca y
su hermana abandonaron a Don Carlos para recibir a los aduladores,
pretendientes y cofrades reverendos de aquella cortesana intriga. En
poco tiempo llenose la cámara de personajes diversos, el conde de Negri,
el padre Carranza, el embajador de Nápoles, vendido secretamente a los
apostólicos desde mucho antes, y D. Juan de Pipaón, que según todas las
apariencias, representaba en el seno de la comunidad apostólica a
Calomarde. Luego aparecieron el obispo de León y el conde de la Alcudia,
y entonces la cámara fue un hervidero de preguntas y comentarios.
Vanidad, servilismo, adulación, los rostros pálidos, las palabras
ansiosas, el respeto olvidado, el rencor no satisfecho, la esperanza
cohibida por el temor\ldots{} todo esto había bajo aquel techo habitado
por sosas ninfas, entre aquellos tapices representando borracheras a lo
Teniers, remilgadas pastoras o cabriolas de sátiros en los jardines de
Helicona.

---Una proposición inaudita, señores---dijo el reverendo obispo con
fiereza.---Veremos lo que opina el Señor. Ahí es nada\ldots{} Quieren
que durante la enfermedad del Rey se encargue del gobierno doña
Cristina, y que el Serenísimo Señor Infante sea\ldots{} su consejero.

Una exclamación de horror acogió estas palabras. La princesa de Beira
casi lloraba de rabia, y a la orgullosa Doña Francisca le temblaban los
labios y no podía hablar.

---Es una desvergüenza---se atrevió a decir Pipaón, que siempre quería
dejar atrás a todos en la expresión extremada del entusiasmo apostólico.

---Es una jugarreta napolitana---indicó Negri, que en estas ocasiones
gustaba de decir algo que hiciera reír.

---Es burlarse de los designios del Altísimo---afirmó Abarca, atento
siempre a entrometer la Divinidad en aquellas danzas.

---Es simplemente una tontería---dijo el de Alcudia.---Veamos la opinión
de Su Alteza.

El ministro y el obispo pasaron a ver a D. Carlos, que hasta entonces
tenía la digna costumbre de huir de los conventículos donde se
ventilaban entre aspavientos y lamentaciones los intereses de su causa,
y al poco rato salieron radiantes de gozo. Su Alteza había contestado
con enérgica negativa a la proposición de la \emph{madre de Isabelita};
que de este modo solían allí nombrar a la Reina Cristina.

Entonce los cortesanos corrieron del cuarto del Infante a la cámara
real, donde, en vista de la denegación, se buscaban nuevas fórmulas para
llegar al deseado arreglo. Hora y media pasó en ansiedades y locas
impaciencias. La Reina y los ministros conferenciaban en la antecámara
del Rey. En la alcoba de este nadie podía penetrar, a excepción de
Cristina, los médicos y los ayudas de cámara de Su Majestad. El Infante
no salía del rincón de su cuarto, en que parecía estar recogido como un
cenobita que hace penitencia; pero la bulliciosa Infanta, la implacable
princesa de Beira, su hijo D. Sebastián y la mujer de este no se daban
punto de reposo, inquiriendo, atisbando, en medio del vertiginoso ciclón
de cortesanos que iba y venía y volteaba con mareante susurro.

Al fin aparecieron el obispo y el conde de la Alcudia, trayendo las
nuevas proposiciones de arreglo. ¿Cuáles eran? «¡Una regencia compuesta
de Cristina y D. Carlos, con tal que este empeñase solemnemente su
palabra de no atentar a los derechos de la Princesa Isabel!» Tal era la
proposición que a unos parecía absurda, a otros insolente, a los más
ridícula. Hubo exclamaciones, monosílabos de desprecio y amargas risas.
«¡Los derechos de Isabelita!» Esta idea ponía fuera de sí a la enfática
y siempre hinchada princesa de Beira.

¿Y quién sabrá pintar la escena del cuarto de D. Carlos, cuando el
obispo y el ministro le comunicaron la última proposición de los Reyes?
Por todos los santos se puede jurar que el que tal escena vio no la
olvidará aunque mil años viva. Nosotros que la vimos la tenemos presente
lo mismo que si hubiera pasado ayer, ¿pero cómo acertar a pintarla? Es
tan rica de matices y al propio tiempo tan sencilla que es fácil se eche
a perder al pasar por las manos del arte. ¡Pasó allí tan poca cosa y fue
de tanta trascendencia lo que allí pasó!\ldots{} No hubo ruido; pero en
el silencio grave de aquella sala se engendraron las mayores tempestades
españolas del siglo.

Al ver entrar al obispo y al ministro, seguidos de las infantas, D.
Sebastián y el agraciadísimo Padre Carranza, D. Carlos se levantó
solemnemente. Era hombre que sabía dar a ciertos actos una majestad
severa que contrastaba con su llaneza en la vida privada. Mientras
Alcudia leía el borrador del decreto en que se establecía la doble
regencia, la princesa de Beira estaba lívida y Doña Francisca mordía las
puntas del pañuelo. Ambas hermanas vestían modestamente. ¿Quién olvidará
sus talles altos, sus ampulosos senos, sus peinados de tres lazos y sus
pañoletas de colores? Estaban como dos estatuas de la ambición
doméstico-palatina, erigidas en el centro del arco que formaba la
comisión de príncipes y magnates. Miraban ansiosas a D. Carlos cual si
temieran que el grande amor que al Rey tenía venciera su entereza en
aquel crítico instante, haciéndole incurrir en una debilidad que se
confundiría con la bajeza.

D.~Carlos no tenía talento ni ambición, pero tenía fe, una fe tan grande
en sus derechos que estos y los Santos Evangelios venían a ser para Su
Alteza Serenísima una misma cosa. Esta fe que en lo moral producía en él
la honradez más pura, y en los actos políticos una terquedad lamentable,
fue lo que en tal momento salvó la causa apostólica, llenando de júbilo
los corazones de aquellos señorones codiciosos y levantiscas princesas.
Mientras duró la lectura, D. Carlos no quitó los ojos del cuadro de la
Purísima, a quien sería mejor llamar Capitana por las prerrogativas
militares que el príncipe le había dado. Después hubo una pausa
silenciosa, durante la cual no se oyó más que el rumorcillo del papel al
ser doblado por el conde de la Alcudia. Las infantas miraban a los
labios de D. Carlos y D. Carlos se puso pálido, alzó la frente más ancha
que hermosa, y tosió ligeramente. Parecía que iba a decir las cosas más
estupendas de que es capaz la palabra humana, o a dictar leyes al mundo
como su homónimo el de Gante las dictaba desde un rincón del alcázar de
Toledo. Con voz campanuda dijo así:

---No ambiciono ser rey; antes por el contrario desearía librarme de
carga tan pesada que reconozco superior a mis fuerzas\ldots{}
pero\ldots{}

Aquí se detuvo buscando la frase. Doña Francisca estuvo a punto de
desmayarse y la de Beira echaba fuego por sus ojos.

---Pero Dios---añadió D. Carlos,---que me ha colocado en esta posición
me guiará en este valle de lágrimas\ldots{} Dios me permitirá cumplir
tan alta empresa.

Aún no se sabía qué empresa era aquella que Dios, protector decidido de
la causa, tomaba a su cargo en este valle de lágrimas. El conde de la
Alcudia que a pesar de estar secretamente afiliado al partido de D.
Carlos, quería cumplir la misión que le había dado el Rey, dijo algunas
palabras en pro de la avenencia. Pero entonces don Carlos, como si
recibiera una inspiración del Cielo, habló con facilidad y energía en
estos términos, que son exactos y textuales:

---«No estoy engañado, no, pues sé muy bien que si yo por cualquier
motivo, cediese esta corona a quien no tiene derecho a ella, me tomaría
Dios estrechísima cuenta en el otro mundo y mi confesor en este no me lo
perdonaría; y esta cuenta sería aún más estrecha perjudicando yo a
tantos otros y siendo yo causa de todo lo que resultare; por tanto no
hay que cansarse, pues no mudo de parecer».

Dijo y se sentó cansado. Las infantas dejaron a sus abanicos la
expresión del orgullo y satisfacción que sentían por aquellas
cristianísimas palabras. ¿Qué cosa más admirable que un príncipe
decidido a reinar sobre nosotros, no por ambición, no por deseo de
aplicar al Gobierno un entendimiento que se siente poderoso, sino por
cristianismo puro, por temor de Dios y por miedo al Infierno? En aquel
breve discurso nos explicó Su Alteza Serenísima la clave de sus ideas y
de su modo de hacer la guerra y de gobernar. No era ambicioso ni
conquistador, sino una especie de cruzado de la Tierra Santa de sus
derechos. Según él, Dios estaba profundamente interesado en aquel
negocio, y tanto, que no se sabe lo que habría pasado en los reinos
celestiales si al buen Infante le da la mala tentación de dejar reinar a
\emph{Isabelita}. Es sabido que estas contiendas de familia se miran
allá arriba como cosa de casa. Bien enterado estaba de todo el confesor
de Su Alteza, que así le había pintado la imposibilidad de ser modesto y
la urgente precisión de ceñirse la corona por estar así acordado allí
donde se hacen y deshacen los imperios. ¿Y cómo se iba a atrever el
pobre D. Carlos a confesar en el temeroso tribunal de la penitencia el
horrible delito de no querer ser Rey? ¿Y además no estaba de por medio
la infeliz España a quien Dios no podía abandonar? ¿Y qué era el
príncipe más que el instrumento de Dios, protector decidido en todos
tiempos de nuestra nación con preferencia a todas las demás que ocupan
la interesante Europa, la América lozana, la negra África y el Asia
opulenta? ¡Instrumento de la Providencia! Esto y no otra cosa era D.
Carlos, y bien lo comprendía así el bueno, el evangélico, el seráfico
obispo de León, cuando al salir de la cámara del Infante se abrió paso
entre la multitud de cortesanos, diciendo con entusiasmo:

---¡Paso al partido del Altísimo!

Olvidábamos decir que D. Carlos, luego que dio aquella respuesta digna
de un arcángel, encargado de defender una plaza del Cielo sitiada por
los pícaros demonios, habló un rato con sus amigos y con su esposa y
cuñada, repitiéndoles lo que ya les había dicho muchas veces, a saber:
que se negaba resueltamente a apelar a las armas, que desaprobaba todas
las conspiraciones fraguadas en su nombre y que se le enterase cada poco
rato del estado de la salud del Rey.

Luego se encerró en su oratorio donde rezó gran parte de la noche,
pidiendo a Dios, su superior jerárquico, y a la Limpia y Pura, su
generala en jefe, que salvaran la vida de su amado hermano Fernando. Tal
era, ni más ni menos, aquel D.~Carlos que en España ha llenado el siglo
con su nombre lúgubre, monstruo de candor y de fanatismo, de honradez y
de ineptitud.

\hypertarget{xxxii}{%
\chapter{XXXII}\label{xxxii}}

Todos los manipuladores de aquella intriga se agitaban mucho, pero
ninguno como Pipaón, el correveidile de Calomarde, el que tan pronto
llevaba un recado al embajador de Nápoles, caballero Antonini, como un
papelito al Padre Carranza para que lo diera a las infantas. Cuando el
barullo cesó en los salones y empezó a reinar un poco de sosiego, el
bueno de Bragas retirose con Calomarde y Carranza a una pieza lejana
donde estuvieron charlando acaloradamente y revolviendo papeles y
haciendo números hasta por la mañana. Cuando amaneció tenía la augusta
cabeza tan caldeada por el hervir de ideas y proyectos que en aquella
cavidad bullían, que juzgó prudente no acostarse y salir a los jardines
para dar algunas vueltas.

Largo rato estuvo recorriendo alamedas y bosquecillos de tallado mirto,
pero sin parar mientes en la hermosura de la Naturaleza en tal hora,
porque su ambición ocupaba al cortesano todas las potencias y sentidos.
Así la deliciosa frescura de la mañana, el despertar de los pajarillos,
la quietud soñolienta de la atmósfera, la gala de las flores humedecidas
por el rocío, eran para aquel infeliz esclavo de las pasiones, como
páginas de un idioma desconocido, del cual no comprendía ni una letra ni
un rasgo.

Ciego para todo menos para su loco apetito no veía sino la cartera
ministerial, el sueldazo, las obvenciones, las veneras, el título de
nobleza y todo lo demás que del próximo triunfo de los apostólicos podía
obtener.

Junto a la fuente de Pomona tropezó con D. Benigno Cordero, que volvía
de su paseo matinal. Era hombre que madrugaba como los pájaros y daba
paseos de leguas antes del desayuno. Aquella mañana el héroe estaba tan
meditabundo como Pipaón; pero por diferentes motivos.

---No he dormido en toda la noche, señor Don Benigno---dijo el cortesano
con énfasis.---Hemos trabajado para evitar derramamiento de sangre. El
Rey se nos muere hoy: no llegará a la noche. ¡España por D. Carlos!

---Yo tampoco he dormido, pero no me desvelan a mí esas trapisondas
palaciegas, no---repuso el héroe melancólicamente.---Barástolis,
rebarástolis\ldots{} ¡pensar que hasta ahora no he podido conseguir de
ese intrigante la cosa más fácil y sencilla que se puede pedir a un
obispo!\ldots{} ¡una firma, una, D. Juan, una firma! He prometido una
gran cesta de albaricoques, amén de otras cosas, al familiar de Su
Ilustrísima y\ldots{} ni por esas\ldots{} Su Ilustrísima no se puede
ocupar de eso, Su Ilustrísima se debe al Rey y al Estado y al\ldots{}
¿En qué país vivimos? ¿Pues así se tratan los intereses más respetables?
¿Es esto ser obispo?\ldots{} ¡Le digo a usted, amigo D. Juan, que estoy
de obispos hasta la corona!\ldots{} ¿Qué es lo que pido? Una firma, nada
más que una firma en documento corriente, informado y vuelto a informar,
y que ha pasado por más manos que moneda vieja\ldots{} ¡Oh!, malhadada
España. ¡Y estos hombres hablan de regenerarte!

¡Una firma, nada más que una firma! Indudablemente el revoltoso obispo
debía ser ahorcado. Pipaón consoló a su amigo lo mejor que pudo
prometiéndole recomendar el caso a Su Ilustrísima, y conseguirle si
triunfaban los apostólicos, no una firma, sino cuatro o cinco docenas de
ellas.

Cuatro o cinco docenas de \emph{Barástolis} echó después de su boca D.
Benigno, y juntos él y Bragas se dirigieron hacia la casa de Pajes.

---Si estuviera aquí Jenarita---decía Cordero,---ella con su
irresistible poder haría firmar a ese condenado.

Pipaón se acostó; pero llamado a poco rato por Su Excelencia, tuvo que
dejar el blando sueño para acudir a los cónclaves que se preparaban para
aquel día. El inconsolable y aburridísimo Cordero, luego que se
desayunó, volvió a los jardines, único punto donde hallaba algún
esparcimiento en su tristeza, y no había llegado aún a la fuente de la
Fama, cuando topó con Salvador Monsalud que de palacio venía cabizbajo y
de malísimo humor. El día anterior se habían visto y saludado un momento
como amigos antiguos que eran desde las trapisondas de la Milicia
nacional del año 22, memorable por la hazaña del nunca bastante célebre
arco de Boteros. D. Benigno se alegró de verle, por tener alguien con
quien hablar en aquella desolada corte, tan llena de interés para otros
y para él más triste y solitaria que un desierto. De manos a boca
Monsalud le habló de Sola, del casamiento, y tales elogios hizo de ella
y con tanto calor la nombró, que Cordero sintió inexplicables
inquietudes en su alma generosa. No sabía por qué le era desagradable la
persona y la amistad de aquel hombre, protector y amigo de su futura en
otro tiempo, y luego nombrado en sueños por ella. Recordó claramente
cuán triste se ponía Sola si le faltaban cartas de él, y cuánto se
alegraba al recibir noticias suyas; pero al mismo tiempo le consoló el
recuerdo de la perfecta sinceridad, signo de pureza de conciencia, con
que Sola le supo referir su entrevista con Salvador en los Cigarrales,
mientras Cordero estaba en Madrid ocupado de los nunca bastante
vituperados papeles. Recordó muchas cosas, unas que le agitaban, otras
que calmaban su inquietud, y por último la fe ciega que tenía en el
afecto puro y sencillo de la que iba a ser su señora le confortaba
singularmente. No obstante, quiso evitar la compañía de aquel hombre, y
ya preparaba la conversación para buscar un pretexto de ausencia, cuando
Salvador dijo:

---Reniego de esta cansada y revoltosa corte. Aquí estoy hace seis días
atado por una pretensión fácil y sencilla, y aunque tengo relaciones en
palacio, nada puedo conseguir. A usted no le sorprenderá el saber que lo
que pretendo no es más que una firma, nada más que una firma en
documento corriente. Pero el señor Calomarde que para daño eterno de
nuestro país, sigue sin reventar todavía, no se ha decidido aún a tomar
la pluma. ¡Y de que la tome y rubrique dependen mi fortuna y mi
porvenir!

---Nuestra cuita es la misma---exclamó Don Benigno sintiéndose consolado
con la desgracia ajena.---Yo también me aburro y me desespero y me quemo
la sangre sólo por una firma.

---¡Qué ministros!

---Están intrigando para arrancar al Rey un codicilo que dé la corona a
D. Carlos.

---¡Qué menguados hombres!\ldots{} ¡Que una nación esté en tales
manos\ldots!

---Y según los vientos que corren, barástolis, lo estará para \emph{in
eternum}. La consigna de esa gente es que el Rey se muere hoy. Parece
que han sobornado al Altísimo.

---Es gracioso.

---Ya tratan a D. Carlos de Majestad.

---Lo creo. Será Rey. Vamos progresando. ¿Piensa usted emigrar?

---¿Yo?---dijo Cordero sorprendido.---Si triunfa ese partido brutal lo
sentiré mucho, porque en fin, tengo ideas liberales\ldots{} algo ha
leído uno en autores filosóficos\ldots{}

---Sí, ya sé que lee usted a Rousseau. Rousseau dice: «no hay patria
donde no hay libertad». ¿Piensa usted emigrar?

---Emigrar no, porque no me mezclo en política. Viviré retirado de estos
trapicheos dejándoles que destrocen a su antojo lo que todavía se llama
España, y con ellos se llamará como Dios quiera. Un padre de familia no
debe comprometerse en aventuras peligrosas. Usted\ldots{}

---Yo no soy padre de familia ni cosa que lo valga---dijo el otro
dejando traslucir claramente una pena muy viva.---No tengo a nadie en el
mundo. No hay casa, ni hogar, ni rincón que guarden un poco de calor
para mí; soy tan extranjero aquí como en Francia; soy esclavo de la
tristeza; no tengo en derredor mío ningún elemento de vida pacífica; la
última ilusión la perdí radicalmente; vivo en el vacío; no tengo, pues,
otro remedio, si he de seguir existiendo, que lanzarme otra vez a las
aventuras desconocidas, a los caminos peligrosos de la idea política,
cuyo término se ignora. Mi antigua vocación de revolucionario y
conspirador, que estaba amortiguada y como vencida en mí, vuelve a nacer
ahora, porque el freno que le puse se ha roto, porque la vocación nueva
con que traté de matar aquella se ha convertido en humo. Hay que volver
al humo pasado, a las locuras, a la lucha, a las ideas, cuya
realización, por lo difícil, toca los límites de lo imposible.

D.~Benigno le oía con estupor. Habíanse internado en uno de aquellos
laberintos hechos con tijeras, que parecen decoraciones teatrales
construidas para una sosa comedia galante o para una opereta de
Metastasio. Solitarias y placenteras estaban las callejuelas y las
bovedillas verdes. Nadie podía oírles allí. Salvador no puso trabas a su
lengua y se expresó de este modo:

---Cuando vine aquí persistía en mi propósito de huir para siempre de la
política, aunque estaba muy indeciso considerando que alguna dirección o
empleo había de dar a mi pensamiento y a mi voluntad. No se puede vivir
de monólogos, como yo vivo ahora. Mi desgracia o mi fortuna, que esto no
lo sé bien, quisieron que entrara algunas veces en Palacio. Allí traté a
gentiles-hombres y cortesanos, hice amistad con ministriles y
empleadillos menudos; todo por el negocio maldito de esta rúbrica que
pido a Su Excelencia y que no me quiere dar. Además soy amigo de un
montero de Espinosa que me ha enterado de todo lo ocurrido ayer y
anoche. ¡Qué cosas, amigo mío; qué horrores! Si cuando se lee la
historia sentimos emociones tan hondas y queremos ser actores en los
sucesos pintados, ¿qué será cuando vemos la historia viva, antes de ser
libro, y asistimos a los hechos antes de que sean páginas? El drama de
anoche me ha espeluznado. Pues se prepara otro drama, junto al cual el
de anoche será comedia. No, no es posible ver esto como se ven por
anteojo los muñecos y las vistas de un \emph{tutilimundi}. De repente me
he sentido exaltado, y mis antiguas vocaciones han renacido con ímpetu
irresistible.

---Cuidado, cuidado---dijo D. Benigno, temeroso del sesgo peligroso que
aquella conversación tomaba.---Los arbolitos oyen; chitón. Le veo a
usted en camino de ser un cristino furibundo.

---Yo no sé por qué camino voy; sólo sé que cuando veo a esa Reina
joven, hermosa, inocente de todos los crímenes del absolutismo: cuando
considero sus virtudes y la piedad con que asiste al Rey enfermo, que
sólo merece lástima; cuando veo los peligros que la cercan, los infames
lazos que se le tienden y el desdén con que la miran los mismos que hace
poco se arrastraban a sus pies, siento arder la sangre en mis venas, y
no sé qué daría, créame usted, D. Benigno, por hallarme en situación de
enseñar a esos murciélagos apostólicos cómo se respeta a una señora y a
una Reina. En la corona que no han podido quitarle todavía, y que sobre
su hermosa frente tiene mayor brillo, veo la monarquía templada que
celebra alianzas de amistad con el pueblo; pero en la corona de hierro
que esos intrigantes clérigos y cortesanos están forjando en el cuarto
de D. Carlos, veo la monarquía desconfiada, implacable, que no admite
más derechos que los suyos. No, no hay ya en España caballeros, si
España consiente que esa turba de fanáticos expulse a la Reina y
arrebate la corona a su hija\ldots{}

---Sí, sí---exclamó Cordero sintiendo que revivía lentamente en su pecho
su antiguo entusiasmo liberalesco.---Pero cuidado, mucho cuidado, amigo.
Lo que usted dice es peligrosísimo. Todo el Real Sitio es de los
apostólicos. No nos metamos en lo que no nos importa.

---¿Cómo que no nos importa?---dijo el otro con viveza.---Es cuestión de
vida o muerte, de ser o no ser. En estos momentos se está decidiendo, y
pronto se probará si los españoles no merecen otro destino que el de un
hato de carneros o si son dignos de llamar nación a la tierra en que
viven. Yo que había tomado en aborrecimiento las revoluciones y el
conspirar, ahora siento en mí un apetito de rebeldía que me llevaría a
los mayores atrevimientos si viera junto a mí quien me ayudase.
Desanimado ayer y deseoso de la oscuridad, hoy que la vida doméstica me
es negada por Dios, quisiera tener medios de revolver a España, y
amotinar gente, y hacer que todo el mundo se rebelara, y romper todos
los lazos, y levantar todos los destierros, y desencadenar todo lo que
está encadenado por este régimen brutal. Yo iría a esa Reina atribulada
y le diría: «Señora, lance Vuestra Majestad un grito, un grito sólo en
medio de este país que parece dormido y no está sino asustado. No tema
Vuestra Majestad; estas situaciones se vencen con el valor y la
confianza. Abra Vuestra Majestad las puertas de la patria a todos los
emigrados, a todos absolutamente sin distinción. Para vencer al Infante
se necesita una bandera; para hacer frente a un principio se necesita
otro; nada de términos medios, ni acomodos vergonzosos; esa gente pide
todo o nada; pues nada y guerra a muerte. Levántese Vuestra Majestad y
ande con paso seguro; no se deje asustar por los errores de los que no
han sabido establecer la libertad. Es preciso tolerarles como son,
porque son la salvación, y si aseguran el trono y la libertad sus
imperfecciones y extravíos les serán perdonados. Y entonces, señora, se
alzará del seno de la nación oprimida y deseosa de mejor suerte, un
sentimiento, un prurito incontrastable, y miles de hombres generosos se
agruparán al lado de Vuestra Majestad protestando con la palabra y con
la espada de que quieren por soberana a la Reina del porvenir, la Reina
liberal, Isabel II».

\hypertarget{xxxiii}{%
\chapter{XXXIII}\label{xxxiii}}

---¡Chitón, chitón por todos los santos del cielo!---dijo D. Benigno
poniéndole la mano en la boca para hacerle callar.

El héroe participaba de aquel noble ardor, pero temía que tales
demostraciones les trajeran a ambos algún perjuicio. Tembloroso y
ruborizado, Cordero llevó a su amigo fuera del verde laberinto,
incitándole a que callara, porque---y lo dijo en la plenitud de la
convicción,---si el obispo Abarca y el ministro Calomarde llegaban a
tener noticia de lo que se habló en los jardines, no firmarían ni en
tres siglos. Salvador tranquilizó al buen comerciante sobre aquel
endiablado negocio de las firmas y cuando se separaron invitole a que
comieran juntos aquella tarde. Excusose D. Benigno, por sentirse, al oír
la invitación, tocado de aquel mismo recelo o inquietud de que antes
hablamos; pero las reiteradas cortesanías del otro le vencieron al fin.
Mientras Cordero entraba en la casa de Pajes pensando en el convite, en
la muerte del Rey, en la firma y sobre todo en los que le esperaban en
los Cigarrales, Salvador penetró en Palacio y no se le vio más en todo
el día.

Era aquel el 18 de Setiembre, día inolvidable en los anales de la guerra
civil, porque, si bien en él no se disparó un solo cartucho, fue un día
que engendró sangrientas batallas, un día en el cual se puede decir
figuradamente que se cargaron todos los cañones. Desde muy temprano
volvió a reinar el desasosiego en los salones y en todas las
dependencias. Su Majestad seguía muy grave, y a cada vahído del monarca
la causa apostólica daba un salto en señal de vida y buena salud; así es
que cuando circulaban noticias desconsoladoras no se veía el dolor
pintado en todas las caras, como sucede en ocasiones de esta naturaleza,
aun en reales palacios, sino que a muchos les bailaban los ojos de
contento, y otros aunque disimulaban el gozo, no lo hacían tanto que
escondieran por completo la repugnante ansiedad de sus corazones
corrompidos.

En medio de esta barahúnda, la Reina apuraba ella sola en el silencio
lúgubre de la alcoba regia el cáliz amargo de la situación más triste y
desairada en que pueda verse quien ha llevado una corona. Los cortesanos
huían de ella; a cada hora, a cada minuto veía disminuir el número de
los que parecían fieles a su causa, y cada suspiro del Rey moribundo
producía una defección en el débil partido de la Reina. El día anterior
aún tenía confianza en la guardia de Palacio; pero desde la mañana del
18 las revelaciones de algunos servidores leales la advirtieron de que,
muerto el Rey, la guardia y probablemente todas las fuerzas del Real
Sitio abrazarían el partido del Infante.

Cristina se vistió en aquellos días el hábito de la Virgen del Carmen, y
con la saya de lana blanca estaba más guapa aún que con manto regio y
corona de diamantes. No salía de la alcoba regia sino breves momentos,
cuando el Rey parecía sosegado y ella necesitaba ver a sus hijas o
desahogar su pena en amargas lágrimas, derramadas sin testigos en su
cámara particular. Allí también había bullicio y movimiento, porque la
servidumbre arreglaba las maletas y embaulaba el ajuar de la Reina en
previsión de una fuga precipitada.

Por la noche la Reina no dormía tampoco. Sentada junto al lecho del Rey,
vigilaba su enfermedad, atendía a sus dolores, preparaba por sí misma
las medicinas y se las daba, le dirigía palabras de esperanza y
consuelo, no permitía que los criados hicieran cosa alguna que pudiera
hacer ella, esclava entonces de sus deberes de esposa con tanto rigor
como la compañera del último súbdito del tirano enfermo. Haciendo
entonces lo que no suelen ni saben hacer generalmente las reinas,
aquella joven se puso una corona de esas que no están sujetas a los
azares de un destronamiento ni a los desaires de la abdicación.

La historia no dice lo que pasó por la mente del atormentador de España
al ver que en pago de sus violencias, de su bárbaro orgullo, de sus
vicios y de su egoísmo brutal, Dios le enviaba aquel ángel en su última
hora para que el autor de tantas agonías viera endulzada la suya y
pudiera morirse en paz, como se mueren los que no han hecho daño a
nadie. Cuando se entraba en la alcoba real no se podía ver sin horror el
enorme cuerpo del Rey en el lecho, hinchado, sin movimiento, oprimido
por bizmas, ungido con emplastos que a pesar de sus virtudes no vencían
los dolores; hecho todo una miseria; conjunto lastimoso de desdichas
físicas, que así remedaban la moral más perversa que ha informado un
alma humana.

Su rostro variaba entre el verdoso de la muerte y el amoratado de la
congestión. Ligeramente incorporado sobre las almohadas su cabeza estaba
inmóvil, su mirada fija y mortecina, su nariz colgaba cual si quisiera
caer saltando al suelo, y de su entreabierta boca no salía sino un
quejido constante que en los breves momentos de sosiego era estertor
difícil. Por fin le tocaba a él también un poco de potro. Debía de estar
su conciencia bastante despierta en aquellos momentos, porque no se
quejaba desesperado, como si en el fondo de su alma existiese una
aprobación de aquel horrible quebrantamiento de huesos y hervor de
sangre que sufría. La cama del Rey por el estado de aquel desdichado
cuerpo que desde algún tiempo vivía corrompiéndose, parecía más bien un
ensayo de las descomposiciones del sepulcro. Esto sólo es un elocuente
elogio de la cristiana abnegación de la Reina.

En la alcoba había dos o tres crucifijos e imágenes, todos solicitados
por la piedad de Cristina para que no permitieran que España se quedase
sin Rey. Mas por el momento no había síntomas de que tan noble anhelo
fuera atendido, porque Fernando VII se moría a pedazos. Aquella masa
inerte, tan sólo vivificada por un gemido, no era ya Rey ni siquiera
hombre. Hacia el medio día se temió la pérdida absoluta de las
facultades mentales y antes que esto llegara, se reconoció la necesidad
de dar solución al problema tremendo. Una chispa de razón quedaba en el
espíritu del Rey. Era urgente, indispensable, que a la débil luz de esa
chispa se resolviese el conflicto.

Cristina hubiera dilatado aquel momento. Ganando algunas horas habría
podido llegar su hermana la Infanta Doña Carlota, mujer de mucho brío y
resolución que para aquel caso era de perlas. Desde que se agravó Su
Majestad le habían enviado correos al Puerto de Santa María, rogándola
que viniese, y ya la Infanta debía de estar cerca, quizás en Madrid,
quizás en camino del Real Sitio. Pero el aniquilamiento rápido del
enfermo no permitía esperar más. Entraron, pues, en la real cámara tres
figuras horrendas: Calomarde, el de Alcudia y el obispo de León. La
Reina y el confesor del Rey habían llegado poco antes y estaban a un
lado y otro de Su Majestad, Cristina casi tocando su cabeza, el clérigo
bastante cerca para hablar al oído del pobre enfermo. Había llegado un
momento en que ninguna alma cristiana podía conservar rencor ante tanta
desdicha. No era posible ver a Fernando VII en aquel trance sin sentir
ganas de perdonarle de todo corazón.

Los tres temerosos figurones se situaron por los pies de la cama.
Después que uno tras otro besaron con apariencia cariñosa aquella mano
lívida, que había firmado tantas atrocidades, se sentaron por los pies
del lecho. El obispo estaba grave e impotente como quien, suponiéndose
con autoridad divina, se cree por encima de todas las miserias humanas;
el conde de la Alcudia estaba triste y acobardado por la solemnidad del
momento, y Calomarde, el hombre rastrero y vil, cuya existencia y cuyo
gobierno no fueron más que pura bajeza e hipocresía, arqueaba las cejas
mucho más que las arqueaba de ordinario, pestañeaba sin cesar y hacía
pucheros. Cruel con los débiles, servil con los poderosos, cobarde
siempre, este hombre abominable adornaba con una lagrimilla la traición
infame que hacía a su amo al borde del sepulcro.

Quien presenció aquella escena terrible cuenta que la luz de la estancia
era escasa; que los tres consejeros estaban casi en la sombra; que el
Rey volvía su rostro hacia la Reina vestida de hábito blanco; que hubo
un momento en que el confesor no hacía más que morderse las uñas; que la
hermosura de Cristina era la única luz de aquel cuadro sombrío, intriga
política, horrible fraude, traidor escamoteo de una corona perpetrado en
el fondo de un sepulcro.

Cuenta también el testigo presencial de aquella escena que el primero
que habló, y habló con entereza, fue el obispo de León. Se puso de pie y
parecía que llegaba al techo. Su voz hueca de sochantre retumbaba en la
cámara como voz de ultratumba. Aquel hombre tan rígido como astuto
principió tocando una delicada fibra del corazón del Rey; habló de
\emph{las inocentes} niñas de Su Majestad y de la \emph{virtuosa Reina},
que según él corrían gran peligro si no pasaba la corona a las sienes de
Don Carlos. Después pintó el estado del reino, en el cual, según dijo,
no había un solo hombre que no fuera partidario de la monarquía
eclesiástica representada por el Infante.

Fernando dio un gran suspiro y fijó sus aterrados ojos en el obispo.
Este se sentó. Puesto en pie Calomarde dijo que su emoción al ver en
aquel estado al mejor de los Reyes y al mejor de los padres, y al mejor
de los esposos, y al mejor de los hombres no le permitía hablar con
serenidad; dijo que se veía en la durísima precisión de no ocultar a su
amado soberano la verdad de lo que ocurría; que había tanteado el
ejército, y todo el ejército se pronunciaría por D.~Carlos si no se
modificaba en favor de este la Pragmática sanción del 29 de Marzo de
1830; que los voluntarios realistas, sin excepción de uno solo,
proclamaban ya abiertamente como Rey de derecho divino al mismo Sr.~D.
Carlos, y que para evitar una lucha inútil y el derramamiento de sangre
convenía a los intereses del reino\ldots{}

El infame hacía tantos pucheros que no pudo continuar la frase. Sintiose
que el cuerpo dolorido del Rey se estremecía en su lecho o potro de
angustia. Oyose luego la voz moribunda que dijo entre dos lamentos:

---Cúmplase la voluntad de Dios.

El confesor silbó en su oído palabras no entendidas por los demás, y
entonces la Reina Cristina, sin mirar a las tres sombras, volviendo su
rostro al Rey y haciendo un heroico esfuerzo para no dar a conocer su
dolor, pronunció estas palabras:

---Que España sea feliz, que en España haya paz.

El Rey exhaló un gran suspiro, mirando al techo, y después dijo algo que
pareció el mugido de un león enfermo. La Reina tomó su pañuelo y sin
decir nada, dejando correr libremente sus lágrimas, limpió el sudor
abundante que bañaba la frente del Rey.

Siguió a esto un discursillo del conde de la Alcudia confirmando el
dictamen de los otros dos apostólicos. Aquel famoso triunvirato traía la
comedia bien aprendida, y en el cuarto de D. Carlos se habían estudiado
antes detenidamente los discursos, pesando cada palabra. El confesor
dijo también en voz alta su opinión, asegurando bajo su palabra que el
Altísimo estaba en un todo conforme con lo expuesto por los
respetabilísimos señores allí presentes. ¡Se quedó tan satisfecho
después de este mensaje\ldots!

El Rey pareció llamar a sí todas sus fuerzas. Claramente dijo:

---¿En qué forma se ha de hacer?

No vacilaron los apostólicos en la contestación, pues para todo estaban
prevenidos. Calomarde fingiendo que se le ocurría en aquel mismo
instante, propuso que el Rey otorgase un codicilo-decreto derogando la
Pragmática sanción del 30, y revocando las disposiciones testamentarias
en la parte referente a la regencia y a la sucesión de la corona.

Después de una pausa el Rey se hizo repetir la proposición del ministro,
y oída por segunda vez, Cristina volvió a limpiar el sudor que corría
por la frente de su marido. Con un gesto y la mano derecha este mandó a
los tres apostólicos consejeros que salieran de la estancia y se quedó
sólo con su esposa y con su confesor, el cual salió también poco
después. Consternados los tres escamoteadores y dudando del éxito de su
infame comedia, no decían una palabra, y con los ojos se comunicaban
aquella duda y el temor que sentían. Calomarde y el obispo dieron
algunos paseos lentamente por la cámara, esperando que el Rey les
volviera a llamar, y el conde de la Alcudia aplicó el oído a la puerta y
dijo en voz baja y temerosa:

---Parece que llora Su Majestad.

---No lo creo---murmuró el obispo acercando también su oído.

Entonces se abrió la puerta y apareció el confesor con las manos
cruzadas y el semblante compungido, imagen exacta de la hipocresía. Los
cuatro cuchichearon un momento como viejas chismosas. Media hora después
Cristina les llamó y volvieron a entrar. Fernando no estaba ya
incorporado en su cama sino completamente tendido de largo a largo,
fijos los ojos en el techo, rígido, pesado, el resuello lento y difícil.
Sin mirar a los que habían sido sus amigos, sus aduladores, terceros de
sus caprichos políticos y servidores de sus gustos con la lealtad y
sumisión del perro, Fernando VII les manifestó en pocas palabras que
aceptaba el sacrificio que se le imponía. Esforzándose un poco, habló
más para exigir secreto absoluto de lo acordado hasta que él muriese.

Los tres apostólicos bajaron; encerráronse en un gabinete. Entre tanto,
la chusma del cuarto de D. Carlos ardía en impaciencias; las dos
infantas estaban tan nerviosas, que no podía ser más. La historia, que
es muy descuidada en ciertas cosas, no dice el número de tazas de tila
que se consumieron aquel día. El obispo, Calomarde y Alcudia se
mostraron tan reservados aquella tarde, que los \emph{carlinos} se
impacientaban y aturdían cada vez más. No obstante, algunas palabras
optimistas, aunque enigmáticas, de Abarca al salir del gabinete en que
los tres se encerraron para extender el decreto-codicilo, hicieron
comprender a la muchedumbre apostólica que las cosas iban por buen
camino. Finalmente, al llegar la noche, y cuando se difundía por
Palacio, corriendo y repercutiéndose de sala en sala como un trueno, la
voz de el \emph{Rey ha muerto}, el señor Abarca entró triunfante en la
cámara donde la corte del porvenir estaba reunida. En su mano alzaba el
reverendo un papel, con el cual parecía amenazar, o que lo tremolaba
como estandarte donde estuviera escrita una ley suprema. Moisés bajando
del Sinaí no estaba seguramente más terrible que el señor Abarca cuando,
mostrando el decreto-codicilo, exclamó:

---Señores, óiganme.

Oyeron leer con atención profunda y poco faltó para que algunos se
prosternaran, quién por servilismo mezclado de entusiasmo, quién por ese
especial y no bien comprendido instinto a lo Nabucodonosor que algunos
entes civilizados no pueden ocultar aunque vistan casaca bordada. Toda
la corte de D. Carlos estaba allí, menos D. Carlos, el candidato divino,
que a tal hora se hallaba en su oratorio con la frente humillada y el
corazón oprimido, pidiendo a Dios que no quitara la vida a su hermano.

\hypertarget{xxxiv}{%
\chapter{XXXIV}\label{xxxiv}}

Al llegar aquí, el narrador no puede contener el asombro que le produce
el peregrino suceso que va a referir, y deteniendo su relato, exclama:
¡Oh admirables designios de la Providencia!, ¡oh vanidad de los cálculos
humanos!, ¡oh peligro de jugar con las cosas del Cielo, eslabonándolas
con los apetitos e intereses de un bando político! De este modo el ánimo
del lector queda perfectamente dispuesto para saber que Dios
Todopoderoso, que sin duda tenía a D. Carlos en más estimación que al
partido apostólico, atendió al ruego que con amor fraternal y piedad
cristiana le dirigió este; y así dispuso que Fernando, ya casi muerto,
tornase a la vida, dando al traste con las esperanzas de lo que el
obispo de León llamaba \emph{el partido del Altísimo}. De este modo el
Padre de todas las cosas abandonaba a su grey en lo mejor de la pelea,
seguido de la Generalísima, a quien también pidió muy ardientemente D.
Carlos la vida de su hermano. Hasta con su cristiandad se perjudicaba a
sí mismo D. Carlos como jefe visible del partido absolutista-religioso,
y si lo dejaran rezar mucho, es fácil que los furibundos apostólicos
perdieran todas las batallas cortesanas y marciales que en lo futuro
habían de dar.

Fernando se aletargó por la noche. Todos le creyeron muerto y la
tremenda noticia circuló por el Real Sitio, llegó hasta Madrid y aun fue
trasmitida a las Cortes europeas. Pero a la mañana siguiente, de aquel
cadáver volvieron a salir quejas y suspiros, se reanimó con oportunas
sustancias y medicinas, y en Palacio y en los jardines no se decía sino
\emph{el Rey vive}, \emph{el Rey vive}; frase de consternación para
algunos, de esperanzas para los menos. Muchas caras variaron
completamente, y Cristina vio sonreír a los que el día anterior estaban
cejijuntos y tenían en su rostro protervo el indefinible airecillo de la
defección. ¡Y el señor obispo que la tarde del 18 salía a los jardines
diciendo en voz alta en un corro de amigos: «Ya no volverán a levantar
la cabeza los liberales»!\ldots{} ¡Y el gracioso Padre Carranza que
aquella noche había prometido solemnemente a sus allegados más de
cuarenta canonjías y beneficios simples!

En todo el día 19 fueron llegando al Real Sitio muchos jóvenes de la
aristocracia y militares de todas graduaciones, que iban a ponerse a las
órdenes de la Reina Cristina. Con estas adquisiciones hechas por un
partido que se creía muerto, iban rápidamente abatiéndose los ánimos de
los apostólicos y no se sabe qué cantidad fabulosa de tazas de tila
tuvieron que tomar Doña Francisca y su hermana para poner a raya sus
desconcertados nervios. ¡Dios y la Generalísima ayudaban a la
napolitana!

Con la irrupción de personajes civiles y militares en el Real Sitio, las
habitaciones escasearon en tales términos que Pipaón tuvo que rogar a D.
Benigno le dejase libre el cuarto que ocupaba en la casa de Pajes, lo
que no sintió mucho el héroe porque estaba hasta la corona de
cortesanos, obispos y palaciegos.

---Lo siento mucho---dijo D. Juan al despedirle.---Pero ya ve usted,
media España ha venido aquí a ponerse a las órdenes de la Reina\ldots{}
¡Es un ángel esa señora! Aunque no lo parezca, sepa usted que yo la
admiro mucho. Dicen que será nombrada Regente\ldots{} y no me pesa, no
me pesa\ldots{}

Cuando Cordero iba por el jardín acompañado de un chico que le llevaba
las maletas encontró a Salvador, el cual se empeñó en compartir con él
su alojamiento, aunque estrecho, suficiente para los dos. Dio mil
excusas D.~Benigno que en aquel momento sintió más vivo que nunca el
misterioso recelo que su amigo le inspiraba; pero al fin no tuvo más
remedio que aceptar, so pena de tener que dormir en la calle o en un
banco de los jardines.

---No hay que pensar ahora---le dijo Monsalud con cariño,---en que esos
señores firmen. Ninguno de ellos sabe ahora dónde tiene la mano derecha.
Esperando a ver en qué para esto, viviremos juntos, charlaremos, nos
contaremos nuestras desdichas y nos consolaremos mutuamente.

Al día siguiente Fernando cobró algunas fuerzas, y serenándose su mente,
empezó a comprender la infame sorpresa de que había sido víctima. No
obstante, todavía los Reyes legítimos estaban en Palacio como cohibidos
por la gente apostólica, cuyo poder era grande aún, a pesar de la
situación desfavorable en que se encontraban. Les esperaba todavía el
golpe de gracia, que había de darles muerte en la esfera cortesana,
cerrándoles todo camino que no fuera el de la guerra. En la madrugada
del 22 llegó a San Ildefonso la infanta Carlota, esposa del infante Don
Francisco y hermana de Cristina, mujer resuelta, varonil, desparpajada,
libre y campechana de palabras, alta, airosa y algo manolesca de figura,
valerosa hasta lo sumo, despótica, y tan ardiente de genio que, según
pública opinión, trataba a bofetadas, cuando el caso lo requería, a las
personas ligadas a ella por el parentesco más íntimo. Odiaba con toda su
alma a las dos princesas brasileñas, Doña Francisca y la de Beira, y
este aborrecimiento podrá explicar quizás mejor que ninguna razón
política, la guerra que había declarado a los apostólicos. ¡Formidable
influencia de la mujer en el destino de los pueblos! Los hombres
pensando, plantean las teorías y los sistemas, crean los partidos; las
mujeres amando o aborreciendo, determinan la acción. Imaginando que la
historia es un drama, el hombre es el histrión y la mujer el autor. No
ha existido ningún gran suceso político que no haya venido a la historia
a impulsos de manos femeninas, y esa académica nave del Estado de que
tanto hablan los tratados políticos no navegaría muchas veces si no
tiraran de ella las voladoras palomitas de Venus.

Doña Carlota entró en Palacio hablando a gritos, tratando con modales
bruscos a todo el mundo, servidumbre, gentiles-hombres y damas;
presentose a su hermana y después de abrazarla la llamó tonta unas
veinte veces. El testigo presencial de estas escenas, que ya no eran de
tragedia ni de drama sino de opereta, cuenta que como Cristina y Carlota
hablaban acaloradamente en italiano, no era posible a los presentes
entender bien lo que decían; sólo se entendían algunas palabras, como
\emph{sciocca}, \emph{pazza}, \emph{regina de galleria},
\emph{sceleratezza}\ldots{} Después la Infanta descansó un momento, y a
hora avanzada de la mañana anunció que recibiría a los ministros y demás
personajes que quisieran cumplimentarla. Cuando Calomarde y el conde de
la Alcudia entraron, Doña Carlota afectó serenidad y preguntó al
ministro de Gracia y Justicia la razón de haber revelado el secreto del
codicilo, contra lo dispuesto por Su Majestad. Tembloroso y cortado, D.
Tadeo se excusó con el letargo del Rey, que parecía muerte.

---Su Majestad---dijo Doña Carlota, disimulando su ira,---quiere recoger
el original del codicilo y me encarga decir a usted que lo presente
ahora mismo.

El ministro se inclinó, saliendo en busca de lo que se le pedía.
Entretanto todos los que no se habían manifestado muy claramente
partidarios del Infante se reunían en la Cámara. En pie y moviéndose sin
cesar de un lado para otro, altiva, nerviosa, respirando fuerte, Doña
Carlota parecía que imaginaba crueldades y violencias impropias de mujer
y de princesa. Los circunstantes no le dijeron nada, y Cristina misma,
con ojos encendidos de tanto llorar y el seno palpitante, enmudecía ante
la arrogantísima actitud de aquella nueva Semíramis, su hermana.

Cuando Calomarde entregó a la Infanta el manuscrito, que tantos desvelos
y fingimiento había costado a los apostólicos, Carlota no se tomó el
trabajo de leerlo y lo rasgó con furia en multitud de pedazos. Con el
mismo desprecio y enojo con que arrojó al suelo los trozos de papel,
echó sobre la persona del ministro estas duras palabras, que no suelen
oírse en boca de príncipes:

---Vea usted en lo que paran sus infamias. Usted ha engañado, usted ha
sorprendido a Su Majestad abusando de su estado moribundo; usted al
emplear los medios que ha empleado para esta traición, ha obrado en
conformidad con su carácter de siempre, que es la bajeza, la doblez, la
hipocresía.

Rojo como una amapola, si es permitido comparar el rubor de un ministro
a la hermosura de una flor campesina, Calomarde bajó los ojos. Aquella
furibunda y no vista humillación del tiranuelo compensaba sus nueve años
de insolente poder. En su cobardía quiso humillarse más y balbució
algunas palabras:

---Señora\ldots{} yo\ldots{}

---Todavía---exclamó la Semíramis borbónica en la exaltación de su
ira,---todavía se atreve usted a defenderse y a insultarnos con su
presencia y con sus palabras. Salga usted inmediatamente.

Ciega de furor, dejándose arrebatar de sus ímpetus de coraje, la Infanta
dio algunos pasos hacia Su Excelencia, alzó el membrudo brazo, disparó
la mano carnosa\ldots{} ¡Plaf! Sobre los mofletes del ministro resonó la
más soberana bofetada que se ha dado jamás.

Todos nos quedamos pálidos y suspensos, y digo \emph{nos}, porque el
narrador tuvo la suerte de presenciar este gran suceso. Calomarde se
llevó la mano a la parte dolorida, y lívido, sudoroso, muerto, sólo dijo
con ahogado acento:

---Señora, manos blancas\ldots{}

No dijo más. La Infanta le volvió la espalda.

Calomarde acabó para siempre como hombre político. Los apostólicos,
cuando se llamaron carlistas, le despreciaron, y el execrable ministril
se murió de tristeza en país extranjero.

\hypertarget{xxxv}{%
\chapter{XXXV}\label{xxxv}}

A la misma hora la muchedumbre, paseando en los amenísimos jardines,
comentaba los sucesos de aquellos días. D. Benigno y Salvador paseaban
juntos como viejos amigos, y ya se habían contado parte de sus secretos.
Cordero estaba triste, Monsalud se iba exaltando más cada día con la
idea política. De pronto vieron que la multitud se agolpaba en un sitio,
por donde discurría en abigarrada procesión mucha gente de Palacio, con
dorados uniformes y huecos casacones. Abría calle el público para dar
paso a estos señores. Cordero y Monsalud se acercaron para ver mejor.
Sostenida por una nodriza, rodeada de damas, seguida de personajes, una
niña de dos años andaba con dificultad, batiendo palmas y riendo de
alegría. Aquellos eran los primeros pasos de una Reina.

Del gentío salió una voz que gritó con furor: \emph{«¡Viva Isabel II!»}
Y una exclamación inmensa recorrió los jardines, perdiéndose y
desparramándose como los primeros ecos de una tempestad naciente.

La tempestad estaba cerca: oíanse los primeros truenos; pero el que
quiera conocer los notables sucesos, ya privados ya públicos, que restan
por referir, tenga paciencia y espere a leer lo que con toda verdad se
dirá en el libro siguiente.

\flushright{Madrid, Mayo-Junio de 1879.}

~

\bigskip
\bigskip
\begin{center}
\textsc{Fin de los Apostólicos}
\end{center}

\end{document}
