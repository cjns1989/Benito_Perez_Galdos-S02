\PassOptionsToPackage{unicode=true}{hyperref} % options for packages loaded elsewhere
\PassOptionsToPackage{hyphens}{url}
%
\documentclass[oneside,14pt,spanish,]{extbook} % cjns1989 - 27112019 - added the oneside option: so that the text jumps left & right when reading on a tablet/ereader
\usepackage{lmodern}
\usepackage{amssymb,amsmath}
\usepackage{ifxetex,ifluatex}
\usepackage{fixltx2e} % provides \textsubscript
\ifnum 0\ifxetex 1\fi\ifluatex 1\fi=0 % if pdftex
  \usepackage[T1]{fontenc}
  \usepackage[utf8]{inputenc}
  \usepackage{textcomp} % provides euro and other symbols
\else % if luatex or xelatex
  \usepackage{unicode-math}
  \defaultfontfeatures{Ligatures=TeX,Scale=MatchLowercase}
%   \setmainfont[]{EBGaramond-Regular}
    \setmainfont[Numbers={OldStyle,Proportional}]{EBGaramond-Regular}      % cjns1989 - 20191129 - old style numbers 
\fi
% use upquote if available, for straight quotes in verbatim environments
\IfFileExists{upquote.sty}{\usepackage{upquote}}{}
% use microtype if available
\IfFileExists{microtype.sty}{%
\usepackage[]{microtype}
\UseMicrotypeSet[protrusion]{basicmath} % disable protrusion for tt fonts
}{}
\usepackage{hyperref}
\hypersetup{
            pdftitle={El Grande Oriente},
            pdfauthor={Benito Pérez Galdós},
            pdfborder={0 0 0},
            breaklinks=true}
\urlstyle{same}  % don't use monospace font for urls
\usepackage[papersize={4.80 in, 6.40  in},left=.5 in,right=.5 in]{geometry}
\setlength{\emergencystretch}{3em}  % prevent overfull lines
\providecommand{\tightlist}{%
  \setlength{\itemsep}{0pt}\setlength{\parskip}{0pt}}
\setcounter{secnumdepth}{0}

% set default figure placement to htbp
\makeatletter
\def\fps@figure{htbp}
\makeatother

\usepackage{ragged2e}
\usepackage{epigraph}
\renewcommand{\textflush}{flushepinormal}

\usepackage{fancyhdr}
\pagestyle{fancy}
\fancyhf{}
\fancyhead[R]{\thepage}
\renewcommand{\headrulewidth}{0pt}
\usepackage{quoting}
\usepackage{ragged2e}

\newlength\mylen
\settowidth\mylen{« Les pieds sauvent la personne,}

\usepackage{stackengine}
\usepackage{graphicx}
\def\asterism{\par\vspace{1em}{\centering\scalebox{.9}{%
  \stackon[-0.6pt]{\bfseries*~*}{\bfseries*}}\par}\vspace{.8em}\par}

 \usepackage{titlesec}
 \titleformat{\chapter}[display]
  {\normalfont\bfseries\filcenter}{}{0pt}{\Large}
 \titleformat{\section}[display]
  {\normalfont\bfseries\filcenter}{}{0pt}{\Large}
 \titleformat{\subsection}[display]
  {\normalfont\bfseries\filcenter}{}{0pt}{\Large}

\setcounter{secnumdepth}{1}
\ifnum 0\ifxetex 1\fi\ifluatex 1\fi=0 % if pdftex
  \usepackage[shorthands=off,main=spanish]{babel}
\else
  % load polyglossia as late as possible as it *could* call bidi if RTL lang (e.g. Hebrew or Arabic)
%   \usepackage{polyglossia}
%   \setmainlanguage[]{spanish}
%   \usepackage[french]{babel} % cjns1989 - 1.43 version of polyglossia on this system does not allow disabling the autospacing feature
\fi

\title{El Grande Oriente}
\author{Benito Pérez Galdós}
\date{}

\begin{document}
\maketitle

\clearpage

\hypertarget{i}{%
\chapter*{I}\label{i}}
\addcontentsline{toc}{chapter}{I}

Sí; era en la calle de Coloreros, en esa oscura vía que abre paso desde
la calle Mayor hasta la plazuela y arco de San Ginés. Allí era, sin duda
alguna, y hasta se puede asegurar que en la misma casa donde hoy admira
el atónito público fabulosa cantidad de pececillos de colores dentro de
estanques de madera y muestras preciosas de una importantísima
industria: las jaulas de grillo. Allí era, sí, y no es fácil que ningún
contemporáneo lo niegue, como han negado que Francisco I estuviese en la
torre de los Lujanes y que Sertorio fundara la Universidad de Huesca
(que es achaque de los modernos meterse a desmentir la tradición). Allí
era, sí, en la calle de Coloreros y en la casa de los rojos peces y de
las jaulas de grillos, donde vivía el gran D. Patricio Sarmiento.

En lugar de los estanques de madera, vierais, corriendo el año 1821, una
ventana baja con rejas verdes a la derecha del portal. Aplicad el oído,
ya que la cortineja de indiana rameada no permita dirigir hacia dentro
la vista, y oiréis una voz sonora y grandilocuente, ante cuya majestad
las de Demóstenes y Mirabeau serían un pregón desacorde. Oíd sin
cuidado. Es de día. Detiénense los curiosos y atienden todos sin que
nadie les estorbe.

«Cayo Graco, hijo de Tiberio Sempronio Graco y de Cornelia, era liberal,
señores; tan liberal, que se rebeló contra el Senado. Decid, niño: ¿qué
era el Senado en aquella época?

Una voz infantil contesta:

---El Senado era una camarilla de serviles y absolutistas que no iban
más que a su negocio».

---«Muy bien\ldots{} Porque habéis de saber que Cayo Graco fijó el
precio del trigo para que los pobres tuvieran el pan barato. Como que
era un hombre que no vivía sino para el pueblo y por el pueblo. Luego
les probó a los senadores que estaban robando el tesoro del
Reino\ldots{} digo, de la República. Así es que aquellos tunantes no
querían que Cayo Graco fuese elegido diputado\ldots{} Decid, niño: ¿cómo
llamaban entonces a los diputados de la

Nación?

---Les llamaban Aglaé, Pasitea y Eufrosina.

---Zopenco, ésos son los nombres de las tres Gracias\ldots{} De
rodillas, pronto, de rodillas\ldots{} ¡Valiente borriquito tenemos
aquí!\ldots{} Tú, Gallipans, responde.

---Les llamaban \emph{tribunos de la plebe}, y había cuatro órdenes de
ellos, a saber: el toscano, el jónico, el dórico y el corintio.

---Has empezado como un sabio y concluyes como una mula. ¿Qué berenjenal
es ese que haces mezclando a los diputados de Roma con los órdenes de
arquitectura?\ldots{} Pues bien: les llamaban \emph{tribunos de la
plebe}. El Senado, aquella pandillita de hombres ambiciosos, que
acaparaban los destinos gordos, las superintendencias, las secretarías
y, ¿por qué no decirlo?, los ministerios, no querían que Cayo Graco
fuese tribuno y estorbaban su elección por medio de intriguillas. ¿Qué
habían de querer, si en todas las sesiones de Cortes les ponía de hoja
de perejil? No se mordía la lengua el gran patriota, y en plazas y
cafés, y en el foro y en los pórticos de las iglesias, por doquiera,
señores, convocaba al pueblo para enseñarle las doctrinas
constitucionales y condenar la tiranía y los tiranos\ldots{} Decidme
ahora, niño: ¿quién era el cónsul Opimio?

---El cónsul Opimio.

---Muy bien dicho. Un fatuo, un pedante, un cobarde, un servilón, una
especie de \emph{persa} que salía siempre a la cal e escoltado por una
cohorte de candiotas, o idiotas, que es lo mismo, para que los
partidarios de Graco no pudieran zurrarle la pavana. Decid, niño: ¿cómo
se llamaba el amigo de Cayo?

Todas las voces infantiles responden a un tiempo:

---Flaco.

---Ese nombre no se os olvida, picarones, porque os hace reír. Muy bien;
pues sabed que un día los partidarios de Opimio, después del sacrificio,
que es como si dijéramos al salir de misa de doce, insultaron a los de
Graco, los cuales asesinaron a un alguacil, macero, lictor o como quiera
llamársele. Vierais allí, cual encrespadas olas de un mar borrascoso,
chocar unos con otros, pueblo y tropa, democracia y tiranía, patriotismo
y servilismo. La sangre corría por las calles de Roma como corre en la
de Coloreros el agua cuando llueve. Se degollaban unos a otros e iban
arrojando cabezas al río. Quién gritaba \emph{viva la Constitución},
quién aclamaba a los cónsules diciendo \emph{vivan los verdugos}, y
hasta los niños pequeñitos tomaban parte en la encamizada refriega, no
de otra manera que los tiernos cachorros del león, cuando se disputan un
huesecillo para jugar. Retíranse Graco y Flaco\ldots{} (\emph{Risas en
el menudo auditorio}).

---¡Silencio!\ldots{} ¿Qué importuno y discorde reir es ese? Retíranse
Graco y Flaco; van en busca de Rufo\ldots{} (\emph{Nuevas risas}.)

---Silencio, digo\ldots{} o ninguno sale hoy de aquí. ¿Qué risas son
ésas? Periquito, Chatillo, Roque\ldots{} ¿no os da vergüenza de profanar
este augusto recinto con vuestras ridículas bufonadas?\ldots{} Orden,
compostura, atención, silencio\ldots{} Pues decía que se retiraron todos
al monte Aventino, que era un monte, pues\ldots{} un monte que se
llamaba Aventino. Pero, ¡ay!, los cónsules les cercan, envían numerosa y
aguerrida tropa para que a cañonazos les destruyan allí, y tienen que
marcharse, señores, al otro lado del Manzanares, o sea el Tíber, que
todo viene a ser lo mismo, a un sitio que bien podría nominarse la
Fuente de la Teja, y que estaba consagrado a las Furias, o si se quiere
con más propiedad, a los demonios. Los partidarios de Graco empiezan a
desertar porque el Gobierno les ofrece destinos y dinero. ¡Perfidia
inaudita, escandalosa traición que no volverá a pasar, yo os lo
juro!\ldots{} Al mismo tiempo, Opimio y sus infames cómplices ofrecen
pagar a peso de oro la cabeza del gran tribuno. Éste se ve perdido. Dice
a su esclavo Filócrates que lo mate. Filócrates vacila\ldots{} ¡momento
de angustia y dolor supremo! Los sicarios llegan, los serviles se
acercan rugiendo, cual manada de famélicos lobos. Consérvase sereno y
tranquilo Cayo. La fuga le es imposible. Suplica a su esclavo por
segunda vez que le dé muerte. Éste obedece. Hiérese él mismo con el
estilete, que era una pluma de las que empleaba aquella gente para
escribir sobre papel de cera, y cae, bañando el suelo con su sangre
preciosa. Los del cónsul llegan, córtanle la cabeza, y van con ella a
pedir el vil premio de su hazaña. Decidme, niño: ¿de qué materia
llenaron la cavidad cerebral de la patriótica cabeza para que pesara más
y aumentase el valor de tan cruento trofeo?

Todas las voces a un tiempo:

---De plomo.

---Perfectamente. Y pesó diecisiete libras. Ahora\ldots{} basta de
historia romana y pasemos a la retórica. Ea, niños: divídanse los dos
bandos. Roma, a la izquierda; Cartago, a la derecha. Veremos quién ciñe
el lauro de la victoria y quién muerde el polvo en esta honrosa lid de
la retórica.

Gran tumulto. Corren unos a este lado, otros al contrario, y agrúpanse
en dos bandos al pie de los estandartes españoles con sendos
cartelillos, en uno de los cuales se lee \emph{Roma} y en otro
\emph{Cartago}. Susurro murmurante, parecido al de las colmenas, precede
a las primeras preguntas. Los combatientes esperan con ansia el inicial
encuentro, y los juveniles corazones palpitan, vacilando entre el miedo
y un honroso tesón.

---Veamos\ldots{} Comience este pindárico certamen por una proposición
máxima. Decid, niño: ¿de cuántas clases son los pensamientos?

---De dos: claros y oscuros.

---Bien por Cartago. A ver, responda ahora la gran Roma. ¿Qué son
pensamientos claros?

No se había pronunciado aún la respuesta, cuando oyose gran tumulto en
la calle, y una voz gritó en la reja:---¡Hoy no hay escuela!

Y esta voz se confundió con alaridos de la bulliciosa turba, que
corriendo decía:

---¡A Palacio, a Palacio!

\hypertarget{ii}{%
\chapter*{II}\label{ii}}
\addcontentsline{toc}{chapter}{II}

La escuela quedó en un instante vacía, y D. Patricio Sarmiento salió a
la puerta de la calle. Sesenta años muy cumplidos; alta y no muy
gallarda estatura; ojos grandes y vivos; morena y arrugada tez, de color
de puchero alcorconiano y con más dobleces que pellejo de fuelle; pelo
blanco y fuerte, con rizados copetes en ambas sienes, uno de los cuales
servía para sostener la pluma de escribir sobre la oreja izquierda; boca
sonriente, hendida a lo Voltaire, con más pliegues que dientes y menos
pliegues que palabras; barba rapada de semana en semana, monda o peluda,
según que era lunes o sábado; quijada tan huesosa y cortante que habría
servido para matar filisteos y que tenía por compañero y vecino a un
corbatín negro, durísimo y rancio, donde se encajaba aquélla como la
flor en el pedúnculo; un gorrete, de quien no se podía decir que fue
encarnado, si bien conservaba históricos vestigios de este color, la
cual prenda no se separaba jamás de la cúspide capital del maestro;
luenga casaca castaña, aunque algunos la creyeran nuez por lo
descolorida y arrugada; chaleco de provocativo color amarillo, con ramos
que convidaban a recrear la vista en él como un ameno jardín; pantalones
ceñidos, en cuyo término comenzaba el imperio de las medias negras, que
se perdían en la lontananza oscura de unos zapatos con más golfos y
promontorios que puntadas y más puntadas que lustre; manos velludas,
nervudas y flacas, que ora empuñaban crueles disciplinas, ora la
atildada pluma de finos gavilanes, honra de la escuela de Iturzaeta; que
unas veces nadaban en el bolsillo del chaleco para encontrar la caja de
tabaco, y otras buceaban en la faltriquera del pantalón para buscar
dinero y no hallarlo\ldots{} Tal era la personalidad física del buen
Sarmiento.

---¡A Palacio!---exclamó, viendo la mucha gente que bajaba hacia San
Ginés por delante de su casa y la muchísima que seguía la calle Mayor
hacia Platerías.---Hoy tendremos otra gresca. ¿A cuántos estamos?

---A 5 de Febrero---repuso un joven que junto a D. Patricio apareció,
con mandil de sastre, sosteniendo en la izquierda mano dos pedazos de
tela y en la diestra una aguja.---Parece ser que Narices ha escrito un
papel al Ayuntamiento quejándose de los insultos, y para que rabie más,
hoy le van a dar más música.

---Aparte de que no me gusta que se hable del Soberano con tan poco
respeto---dijo el maestro,---lo que has dicho, querido Lucas, me parece
muy bien. Pues que no quiere música, désele más música. Si no, que
cumpla sus deberes de rey constitucional y marche francamente por la
senda aquella de que nos habló el 10 de Marzo del año pasado\ldots{} Va
mucha gente. ¿Por qué no dejas la obra y corres allá? Tal vez ocurra
algún acontecimiento digno de ser transmitido a la posteridad. Yo iré
después a la Cruz de Malta, a ver qué se decide esta noche respecto a la
exposición que se proyecta dirigir al Rey contra el Ministerio. Me
parece admirable idea, querido Lucas, porque has de saber que yo combato
a Argüelles.

---Y yo también---replicó el sastre.---O nos dan un Ministerio
liberalísimo, que de una vez acabe con todos los tunantes, o el pueblo
soberano decidirá en su sabiduría\ldots{} ¿Dejo el trabajo? ¿Cierro el
puesto?

---Deja el trabajo, \emph{dimitte laborem}, y cierra el puesto, que
tiempo hay de mover el paño. Día llegará en que la patria más necesite
de bayonetas que de agujas. Si no tuviera que copiar esos pliegos,
también husmearía un poco. Ponte el uniforme, hijo, que en estos sucesos
públicos bueno es que cada cual se presente con los arreos de su
jerarquía. Los uniformes dan respetabilidad. Procura que la muchedumbre
no se desborde; amonéstala, que, al verte, ella respetará la gloriosa
institución a que perteneces. No grites, no vociferes, que eso no es
propio de quien representa la autoridad, la fuerza pública y la
soberanía armada. Consérvate sereno en medio del tumulto, y si tocan a
formar y hay lucha con los guardias y demás cohortes del absolutismo,
despliega, querido hijo, todo el valor de tu pecho, todo el brío de tu
raza, y sé cual indomable león, que no conoce riesgo y hace estremecer
al cobarde lobo sólo con el rugido de su cólera.

El joven sastre, mientras esto decía su venerable padre, vestíase a toda
prisa en el mismo portal que era albergue de la sastrería. En el momento
de abandonar la tienda para mezclarse al popular tumulto, un hombre
llegó a la puerta y se detuvo en ella, saludando cariñosamente al señor
Sarmiento.

---¡Hola, hola\ldots{} Sr.~Monsalud!---dijo éste.---¿Tan pronto de
vuelta? ¿No va usted a Palacio? Dicen que habrá tocata de
\emph{trágalas} y sinfonía de \emph{mueras y vivas}.

---¿Ha salido mi madre?---preguntó el joven sin hacer caso de las
observaciones de su amigo.

---No he visto salir a la señora Doña Fermina---replicó
Sarmiento.---Debe de estar arriba, acompañando a doña Solita y al
Taciturno.

---Subiré a decirle que no salga esta tarde.

---Aguarde usted, D. Salvador. Si no va usted más que a eso, le mandaré
un recado con Lucas. Quédese usted aquí. Vámonos a la esquina a ver
pasar la gente y hablaremos un rato. ¿Qué me dice usted de estas cosas?

---¿Pero no tiene usted escuela?

---He soltado al infantil rebaño. Si no lo hiciera, me alborotaría la
escuela, y mis lecciones se perderían en la algazara como semilla que se
arroja al viento. Es preciso transigir un poco con la inquietud
bulliciosa y la precocidad patriótica de estos chiquillos que han de ser
ciudadanos. De esta manera les voy educando sin tiranías, y mansamente
les inculco sus deberes y les preparo para que ejerzan la soberanía en
los venideros años venturosos, en los cuales nuestra Nación se ha de
empingorotar por encima de todas las Naciones.

El amigo y vecino de nuestro excelente D. Patricio sonrió.

---No crea usted---continuó el maestro,---que imitaré la conducta de ese
pedante insoportable, émulo y antagonista mío, el maestro Naranjo, de la
calle de las Veneras, el cual, cada vez que hay bullanga o revista de
milicianos u otra cualquier función vistosa, encierra a los chicos y no
les permite ver ni que regocijen sus tiernas almas con las emociones de
la cosa pública. Pero bien sabe usted que Naranjo es un poco y un mucho
servilón, hombre forrado en oscurantismo y encuadernado en
intolerancias, amigo de los enemigos de la Constitución, indiferente en
efigie, pero absolutista en esencia, con vislumbres de \emph{persa}
vergonzante y amagos de realista monacal. ¿Qué ha de hacer con los
pobres chicos un hombre de estas cualidades? Tiranizarlos, ennegrecer su
espíritu, imbuirles ideas despóticas, educarles en el desprecio de la
Constitución y en el amor al servilismo. ¡Desgraciada nación la nuestra
si prevalecieran en ella los alumnos de Naranjo! Vea usted, Sr.~D.
Salvador, una cosa de que el Ministerio debiera ocuparse sin levantar
mano: extirpar esas infames cátedras, suprimiendo todos los maestros de
escuela que con su conducta están sembrando la cizaña del servilismo,
para que en lo venidero estorbe y ahogue la frondosa planta de la
Constitución.

---Sí, es preciso poner mano en eso---respondió distraídamente
Monsalud.---Me parece que ya no pasa tanta gente.

---Si no tuviera que barrer la escuela y copiar unos pliegos, señor D.
Salvador, nos iríamos usted y yo a meter nuestro hocico en la plaza de
Palacio y oír algo de la rechifla\ldots{} pero ¡cómo ha de ser!\ldots{}
Primero es la obligación que la devoción.

Diciendo esto, D. Patricio entró en el aula, y tomando la escoba que
detrás de la puerta estaba, empezó su tarea.

---Si usted me lo permite---dijo Salvador, siguiéndole también
adentro,---escribiré una carta aquí en la mesa de usted.

---Gran honor es para mí\ldots{} Aquí tiene usted la pluma que he
cortado hace poco; aquí, la tinta; aquí, el papel. Me callaré para que
usted pueda escribir tranquilo\ldots{} Pues, como iba diciendo, yo me
alegro de que a Su Majestad, de quien siempre hablaré con mucho respeto,
le den estas lecciones de constitucionalismo. Los reyes, amigo mío, no
aprenden de otra manera. Les dice uno las cosas, y nada; se las repite,
se las vuelve a repetir, y ni por ésas; es preciso gritar y manotear
para que fijen la atención\ldots{} ¡Ah!\ldots{} Perdone usted. Estoy
levantando mucho polvo. Regaré un poquito.

Salvador Monsalud escribió lo siguiente:

\begin{center}
\normalsize
«A L.\textsuperscript{\textbf{.}}. G.\textsuperscript{\textbf{.}}. D.\textsuperscript{\textbf{.}}. G.\textsuperscript{\textbf{.}}. A.\textsuperscript{\textbf{.}}. D.\textsuperscript{\textbf{.}}. U.\textsuperscript{\textbf{.}}. \\
Pod.\textsuperscript{\textbf{.}}. Sob.\textsuperscript{\textbf{.}}. Gr.\textsuperscript{\textbf{.}}. Com.\textsuperscript{\textbf{.}}. y Secr.\textsuperscript{\textbf{.}}. Gran Maest.\textsuperscript{\textbf{.}}. \\
del Gran Oriente de España. \\
S.\textsuperscript{\textbf{.}}. F.\textsuperscript{\textbf{.}}. U.\textsuperscript{\textbf{.}}. \\
Aristogitón.\textsuperscript{\textbf{.}}. gr.\textsuperscript{\textbf{.}}. 18. \\
\end{center}
\begin{flushright}
\textsc{(Salvador Monsalud.)}»
\end{flushright}

~

Después se quedó un rato pensativo mordiendo las barbas de la pluma.

---Cuidadito; retire usted un poco los pies, que mojo---dijo Don
Patricio, agitando la regadera junto a la mesa.---Ahora se puede barrer
sin cuidado\ldots{} No de otra manera la benéfica lluvia de la libertad
impide que se levante el sucio polvo de la tiranía\ldots{} Vea usted,
Sr.~D. Salvador, qué poco aprenden los reyes. Como los chicos, no
entienden sino a palos. Yo digo que la Constitución con sangre entra. En
Octubre del año pasado, cuando Su Majestad no quería sancionar la
reforma de monacales, por instigación de D. Víctor Sáez y del
embajadorcillo de Su Santidad, el pueblo amenazó con una revolución y
Fernando no tuvo otro remedio que sancionar. ¿Pero sirviole de enseñanza
este suceso? No, señor, porque en El Escorial conspiraba contra el
Gobierno, y el nombramiento de Carvajal en decreto autógrafo era un
proyecto de golpe de Estado. ¡Iniquidad funesta! Pero el pueblo no se
duerme. Cuando Fernando entró en Madrid\ldots{} ¡qué día, qué solemne
día! ¡qué 21 de Noviembre! En vez de vítores y palmadas, galardón propio
de los sabios monarcas, Fernando oyó gritos rencorosos, mueras
furibundos, amenazas, dicterios, oyó ternos como puños y vio puños como
ternos. No ha presenciado Madrid una escena tan imponente. Allí era de
ver el pueblo ejerciendo el soberano atributo de amonestación; allí era
de oír el trágala, cantado por las elegantes mozas del Rastro. Miles de
brazos se agitaban amenazando y todas las bocas espumarajeaban de rabia.
Los que llevábamos en la mano el libro de la Constitución, lo besábamos
en presencia del Rey. Un fraile pronunció varios discursos que encendían
más los ánimos. De repente, por entre apiñadas cabezas, se alzan
multitud de manos que sostienen un niño. Es el hijo de Lacy. La multitud
soberana grita: «¡Es el vengador de su padre! ¡Es el hijo del gran
patriota! ¡Mueran los tiranos! ¡Viva la Constitución!» El Rey oía todo,
y su semblante echaba fuego\ldots{} Pues bien: ¿cree usted que esta
lección fue provechosa? Nada de eso. La camarilla sigue conspirando; la
Corte desafía a la nación, al mundo y al linaje humano con la infame
conspiración y plan de D. Matías Vinuesa que ha escandalizado a Madrid
días pasados.

Salvador prestando escasa atención a las palabras del maestro, escribió
despacio y con largos descansos lo siguiente:

«Dispensad, H.\textsuperscript{\textbf{.}}. y
M.\textsuperscript{\textbf{.}}. Q.\textsuperscript{\textbf{.}}.
H.\textsuperscript{\textbf{.}}. la libertad con que os manifiesto mi
pensamiento después de saludaros con los s.\textsuperscript{\textbf{.}}.
y b.\textsuperscript{\textbf{.}}. c.\textsuperscript{\textbf{.}}. en
este Or.\textsuperscript{\textbf{.}}. de Madrid.

«Faltaría a los más altos deberes si no me negara a aceptar vuestros
ofrecimientos y la misión que me encomendasteis, porque estando
convencido de que ese Or\ldots{} es un centro de libertinaje y de
anarquía, y tal como está organizado produce efectos contrarios a los
verdaderos principios liberales, deseo que se me considere como
H.\textsuperscript{\textbf{.}}. D.\textsuperscript{\textbf{.}}. y se
aparte mi humilde persona de todos los trabajos de la
O.\textsuperscript{\textbf{.}}. Quizás sea mío el error y no de los de
V.\textsuperscript{\textbf{.}}. H.\textsuperscript{\textbf{.}}.
pero\ldots»

Al llegar a este punto se detuvo, recorrió con la vista lo escrito, hizo
un gesto de disgusto, y, rompiendo el papel, empezó a escribir otro.

---¿No sale, no sale la cartita?---dijo D. Patricio, sonriendo.---Se
conoce que es de amores. No a todos los mortales es dado manifestar
elegantemente sus pensamientos en forma literaria. ¿Quiere usted que vea
si puedo yo sacarle del paso?

---Gracias; no es preciso\ldots{} ¿Con que decía usted, Sr.~D. Patricio,
que el

Rey\ldots?

---No aprende nunca. Veremos qué tal efecto produce la amonestación de
esta tarde. Observe puntualmente la Constitución; sea amigo del pueblo;
ame la libertad como la amamos todos, y entonces no habrá más que
aclamaciones y flores\ldots{} Pero ¿estuvo usted anoche en \emph{Malta}?

---Yo no voy a ese manicomio.

Y en \emph{La Fontana}? Dicen que van a cerrar los cafés patrióticos.

---Harán bien.

---Bien sé que usted al hablar de este modo, lo hace por espíritu de
oposición, y que dice lo contrario de lo que piensa. Es particular que
le parezcan a usted detestables esas sociedades tan propias de un pueblo
libre, y que se le antojen majaderos y charlatanes los hombres eminentes
que en ella derraman el fructífero rocío de la palabra constitucional.
Si no conociese el gran entendimiento de usted\ldots{}

El joven siguió escribiendo sin atender a las palabras del dómine. Pasó
un rato, durante el cual uno y otro callaron. Después, Monsalud rompió
por segunda vez el papel escrito y empezó otro.

---Vamos, que está durilla esa oración primera de activa. Ya van dos
pliegos rotos.

---Antes me dejaré matar---dijo Monsalud en un arranque
espontáneo,---que contribuir a este desorden y figurar en una sociedad
que es un hormiguero de intrigantes, una agencia de destinos, un centro
de corrupción e infames compadrazgos, una hermandad de
pedigüeños\ldots{}

---¡Ah, ya veo, ya comprendo de quién habla usted!---exclamó Sarmiento,
soltando rápidamente la escoba y sentándose frente a su amigo.---Esos
intrigantes, esos compadres, esos pedigüeños, esos hermanos son los
masones. Bien, muy bien dicho; todas esas picardías las he dicho yo
antes que usted y las repito a quien quiera oírlas. El Grande Oriente
perderá a España, perderá a la libertad, por su poco democratismo, sus
transacciones con la Corte, su repugnancia a las reformas violentas y
prontas, su templanza ridícula, su orgullo, su justo medio, su
doceañismo fanático, su estancamiento en las pestíferas lagunas de lo
pasado, su repulsión a todo lo que sea marchar hacia adelante, siempre
adelante, por la senda constitucional. O hay progreso o no lo hay. Si lo
hay, si se admite, fuerza es que demos un paso cada día, que a cada hora
desbaratemos una antigualla para construir una novedad, que a cada
instante discurramos el modo de dar al pueblo una nueva dosis de
principios, y que no se aparte de nuestra mente la idea de que hoy hemos
de ser más liberales que ayer y mañana más que hoy\ldots{} Pero ¿se ríe
usted?

---No, no me río. Oigo al Sr.~D. Patricio con muchísimo gusto.

---Adelante, siempre adelante---añadió Sarmiento con calor.---En virtud
de este criterio, yo y todos los verdaderos patriotas hemos dado de lado
a la masonería para fundar la grande y altísima y por mil títulos
eminente y siempre española sociedad de \emph{Los Comuneros}.

---He estado mucho tiempo fuera de Madrid---dijo Salvador,---y al
regresar he oído hablar mucho de esa nueva hermandad. Por lo visto, el
Sr.~Sarmiento pertenece a ella. Sírvase usted explicarme en qué
consiste.

---¡Explicar! ¿A qué vienen esas explicaciones? ¿Por qué no ha de
conocer usted de \emph{visu} lo que difícilmente podrá comprender
\emph{ex audita}? Véngase usted conmigo. Le presentaremos en la
sociedad, le haremos caballero de Padilla, y para mí será tan grande
honor presentarle como para la Confederación recibirle.

---¡Confederación! ¡Padilla! ¿Qué ensalada es ésa?

---En el primer artículo de los estatutos se dice que nos
\emph{reunimos} y nos \emph{esparcimos} por el territorio de las Españas
con el propósito de \emph{imitar las virtudes de los héroes que, como
Padilla y Lanuza, perdieron sus vidas por las libertades patrias}.

---¿Y la Confederación se divide en talleres?

---¿Qué talleres? Eso es cosa de artesanos. Aquí todos somos caballeros.
Llámase nuestro jefe el \emph{Gran Castellano}; la Confederación se
divide en \emph{Comunidades}, éstas, en \emph{Merindades}; éstas, en
\emph{Torres}, y las \emph{Torres} en \emph{Casas Fuertes}. Todo es
caballeresco, romancesco, altisonante. Si la masonería tiene por objeto
auxiliarse mutuamente en las pequeñeces de la vida, nosotros nos
\emph{reunimos} y nos \emph{esparcimos}, asimismo se dice\ldots{} para
\emph{sostener a toda costa los derechos y libertades del pueblo
español, según están consignados en la Constitución política,
reconociendo por base inalterable su artículo} 3.º Nada de empeñitos;
nada de lloriqueo de destinos ni de asidero de faldones. El artículo 17
del capítulo 2.º, dice que ningún caballero \emph{interesará el favor de
la Confederación para pretender empleos del Gobierno}. ¿Qué tal? Esto se
llama catonismo. ¡Hombres incorruptibles! ¡Pléyade ilustre! Tenemos
Código penal, alcaides, tesoreros, secretarios. Nuestras logias se
llaman Fortalezas, a las cuales se entra por puente levadizo nada menos.
La admisión es peliaguda. Está mandado que al iniciar a alguno no se
revele nada del objetivo y modo de la Confederación; pero yo le digo a
usted todo, todito, porque confío en su discreción y prudencia.

---¿Y se puede ver eso? ¿Se puede ir allá?---dijo Salvador, demostrando
curiosidad.---Supongo que habrá juramentos y pruebas\ldots{}

---Le presentaré, Sr.~D. Salvador. Nuestra Confederación se honrará
mucho con que usted entre en ella.

---No; preguntaba si se puede ir a las Fortalezas como se va al teatro,
para ver, para reírse un rato.

---Amigo mío---dijo Sarmiento con gravedad,---no es cosa de risa una
sociedad donde se jura morir defendiendo a la patria y donde se cumple
lo que se jura.

---Eso es lo que no se ha probado todavía.

---Yo se lo probaré a usted, se lo probaré---exclamó vivamente Don
Patricio, apoyándose en la escoba como un centinela en el fusil.

---Si usted me hiciera el favor\ldots---indicó sonriendo Monsalud.

---¿De probárselo?

---No; de callarse. Un momento nada más, queridísimo amigo mío.

---Si no digo una palabra\ldots{} Escriba usted---indicó el maestro,
recomenzando su interrumpida tarea.---Voy a purificar mi escuela, a
barrer, digámoslo así, mientras usted escribe la carta. ¿Quiere usted
que se la dicte?

---No, gracias. El asunto es delicado; pero a la tercera ha de salir.

Y en efecto, salió.

\hypertarget{iii}{%
\chapter*{III}\label{iii}}
\addcontentsline{toc}{chapter}{III}

Es indispensable el conocimiento de todas las familias que vivían en
aquella casa. Ocupaba el principal Salvador Monsalud con su madre, y el
segundo, un señor taciturno y reservado, del cual los vecinos, a
excepción de Salvador, no conocían más que el nombre, ignorando sus
antecedentes y sus ideas políticas, a pesar de las impertinentes
pesquisas que por averiguarlo hacía diariamente el curioso Sarmiento.
Este y su hijo Lucas, sastre de oficio, ocupaban una de las habitaciones
del piso tercero, sirviendo la otra de morada a Pujitos, gran maestro de
obra prima, miliciano nacional, patriota, cuasi orador, cuasi héroe, y
un si es no es redactor de diarios políticos, que para todo había en
aquel desmesurado entendimiento.

El habitante del cuarto segundo era un hombre decente, con indicios en
toda su persona de pobreza decorosamente combatida y disimulada por el
aseo, la economía, las cepilladuras de la ropa y otros artificios que no
siempre realizaban el fin deseado. Tenía más de cincuenta años, aspecto
débil y enfermizo, rostro muy melancólico, apagados ojos, ademanes
corteses y fríos, escasísima propensión comunicativa y costumbres tan
tranquilas como metódicas. Jamás anochecía sin que estuviese dentro de
su casa. A horas fijas salía y a horas inalterables entraba. Era
rarísimo acontecimiento que alguien le visitase, y su morada era
silenciosa y triste, como vivienda de cartujos.

Antes de que penetrara en ella cualquier extraño, tomábanse minuciosas
precauciones, y dos ojos negros miraban por la cruz del ventanillo,
examinando atentamente al inoportuno. Estos ojos negros eran los de una
señorita, hija del señor Gil de la Cuadra (que así llamaban al
taciturno) y única compañera suya, a más de una criada, en la triste
mansión. Todo lo que tenía de antipático el padre entre los habitantes
de la casa, lo compensaba en simpatías la hija. A todos agradaba; solía
conversar con D. Patricio al entrar y salir, y muy a menudo pasaba a la
habitación de Doña Fermina Monsalud, charlando con ella largas horas.
Tenía por nombre Soledad, pero como su padre la llamaba Solita, así la
decían todos, y más comúnmente doña Solita; que entonces las señoritas
cargaban todavía con un Doña no menos grande que el de cualquiera
quintañona.

Como cronistas sentimos tener que decir que Solita era fea. Fuera de los
ojos negros, que aunque chicos eran bonitos y llenos de luz, no había en
su rostro facción ni parte alguna que aisladamente no fuese
imperfectísima. Verdad es que hermoseaban la incorrecta boca finísimos
dientes; mas la nariz redonda y pequeña desfiguraba todo el rostro. Su
cuerpo habría sido esbelto si tuviera más carne; pero su delgadez
exagerada no carecía de gracia y abandono. Mal color, aunque fino y
puro, y un metal de voz delicioso, apacible, que no podía oírse sin
sentir dulce simpatía, completaban su insignificante persona. Es
sensible para el narrador que su dama no tenga siquiera un par de
maravillas entre la raíz del cabello y la punta de la barba; pero así la
encontramos y así sale, tal como Dios la crió y tal como la conocieron
los españoles del año 21.

El gran misterio de D. Urbano Gil de la Cuadra, lo que traía en gran
inquietud a los vecinos, y principalmente a D. Patricio, era la
ignorancia en que todos estaban acerca de sus ideas políticas. ¿Era
liberal? ¿Era servil? Enigma terrible que daba vueltas como una rueda
pirotécnica dentro del febril cerebro de Sarmiento, sin ser descifrado
jamás. A veces, fundándose en conjeturas, en palabras sueltas, en la
letra \emph{sui generis} del sobre de una carta recibida por Gil,
Sarmiento le declaraba absolutista. Otras veces, fundándose en iguales
datos, diputábale revolucionario. Causaba desesperación al buen
preceptor que Monsalud lo supiese todo, y no lo revelase a los vecinos.

---O este hombre es un emisario de la Santa Alianza---solía decir
Sarmiento,---o un apoderado de los republicanos franceses. A estos
viejos ojos que tanto han visto, no se les escapa nada.

Al anochecer de aquel día en que nuestra relación comienza, entró, como
de costumbre, en su casa el padre de Solita. Ésta, que se hallaba
acompañando a Doña Fermina, subió a su habitación cuando sintió los
pasos de Gil. Al poco rato subieron también Sarmiento y Monsalud,
acompañados de Lucas, que a la sazón volvía de la plaza de Palacio, y
los tres entraron en el principal, porque el maestro de escuela gustaba
de platicar con Doña Fermina sobre la cosa pública, en que él era, como
el lector sabe, tan experto.

Reunidos los cuatro, Lucas contó los sucesos de aquella tarde, que
consistían en dos piedras arrojadas al coche de Su Majestad, en diversos
gritos patrióticos, en un miliciano herido por un guardia, y algunas
contusiones y corridas de escasa importancia.

---A pesar de eso---dijo Sarmiento gravemente,---no aprenderá. Seguirá
oponiéndose a la plantificación lógica del sistema constitucional;
fomentará la superstición y el fanatismo. Si yo fuera llamado a regir
los destinos de la nación; supongan ustedes que lo fuera\ldots{} ¿eh?,
pues bien: mi primer decreto sería suprimir el cuerpo de Guardias.
Mientras la camarilla tenga la probabilidad de ese apoyo, la libertad no
echará profundas raíces en el hispano suelo.

---Esta tarde se ha dicho---indicó Lucas,---que el Gobierno va a
disolver la guardia.

---¿Lo ven ustedes? Mi idea\ldots{} es idea mía.

---Y a cerrar las sociedades patrióticas.

---Ésa no es idea mía. La rechazo. Por el contrario, Sr.~D. Salvador,
Doña Fermina, yo abriría en cada calle dos por lo menos, dos cafés
patrióticos, y los subvencionaría con fondos del Estado, para que se
propagase la idea constitucional. ¿Qué le parece al Sr D. Salvador mi
idea?

---Excelente---respondió el joven, ocupado a la sazón en hojear varios
libros que sobre la mesa de la habitación había.

---Ya que está aquí el Sr.~D. Patricio---dijo Doña Fermina, después de
hablar un rato con la criada,---no se irá sin tomar chocolate. Y lo
mismo digo a usted, Lucas.

Sarmiento que, dicho sea en honor de la verdad histórica, no había ido a
otra cosa, respondió de este modo:

---No se moleste la señora\ldots{} Siento haber venido; pero si se ha de
enojar usted con nuestra negativa, aceptamos\ldots{} Madre e hijo son
tan amables que, la verdad, cuando uno entra en esta casa, no encuentra
la puerta para salir.

---Gracias, Sr.~D. Patricio.

---¿Saben ustedes---dijo con aire misterioso Lucas,---que esta tarde vi
en la plaza de Palacio al vecino del cuarto segundo? Estaba hablando con
un guardia.

---¿Pero no saben ustedes lo mejor?---indicó Sarmiento, padre.---Pues ya
me olvidaba\ldots{} Que tengo nuevos datos para juzgar de las opiniones
políticas del Sr.~Gil de la Cuadra.

Monsalud miró fijamente al preceptor.

---Un precioso dato. Tengo por seguro que es \emph{despótico}.

---Vamos, no hable usted mal de los vecinos, y menos de ese buen
sujeto---dijo Doña Fermina.---Él y su niña son personas muy decentes que
merecen el mayor respeto.

---¿Respeto? No se lo niego. Oiga usted el dato, Sr.~D. Salvador. Ayer
tarde entró en mi academia para que le cortase una pluma. Ya sabe usted
que en la pared de enfrente tengo un buen retrato de Riego. Como el
Sr.~Gil le mirase atentamente, yo dije: «ése es el grande hombre».
Advertí en el semblante de nuestro vecino una sonrisa picaresca. Mirome,
y con mucha suficiencia y pedantería, exclamó: «Es un majadero».

---Lo mismo dice mi hijo---manifestó la Monsalud, ofreciendo el
chocolate a sus dos vecinos.

---¿Lo mismo dice? Será por broma. ¡Riego, D. Rafael del Riego! Inmensa
figura que se alza sobre el suelo de la patria, y con su majestuosa
cabeza toca las nubes! ¡Riego, sol refulgente que todo lo inunda con su
luz! ¿A quién sino a él se debe la libertad que gozamos? ¿A quién sino a
él debe España el haberse puesto por montera del mundo y el estar por
encima de toditas las Naciones?

---Pues Salvador dice que es una cabeza llena de viento---dijo Doña
Fermina, gozando en mortificar al maestro.

---Bromas; son bromas, Sr Sarmiento---dijo el joven con benevolencia.

Monsalud había encendido una luz y examinaba cartas y papeles.

---Como bromas pueden pasar; pero son de mal género. Esas bromas puede
oírlas cualquiera que no sepa discurrir\ldots{} Yo no me tengo por
ignorante; yo creo haber leído algo; creo poseer alguna ciencia\ldots{}
digo, me parece a mí\ldots{}

---Por de contado.

---Algo sabe uno de lo que ha pasado en el mundo: memorables hechos y
preclaras acciones, o sea lo que los eruditos llamamos historia. Y si
no, que lo diga el Sr.~D. Salvador.

Monsalud no dijo nada.

---Pues bien---añadió Sarmiento sorbiendo la mitad de lo que contenía la
jícara,---yo declaro que conozco pocos varones de la antigüedad (y ahí
está Plutarco que lo certifique\ldots) sí, conozco pocos que se igualen
a este atrevido comandante, que desafió al absolutismo, a toda la
Europa, señora; a la Santa Alianza, a los Borbones todos, a los serviles
todos. Y tan gran fin realizó sin derramamiento de sangre,
porque\ldots{} vean ustedes la historia: Harmodio y Aristogitón
derramaron mucha sangre; las sediciones de los Gracos también fueron
cruentas; Bruto mató a César; Robespierre y Danton, ya sabemos que
cortaban cabezas como yo plumas; Cromwell degolló a Carlos I,
\emph{etcétera}. Pero nuestro hombre ha dicho \emph{sea la libertad}, y
la libertad ha sido. Su espada no ha necesitado herir para vencer. Con
su vívido fulgor deslumbráronse los tiranos, y, despavoridos, huyeron
cual asustadas liebres. ¿No es verdad, señor D. Salvador? ¿No es verdad
esto?

Monsalud tampoco dijo nada, ni hacía caso de la disertación sarmentil.

---¡Y a hombre tan insigne, a este campeón que le dijo a España, como el
ángel a María: \emph{El Señor} o \emph{la Libertad es contigo}; a ese
apóstol, señores, se le tiene alejado de la Corte, como si fuera una
plaga, un pedrisco u otra calamidad aterradora! Se le desterró primero a
Asturias; se le desterró después, porque destierro es, a la Capitanía
General de Aragón\ldots{} ¡Oh! si yo llegase a regir los destinos de la
España; si yo\ldots{} pongamos por caso, llegase a ser ministro\ldots{}
mi primera disposición sería para recompensar dignamente a ese héroe
inaudito\ldots{}

---¿Más todavía?\ldots---indicó festivamente Monsalud.

---¿Pues qué?---dijo Sarmiento con ciceroniano ademán, poniendo sobre la
mesa la jícara vacía,---acaso se le han tributado honores
correspondientes a sus servicios? Ni aun en la jerarquía militar ha
tenido la elevación a que es acreedor. Él era comandante: le plantaron
en mariscal de campo\ldots{} Bueno; pues eso, digan lo que quieran, es
bien poco, es poquísimo; y aún me parecían una bicoca los tres
entorchados. Usted tenga presente cómo recompensó Inglaterra a Lord
\emph{Vellintón} después de la campañita aquella en que derrotó a
Bonaparte. Así se premian los grandes servicios, no con estas
mezquindades de aquí.

---Tiene razón el Sr.~Sarmiento---dijo Doña Fermina.---Si por lo de
militar merece los tres entorchados, por lo que tiene de orador y de
hombre discreto se le puede señalar una renta. Vaya, que la escena y los
discursos aquellos del teatro fueron cosa bonita.

---Extraordinariamente buena, aunque usted, señora mía, lo diga con
cierto tonillo zumbón. Lucas, ¿te acuerdas?\ldots{} Nosotros fuimos
desde muy temprano a la cazuela. ¡Qué tumulto, qué palmadas, qué
entusiasmo! Yo me puse tan ronco que en ocho días no pude dar lección a
los chicos. Aún me parece que veo a nuestro querido General levantarse
del asiento con aquella majestad que él sólo tiene, y echarnos un
discurso que me pareció de perlas, si bien con el mucho alboroto no se
oía una palabra desde arriba. Aún me parece que estoy oyendo la pomposa
música del himno que entonó el público. Riego, con aquella gracia suma
que Dios le ha dado, levantose y dijo: «La música del himno no es así,
sino de esta otra manera». Y se puso a cantarlo. Sus ayudantes llevaban
el compás.

---¡Estaría bonito!\ldots{}

---Después, uno de los ayudantes cantó el \emph{trágala}, \emph{perro},
y aquí fue Troya. Yo creo que hasta las figuras pintadas en el techo
cantaron en aquel instante. ¡Sublime momento, señora!\ldots{} Pero los
envidiosos no faltan en ninguna parte. Empéñase el jefe político en
decir que aquello era un desorden. Quiere hacernos callar; encréspase el
público como el Océano agitado por rabioso Noto; empiezan las puñadas,
los dimes, los diretes, los ternos de pimentón, las cantáridas
gramaticales. Riego mira con desdén al jefe político. Algunos de sus
ayudantes, mostrando una impavidez pasmosa, le insultan. Aporréanle dos
o tres paisanos, Paco Rincón y Blas Cortada, si no me engaño; el teatro
parecía una caldera hirviendo; el General se retira al fin, y, ¡oh,
pavor!, las calles están llenas de gente, la tropa se encierra en los
cuarteles, y todo es zozobra y miedo de trifulcas. Sin la imprudencia
del jefe político, nada habría pasado. Pero el despotismo es así: no le
gusta oír el himno ni el \emph{trágala}; no quiere ver la faz del
libertador del hesperio suelo, y aquí tienen ustedes el resultado:
\emph{guerras}, \emph{asolamientos}, \emph{fieros males}, como dijo el
poeta. Nada, nada; según esa gente estólida, a la Libertad debe
ponérsele bozal para que no muerda.

---Bozal para que no muerda---repitió taciturnamente Monsalud.

---De la cosa más sencilla, del desahogo más ingenuo---continuó el
vehemente preceptor,---toma pie el despotismo para extender su férreo
dominio\ldots{} Volvamos a nuestro invicto Don Rafael. De nada vale el
popular deseo. Se empeñan en que ha de salir de aquí, y le echan como se
echa un perro que incomoda. Las sociedades patrióticas dejan oír su
autorizada voz en contra de tal vilipendio; pero no son oídas.
Manifiesta el pueblo su voluntad de mil maneras; fíjanse pasquines;
gritamos, pedimos, suplicamos, amenazamos. Yo pongo a todos los niños de
mi academia la cinta verde con el lema \emph{Constitución o muerte.} Ni
por ésas. ¿Cómo contestan a nuestras honradas exhortaciones? Echando los
cañones a la calle; lanzando de los cuarteles la caballería para que
pisotee al pueblo; acuchillando sin piedad a la gente indefensa. En
tanto Argüelles habla en las Cortes de las célebres \emph{páginas}, y
Feliú habla de los \emph{hilos}; se alborotan también los diputados, y
cuando un gran patriota como Romero Alpuente se dispone a defender al
pueblo, ahogan su generosa voz los chillidos de los serviles. Riego es
desterrado, y ¡qué ignominia! disuelven el ejército de la Isla, que
había proclamado la Constitución; y por este camino volveremos a la
tiranía y oscurantismo del año 14, y al despotismo puro, el cual,
después de todo, es mejor que el mixto, vergonzante, tibio o moderado
que ahora tenemos. ¿No es verdad, Sr.~D. Salvador?

---Sí, amigo D. Patricio; todo lo que usted quiera. ¡Y pensar que tantas
cosas malas se remediarían con que el Sr.~D. Patricio fuese ministro
media docena de días!\ldots{}

---No se burle usted---dijo el preceptor, algo picado.---Yo no seré
ministro, yo no puedo ser ministro, porque soy muy honrado, porque no
soy intrigante, porque no soy ambicioso. Si tuviera un duro por cada vez
que me he negado a aceptar este o el otro destinillo, sería un
Fúcar\ldots{} Pero supongamos que fuera ministro, y sentemos esa
atrevida hipótesis\ldots{}

---Silencio---dijo Monsalud.---Están llamando a la puerta.

Atendieron todos. Oyéronse fuertes golpes en la puerta de la casa.

---¿Quién será?---murmuró con temor Doña Fermina.---Aquí no viene nadie
después de anochecido.

---Iré a ver---dijo Lucas, a quien los golpes sorprendieron descabezando
un sueño.

Pocos momentos después entraba Solita, con semblante pálido y
consternado, sin aliento, encendidos de llorar los ojos.

¡Mi padre está enfermo!---exclamó, dirigiendo a todos una mirada
suplicante.

---Iremos a buscar un médico---dijo D. Patricio con
oficiosidad.---¡Lucas!\ldots{} Corre al momento.

---No es preciso médico---dijo Solita, deteniendo a los Sarmientos con
un expresivo ademán.

---Yo entiendo algo de medicina\ldots{}

---No necesitamos cosa alguna---añadió la joven, mirando a Doña
Fermina.---Lo que tiene mi padre es muy singular.

---¿Congestión cerebral, ataque de gota, síncope, jaqueca\ldots?

---Mi padre está enfermo del ánimo---dijo tristemente Soledad.---No
quiere médicos ni medicinas; lo que quiere es hablar con el señor
Monsalud, y por eso vengo a rogarle que pase ahora mismo a casa.

Asombráronse todos de ver enfermedad que se aliviaba hablando.

---También puede que tenga algo que revelarme a mí---dijo Sarmiento,
dando algunos pasos.---Voy allá corriendo.

---No, usted no---replicó la joven, deteniéndole.---Salvador solo. Mi
padre desea verle y hablarle ahora mismo, ahora mismo.

Salvador subió sin tardanza al segundo piso.

Malísimo humor tenía Sarmiento cuando se retiró a su casa. No pudiendo
refrenar la abrasadora curiosidad que le consumía, detúvose junto a la
puerta del misterioso vecino, y aplicó el oído, anhelando percibir algo
de la conversación o confidencia que dentro se efectuaba; pero ni una
sílaba llegó a sus grandes orejas. Resignose a no saber nada, y al
entrar en su casa, dijo a Lucas:

---Insisto en que es \emph{absolutista}, hijo; un infame \emph{persa}
que nos ahorcaría a todos si le dejáramos.

\hypertarget{iv}{%
\chapter*{IV}\label{iv}}
\addcontentsline{toc}{chapter}{IV}

Halló Monsalud al Sr.~Gil de la Cuadra en un gabinete estrecho, donde
tenía cama y mesa de escribir. Estaba el taciturno sentado en un viejo
sillón, donde se hundía su flaco y miserable cuerpo, y todo en él
revelaba perniciosa mezcla de abatimiento y exaltación, cual si su
espíritu aumentase en actividad y la perdiera a toda prisa el cuerpo,
reclamando el final descanso de la sepultura. Sus ojos brillaban,
moviéndose en los irritados huecos, y con vaguedad calenturienta y
voluble fijábanse en todos los objetos. Movía la cabeza y los brazos sin
descanso, asemejándose su inquietud a tentativas de acciones concebidas
rápidamente y desechadas antes de la realización. Cada segundo
determinaba en aquella alma llena de zozobra un nuevo proyecto, un nuevo
plan, un nuevo deseo. Las luchas de insomnio le conmovían, pugilato
horrendo que el alma sostiene consigo misma creyéndose otra, y en el
cual hay formidables encuentros, caídas y elevaciones, un espantoso
temblor de congojas, contra las cuales no hay voluntad ni razón que
prevalezcan.

El personaje que ahora nos ocupa no es desconocido para los lectores de
estos libros\footnote{Veáse \emph{El Equipaje del Rey José}.}. Apareció
brevemente cuando describimos la retirada de los franceses en 1813.
Entonces abandonaba el suelo patrio como adicto al Intruso, a quien
había servido, desempeñando una plaza de oidor en la Chancillería de
Valladolid. Estableciose con su esposa, doña Pepita Sanahúja, en un
pueblecillo del Poitou, y poco después de estar allí hizo que le
llevaran su única hija, Soledad, a quien, por no exponerla a los
peligros de la retirada, dejó en el pueblo natal confiada a los
parientes de su primera esposa. Gil de la Cuadra había sido casado dos
veces, y Solita era hija del primer matrimonio, pues la señora que el
lector conoció en los campos de Álava no tuvo prole. La emigración fue
tristísima para el oidor de la Chancillería de Valladolid, a pesar de la
dulce compañía de su adorada hija, porque después de haber perdido casi
toda su fortuna en el gran conflicto de la monarquía extranjera, tuvo el
dolor de ver expirar a su segunda mujer en el invierno del año 18.

De regreso a España, cuyas puertas abrió para los infelices renegados la
revolución de 1820, se estableció con su hija en La Bañeza; pero
circunstancias funestas que él mismo nos dará a conocer le obligaron a
trasladarse a Madrid, donde la casualidad le llevó a la misma casa que
habitaba Salvador Monsalud, cuya suerte tan unida estuvo después de la
batalla de Vitoria a la del fugitivo matrimonio. A pesar de la amistad
contraída en la fatal jornada del 21 de Junio y de las buenas relaciones
que sostuvieron en la emigración, pues Salvador vivió también algunos
meses en Poitiers, Gil de la Cuadra se mostraba en Madrid muy poco
comunicativo y afectuoso con su vecino. Era su carácter en verdad
inclinado a la reserva, a cierta aspereza misantrópica que entibiaba las
amistades. Visitábanse, sí, con frecuencia, y Soledad pasaba algunos
ratos acompañando a Doña Fermina; pero Gil de la Cuadra, en sus
entrevistas con el antiguo jurado, mostraba el singular recato y la
estudiada sobriedad de palabras que indican empeño de ocultar
ocupaciones o designios. Por esta misma razón causó sorpresa al joven
verse llamado tan a deshora y con tanto anhelo.

Indicándole con una seña que se sentara a su lado, Gil de la Cuadra le
habló de este modo:

---Dispénseme usted si me he tomado la libertad de hacerle subir para
confiarle un asunto grave. Sepa usted que yo soy muy desgraciado, el más
desgraciado de los hombres\ldots{} Necesito el amparo de un ser
generoso, de un buen amigo, de una persona discreta y al mismo tiempo
poderosa.

---Yo no puedo ni valgo nada---replicó Salvador,---pero lo que de mis
escasas facultades dependa, está a disposición de usted.

---Revelaré todo y decidiremos---dijo Gil de la Cuadra con esforzada
voz.---Mi estado nervioso, la furia y exaltación de mi cerebro son tales
esta noche que creo moriré si no tomo una determinación
salvadora\ldots{} ¿Quiere usted que le hable con toda franqueza? Pues,
amigo mío, yo soy muy cobarde.

Después de esta declaración, Monsalud creyó que el Sr.~Gil iba a poner
en su conocimiento cualquier contrariedad insignificante.

---Muy cobarde---añadió el extraño enfermo.---Verdad es que lo que me
pasa es gravísimo. Si no tuviera una hija a quien adoro, a estas horas,
Sr. Monsalud, ya me habría dado muerte. En un momento de exaltación,
casi llegué a olvidarme de mi pobre Solita, y abrí esa ventana para
arrojarme a la calle. Vivir así, no es vivir.

---Dígame usted con calma lo que tanto le mortifica, y resolveremos.

---Ante todo debo recordarle a usted una deuda que conmigo
tiene---indicó el taciturno, fijando en su amigo los ojos con expresión
patética.---Mi esposa, que en gloria esté, y yo le salvamos a usted la
vida en aquellos aciagos días de Junio de 1813, que no puedo recordar
sin espanto.

---Tampoco yo---dijo Monsalud palideciendo.

---Le salvamos a usted la vida---añadió Gil de la Cuadra complaciéndose
en esta idea fundamental de su argumentación.---Después de ocultar a
usted diferentes veces, yo autoricé a mi esposa para que, cediendo todas
sus alhajas, que eran gran parte de nuestra fortuna, le rescatara a
usted del poder de aquellos malvados guerrilleros que querían
sacrificarle.

---¡Es cierto!---murmuró Salvador con voz grave.

---¿Cabe mayor abnegación tratándose de un desconocido?

---No, no cabe más. Cien vidas de agradecimiento no bastarían para pagar
eso que usted llama deuda, y como tal, con todo mi corazón la reconozco.

---¿De modo que usted, amigo mío, se halla dispuesto a hacer por mí, si
me veo en un conflicto supremo, lo que mi esposa y yo hicimos por usted
cuando peligraba su vida?

---Dispuesto con toda mi alma---afirmó el joven lleno de piedad y
efusión.---Ordene usted lo que debo hacer. Cuanto tengo, cuanto valgo,
mi vida y mi nombre están a disposición de usted. No es un sacrificio,
es un deber; y si no recuerdo mal, no ha sido preciso que llegaran
ocasiones supremas para hacer este ofrecimiento, porque desde nuestra
primera entrevista en Madrid me declaré deudor eterno de usted.

---Es verdad; gracias, gracias---dijo el enfermo, estrechando con sus
flacas y amarillas manos las de Monsalud.---Mucha atención a lo que voy
a referir. Creo haber indicado a usted cuando estábamos en Francia que
mis ideas han sido siempre favorables a los derechos absolutos de la
Corona y a la monarquía pura tal como durante siglos la disfrutaron las
más gloriosas naciones de la tierra. La ambición de mi segunda esposa y
debilidades mías, que deploro amargamente, me indujeron a reconocer y
servir al intruso Bonaparte. No necesito recordar la ignominiosa caída
del partido afrancesado. Yo, que no pertenecí a él de corazón, sino por
las sugestiones de mi mujer, tengo más derecho que los demás a quejarme
de mi detestable suerte. Volví del destierro sin que mis ideas sufriesen
mudanza alguna, y es singularísimo, y a la par muy triste, que los
absolutistas del 14, con quienes mi corazón simpatizaba, me cerraran las
puertas de la patria, y me las abriesen los liberales, a quienes tengo
la desgracia de aborrecer. Esta contradicción real y molesta entre mi
modo de pensar y mi gratitud, obligome el año pasado a huir
prudentemente de las cosas políticas.

Retireme a mi pueblo natal, La Bañeza. Como allí conocían todos mis
ideas, un día los liberales me acometieron con palos, ordenándome que
diese vivas a la Constitución; negueme a tal vilipendio, y aquella deuda
que para con ellos contrajeron mis honrados labios, pagáronla mis
costillas con buenos cardenales. No obstante, tuve paciencia, señor y
amigo mío, y seguí pacíficamente en mi casa, pidiéndole a Dios que ponga
fin a esta insoportable tiranía del populacho, mas sin buscar venganza,
resistiéndome a tomar parte en los trabajos que algunos realistas traían
entre manos para levantar partidas. En estas andadas, organizose en La
Bañeza la llamada Milicia Nacional, que yo llamaría Infernal hablando
propiamente, y para dar pruebas de su existencia y hacer el estreno de
su bárbaro poder, emprendiendo con brillo el camino de la gloria, creyó
que lo mejor era adjudicarme una nueva paliza, como lo hizo el 3 de
Septiembre del año pasado, pretextando que yo conspiraba.

---Ya van dos, Sr.~Gil. En verdad que admiro la resignación y
sufrimiento de usted.

---Mes y medio de cama me costó la hazaña de los milicianos de mi
pueblo. ¿Creerá usted que ni tales razones pudieron persuadirme a que
dejara mi pacífico y santo retiro? Aguanté, callé y esperé. Mi actitud
digna y cristiana debió ponerme a cubierto de nuevos ataques, ¿verdad?

---Seguramente.

---Pues no fue así. Precisamente por la razón de que yo sufría y
callaba, debieron aplacarse en ellos la feroz intolerancia y salvajismo;
pero no fue así, sino que mi humildad les hacía más bravos cada vez; y
alegando conspiraciones que sólo en su obtusa mente existían, me
atacaron de nuevo\ldots{}

---¿Otra vez?

---Sí, señor, y se lo digo a usted francamente. A la tercera paliza ya
no pude aguantar más, y lo que no había hecho hasta entonces, lo hice
desde aquel día.

---¿Conspirar?

---Justamente. Ellos se empeñaron en que conspirara, y conspiré. Aquí
tiene usted la sabiduría de los liberales. Con su imbécil sistema de
apalear a los que no piensan como ellos, van poco a poco convirtiendo en
enemigos a todos los españoles. Yo, que había hecho propósito firme de
no mezclarme en la política activa, ni contribuir al levantamiento de
partidas, ni conspirar, salí de mi casa decidido a todo, a todo
absolutamente; vine a Madrid, y mi mala suerte deparome aquí el
encuentro con un amigo de mi juventud, D. Matías Vinuesa, cura que fue
de Tamajón, y a quien Su Majestad, en premio de los méritos que contrajo
durante la guerra, hizo capellán de honor y arcediano de Tarazona.

---Ya sé a dónde va usted a parar---dijo Monsalud con
benevolencia.---Vinuesa le indujo a usted a intervenir en esa
descabellada conspiración que le ha llevado a la cárcel y que
probablemente le llevará también al patíbulo.

Al oír esto, el enfermo palideció y sus labios pronunciaron algunas
palabras a guisa de oración.

---Puesto que todo se lo he de confesar a usted---añadió, exhalando un
suspiro,---diré que, en efecto, he sido confidente y amigo de D. Matías
Vinuesa. Obra de muchos es el célebre plan, cuyo descubrimiento ha
ocasionado la prisión de ese bendito, y que, con perdón de usted, no es
descabellado ni mucho menos, y nos habría conducido al glorioso objeto
que anhelamos los buenos españoles, si la imprudencia, el soborno o la
traición no lo hubieran descubierto. Presumo yo que alrededor del Trono,
donde tanto se trabaja por derrotar al Gobierno y a los liberales,
existen la venalidad y la corrupción más que en parte alguna, y que de
los mismos que nos han incitado a conspirar partió la infame denuncia,
fundada en móviles que no comprendo. Ya estoy aburrido, desengañado de
la mala fe de todos, convencido de que tan pícaro es Juan como Pedro, y
de que no es posible tomar parte activa en la cosa pública sin meterse
en el fango hasta el coronilla.

---¡Lástima que no lo conociera usted antes de pringarse en la
desdichada conjuración palaciega de Vinuesa, que es, según he oído, una
de las mayores aberraciones que puede concebir la imaginación!.

---Siento que usted califique tan duramente un plan que no
conoce---repuso Gil de la Cuadra en el tono del amor propio herido.---Y
como no puede conocerlo si yo no se lo revelo, lo haré, porque después
de la prisión de mi amigo, no hay en ello inconveniente. La primera
condición de nuestro plan era el secreto. Sólo debían tener noticia de
él Su Majestad, el infante D. Carlos, el duque del Infantado y el
marqués de Castelar, como los únicos encargados de ponerlo en ejecución.
Llegado el momento del golpe, Su Majestad debía llamar a los ministros,
al Capitán general y al Consejo de Estado, y una vez que los tuviera a
todos bien agazapados en la real cámara, debía entrar una partida de
guardias de Corps, mandada por el serenísimo señor Infante, y prenderlos
a todos, luego que el Rey saliese de la estancia. Vea usted qué ardid
tan sencillo y al mismo tiempo tan fácil.

---Sí: todo es fácil y sencillo en las cabezas de los conspiradores.
Prosiga usted.

---Inmediatamente después el mismo señor infante D. Carlos debía pasar
al cuartel de guardias y mandar arrestar a todos los individuos poco
afectos a Su Majestad y a nuestras ideas.

---¿También es eso fácil y sencillo?

---Déjeme usted seguir. Al mismo tiempo el señor duque del
Infantado\ldots{} bien le conoce usted ¡qué imponente figura, qué aire
marcial! Sólo con presentarse inclina los ánimos a la obediencia\ldots{}
Pues digo que el señor Duque debía marchar en el mismo momento a Leganés
a ponerse al frente del batallón de guardias que hay allí.

---Suponga usted que los guardias de Leganés le recibieran a tiros, que
también puede ser\ldots{}

---No es probable que a tan grande prócer y cumplido caballero le
faltaran de ese modo\ldots{} Pero aún resta algo\ldots{} Excuso decirle
a usted que todo debía hacerse en el mismo momento.

---Es natural, y en un mismo momento dado también debía hundirse todo.
Adelante.

---Se sobrentiende que lo referido había de acontecer por la
noche---continuó el anciano.---Dado el primer golpe, veamos ahora su
desarrollo. A las doce en punto, ni minuto más ni minuto menos, debía
ponerse en camino para Madrid el batallón de Leganés, entrando en esta
Corte a las dos. A las tres en punto, el regimiento del Príncipe, con
cuyo coronel se contaba, debía ocupar todas las puertas de la villa, y a
las cinco y media, ni minuto más ni minuto menos, debían las tropas y el
pueblo empezar a dar \emph{vivas} a la Religión, al Rey, a la patria, y
\emph{mueras} a la Constitución y a los ministros\ldots{} Luego, el plan
contenía una multitud de determinaciones, consecuencia natural del
triunfo. Debían ordenarse varias cosas, verbigracia: que se celebrase un
Concilio nacional\ldots{} que los cabildos se encargaran otra vez de la
administración del \emph{Noveno}\ldots{} que hubiese tres días de
rogativas\ldots{} que se rebajase la tercera parte de la
contribución\ldots{} que los gastos de iluminaciones y festejos fueran
muy moderados\ldots{} que los milicianos sirvieran en el ejército ocho
años o pagaran veinte mil reales de redención\ldots{} que se trasladara
al obispo de Mallorca\ldots{} que se imprimieran por cuenta del Estado
las cartas del padre Rancio\ldots{} que el obispo auxiliar, portador del
libro de la Constitución el año 20, lo llevase también ahora, y con su
propia mano se lo diese al verdugo para quemarlo\ldots{} en fin, ya ve
usted que nada faltaba.

---Nada faltaba, a no ser sentido común. ¿Son también obra de usted los
papeles \emph{El Grito de un Español} y \emph{La Papeleta de León}?

---En esta misma mesa he escrito parte de ellos---repuso el enfermo con
disgusto.---Pero no disputemos ahora sobre la ruindad o excelencia del
plan. Yo sigo creyendo que sin los infames sobornos y traiciones que han
mediado, nuestra obra nos habría proporcionado un verdadero triunfo. No
es posible formar juicio de lo que no ha podido pasar del pensamiento a
la irrecusable prueba de los hechos. Lo real, lo positivo, lo que vemos
y tocamos, amigo mío, es que yo me encuentro comprometido, expuesto a
perder la libertad y quizás la vida, si no hallo un hombre discreto,
astuto, hábil y poderoso que me ampare en trance tan aflictivo.

---Pero la Corte, esa Corte que es la que alienta, paga y sostiene las
conspiraciones realistas, no le abandonará a usted\ldots{}

---¡Ah! Sr.~Monsalud de mis pecados---exclamó Gil de la Cuadra con
amarga tristeza,---la Corte, o no puede nada, o teme comprometerse
dándome el amparo que de ella he solicitado. Preso D. Matías, sin que ni
Rey ni Roque lo hayan podido evitar, hecha pública la conjuración, no
hay ningún prócer ni potentado de Palacio que no proteste de su adhesión
al liberalismo. ¡Pecador de mí! ¡Mil veces pecador! La circunstancia de
haber sido afrancesado me hace sospechoso a los absolutistas. Ésa es mi
fatalidad; ésa es mi estrella negra; ésa es la funesta herencia que me
dejó mi esposa. ¡Si viera usted cuántas puertas se han cerrado hoy ante
mí! Es particular: de la noche a la mañana ya nadie me conoce. Soy un
extraño, un importuno; creen, sin duda, que les voy a pedir un socorro
pecuniario, y me reciben de malísimo talante. La única muestra de
benevolencia que he recibido es muy triste, señor Monsalud. Diomela un
caballero de Palacio, avisándome hoy el peligro que corro, porque
halladas varias cartas y notas mías entre los papeles de Vinuesa, no han
de tardar en venir por mí para embaularme en la cárcel, donde, si Dios
no lo remedia, nos pudriremos el cura y yo, a no ser que nos cuelguen en
la plazuela de la Cebada. ¿No es verdad, Sr.~Monsalud, que debí preferir
el tratamiento de los milicianos de La Bañeza?

---¿Usted espera que le prendan? ¿Lo sabe usted?

---Lo sé.

---Pues en tal caso---dijo Salvador con asombro,---¿por qué no huye
usted? ¿Por qué no se oculta al menos?

---Precisamente de eso quiero hablarle---manifestó Gil de la Cuadra,
cayendo de nuevo en el lúgubre abatimiento en que Salvador le
encontrara.---¡Huir!\ldots{} Creo que no habrá otro remedio.

---Es el más seguro por ahora.

---Mis achaques me hacen de tal modo cobarde, que no acertaré a dar un
paso\ldots{} ¡Si parece que me convierto en un niño!\ldots{} ¡Si se me
oprime el corazón!\ldots{} Luego doy en pensar en la desdichada suerte y
desamparo de mi pobre hija\ldots{} ¿Qué será de ella si muero? De tal
manera se perturba mi alma y se enflaquece mi razón pensando en esto,
que no puedo discurrir los medios de mi fuga o escondite. Piense usted
por mí, pues no con otro objeto he solicitado su amparo; dígame usted lo
que debo hacer\ldots{} tráceme un plan.

---No sólo indicaré lo conveniente, sino que haré cuanto pueda para que
usted quede en salvo esta misma noche. Es preciso tomar una resolución
pronta. Ánimo, Sr.~Gil, no acobardarse, y triunfaremos.

---¡Oh!, gracias, gracias mil---exclamó el enfermo, estrechando las
manos de Salvador.

---El infeliz conspirador lloraba.

---No perdamos tiempo\ldots{} Saldremos juntos para que vaya usted más
tranquilo---dijo Monsalud, restaurando más a cada palabra la energía
moral y física de su vecino.---No carecerá usted de nada.

---¡De nada!\ldots{} ¡Qué bendición de Dios! Usted me devuelve la
vida\ldots{} Yo que empezaba a carecer de todo, hasta de lo más
preciso\ldots!

---El conflicto de usted, amigo D. Urbano, es poca cosa. Creo que nadie
nos estorbará la fuga. Le llevaré a usted a un paraje seguro, donde
vivirá tranquilo y oculto hasta que podamos conseguir un sobreseimiento,
una absolución\ldots{} allá lo veremos.

---¡Benditas mil veces sean esa boca y esas manos!---dijo Gil de la
Cuadra con emoción profunda.---Usted me salva; yo me arrojo en sus
brazos como en una playa hospitalaria después de ser juguete de las
olas\ldots{} ¿Con que usted, después que me ponga en lugar seguro,
conseguirá un sobreseimiento, una absolución?\ldots{} ¡Cuánto lo
agradeceremos mi hija y yo!\ldots{} Sola, Solita, ¿dónde estás?\ldots{}
Ven, corre a abrazar a este caballero.

---Vale más que nos dediquemos sin perder un instante a preparar todo lo
necesario\ldots{} ¿Qué hora es?

---Las once---dijo el anciano, levantándose con dificultad.---Me siento
mejor; me siento más ligero; se me ha despejado la cabeza; muevo las
piernas con flexibilidad; en fin, soy otro\ldots{} ¿Con que a
disponer\ldots?

---Sí, a disponerlo todo. Arregle usted lo que ha de llevar de su casa.
Yo me encargo de todo lo demás.

---¡Oh!, idolatrada hija mía, ya tienes padre otra vez; viviremos tú y
yo\ldots---exclamó Gil de la Cuadra con viva excitación de
espíritu.---Lo que va a hacer por mí, Sr.~Monsalud, supera a cuanto
hicimos por usted en aquel horrendo día. Si consigue ponerme en salvo
esta noche, me parecerá que resucito, y el horroroso aspecto de la
cárcel dejará de atormentar mi imaginación\ldots{} Con que
apresurémonos. Soledad, hija mía, ven\ldots{} Una vez que esté libre de
las garras de esos infames, fácil le será a usted sacarme del atolladero
de la causa. Las sociedades secretas a que usted pertenece lo hacen y
deshacen todo. Además, el señor duque del Parque, de quien es usted
secretario, administrador o no sé qué, pasa por uno de los hombres de
más valimiento que existen en España.

---Antes de medianoche estaremos fuera de Madrid---dijo Monsalud,
haciendo sus cálculos.---No conviene perder tiempo.

---Ese ánimo y decisión me regeneran---dijo Cuadra, dando algunos pasos
vacilantes por la habitación.---Déjeme usted que antes de ocuparme en
los preparativos de la fuga le dé a usted un abrazo, un estrecho abrazo
de amigo\ldots{} así\ldots{} Ahora veamos lo que se lleva\ldots{}
¡Soledad, Solita!

La muchacha apareció de repente, pálida, desconcertada. Su semblante
expresaba el terror más vivo, y sus descoloridos labios no acertaban a
pronunciar palabra alguna. El padre participó al punto por simpatía
natural del pavor de su hija; miró a Monsalud; éste formuló con ansiedad
una pregunta.

No pudo dar contestación la atribulada niña. Oyéronse terribles golpes
que resonaban en la puerta de la casa, haciendo retemblar a ésta de los
cimientos al tejado\ldots{} Oyéronse al mismo tiempo pasos de mucha
gente, palabras, un rumor soez que llenó de espanto el alma de los tres
personajes.

---¡Ahí están!---murmuró con voz tétrica Gil de la Cuadra.

---¡Ahí están!---repitió Monsalud, golpeando el suelo con tanta fuerza
que la casa redobló su temblor convulsivo y profundo, como contestando a
las llamadas de los polizontes.

\hypertarget{v}{%
\chapter*{V}\label{v}}
\addcontentsline{toc}{chapter}{V}

El amigo de Vinuesa cayendo en el sillón, se oprimió con ambas manos la
desnuda calva.

---Se me ha partido el alma\ldots---exclamó sordamente.---Parece que me
han arrancado la última raíz de la vida\ldots{} ¡Yo me muero!\ldots{}
¡Pobre hija mía!\ldots{}

Solita corrió hacia él. Hija y padre se unieron en estrecho abrazo.

---Ya no hay remedio---dijo el segundo con amargura.

Los golpes se repetían con más fuerza. Salvador, agitado por violenta
cólera y despecho, se golpeaba la frente con el puño. En algunos
momentos se sentía impulsado a una resolución desesperada; pero tenía
demasiado buen sentido para no refrenarse al punto.

---No hay remedio---dijo Gil de la Cuadra con acento solemne.---Hija
mía, oye lo que voy a decirte. ¿Ves este hombre?\ldots{}

Solita fijó en Monsalud sus ojos llenos de lágrimas.

---Salve usted a mi padre---gritó.---Discurra usted algún medio para
ocultarle, para sacarle de la casa sin que esos malditos le vean.

El tétrico silencio del joven indicó claramente que no podía discurrir
medio alguno que no fuese una locura.

---No puede ser, no puede ser---dijo el anciano.---¿Ves este hombre? Es
el único que puede hacer algo por mí, por nosotros. Mientras vivamos
separados, recuérdale un día y otro que tu padre está en la cárcel. Se
me figura\ldots{} se me figura que será un buen hermano para ti.

Los golpes redoblaron. Parecía que cien puños de hierro martillaban la
puerta, y la campanilla sin cesar movida, cayó de su sitio.

---Es preciso abrir al instante---manifestó con vivísima agitación Gil
de la Cuadra.---Una palabra más, amigo mío, hija de mi alma. Mientras
viene de Asturias tu primo Anatolio, que ha de ser, amén de tu marido,
tu único amparo después que yo falte, te dejo encomendada a este buen
amigo. Él será tu padre y tu hermano. Sr.~Monsalud, si acepta usted el
encargo, me voy más tranquilo a la cárcel, y de allí\ldots{}

---Acepto---dijo con grave acento el joven.---Solita será mi hermana.
Además juro por todos los santos y por Dios, que es mi padre, que le he
de sacar a usted de la cárcel a donde va esta noche.

Los tres se abrazaron sin añadir una palabra más. En el mismo instante,
despedazada la puerta de la casa, entró en la estancia un hombre brutal
y grosero, uno de estos que no creen representar bien a la autoridad si
no la hacen antipática y aborrecible.

---¿Quién es aquí el bribón de Gil de la Cuadra?---dijo mirando
alternativamente al joven y al anciano.---¡Ah! Conozco al mozo, que es
Monsalud\ldots{} Supongo que Cuadra será el vejete\ldots{} Véngase usted
conmigo a la cárcel de Villa\ldots{} no, a la de la Corona, porque en
aquélla no cabe más gente.

---El señor es Gil de la Cuadra---dijo Salvador.---Por el bribón no
preguntes, que aquí no hay otro que tú.

Dos, tres, cuatro individuos no menos simpáticos que su lindo jefe,
penetraron en la estancia.

---¿Y a esta tortolilla, la llevamos también?---preguntó uno,
atreviéndose a poner la mano en el hombro de la joven.

---Para preguntar una estupidez---repuso Monsalud, rechazándole
violentamente,---no se necesita dar coces.

---Juan Violín, no seas bruto---gruñó el jefe.---Deja a esa señorita y
alcánzame las esposas.

Gil de la Cuadra al ver que le iban a atar las manos huyó despavorido a
la pieza inmediata. Siguiéronle todos. Rogole Salvador que se sosegase,
no haciendo resistencia a sus bárbaros aprehensores, y cedió al fin el
anciano, y ofreció sus manos a las argollas de hierro. Abrazole
estrechamente Solita, diciendo con lastimeros ayes y lamentos que no se
apartaría de él, y fue necesario separarla. En la sala, Gil de la Cuadra
agobiado por la amarga pena, exánime y aturdido, cayó al suelo. Los
polizontes tiraron de él como se tira de un perro que se detiene a
hociquear en el suelo. Ayudole Salvador a levantarse y salieron de la
casa.

Cuando bajaban la escalera, D. Patricio y su hijo salieron a ver la
tristísima comitiva, y Fermina Monsalud quiso que Soledad entrase desde
luego en su casa. Detuviéronla todos, procurando consolarla; pero ella
insistió en bajar, y luchando con todas sus fuerzas, que no eran muchas,
procuraba desasirse de los brazos de Sarmiento y Doña Fermina.

---Le soltarán pronto\ldots{} No llore usted, niña---le decía el
preceptor.---Este Gobierno es como Dios lo ha hecho\ldots{} no persigue
más que a los liberales\ldots{} ¿Con que el señor Gil de la Cuadra era
la mano derecha de Don Matías Vinuesa?\ldots{}

Soledad bajó rápidamente, y tras ella Sarmiento. En la calle arrojose
otra vez la joven en brazos de su padre, manifestando inquebrantable
resolución de seguirle; pero las fuertes manos de los corchetes la
separaron. Gil de la Cuadra, negándose a dar un paso en compañía de la
soez cuadrilla, dejose caer en el suelo, y otra vez el egregio polizonte
tiró de la soga.

---Tengo sed---dijo el anciano, respirando con ansia.

Delante de él estaba D. Patricio, con las manos a la espalda, fijando en
el reo una mirada maliciosa y nada compasiva.

---Tengo sed---repitió Gil de la Cuadra.

---Sr.~Sarmiento---dijo Monsalud vivamente,---en la escuela de usted hay
una alcarraza con agua\ldots{}

---Mire usted qué demonches de casualidad---repuso Sarmiento, sin
moverse del sitio en que al anciano contemplaba;---se me ha olvidado
dónde puse esta tarde la dichosa alcarraza.

---Subiré yo---dijo Soledad procurando sobreponerse a su pena.

---Subiré yo---dijo Monsalud tomándole la delantera con rapidez
suma.---Aguarde usted abajo y procure calmar al pobre viejo.

Pocos instantes después, Salvador daba de beber a su amigo.

---La noche está fría---manifestó imperturbable y sin dejar su sonrisa
picaresca el gran Sarmiento,---y cuando la noche está fría\ldots{} y el
tiempo fresco\ldots{} pues no se tiene sed.

Los polizontes tiraron de la soga, acompañando su movimiento de ese
chasquido de lengua que tan bien entienden los animales.

---Ánimo, amigo---le dijo Monsalud.---No olvide usted mi promesa.

Pareció que el infeliz colega de Vinuesa recibía ánimo y vida al oír
estas palabras.

---¡Pobre hija mía!---exclamó, bebiéndose las lágrimas que copiosamente
corrían por sus mejillas.

---Solita es mi hermana---dijo Salvador, abrazándola.---Vamos: esto debe
acabarse. Se reúne gente.

Cuadra se levantó con dificultad. En su espíritu había seguramente
poderoso anhelo de colocarse a la altura de su situación, sofocando la
ruin pusilanimidad que le abatía.

---¡Mi hija!\ldots{} ¡Mi pobre hija!---gritó, clavando los tristes ojos
en el semblante de su joven vecino.

Con aquella mirada, su afligido corazón de padre dijo cuanto las
circunstancias exigían que dijera.

Solita perdió el conocimiento. Sarmiento, que estaba a dos pasos de
ella, la sostuvo en sus brazos.

---¿En dónde pongo esto?---murmuró festivamente.

---Subiré a Soledad a mi casa---dijo Salvador tomando en brazos a la
joven como si fuese un niño,---y después, Sr.~Gil, le acompañaré a usted
a la prisión.

Como lo dijo lo hizo, y poco después de medianoche todo estaba
terminado.

\hypertarget{vi}{%
\chapter*{VI}\label{vi}}
\addcontentsline{toc}{chapter}{VI}

Todavía no se había \emph{descubierto} el templo. No era aún la hora de
la \emph{tenida}, y los \emph{Hijos de la Viuda}, descansando de las
fatigas políticas en sus casas o en los cafés, esperaban que la
\emph{luz astral} de la noche marcase la hora propia para los trabajos
del \emph{Arte-Real}. Los \emph{Maestros Sublimes Perfectos}, los
\emph{Valientes Príncipes del Líbano o de Jerusalén}, los Caballeros
\emph{Kadossch}, los que antaño se llamaban \emph{Gerográmatas}, los
\emph{Hierorices}, los \emph{Epivames}, los \emph{Dadouques}, los
\emph{Rosa-Cruz} de hogaño, los hermanos todos, desde el \emph{Terrible}
hasta el \emph{Sirviente}; los aprendices, compañeros y maestros, desde
los de mallete hasta los de cuchara, estaban ocupados en el \emph{ágape}
doméstico, o bien conversando con sus \emph{mopsses}, jugando con sus
\emph{lovatones} o matando el tiempo en las reuniones profanas, lejos de
la \emph{verdadera luz}. Las \emph{estrellas} no se habían encendido
todavía, ni el \emph{mirto elusiaco} exhalaba su aroma. Imperaba la
rosa, emblema del silencio, y la imponente exclamación \emph{Ossé} no
había resonado aún bajo las \emph{bóvedas} orientales. En una palabra (y
hablando con claridad para inteligencia de los ignorantes), la sesión de
la logia no había empezado todavía.

En la \emph{Caverna del Mithra}, o sea el Universo, hay un punto que se
llama \emph{Mantua}, o Madrid, en cuyo punto es evidente la existencia
de una calle llamada de las Tres Cruces. En esa calle, cualquier
curioso, aunque no tenga sus oídos abiertos a la \emph{verdadera luz},
podrá ver una tienda de sastre, y si penetra en ella para que el supremo
arquitecto de las levitas le tome medida de una; si durante esta
fastidiosa operación alza los ojos a la \emph{bóveda del firmamento},
vulgo cielo raso, verá sin duda que por aquellos descoloridos y
descascarados yesos se pasean soles, lunas, rayos que fueron de oro,
cordones, triángulos, estrellas pitagóricas y otros signos. Al ver esto,
sentirá en su alma profundísima emoción de respeto, y dirá: «Aquí estuvo
el gran templo masónico en los tres \emph{llamados} años, del 20 al 23».

Siguiendo nuestra relación (y dejando que pasen algunos días después de
las escenas últimamente referidas, lo cual nos lleva a los últimos de
Febrero de 1821), nos dirigimos allá. Es temprano: es la hora en que
hierven los clubs, la hora en que \emph{Lorencini}, \emph{La Cruz de
Malta} y \emph{La Fontana} son otras tantas ollas donde burbujean con
rumoroso y mareante zumbido las pasiones políticas, entre el
chisporroteo de las envidias y el resoplido de las ambiciones. Todavía
es temprano, porque los trabajos masónicos \emph{se abren} (este
tecnicismo obliga frecuentemente a no hablar en castellano) a hora más
avanzada.

Aún está a oscuras el edificio de la calle de las Tres Cruces.
Reconocemos el \emph{vestíbulo}, la sala de \emph{Pasos perdidos}, donde
campean los \emph{Cuadros lógicos}, y no hallamos persona viva. Óyense
tan sólo los pasos de un \emph{hermano sirviente} que va y viene,
poniendo en su sitio las lámparas de aceite que bien pronto se han de
llamar \emph{estrellas polares}, \emph{astros} o \emph{nebulosas}. Por
último, vemos que entra un hombre con ademán resuelto, como persona muy
hecha a semejantes lugares, y observando que adelanta sin recelo alguno,
nos apresuramos a seguirle, tomándole por guía en el laberinto de
galerías y salas. El desconocido se acerca al \emph{sirviente}, y
después de saludarle con signos que no nos es posible determinar,
pronunciando una especie de santo y seña, le hace esta pregunta:

---¿Está el Sr.~Canencia?

---En la \emph{Cámara de Meditaciones} le hallará usted, Sr.~Monsalud.

Le seguimos denodadamente, aunque el nombre de \emph{Cámara de
Meditaciones} nos da cierta comezoncilla de miedo, por haber oído que es
un recinto pavoroso que hace enflaquecer el ánimo más esforzado. A pesar
de esto, penetramos detrás del gallardo joven, y desde el mismo instante
sentimos temblores y escalofríos al ver una habitación toda colgada de
negro, no puede decirse que alumbrada, sino entristecida por macilenta
luz. Damos diente con diente y el cabello se nos eriza al observar que
en diversas partes de la triste estancia cuelgan, cual objetos en
testero de tienda, cantidad de huesos y calaveras, y que medio esqueleto
se apoya contra la pared, mirando con desconsuelo al otro medio, o sea
los fémures y tibias que fueron de su pertenencia y ora yacen en el
suelo.

En la sepulcral pieza hay una mesa, y junto a esta mesa se ocupa en
\emph{burilar una plancha}, o sea extender un acta (hablando a lo
cristiano), un viejo de cabellos blancos. No atendemos a las
demostraciones amistosas que hace a nuestro introductor ni a las
palabras de éste; por ahora, atentos sólo al conocimiento del local,
fijamos los atónitos ojos en algunos letreros que entre hueco y hueco
adornan las paredes, y leemos: \emph{«Si vienes impulsado por una mera
curiosidad o por otro móvil aún peor, retírate; no trates de
descubrirla, porque penetraremos tus intenciones»}. Volvemos la cabeza,
y nos sale al encuentro otro parrafillo: \emph{«Si tu conciencia está
tranquila, ¿por qué sientes disgusto ante estos despojos que te
recuerdan el fin de tu vida?»} Otro letrero dice: \emph{«¿Siente tu alma
temor? Pues retírate, porque sólo un espíritu fuerte puede soportar las
pruebas a que has de ser sometido». «¿Te hallas dispuesto a sacrificar
tu vida en aras del progreso humano?»}

Poco a poco nos vamos familiarizando con el fúnebre y medroso
espectáculo, y echamos de ver que la Cámara, lo mismo que su extraño
mueblaje, tienen cierto sello de arrinconados cachivaches de teatro,
dicho sea con perdón de las humanas calaveras. El polvo que los cubre,
el desorden y abandono con que están colocados los huesos y las
inscripciones indican que todo aquello está en lamentable desuso. Era la
\emph{Cámara de las Meditaciones} un recinto donde encerraban al
catecúmeno para que preparara su ánimo antes de ser recibido como
aprendiz por la congregación masónica. Lo primero que tenía que hacer el
pobre profano, una vez que lo metían bonitamente allí, era otorgar su
testamento y contestar por escrito a varias preguntas, con objeto de
mostrar su manera de discurrir y los gramos de sal que tenía en la
mollera.

Formuladas las respuestas, un hermano entraba con el rostro cubierto en
la Cámara, y recogiendo aquéllas, las entregaba al \emph{Venerable}, que
ya estaba presidiendo la sesión o \emph{tenida}. Leíanse las pruebas del
talento del neófito, y si no resultaba alguna barbaridad estupenda,
concedíanle el goce de la verdadera luz. Aquí empezaba una serie de
ceremonias de que la gente de todos tiempos se ha reído mucho; pero
dicen los masones que hasta sus más insignificantes gestos y signos
tienen un sentido no menos profundo que los ritos de las religiones
india, judaica y cristiana. Digan lo que quieran, las ceremonias de
estas religiones, aun consideradas tan sólo bajo el punto de vista
artístico, tienen un sello especial de grandeza e idealidad; las
masónicas, que sólo vagamente responden a una idea filosófica, parecen,
por lo general, un juego de chiquillos, dicho sea con perdón de los
\emph{Valerosos} y \emph{Soberanos Príncipes}.

Cuando se acordaba que el profano tenía bastante entendimiento para ser
masón (y no debían de ser grandes las exigencias del tribunal),
vendábanle a mi hombre los ojos para conducirle a la logia, que estaba
comúnmente a dos pasos de la \emph{Cámara de Meditaciones}. Daba él un
golpecito en la puerta, y un masón, a cuyo cargo corrían las funciones
de \emph{primer celador}, decía con la voz más campanuda posible:
«Venerable, llaman profanamente a la puerta del templo».

El \emph{Venerable}, aunque sabía quién llamaba y por qué llamaba, se
hacía el sorprendido, diciendo con acento solemne: «Ved quién es».
Intervenía entonces otro funcionario, que se llamaba el \emph{guarda
interino}. Éste salía en averiguación del profano forastero que a
deshora turbaba la tranquilidad augusta de la logia, y entonces el
hermano que acompañaba al neófito decía: «Es un profano que desea ser
iniciado en nuestros secretos».

Por fin, después que habían mareado bastante al pobre lego, le dejaban
entrar, no sin que dijera antes su nombre, edad, naturaleza, estado,
religión, profesión y domicilio. El hermano que le presentaba ponía fin
a su alta misión con estas palabras: «Ahí os lo entrego; ya no respondo
de él».

Sería molesto y ocioso referir la serie de preguntas que el
\emph{Venerable}, desde la celeste luminosa altura del Oriente, dirigía
al neófito. Después de las preguntas empezaban las pruebas, a fin de
ver, según el código masónico, \emph{hasta qué punto la tortura física
influye en la lucidez de las ideas del neófito, y conocer su energía, su
carácter}, etc. Aquí venían las figuradas copas de sangre; los
homicidios de mentirijillas; los testarazos que no pasaban de broma; los
\emph{cálices de amargura}, cuyo licor ha sido siempre muy conocido en
la Fuente del Berro; las abluciones en un pilón denominado \emph{Mar de
bronce}, y otros sainetes, algunos de los cuales recibían el nombre de
\emph{viajes}, y lo eran en efecto, por los imaginarios países de Babia.
Al \emph{recién nacido} le asistía en tales actos un individuo a quien
llamaban el \emph{hermano terrible}, siendo común que desempeñara tal
comisión y llevase el atroz mote algún bonachón tendero de la plaza
Mayor o manso escribientillo de cualquier oficina.

En seguida juraba el recipiendario, prometiendo realizar cosas muy
buenas, para las cuales no es preciso seguramente hacer el payaso, pues
multitud de personas socorren a sus hermanos en la \emph{Caverna del
Mithra}, vulgo mundo, sin necesidad de que se lo mande un
\emph{Venerable} ni de que le mareen con preguntas vanas después de
bailar el minueto entre un \emph{Caballero Kadossch} y un \emph{Príncipe
del Líbano}. El juramento no era la última ceremonia, pues ningún
profano podía dejar de serlo, hasta que no le sobaban de lo lindo. Al
golpe de los \emph{malletes}, o sea martillos de palo, caía la venda de
los ojos del neófito, y se encontraba rodeado de llamas y espadas.

¡Tremendo, crítico instante para aquel que creyera iba a ser mechado y
asado culinariamente!\ldots{} Pero las llamas eran pintadas, y las
espadas, de hojalata. El \emph{Venerable}, compadecido entonces sin duda
de la situación de aquel pobre hermano metido dentro de una hoguera y
entre punzantes aceros, procuraba tranquilizarle, diciéndole que las
llamas y espadas no eran otra cosa que una imagen del remordimiento que
\emph{desgarraría el alma del recién nacido} si llegaba a vender los
secretos de la sociedad. Con esto quedaban terminadas las fórmulas, y
respiraba con libertad el iniciado viendo concluidas las pesadeces del
rito. Pero a lo mejor tomaba la palabra el \emph{Venerable}, que era por
lo común un hombre, si no digno de veneración, muy convencido de la
importancia de aquellas comedias, y le espetaba un discursazo, llamado
entre ellos \emph{pieza de arquitectura}, encareciendo la sublimidad de
la masonería y revelándole algo de lo concerniente al grado primero o de
aprendiz. Éste dejaba de llamarse Juan o Pedro, y tomaba con singular
modestia el nombre de Catón, Horacio Cocles, Leibnitz u otro cualquier
personaje célebre.

No puede formarse juicio exacto de la masonería por lo que esta
institución ha sido en España. Los masones de todos los países declaran
que la sociedad del compás y la escuadra existe tan sólo para fines
filantrópicos, independientes en absoluto de toda intención y propaganda
políticas. En España, por más que digan los sectarios de esta orden,
cuyos misterios han pasado al dominio de las gacetillas, los masones han
sido en las épocas de su mayor auge, propagandistas y compadres
políticos. Tampoco puede formarse juicio de la masonería española de
antaño por los restos de ella que existen hoy, y que, al decir de los
devotos, se reducen a unas juntillas diseminadas e irregulares, sin
orden, sin ley, sin unidad, aunque cumplen medianamente su objeto de dar
de comer a tres o cuatro hierofantes. Esta antigualla oscura, que
algunos sostienen como una confabulación caritativa para fines positivos
o menudencias individuales y para protegerse en uno y otro continente
(por lo cual son masones casi todos los marineros que hacen la carrera
de América), no tiene nada de común con la asociación de 1820.

Era ésta una poderosa cuadrilla política que iba derecha a su objeto;
una hermandad utilitaria que miraba los destinos como una especie de
religión (hecho que parcialmente subsiste en la desmayada y moribunda
Masonería moderna), y no se ocupaba más que de política a la menuda, de
levantar y hundir adeptos, de impulsar la desgobernación del Reino; era
un centro colosal de intrigas, pues allí se urdían de todas clases y
dimensiones; una máquina potente que movía tres cosas: Gobierno, Cortes
y Clubs, y a su vez dejábase mover a menudo por las influencias de
Palacio; un noviciado de la vida pública, o más bien ensayo de ella,
pues por las logias se entraba a \emph{La Fontana} y \emph{La Cruz de
Malta}, y de aprendices se hacían diputados, así como de
\emph{Venerables} los ministros. Era, en fin, la corrupción de la
masonería extranjera, que al entrar en España había de parecerse
necesariamente a los españoles.

Durante la época de persecución, es notorio que conservó cierta pureza a
estilo de catacumbas; pero el triunfo desató tempestades de ambición y
codicia en el seno de la hermandad, donde al lado de hombres inocentes y
honrados había tanto pobre aprendiz holgazán que deseaba medrar y
redondearse. Apareció formidable el compadrazgo, y desde la simonía, el
cohecho, la desenfrenada concupiscencia de lucro y poder, asemejándose a
las asociaciones religiosas en estado de desprestigio, con la diferencia
de que éstas conservan siempre algo del simpático idealismo de su
instituto original, mientras aquélla sólo conservaba, con su embrollada
y empalagosa liturgia, el grotesco aparato mímico y el empolvado
\emph{atrezzo} de las llamas pintadas y las espadas de latón.

A medida que iba avanzando el triunfo iba decayendo el ritual masónico,
simplificándose los símbolos, relajándose la disciplina en lo relativo a
juramentos, pruebas, iniciación. Por eso hemos visto tan empolvados y
rotos los tarjetones y huesos de la \emph{Cámara de Meditaciones}, cuya
inutilidad empezaba a ser reconocida. Es propio de gente tocada del afán
de codicia el no preocuparse de detalles tontos, y bien se sabe que
hambre o ambición no tienen espera.

\hypertarget{vii}{%
\chapter*{VII}\label{vii}}
\addcontentsline{toc}{chapter}{VII}

---Gracias a Dios que se te ve por aquí---dijo Canencia dando un
apretado abrazo al joven.---Sé que has venido de Francia hace más de
veinte días\ldots{} ¡tunante! y no te has dignado dar una vuelta por la
logia\ldots{} ¡cuando sabes que te queremos tanto; cuando sabes que los
señores te estiman mucho y desean hacerte hombre de pro\ldots!

---Por tener ocupaciones graves no he podido venir---repuso Monsalud
sentándose.---Me han dicho que esto anda muy revuelto, papá Canencia.

---No es esto un modelo de paz y concierto---dijo Canencia con cierto
desconsuelo.---Las diversiones crecen, y la reciente fundación de los
comuneros ha hecho mucho daño a la sociedad\ldots{} ¿Y tú en qué
piensas? Me han dicho que los negocios del duque del Parque te dan de
comer\ldots{} lo celebro.

---Vivo regularmente; no como ustedes, los hombres mimados de la
situación, que están hechos unos bajás.

---¿Lo dices por mí? ¡pobre Aristogitón!---exclamó Canencia con
filosófica humildad.---Yo no disfruto otras delicias de Capúa que las
emanadas de un miserable destino en Correos. Pero estoy contento,
contentísimo. Ya sabes que no soy ambicioso, que me precio de filósofo
en la verdadera acepción de la palabra\ldots{} Hijo mío, un pedazo de
pan, un vaso de agua clara, un buen libro, un tiesto de flores: he aquí
mis tesoros, he aquí mis necesidades, he aquí mi sibaritismo. Recordarás
lo que dice el gran Juan Jacobo acerca de\ldots{}

---Yo no recuerdo nada.

---Pues el filósofo de los filósofos dice que no hay verdadera felicidad
sin sabiduría\ldots{} ¡Oh!, ¿de qué sirven las grandezas humanas? Hasta
el heroísmo es cosa que no tiene simpatías, porque, como dice el
Ginebrino, «la continuidad de pequeños deberes bien cumplidos no exige
menos fuerza moral que las acciones heroicas». Mira tú cómo un hombre
humilde, que no va más que de su casa a la de Correos y de la casa de
Correos a la suya o a la logia, y carece de esposa y de prole, puede ser
un grande hombre, es decir, un sabio, o si lo quieres más claro, un
hombre feliz\ldots{} Que suban los comuneros, que bajen o suban o se
estén quedos los masones\ldots{} es cuestión que no me importa mucho. El
zoquete de pan, la cántara de agua, el tiesto de flores y el buen libro
no han de faltar. Convéncete, ¡oh joven inexperto!, de que la ambición
no ocasiona más que disgustos y enfermedades en el hépate\ldots{} en el
hígado, para hablar claramente\ldots{} Se me figura que tú estás
carcomido por la ambición, ¿eh? Tú traes algo entre manos. Dime---añadió
poniéndole la mano en el hombro con patriarcal cariño,---¿por qué has
escrito aquella carta a Campos, diciéndole que te retiras de la
masonería y poniéndonos de oro y azul?\ldots{} ¿Tratas de pasarte a los
comuneros? Ahí tienes una apostasía que me parece tonta. Pareces un
chiquillo. El creer que esto es una casa de locos no es motivo para
querer salir de ella, señorito Aristogitón. Quédate aquí, quédate sin
perjuicio de que, \emph{in foro conscientiæ}, te rías un poquillo de la
parte externa, ¿entiendes? Yo también, si he de decirte la verdad, me
río algunas veces.

---Pues si usted se ríe, amigo D. Bartolo---dijo Monsalud siguiendo el
consejo del anciano,---es un hipócrita, porque usted es el hermano
secretario y orador de la sociedad; usted es el erudito, el que explica
las leyes de la masonería, el consultor general, el que lo sabe todo
dentro de esta casa, el que ordena los ritos, el que explica lo que los
demás no entienden; usted es el sacerdote, el mago, el patriarca, el
senescal, el archimandrita, el santón, el hierofante o no sé qué nombre
darle, porque no sé todavía qué especie de religión, secta o jerigonza
es ésta. Usted es el que predica cosas enrevesadas y enigmáticas que no
entendemos; usted es el que dibuja garabatos en los diplomas; usted,
asistido de su ayudante, el señor Regato, fue quien puso aquí esos
huesos y esas calaveras que están abriendo la boca para decir que las
vuelvan a la tierra; usted escribió estos tarjetoncillos y puso las
granadas abiertas, las columnas, los triángulos y la soga, y lo que
llaman el \emph{Delta}, el Sol, la Luna, el dosel, la J y la B, el cirio
y demás signos y majaderías. Si después de hacer esto se ríe usted de
los masones\ldots{} vamos, se comprende en qué consiste el ser sabio y
filósofo.

Durante el discursillo, el anciano Canencia sonreía socarronamente,
acariciándose la barba. Cuando le tocó hablar volvió a poner su mano en
el hombro del amigo, y bondadosamente le dijo:

---Tú no sabes que al pueblo, al vulgo, al común de las gentes, o como
quiera llamarse a esa turbamulta ignorante e impresionable, es preciso
meterle las ideas por los ojos? Ya es un gran adelanto que hayamos
desterrado los símbolos y fórmulas absurdas de las religiones. Para
inculcar en esas cabezas de estuco el culto y veneración del Ser Supremo
hay que proceder con paciencia. ¿Hemos de decirles que lo mejor es
adorar a Dios bajo la bóveda de los cielos? No, mil veces no; mientras
haya hombres es preciso que haya simbolismos, y mientras haya simbolismo
es preciso que haya imágenes, o a falta de imágenes, garabatos, cositas
raras y de difícil inteligencia\ldots{} Vaya, amiguito, no repitas la
vulgaridad de que soy un farsante. Equivaldría esta calumniosa especie a
llamar farsantes al Papa y demás gigantones del catolicismo, y no lo
son: dentro de su esfera, bajo su punto de vista, no lo son\ldots{} Lo
que yo siento es que la gente va perdiendo el respeto al ritual, y
llegará día en que miren todo esto como miran los curas dentro de la
sacristía los objetos de su oficio. ¡Pícara humanidad! Verdaderamente es
una bestia. No se la puede tratar sino a palos. Acá para entre los dos,
Aristogitoncillo de mil demonios, desde que se planteó aquí la libertad,
voy creyendo que Atila, Omar, Felipe II y Bonaparte han tratado a los
hombres como se merecen. ¡Mientras todo no vuelva al estado
primitivo!\ldots{} Pero tú no entiendes de esto, ¿no es verdad? ¡El
estado primitivo! ¡Ah! ¡Imagínate el estado anterior a este funesto
pacto que hemos hecho para destrozarnos los unos a los otros y hacernos
todo el daño posible!\ldots{} No hay nada comparable al pacto. La
verdadera sabiduría debe dirigirse a ese fin; un fin, muchacho, que
consiste en volver al principio. Mas no puede formar idea de esto quien
está devorado por la ambición y tiene lleno el espíritu de ansiedades
mundanas, en vez de conformarse a vivir modesta y primitivamente con un
pedazo de pan y un vaso de agua cristalina, un tiesto de flores y un
buen libro\ldots{}

Monsalud no podía tener la risa. Durante un rato, Canencia, poniéndose
las antiparras, siguió \emph{burilando}, o sea escribiendo \emph{la
plancha}, o mejor, el acta.

---Tú te ríes---dijo en el momento en que echaba polvos para volver la
hoja---porque crees que ganarse la vida de esta manera no cuesta
trabajo. Niño mimado de la fortuna, yo quisiera saber qué sería de ti
sin la prebenda que tienes en casa del duque del Parque.

---Las prebendas---repuso Salvador,---no existen hoy sino en este manejo
de la J y la B, y en este cepillo o tronco masónico, que es el mejor del
mundo después del de las Ánimas. ¡Ah, papá Canencia, ya podía usted
echar un remiendo a estas pobres calaveras, que están diciendo con sus
bocas sin lengua la inmensa tacañería del sacristán mayor de este
templo?

---Así como no tienen lengua para pedir---dijo D. Bartolomé con
malicia,---tampoco tienen paladar, y puesto que no comen más que polvo,
no puede haber cocina más económica, y limpiarlas sería ponerlas a
dieta. Bien dijo el otro que en polvo nos hemos de convertir.

---No lo dije por usted, que se está convirtiendo en momia de Egipto
forrada en oro y plata, por obra y gracia de los misterios de Isis, de
Eleusis o de Patillas.

---Ésa es la opinión de esos bobos de comuneros---dijo Canencia, algo
amostazado.---¿Por ventura este granuja se nos ha hecho comunero?

---Tal vez---replicó Salvador.---Allá parece que están por la
formalidad. ¿Hay también cepillo y colectas?

---Más que aquí. Pregúntaselo al Sr.~Regato, que ha contribuido a fundar
aquella sociedad después de haber comido a dos carrillos en nuestro
plato y hecho \emph{salvas} con nuestra \emph{pólvora}.

Los masones llamaban al vino \emph{pólvora roja}, al vaso \emph{cañón},
y a los brindis \emph{salvas}. No es fácil comprender la misteriosa
relación simbólica entre la embriaguez y la artillería.

---Pero te advierto---continuó Canencia,---por si es tu intención
pasarte a los comuneros, que aquí no tienes más que boquear para obtener
lo que mejor te cuadre. Campos te quiere mucho\ldots{} Anoche mismo
habló mucho de ti, y aun se me figura que te va a sorprender con un buen
regalito. Has hecho bien en venir esta noche.

---Lo celebro, porque vengo a pedir.

---¿A pedir?\ldots{} Gracias a Dios, hombre. Eres de los nuestros. Veo
que entras en el buen camino---dijo Canencia mirando su reloj.---El acta
está lista. Ya es hora de empezar la \emph{tenida}. ¿Y qué pides?

---Dígame, Sr.~Canencia---preguntó Monsalud con gran interés:---¿cuál es
el criterio del Orden respecto a la suerte de los que están presos por
conspiraciones absolutistas?

---¿Cuál ha de ser? Que los ahorquen. ¿Te has echado a filántropo? ¿Hay
algún pariente tuyo en la cárcel de Villa?

---Sí, señor; hay un pariente mío en la cárcel de la Corona---repuso
Salvador con firmeza,---y es preciso sacarlo de allí.

---¿Es rico?

---Es pobre.

---Pues veo muy difícil que tu pariente coma los buñuelos de San Isidro
de este año\ldots{} Sin embargo, puedes trabajar. Campos te quiere
mucho. El Duque pertenece al Supremo Consejo. Ya sabes que lo que aquí
se ata, atado será en el Gobierno, y lo que allá dentro desatemos,
desatado será\ldots{} allá arriba. Esta noche, después de la
\emph{tenida} ordinaria, hay \emph{tenida de Príncipes del grado} 31.
Creo que se tratará de cosas muy altas. Si consigues tener de tu parte a
Campos\ldots{}

---En la \emph{tenida} ordinaria, ¿quién preside esta noche?

---El mismo Campos\ldots{} Ya comienza a venir gente. Señor Aristogitón,
orden y compostura.

Ambos personajes se trasladaron a la sala de \emph{Pasos perdidos},
donde encontraron varias personas. La concurrencia aumentaba cada
instante con la entrada de nuevos hermanos, entre los cuales los había
de todas clases, edades y figuras; muchos militares, aunque sin
uniforme, y no pocos clérigos, aunque sin hábitos. El hermano
Aristogitón, que por espacio de algunos meses había estado
\emph{dormido}, saludó a sus compañeros de taller. Pasó algún tiempo en
animadas conversaciones particulares hasta que el templo \emph{fue
descubierto}, mejor dicho, se abrió una puertecilla que daba entrada a
la logia.

\hypertarget{viii}{%
\chapter*{VIII}\label{viii}}
\addcontentsline{toc}{chapter}{VIII}

La logia era un salón cuadrangular, muy mal alumbrado y peor ventilado,
de techo plano y no muy alto, de paredes sucias y más parecido a cuadra
o almacén que a templo de una religión que dicen tenía entonces en todo
el mundo ocho o diez mil logias. En los cuatro testeros otras tantas
palabras de doradas letras indicaban los puntos cardinales,
correspondiendo el \emph{Oriente} a la presidencia, presbiterio,
\emph{santa sanctórum}, altar mayor o como quiera llamársele, a cuyo
sitio, más elevado que el resto del local, se subía por tres escalones.
Para que todo se pareciera a un recinto religioso serio, había un
doselete de terciopelo, en cuyo centro resplandecía un triangulillo, al
cual, para hablar con la menor claridad posible, llamaban ellos
\emph{Delta}. Dentro de él se veían unos garabatos que indicaban el
nombre de Dios puesto en hebreo, también para mayor claridad; pero ya es
sabido que ningún signo masónico ha de estar al alcance de los tontos.
Lo que sí se entendía perfectamente era el Sol y la Luna, dos
caricaturas de aquellos astros pintadas a derecha e izquierda del Delta,
o como si dijéramos, al lado del Evangelio y al de la Epístola.

En igual disposición respecto al presidente estaban los sitios del
hermano Orador y del secretario. Cierto es que las mesillas de que se
servían fueran más útiles teniendo la forma cuadrada; pero era
indispensable no abandonar el triangulillo siempre que se pudiera, y por
esto las mesas eran de tres picos. También tenían un poco más abajo
bufetes trípicos el Tesorero y el Hospitalario. En el remoto Occidente,
es decir, junto a la puerta, se elevaban dos columnas rematando en
granadas entreabiertas. Una columna tenía la J y otra la B, letras que
al parecer querían decir \emph{Juan Bautista}, pues también al precursor
del Mesías le metieron de cabeza en la heterogénea liturgia masónica,
donde los misterios egipcios y mil desabridas fábulas se mezclan
gárrulamente con el mosaísmo, el paganismo, la religión cristiana, la
revolución inglesa y la filosofía del siglo de Federico. Junto a las
columnas se repetían las mesillas triangulares, una para el primer
vigilante y otra para el segundo.

El techo no carecía de interés. Por encima del doselete destinado a
guarecer la calva del Presidente, asomaban unas listas doradas
representando los rayos del sol con dudosa fidelidad. En el friso había
varios garabatos, obra de indocto pincel, a los cuales se atribuían
intenciones de querer expresar los signos del zodiaco; y por debajo de
ellos corría, también pintada, una soga, símbolo de unión y fuerza. La
estrella pitagórica andaba también de paseo por aquellos altos cielos,
testimonio de grandeza del Supremo \emph{Demiurgos} (Dios), y en su
centro llevaba la letra G., significando \emph{gnos}, palabreja que
hasta los niños entienden, sin necesidad de aprender, que significa
\emph{generación}. Completaban el sublime ajuar cuatro candelabros con
sendas \emph{estrellas}, que en el mundo ordinario llamamos velas, y por
último, la consabida batería de trastos, espada ondulante, compás,
escuadra y el ejemplar de los Estatutos. No había ventanas ni más
puertas que la de entrada, porque era de rito el ahogarse.

El \emph{Venerable} o Presidente era un hombre como de sesenta años, de
agradable y aún hermosa presencia, fisonomía simpática, sonrisa
esculpida, más bien de cortesía que de burla. En todo él había
marcadísima expresión de contento de la vida, un singular convencimiento
de que el mundo era bueno, y si se quiere, de que el Arte Real era
óptimo. Vestía con elegancia, y los atributos y arreos de la masonería,
que no tienen comúnmente nada de airosos, le sentaban a maravilla. Había
en su bizarra apostura corpulenta cierto aire de obispo y también algo
de hombre de mundo, sin que pudiera adivinarse cómo se verificaba la
síntesis de estos dos términos tan diversos.

Aquel personaje, que a pesar de su indudable influjo en los sucesos de
su época ha escapado, por extraño fenómeno, a las fiscalizaciones
entrometidas de la Historia, se llamaba D. José Campos. Éste era su
verdadero nombre, y no anagrama impuesto por el novelador para tapar una
celebridad; mas no lo busquéis en la Historia, como no sea en algún
olvidado y oscuro libro de masones; buscadlo en la \emph{Guía de
forasteros}, porque era director general de Correos.

A pesar de la poca resonancia de su nombre, a pesar de no estar asociado
a ningún ministerio, a ningún gran discurso, ni menos a batallas o
sediciones, es indudable que el portador de él fue uno de los hombres
más importantes del célebre trienio. A él se debió la organización de la
Masonería en aquel pie de ejército poderoso. Lo que no se comprende
fácilmente es la razón de su modestia. Campos no quiso nunca salir de la
Dirección de Correos, aunque su familiaridad con ministros, generales y
consejeros le ponía en la mejor situación del mundo para satisfacer su
vanidad si la hubiera tenido. De las más verosímiles tradiciones
masónicas se desprende que el \emph{Venerable} en cuestión era de los
que se agachan para dejar pasar las turbonadas y los pedriscos,
conservando siempre el mismo sitio y no dejándose arrastrar por la furia
de las pasiones, con lo cual, si aparentemente adelantan poco, en
realidad salen siempre ganando y no están sujetos a las caídas y
vaivenes de la gente muy visible y muy talluda. Más hábil vividor no lo
conocieron los pasados ni conocerán los venideros siglos.

Los anales masónicos están conformes con asegurar que Campos tenía en
las logias el nombre de \emph{Cicerón}.

Tomaron todos asiento, siendo de notar que algunos tenían mandil y
banda, y otros no. Hubo no pocos pasos de baile francés, tocamientos y
signos que no describiremos por ser demasiado conocidos. La patriarcal
fisonomía y espesa cabellera blanca de Canencia se destacaban al lado de
la Epístola, y al verle tan circunspecto y hasta con cierta expresión
beatífica, se creería que los templos elevados a la Gloria del Gran
Arquitecto \emph{Iod}, también tenían sus santos. El \emph{Venerable},
usando las fórmulas rituales, mandó al primer vigilante que \emph{se
asegurase si el templo estaba a cubierto}, y el primer vigilante,
después de hacer la pantomima de salir y volver a entrar, declaró que no
\emph{llovía}, es decir, que el templo estaba libre de entrometidos y
que podían empezar los trabajos. Un martillazo presidencial abrió éstos
en el grado convenido.

El \emph{Maestro de ceremonias}, que era uno de los oficiales
dignatarios, recorrió los asientos presentando el \emph{saco} de las
proposiciones. Algunos masones depositaron un papelillo como los que se
usan en las rifas domésticas. El \emph{Venerable} extrajo todas las
proposiciones, y escogiendo la que le pareció más grave, leyó lo
siguiente:

\emph{«Proposición de
Aristogitón}.---\emph{Gr}.\textsuperscript{\textbf{.}}. 18:
\emph{Salvador Monsalud}.---Pido a este Grande Oriente de Madrid, se
sirva declarar que reprueba las prisiones ordenadas por el Gobierno con
motivo de inofensivas conspiraciones absolutistas, y que se apresure a
interponer su mediación benéfica para que D. Matías Vinuesa y los demás
infelices encarcelados por causa del ridículo plan descubierto el 21 de
Enero, se libren no sólo de ejecución capital, sino del largo cautiverio
a que los condenará la pasión política».

Cuando el Venerable concluyó de leer, rumores de desaprobación sonaron
en la logia; pero el martillo del Venerable impuso silencio, y algunos
instantes después, Aristogitón se expresaba en estos términos:

---He presentado esa proposición por pura fórmula y para cumplir con los
Estatutos del Orden, que disponen sean tratados todos los asuntos en
sesión reglamentaria, y no en conciliábulos reservados entre dos o tres
hermanos bullidores que arreglan el mundo y la nación para su uso
particular.

Nuevos rumores interrumpieron al orador, y Cicerón, después de
acallarlos a golpes, recomendó a todos moderación.

---Temprano empiezan las interrupciones---prosiguió el masón del
gr.\textsuperscript{\textbf{.}}. 18,---y lo siento, no por mí, que estoy
dispuesto a decir todo lo que sea preciso, sino por mis queridos
hermanos, que van a perder la paciencia y la voz, si continúan
haciéndome coro hasta el fin de mi discurso\ldots{} Decía que desconfío
de que mi proposición tenga éxito aquí, a pesar de ser la expresión más
leal y clara del espíritu y de las prácticas constantes de este
respetable Orden en todos los países del mundo; y no tendrá éxito,
porque este Gran Oriente y los individuos que en diversos grados
dependen de él, han olvidado completamente los fines benéficos,
desinteresados y filantrópicos de tan antiguo Instituto, para
desvirtuarlo y corromperlo, haciéndole instrumento de intereses
políticos y de la codicia\ldots{}

El martillo del Venerable, interpretando el descontento de la asamblea,
advirtió al orador que hablaba con la pasión y vivacidad propias de un
Congreso. Cicerón rogó en breves palabras al orador tuviese presente que
aquello era un templo y no un club.

---Hermano Venerable---indicó Aristogitón;---si la condición de templo
impide a este local oír la verdad, me callaré. Cuantos me escuchan saben
ya por su conciencia lo que yo estoy diciendo. ¿Por qué no me lo han de
oír a mí, si ya lo saben, y no les digo nada nuevo?\ldots{} Continuaré,
pues, procurando ser breve y herir lo menos posible la susceptibilidad
de mis hermanos, a quienes ofende más lo dicho que lo sentido; más las
palabras que los hechos\ldots{} Al proponer al Oriente que temple en lo
posible el ardor de las luchas políticas, he querido protestar contra la
tendencia a fomentarlas y exacerbarlas. El Instituto masónico debe ser
extraño a la política, debe ser puramente humanitario, debe proteger a
los desvalidos sin pedirles cuenta de sus ideas, y aun sin conocer sus
nombres. Está fundado en la abnegación y en la filantropía. Lo dicen así
su historia, sus antecedentes, sus símbolos, que o no representan nada,
o representan una asociación de caridad y protección mutua. Lejos de
practicarse estos principios en España, el Orden se ha olvidado de los
menesterosos, constituyéndose en agencia clandestina de ambiciones
locas, en correduría de destinos y en\ldots{}

Protestas, amenazas y tal cual palabreja puramente española, que no fue
conocida de Salomón ni de Hiram-Abí, ahogaron la voz del orador. El
tumulto fue tan grande como cuando en el templo de Salomón se dispuso
que la multitud prorrumpiese en gritos para que la palabra Jehová,
pronunciada por el Gran Maestro, no llegase a oídos profanos. Del mismo
modo los martillazos de Campos-Cicerón no llegaban a profanas orejas.
Por último, entre Canencia y el Venerable, lograron restablecer el
orden.

---Esto no se puede tolerar---gritó un compañero.---Si el hermano
Aristogitón quiere abogar por los absolutistas, que tanto nos han
perseguido; si es absolutista él mismo, dígalo de una vez, sin necesidad
de insultarnos, ni de manchar tan audazmente la honra inmaculada de esta
santa Sociedad.

---Hermano Arístides, o mejor, Pipaón, pues no puedo acostumbrarme a
prescindir de los nombres verdaderos---dijo Salvador, sin perder ni un
instante su serenidad;---tú que has cantado en todos los corrales y has
venido aquí mandado por los absolutistas, para referirles lo que
hacemos, debes callar para no exponerte a que se descubra bajo la piel
de ese ridículo celo la verdadera oreja asnal de tu conciencia negra.

---Que se \emph{burilen}, que se escriban ahora mismo esos
insultos---gritó Pipaón fuera de sí.---Hermano Venerable, pido que el
Oriente formule ahora mismo el acta de acusación contra el hermano
Aristogitón y que pase a la Cámara de Justicia.

---¿Para qué se ha de escribir lo que he dicho?---añadió
Monsalud.---Mejor es que lo repita, y lo repetiré cuantas veces queráis.

---¡Orden, orden!

Cicerón rompía la mesa a martillazos.

---¡Fuera, fuera!

---Hermanos queridos---dijo el Venerable haciendo un esfuerzo para que
su sonora voz fuese oída,---tengamos calma. Ruego al orador tenga
presente que estamos en un templo, en el santo templo abierto a las
luces, a la honradez pura, a la filosofía pura, a los nobles
sentimientos filantrópicos de la humanidad toda, sin distinción de
clases, iglesias, castas, ni estados\ldots{}

---¡Bien, muy bien!

---Pues decía al orador que estamos en un templo y no en un Congreso y
menos en un club.

---¡Bien, muy bien!

---Hecha esta advertencia, y rogando a los hermanos de las columnas
septentrional y meridional que se calmen y tengan prudencia, oigamos a
nuestro hermano; que después el Oriente tomará las medidas que crea
necesarias. Adelante, hermano Aristogitón.

---Es el colmo de la insolencia---gritó un hermano sin hacer caso de los
martillazos ciceronianos,---que aquí dentro se levante una voz a
defender al cura Vinuesa y a los demás conspiradores absolutistas.

---Yo no defiendo a los conspiradores---exclamó el orador.---Lo que pido
al Oriente es protección para los que padecen, martirizados por una
populachería indigna que no sabe oponerse a las conspiraciones de la
Corona sino insultando al Rey; que no sabe sofocar las conspiraciones
realistas, porque perdona, tolera y agasaja a los hombres verdaderamente
temibles, mientras encarcela y atormenta y ahorca a infelices clérigos y
ancianos ineptos, incapaces de hacer cosa alguna de provecho contra el
régimen establecido. La populachería, a cuyo servicio se ha puesto este
Orden, no ve los enemigos reales y poderosos que se unen astutamente al
pueblo y se meten aquí, minando el terreno en que la libertad trata de
fundar, sin poderlo conseguir, un edificio más o menos perfecto. La
populachería, mientras deja de trabajar en silencio a los que odian la
libertad, se entretiene en dar tormento a la gente menuda.

»Señores masones, o señores liberales templados, que ahora todo viene a
ser lo mismo, sois como aquel emperador romano que se ocupaba en cazar
moscas, y mientras mortificaba a estos pobres insectos no veía a los
pretorianos que se conjuraban para echarle del trono. Éste era
Domiciano. Así sois vosotros. Yo quiero que variéis de conducta, y
principio por pedir que se deje en paz a las moscas\ldots{} No conozco a
Vinuesa; pero si a compañeros y amigos suyos, que comparten su suerte en
la cárcel de la Villa o de la Corona. He visto la feroz excitación que
existe en el pueblo contra ellos, y esta excitación creada y fomentada
por este Orden y más aún por la Asamblea de los Comuneros, es una
barbarie y al mismo tiempo una imprudencia política. El vil populacho a
quien instruís en el inicuo arte de hacerse justicia por sí mismo,
aprenderá al cabo, y una vez maestro, querrá dar todos los días una
prueba de esa atroz soberanía que le habéis enseñado. Tengo la seguridad
de que si el tribunal que va a juzgar a Vinuesa se mostrase benigno, la
canalla destrozaría a Vinuesa, al tribunal y luego a vosotros, que
habéis hecho creer a la bestia en la necesidad de los sacrificios
humanos. Mientras la Corte juega con vosotros y os lanza de desacierto
en desacierto para desacreditaros y para que os devoréis los unos a los
otros, os entretenéis en menudencias ridículas, os debilitáis en
rivalidades indignas y aduláis las pasiones de la canalla, que si hoy
ladra libertad, ladrará mañana absolutismo. Todo depende de la mano que
arroje el pedazo de pan.

»Poniéndome, pues, en el terreno político, a pesar de creerlo impropio
de esta Sociedad; hablando el único lenguaje que entienden aquí, declaro
que la persecución de Vinuesa, y mucho más la sañuda irritación del
pueblo contra ese hombre infeliz, me parecen una desgracia casi
irreparable para la libertad, un mal gravísimo, que este Orden debe
evitar a toda costa, principiando por propagar la tolerancia, la
benignidad, la cordura, y concluyendo por emplear toda influencia en pro
de los procesados. Si no se hace así, esto que llamamos templo merece
que el mejor día entren en él cuatro soldados y un cabo, y que después
de entregar todos los trastos del rito a los chicos de las calles para
que jueguen, recojan a los hermanos todos para llenar otras tantas
jaulas en el Nuncio de Toledo.»

Las últimas palabras del orador apenas fueron entendidas, a causa del
gran alboroto que se armó dentro del templo, que representaba la
grandeza y maravillosa arquitectura del mundo.

---¡Fuera, fuera!\ldots{} El mismo se ha desenmascarado y ya sabemos lo
que quiere.

---A votar\ldots{} Que se vote la proposición en escrutinio secreto.

---Ahora mismo se va a redactar el acta de acusación.

---¡Fuera!

---¡El acta de acusación!\ldots{}

---Pedimos que pierda en absoluto los derechos masónicos. Tanta
insolencia, esas brutales amenazas, la defensa de nuestros enemigos, no
pueden quedar sin castigo\ldots{}

Estas y otras frases pronunciadas en indescriptible tumulto, indicaban
la efervescencia que en el templo reinaba, y por largo rato Cicerón se
rompía las manos dando martillazos sin poder calmar las olas de aquel
mar embravecido. Al fin, auxiliado de Canencia y de otros, lograron
serenar un tanto los irritados ánimos, librando asimismo al insolente
orador de las manifestaciones un poco brutales que el grupo más
entusiasta, la columna del septentrión, si no estamos equivocados, se
dispuso a emplear contra él.

---Después de ver lo que veo me preocupa poco que se vote o no lo que he
propuesto---dijo Salvador.---Y en cuanto al acta de acusación, no se
tomen mis hermanos el trabajo de redactarla, porque no es preciso que me
expulsen. Me expulsaré yo mismo, abandonando para siempre este Orden
inútil, enfermo, podrido, que si aún respira y habla como los vivos, ya
infesta como los cadáveres.

¡Escándalo inaudito! Aunque lo normal en las \emph{tenidas} era que se
discutiera con tranquilidad, cuando la congregación salomónica se
alborotaba parecía un club de los más fogosos. Unos rugían tan cerca del
atrevido Aristogitón, que fue necesario que interviniera personalmente
al \emph{Venerable} para impedir cosas mayores entre hermanos, olvidados
de la santidad que infunde un mandil de cocinero. De las columnas
septentrionales partía el más atroz nublado de amenazas y
recriminaciones. Las columnas del Mediodía estaban más tranquilas.
Indudablemente había allí no pocos compañeros que opinaban lo mismo que
el orador, hallando tan sólo reprensible la forma violenta del discurso.

\emph{---¡Radiación, radiación!}---gritaron algunos.---Sin alborotar se
puede imponer castigo al delincuente.

\emph{Radiar} significaba dar de baja.

---Que se le inscriba en el \emph{Libro Rojo}.

Era un librote donde se inscribían los hermanos \emph{radiados} por
sentencia masónica.

---Que se vote antes por \emph{esferas} esa absurda proposición.

\emph{Esferas} llamaban a las bolas.

---Queridos hermanos---repetía el Venerable con mansedumbre,---estamos
en un templo, no en un club. Orden.

El orador se hubiera marchado de la logia sin esperar las resoluciones
del templo; pero un resto de consideración hacia los que aún le llamaban
hermano detúvole allí. Vio que Canencia desde su tripódica mesilla le
hacía señas de reprobación y pesadumbre; vio que el \emph{Venerable} le
miraba con expresión de lástima; oyó algunas palabras rencorosas de tal
cual hermano que no lejos de él tenía su asiento; observó que muchos,
mayormente los del Mediodía guardaban una actitud reservada, como
hombres demasiado prudentes que no se atreven a poner su opinión frente
a la opinión de la mayoría; vio después que votaban su proposición, y
por unanimidad la desechaban; pero lo que más sorpresa le causó fue que
en la sala de \emph{Pasos perdidos}, concluida la sesión, le dijera al
oído algún hermano de los más callados bajo la \emph{bóveda del
Universo}:

---Hermano Aristogitón, yo pienso como usted en lo de dejar en paz a las
moscas y hacer puntería a los pajarracos; pero esto no se puede decir
aquí. Conviene seguir la corriente y no chocar con la mayoría. A donde
nos lleven iremos.

Y otro le dijo, también en secreto:

---Lo mismo que usted hubiera dicho yo, aunque en tono menos agresivo.
No conviene ensoberbecer al pueblo ni adular sus instintos sanguinarios,
pero, amigo, la consigna de estos días es sacrificar algún absolutista a
la implacable furia populachera, y como no ha caído en nuestras redes,
ni caerá, ningún tiburón, fuerza es echar en la sartén los pececillos de
redoma. Vinuesa morirá.

Y un tercero le dijo, también en secreto:

---Le hubiera aplaudido a usted con toda mi alma; pero, amigo, estas
cosas se sienten y no se dicen. Ni vale la pena de que pierda uno su
destino y el pan de sus \emph{lobatones} (hijos) por una apreciación
política. Yo creo que esto se lo lleva la trampa. Estamos dentro de un
torbellino que nos arrastra, nos hace dar mil vueltas, nos marea, y no
para nunca, y nos llevará a donde quiera el Gran \emph{Demiurgos}. Creo
que hace usted mal en manifestar tan crudamente sus ideas. La masa
popular tiene ya a Vinuesa entre los dientes, y no seré yo el guapo que
pretenda quitárselo. Ese clérigo es bastante criminal, es un disoluto,
un perdido. ¿Por qué le defiende usted?

Y un cuarto le dijo, en secreto también:

---Siento mucho que le tengamos que \emph{radiar} a usted y apuntarlo en
el Libro Rojo, pero no hay más remedio. No se puede tratar al Orden como
usted lo ha tratado\ldots{} Por mi parte, acepto esa idea de no hacer
caso del bajo pueblo: pero ¿quién le pone el cascabel al gato? Soltamos
los mastines, y ahora tenemos que andar brincando y corriendo huyéndoles
el bulto para que no nos muerdan. Si he de hablarle a usted con
franqueza, creo que nada se pierde con quitar de en medio a los autores
de ese monstruoso plan; pero al mismo tiempo opino, como usted, que hay
otros peores, sí señor; otros que trabajan en obra fina, y no digo
más\ldots{} Dios nos tenga de su mano, Aristogitón, y lo que fuere
sonará\ldots{} Allí veo a Argüelles, a Calatrava y a Feliú que acaban de
entrar. Esta noche hay \emph{tenida de Maestros Sublimes
Perfectos}\ldots{} Parece que en Palacio anda la cosa mal, y que las
Cortes nuevas no serán muy sumisas\ldots{} Yo me voy, porque, según me
ha dicho Campos, debo perder la esperanza de un ascenso por ahora.

Y un quinto le dijo en voz alta:

---¡Buena la has hecho\ldots! Yo que pensaba decirte que te empeñaras
con Campos para que me trasladasen a la vacante de la secretaría\ldots{}

---El duque del Parque acaba de entrar---le dijo un sexto.---Hay
\emph{tenida de Valientes y Soberanos Príncipes}. Sentiré que te
\emph{radien}, hermano Aristogitón. Aunque grité contra ti y te llamé
insolente y procaz, no hagas caso. Somos amigos. Algo de lo que dijiste
me gusta; principalmente, el apóstrofe a Pipaón. Ese canalla va a ser
presentado esta noche en un grado superior. No hay quien pueda con él.
¿Creerás que la plaza que estaba destinada para mí la pescó Pipaón para
su criado?

Otros pasaban sin mirarle o mirándole con provocativo enojo.

Mientras entraban diversos hermanos, que en el siglo respondían a los
nombres de Quintana, Argüelles, Valdés, San Miguel, etc., salieron
otros, entre los cuales también había nombres que después fueron
ilustres, pero que callamos por varias razones.

Quedose Monsalud en la sala de \emph{Pasos perdidos}, esperando el
resultado de la \emph{tenida de Maestros Sublimes Perfectos}.

La logia se iba a abrir en uno de los grados superiores.

\hypertarget{ix}{%
\chapter*{IX}\label{ix}}
\addcontentsline{toc}{chapter}{IX}

Duró la reunión de los padres graves bastante tiempo, porque además de
que en ella trataron diversos asuntos de política elevada, hubo admisión
de un hermano que había recibido \emph{aumento de salario}, es decir,
ascenso en la escala masónica. La ceremonia de recepción en los grados
superiores no era más seria que el grado de aprendiz, y se hablaba mucho
de la \emph{Acacia}, de la \emph{Sala de en medio}, de la \emph{Luz
opaca} y otras lindezas. Para explicarlas sería preciso entrar con brío
en la leyenda del Arte Real; pero como ésta y cuanto a ella se refiere
es fastidioso en grado sumo, recomendando al lector se abstenga de
perder el tiempo averiguando el significado de los millares de emblemas
diversos usados por las doscientas o trescientas disidencias o cisma del
primitivo Francmasonismo, y entre los cuales el rito \emph{Escocés y
aceptado}, que parece predominante en nuestros tiempos, tiene por
liturgia un enredado berenjenal de alegorías, entre místicas y
filosóficas, donde fracasa la más segura y sólida cabeza.

Los \emph{Maestros Sublimes Perfectos} se retiraron muy tarde, y a la
madrugada no quedaban en el local más que cuatro individuos, reunidos en
torno a la mesa en la \emph{Cámara de Meditaciones}. Eran
\emph{Cicerón}, Monsalud, D. Bartolomé Canencia y otro cuyo nombre y
persona serán conocidos en el transcurso del diálogo. Este (que acababa
de entrar concluidas las sesiones) y Canencia fijaban su atención en
unos papeles llenos de guarismos y en un saquillo de monedas, contando a
ratos y a ratos apuntando cifras. Los otros dos hablaban.

---La \emph{Cámara de Perfección}---dijo Campos,---no ha querido
mostrarse severa contigo. Ha decidido que no seas \emph{radiado} por
ahora, y que, en vez de \emph{dormir}, pidas una licencia ilimitada, que
se te dará.

---Tonterías y debilidades---respondió Salvador riendo.---Ni yo quiero
licencia, ni la necesito, ni la pediré, ni me importa que me
\emph{radien} o me escriban en todos los libros rojos o amarillos.

---Hazme el favor---indicó Campos con socarronería,---de no echártela de
hombre superior. No valemos tan poco como crees. El discursillo de esta
noche, que tan justamente alborotó la logia, y la carta que me
escribiste renunciando las comisiones que yo quería encargarte en
provincias, me prueban que estás en un período de hipocondría o satánico
orgullo\ldots{} Sr.~Aristogitón, hay que civilizarse; hay que aceptar
las cosas como son; hay que renunciar a esos humos de hombre puro, so
pena de anularse y caer en triste olvido\ldots{} Es particular: yo te
alargo la mano para sostenerte y elevarte, y me la rasguñas. ¡Pobre
gatillo inocente! El discurso de esta noche bastaría para expulsarte
definitivamente de entre nosotros, y, sin embargo, gracias a mí te
quedarás; gracias a mí\ldots{}

---Para nada quiero seguir.

---Seguirás---repitió Campos con benévola insistencia,---y no sólo
seguirás, sino que nos serás útil. ¡Tunante! Más de cuatro quisieran
verse en tu lugar. Has de saber que tus salidas de tono y tus desaires,
en vez de ocasionarte disgustos, te proporcionan gangas. Ya verás qué
pedrada te voy a dar esta noche.

---A nada conduce tanto hablar, Sr.~Campos---repuso Aristogitón con
impaciencia.---Es tarde: de una vez dígame usted si han tratado esos
señores algo referente a Vinuesa y su conspiración.

---Eres en verdad sospechoso. ¿En qué se funda tu interés por ese Gil de
la Cochera, de la Cuadra o no sé de qué?

---Es pariente mío.

---¿Cercano?

---Muy cercano.

---Quizás sea su padre---dijo para sí.---Estos hijos de nadie se exponen
a que de buenas a primeras les salga un padre en cualquier calabozo».

---¿Se ocupan de esto? sí, o no.

---Nos ocupamos, sí. El castigo de Vinuesa y sus cómplices es una de las
cosas que más preocupan a la gente política. No han sido olvidados otros
asuntos graves, como la disolución del cuerpo de Guardias, los insultos
al Rey, las nuevas Cortes, que se abrirán dentro de unos días; la
sociedad de los comuneros, que está metiendo demasiado ruido, y las
partidas de guerrilleros que comienzan a aparecer. Es un hormiguero de
asuntos graves, que hacen de España un país de delicias.

---Por supuesto, no habrán resuelto nada. Los \emph{Maestros Sublimes
Perfectos} se parecen al Gobierno como una calabaza a otra. Aquí como
allí se procede de la misma manera. Habrán decidido que no conviene
absolver a Vinuesa ni tampoco condenarlo; que no conviene castigar a los
insultadores del Rey ni tampoco alentarles; que el cuerpo de Guardias
está bien disuelto, pero que se debe crear otro; que la mejor manera de
acallar el ruido que hacen los comuneros es alborotar mucho aquí; que
las nuevas Cortes no son buenas, pero tampoco malas, y que la política
debe ser exaltada para contentar al populacho, y al mismo tiempo
despótica para contentar a la Corte.

---Atacas el justo medio, que es el arte político por excelencia,
bribón---dijo Campos riendo.---¿Tú qué entiendes de eso? Sin este tira y
afloja, sin esta gracia de Dios que consiste en no hacer las cosas por
temor de hacerlas a disgusto de Juan o de Pedro, no hay Gobierno
posible.

---En una palabra: los \emph{sublimes} no han decidido nada. Ya dijo
Voltaire hace muchos años: \emph{«La masonería no ha hecho nunca nada,
ni lo hará».} Tenía razón.

---Protesto---gritó Canencia, apartando por un momento su atención de
las monedas, de los guarismos y del amigo que con él contaba y
escribía.---El buen Aroüet no ha dicho semejante cosa. No calumniemos al
gran filósofo, señores.

---Quienes le calumnian, querido Sócrates---dijo Campos en un momento de
ira,---son los volterianos que fuera de aquí se fingen beatos para
halagar a los curas.

---Pero si halagan a los curas honrados---repuso Canencia volviendo a
contar,---no trabajan por la impunidad de los curas absolutistas, que
escandalizan al país con sus conspiraciones\ldots{} Cuarenta y cinco
reales en medias pesetas.

---Usted, papá Sócrates---dijo Monsalud con mal humor---reparta el
\emph{dinero de la Viuda} y deje lo demás.

---Volviendo a nuestro asunto, hermano Aristogitón---manifestó
Campos,---te conviene mucho no meterte a redentor de cautivos. El Grande
Oriente no puede aplacar la efervescencia del pueblo contra Vinuesa ni
absolver a éste, aunque hará todo lo posible porque no se le condene a
muerte, ni tampoco pondrá en libertad al de Tamajón, ni a tu Gil de la
Cuadra, porque si lo hiciera, se supondrían complicidades absurdas. Ya
sabes lo que es el vulgo\ldots{} y por más que digan, los Gobiernos
deben dar algo al señor vulgo en compensación de lo mucho que a todas
horas le piden.

---Pues yo me retiro---dijo Monsalud resueltamente.

---Aguarda, torpe, ingrato. Te he dicho que iba a darte una pedrada esta
noche.

---No estoy para bromas.

---Vamos, será preciso cogerte con lazo, y luego atarte las manos para
que no des bofetadas a tus favorecedores.

Campos sacó del bolsillo un pliego doblado en cuatro.

---Aquí tienes tu destino.

---¿Qué destino?---preguntó el joven con asombro.

---No te hagas el tonto, Salvador, ni vengas acá con ridículas y
mentirosas modestias. Con esta clase de latigazos se domestica a las
fieras catonianas. Ya sé que no te gusta pedir nada; ya sé que te falta
boca para proclamar tu horror a los destinos públicos y censurar la
ambición y a los ambiciosos. Todos hacemos lo mismo; pero cuando nos dan
algo\ldots{} lo tomamos.

---Yo no entiendo una palabra de lo que usted me dice.

---Vamos, que no te falta ya más que hacerte anacoreta y excomulgarme
por favorecerte. No tanto, joven modesto. Aquí tengo una credencial de
treinta mil reales, una canonjía admirable en la secretaría del Consejo
de Indias. Poco trabajo, ninguna responsabilidad. Con los suspiros que
otros han exhalado por esta plaza se podría dar a la vela un navío. El
ministro, al dármela esta noche en el capítulo, me dijo que desde que
vacó ese puesto lo han solicitado unos cien o doscientos \emph{adictos}.
Pero yo la había pedido para ti con muchísimo empeño, y el ministro no
podía desairarme; el ministro me ha dado la plaza a pesar de tu
irreverente y sacrílego discurso de esta noche.

---Estoy muy agradecido a usted; pero no acepto.

---Es el primer caso que veo en España, querido Salvador---dijo Cicerón
con la malicia escéptica que le era habitual;---es el primer caso que
veo de un hombre a quien le dan esta bendición de Dios que yo tengo en
la mano y se queda sereno y frío como tú estás ahora. Tú no eres hombre,
tú no eres español.

---Pero ¿usted, por su propia iniciativa, ha pedido para mí ese destino
no habiéndolo solicitado yo?---preguntó el joven, tratando de averiguar
el motivo de aquella protección sospechosa.

---Hombre, la verdad\ldots{} a mí no se me ocurría tal cosa; pero mi
sobrina Andrea, que a todo atiende, que todo lo prevé, que sabe tan bien
adivinar las necesidades, me dijo no hace muchos días: «Es una vergüenza
que hayan colocado tanta gente inepta y esté sin destino Salvador
Monsalud». Comprendí que tenía razón, y le contesté que tú nunca habías
pedido nada y que en la casa del señor duque del Parque estabas muy
bien\ldots{} Ella me dio a entender que deseas la plaza.

---¡Yo!

---Tú. Andrea es excelente, es caritativa como ninguna, y estima mucho a
todos mis amigos. Me ha dicho que habías estado en casa a verme; que no
hallándome, esperaste largo rato; que estabas meditabundo y
cariacontecido; que te dio conversación para distraerte; que hablando de
cosas de la vida, le diste a entender con frases delicadas y parabólicas
que deseabas un buen empleo; en suma, según mi sobrina, tú le rogaste
con buenos modos que influyera conmigo para que el Grande Oriente te
proporcionara una pingüe colocación.

---¡Qué falsedad!\ldots{} ¿pero lo dice usted seriamente?---exclamó
Monsalud con ira.

---¿Desmentirás a mi sobrina?

---Yo no desmiento a nadie. Simplemente digo que muchas gracias y que
guarde usted su credencial para otro.

Diciendo esto, Salvador clavó tenazmente los ojos en el semblante de
Cicerón, tratando de leer en él los móviles de conducta tan extraña.
Aquella extemporánea protección del \emph{Maestro Sublime Perfecto},
otorgada precisamente a quien acababa de hacer a la congregación una
ofensa grave, encerraba sin duda algún misterio. Conocía bastante
Monsalud el carácter de Campos para creer en su benevolencia, y conocía
bastante el Orden para suponerle capaz de dar a los que no pedían. Ni
consideraba tampoco verosímil la intervención de Andrea en aquel asunto.
Hizo diversos juicios y sentó varias hipótesis; pero ni de aquéllos ni
de ésta resultó nada correcto. También fue inútil la observación
analítica del plácido rostro de Campos, pues el gran masón no era hombre
que a su cara permitía vender los secretos del entendimiento.

---Yo lo agradezco mucho---repitió el joven;---pero de ningún modo puedo
aceptar.

---Basta; para fórmula modesta, para vergüencilla de niño bien educado,
basta ya ---dijo Campos burlonamente.---Pues eso que ahora te doy no es
más que para hacer boca. Ya he hablado al ministro de enviarte a
desempeñar una de las superintendencias de Indias, con la cual puedes
ser hombre rico en diez años.

Aquel proyecto de envío a Ultramar, aumentando al principio la confusión
del joven, confirmó sospechas dolorosas que en su alma empezaban a
nacer.

---¡Repito que no y que no!---dijo con la mayor energía.,---Muchas
gracias por todo; pero celebraré que no me vuelva usted a hablar de eso.

---Entonces---indicó Campos, cruzando los brazos en señal de
perplejidad,---pide por esa boca. Imagina algún imposible: pide la luna,
a ver si te la podemos dar.

---Lo que deseo, ya lo pedí en la tenida.

---Pues eso es un disparate. Ya te he dicho que no podemos decidir nada.
Hay cuestiones que no se resuelven sino dejándolas sin resolución. ¿Te
ríes?\ldots{} ¡Maldita sea tu filantropía! Yo quisiera comprender en qué
consiste tu interés por Gil de la Cuadra.

---En que le debo la vida.

---¿Y qué es eso de deber la vida?

---Una cosa que no entienden los egoístas.

---Tú estás loco---dijo Cicerón, haciendo gestos de
desdén.---Sr.~Regato, ¿qué le parece a usted la pretensión de nuestro
joven filántropo?

El Sr.~D. José Manuel Regato alzó los ojos del montón de dinero para
fijarlos en el cercano grupo. Hombre tan célebre merece algunas líneas.

\hypertarget{x}{%
\chapter*{X}\label{x}}
\addcontentsline{toc}{chapter}{X}

Era de mediana edad y fisonomía harto común, ni alto ni bajo, moreno y
curtido de rostro, a excepción de la frente, que era muy blanca. Sus
pobladas cejas negras y el pelo espeso y cerdoso indicaban fortaleza.
Había en sus ojos la vaguedad singular propia de los tontos o de los que
aparentan serlo, y a menudo reía, como tributando de este modo
complaciente lisonja a cuantos le dirigían la palabra. Vestía
completamente de negro, asemejándose por esta circunstancia a una
persona de estado eclesiástico; afectaba la más refinada compostura, y
al mirar contraía los párpados a manera de los miopes. Si los abría en
momentos de sorpresa, de miedo o de ira, distinguíanse los verdosos y
dorados reflejos de su iris, muy parecido al de los gatos. Cuando quería
hablar algo de interés iba acercándose poco a poco al asiento de su
interlocutor, y su manera de acercarse, su especialísima manera de
sentarse, arrimando el codo o el hombro a la persona, eran fiel copia de
los zalameros arrumacos del gato. Muchos habían observado esta
semejanza, y hasta en el apellido de Re-gato, es decir, reiteración en
las cualidades gatunas, hallaban motivo de burla los maliciosos.

---Antes de pedir con tanto empeño la impunidad de Vinuesa y
compañeros---dijo D. José Manuel,---yo me pondría en paz con Dios por lo
que pudiera tronar. Defendiendo a tales víctimas se corre el peligro de
ser una de ellas. Gil de la Cuadra es uno de los peores. ¡Valiente
pajarraco defiende usted, amiguito Monsalud! Con la mitad de lo que él
ha hecho se va de bureo a la plazuela de la Cebada. No es crueldad,
señores; pero si a este candoroso anciano no le ponen la corbata de
cáñamo, no hay justicia en el mundo.

---A quien hay que poner la corbata de cáñamo---dijo Salvador con súbita
ira,---es a los serviles que impulsaron a Vinuesa y compañeros mártires
para abandonarles en el momento del peligro. Quizás celebran hoy que la
muerte de esos infelices borre la huella de trabajos más formales;
quizás se mezclan hipócritamente a la canalla soez que pide horca y
hogueras\ldots{} para distraer de sí la atención del pueblo honrado y
del Gobierno.

---Quizás\ldots---repitió serenamente Regato.

---Si sigues por esa senda de sentimentalismo---dijo Campos, dando a
Monsalud familiar espaldarazo,---es muy posible, ¡oh joven!, que te
pongan entre los sospechosos o poco adictos al sistema.

---Pónganme donde quieran---manifestó Salvador.---Yo sé dónde estoy y
conozco bien los sitios y las personas. Desprecio los juicios malignos
que aquí o fuera de aquí puedan hacerse de mi conducta.

---Enérgico estás---dijo Cicerón con jovialidad.---Verdad es que quien
se ha extralimitado en el templo, bien puede salir de sus casillas en la
sacristía.

---¿Qué es eso de sacristía?---indicó Canencia, desperezándose, después
de contado el dinero, como hombre que ha terminado un gran trabajo.---No
se pongan motes de clerigalla a estos venerables lugares. Esto se llama
la \emph{Cámara de Meditaciones}\ldots{} Cuente usted otra vez lo suyo,
señor Regato. Son 836 reales y tres maravedises.

---No vuelvo a ensuciar mis manos en esta inmundicia---dijo
Regato.---¡Válgame Santa Mónica, cuánta calderilla! Parece mentira que
una hermandad tan ilustre y a la cual pertenece tanta gente adinerada no
ponga más que estos miserables huevecillos.

---Los gordos son para el hermano Sócrates---dijo Monsalud.---Mire
usted, Sr. Regato, cómo va echando carrillos y rejuveneciéndose el buen
masón de Salamanca.

---Cállate, picarillo---repuso Canencia.---Ya sabes que puedo sacarte
los colores a la cara siempre que quiera.

---Señal de que tengo vergüenza.

---O de que la tuviste\ldots{} Pero basta de boberías. Cobre usted,
señor Regato, y venga recibo.

---Las cuentas de estos señores---dijo Salvador,---son tan embrolladas
como las leyes masónicas.

---Es sencillísimo---contestó Regato.---Se me deben 1.233 reales. Aquí
está mi cuenta\ldots{} «Por dos calaveras que mandé traer de la bóveda
de San Ginés en 6 de Noviembre, 42 reales\ldots{} Por el bordado de
cuatro mandiles, 268\ldots{} Por echar una pieza al sol, 12\ldots{} Por
pintar las llamas, 30\ldots{} Por una escuadra nueva y siete malletes,
58\ldots{} Por aguardiente que se dio a los de policía el 5 de Enero,
14\ldots{} Por lo que se repartió cuando tiraron la pedrada al coche de
Narices, 4 10\ldots{} Por papel de circulares, 60\ldots{} Por saldo del
piquillo que se le debía a Grippini el cafetero de La Fontana,
140\ldots{} y así sucesivamente, señores. Total, 1.233 reales». Ahora
papá Sócrates ajusta las cuentas de otro modo, y no quiere darme más que
836 reales. Estas mermas son las recompensas de un hombre de bien que
consagró su tiempo a ser secretario de la masonería durante cinco
meses\ldots{} ¡Vean ustedes qué pago! Adelanta uno su dinero para que el
Orden no carezca de nada, y al pagar\ldots{} ¡Luego se espantan de que
me haya hecho comunero!\ldots{}

---Bendito D. José---dijo vivamente Cicerón,---poco a poco. No nos
espantamos de que usted se haya hecho comunero; nos espantamos y nos
enojamos de que usted, tan favorecido por este Gran Oriente,
prescindiendo de piquillos, alcances y descuentos, fomentara la escisión
funesta que acaba de realizarse en la sociedad; que arrastrara fuera del
Orden a esos desgraciados fundadores de la gárrula comunería, y que
ahora, después que forman iglesia aparte, les incite contra nosotros,
les predique la anarquía y el desorden, convirtiéndoles en desalmados
jacobinos.

---Yo me marché de la masonería---dijo Regato con firmeza,---yo fomenté
el cisma, yo contribuí a fundar la Sociedad de los Hijos de Padilla,
porque la masonería vino a ser rápidamente una sociedad ñoña y que no
sirve para nada, como dijo Voltaire. Yo no oí las verdades amargas que
dijo el Sr.~Monsalud esta noche, porque como hermano \emph{durmiente} a
perpetuidad, no puedo pasar de la sacristía ni aun entrar aquí, sino
recatadamente y a ciertas horas; pero por lo que me contó el
Sr.~Canencia, sé que este joven puso el dedo en la llaga. Señores, esto
es una farsa; esto no conduce más que a un servilismo no menos infame
que el servilismo del año 14. Aquí se hacen los decretos a gusto de dos
o tres maestros del grabado sublime; aquí se eligen los diputados; aquí
no hay otra cosa que los manejos de cuatro fatuos que mandan y a su
gusto disponen de todo. No les quiero citar, porque no hay para qué.
Pero ellos quieren establecer el gobierno perpetuo de los tibios y
adjudicarse todos los destinos. Esto no puede ser, y no será. Hemos
fundado la comunería para establecer la verdadera libertad, sin boberías
de orden y servilismo encubierto, para darle al pueblo su total
soberanía, y que se hagan todas las cosas como al santo pueblo le dé la
gana; para desenmascarar a tanto pillo farsante y hacer que obtengan
destinos los verdaderos hombres de bien, adictos al sistema. Basta de
papeles y comedias bufonas. Nosotros vamos a la verdad, a la realidad.
Odio eterno, señores, entre unos y otros; queremos separación eterna,
irreconciliable, de los que desterraron a nuestro querido héroe, de los
que contemporizan con la Corte y la Santa Alianza, de los que disuelven
el ejército libertador, de los que persiguen a las sociedades
patrióticas de \emph{La Fontana} y \emph{La Cruz de Malta}, de los que
hacen la mamola a los obispos y al Papa, de los que ponen dificultades a
la organización de la Milicia Nacional; separación eterna de los que en
una mano tienen el libro de la Constitución y en otra el cetro de hierro
del \emph{Rey neto}. Éste es el Orden de Padilla; ésta es la
Confederación de Padilla, que hará en España la revolución verdadera,
que establecerá el sistema constitucional en toda su pureza y pondrá fin
al reinado de los pillos e hipócritas. El Orden de Padilla derribará el
infame Ministerio de las \emph{páginas} y de los \emph{hilos} antes de
ocho días, señores; óiganlo bien, antes de ocho días.

Nadie contestó en los primeros momentos. Cicerón meditaba apoyando su
sien en el dedo índice. Canencia sonreía. Monsalud, indiferente a la
perorata, se levantó para retirarse.

---¡Gran suerte será para nosotros---dijo al fin Campos,---que el señor
Regato nos perdone la vida!

---Yo no amenazo. Al contrario, invito a todos los buenos amigos a que
se vengan conmigo.

---Es muy cómodo eso---indicó Cicerón.---Vivir con la Masonería, cobrar
800 reales por calaveras, remiendos echados al sol y aguardiente dado a
la policía, y marcharse después con los comuneros para hacernos la
guerra.

---No pueden ustedes acusarme de interesado---dijo Regato, levantándose
también para marcharse.---La Comunería es pobre; no da destinos.

---Pero los dará tal vez dentro de ocho días. Ya se puede esperar.

---Antes que se me olvide, Sr.~D. José Manuel---dijo el filósofo
Canencia, que no se apartaba de lo positivo.---Me han dicho que allá
tienen falta de espadas y broqueles. Aquí tenemos algunas piezas de
sobra.

---Veo que esto acabará en Rastro---repuso el comunero, guardando sus
cuartos .---Nosotros usamos espadas de acero, no de latón.

---Pues buen provecho, hombre, buen provecho.

---Para mis amigos soy el mismo de siempre---dijo Regato echándose la
capa sobre los hombros.---¿Quién sabe si\ldots?

---El hermano Sócrates y yo tenemos que ajustar ahora otra especie de
cuentas. Buenas noches, señor Regato.

---Yo me retiro también---dijo Monsalud.---Repito lo del destino, señor
Marco Tulio. Muchas gracias, muchas gracias por la secretaría; pero que
sea para otro.

---Adiós, puerco espín\ldots{} Señor Regato, mucho cuidado con ese
granuja que sale con usted. Es capaz de hacerse comunero si usted se lo
dice tres veces.

Cuando ambos salieron a la calle, el más joven dijo:

---Sr.~D. José Manuel Regato, yo quiero ser comunero.

Uno y otro hablaron breve rato, separándose después.

\hypertarget{xi}{%
\chapter*{XI}\label{xi}}
\addcontentsline{toc}{chapter}{XI}

Seguía viviendo Solita en casa de Doña Fermina Monsalud, adonde trasladó
el pequeño mueblaje matrimonial; y su bondad y sencillez nativas, así
como la gran desgracia que padecía, abriéronle pronto el corazón de la
madre y el hijo. Otras personas necesitan largo tiempo y trato para
ganarse una amistad profunda; pero Solita, a los ocho días ya era de la
familia. Durante las largas ausencias de Salvador, que estaba fuera casi
todo el día y parte de la noche, la señora mayor y la señorita, sin
dejar de la mano una y otra labor de utilidad y entretenimiento, no
cesaban de discurrir sobre las probabilidades de que el Sr.~Gil de la
Cuadra fuese puesto en libertad; y como el tema llevaba al áspero
terreno de la política, concluían siempre diciendo mil desatinos, que en
su buena fe y candor les parecían discretas observaciones o grandiosos
descubrimientos.

---Dicen que va a caer el Gobierno---indicaba Doña Fermina.---Si entran
después los que quieren que todo sea libertad y más libertad, no habrá
presos.

---Lo que yo creo más probable---respondía Soledad,---es que el Rey se
levante de mal humor cualquier mañanita, y mande a su caballerizo mayor
a las Cortes. Desengáñese usted: de ahí viene todo el mal.

Algunos días veían los sucesos con alegres ojos; otros, sombríamente y
con tristeza.

---Tengo el corazón traspasado---decía Solita, dejando caer sus lágrimas
sobre la costura.---He cerrado un momento los ojos para rezar, y he
visto a mi padre expirando en el calabozo.

---No pienses tonterías---contestaba la Monsalud.---Yo he cerrado
también los ojos para rezar, y he visto al señor Gil poniéndose la capa
para salir de la cárcel. El mejor día le ves entrar por esa
puerta\ldots{} Mi buen hijo ha tomado con empeño este negocio.

Entraba entonces Salvador, fatigado y sombrío, y al punto las dos
mujeres clavaban en él la vista para adivinarle los pensamientos antes
que los manifestase. Solita se lo comía con los ojos, y había adquirido
tal arte para leer en la expresiva fisonomía del joven, que al verle
entrar decía para si: «Hoy tenemos malas noticias», o: «Hay esperanzas».

Soledad creía deber suyo pagar con pequeños trabajos y servicios los
favores sin cuento que en aquella casa recibía. En un par de días
enterose minuciosamente de los hábitos de la familia y procuraba que su
presencia en la humilde vivienda fuera de lo más útil posible. Aguzaba
su ingenio para introducir en el cuarto de Salvador refinadas
comodidades, previendo cuanto el buen muchacho necesitar pudiera; se le
conocía en la cara y en el modo de mirar que no abandonaba un punto la
observación cariñosa y vigilante de todo cuanto a su hermano postizo se
refiriese.

Separada de su padre y de los parientes maternos, la persona a quien
tenía mayor respeto era aquel protector advenedizo en cuyos brazos había
caído. Con la madre tenía confianza; con el hijo, no. Además de que no
osaba entablar conversación con él, fuera de las preguntas propias de
las circunstancias, manteníase siempre distante y respetuosa. Salvador,
a los pocos días de vida común, la tuteaba. Como pasasen muchos sin que
ella correspondiese a esta familiaridad, él le dijo:

---Cuando el pobre Gil se separó de nosotros, quiso que fuéramos
hermanos. Trátame como se tratan los hermanos, y llámame \emph{Salvador}
a secas y \emph{tú}.

---Me parece que no podré acostumbrarme a eso---respondió la niña,
ruborizándose.

Contradiciendo su propia opinión, se acostumbró muy pronto.

Cuando el joven dormía, avanzada la mañana, una como divinidad del
silencio cuidaba de evitar los más ligeros ruidos de la casa. Cuando
volvía muy tarde, las más veces en el último confín de la noche, Solita
velaba sin fatiga ni sueño para que no esperase ni un minuto en la
puerta ni le faltara nada al entrar. Nunca se había permitido la más
ligera broma con él, ni dejó de emplear, para decirle algo, el tono más
comedido y serio. Una noche, sin embargo, le salieron las palabras a la
boca con tal ímpetu, que se extralimitó a hablarle así:

---¡Qué tarde has venido esta noche, hermano! Se conoce que tú y tu
novia habéis tenido muchas cosas que deciros.

Soledad no comprendía que un hombre trasnochase por otra razón que por
estar hablando con su novia.

Salvador acogió la observación con amable sonrisa. Arrojándose en una
silla con muestras de gran cansancio, contempló a su improvisada
hermana, que estaba ante él sosteniendo una luz, y se fijó más que nunca
en las graves imperfecciones de su rostro, no tantas, sin embargo, que
disminuyese el fuerte atractivo simpático que existía en ella, a manera
de reflejo o anuncio del alma.

---Solita---le dijo Monsalud riendo,---con esa luz en la mano te pareces
a la Fe iluminando el mundo. Yo he visto en alguna parte una estatua,
cuadro o estampita igual a ti en este momento\ldots{} Dime, hermana, y
perdona mi curiosidad: y tú, ¿no tienes novio?

Solita volvió rápidamente la espalda para retirarse; pero arrepentida
sin duda, tornó a mirar a su hermano.

---Bien sabes que lo tengo. Mi primo Anatolio\ldots{}

---¡Ah, ya recuerdo! Tu papá me habló de un primo tuyo, que también será
ahora primo mío\ldots{} Ya recuerdo, sí, el primo Anatolio, que va a ser
mi cuñado.

---Justamente. ¿Quieres algo?

---Aguárdate y respóndeme. ¿Quieres mucho a nuestro primo?

---Ya sabes que mi padre ha dispuesto que sea mi marido.

---¿Le has visto alguna vez?

---Cuando éramos niños. Yo no me acuerdo bien cómo es. Mi padre hace
poco me solía decir: «Tu primo Anatolio ha de ser a esta fecha un
arrogante hombrazo, como Salvador, el de Doña Fermina».

---Pero no me has dicho si quieres mucho a tu Anatolio.

---Eso no se pregunta. ¿No he de quererle si mi padre me ha mandado que
le quiera y me case con él?

---A eso no hay nada que decir, hermana. Cuando te cases y vayas a
Asturias, te prometo hacerte una visita. ¿Qué te parece?

---Me parece muy bien.

---Y seré padrino de tu boda\ldots{} y seré padrino de tus niños, de mis
sobrinillos.

---Buenas noches, compadre.

Pero esta clase de diálogos eran una excepción. Generalmente, cuando
Salvador entraba, Soledad le hacía preguntas referentes a la deseada
libertad de su padre.

---Hermano---le dijo una noche,---tu cara me anuncia malas noticias.
¿Qué hay?

---¿Malas noticias?---repuso el joven dando un suspiro y meditando breve
rato.---La verdad, este asunto es difícil. Se sacan piedras del fondo
del mar; pero ¿quién saca la pobre víctima que cae en el inmenso fondo
de barbarie del populacho?

Solita dio un suspiro y elevó sus expresivos ojos al cielo.

---Pero no hay que desesperar, hermanita---añadió Salvador
consolándola.---Cuando yo llegue al último extremo en mis fatigas y
empeños por salvar la vida al pobre reo; cuando yo no pueda más, vendrá
lo imprevisto, vendrá Dios y lo salvará.

---Según eso, traes malas noticias---dijo Soledad con abatimiento.

---Malas no, regulares. He adelantado algo. Mañana veremos. Con que
buenas noches, comadre.

Solita dio otro suspiro y se alejó; pero retrocediendo al instante, hizo
esta pregunta:

---¿Y le has visto?

---Todavía no he podido verle. Ponen mil dificultades; pero me voy a
hacer amigo de los comuneros, a ver si por este medio\ldots{}

---Los comuneros\ldots{} es decir, D. Patricio. Dime, hermano, ¿son
todos tan tontos y tan crueles como nuestro vecino?

---Allá se le van\ldots{} Creo que me será fácil ver a tu padre.
Descuida, que si no podemos conseguir su absolución, trataremos de
arreglarle la escapatoria.

---¡Qué bueno eres, pero qué bueno!---exclamó Sola.---Siempre que te
oigo hablar se me llena el corazón de esperanza y veo a mi pobre padre
libre y feliz. Lo que haces por nosotros Salvador, es más que cuanto
pueden hacer los hombres más generosos. Mucho ha de darte Dios en esta
vida o en la otra para poderte premiar.

---Dios no tiene que darme nada, tonta. Esto es una deuda, mejor dicho,
aquí hay varias deudas que pesan sobre mi alma. Si salvo a tu padre de
la muerte primero, de la cárcel después, sentiré un alivio\ldots{}

---Ya sé\ldots{} Cuando mis padres marcharon a Francia hace ocho años,
ocurrieron cosas terribles.

---Sí, muy terribles. Algunas de ellas no las puedes comprender. Por
fortuna tú no estabas allí; te dejaron en La Bañeza.

---Pero todo me lo contó mi madrastra---manifestó Solita con
emoción.---La pobre te estimaba mucho, y constantemente hablaba de ti.
Hasta en el día de su muerte te nombró varias veces\ldots{}

Salvador callaba, fijando la vista en el suelo.

---No digas que soy generoso si saco a tu padre de este mal
paso---manifestó después de una pausa.---Di más bien que soy un malvado
si no le salvo.

---¿Y si es imposible?

---No hay nada imposible---repuso el joven con brío.---Soledad, tendrás
padre, tendrás marido\ldots{} ¿Sabes que conviene escribir a tu primo
Anatolio, refiriéndole la situación en que te hallas?

---Como tú quieras---respondió la joven con indiferencia.

---Le escribiré, vendrá, te casarás. Para entonces, vive Dios, o soy
digno del desprecio de todos, o estará tu padre libre. Viviréis felices
y tranquilos\ldots{} ¡Oh, qué hermosa familia vamos a tener
aquí!\ldots{} Porque supongo que el Sr.~Gil se verá rodeado de nietos
dentro de algunos años\ldots{} ¡Pobre anciano, cómo gozará, jugando con
los pequeñuelos!\ldots{} ¿Y ese Anatolio será un buenazo, un corazón de
oro?\ldots{} Lo dicho: seré padrino de tus muñecos.

---Buenas noches, compadre. Que duermas bien.

---Buenas noches.

Y al acostarse se decía a sí mismo:

---¿La ves tan desgraciada, tan pobre, tan sola? Pues con su sencillez,
su ignorancia y su Anatolio, será más feliz que tú.

\hypertarget{xii}{%
\chapter*{XII}\label{xii}}
\addcontentsline{toc}{chapter}{XII}

El personaje a quien los de \emph{la Acacia} daban el nombre de
\emph{Cicerón}, vivía en una hermosa casa a la extremidad de la calle de
D. Pedro, junto a las Vistillas. La Dirección de Correos, que hoy
constituye una posición decente, era en aquellas calendas una verdadera
mina, y ahondando en ella, el señor Campos, a pesar de su oscuridad
política, había conseguido manejando cartas, y no de baraja, allegar un
capitalejo que en lo sucesivo sirvió de tema de maledicencia al
envidioso vulgo. Entró con pie derecho este insigne personaje en la
burocracia revolucionaria por reunir los tres requisitos indispensables
para medrar durante aquel período, los cuales eran: haber padecido
durante el régimen absoluto, haber intervenido en la mudanza del 20 y
estar afiliado en las sociedades secretas.

Vivía, pues, pacífica y cómodamente con su familia, que no era por
cierto muy numerosa, pues constaba tan sólo de dos personas: su hermana
doña Romualda (señora de muy poco seso en su juventud, al decir de la
gente, pero que en la época de nuestra historia parecía querer apaciguar
su conciencia dándose a la devoción con ardiente celo) y su sobrina
Andrea, hija de Mauricio Campos, que volvió de Indias el año 12 con una
regular fortuna de que no pudo disfrutar porque le sobrevino la muerte.
Huérfana de padre y madre a los once años de edad, la hermosa niña quedó
bajo la tutela de su tío, que no tuvo reparo en empezar su
administración disipando en conspiraciones una parte de la fortuna de la
pobre indianilla; y para mayor perjuicio de ésta, los frecuentes viajes
de Campos la ponían bajo la inmediata protección de doña Romualda, que
por aquellos días no había salido aún de la etapa de las calaveradas
amorosas.

Andrea, cuya crianza en América no había sido ejemplar a causa de la
temprana muerte de su madre, tuvo una escuela lamentable en la peligrosa
edad del cambio de juguetes, es decir, cuando se decreta la jubilación
definitiva de las muñecas y el planteamiento de los novios. Mal atendida
por su tío y peor tratada por doña Romualda, a quien aborrecía
cordialmente, la joven vivía ensimismada, cultivando con ardor su propia
imaginación. Contrajo amistades que una madre prudente hubiera
prohibido; intimó excesivamente con las criadas; paseaba en compañía de
éstas más de lo conveniente, y en cambio del cariño y el agasajo que le
negaran dentro de casa, disfrutaba de una libertad que no conocían las
señoritas de aquella época y rara vez las de ésta. Por esto Andrea se
parecía tan poco a las niñas españolas de su tiempo. Era una criolla
voluntariosa, una extranjera intrusa que habrían repudiado Moratín y
Cruz. Su familia favorecía más cada vez aquella libertad. Doña Romualda,
que empezaba a sufrir la transformación de la edad paleolítica de los
amores a la edad neolítica de las devociones, tenía mucho que hacer:
estaba en la iglesia. El buen Campos también era hombre ocupadísimo por
aquellos días: estaba conspirando.

Era la indiana buena y sensible. Fácilmente comprendía la verdad por
poco que se la mostraran. Fácilmente acertaba con lo justo y honrado,
por simple iniciativa de su conciencia. Pero tenía ansia de afectos
ardientes, y miraba sin cesar a todos lados buscándolos. Su desgracia
consistía en que le era forzoso abrirse sola y sin ayuda de nadie el
áspero camino de la juventud. Habría necesitado para esto tener un
caudal de energía y de entereza moral que rara vez da Dios a las
criaturas, pero que suplen, según admirable orden de la sociedad, las
personas allegadas y mayores de la familia. Careciendo de fuerza propia
y de sostén extraño, hubiera sido un prodigio que la gallarda flor se
mantuviera derecha. Los prodigios son muy raros en el mundo. Bueno es
hacer constar que la pobre Andrea, avisada del peligro por una intuición
potente, hizo esfuerzos instintivos para sostenerse erguida y pomposa,
vuelta hacia el sol la virginal corola; pero el viento soplaba con
demasiada fuerza y se dobló.

Era tan guapa, que su vanidad (otra desgracia no pequeña) resultaba cada
vez más lógica. Habría sido conveniente que ignorara algún tiempo la
riqueza de seducciones que atesoraba en sus ojos, en su boca, en todas
las partes de su cara morena y alegre, llena de inexplicables gracejos y
atractivos; en su cuerpo delgado y flexible, de esos que no tienen
clasificación fácil en el cuadro ginecológico, y son tales, que para
buscarles semejante necesita el observador descender en busca de un ser
antipático y que se arrastra: la culebra.

Pero Andrea no tuvo a nadie que le hiciera el sumo bien de engañarla
durante algún tiempo respecto a su belleza, y entregose desde muy niña
al fascinador deleite de los espejos. Las criadas cantaban a su oído un
coro de lisonjas. En la sala de su casa había una hermosa estampa que
representaba la famosa escena de Phrine entre los jueces de Atenas, y
Andrea, de tanto leerla, se sabía de memoria la leyenda grabada al pie
con resplandecientes letras de oro. Aunque parezca extraño, conocidos
los tiempos y el lugar, no puede menos de suponerse que en aquella
cabeza hervían ideas gentílicas; pero el paganismo es de todas las
edades, y buscando sin cesar dónde establecerse, se mete y se acomoda
allí donde no hay otra religión que haya echado raíces.

Andrea fomentó su vanidad y la adoración de sí misma, consagrando al
adorno de la persona mucho tiempo, mucha atención y todo el dinero de
que podía disponer. Si éste no abundó durante los ominosos tiempos en
que Campos conspiraba, luego que vino la era feliz y fue restablecido en
parte el patrimonio de la huérfana, el buen tío, que no era tacaño y
gustaba de que su pupila se presentase bien, abrió bastante la mano en
lo relativo al lujo. Ésta era la fórmula de su cariño, porque sin duda
hay distintas maneras de amar a las sobrinas. Además, Campos, por
razones de egoísmo, tenía empeño en no contrariarla, deseando alcanzar
de ella consentimiento para un proyecto nupcial que entre manos traía
después de la revolución.

No se crea que el \emph{Venerable} se parecía a los grotescos tutores
que son el elemento bufón de las comedias italianas del siglo
{\textsc{xviii}} y que también abundan en el repertorio de las óperas.
Campos no quería que su sobrina se casase con él. Era viejo, habíase
entregado al volterianismo, que en aquellos tiempos empezaba a propagar
tanto las cómodas prácticas del celibato; era además un epicúreo
refinado de esos que nos legó el siglo {\textsc{xviii}}, y que ya
comenzaban a desbancar a los rancios egoístas de chocolate y bollos de
monjas. Otrosí: tenía Campos sus entretenimientos fuera de casa, con los
cuales le iba muy bien al parecer. Su claro talento, además, no le decía
nada favorable a su enlace con muchacha primaveral. Su amigo D. Leandro
no escribió para él \emph{El viejo y la niña} ni \emph{El sí}.

El proyecto consistía en casarla con un señor de edad algo avanzada,
pero entero, arrogante, fino, discreto, y que sabía ocultar sus años y
aun hacerse amable, pues a tanto llega en privilegiados individuos el
arte social. El marqués de Falfán de los Godos era un medio siglo bien
conservado, gracias a reparaciones hábiles y a un cuidado continuo.
Había sido exento de Guardias, compañero de Palafox y de Godoy, y en
aquellos tiempos en que los mozos guapos desempeñaban grandes papeles en
la Corte y en que se hablaba, como lo prueba el desvergonzado libro de
un fraile, de serrallos a la turca, de envenenamientos proyectados, de
matrimonios dobles y otras barbaridades ante las cuales la discreta
historia se complace en cerrar los ojos. Así como el duque de Zaragoza
fue célebre y simpático por sus hurañas resistencias, Falfán de los
Godos tuvo fama por lo contrario. En 1821 era general; tenía fama no
sólo de honrado y decente, sino también de gastrónomo y mujeriego, cosa
natural en un solterón riquísimo y bien parecido, de ancha conciencia
formada en la escuela enciclopedista del siglo pasado.

Hacia 1820 comenzó a pesarle el celibato; echó de menos algo amante,
tierno y cariñoso; es decir, los hijos que debía tener y no tenía, la
esposa que siempre había rechazado como una fastidiosa carga de la vida.
Falfán de los Godos pensó en casarse, y supuso que sus cincuenta años, a
pesar de la madurez consiguiente, podían dar aún mucho de sí. Acontece a
menudo que estos hombres listos y conocedores del mundo, pierden la
chaveta cuando tratan de poner algún orden en su vida, y bastardean
completamente la meritoria idea de ser padres, que tan a deshora les
ocurre. Falfán de los Godos, maestro en el arte de vivir, perdió el
tino, como todos los de su clase, y en vez de buscar para esposa un tipo
de bondad reposada, una madura belleza asegurada de peligros y que se
acomodase fácilmente a los gustos e ideas del trasnochado esposo, fue a
incurrir en el maldito antojo de la niña fresca y tiernecita que apenas
ha empezado a vivir y tiene un porvenir ignoto delante de sus ojos
chispeantes. Él no dejaba de comprender en ratos lúcidos su error; pero
se engañó a sí mismo vanidosamente trayendo a la memoria su buena
presencia, su gran fortuna, su fama, sus gustos artísticos, su finura,
rica herencia del antiguo régimen que contrastaba con la grosería de los
revolucionarios.

Si todo hubiera de resolverse entre el acartonado Marqués y Campos, la
cuestión habría estado concluida en un par de semanas; pero Andrea no
quería casarse con Falfán de los Godos porque amaba a otro. Esto sí que
se parece a todas las comedias italianas del siglo {\textsc{xviii}}, a
las óperas del primer repertorio y a muchas novelas de aquel tiempo,
principalmente a las de D'Arlincourt, Mad. Cottin, Florian y Mistress
Bennet; pero no es culpa nuestra que esta vieja historia se nos venga a
las manos. Acontece alguna vez que las cosas vulgares son las más dignas
de ser contadas.

En los días que van corriendo para nuestra relación hacía tres años que
Andrea había entablado amistades íntimas con un hombre que cierto día se
metió en su casa buscando refugio contra los corchetes que le
perseguían. Cómo nacieron y rápidamente tomaron vuelo a manera de
incendio estos amores, es cosa que ahora no nos importa; pero la
libertad de que disfrutaba Andrea explicaría muchas cosas. Pasaron días,
muchos días, y con ellos sucesos buenos y malos que no merecen ser
referidos. En 1821, la casualidad, o mejor dicho, la política, juntó en
un círculo al amante de Andrea y a Campos: hiciéronse amigos, y cuando
éste le llevó a su casa no tenía ni vagas sospechas del interés que
aquella amistad inspiraba a su sobrina. De este modo, Píramo y Tisbe no
tuvieron que horadar paredes para hablarse, y aunque la presencia casi
constante del tío les estorbaba, viéndose a menudo aun delante de
testigos, tenían medios para preparar sus conferencias reservadas, las
cuales no eran ya frecuentes porque la libertad de Andrea empezaba a
disminuir.

El favorecido conocía perfectamente las horas que doña Romualda
consagraba a la grave faena diaria de sus devociones, las de oficina y
la logia para Campos. Aplicando bien la sentencia profundísima de uno de
los siete sabios de Grecia, que dijo \emph{aprovecha la ocasión}, aquel
hombre enamorado hasta la ceguera y el aturdimiento entraba en la casa.
Estas atrevidas invasiones del templo de un exaltado amor no eran ni
podían ser frecuentes, y exigían gran cautela con criados y gente
menuda; pero los amantes habían discurrido mil triquiñuelas y contaban
con la fiel complicidad de una criada antigua. Su ceguera, con todo, no
era tanta que se ocultase a entrambos la necesidad de poner término a
tal género de vida.

\hypertarget{xiii}{%
\chapter*{XIII}\label{xiii}}
\addcontentsline{toc}{chapter}{XIII}

Una mañana, Salvador entró. Como no había temor de sorpresas, Andrea,
después de poner en escucha a su criada, según costumbre, abrió al
amante las puertas de su habitación.

---Ven aquí---le dijo asomando la linda cara y la mano tras la cortina
de la sala donde él esperaba.---Estaremos solos hasta que venga mi tía.

El amante se sentó sin decir nada en un canapé, y Andrea volvió al
espejo de donde poco antes se había apartado. Con su preciosa mano se
tocaba aquí y allí el recién peinado cabello, dándole la última forma,
como artista que remata su obra. Después se puso una flor. Sin retirarse
del espejo, porque en él veía la figura del hombre, le habló así:

¿Qué tienes hoy, que estás tan callado?

---Hace pocas noches vi a tu tío, ¿te lo ha dicho?---contestó Salvador.

---Sí, me contó que te había ofrecido un destino y no lo quisiste.
¡Bonito modo de ser agradecido!---dijo Andrea, moviendo su cabeza ante
el espejo.---¡Qué orgullo!\ldots{} porque no es más que orgullo.

Gracias por tu protección.

¿Qué protección?

---¿No fuiste tú quien dijo a Campos que me proporcionara una posición
decente?

---¡Yo! ¿Estás loco?---exclamó Andrea con sorpresa, volviéndose, porque
para manifestar cosas importantes no satisface ver la figura del
interlocutor reflejada en un espejo.

---No te esfuerces en convencerme de que no fuiste tú---dijo
Salvador.---Desde luego, comprendí que tu tío me engañaba.

---Seguramente te engañaba. Bien sabes que nunca me atrevo a hablarle de
ti; y cuando lo hago es de la manera más indiferente.

---Extraño que Campos, hombre muy listo, urdiera tan mal su farsa---dijo
Salvador.---¿En qué se funda ese oficioso empeño de favorecerme? No
creas, quiere mandarme a América nada menos. Seguramente le estorbo.

---No lo comprendo así. Si quiere favorecerte es porque te
estima---repuso Andrea, volviéndose hacia el espejo.

---¿Tú también?---dijo Monsalud con impaciencia y desasosiego.

---¿Qué es eso de yo también?---indicó la indiana jovialmente.

---Quizás tú puedas explicarme lo que la astucia de Campos no ha dejado
entrever.

---Querido, yo no puedo explicarte nada, ¿estamos?\ldots{} Hoy has
pisado mala yerba. Ya veo que no me libraré hoy de un poquillo de mareo.
¿Y por qué? por la cosa más natural del mundo: porque mi tío ha querido
darte una prueba de lo mucho que te aprecia.

---Sería, no muy natural, sino algo natural esa prueba de estimación si
tu tío después de ofrecerme el destino, no me hubiera dicho una cosa
grave.

---¿Qué cosa?

Salvador la miró con fijeza.

---Me dijo que pensaba casarte.

Como el lector recordará, Campos no había dicho tal cosa; pero el
inquieto joven practicaba el aforismo vulgar que ordena decir mentira
para sacar verdad.

---¡Ah!---exclamó Andrea riendo.---Eso es lo que traes hoy. Te conozco,
tunante. Vienes mascullando esa idea.

Diciendo esto tomó un abanico, y con expresión de graciosísima burla,
sonriente la boca, húmedos los ojos, acercose al joven y empezó a darle
aire rápidamente.

---¿Estás sofocado?\ldots{} Aire, aire, no sea que te dé un síncope.
Refréscate, hombre\ldots{} Que se te quite eso de la cabeza.

Monsalud le arrebató violentamente el abanico, lanzándolo al aire. El
abanico atravesó el recinto de un extremo a otro, abriéndose como un
pájaro que extiende las alas.

---¡Qué modo de tratar mis joyas!\ldots{} Pues me gusta---dijo Andrea,
corriendo tras el abanico.

Arrodillose para cogerlo del suelo, cerrolo, y empuñándolo a manera de
puñal, amenazó a su amante diciéndole:

---Te voy a matar.

Monsalud contemplaba, primero sin enojo, después con gozo, la hermosa
figura juguetona y ligera que tenía delante. De súbito Andrea corrió
hacia él con los brazos abiertos, y abrazándole el cuello, le apretó
fuertemente diciendo:

---Ya me casé, ya me casé, ya me casé.

Repitió esto unas cuarenta veces.

Salvador la obligó a sentarse a su lado.

---A mí se me está preparando una desgracia---le dijo
cariñosamente.---Andrea, tengo desde hace muchos días el presentimiento
de que esta preciosa cabeza me hará traición. ¿No recuerdas lo que te he
dicho tantas veces? Desde que tengo uso de razón no he intentado cosa
alguna que haya tenido un desenlace lisonjero para mí. Si alguna vez he
conseguido el objeto por mucho tiempo deseado, mi dicha ha sido corta.
Siempre que cavilo acerca del resultado de un asunto cualquiera que me
intranquiliza, no puedo apartar de mi pensamiento la idea de un éxito
desgraciado, y siempre acierto\ldots{} Tengo la desdicha de no haberme
equivocado una sola vez. Yo no sé qué pensar de mí. Si se castigan en la
tierra las faltas, las que yo he cometido no corresponden a los golpes
que en diversas ocasiones me han venido de arriba. Fui jurado y cayó
José I; tuve amores, y por poco muero en ellos; conspiré, y la
conspiración salió mal; dejé de conspirar, y salió bien\ldots{} En fin,
tú sabes mi vida toda y podrás juzgarlo. Si es verdad que los hombres
nacen con buena o mala estrella, la que andaba por los cielos el día en
que yo vine al mundo era la más mala, la más perra de todas.

---Eso que dices, ¿tiene algo que ver con mi casamiento?---preguntole
Andrea con malicia.

---Tiene que ver, sí. Te quise y te quiero. Si tú me correspondieras con
la fidelidad constante que yo merezco y que me debes\ldots{} esto sería
una suerte, una felicidad, y yo no puedo tener suerte alguna ni
felicidad.

---¡Qué majadero!---dijo la sobrina de Cicerón con desdén humorístico.

---Cuando pienso en esto, Andrea---prosiguió el joven, enlazando con su
brazo el cuerpo de ella,---me asombro de que tal absurdo haya durado dos
años sin desvanecerse, y hace tiempo estoy pensando que concluirá
pronto, y que tú, como todo lo que interesa a mi corazón, te vas a
desvanecer, a alejarte de mí, dejándome solo con mi desgracia.

---¡Caviloso!\ldots{}

---¡Veo que no te defiendes con ardor; veo que no protestas como yo
protestaría en tu caso!---exclamó Monsalud con la impertinente comezón
de los celosos.---Andrea, tú meditas algo, tú me ocultas algo.

---Medito que te quiero más que a mi vida---repuso ella, cerrando los
ojos y apoyando la cabeza en el hombro de Salvador, mientras le deshacía
el nudo de la corbata.

---Ya sabes, querida mía---repuso él, moviendo la cabeza
negativamente,---que tengo motivos para no creer en palabras de mujeres.
Déjame que te diga una cosa. Yo creo que tu tío tiene razón al querer
casarte; pero el pobre señor ignora que no puedes casarte sino conmigo.
Eres tal para mí, que sin poseerte no comprendo la vida. Si me amas del
mismo modo, demos fin a estas relaciones peligrosas. Casémonos, Cielo.

---Casémonos, Tierra---repitió maquinalmente Andrea.---Cuando quise no
quisiste\ldots{} Está bien. Es verdad que así no podemos seguir\ldots{}
Pero si le dices a mi tío que seré tu mujer, te arrojará por el balcón.

---Me arrojará por la puerta. Verdaderamente no me importa gran cosa,
llevándote conmigo.

---¡Huir!---exclamó la joven con terror.

---¡Huir!---dijo Monsalud, remedándola.---Siempre eres tímida para todo
lo que me favorece. ¡Huir! No te llevaré a ningún desierto\ldots{} Nos
quedaremos aquí.

---Tú estás loco---dijo Andrea levantándose pensativa.

---Pues entonces, hoy mismo le diré al gran Cicerón que te adoro\ldots{}

---Si haces eso, si haces eso\ldots---dijo vivamente Andrea poniéndose
pálida.---Pero tú estás loco, Salvador. Mi tío te aprecia mucho, te
aprecia muchísimo; pero, ¡ay!, tú no le conoces. Temo cualquier
atrocidad si le dices eso.

---Pues no te comprendo. ¿Creerá tu tío que te morirás de hambre en mi
casa? ¿Creerá que no vas a tener una posición decorosa?

---No\ldots---dijo Andrea con los ojos fijos en el suelo;---pero mi tío
es ambicioso\ldots{} tú no sabes quién es mi tío\ldots{} tiene ahora la
cabeza llena de vanidades, y yo no sé\ldots{} Se le figura que yo valgo
mucho, que merezco la mano de reyes y emperadores\ldots{} tonterías.

---Si tú le ayudas, si tú favoreces en él esas ideas, entonces todo se
acabó\ldots{} Yo me voy---dijo Monsalud con repentina cólera.

---Te enfadas contigo mismo---dijo Andrea mirándole con dulces
ojos.---Hazme el favor de no ser terrible. Por ahora no le digas nada a
mi tío. Ya veremos.

---Tu tío quiere casarte; tu tío piensa en ello, y sin duda ha formado
ya su plan. Andrea, tú no quieres decirme la verdad.

---La verdad es que te quiero con toda mi vida---repitió amorosamente la
indiana, repitiendo también el abrazo.---Cállate. Haz lo que te mando, y
espera.

---¿Crees tú que se puede vivir mucho tiempo de esta manera, a
escondidas, ideando mentiras y con absoluta ignorancia del porvenir?

---Es verdad, no se puede vivir así---repuso Andrea con tristeza.

---No puedes ocultar que te agrada este sistema de vida; que no deseas
como yo una paz dichosa al lado de la persona amada. Andrea, en ti
ocurre algo. Tú no eres la que eras; tú has variado mucho; en tu cabeza
hay una idea nueva. Recuerdo que hace tiempo deseabas lo que yo te
propongo ahora. ¿Crees que podrás engañarme muchos días? O te sacaré la
verdad, o te venderás tú misma.

---¿Qué sospechas de mí?

---No lo sé---dijo Monsalud lleno de confusión.---Los que aman no
sospechan poco ni mucho: lo sospechan todo de una vez. Cualquier indicio
o traición. Andrea, tú no eres la misma; repito que no eres la misma.

La estrechó entre sus brazos, apretándola con una fuerza que más que
frenesí de amante parecía el fatal abrazo de Otelo.

---Que me ahogas, tigre---gritó Andrea.

Y entre festivas risas le mordió el brazo. En el mismo instante, de las
ropas de la joven cayó una llave, que, escurriéndose por la alfombra,
brilló, al detenerse, sobre el pétalo de una flor pintada.

---¿Qué llave es ésta?---preguntó Monsalud, cuya excitación suspicaz le
obligaba a fijarse en el más ligero incidente.

---Es la llave de mis secretos.

Salvador con su perspicacia sutil creyó ver en el semblante de Andrea
ligerísimo indicio de contrariedad.

---¿La llave de tus secretos?

---Sí: dámela---dijo ella apresurándose a recogerla.

---Es la llave de la cajita negra. Se me ha antojado abrirla; ¿dónde
está?

Andrea vaciló un instante. Pareció que meditaba y que con el pensamiento
exploraba todo el interior de la cajita negra antes de entregarla a las
pesquisas del receloso amante.

---Ábrela---dijo al fin.---Allí están tus cartas y tu retrato.

---¿Dónde está?

Andrea vaciló otra vez. Al fin, sacando de la cómoda una caja de
finísima madera negra, la puso en manos de su cortejo.

---Si encuentras en ella cartas que no sean las tuyas, y un retrato que
no sea el tuyo---dijo con gravedad,---puedes matarme. ¿Crees que no hay
armas aquí? Mira esto.

Conservando la caja en la mano izquierda, metió la derecha en otro cajón
de la cómoda y sacó un puñal. Era un arma preciosa, damasquinada y
nielada, con puño berberisco adornado de turquesas.

---Éste era de mi padre\ldots{} ya lo has visto---dijo la indiana,
riendo.---Está destinado a mi esposo, para que me mate el día que le sea
infiel.

Monsalud, poniendo a su lado el arma, tomó la caja y la abrió.

---Mi retrato---dijo, sacándolo.

Andrea se apoderó del medallón y lo cubrió de besos.

---Tú sí que no me riñes, tú sí que no dudas de mí---le dijo a la
pintura.---Tú sí que eres bueno, y cariñoso y pacífico.

---Un paquete de cartas---dijo Salvador Monsalud.---Son las mías.

---Dámelas. Valen más que tú.

Andrea desató el paquete. Varias cartas cayeron al suelo. Al inclinarse
para recogerlas se sentó en una preciosa piel de tigre que cubría en
parte la alfombra. Un rayo de sol que por la ventana entraba inundó de
luz el pellejo muerto del animal y el cuerpo extraordinariamente vivo de
la hermosa americana.

---Venid acá, prendas de mi corazón---exclamó, recogiendo los papeles
diseminados a su lado y poniéndolos sobre su lindo pecho.---Vosotras sí
que sois amables y cariñosas; vosotras no reñís ni amenazáis.

Monsalud, que en el canapé inmediato registraba la cajita, alargó la
mano, mostrando a Andrea un pequeño estuche abierto.

---¿Quién te ha dado esta joya?---preguntó con calma.

En el estuche brillaba un diamante de gran tamaño. Como al extender la
mano entrase en la esfera del rayo de sol, Monsalud parecía estar
enseñando una estrella.

---La he comprado yo---repuso Andrea.

---¿Tú?---manifestó Salvador en tono de amarga duda.---Ya sé que tu tío
te da de un tiempo a esta parte bastante dinero para tus vanidades; pero
esto es joya cara. ¿Cómo es que siendo tu costumbre consultarme hasta
cuando compras una vara de cinta, no me has dicho nada de este
despilfarro?

---Pensaba decírtelo hoy---repuso Andrea, soportando con heroísmo la
mirada penetrante del hombre.

---Entonces lo has comprado ayer.

---Ayer, sí. ¿Eso te sorprende? Ya sabes que me gustan las joyas
bonitas\ldots{} Pero ¿por qué pones esa cara? ¿Qué piensas?

---Pienso que lo que me dices no será tal vez la verdad---afirmó
Monsalud severamente.

---¿De modo que yo no puedo comprar un diamante?

---Pero este diamante es muy caro.

---No tanto como crees, niñito---dijo Andrea tomando la sortija y
poniéndosela en el dedo.---No es muy fino. ¡Pero qué bonito!

Movía su mano al sol, y los reflejos que partían de ella semejaban hilos
de luz enredándosele en los dedos.

---¿Y este collar de perlas?---preguntó el amante, sacando de la caja
una magnífica madeja de diez hilos con perlas pequeñas, pero muy
iguales.---No dirás que no es fino. Entiendo algo de perlas, y éstas son
de las mejores.

---Ya lo creo---dijo Andrea, sin dejar su cómodo asiento sobre la piel
de tigre, entre cuyos pelos habían vuelto a desparramarse aquí y allí
las amorosas cartas.---Buen dinero me ha costado.

Salvador la miró de tal modo, que la indiana no pudo permanecer en
silencio. Necesitaba hablar con cháchara festiva para borrar de su
rostro todo rasgo que indicando la presencia de ciertas ideas en su
mente, confirmara las sospechas del hombre.

---Veo que estás muy fastidioso---dijo.---Dame acá.

Tomando vivamente el collar, se lo puso.

---¿No es verdad que es precioso?---añadió, inclinando la cabeza hasta
unir la barba con la garganta y bajando todo lo posible los ojos para
recrearse en la voluptuosa hermosura de su propio seno.---Sostén que no
es bonito.

---¿Lo has comprado tú?

---No, que me cayó del cielo. ¿Pues cómo lo tendría si no lo hubiera
comprado?\ldots{}

Monsalud movió la cabeza con triste expresión.

---Vamos, que no se puede tener nada sin tu permiso\ldots{} Precisamente
hoy pensaba hablarte de esas magníficas compras. Mi tío me dio anteayer
una gran cantidad; no sé cuánto, mucho, muchísimo dinero. Compré estas
joyas a una señora viuda de un intendente\ldots{} ¡Qué ojos pones!
Parece que eres tonto\ldots{} Sí, señor, las compré con mi dinerito. Me
gustan las cosas buenas. También compré en casa del francés de los
portales de Bringas una \emph{citoyenne} preciosísima y un chal muy
rico. ¿Qué tiene usted que decir a eso, Sr.~Majaderito?

Como un pájaro que vuela, corrió a la cómoda y sacó las dos prendas
mencionadas. La \emph{citoyenne}, guarnecida de pieles de armiño, con
forro de seda azul y recamada con cordonadura de oro, presentaba rico y
lujoso aspecto. El chal era de color de rosa con listas blancas que
brillaban como la más deslumbradora plata. Con esa rapidez de manos que
acompaña siempre al instinto del bien parecer, Andrea se puso la
\emph{citoyenne}; después arrojó la \emph{citoyenne} para ponerse el
chal.

---¿Estoy bien?

---Demasiado bien---repuso Monsalud, contemplando con arrobamiento la
hermosísima figura de la indiana, que volvía la cabeza ante el espejo
para verse la espalda.

---Si me lo permite el Sr.~Majaderito---dijo dirigiéndose a él con
ademán ceremonioso,---usaré estas prendas que me han costado mi dinero.

Salvador no contestó. Hallábase en un estado de estupor cercano al
embrutecimiento. Andrea se quitó el chal y lo envolvió rápidamente en el
cuello de su amante, diciendo:

---¡Te ahorcaré!

Había puesto la rodilla en el canapé, y su cuerpo gravitaba con dulce
pesadumbre sobre el pecho y los hombros de Monsalud.

---Andrea---dijo éste, rechazándola suavemente,---si mintieras, si me
engañaras, si estuvieras jugando conmigo, no tendrías perdón de Dios.
Quiero creer que no es así. Casi prefiero una ceguera estúpida a perder
la idea que tengo de ti.

---Pues si te enfadas---declaró ella con vehemencia,---no quiero el
diamante, no quiero el collar, no quiero el chal.

Quitose rápidamente las tres cosas y las arrojó lejos de sí dando al
mismo tiempo con el pie a la \emph{citoyenne} que estaba en el suelo.
Las perlas chocaron contra el cristal de una lámina, y el diamante cayó
detrás de la cortina de uno de los balcones, sin producir ruido alguno.
Monsalud fue allá.

---Ha caído sobre un ramo de flores---dijo con asombro.---Andrea, ¿quién
te ha dado este ramillete?

Señaló el objeto mencionado, que estaba en el suelo junto a los
cristales del balcón, dentro de un hermoso búcaro de la Moncloa.

Andrea permaneció breve rato sin contestar.

---¿No te dije que me lo trajo mi tío esta mañana?

---Nada me has dicho. ¡Hermoso ramo! Violetas, pensamientos y rosas
tempranas. ¡Qué galante es tu tío!

---¡Si creerás que me pretende por esposa!

---¿Por qué no?---dijo Salvador, tomando el ramo y aspirando su delicado
aroma.---El señor Campos está todavía en buena edad.

---Pero no quiere hacer el papel de D. Bartolo. Dame el ramo. Quisiera
que la belleza de tantas flores estuviese en una sola para dártela, y
que el olor de todas también en una sola estuviese para que, guardándola
siempre, te sirviera de memoria mía.

Dicho esto con voz tierna, que sorprendió mucho a su interlocutor, sacó
del ramo una rosa para ofrecerla a Monsalud.

---¿Es la primera vez que tu tío te regala flores?---dijo éste,
meditabundo.

---¿No la quieres? ¿No quieres una flor que te doy? Pues toma, toma,
toma.

Andrea se había sentado otra vez sobre la piel de tigre, y desbaratando
el ramo, cada vez que decía \emph{toma}, arrojaba una flor a su cortejo,
apedreándole de este modo lindamente. Él se las devolvía.

Concluido esto, extendió sus brazos sobre la piel, ocultando el rostro
entre ellos. Yacía dulcemente contorneada en el suelo, y en ella se
enroscaba como una culebra de rosa y plata. El desorden de tal escena
era encantador. Las pieles de armiño de la \emph{citoyenne}, semejantes
a copos de nieve, eran hollados por los pies de la preciosa indiana, y
las ricas telas y la cordonadura de oro se revolvían entre los pliegues
de sus vestidos; las flores aparecían diseminadas en distintos puntos;
algunas cayeron sobre las sillas, otras sobre la misma piel de tigre;
violetas y jacintos veíanse deshojados y rotos, quier sobre las mismas
piernas de Monsalud, quier en los propios rizos del negro pelo de ella.
Las perlas extendían diversos circuitos irregulares sobre la alfombra, y
el diamante fulguraba sobre el velador como una mirada satisfecha,
recreándose en aquel pintoresco y brillante desconcierto.

Uno y otro callaban. Únicamente se oía el ruido que hacía un jilguero en
el balcón, escarbando su alpiste y limpiándose después el pico contra
los alambres de la jaula. Monsalud, con el codo puesto en uno de los
cojines de la cabecera del canapé y la barba en la mano, hallábase en el
estado de atonía y silencio que anuncia miradas interiores u observación
de fenómenos propios que impresionan profundamente. Andrea no chistaba.
Las elegantes ondulaciones de su cuerpo yacente alterábanse un poco con
los movimientos propios de la impaciencia contenida o con los de la
respiración. De pronto movió la cabeza. Monsalud se estremeció todo al
ver aquel movimiento que le mostró la hermosa fisonomía de la indiana y
sus ojos arrasados en llanto.

---¡Andrea!---exclamó movido de sorpresa y pasión.

La indiana saltó como una ondina, y corriendo a abrazarle, secó sus
lágrimas junto a él.

\hypertarget{xiv}{%
\chapter*{XIV}\label{xiv}}
\addcontentsline{toc}{chapter}{XIV}

Cuando la criada les avisó que había peligro, Monsalud pasó a la sala.
No era Doña Romualda quien venía, sino el mismísimo Campos, acompañado
del marqués de Falfán de los Godos.

---¿Has esperado mucho?---preguntole Cicerón.---¿Y Andreílla, no ha
salido a acompañarte?

Salvador, contestando lo que le pareció, estrechaba fríamente la mano
del Sr. Campos y la del Marqués.

---Ya sé a lo que vienes---dijo el \emph{sublime perfecto---}. Siempre
con el tema de ese bribón de Gil de la Cuadra\ldots{} Ahora quizás sea
más fácil. Ya sabes que cae el Ministerio.

---¿Es positivo?

---Figúrate que hoy en la apertura de las Cortes, Su Majestad ha añadido
por cuenta propia un parrafillo al discurso de la Corona, en el cual con
buenas palabras pone cual no digan dueñas a sus ministros.

---Y en cuanto ha llegado a Palacio, le ha faltado tiempo para
exonerarles\ldots{} ---dijo Falfán.---Yo me río de las singulares
prácticas constitucionales de nuestro Soberano.

---Mientras no se sepa quién nos gobernará mañana---añadió Campos,---hay
que dejar a un lado todos los negocios pendientes. ¡Oh!, mi buen
Aristogitón, no pienses que te olvido. Aunque tú pagas con desaires y un
hocico de tres varas los beneficios que se te hacen, ¡qué demonios!, me
he propuesto complacerte y lo conseguiré. Encuentro muy meritorio ese
interés que tomas por un pobre anciano desvalido. Hay que trabajar, hay
que trabajar, granujilla, porque satisfagas tus sentimientos
caritativos. Eres todo un hombre de bien\ldots{}

---Gracias---repuso Salvador cavilando acerca de la nueva ingeniosidad
de su amigo.

---Ya hablaremos, ya hablaremos---dijo Campos.---Ahora tenemos el
Marqués y yo muchas cosas en qué pensar. Y puesto que te hallamos aquí
tan a punto, querido Monsalud, vamos a darte una buena noticia. ¿Se lo
digo, señor Marqués?

---¿Por qué no?---indicó Falfán de los Godos promulgando el gozo de su
alma por medio de sonrisillas y gestos.

---El Sr.~Marqués se nos casa---dijo Campos, acariciando la espalda del
exento.---Ya supondrás con quién. Con mi sobrina.

Monsalud se quedó blanco y frío. Punzada agudísima hizo estremecer de
dolor su corazón. Afortunadamente, la sala estaba oscura, y la emoción
del joven, que se esforzaba en disimular, no fue advertida.

---Es un proyecto improvisado, sin duda---dijo pasándose la mano por la
frente para apartar la negrura que le caía sobre los ojos.

---Ya venimos pensando en esto hace algún tiempo. Pero el Sr.~Marqués no
ha necesitado hacer grandes esfuerzos para cautivar a la hermosa
americanilla.

---Pongamos las cosas en su verdadero lugar---dijo Falfán de los Godos
haciendo alarde de buen sentido.---No soy un vejete de comedia, bien lo
sabe el amigo Monsalud. Conozco la fecha de mi nacimiento y la
desproporción que existe entre mi edad y la de Andrea. Por eso no he
caído en la ridiculez de pretender inspirar a la niña una pasión
formidable\ldots{} Verdad es que no soy un mamarracho, y mis cincuenta
ofrecen un aspecto tolerable\ldots{} pero no; nada de pasiones
exaltadas. Yo me contento, amigos míos, con haber logrado, como es
evidente, inspirar a Andreíta un amor tranquilo y sesudo\ldots{} pues,
sesudo; un amor que a las dulzuras propias de este sentimiento reúna las
sabrosas insulseces de la amistad. Me satisface, además, completamente,
el saber que las primicias sentimentales del corazón de esa tierna
criatura van a ser para este goloso que indudablemente no las merece.

---Eso sí, amigo Falfán---manifestó Campos:---la prenda que se lleva
usted excede a todos los elogios. No es porque sea hija de mi querido
hermano, ni me ciega el amor de tío que le profeso; pero la verdad por
delante. Existen pocas muchachas como Andrea. Nada hay que decir de su
belleza que está a la vista de todos; ¿pero y su talento, y sus
virtudes, y su piedad, y su genio manso y apacible, y aquella bondad
deliciosa que convida a entregarle el corazón? Un defecto tiene, y por
lo mismo que está delante el que va a ser su marido, lo digo\ldots{} ya
hemos hablado de esto el Marqués y yo; pero este defecto es de los que
dejan de serlo cuando se está en posición holgada y opulenta, como la
que tendrá la marquesa de Falfán de los Godos\ldots{} la marquesa, sí,
sí; ¿por qué no se ha de decir? He encargado hoy mismo una magnífica
palangana de plata con las armas y el hermoso lema \emph{Vallifanius
Gothorum}\ldots{} pues volviendo al defectillo\ldots{}

---No hay que fijarse en una inclinación propia del bello sexo y que
frecuentemente adorna a las que han nacido hermosas---dijo el
Marqués.---¿No es verdad, querido Aristogitón?

---Seguramente. El señor Campos se refiere a la pasión del lujo y al
delirio de las galas y atavíos para realzar la hermosura.

---Andrea se ocupa excesivamente de engalanar su persona---dijo
Cicerón;---pero esto, que sería imperdonable en la esposa de un
menestral, ¿puede vituperarse en la mujer de un prócer millonario? De
ninguna manera.

---Al contrario---indicó Monsalud,---la alta posición exige un esmero
constante en la persona, cultivar el lujo, favorecer las artes; con lo
cual, una dama elegante da lustre a su marido y a la casa cuyo nombre
lleva.

---¡Oh! Ha hablado usted acertadamente---dijo el Marqués, echándose
atrás y dándose golpecitos en la boca con el puño de su bastón.

---¿Pero qué hace esa chiquilla, que no viene?---exclamó con impaciencia
Campos.---¡Andrea, Andrea!

Monsalud ante la anunciada presencia de Andrea, sintió una llama en su
pecho. Resolvió esperar.

---Voy a buscarla---dijo Campos.---Vaya, que nos obliga a hacer unas
antesalas\ldots{}

Cuando el Marqués y Salvador se quedaron solos, aquél pegó la hebra como
suele decirse, en la política, espetando a nuestro amigo un trozo
literario que bien podría haber pasado por artículo de fondo en las
graves columnas de \emph{El Universal}, órgano entonces de la gente
templada. Poca o ninguna atención ponía el angustiado joven a los
atildados párrafos y discretas observaciones del Marqués, que supo hacer
un resumen de la famosa \emph{coletilla} añadida por el Rey a su
discurso de apertura en la solemnidad constitucional de aquel día 1.º de
Marzo de 1821. Emitió después varios juicios, todos muy templados y
sesudos, acerca del estado general de la cosa pública, de la caída del
Ministerio, del conflicto parlamentario que debía suceder al acto
imprudente de la Corona; dirigió una ojeada en redondo al inmenso
círculo de los sucesos y de las personas, señalando fenómenos
desconsoladores, previendo desastres, anunciando terribles hundimientos
y naufragios de esa viejísima \emph{nave del Estado}, en la cual la
literatura política de todos los tiempos y lugares ha hecho tantas
travesías.

Como se atiende a la lluvia cuando no se piensa salir a la calle, así
atendió Monsalud al chubasco verbal del Marqués. Dejábale hablar. Al
través de aquel nublado, el desairado amante no veía más que el cielo
que había perdido. Estaba anonadado cuando regresó Campos. El semblante
de éste revelaba tristeza y contrariedad.

---¿Qué hay?---le preguntó Falfán.

---Nada, que esa mocosilla se nos ha puesto mala.

---Que vayan a buscar un médico\ldots{} ¡Pronto, un médico!---exclamó
con agitación el exento, levantándose y dirigiendo brazo y bastón al
Oriente y Occidente, como general que da órdenes en una batalla.

---No es para tanto.

---¿Puedo pasar a verla?

---Creo que sí---dijo Campos con oficiosa complacencia.---Pero
ahora\ldots{} Querrá dormir un rato\ldots{} Puede usted pasar si gusta,
al cuarto de Romualda, que acaba de llegar.

Falfán salió.

Al verse solo con Campos, sintió Monsalud, que en su pecho nacía uno de
esos accesos de coraje que al varón más prudente impulsan a acciones
violentas y brutales. Levantose con los dientes apretados, las manos
crispadas\ldots{}

Campos vio que sobre él caía una tempestad. Cruzando las manos en ademán
de súplica, detuvo al joven, diciéndole:

---Monsalud, por tu honor, por tu vida, cálmate\ldots{} Soy tuyo, soy
todo tuyo, te pertenezco. Pídeme lo que quieras. Da conseguido lo que
pretendes. Tu pariente, tu padre o lo que saldrá de la cárcel\ldots{}
pero no hagas escándalos, no me comprometas\ldots{} por Dios y por la
Virgen Santísima, no alces la voz.

Monsalud vaciló un instante, hizo un esfuerzo para dominar su cólera, y
después dijo:

---¿A qué tanta farsa? Hablemos con claridad.

---Sí, con claridad---repuso Campos muy agitado.---He descubierto todo.
Yo soy aquí el engañado, yo soy aquí el ofendido, porque has infamado mi
casa; pero te perdono, te lo perdono todo con tal que te vayas y no
vuelvas más, con tal que desaparezcas y no existas para mi
sobrina\ldots{} Yo tengo derecho ello; tendría derecho a quitarte hasta
la vida; pero lo pasado, pasado. Vete. Ya sabes que he querido
favorecerte; no te quejarás de mí. En cambio te pido que huyas, que
desaparezcas, que no existas más para mi sobrina. Si quieres, te lo
pediré de rodillas, y será gracioso ver a un \emph{Valeroso Príncipe del
Real Secreto} de hinojos ante un triste \emph{Caballero Kadossch}. Vete
y búscame lejos de aquí para ponerme a tus órdenes. ¿Quieres que se
suelte a todos los reos que hay en Madrid? Se soltarán, se soltarán con
tal que no existas más para Andrea.

---¡Andrea!---exclamó Monsalud procurando traducir en expresiones de
desprecio la furia de su alma.---¡Yo la desprecio como te desprecio a
ti, farsante!

Sin oír las palabras que Campos balbucía, el amante engañado salió de la
casa.

\hypertarget{xv}{%
\chapter*{XV}\label{xv}}
\addcontentsline{toc}{chapter}{XV}

Monsalud se ocupó durante gran parte del día en diversos asuntos que no
podía abandonar, por muy perturbado que su ánimo estuviese. Cuando fue a
su casa, mucho más temprano que de costumbre, Solita con toda la
inocencia de su alma, le dijo estas palabras:

---Hermano, hoy sí que te ha soltado pronto tu novia.

La muchacha se quedó muda de asombro y terror al ver que la broma no era
recibida, como de costumbre, con simpatía y buen humor. El semblante de
su hermano indicaba una agitación extrema, y sus labios descoloridos
articulaban sílabas silenciosas.

---Déjame en paz---le dijo con bruscos modos.---No seas impertinente.

Solita temblaba como un criminal arrepentido. Su impertinencia se le
representaba en la imaginación cual horrendo delito. Después de meditar
breve rato, creyó que el mejor medio para lavar su falta era pronunciar
algunas palabras que destruyeran el deplorable efecto de las anteriores.

---¿Te pasa algo?---preguntó con mucho interés.---¿Estás enfermo?

Monsalud alzó la cabeza, mostrando a los atónitos ojos de Solita los
suyos, llenos de extraño fuego.

---No me pasa nada. Ya hace media hora que estás plantada en la
puerta---dijo el hermano en tono durísimo.---¿Me dejarás al fin en paz?
Sola, Sola, ¿por qué eres tan pesada?

Esta reprensión era demasiado fuerte para el alma asustadiza de la hija
del realista. Sintió una congoja que le desgarraba el corazón, y casi,
casi estuvo dispuesta a arrojarse de rodillas delante de su hermano,
pidiéndole que la perdonase. Pero el temor de enojarle más la contuvo.
Tal era su sobresalto, que hasta temía molestarle con el ruido de sus
pasos al retirarse. Hubiera deseado poder huir sin moverse, sin correr,
sin andar, desapareciendo como una sombra o apagándose como una luz.

---Te he dicho que no necesito nada---repitió Salvador, deteniéndose
ante ella, después de dar varios pasos por la habitación.

Un instante después Monsalud se hallaba solo consigo mismo. Midió la
pieza de largo a largo varias veces con agitado paseo; sentose luego, y
apoyando los codos en la mesa, puso la cabeza entre las manos, como si
necesitara aquélla de estos dos puntales para no caerse del busto. Al
cabo de un rato de dolorosa meditación sobre su desaire, la voluntad, o
mejor dicho, la misteriosa fuerza reparadora que en el orden físico
poseemos, empezó a trabajar dentro de él. Trataba de consolarse,
imaginando razones positivistas que atenuaran el desconsuelo total de su
alma, curando además la profunda herida abierta en su amor propio. Pero
en estos casos de sensibilidad hondamente excitada, las razones
positivistas, por ingeniosas que sean y aunque emanen de la dialéctica
más segura, son como los medicamentos que el criterio vulgar llama paños
calientes, que o no hacen nada o exacerban el mal.

El dolorido razonaba admirablemente, y mientras mejor razonaba,
argumentando contra su propio dolor, más crecía éste, con más fuerza
hincaba su agudo diente, más avivaba sus inextinguibles ascuas. Una
lógica incontrovertible demostraba que habría sido gran error contraer
matrimonio con Andrea: en el carácter de la americana había un germen
maléfico cuyas consecuencias érale fácil prever a la razón fría. Pero
armas tan sutiles no eran poderosas contra la sensibilidad inflamada.
Calmada ésta, consideraba Monsalud con elevación el mal que padecía,
generalizando sus desgracias y sometiendo todas las ocurrencias
desdichadas de su vida a una ley fatal, que presidía sus tristes
destinos, como las estrellas de la antigua nigromancia.

---Otra equivocación---decía,---otra caída, otro desengaño. Todo aquello
en que pongo los ojos se vuelve negro. Si mi corazón se apasiona por
algo, persona o idea, la persona se corrompe y la idea se envilece.
Conspiro, y todo sale mal. Deseo la guerra, y hay paz. Deseo la paz, y
hay guerra. Trabajo por la libertad, y mis manos contribuyen a modelar
este horrible monstruo. Quiero ser como los demás, y no puedo. En todas
partes soy una excepción. Otros viven y son amados; yo no vivo ni soy
amado, ni hallo fuente alguna donde saciar la sed que me devora.
¿Amigos? Ninguno me satisface. ¿Artes? Las siento en mí; pero no tengo
educación para practicarlas. ¿Amor? Siempre que me acerco a él y lo
toco, me quemo. ¿Religión? Los volterianos me la han quitado, sin
ponerme en su lugar más que ideas vagas\ldots{} Dios mío, ¿por qué estoy
yo tan lleno y todo tan vacío en derredor de mí? ¿En dónde arrojaré este
gran peso que llevo encima y dentro de mi alma? Voy tocando a todas las
puertas, y en todas me dicen: «Aquí no es, hermano; siga usted
adelante». Voy siempre adelante. Algún ser existe, sin duda, que está
sentado junto a su casa, esperándome con ansiedad; pero yo paso y vuelvo
a pasar, subo y bajo, entro y salgo con mi carga a cuestas, y no doy
jamás con la puerta de mi semejante. Voy aburrido y desesperado, ando
sin cesar. «¿Será aquél?», me pregunto. Creo haber acertado, y una
brutal mano me lanza al camino diciendo: «Sigue adelante, que aquí no
es\ldots» «Aquí no es, aquí no es, aquí no es». En toda mi vida no oiré
sino estas desesperantes palabras. «Aquí no es», me dijo Jenara. «Aquí
no es», me dijo el partido jurado. «Aquí no es», me dijo la emigración.
«Aquí no es», me dijo la patria. «Aquí no es», me dijeron las logias del
año 19. «Aquí no es», me han dicho los liberales de ahora. «Aquí no es»,
me acaba de decir Andrea. No es en ninguna parte, y yo moriré de
cansancio y fastidio en medio del camino. ¡Maldita sea la hora en que
nací! Hijo soy del crimen, y la expiación de él tomó carne y vida en mi
persona miserable\ldots{} ¿Por qué soy tan distinto de los demás, que en
ninguna parte encajo? ¿Por qué ningún hueco social cuadra a mi forma?
Mejor es desbaratarse y morir, ¡Dios mío!, que estar siempre de
más\ldots{}

Al concluir esta serie de razonamientos, que brotaban en su cerebro como
chispas de un hierro candente herido en la fragua por el martillo, dio
repetidos golpes con la frente en la dura tabla de la mesa.

¡Pobre hombre! La verdad es que teniendo los medios vulgares para ser
feliz, no podía serlo, sin duda por repugnar a su naturaleza los
vulgares medios. Pero se equivocaba al echar la culpa de sus
contrariedades al destino, a las estrellas, a una crueldad sistemática
de la Providencia, como es frecuente en los que razonan poco; las causas
de su constante desaliento y de sus caídas teníalas dentro de sí mismo,
y se atormentaba constantemente en virtud de una poderosa fuerza
crítica, compañera de todos sus actos. Sin quererlo, su mente le
presentaba con claridad suma todas las abominaciones y fealdades de
hombres y de la vida, exagerándolas quizás, pero sin perder ninguna. Por
eso, cuando el natural orden de compensaciones que preside a la
existencia le conducía a una situación lisonjera y optimista, el amor,
por ejemplo, se abrazaba a ella con la desesperación del náufrago; y
despertando todas las fuerzas de su ser, las dirigía al caro objeto; se
apasionaba y exaltaba tanto, como si toda la vida debiera condensarse en
una semana y el universo entero en las sensaciones y los espectáculos de
un día. Cuando el desengaño llegaba, natural invierno que con orden
incontrovertible sigue al verano de la pasión y del entusiasmo, le
sorprendía a tanta altura que sus caídas eran desastrosas. Otros caen de
una silla y apenas se hacen daño. Él, que siempre se encaramaba a las
más altas torres, quedaba como muerto.

Otra causa le hacía infeliz, la desproporción inmensa entre sus
condiciones sociales o de nacimiento y la superioridad ingénita de su
inteligencia y de su fantasía. La fantasía le incitaba a todas horas con
vivaces estímulos: era como un aguijón constante que intentara hacer
correr a quien carece de pies. Considerad una inspiración ardiente sin
medios de manifestarse, semejante a la curiosidad óptica del ciego; una
inspiración que daba el fuego sin combustible, el agua sin vaso, la idea
sin la palabra, sin la línea, sin la nota; considerad un alto ingenio
que no sabe más que leer y escribir en una época en que el arte tiene
que ser letrado porque han desaparecido los bardos y los trovadores de
camino, y comprenderéis cómo pesa sobre un alma la fantasía cuando la
falta de educación la ha privado de sus sentidos propios. Es verbo
inencarnado que lucha en las tinieblas con horrendo torbellino,
queriendo ser forma y sin satisfacer jamás su anhelo doloroso.

Salvador tenía pasión por la música. Al establecerse en Madrid el año 18
creía en su candor (pues su alma era en el fondo excesivamente
candorosa), que aquel arte estaba al alcance de todo el mundo. Ignoraba
las inmensas dificultades técnicas, jamás vencidas después de la
infancia, que caracterizan el arte más amable y más profundamente
patético en la vaguedad soñadora de su expresión. Con estas ideas,
Monsalud compró un piano. Creía que en el clave todo es, como
vulgarmente se dice, coser y cantar. El desengaño vino al instante, y el
pobre joven se encorvaba con desesperación sobre el ingrato instrumento,
y sus dedos de hierro herían las teclas sin poder hacerles hablar más
que un lenguaje discorde y estrepitoso. Al mismo tiempo trataba de
explorar el mundo de aritmética y de armonía comprendido en las cinco
rayas de la cábala musical, y su mente caía rendida ante un trabajo que
exige paciencia sinfín y árida práctica. Un día le sobrevino un arranque
de ira durante los estudios musicales, que asemejaban su casa a un
conservatorio de locos, y tomando un martillo, dijo a las teclas:

---¿No queréis responderme? Pues tocad ahora.

Y las despedazó. La caja no tuvo mejor suerte, y una vez vacía, la llenó
de legajos. El clave sufrió la suerte de los hombres que a cierta edad
se vacían de ilusiones y se llenan de positivismo.

La poesía escrita le cautivaba sobremanera. También se le antojó ser
poeta escrito, lo cual es muy distinto de poeta sentido; pero tropezó
con el inconveniente de no saber de nada, grave contrariedad que estorba
mucho, aunque no tanto como al músico la ignorancia técnica de su arte.
El poeta puede salir de su atolladero con libros, y en aquel tiempo,
aunque pocos, había libros. Lo que principalmente faltaba era espíritu
literario, que es la atmósfera del artista; faltaban público y amigos
tocados de la misma debilidad versificante, porque cuanto respiraba,
respiraba entonces con los pulmones de la política. Salvador creyó, sin
embargo, que en sí mismo encontraría todo lo necesario, es decir, poeta,
espíritu poético, público y hasta el aplauso, que también es musa.
Compró libros, empezó a desflorar aquí y allí; pero ¡ay! a las primeras
tentativas vio que le faltaba una musa imprescindible, una musa sin cuya
condescendencia no es posible hacer absolutamente nada: le faltaba
tiempo. No sabemos lo que habrían hecho Homero y el Dante con su inmensa
inspiración si no hubieran podido consagrar a los versos ni aun medio
minuto; si hubieran tenido que ganarse la vida trabajando dieciséis
horas en áridas cuentas y fatigosos menesteres; si la obligación sagrada
de mantener a su madre les hubiera quitado toda ocasión de renunciar al
trabajo lucrativo para emprender la gloriosa, agitada y vagabunda vida
de la imaginación.

Un día Salvador se sintió muy malhumorado. Cogió los poetas, y
acordándose de Felipe II, les trató como a herejes.

Aún le quedaba un respiradero, un escape, una vía libre, aunque muy
estrecha, para salirse a sí mismo y quebrantar la ley de concentración y
encierro que le estaba emparedando el alma, digámoslo así; le quedaba el
periodismo, y entonces había una prensa no despreciable, donde la
juventud podía hacer sus juegos. \emph{El Espectador} y \emph{El
Universal}, que hoy nos hacen reír, eran órganos hasta cierto punto
afinados y sonoros. Salvador no dejó de hacer la prueba; pero bien
pronto aquel displicente espíritu crítico de que antes hablamos le hizo
aborrecibles las redacciones, como le hizo aborrecibles más tarde las
logias, los clubs y la política.

Mas de repente descendió para él de ignorado cielo la hermosa figura de
Andrea. Entonces las artes todas, que antes no habían tenido nota ni
palabra, se realizaron. Andrea era la música, la poesía, la pintura, la
estatuaria, hasta la arquitectura y la danza; era también, si se quiere,
el periodismo, la gran política, la vida toda en fin. El arte tiene
distintos caminos para satisfacer el alma: unas veces va por el camino
de los lienzos y de las notas; otras por los derrumbaderos de la pasión,
entre tormentos y goces infinitos. Como quien lo tiene todo, como quien
recoge a manos llenas abundantes frutos y flores en todas las ramas del
gran árbol del espíritu, Salvador estaba satisfecho: las teclas habían
respondido, y sin notas ni versos, poesía y música habían saciado su
sediento afán.

Corrieron días felices. Él, sin embargo, se proporcionaba el placer de
atormentarse pensando en la probabilidad de perder a su amada; y su
cavilación, despertando otros recuerdos y estableciendo los términos
sistemáticos de su desgracia, llegó a darle la seguridad completa de un
conflicto. El alma se defendía rabiosamente contra aquella alevosa
guerra de distingos y sutilezas. Por adorar, hasta adoraba los defectos
de Andrea, mejor dicho, veía en ellos gracias nuevas y donaires
desconocidos, por cuyo motivo, en el momento de la catástrofe, le hemos
visto rechazando las razones positivistas con que el pérfido
\emph{intellectus} trataba de arrancarle su hermoso sueño. Andrea era
para él la totalidad de las satisfacciones humanas y el ideal de la
vida. La amaba en globo, con sus defectos, conociéndolos y aceptándolos
como se aceptan sin la más leve protesta de los ojos las manchas del
Sol. Ni por un momento pensó en apartarse de ella por causa de tales
lunares, accidentes encantadores que se confundían con las perfecciones,
sin que el ciego amor pudiera decir dónde acababa Dios y empezaba Satán.
El egoísmo estupendo del amor ahogaba entonces en Monsalud la potencia
crítica que en él hemos reconocido. Para que uno y otro se separaran era
preciso, pues, que mediase una gran violencia o una traición de ella.
Ésta vino, como hemos visto, y el pobre hombre, dolorido y desesperado
por la conmoción de la caída, meditaba en la noche que siguió al día del
desengaño, buscando una especie de recreo en su propia pena, y golpeaba
en la tabla del bufete con su cabeza, cual si ésta fuera un caldero
lleno de absurdos, que merecía ser roto y desocupado.

Entre tanto, Solita, llena de consternación por lo que había visto y
oído, se retiró. No se apartaba de su mente la idea de que Salvador
sufría algún mal muy grande. ¿Cómo consolarle, cómo aliviarle al menos?
Por último, cavilando durante largo rato, sus ideas variaron.

---Ya adivino lo que es---dijo.---Salvador está triste y enojado porque
tiene malas noticias de la causa de mi padre.

Al instante corrió en busca de Doña Fermina. Manifestole lo que había
pasado, y las dos deliberaron si debían esperar a que él revelase la
causa de su malestar o interpelarle desde luego sin miedo.

---Esperemos---dijo la madre.---Si da en callar, no le sacaremos una
palabra.

No había concluido de decirlo, cuando sintieron la voz de Monsalud.

---¡Madre, madre!\ldots{} ¡Soledad!

Corrieron allá.

---Madre\ldots{} Soledad\ldots---repitió Salvador viéndolas
entrar.---Aquí no tiene uno quien le acompañe\ldots{} le dejan a uno
morirse de tristeza. Ni siquiera vienen a preguntar si se me ofrece
algo.

El semblante del joven expresaba una reacción viva en sentido
consolador. En lo más extremado de su pena, sintió que ésta se agrandaba
con el aislamiento, y un poderoso instinto de restauración le impulsaba
a rodearse de personas queridas.

---Hijo, si estamos aquí\ldots{} Sola me ha dicho que la has despedido
con dos piedras en la mano---dijo Doña Fermina.

---Ha sido una broma---indicó Monsalud, sintiendo remordimiento por
haber tratado mal a su protegida.---Solilla, siéntate aquí y trabaja en
mi cuarto. Necesito que me acompañes.

---¿Tienes que decirnos algo desfavorable del pobre D. Urbano?

---Nada, nada; todo lo contrario. Espero sacarle pronto de la cárcel.
Hoy precisamente han variado las cosas.

Solita miró con expresión de incredulidad a su hermano.

---¿No lo crees?\ldots{} Pronto verás que no te engaño\ldots{} Una
circunstancia imprevista lo arreglará todo. ¿Estás enfadada conmigo
porque te dije impertinente?

---¡Qué tonto eres!---respondió la de Gil de la Cuadra, toda ruborosa y
turbada.---Nada de lo que tú hagas o digas me puede enfadar. ¿Qué
importa una palabra de más o de menos? Bien sé que eres muy bueno para
mí.

---Gracias, hijita. Haces bien en tener esa confianza en el hombre que
va a ser\ldots{}

---¿Qué?

---Padrino de tus muñecos. Tengo ganas de ser padrino de algo. Sin
embargo, más vale que no sea yo padrino de ellos.

---¿Por qué?

---Porque se morirían.

---¿Pero es verdad que no nos engañas? ¿Hay esperanzas de que el Sr.~D.
Urbano?\ldots---volvió a preguntar Doña Fermina.

---Sí, empiezo a creer que lograré mi objetivo. ¡De qué caminos tan
extraños se vale la Providencia!

---¿Pero es cierto, es verdad lo que dices?---exclamó Sola derramando
lágrimas de ternura.---¡Mi padre libre!

---El corazón---dijo Doña Fermina,---me ha estado diciendo todo el día
que se nos preparaba un acontecimiento feliz.

---Y yo---añadió Solita con emoción profunda,---también he tenido hoy
unas corazonadas\ldots{} Anoche soñé que me asomaba al balcón y que veía
a mi padre entrando en la calle. El pobrecito me saludaba con la mano,
dándose tanta prisa a entrar y subir la escalera, que tropezaba a cada
momento.

---Es particular---dijo la madre.---Yo también soñé anoche una cosa
parecida.

---Es particular---dijo Monsalud.---Sin duda es ésta la casa del sueño.
Hace poco me quedé aletargado y soñé\ldots{}

---¿Que mi padre estaba libre?

---Sí; pero mira de qué modo tan extraño. Yo me dirigía por la calle de
la Cabeza a la cárcel de la Corona. Llegué a la puerta y me salió al
encuentro, ¿quién creerás que me salió al encuentro?

---¿Un centinela?

---¿Un carcelero?

---Un perro.

---Lo mismo da.

---Un perro, no de tres cabezas, como el del Infierno, sino de una sola;
pero tan horrible, que su vista me hacía temblar de sobresalto y pavor.
Sus ojos despedían fuego, y su espantosa boca, llena de cuajarones de
sangre, se abría hasta las orejas dejando ver feroces dientes agudísimos
y una lengua que vibraba como hoja de metal. Era la bestia más
repugnante y fea que imaginarse puede. Pero lo más raro era que aquel
horrendo animal hablaba.

---¿Hablaba?\ldots{}

---Yo le dije que iba a buscar a un infeliz encerrado en la cárcel. El
perro fijó en mí sus ojos de fuego, cuya claridad me llegaba al alma,
estremeciéndome todo.

Las dos mujeres se estremecían también, y los ojos de Solita no estaban
menos espantados que si tuvieran enfrente al temible can.

---El perro dio un gruñido---continuó Monsalud,---y con su voz, que
resonaba como si saliera de honda caverna, me dijo: «Está bien, amigo
mío\ldots»

---¡Amigo mío!\ldots{} Pues no dejaba de ser cortés.

---Está bien, amigo mío---me dijo;---puedes llevarte al preso con una
condición. Ya sabes que yo me alimento de corazones. Dame el tuyo, y
hemos concluido.

---¿Y se lo diste?\ldots{} pero hombre\ldots{} pero hijo\ldots---gritó
Doña Fermina con impaciencia.

---Me clavé las uñas en el pecho, apreté fuertemente, metí la
mano\ldots{}

---¡Jesús!---exclamó Solita, apartando el rostro.

---Metí la mano, me saqué el corazón y se lo arrojé a la bestia, que con
su feroz boca lo cogió en el aire. Entré, y cuando salía, sacando al
señor Gil, vi que el perro mascullaba el pedazo de carne, saciándose en
él. ¡Ay, cuánto me dolía!

\hypertarget{xvi}{%
\chapter*{XVI}\label{xvi}}
\addcontentsline{toc}{chapter}{XVI}

Salvador se inquietaba bien poco de un acontecimiento que por aquellos
días, los primeros de Marzo, agitaba hondamente el mar de la política,
produciendo borrascas, zozobras y naufragios. ¿Necesitaremos recordarlo,
a pesar de haber hablado de él, por cierto con mucha discreción, el
marqués de Falfán de los Godos? Olvidando las prácticas constitucionales
o haciéndose el tonto, que es la opinión más autorizada, añadió el Rey
al discurso de la Corona un parrafillo de su invención, en el cual se
quejaba de los insultos que diariamente recibía, y acusaba con este
motivo a los ministros y a las autoridades de Madrid. Alborotose el
Congreso, alborotáronse más los clubs, los ministros estaban con medio
palmo de boca abierta, sin saber lo que les pasaba, y mientras el Rey
les destituía arrebatadamente, dábales el Congreso un voto de confianza
y una pensioncita de sesenta mil reales; admirable almohada para
reclinar la gloriosa cabeza después de una caída.

Su Majestad, firme en el propósito de hacerse el tonto (y quien crea
otra cosa no sabe hasta dónde llegaba la malicia del astuto \emph{Rey
neto}), pidió consejo a las Cortes para la formación del nuevo
Ministerio, inaudita aberración constitucional, pues el Gabinete caído
tenía mayoría. Los diputados contestaron al mensaje del Rey con un
refunfuño de desconfianza, achacaron a la \emph{mano oculta} los
insultos consabidos, y negáronse a proponer los nuevos-ministros, dando
a entender al Soberano que el Ministerio Argüelles era el mejor de los
ministerios posibles. Fernando consultó entonces al Consejo de Estado, y
de esta consulta salió el Ministerio del 4 de Marzo.

Era natural que el nuevo Gabinete no gustase a nadie. Los tibios le
tenían por exaltado, y los exaltados por tibio. Procedente, como el
anterior, de la mayoría, el Gabinete Valdemoro-Feliú, representaba las
mismas ideas, la propia indecisión, idéntica dependencia de manejos
secretos; representaba también la debilidad frente a los alborotadores,
las pedradas al coche del Rey, la tolerancia de las grandes
conspiraciones y la persecución sañuda de las pequeñas. De entonces
data, si no estamos equivocados, la célebre frase de \emph{los mismos
perros con distintos collares}. Más adelante, cuando Feliú pasó de
Ultramar a Gobernación, el Gabinete se enderezó como una planta cuya
savia se regenera, y supo desplegar contra los alborotadores y los clubs
una energía que hasta entonces no se había visto en el Gobierno después
de la revolución.

Tal era la situación política a principios de Marzo. En el Gobierno,
debilidad; en el Congreso, confusión; en Palacio, solapadas intrigas,
cuyas resultas se verán más adelante. El pueblo, desbordado y sin
reconocer ley ni freno alguno, expresaba su voluntad ruidosa y
groseramente en los clubs. A fuerza de oír hablar de su soberanía,
empezaba a creer que consistía ésta en el uso constante de la iniciativa
revolucionaria y en el ejercicio atropellado de la sanción popular en
asonadas, violencias y atrocidades sin cuento. Romero Alpuente, un
vejete furibundo a quien después conoceremos, había dicho que la
\emph{guerra civil era un don del cielo}. Istúriz, joven y exaltado,
había dicho que la palabra \emph{Rey era anticonstitucional}. Moreno
Guerra, había dicho que \emph{el pueblo tiene derecho a hacerse justicia
y vengarse a sí propio}. Golfín, que la anarquía purgaba a la tierra de
tiranos. Otro llamaba al Trono \emph{cadalso de la libertad}.

Entre tanto las sociedades secretas estaban desconcertadas; porque si
bien el nuevo Ministerio saliera de ellas como el anterior, no había
gran seguridad de que se dejase gobernar por los \emph{Valerosos
Príncipes}.

---Estamos---decía Campos,---en la situación más oscura que puede
imaginarse. Yo no he tenido nunca a Feliú por muy afecto a nuestro
Orden, y temo mucho que se nos vuelva en contra. Sin embargo, anoche nos
ha echado un discursejo con muchos ofrecimientos y palabrotas; pero no
me fío, no me fío.

Esto lo decía el gran Cicerón sentado junto a una mesa del café de
\emph{La Fontana}, teniendo enfrente a Salvador Monsalud, que entre
sorbo y sorbo de café leía \emph{El Espectador}. Cómo se habían juntado
después de su violenta separación, cómo habían ido allí, apareciendo
amistosamente reconciliados merced a un par de tazas y otras tantas
copas, es cosa que se explica fácilmente. Campos fue a casa de Monsalud
una mañana, anunciándole que tenía que hablar de asuntos igualmente
graves para los dos, y aunque el joven le recibió con los peores y más
ásperos modos, como

Cicerón no se daba por ofendido y era hombre que respondía con risas a
las palabras duras, bien pronto uno y otro, a pesar de su desacuerdo,
hallaron un término común de reconciliación pasajera. Campos convidó a
Aristogitón a pasar un par de horas en \emph{La Fontana}, y una vez allí
sentáronse en el más apartado y oscuro rincón del local, tras la tribuna
y no lejos del mostrador. Casi estaban solos, porque en tal hora el
célebre club \emph{de los amigos del orden} descansaba de sus fatigas.

---Pero a pesar de todo, nosotros no hemos perdido nada todavía---añadió
Campos,---y yo quiero ver quién es el guapo que se atreve a dar un golpe
a las sociedades secretas, autoras no sólo de la revolución de España,
sino de las de Portugal y Nápoles. Este poder inmenso no se pierde por
una veleidad ministerial\ldots{} Conque, amado Aristogitón, yo planteo
nuestra cuestión en los mismos términos en que la planteé en mi casa
hace ocho días, cuando te pusiste como un basilisco, y aun creo que
intentaste pegar a tu maestro\ldots{} Pero, hombre de Dios, ¿no me haces
caso de lo que te digo? Mientras hablo, tú lees.

---Oigo perfectamente---dijo Monsalud, dejando el periódico y tomando la
taza.---La cuestión planteada en los mismos términos de aquel
día\ldots{}

---Cuando me quisiste pegar---repitió Campos con burla.---Después me
estuve riendo de ti dos horas. Si yo fuera un hombre terrible, te
hubiera echado por el balcón; estaba en mi derecho.

---No lo niego. Si yo hubiera sido un hombre imprudente, le hubiera roto
a usted la cabeza; también estaba en mi derecho por haber sido engañado.
Usted intentó comprarme con viles ofertas de destinos y menudencias.

---Y ahora te compro por el precio que tú te has puesto: por la
concesión de una gracia a que das suma importancia. La cosa en sí es la
misma: no varía más que el precio y la clase de moneda. Tú me dejas en
paz a mi sobrina\ldots{}

---Y usted me pone en la calle a un pobre preso que será ahorcado si las
cosas siguen por el camino que llevan.

---Perfectamente. Trato clarísimo y que no da lugar a engaños ni malas
interpretaciones. \emph{Do ut des}.

Campos como hombre que ve adelantar satisfactoriamente una negociación
de importancia, respiró con fuerza, embaulando después media taza.
Robespierre\footnote{Un gato. Veáse \emph{La Fontana de Oro}.} subió a
sus rodillas. Uno y otro se acariciaron.

---No debieras extrañar---añadió,---que yo quisiera favorecerte con un
buen destino y aun alejarte. A mí me gusta hacer las cosas con
delicadeza. De este modo se llega al objeto sin ofender a nadie, sin
ruido y sin dimes ni diretes. Creí que tú, hombre listo, me entenderías
después del primer avance, y tomando lo que te daba, te dispondrías a
callar y obedecer, dejándome el campo libre. Pero no entendiste. Tienes
un candor honradillo que exige se te digan las cosas claras, y en
verdad, a mí me repugnaba hablarte con claridad en asunto tan espinoso.

---Algo creí entender; pero como no contaba con la traición de Andrea,
no pasé de sospechas vagas.

---¡La traición!---dijo Campos con gravedad irónica.---Pero
hombre\ldots{} ¡qué palabrotas se estilan ahora! Di más bien que mi
sobrina comprendió lo que sacaba del noviazgo contigo. Por mi parte, de
algún tiempo acá me desvelo porque disfrute una posición tónica y como
corresponde a sus méritos. Es tiempo ya de que tenga un padre vigilante
y cariñoso. Te confieso, amigo Aristogitón, que cuando sospeché tus
niñadas con ella, y mas aún, cuando las sospechas se trocaron en
certidumbre\ldots{} ¡ay! sentía impulsos de despedazarte. Pero meditando
bien, resolví tener mucha calma, abordar la cuestión con astucia, evitar
un escándalo que pudiera turbar la paz espiritual del buen Falfán de los
Godos. De esta manera todos quedan contentos. No creas que me ha costado
poco cautivar a Andreílla. La pícara se nos escapaba como una mariposa
cuando creíamos tenerla segura; pero conquistado tú, que eres el
Montjuich, la rendición de la ciudadela es inevitable\ldots{} ¿Te das
por conquistado?

---Me doy por conquistado.

---¿Renuncias por completo y en absoluto a ella? ¿Huirás de su trato y
de su vista, y en caso de que la casualidad te la ponga delante, harás
con ella como si nunca la hubieras conocido?

---Lo haré.

---¿La despreciarás, la arrojarás de tu lado, le harás ver de una manera
indudable que tú y ella sois como el agua y el fuego, que no se pueden
juntar?

---Como el agua y el fuego.

---Y si la tempestad arrecia, ¿serás capaz hasta de hacerla creer que
estás enamorado de otra?

---También.

---Vamos, eres un hombre. Tus declaraciones merecen una \emph{salva}.
Echemos \emph{pólvora fulminante} en el \emph{cañón} y disparemos.

Los masones llamaban pólvora fulminante al \emph{ron}. El \emph{cañón} y
la \emph{salva} ya sabemos lo que eran.

\emph{---¡Fuego!}---dijo Monsalud, llevando la copa a sus labios.

\emph{---¡Fuego!}---repitió Campos.

Los del \emph{Arte-Real}, en sus tenidas de banquetes, pronunciaban esta
voz de mando para indicar los brindis.

---¿Pero a qué vienen tantas exigencias, que parecen pruebas
masónicas---dijo Salvador,---si Andrea no necesita de mis desdenes para
obedecerle a usted? ¿No ha dado su consentimiento?

---¡Ah!, ¡ah!\ldots{} fíate de consentimientos. Dicen que la palabra
\emph{veleidad} es femenina en todas las lenguas. Prueba de que todas
las mujeres son veleidosas. Es verdad que Andrea, a fuerza de ruegos, de
razones, de regalos, de mimos, de promesas, me prometió ser
marquesa\ldots{} ¡marquesa, ya ves qué pedrada!\ldots{} y la muy
tonta\ldots{} Por algo se ha dicho que \emph{entre el sí y el no de una
mujer no se puede poner la cabeza de un alfiler}.

---Ella apetece más. La ambición, una vez desarrollada, no se satisface
fácilmente. Creerá que Falfán de los Godos no es bastante rico.

---Si es millonario. No va por ahí la corriente---dijo Campos con
desaliento.---Es que Andrea vuelve los ojos a este tunante y se
arrepiente, se arrepiente la muy pícara de la promesa que me dio. Desde
el otro día\ldots{} Pero yo quisiera saber qué tienes tú para trastornar
de este modo un cerebro, que después de todo es un cerebro de la raza de
Campos, fecunda en gente sesuda.

---Andrea tiene conciencia; no es una muchacha corrompida---afirmó
Monsalud, disimulando el interés que aquella parte de la conversación le
producía.

---¡Qué conciencia ni conciencia!\ldots{} Resabios tontos de su
enamoramiento infantil. Yo sé que eso desaparecerá; pero por de pronto
me tiene inquieto. Desde aquel día que tú y yo estuvimos a punto de
machacarnos las liendres, no sabes cómo se ha puesto esa muñeca. Está
loca, rematadamente loca, y anoche tuve que encerrarla, porque quería
salir.

---¿Salir?

---A buscarte; y se nos escapará, porque la niña es sutil. Por eso
quiero estar seguro de ti. Querido Aristogitón, si tú no me ayudas, todo
se pierde. No puedes tener idea de cómo está esa criatura. En mi casa no
se oyen más que suspiros, y con las lágrimas que unos ojitos negros han
derramado estos días se podía haber hecho otro estanque del Retiro.
Sorprendila ayer desenvainando el puñal que conserva como recuerdo de su
padre. ¡Ay! qué susto. Te aseguro que si no llego a tiempo, tenemos en
casa una degollina, un suicidio, una de esas gracias que mi sobrina ha
leído en las historias de griegos y romanos, y que ahora las novelas
sentimentales tratan de poner en moda. ¿Has leído el \emph{Werther}? Es
un Dido macho que se mata por amor.

Salvador estaba pálido y no acertaba a decir nada.

---Por esta causa he querido prevenirte, asegurarme de tu formal
renuncia, que espero cumplirás con honradez. Es posible que recibas
alguna esquelita, aunque la hemos privado de tinta y papel; es también
muy probable que la mariposa tienda sus alas y se eche a volar
poéticamente por las calles de Madrid, y te busque y te
encuentre\ldots{} Veo que suspiras\ldots{} mira, no vengas tú también
con suspiros. En una mujer, pase; pero un hombre es un hombre, Salvador,
y, sobre todo, un hombre que tiene a su padre en la cárcel a punto de
ser ahorcado, debe tener corazón de bronce, portarse caballerosamente y
cumplir su palabra.

---Yo la cumpliré---murmuró Salvador.

---Bueno, señor \emph{Caballero Kadossch}. ¿Tú repites las ofertas que
hace poco me has hecho?

---Las repito.

---¿Acabaste para mi sobrina?---preguntó Cicerón en un tono que indicaba
la idea de las resoluciones categóricas.

---Acabé---respondió Salvador en el propio tono del suicida que dice
adiós a la vida.

---¿De modo que no harás caso de esquelitas, ni de recados, ni de
visitas?

---No.

Se frotó los ojos con la mano derecha, cual si quisiera reducírselos a
polvo.

En aquel momento arrojaba su corazón al perro.

\hypertarget{xvii}{%
\chapter*{XVII}\label{xvii}}
\addcontentsline{toc}{chapter}{XVII}

---Pues lo pasado, pasado---dijo Campos.---Amigos otra vez. Olvidemos
las ofensas que mutuamente nos hayamos hecho.

\emph{---Pasemos la trulla}.

\emph{Trulla} era la cuchara de albañil, y la idea de \emph{pasarla}
indicaba olvidar y perdonar las injurias, idea que bien podía expresarse
hablando como la gente.

---Ahora me toca a mí---dijo Salvador.

---Ahora te toca a ti---añadió Campos sacando dos cigarros habanos y
ofreciendo uno a su amigo.---Ahí va esa \emph{pólvora del Líbano}.
Fumemos.

---¿Usted me promete que Gil de la Cuadra no será condenado a muerte?

---Eso no.

---¿Usted me promete que se sobreseerá su causa?

---Tampoco.

---Entonces\ldots{}

---Lo que prometo es que tu padre, tu tío, tu pariente o lo que sea,
saldrá de la cárcel.

---¿Cómo?

---Escapándose de ella, lo cual no es fácil, pero sí posible, sobre todo
si tú y yo nos proponemos hacerlo. No hay que pensar en que el Gobierno
suelte la presa absolutista que tiene entre las garras. Es preciso
ofrecer un par de víctimas al pueblo, y como no se le puede dar un león,
se le da un conejo. Ya sabes que el cura Merino ha aparecido en
Castilla; el \emph{Abuelo} ha levantado también una partida cerca de
Aranjuez y Aizquíbil recorre con su gente el país de Álava. El
\emph{Pastor} entra también en campaña, y a varios de su partida que han
sido pescados, se les encontraron muchos ochentines de los que acuñó el
Gobierno hace poco. Estos ochentines se dieron todos a la Casa Real, de
modo que no hay duda alguna respecto a la mano que está moviendo esta
vil máquina de las partidas.

---El Rey.

---Sí, y cuando los Ministros le hicieron notar la coincidencia,
respondió tranquilamente: «Es muy extraño eso», y no dijo más. La Corte
trabaja con desesperación por encender la guerra civil, y los curas y
los guerrilleros, amparados por ella y por las juntas extranjeras, harán
un esfuerzo terrible para restablecer el absolutismo. Nos aguarda un
porvenir de rosas. Ya sabes lo que significan en nuestro amado país
estas dos fuerzas: \emph{curas}, \emph{guerrilleros}.

---No tengo ilusiones en ese particular. La estupidez de los liberales,
su corrupción y falta de sentido, anuncian a voces que volverá el
absolutismo.

---Pues bien; cuando por todas partes no se ven mas que peligros; cuando
el Gobierno se mira amenazado y provocado por los absolutistas, ¿no es
natural que si logra poner la mano encima de alguno, apriete firme hasta
ahogarle?

---Es natural. Los pobres gazapos que se han dejado coger, pagarán las
culpas de los lobos y de la Corte que los azuza.

---Evidentísimo. Por consiguiente, amigo Monsalud, no hay que pensar en
que el Gobierno perdone a ninguno de los que hoy están presos por
conspiraciones realistas.

---Serán condenados\ldots{}

---A muerte. El juez, Sr.~Arias, confiesa privadamente que no halla
motivo para tanto; pero la presión popular y la necesidad de hacer un
escarmiento, la conveniencia de amedrentar a la Corte, levantará el
cadalso. Aquí tienes la libertad en tales trances que no puede pasarse
sin el verdugo.

---¿De modo que no hay que soñar con un sobreseimiento?

---Locura. Vinuesa no se escapa de la horca. Los demás serán condenados
a presidio\ldots{} Puesto que no podemos evitar la sentencia, tratemos
ahora de salvar a tu hombre. Yo estoy tan comprometido a ello moralmente
como tú. Planteemos la cuestión. Primer punto. Todo el personal de la
cárcel está en poder de gentuza comunera o milicianos nacionales de los
más majaderos.

---Lo sé, y he resuelto hacerme comunero.

---Admirable idea---dijo Campos en tono de lisonja.---Y si procuras
retener en la memoria todos los disparates y gansadas de los hijos de
Padilla para contármelos, tu idea será sublime.

---Yo iré allá tan sólo con el fin de contraer amistades que me sirvan
para nuestro objeto.

---Excelente plan. En tanto el Grande Oriente se encarga de hacer en el
personal de cárceles alguna variación.

---Cosa facilísima.

---No tanto, joven, no tanto. Tú no sabes cuánto se ha alambicado ya en
la cuestión de destinos. No se puede estar trasegando la gente todos los
días. Lo peor de todo es que hacemos una variación, y al punto nos
conquistan los comuneros el nuevo personal. Se varía otra vez, y la
defección se repite. Hacemos tercera hornada; pero llega un momento en
que no se puede más, porque se acaban los carniceros, panaderos y
pasteleros que quieren ser funcionarios públicos en las porterías de los
ministerios, en cárceles, en correos\ldots{} Por este camino va a
desaparecer en Madrid toda la clase menestral.

---Pero los cambios traen numerosas cesantías.

---Pero los cesantes, esos insignes patricios desairados, no quieren
volver a las panaderías, carnicerías y molinos de chocolate de donde
salieron. Encuentran más fácil encastillarse en las \emph{fortalezas} de
Padilla, donde, haciendo comedias, se van adiestrando en la oratoria y
en el arte de conspirar.

---¿Y cómo viven?

---Ese es el misterio. Lo evidente es que tienen dinero. ¿Ves esa
turbamulta de vagos que aúllan en los cafés, que alborotan en la plaza
de Palacio, que apedrean las casas de los Ministros, que van a cantar
coplas indecentes junto a las rejas de la prisión de Vinuesa?\ldots{}
Pues todos ellos viven, y viven bien.

---Los ochentines del \emph{Pastor} harán ese milagro.

---Eso creo yo. Los ochentines\ldots{}

---Pero contra los ochentines, el Gobierno tiene los empleos públicos.
Póngame usted en la cárcel de la Corona a un empleado que se preste a
favorecer nuestro plan.

---Precisamente hay una vacante. Me he informado hoy.

---Mejor que mejor.

---Bueno; pues elige tú el candidato.

Salvador meditó breves instantes.

---Lo mejor será un hombre de bien, pues no se trata de salvar a
ladrones y asesinos; se trata de hacer una buena obra, librando a un
pobre anciano inocente, inocente, sí\ldots{} porque Gil de la Cuadra,
aun conspirando con todas sus fuerzas, no es capaz de hacer daño a un
semejante ni a la sociedad.

---Pues mi opinión es que elijamos un tonto. Es fácil de encontrar.

---Ya tengo mi hombre---dijo vivamente y con alegría Monsalud.

---¿Has hallado el tonto?

---Un maestro de escuela.

---Viene a ser lo mismo. Apuesto a que has pensado en Sarmiento.

---No: lo echaríamos todo a perder---dijo Salvador
arrepintiéndose.---Sarmiento es sencillo, pero su fanatismo rabioso le
transfigura, haciéndole cruel. Me parece que debemos elegir un discreto.

---Bien puedes coger la linterna de Diógenes. Échate a buscar el
discreto.

---Ya lo hallé---exclamó Monsalud, dándose una palmada en la frente.

---¿Quién?

---Yo mismo.

---Hombre\ldots{} la idea no es mala---repuso Campos sonriendo.---Pero
la verdad\ldots{} ese destino no es propio para ti. Vales tú mucho más.

---¿Y qué me importa?

---El duque del Parque no querrá tener a su servicio a un sota-alcaide.

---Dejaré el servicio del duque del Parque.

---¿Pero no se te ocurre otra persona?

---No me fío de nadie. Estoy decidido. Seré sota-alcaide.

---Vas a bregar con la gente más cruel, más perdida y más infame de la
sociedad. El personal de cárceles allá se va con el de encarcelados.

---No me importa. He tenido una idea feliz.

---Pues adelante, y realicemos la idea feliz. Serás sota-alcaide. En
tanto que te nombro\ldots{} pues no creas que es cosa de un momento: lo
menos hay treinta candidatos\ldots{} hablaré a Copons.

---¿El jefe político?

---¡Ah!---exclamó Campos con gozo.---Le tengo cogido, le tengo preso en
mis redes. Precisamente anda tras de mí para que le favorezca en ciertas
pretensiones que trae en Gracia y Justicia. Una bicoca; tres primos que
fueron beneficiados y ahora se les ha antojado ser deanes. Son de la
pacotilla de los que llaman modestos\ldots{} ¡pobrecitos! Copons es muy
exaltado; el Gobierno, que le puso en lugar de Palarea, no está muy
contento con él. Necesita todo el arrimo del Grande Oriente para no
venir a tierra. Muy bien; esto va a pedir de boca. Tu padre, tu abuelo,
o lo que sea, se ha salvado.

Hablaron algo más, determinando algunos detalles del plan, y se
separaron. Campos tenía que revisar unas cartas detenidas por orden
superior. Salvador debía consagrarse a sus ocupaciones. Cuando volvió a
su casa, entregáronle un billete que acababa de llegar. Habiendo
conocido en el sobre la letra de Andrea, sintió tanta ansiedad como
pavor. La carta estaba trazada a prisa, con indecisos rasgos, y decía:

«Arrepentida, arrepentida, arrepentida de lo que he hecho.

»Ven al instante. Estoy esperándote en el Retiro, junto al Observatorio.
Me he escapado de mi casa. Querido mío, mi vida y mi muerte: si no me
perdonas, si no vienes al instante a mi lado, me moriré de
desesperación.

»Lo que he hecho contigo es una villanía, una ofuscación.

»Un poco tarde lo he conocido; pero lo conozco al fin, lo confieso y te
pido perdón.

»Te adoro, y ni Dios podrá hacer que yo pertenezca a otro. Eres mi dueño
y puedes abofetearme, puedes matarme si me porto mal.

»Salvador, sácame del infierno en que estoy. Ven, no tardes ni un
segundo. No vuelvo más a mi casa. Iré contigo a donde quieras: seré tu
esposa, tu criada o lo que tú quieras\ldots{} Sácame los ojos y dentro
de ellos verás tu cara. Ya me parece que te siento venir\ldots{}
¿Vendrás?\ldots{} En el Retiro junto al Observatorio. Voy corriendo, no
sea que llegues antes que yo. Adorado mío, te quiere con toda su alma y
te ofrece el corazón y la vida,

\begin{flushright}
\textsc{Andrea}.»
\end{flushright}

Soledad, que entraba cuando Salvador concluía de leer la carta, notó su
palidez y agitación.

---¿Qué tienes, hermano?---dijo llena de pesadumbre.---¿Ese papel te
dice algo desfavorable a mi pobre padre?

---No, no---dijo el hermano con desesperación.---Es todo lo contrario.
Sola, abrázame, abraza a tu hermano.

La muchacha se arrojó llorando en brazos de Salvador.

---¿Pero te causan pena las buenas noticias?

---¡No, no!\ldots{} La carta no dice nada---exclamó él, sofocando la
tempestad que bramaba en su alma.---Estoy alegre, hermana, hermana
querida, abrázame otra vez. Tu padre se ha salvado.

~

Pasó Monsalud todo el día y toda la noche en un estado de agitación muy
viva. A la mañana siguiente, cuando entró en casa del duque del Parque,
un criado le dijo: «Han estado aquí dos mujeres buscándole a usted.
Parecían ama y criada».

---Si vuelven---repuso,---dígales usted que he salido de Madrid.

Para evitar un encuentro que temía, salió del Palacio por una puerta de
servicio que daba a otra calle. Pero más tarde, al entrar en su casa, D.
Patricio Sarmiento repitió la noticia.

---Aquí han estado dos damiselas a preguntarme cuándo volvía usted.
Parecen ama y criada\ldots{} ¡oh, edad dichosa esta en que nos vienen a
buscar dos y tres veces en el breve espacio de unas horas!\ldots{} Yo
también en mis juveniles años\ldots{}

Sarmiento exhaló un suspiro.

---Si vuelven, dígales usted que he salido de Madrid y que no volveré
hasta dentro de un mes.

---¡Cuánta esquivez!\ldots{} Pero en esa edad feliz\ldots{} También uno
ha tenido sus dulzuras ¿eh? No crea usted: este arrugado semblante y
este flaco y débil cuerpo no han sido siempre así. Aquí, amiguito
Salvador, aquí se sabe lo que es afán de amores; aquí se comprende bien
eso de despreciar a una por apasionarse de la otra, volando de flor en
flor cual inconstante mariposa\ldots{} ¿Pues y estar penando días y días
por una mirada, sólo por una mirada?\ldots{} ¡ay!, ¿y aquello de estar
cavilando por qué me miró así, o dejó de mirarme?\ldots{} Todos hemos
tenido nuestro Abril, todos hemos revoloteado y sacado la miel hiblea
del cáliz de las frescas flores, Sr.~Monsalud.

Cuando este se dirigió después de medio día a una tienda de la calle
Mayor, donde solía hacer tertulia, un mancebo le dijo la muletilla:

---Han estado dos hembras a ver si había usted venido.

Más tarde pasó por la parte baja de la calle de Atocha. Detúvose de
repente porque un objeto lejano llamó su atención: era el Observatorio
astronómico. Singular trastorno debió de producir en las ideas del joven
la vista del hermoso edificio, porque apresuró el paso como quien huye
de un fantasma temible.

¡Cosa extraña! Al anochecer, cuando fue al local ocupado por la
masonería en la calle de las Tres Cruces, con objeto de hacer unas
preguntas a Sócrates, o como si dijéramos, a Canencia, un portero le
cantó el atormentado estribillo de todo el día:

---Aquí han estado dos damas a preguntar si vendría usted esta noche.

Después marchó a \emph{La Cruz de Malta}, café situado en la calle del
Caballero de Gracia. Aguardábale allí D. José Manuel Regato.

\hypertarget{xviii}{%
\chapter*{XVIII}\label{xviii}}
\addcontentsline{toc}{chapter}{XVIII}

En la calle que hoy se llama de Isabel la Católica, y antes de la
Inquisición, pasando así bruscamente del nombre más horrible al más
hermoso, hay una casa que hoy lleva el número 25 y antes tenía el 2,
edificio perteneciente en su juventud al conde de Revillagigedo y que
después fue Conservatorio de Música y Declamación. Diversas oficinas se
han sucedido en dicha casa, y hoy sirve de albergue, si no estamos
equivocados, a una Dirección del ramo de guerra. Pero lo más importante
de este caserón en su variada y larga historia, es que dentro de él
estuvo la \emph{Asamblea de los Comuneros} durante los tres
\emph{llamados años}. Ya se habrá comprendido quiénes eran estos bravos
hijos de Padilla. Cualquiera que haya vivido en España y prestado
atención a sus cosas políticas, comprenderá que en aquella época, como
en todas, los descontentos y los cesantes y los atrevidos y los
pretendientes y los envidiosos, que son siempre el mayor número, no
podían tolerar que determinada pandilla gobernase siempre el país y las
Cortes. Este afán de renovación periódica del personal político que en
otras partes se hace por razón de ideas y de aspiraciones elevadas, se
suele hacer aquí, y más entonces que hoy, por el turno tumultuoso de las
nóminas. Esto es una vulgaridad tan manoseada, y ha trascendido de tal
modo hasta llegar a las inteligencias más oscuras, que casi es de mal
gusto ponerlo en un libro.

Los comuneros querían reformar la Constitución, porque no era bastante
liberal todavía. Los ministeriales (nos referimos a la primera mitad de
1821) o doceañistas, o si se quiere, los \emph{masones}, convencidos de
que su Constitución era la mejor de las obras posibles, y que la mente
no concebía nada más perfecto, querían que se conservase intacta y sin
corrección ni reforma como la Naturaleza. De repente apareció un tercer
partido llamado de los \emph{anilleros} que quiso modificar la
Constitución en sentido restrictivo, aspirando a una especie de
transacción con la Corte y la Santa Alianza. Sobre estas tres voluntades
giraba aquel torbellino que empezó con una sedición militar y terminó
con una intervención extranjera.

Los comuneros, que nacieron del odio a los masones, como los hongos
nacen del estiércol, creyendo que los ritos y prácticas de la Masonería
eran una antigualla desabrida, anti-española, prosaica y árida,
imaginaron que les convenía establecer un simbolismo caballeresco y
nacional, propio para exaltar la imaginación del pueblo y aun de las
mujeres, que por entonces tenían parte muy principal en estos líos.
Siendo la representación primaria de los masones un templo en fábrica y
los hermanos, arquitectos o albañiles, los comuneros, formaron su
partido de Comunidades, divididas en Merindades y Torres y
Casas-Fuertes, y a sus logias llamaron \emph{Castillos} y a sus
Venerables \emph{Castellanos}, \emph{Alcaides} a sus Vigilantes, y así
sucesivamente. En los ritos y ceremonias modificaron todo lo que hay de
teatral en la Masonería; pero dándole forma caballeresca, e ideando
ilusorias fortalezas, puentes levadizos, barbacanas, recintos, salas de
armas, cuerpos de guardia, almacenes de enseres y demás mojigangas, todo
creado por sus exaltadas fantasías, de tal modo, que más que militantes
caballeros parecían rematados locos.

Su color distintivo era el morado, así como los masones adoptaron el
verde. La Asamblea general recibía el nombre de \emph{Alcázar de la
Libertad}, y el recinto donde se reunían, llamado \emph{Plaza de Armas},
estaba adornado con embadurnados lienzos y telones, representando
torreoncillos con banderolas, lanzas y las indispensables inscripciones
patrioteras. El Presidente llamaba a los socios la \emph{guarnición} y a
los neófitos \emph{reclutas}. Abríanse y cerrábanse las sesiones con
fórmulas que harían reír a la misma seriedad, siendo de notar
principalmente el parrafillo con que se despedían después de discutir
largamente sobre mil innobles temas sugeridos por el egoísmo, el hambre
o la envidia: «Retirémonos, compañeros, a dar descanso a nuestro
espíritu y a nuestros cuerpos, para restablecer las fuerzas y volver con
nuevo vigor a la defensa de las libertades patrias».

Poco después de las diez de la noche Salvador Monsalud, acompañado del
Sr. Regato, penetró en el \emph{Alcázar de la Libertad} de la calle de
la Inquisición. Era el local grande y espacioso, consistente en una
serie de salas abovedadas a las cuales se descendía por media docena de
escalones. Pobres farolillos que aquí no cometían la fatuidad de
llamarse \emph{estrellas} las alumbraban, y un sordo rumor de gente
anunciaba desde el vestíbulo que la colmena se había llenado ya de
zánganos.

---El ceremonial nos manda esperar aquí---dijo Regato a su recluta,
deteniéndose en la primera sala.---Voy a llamar al Alcaide.

Durante el breve rato de espera Monsalud tuvo que resignarse a oír las
felicitaciones de D. Patricio Sarmiento que a la sazón entraba, y que
atronó la estancia con sus gritos y encarecimientos por el feliz suceso
de aquella iniciación. Todo su porvenir caballeresco comunero diera el
joven por sacudírselo de encima; pero al fin sacole de tan mal paso el
Alcaide apareciendo con Regato, y en seguida vendaron los ojos del
recluta, mandándole que marchase apoyado en el brazo del comunero
proponente.

---¿Quién es?---preguntó una voz.

---Un ciudadano---respondió Regato con toda la seriedad posible,---que
se ha presentado en las obras exteriores con bandera de parlamento a fin
de ser alistado.

La misma voz gritó:

---Echad el puente levadizo.

Oyó entonces el neófito un espantable ruido que en derredor suyo sonaba,
con tal estrépito que no parecía sino que todos los alcázares y torres
de España caían en ruinas; mas no se turbó por esto su esforzado
corazón, ni aun se le mudó la color del rostro, que para mayores trances
tenía coraje y alientos el bravo recluta. Además bien sabía él, como
todos, que aquel rumor provenía de una plancha de hierro semejante a las
que usan en los teatros para imitar los fragorosos ecos del trueno, y
que el ruido del hierro y cadenas era producido por una sarta de
cacharros que tras de la puerta agitaba bestial paleto simulando de este
modo con notoria perfección el acto de bajar el puente levadizo.

Quitáronle la venda; retiráronse Alcaide y proponente, y quedó solo con
el centinela, que estaba enmascarado. Estaba en el \emph{Cuerpo de
guardia}, y allí como en la \emph{Cámara de Meditaciones}, debía el
candidato reflexionar sobre su situación y contestar por escrito a
varias preguntas referentes a las obligaciones y derechos del comunero.
Monsalud observó el local de cuyas paredes pendían varias armaduras
mohosas y algunas espadas mojadas en sangre de cabrito, que para tan
terrorífico uso suministraba un día sí y otro no el conserje de la
Sociedad. Leyó los letreros conteniendo sentencias vulgares de la
religión de honor, y se dispuso a tomar asiento junto a la mesa donde
debía extender sus respuestas.

El centinela, que había permanecido tieso y grave, desempeñando su
imponente papel, soltó de repente la risa y dijo al neófito:

---¿También tenemos por aquí al Sr.~Monsalud?

Monsalud miraba a su interlocutor y no veía más que una máscara
horrible, una figura espantosa con casco empenachado de gallináceas
plumas y un babero a guisa de celada de encaje.

---¿Qué, no me conoce usted? Soy Pujitos---dijo el centinela quitándose
la máscara.

---Cómo te había de conocer, vecino, si parecías un valiente. ¿También
tú te diviertes con estas mojigangas?

---Vaya un modo de prepararse\ldots{} Llamar mojigangas a una cosa tan
seria, que va a derribar el Ministerio y a poner un Gobierno
republicano. Sr.~D. Salvador, ¿usted viene aquí a burlarse? Le aviso que
los que se han burlado de esto no lo han hecho dos veces. Con que
escriba el papelito y me volveré a poner la careta. Acabe usted pronto,
que me sofoco y este demonche de cartón huele muy mal.

---¿No te fatiga esta tarea? ¿No es mejor que descanses en tu casa toda
la noche después de haber trabajado todo el día?

---¡Quia!, si yo no hago más zapatos---dijo el gran patriota con
expresión de hombre perspicuo.---El Sr.~Regato me ha prometido darme un
destino en la Contaduría de Propios. D. Patricio me enseña a echar la
firma, que es lo que necesito, y salga el sol por Antequera.

---Ya sabía que eres de los que vocean en los motines, patean en
\emph{La Cruz de Malta} y apedrean el coche del Rey. ¿A cómo pagan esto?

Pujitos se puso serio al oír tamaña injuria.

---Vamos---dijo.---Está visto que usted viene aquí a mofarse. Pero
siempre seremos amigos, o mejor dicho, compañeros de armas. Escriba el
papelito y despache pronto. Me pongo la careta porque el Alcaide va a
venir.

---No hay prisa. Dime, Pujitos, ¿vienes aquí todas las noches?

---Todas, desde el primer día. Soy caballero fundador, y el día lo paso
en las cosas de la Milicia. Soy teniente, ¡uf!, ¡usted no sabe el
trabajo que da esto! A la parada, a pasar lista, a revisar los
uniformes, a hacer ejercicio de tiro, a aprender los reglamentos, a
echar unas copas con los oficiales para discutir lo que ha de hacerse el
día siguiente\ldots{} Y luego guardias y más guardias.

---¿Haces guardias de noche?

---Pues no. Anoche me tocó en el Principal, y mañana me toca en la
cárcel de la Corona.

---¡En la cárcel de la Corona\ldots{} mañana!---dijo Monsalud con
interés.---Ya sé\ldots{} es donde están presos esos cleriguillos que han
hecho planes horribles para quitar la libertad.

---Y algunos que no son clérigos. Pero esos tunantes morirán, o no hay
justicia en España. Dicen que el Gobierno quiere condenarles a presidio
nada más: esto se llama protección, ¿no es verdad?

---¿Y me has dicho que eres teniente?

---Nada menos; y si no fuera por las intrigas que hay en el
batallón\ldots{}

---Yo también seré miliciano y me afiliaré en tu batallón, gran
Pujos---dijo Monsalud riendo.---Se me figura que entre tú y yo hemos de
hacer algo extraordinario.

---Me alegraría de ello.

---Nos veremos pronto, y hablaremos\ldots{} quizás mañana\ldots{} Pero
el tiempo pasa y hay que contestar a estas endiabladas preguntas.

---Escriba usted\ldots{} Me parece que vienen ya.

Salvador escribió sus respuestas que fueron llevadas a la \emph{Plaza de
Armas} para que las examinara la guarnición. No tardaron el Alcaide y el
proponente en conducirle vendado otra vez a la puerta del salón de
sesiones, que estaba cerrada. Por dentro una voz gritó:---¿Quién es?

---Esta voz áspera y hueca como una campana rajada---dijo Monsalud para
sí,---es la de Romero Alpuente.

Entre tanto el Alcaide respondía:

---Soy el Alcaide de este castillo, que acompaño a un ciudadano que se
ha presentado a las avanzadas pidiendo parlamento.

---Por Dios, amigo Monsalud---indicó en voz baja Regato,---no se ría
usted; le suplico encarecidamente que sofoque toda manifestación de
burlas. Usted no quiere creerme y yo repito que esto es serio, pero muy
serio.

Abrieron la puerta de la \emph{Plaza de Armas}, que más parecía bodega
que plaza, con diversas series de asientos ocupados por los caballeros,
y un estradillo donde estaba el Presidente, teniendo detrás fementido
torreón de lienzo embadurnado, y un harapo que llamaban estandarte de
Padilla, y una urna donde \emph{se debían colocar todas las cenizas de
los comuneros que se pudieran haber}.

El Presidente le preguntó su nombre, edad, pueblo natal, empleo o
profesión; luego le habló de las obligaciones que contraía y del valor y
constancia que había de mostrar para desempeñarlas. Levantáronse en
seguida los caballeros, y Monsalud vio que todos ellos tenían una banda
morada en el pecho, y una como espada o asador en la mano.

---Ya estáis alistado---le dijo el Presidente.---Vuestra vida depende
del cumplimiento de las obligaciones que habéis contraído, y vais a
jurar. Acercaos y poned la mano sobre este escudo de nuestro jefe
Padilla, y con todo el ardor patrio de que seáis capaz, pronunciad
conmigo el juramento que debe quedar grabado en vuestro corazón.

Hecho lo que al neófito se le mandara, empezó este la retahíla del
juramento, que abrazaba diversos puntos, y que concluía con la consabida
conterilla que tanto ha hecho reír a la generación siguiente: «Juro que
si algún cab. com. faltase en todo o en parte a estos juramentos, le
mataré luego que la Confederación le declare traidor; y si faltase yo,
me declaro yo mismo traidor y merecedor de ser muerto con infamia por
disposición de la Confederación de cab. com., y para que ni memoria
quede de mí después de muerto, se me queme, y las cenizas se arrojen a
los vientos».

---Cubríos---le dijo el Presidente,---con el escudo de nuestro jefe
Padilla.

Tomó entonces el joven un mohoso broquel que le presentaron, y cubiertos
pecho y cara con tal defensa, pusieron en él todos los demás comuneros
la punta de sus espadas, mientras el Presidente dijo entre otras
majaderías:

---Si no lo cumplís, todas estas espadas no sólo os abandonarán, sino
que os quitarán el escudo para que quedéis al descubierto y os harán
pedazos en justa venganza de tan horrendo crimen.

Poseídos algunos caballeros, como gente candorosa, del papel que estaban
desempeñando, hincaban con excesiva fuerza la punta de sus asadores o
espadas en el escudo o sartén que resguardaba la cara y busto del joven.
El Sr.~Regato, temeroso de que por desmedido celo de los caballeros se
agujerease el escudo y perdiera un ojo su ahijado, creyó necesario
interrumpir por un momento la majestad del ceremonial, diciendo:

---Cuidado, señores, que es de hojalata\footnote{Todavia vive un
  comunero que corrió igual peligro.}.

La farándula no había terminado aún, porque tras la ceremonia del
escudo, el Alcaide calzó la espuela al caballero, dándole espada y
banda, con lo cual y con acompañarle a recorrer las filas para que fuera
dando la mano uno por uno a todos los confederados, el novel comunero
descansó a la postre de tantas fatigas.

\hypertarget{xix}{%
\chapter*{XIX}\label{xix}}
\addcontentsline{toc}{chapter}{XIX}

Salvador observó la diversidad de fisonomías que presentaba en su
innoble recinto la \emph{Plaza de Armas}, y halló entre sus compañeros
de caballería muchas caras conocidas. Había unos pocos que eran
diputados en el Congreso, y estaba también el célebre Mejía, que algunos
meses después fundó \emph{El Zurriago}. Aunque el elemento principal de
la Sociedad era la juventud, había bastantes viejos, no todos tan
inocentes como D. Patricio Sarmiento. Milicianos nacionales los había
por docenas; la gente de poca instrucción y de locos apetitos
burocráticos imperaba, y en todos los incidentes de la sesión salía a la
superficie un espumarajo de gárrula patriotería, que era la fermentación
de aquel elemento. No habrían trascurrido veinte minutos después de la
admisión del nuevo caballero comunero, cuando un hombre desenfrenado que
se ocupaba del asunto puesto a discusión, pronunció estas palabras:

---Yo propongo a nuestra Asamblea que cesen las contemplaciones con la
Corte y que se dé el grito de \emph{¡Viva la República!}

Alborotose la guarnición con tales palabras, que algunos calificaron de
admirable ocurrencia, otros de desatino mayúsculo, y si bien el
Presidente trató de volver la discusión al terreno que marcaba el tema,
no fue posible conseguirlo. Entonces el Sr.~Regato, manifestando
ruidosamente que deseaba decir algunas cosas estupendas que agradarían a
la reunión, usó de la palabra, en estos términos:

«Señores, lo que ha dicho nuestro ilustre y valerosísimo compañero de
armas, el caballero X\ldots, ha asombrado a muchos; pero a mí no me
asombra, porque yo soy más liberal hoy que ayer, y mañana más que hoy,
porque mi lema, señores, es adelante y siempre adelante. Estamos
cansados de sufrir, estamos cansados de esperar. ¿Os aterra la palabra
\emph{república}? Pues yo digo que a mí no me ha aterrado nunca esa
palabra, ni me aterra hoy. Perdamos el miedo y seremos fuertes.
Amenacemos y nos temerán. Somos los más, somos lo más granado de la
España liberal. La Europa nos contempla, el Piamonte nos imita, Nápoles
nos copia, Portugal se llama nuestros \emph{discípulo}. Señores, seamos
dignos de la Europa liberal, y ante nosotros temblarán el Trono y los
masones».

Después de dar las gracias por los aplausos y de limpiarse el sudor, el
orador prosiguió así:

«No creáis que la idea republicana es nueva en España. Padilla y Lanuza,
nuestros maestros, fueron republicanos. Viniendo a los tiempos modernos,
en la proclamación de los derechos del hombre hecha por Muñoz Torrero en
las Cortes del año 10, veo yo también la idea republicana. Leed las
obras de Marina y de Sempere, y veréis que en ellas palpita la
república. \emph{(Gran estupor.)} Ahora, señores, volved los ojos a
todos los ámbitos de la hispana península \emph{(el orador, excitado por
la admiración general, se cree en el caso de tener estilo)}, volved los
ojos por doquiera, ¿qué veis? \emph{(Gran silencio; indicio cierto de
que nadie veía nada)}. Pues veréis allá en las Andalucías, allá en la
populosa ciudad de Málaga, bañada por las ondas del Mediterráneo, a
Lucas Francisco Mendialdúa que concibió el plan de establecer la
República, como consta en la proclama que imprimió, encabezada con las
mágicas palabras \emph{República Española} y firmada por \emph{Un
tribunal del pueblo}. Como acontece a los grandes genios innovadores,
como aconteció a Colón, Galileo, Savonarola, etc., etc\ldots. Mendialdúa
fue preso\footnote{En Enero del 24.}. Pero así como de la noche sale el
claro día, de las cárceles sale la libertad. \emph{(Atronadores
aplausos.)}

»Volved ahora los ojos al llamado reino de Aragón y veréis allí a
nuestro insigne jefe, al valiente entre los valientes, al político entre
los políticos, al altísimo Riego, que desempeña el cargo de capitán
general en aquella extensa y rica provincia. ¿Creéis que no hace nada?
Indigno sería esto de su perspicua mirada, que cual la mirada del águila
penetra en lo más alto del cielo. No creáis que nuestro jefe está mano
sobre mano, no; nuestro jefe trabaja por la República. \emph{(Asombro
general e innumerables bocas abiertas.)} En Zaragoza están a la sazón
algunos beneméritos patriotas franceses, cuyos nombres no
pronunciaré\footnote{Llamábanse Uxón y Cugnet de Montarlot.}. Esos
patriotas, pertenecientes a la gran Confederación francesa, están de
acuerdo con nuestro jefe, no lo dudéis, están de acuerdo. Unidos todos,
discurren cuál será el mejor medio de ponernos la República en
España\ldots{} ¡Guay de nosotros si no les ayudamos!\ldots{} ¡guay de
nosotros si nos dormimos mientras ellos velan!\ldots{} ¡guay,
guay!\ldots{} Lo que puedo aseguraros es que si no nos ven dispuestos y
valientes, irán con su proyectillo a Francia. Aquel país no se anda con
chiquitas ni repara en niñerías. Estad seguros de que si nuestro jefe se
presenta en el Pirineo enarbolando la bandera tricolor y gritando,
\emph{¡viva la República!}, todo el ejército francés se le unirá en
seguida, y llegará a París en triunfal paseo, como Napoleón cuando
volvió de la isla de Elba. \emph{(Los comuneros acogen esta bola con
grande algazara, señal cierta de que se la han tragado.)}

»Ahora volved los ojos a Galicia, donde está el general Mina; volvedlos
luego a Barcelona, donde está el gran patriota Jorge Bessières y veréis
que estos campeones de la libertad tampoco están mano sobre mano.
¿Seremos menos aquí? ¿Nos espantaremos de la libertad? No, señores.
Adelante, siempre adelante. ¡Viva la libertad! Yo, el más humilde de
esta Asamblea; yo, que he venido aquí porque me repugnaban los infames
manejos de los de allá; yo, que estoy pronto a derramar hasta la última
gota de mi sangre, hasta la última, señores, por el triunfo de la causa;
yo, que jamás recibí destino de los tibios ni lo solicité; yo, que soy
hombre puro, si hay hombres puros en España, os propongo con el corazón
henchido de patriotismo que aceptéis desde luego la idea republicana,
como ha propuesto mi esclarecido amigo el ciudadano X\ldots»

Varios oradores pidieron la palabra. Después de una breve disputa sobre
quién había de usarla, D. Patricio Sarmiento se levantó y habló de este
modo:

---Después del elocuentísimo discurso del fénix de los ingenios
comuneros, D. José Manuel Regato, ¿qué puedo decir yo, que soy un triste
maestro de escuela, un oscuro preceptor de la tierna juventud? Pero si
de algo sirven los consejos de un viejo que se ha quemado las cejas
estudiando la historia del pueblo romano, quiero alzar esta noche mi
humilde voz en este augusto recinto para enseñaros lo que no sabéis.
Vuelvo los ojos en torno mío y veo zapateros, sastres, talabarteros,
comerciantes, taberneros, colchoneros y otros artífices, gente toda muy
honrada, muy patriota, muy digna, pero que no está versada en la
historia romana. \emph{(Rumores de disgusto.)} No trato de ofender a
nadie: afirmo un hecho y nada más; y como yo creo que para tratar
ciertos asuntos es necesario haberse quemado las cejas\ldots{}
\emph{(Interrupciones donosas)}, haberse quemado las cejas, como me las
he quemado yo, de aquí infiero\ldots{} Esas interrupciones y cuchicheos
no hacen mella en mi ruda entereza, no señor; \emph{(El orador se
amostaza)} y así digo como el gran Temístocles: «pega, pero escucha».
¿De qué se trata? De adoptar la idea republicana. Bien; yo pregunto a la
docta Asamblea: ¿Cuándo se estableció la República en Roma? Y la docta
Asamblea me contestará que el año 509 antes de Jesucristo. Muy bien
contestado. ¿Y cuándo concluyó la República de Roma? El año 29. Total de
tiempo en que existió la forma republicana: 480 años. Está muy bien.
\emph{(Más fuertes rumores.)} Ahora pregunto: ¿cuáles fueron las causas
que determinaron a los romanos a cambiar de forma de Gobierno?

Los rumores se trocaban en tumulto, y una voz gritó:

---¡Que se calle ese pedante!

---¡Que se vaya a la escuela!

---Al indocto grosero que de este modo me interrumpe---gritó D. Patricio
agitando los brazos y poniéndose muy encendido,---le contestaré que él
es quien debe ir a la escuela a aprender lo que ignora.

---¡Aquí no se quieren estafermos!---aulló una voz, de la cual no se
tendrá idea sino considerando de qué modo puede hablar el aguardiente.

---Señores---dijo el Presidente con aquel formulismo parlamentario que
algunos hombres quieren llevar a donde quiera que se oiga el sonsonete
de un discurso,---no demos a España y a Europa el triste espectáculo de
una discordia entre individuos de esta nobilísima Asamblea. No se diga
que andamos a la greña como los masones, a quienes yo aplico aquello de
\emph{riñen los pastores y se cubren los hurtos. (Prolongadas risas.)}

---¡Que se calle D. Patricio!

---¡Que se calle Pelumbres!

---Pues a mí no me da la gana de callarme\ldots{} a ver---exclamó una
voz que salía del formidable pecho de un hombre tiznado, fiero,
corpulento, que parecía personificación de una fragua.---Y si a mí no me
da la gana de callarme, a ver quién es el guapo que me cierra el
pico\ldots{} ¡a ver!

Diciendo esto, se levantaba el Sr.~Pelumbres entre la multitud apiñada
en los bancos. Su figura, así como su voz, pondrían miedo en toda
Asamblea que no fuera la de los Comuneros.

---Ciudadano Pelumbres---dijo el Presidente,---¿qué dirá la Europa si no
guardamos la compostura propia de hombres de Gobierno?\ldots{} ¿qué
dirá?

---Eso es, ¿qué dirá?---repitieron D. Patricio y los que deseaban que
hablase.

---Es preciso tener moderación---continuó el Presidente.---Puesto que el
ciudadano Sarmiento estaba en el uso de la palabra, continúe su erudito
discurso, que tiempo tiene de hablar el ciudadano Pelumbres. Yo le
concederé la palabra, esperando en tanto de su finura y buen sentido que
no interrumpa al orador en este importantísimo debate.

Ya entonces empezaba a ser costumbre el llamar \emph{importantísimo
debate} a cualquier inútil disputa suscitada por la envidia o la
vanidad.

---Señor Presidente---gruñó Pelumbres, tambaleándose como un yunque sin
equilibrio,---lo que digo es que el ciudadano Sarmiento es un
animal\ldots{} y a mí no me soba nadie.

Cayó en el asiento como quien se echa a dormir.

---Señor Presidente---dijo con trémula voz Sarmiento.---La Asamblea
conoce bien mi carácter y mis servicios\ldots{} no necesito responder a
los \emph{cargos} que me ha dirigido el ciudadano Pelumbres, porque la
Asamblea sabe muy bien que yo\ldots{}

---Sí, sí---gruñó la Asamblea.

Estaba el buen Sarmiento en pie, con el cuerpo doblado por la cintura,
recogiéndose a un lado y otro los faldones de la levita, como quien se
va a sentar y no se sienta.

---Agradezco las manifestaciones de simpatía de este ilustre
Areópago---dijo el orador,---y me parece que no debo molestar más al
ilustre Areópago, y que los injustos cargos que el ciudadano Pelumbres
me ha dirigido, no deben contestarse sino con un magnánimo silencio.

---Bien, muy bien.

---Por lo cual me siento, dejando a nuestro esclarecido Presidente la
alta honra de continuar este \emph{importantísimo debate}, para que nos
diga su opinión, que es lo que más nos importa.

Rumores diversos manifestaban el deseo de que hablase el Castellano.
Romero Alpuente se dispuso a hacer el gusto a sus presididos. Antes de
atender a su discurso, convendrá decir que el célebre demagogo de los
tres años no era un jovenzuelo fogoso, como algunos creen, sino un
vejete atrabiliario y furibundo, alto, flaco, descuadernado, anguloso,
de gárrula elocuencia, de vulgares modos. Era tanta su fealdad, debida
en primer término a la longitud de sus narices, que no es fácil se
encontrara entonces ni se haya encontrado después su pareja. Alcalá
Galiano, al lado suyo, se tenía por un Adonis.

Había sido magistrado de la Audiencia de Madrid, y en su vida privada
era el hombre más inofensivo, más manso y para poco que imaginarse
puede. El mismo que en público encarecía la necesidad de cortar no sé
cuántos miles de cabezas, era incapaz de matar un mosquito. ¡Pobre
carnero viejo que, habiendo leído algo de Robespierre y de Marat, quería
parecerse a ellos! Pero sólo los tontos confundían su clueco balido con
el rugir de leones y panteras. Sus discursos, que alborotaban las Cortes
y los clubs, eran un conjunto de garrulidades terroríficas, de
chascarrillos y vulgares idiotismos. Carecía de formas literarias, y su
lenguaje familiar era a veces tan divertido como sus amenazas
demagógicas, que aquella bendita generación no tomaba siempre en serio.
Algunos le llamaban el \emph{Guzmán} (el gracioso) de las Cortes. Tuvo
además el pobre \emph{D. Juan Romero Alpuente} la desgracia de que en lo
mejor de sus triunfos parlamentarios le saliera un enemigo folletinista,
que usando el nombre de \emph{D. Pedro Tomillo Al-vado}, le puso de hoja
de perejil.

«Caballeros comuneros---dijo Alpuente con voz que no tenía nada de
temerosa,---o hay confianza en los hombres del partido, o no hay
confianza en los hombres del partido. Si hay confianza en los hombres
del partido, no se planteen cuestiones prematuras. Si algo debe hacerse
se hará. No conviene precipitarse, no conviene comprometerse. Las cosas
vendrán por sus propios pasos. El partido es el partido, y el que no
crea que el partido es como debe ser, espere a ver en qué para el
partido y se convencerá. \emph{(Rumores. Asentimiento general.)}

»Por consiguiente---prosiguió, satisfecho del éxito de su
exordio,---esperemos llenos de patriotismo, y no hablemos por ahora de
republicanismo. El partido es un partido que debe estar preparado para
empuñar el timón de la nave del Estado si se le llama con este fin.
\emph{(Muestras de regocijo.)} Y se le llamará, ciudadanos caballeros,
¿pues quién lo duda? El segundo Gobierno constitucional sigue la misma
desatentada senda que el primero. El país está lo mismo hoy que ayer. El
pueblo soporta las mismas cadenas; los tiranos no han cambiado, los
mandarines siguen, los peligros crecen. El Gobierno cree que va a durar
mucho, ¿pues no lo ha de creer? Pero yo quiero ver cómo se las compone
con las tramas de la Junta Apostólica en Galicia, con los guardias
destituidos, con los obispos rebeldes, con la conspiración de Vinuesa,
con la del Abuelo, con los tumultos de Zamora, con el motín de Alcoy,
donde han sido destrozadas todas las máquinas, con el robo de la valija
de Aragón, con los sucesos de Valladolid\ldots{} Me parece que les cayó
que hacer, ¿eh? \emph{(Risas.)} Yo pregunto, ¿cuál es el medio de que se
acaben los trastornos? Establecer la libertad en toda su integridad.
Esto es axiomático. Que los absolutistas vean una mano terrible
dispuesta a caerles encima en cuanto chisten, y entonces se meterán bajo
una silla. Y no me hablen a mí de conspiraciones demagógicas y
republicanas. Aquí no hay nada de eso, y si lo hay es amaño de los
constitucionales masones para desacreditar a nuestro partido. Ellos
tienen el lema de \emph{dar palos y gritar «que nos pegan,»}, lo cual ya
no hace efecto porque se va descubriendo la picardía. \emph{(Carcajadas
y bravos.)}

»Seamos prudentes, seamos cuerdos. Sigamos defendiendo nuestros
sacrosantos principios\ldots{} Hoy más libertad que ayer y mañana más
que hoy\ldots{} No nos arredremos, no volvamos la cara atrás. Adelante,
siempre adelante. Pero vayamos con pie seguro. A su tiempo se enseñarán
los dientes. Pues qué, ¿creen que si logramos empuñar el timón de la
nave del Estado (esta figura de la \emph{nave} era la única que se había
asimilado en su carrera parlamentaria el orador comunero), vamos a
estarnos mano sobre mano, sin hacer nada, como el Gobierno de la
\emph{coletilla}? Y ahora viene el repetir lo que ya se dijo en 1811:

\small
\newlength\mlena
\settowidth\mlena{¡Ser gobernados los buenos}
\begin{center}
\parbox{\mlena}{¡Mirad qué gobernación!                              \\
                ¡Ser gobernados los buenos                           \\
                por los que tales no son!}                           \\
\end{center}
\normalsize

»No, señores, es preciso que no se pueda decir de nosotros lo que de
estos mandarines chinos. No seguirá el tole tole de oprimir al patriota
y ensalzar al que no lo es. Se encomendarán los destinos de la Nación a
los comprometidos por el sistema, no a los que no lo están. Se harán
castigos ejemplares, se volverá todo del revés para que los pillos bajen
y los patriotas suban. \emph{(Muy bien.)} No se dará el caso de que de
los veinte millones de españoles, suden y trabajen los diez y ocho y
apenas puedan llevar a la boca un pedazo de pan moreno, para que los
otros dos millones se abaniquen y vivan rodeados de placeres. Entonces
se permitirá que eso que llaman los infames \emph{populacho} se reúna
donde le dé la gana y grite y diga todos los defectos del Ministerio. La
suspirada libertad será un hecho y no llevarán \emph{albarda} más que
los que quieran llevarla\footnote{Casi todos los párrafos de este
  discurso son auténticos.}. \emph{(Grandes aplausos).}

»En suma, señores, el partido declara por mi conducto que no quiere ser
vasallo; que planteará el sistema en toda su pureza. Si para esto es
preciso la violencia, venga la violencia. Si es preciso la guerra civil,
venga la guerra. La Providencia salvará al partido. No olvidéis,
señores, que el \emph{Criador del Universo bendijo también los esfuerzos
que hicieron Matatías y sus hijos para evadirse de la justa dominación
del impío Antíoco Epifáneo}. Entre tanto, desechemos la idea de
República. La Constitución establece la Monarquía y nosotros respetamos
al Rey constitucional. No se diga que el partido ha sido el primero en
alterar la augusta ley. Dejémosles que ellos se caigan solos; y si nos
hicieren ascos y no quisieren nuestra ayuda para mantenerse derechos,
¿me entiende usted?, si prefieren apoyarse en la Santa Alianza y en sus
diplomáticos, enviados, farsantes, zascandiles, espías y soplones, en
los que fueron pajes de escoba del Rey Pepillo, en los serviles
españoles de todas clases y ropajes, con bandas, cruces y calvarios, en
los de mitra, bonete e hisopo; en los seráficos, angélicos, en los
tostadores y sus familiares, plumistas, guardas, alfileres, corchetes y
agarrantes, en los que dicen \emph{el Rey mi amo}\ldots{} entonces nos
retiramos, dejándoles que vayan a donde quieran, pues como dicen en mi
tierra, \emph{cuanto más se desvía el borrego mayor topetazo pega».}

Atronadoras exclamaciones de entusiasmo acogieron la frase final del
discurso de Romero Alpuente, orador que, como se ha visto, no ha dejado
de tener herederos en la política española.

Una voz que parecía cien voces, gritó:

---¡Viva Riego!

Contestó un alarido, y desde entonces el \emph{importantísimo debate} se
convirtió en un importantísimo aquelarre. Romero Alpuente se fue, y en
su lugar el Sr. Regato se dispuso a presidir (no hay otro verbo que
pueda emplearse propiamente) el resto de lo que no hay más remedio que
llamar sesión.

Un orador pidió que se hiciesen manifestaciones contra la Santa Alianza
en la persona de sus plenipotenciarios, idea que fue acogida con
satisfactorio y general asentimiento por la Asamblea, y procediose al
nombramiento de una comisión que se encargase de ajustar las cuentas a
los cristales de las casas donde vivían los embajadores de Austria y
Rusia. No se había calmado la efervescencia causada por este suceso
cuando un joven de buen porte tan correcto de traje como de estilo y
hasta afeminado, pronunció un discurso de energúmeno sobre el plan de
Vinuesa y el escarmiento que debía hacerse en la persona de aquel
malvado \emph{aborto del Infierno, compendio de todos los crímenes}.

Aseguró también que Vinuesa estaba conspirando dentro de la cárcel, y
que si no se ponía remedio en ello, imaginaría un nuevo plan absolutista
para matar la libertad. Acusó al infante D. Carlos de complicidad con el
cura de Tamajón, y afirmó que todo porrazo dado a Vinuesa sería porrazo
dado a la Corte. Aumentando en fogosidad a cada instante, llegó a
sostener que el Gobierno se estaba \emph{portando traidoramente} en este
negocio, y que a él (al orador) le constaba que había intenciones de
absolver al de Tamajón y aun darle una mitra, si era menester. Aseguró
que el pueblo no debía consentir tal iniquidad, porque si la consentía
no era digno de la fama que había adquirido en Portugal, Nápoles y el
Piamonte, países que nos habían tomado por modelo, estableciendo la
libertad al mágico grito de \emph{«¡vivan los discípulos de España!»}

Al discurso del joven contestó otro joven de muy distinta figura,
educación y modales, (pues en aquella asamblea había locos de todas
clases) diciendo que la culpa de todo la tenían los masones, que dando a
la Nación el nombre de populacho y haciendo el bu con la anarquía,
estaban poniendo las cosas como en los tiempos ominosos. Hizo reír al
auditorio, afirmando que bien pronto se prohibiría \emph{con pena de
pecado mortal} pronunciar el nombre de Riego; pero que él (el orador)
estaba resuelto a exhalar el último suspiro diciendo \emph{¡Viva Riego!}
en atención a que Riego \emph{había enjugado el llanto del pueblo
español}. Esta figura, tan original como patética, produjo gran
entusiasmo, con el cual, excitándose el espíritu del orador, dijo que él
sabía el modo de resolver el asunto de Vinuesa; que el pueblo, como
soberano que era, podía hacer su real gana, porque el Gobierno recibía
dinero de la Santa Alianza para ir arreglando la cama al despotismo, y
esto no se debía consentir.

Mezclando berzas con capachos, aseguró que él había entrado en la
prisión de Vinuesa y le había visto escribiendo planes y más planes; que
corría mucho dinero absolutista para sacarle de la prisión y ponerle al
frente de un Gobierno despótico, y que el orador y Pelumbres, al salir
una mañana de la taberna, habían oído una conversación sospechosa entre
dos clérigos, de la cual dedujeron que Vinuesa se comunicaba
constantemente con sus cómplices. Concluyó diciendo que él (el orador)
no se pararía en barras, y que si los conspiradores vieran media docena
de cabezas clavadas en otras tantas pértigas junto a la Mariblanca de la
Puerta del Sol, doblarían la \emph{cerviz} (única palabra pedantesca que
se permitió el orador en su largo discurso) ante el pueblo
\emph{re-soberano}.

Después de este joven plebeyo, otro joven decente habló de los que
\emph{clavaban constantemente el puñal en las entrañas de la madre
patria}, y anunció su resolución de ocupar el primer puesto el día del
peligro, sacrificando su existencia al triunfo de la libertad. Puso cual
no digan dueñas a los masones, acusándoles de afrancesados e impostores,
pues muchos, dijo, profanaban el nombre de Riego, tomándole en sus
\emph{asquerosas bocas}, siendo así que para pronunciar palabra tan
angélica \emph{debían enjuagarse un mes antes con miel rosada}. Afirmó
que Calatrava era un bajo adulador, Feliu un traidor, Martínez de la
Rosa un mandria, Cano Manuel un bobo, Toreno un pedante, Argüelles un
embustero. Después de mucho divagar, propuso a la Asamblea que se diese
un voto de gracias a D. José Manuel Regato por lo bien que había
conducido todos los asuntos de la Comunería desde su origen. Regato
estuvo a punto de llorar de emoción, y para demostrar de un modo
incompleto su agradecimiento, convidó a cenar a varios de los más
granaditos. La sesión terminó alegremente entre las alegres endechas del
himno, que sonaban bajo las bóvedas de la fortaleza:

\small
\newlength\mlenb
\settowidth\mlenb{hasta el borde del hondo sepulcro}
\begin{center}
\parbox{\mlenb}{Es en vano calumnie la envidia                       \\
                al caudillo que adora el ibero;                      \\
                hasta el borde del hondo sepulcro                    \\
                nuestro grito será: ¡viva Riego!}                    \\
\end{center}
\normalsize

El lector no será español si no recuerda al punto la música.

\hypertarget{xx}{%
\chapter*{XX}\label{xx}}
\addcontentsline{toc}{chapter}{XX}

En lo restante de la noche oíase por aquellos barrios el aullido de la
Orden de Padilla, suelta por las calles. El himno, el \emph{lairón},
cántico que por aquellos días había sustituido al feroz \emph{trágala},
sonaba de calle en calle, como el ronquido de vinoso trasnochador.
Íbanse perdiendo en el silencio de la noche, a medida que los grupos
desaparecían, entrando en las tabernas, botillerías y cafés patrióticos.
En uno de estos se vio que a deshora penetraba el Sr. Regato, acompañado
de Pelumbres, Pujitos, dos de los jóvenes que pronunciaron discursos
aquella noche, Salvador Monsalud y otros. Cenaron alegremente, sin dejar
de la boca los negocios políticos, y sus proyectos eran atrevidos y
grandiosos como las concepciones del genio. El Sr.~Regato, no sólo pagó
todo el gasto, sino que ofreció dinero a los más necesitados, los cuales
no tuvieron escrúpulo en tomarlo patrióticamente, por aquello de que
tripas llevan pies, que no pies tripas.

Si Salvador Monsalud no se separara antes de tiempo de tan escogida
sociedad, pretextando una enfermedad que no tenía, hubiera visto que el
Sr.~Regato, hombre opulentísimo, aunque nadie le conocía rentas, ni
sueldo, ni industria, recompensó largamente a todos, dándoles lo
necesario para la existencia y sostén de sus respectivas familias.
Cuando esto pasaba, habíanse retirado también los dos oradores con el
gran Pujitos, y sólo quedaban en compañía del generoso comunero
Pelumbres el herrero, D. Bruno, \emph{Chaleco}, y otros padres de la
patria, de cuyas hazañas no puede tenerse idea sino presenciándolas,
como las presenciará el lector en lo restante de este libro.

Salvador Monsalud fue a su casa cerca del día. Su cabeza era un volcán.
Los discursos que había oído, las caras de los oradores, la fisonomía
astuta de Regato, la candidez estúpida de otros, el ramplón jacobinismo
de Romero Alpuente, hervían dentro de ella. Trató de dormir, pero la
Asamblea sin apartarse de sus excitados sentidos, continuaba zumbando y
gesticulando con sus cien voces roncas y sus doscientas manos
amenazadoras. Al punto comprendió que era producto infame de candidez y
de perversidad, la gárrula bastardía del entendimiento, explotada por
una diplomacia satánica. Comprendió que se había metido entre hombres,
la mitad tontos, la mitad feroces, pero que marchaban juntos a un fin
claro, con alianza parecida a la del asno y el lobo en más de una
fábula. Del esfuerzo que necesitaba hacer su espíritu para descender al
trato con tales gentes no hay que hablar, porque se comprenderá
fácilmente.

Había avanzado la mañana, sin que el novel hijo de Padilla hubiera
podido conciliar el sueño, cuando entró Campos lleno de zozobra y
agitación.

---Esto ya pasa de broma---le dijo.---La niña no parece. Hemos estado en
el Retiro, y no está en el sitio que me indicaste. Valiente bromazo nos
está dando la tonta\ldots{} ¡Por los clavos de Cristo!, si no diera la
casualidad de que Falfán de los Godos está fuera de Madrid, no sé cómo
podríamos ocultarle que su novia se ha escapado de mi casa anteayer, y a
estas horas no sabemos dónde está.

---En la carta que enseñé a usted me decía que no volvería a su casa.

---Temo cualquier necedad\ldots{} Salvador, estoy muy inquieto---dijo
Campos perdiendo aquella serenidad que indicaba en él un gran contento
de la vida.---Sin duda esa loca está vagando por Madrid, y te busca de
casa en casa, de café en café, como una perdida. ¡Qué deshonra!

---Creo lo mismo. Pero esto tiene que concluir.

---¿Estuvo ayer aquí?

---Dos o tres veces. Como no me ha encontrado en ninguna parte presumo
que volverá. Si vuelve, Sr.~Campos, ofrezco remitírsela a usted sin
pérdida de tiempo.

---Es que debes hacerlo---dijo Cicerón con energía.---Es que si no lo
haces, faltas a la solemne palabra que me diste, y entonces, amiguito,
no hay nada de lo dicho. Ya tengo en mi casa tu nombramiento para la
cárcel de la Corona; pero como yo no recoja hoy mismo esa oveja
descarriada, creeré que me estás engañando, creeré que estás de acuerdo
con ella, que la escondes en alguna parte, y\ldots{}

El plácido semblante de Campos se enrojeció todo por la congestión que
determinaba la ira.

---Mi determinación es irrevocable---contestó el
joven.---Supongo\ldots{} casi estoy seguro de que volverá hoy. Avisaré a
Lucas para que la deje subir.

---¿Convendrá traer acá dos individuos de la policía y un coche, que
debe esperar en la calle de Bordadores? Conozco a Andrea y sé que no
cederá por buenas.

---Nada de eso me corresponde a mí. Usted puede emplear los medios que
quiera para llevársela. Yo no tengo que hacer sino poner fin a sus
correrías y convencerla de que por más que me busque, no me encontrará
en ninguna parte.

---Te comprendo---dijo Campos con viveza y señales de contento.---Tomaré
mis medidas. No me moveré en todo el día de la tienda de Requejo, y
Sarmiento y yo nos pondremos de acuerdo para que si la oveja viene a
este aprisco no se nos escape.

Después de este diálogo, que se prolongó un poco más, aunque sin ofrecer
en el resto de él nada digno de contarse, Campos se retiró. Monsalud,
contra su costumbre, hizo propósito de permanecer en su casa todo el
día. Sin hacer nada en ella, tenía la agitación y la movilidad exaltada
de quien trae entre manos una ocupación grave. Iba y venía de una pieza
a otra; hacía a su madre y a su hermana preguntas que ninguna de ellas
entendía; se asomaba al balcón; hacía subir a D. Patricio para darle
órdenes; censuraba a veces que la casa no estuviese mejor dispuesta, y
reprendía luego a las dos mujeres porque se agitaban para arreglar las
habitaciones.

Cerca del medio día se retiró a su cuarto. Solita entró en él. Llevaba
un pañuelo atado alrededor de la cabeza para resguardarse del sutil
polvo que zorros y escobas levantaban, y cubría su cuerpo con una falda
bastante antigua, pieza de desecho cuyas funciones se concretaban a los
días de limpieza. La figura de la joven no era con tal atavío un modelo
de elegancia.

---Hermana, estás que no se te puede mirar---dijo Salvador observándola
con cierta pena.---Es preciso que te pongas guapa.

---¿Yo?\ldots{} ¿Cuándo?---repuso la joven con la mayor turbación.---¿Y
a qué vienen ahora esas guapezas?

---Me gustaría verte hoy arregladita y linda, como tú sabes ponerte
cuando quieres. No es esto decir que me disguste verte así. Acá entre
los dos, siempre estás bien; pero\ldots{}

---¿Vamos a algún baile?---preguntó Sola con malicia.

---No vamos a ningún baile---dijo Salvador con la torpeza que acompaña a
las ideas de difícil explicación;---pero quisiera verte hoy como
realmente eres; quisiera que cuantos entraran aquí te admirasen y
reconocieran en ti\ldots{}

---Tú te burlas de mí---dijo Solita llena de rubor.---Yo siempre estaré
mal.

---¡Oh!, te equivocas---manifestó Salvador con un tono que antes era de
benevolencia que de convicción.---Vamos, también querrás sostener que no
eres guapa. Más de cuatro quisieran\ldots{}

---No sé por qué me dices esas tonterías.

---Mira, hermana, te agradeceré que te pongas tu mejor vestido, que te
arregles bien; pero muy bien.

---Ya sabes que estando mi padre en la cárcel no puedo ir a paseo ni al
teatro.

---Si no pretendo llevarte a ninguna parte---dijo Salvador con
impaciencia.---En fin, ¿te compones o no?

---Me compondré.

---Hazme ese gusto, hermana. Así no estás bien, y tú vales mucho. Yo
quiero que se vea que tengo una hermana simpática, bonita\ldots{} ¿me
entiendes?

---Como si hablaras en griego.

---Pues vístete: ponte tu mejor vestido, ya sabes. Figúrate por un
momento que soy tu novio. Vaya, ¿no tendrías interés en agradar a tu
novio; no tendrías interés en que él te encontrara siempre linda?

---Si dijera que no, sería una melindrosa---respondió Soledad fingiendo
que ponía en orden las sillas para que, vuelto el rostro, no se le
conociera la emoción que experimentaba.---Pero como no eres mi novio ni
lo serás\ldots{}

---¿Te vistes, sí o no?

---Al momento, hombre, al momento.

Voló fuera del cuarto. Algún tiempo después regresaba vestida y ataviada
con lo mejor que tenía.

---¡Oh!, ¡qué bien!---dijo Monsalud con sincera admiración.---Hermosa
prenda se va a llevar ese bruto de Anatolio. Hermanita, estás
preciosísima: te lo digo sinceramente.

El rostro de Soledad se encendió más, y viose en aquel puro cielo de
modestia una chispa de vanidad que lo iluminó momentáneamente. Salvador
no mentía, porque de muy distintas maneras está preciosa una mujer. En
las incorrectas facciones de la hija del absolutista, en su descolorido
semblante que a intervalos se inflamaba, en sus ojos donde jugueteaba el
alma escondiéndose en la penumbra del pudor o mostrándose en la claridad
del cariño, había lo bastante para turbar la paz de cualquiera.

---Siéntate a mi lado---le dijo Salvador;---parece que estás asustada.

---¿Yo?\ldots{} no.

---Dame acá esa mano. Tienes las manos más bonitas que he visto. ¿Por
qué las tienes tan frías y temblorosas?

---Es que las tuyas echan fuego y cuanto tocan lo encuentran helado.

---Ahora te has puesto como el papel\ldots{} ¡qué palidez! Pues
mira\ldots{} así descoloridita es como estás mejor. En tu cara se ve tu
alma bondadosa. Me consuela mucho verte a mi lado. Necesita uno personas
así, que le compadezcan mucho, que le tengan lástima, que le mimen.

---Y por qué te he de compadecer, si tienes todo lo que deseas, si estás
como nadie. Yo sí que soy digna de lástima.

---Pero tú tendrás a tu padre, y yo jamás, jamás recobraré lo que he
perdido.

Ambos callaron, inclinando cada cual su cabeza cargada de pesos enormes.

---Me parece que siento ruido---dijo Solita vivamente.---Bueno será
prevenir a Rosa, para que si llega esa mujer que ayer estuvo tres veces
y que tanto te molesta, no la deje entrar.

---No; ya he advertido a Rosa que la deje pasar---dijo Salvador con
turbación.---Quizás no venga más.

El ruido cesó y la casa continuaba en silencio.

---Me alegro de que mi madre haya salido hoy---indicó Salvador.

---Me parece que está ahí---repuso Solita poniendo atención.---Siento
pasos en la escalera.

---No; no es mi madre---indicó Monsalud con ansiedad vivísima.

---Los pasos son precipitados\ldots{} Se oye una voz de mujer\ldots{}
¿Voy a ver?

---No; estate aquí, y no te muevas de mi lado.

Callaron los dos. Solita miró a su hermano como asombrada. Salvador
clavaba sus ojos en la puerta, donde no había nada todavía; pero de
antemano su alma llena de ansiedad, observaba lo que había de venir.

Andrea apareció en la puerta. Estaba desfigurada por enfermiza palidez;
sus ojos miraban todo con febril extravío, y el desmelenado cabello así
como el vestido en desorden indicaban largas horas de insomnio, de lucha
y de amargura.

Su primer movimiento fue un impulso poderoso hacia el hombre que buscaba
y que había encontrado. Viose en su semblante la contracción que
acompaña a un repentino desbordamiento de lágrimas. Pero dio tres pasos,
y viendo que no estaba solo, se detuvo. ¡Qué choque de ideas en aquella
cabeza! El impulso, el tierno avance expansivo, habían encontrado un
obstáculo, un muro frío, y contra este la exaltada mujer se estrellaba
palpitando y llena de congoja. Sus ojos atónitos, enrojecidos por el
llanto, preguntaban sin pestañear: «¿qué chiquilla es esta?»

Salvador se levantó. Estaba lívido.

---Tengo que hablarte---balbució Andrea, viendo que daba un paso hacia
ella.

Después dirigió a Soledad miradas recelosas e impacientes, como
diciendo: «¿qué hace aquí esta mujer extraña? Que se vaya».

---Es un error---dijo Salvador.---Usted no tiene nada que decirme, y se
ha equivocado, sin duda. Yo no sé quién es usted.

---¿No sabes quién soy?\ldots{} Yo te lo diré---exclamó Andrea, cruzando
las manos.---¡Que se marche esa mujer!

Con imperioso gesto señaló la puerta.

Soledad, tan aterrada como curiosa, pero sumisa siempre, se levantó.
Salvador le dijo severamente:

---Quédate.

---¡Con que es decir!\ldots---gritó Andrea con espantosa alteración de
voz y semblante.

---Que usted es quien no está en su sitio aquí y debe
retirarse---respondió el joven.---Sin duda ha padecido una equivocación.

---¡Perverso!\ldots{} ¿dices eso de veras?

Andrea, al decir estas palabras, que salían de su pecho como bramidos,
adelantó con los brazos abiertos hacia su amante. Los brazos tropezaron
con dos manos de acero que los retorcieron, rechazando el hermoso cuerpo
a que pertenecían.

---¡Oh, qué vil soy!\ldots---gritó la indiana cayendo al suelo de
rodillas.---¡Rebajarme así!\ldots{}

---¡Rebajarse así una marquesa!\ldots---murmuró Salvador con sorda
voz.---Señora, sentiré mucho que se ponga usted mala. ¿Quiere usted que
se mande traer un coche para llevarla a su casa?

Andrea se levantó de un salto. La mirada que arrojó a su amante, como
una saeta furibunda, turbó tanto a Monsalud, que este en breve rato no
supo qué decir.

---Yo creí que eras un caballero---dijo la americana.

Se le conocía que estaba haciendo esfuerzos terribles para conservar una
actitud digna. Los impulsos naturales la incitaban a gritar, a
arrancarse el cabello, a coger entre las manos al hombre, como se coge
un abanico, un juguete cualquiera, y destrozarle, haciéndole pedazos
pequeñitos.

Monsalud se dirigió hacia la puerta. Sus ojos y su gesto
decían:---Váyase usted.

---¡Pero si tú me oyeras!\ldots---murmuró Andrea, pasando súbitamente de
la ira a una aflicción profunda.

---No, no puedo oír a quien no conozco---repuso el hombre volviendo el
rostro.

---¿No me conoce usted?---gritó Andrea con voz semejante a un rugido.

Parecía que se alzaba sobre las puntas de los pies. La mujer crecía. Sus
brazos, tiesos hacia atrás; sus puños cerrados; sus labios descoloridos
que temblaban; su fina nariz, que con nerviosas contracciones también
expresaba la pasión desbordada; los músculos de su hermoso cuello,
tirantes; sus ojos, que amenazaban entre llamaradas de despecho; el
golpe violento de su pie en el suelo, como buscando apoyo para
levantarse más\ldots{} todos estos accidentes hubieran puesto miedo en
el corazón más acostumbrado a tales embates.

---¿No me conoce usted?---repitió.

---No---repuso Monsalud.

---¿No me conoció usted?

---Tal vez, pero\ldots{} ya no me acuerdo.

---Pues me conocerá usted---dijo Andrea con sofocada voz.

Dio algunos pasos fuera de la habitación; pero de súbito, con brusco
movimiento, se volvió y entró resueltamente. Detúvose; miró a Solita.
Hubo un momento de esos en que se ve inminente e inevitable el peligro
de un choque material, aun contando con la reconocida dignidad de las
personas.

Con la voz más áspera, más impertinente, más insolente y procaz que
puede imaginarse, Andrea hizo esta pregunta:

---¿Y tú quién eres?

Solita quedose muerta de espanto. Su propia turbación le impidió correr
hacia su hermano y abrazarse a él, buscando un refugio.

---Eso no se pregunta a los que están en su casa, sino a los que vienen
de fuera.

Al oír esto Solita se reanimó. En aquel momento pensaba una cosa.
Pensaba que si ella fuera mujer valerosa, echaría a escobazos de la casa
a la insolente dama.

---¡Oh, qué vil soy!---repitió Andrea corriendo otra vez hacia la
puerta.---¡Rebajarme así\ldots!

Apartando el rostro para no ver el de su amante, salió precipitada y
atropellándose, de la casa. Habiéndosele unido su criada en la escalera,
ambas bajaron.

Salvador se dejó caer en una silla, y apretando la cabeza entre las
manos, se clavaba en el cráneo las uñas.

---¡Oh! ¡Dios mío!, ¡qué infeliz soy!\ldots{} Sola, Sola, ¿has
visto?\ldots{} ¡Maldito sea yo mil veces! ¡Maldito sea el día en que
nací!

---Pero esa mujer---balbució la muchacha, saliendo de su
estupefacción,---es un demonio\ldots{} Comprendo que te cause tanto
furor\ldots{}

---¡No es demonio, es un ángel; y no me causa furor, sino que la
adoro!\ldots{} ¿No la viste? ¿Has visto mujer más hermosa?

---Tú\ldots{}

---¡La adoro, me muero por ella!\ldots{} Pero tú eres una tonta y no
puedes comprender esto. Sola, hermana mía, lloro porque\ldots{} no
puedo\ldots{} ten compasión, ten lástima, mucha lástima de mí.

Solita tuvo tanta lástima, que se echó a llorar.

\hypertarget{xxi}{%
\chapter*{XXI}\label{xxi}}
\addcontentsline{toc}{chapter}{XXI}

La calle de la Cabeza es una de las más tristes de Madrid. Compónese
toda ella de casas viejas y feas, entre las cuales descuellan la enorme
fachada meridional de la del marqués de Perales y otra que tiene grabada
sobre la puerta esta inscripción: \emph{Aparta, Señor, de mí lo que me
apartó de ti}. Contrastando con las vías cercanas, aquella no tiene
tiendas, y la mayor parte de las puertas están cerradas, a excepción de
las cocheras y cuadras que por allí mucho abundan. Hacia el Ave María la
calle se eleva, como si quisiera subir a los balcones de las casas.
Hacia la Comadre se hunde, buscando los sótanos. Algunas acacias, que se
asoman por encima de altos muros junto a San Pedro Mártir están mirando
con tristeza al escaso número de transeúntes. Se oyen tan pocos ruidos
allí que la calle no parece estar en Madrid y a dos pasos del Lavapiés.
Toda ella tiene un aspecto sombrío, un tinte lúgubre, una mala sombra
que no puede definirse, una atmósfera que abruma, un silencio que hiela.
Las calles, como las personas, tienen cara, y cuando esta es antipática
y anuncia siniestros designios, una fuerza instintiva nos aleja de ella.

Vulgarmente se cree que en la calle de la Cabeza no ha pasado nunca nada
digno de contarse. Por el contrario, es una calle trágica, quizás la más
trágica de Madrid. La tradición que le da nombre, y que no carece de
mérito en lo que tiene de fantasía, es como sigue: Vivía por aquellos
barrios un cura medianamente rico. Su criado, por robarle, le asesinó,
cortándole ferozmente la cabeza, y con todo el dinero que pudo encontrar
huyó a Portugal. No fue posible descubrir al autor del crimen, y
enterrado el clérigo, bien pronto su desastroso fin quedó olvidado. Pero
el asesino, después de haberse dado muy buena vida en Portugal durante
muchos años, volvió a Madrid hecho un caballero, aunque no tanto que
olvidase su primitiva condición de criado. Solía ir él mismo al Rastro
todas las mañanas a hacer su compra, y un día adquirió una cabeza de
carnero. Llevábala bajo la capa, y como chorreaba mucha sangre, que iba
dejando rastro en el suelo, fue detenido por un alguacil, que le mandó
mostrar lo que oculto llevaba. ¡Horrible espectáculo! Al echar a un lado
el embozo, el criado alargó en la derecha mano la cabeza del sacerdote a
quien le diera muerte.

¡Milagro, milagro! Este fue el grito general. Confesó todo el asesino y
le llevaron a la horca, acompañado de la cabeza del sacerdote que había
sido de carnero, y cuya vista horrorizaba y edificaba juntamente al
pueblo. Murió, según dicen, con grandísima devoción y arrepentimiento, y
hasta que no entregó su alma a Dios, no recobró la testa del cura su
primitiva forma carneril. Felipe III, que a la sazón nos gobernaba,
mandó labrar en piedra una cabeza que se puso en la casa del crimen para
memoria de aquel estupendo suceso.

En este siglo la calle de la Cabeza presenció muy de cerca el horrible
asesinato del marqués de Perales el l.º de Diciembre de 1808\footnote{Veáse
  \emph{Napoleón en Chamartín}.}. Cuando las revueltas políticas del 14,
vio encarcelar a los diputados y ministros, y aquel silencio tétrico fue
turbado en más de una ocasión por los rugidos de la plebe furiosa
embriagada. Nuestra narración nos lleva ahora a la citada calle y a uno
de sus edificios más antipáticos y más feos: la cárcel eclesiástica o de
la Corona, que estaba en la esquina de la calle Real de Lavapiés, y que
todavía existe, aunque destinada a cuadras o cocheras.

Un portalón daba entrada al patio, que no había sufrido variaciones
esenciales y tenía en dos de sus lados columnas de piedra para sostener
la crujía alta. Las prisiones estaban en el piso bajo y en los sótanos,
y consistían en calabozos inmundos, algunos con rejas a la calle. Dos
puertecillas abiertas a un lado y otro del zaguán indicaban el cuerpo de
guardia y las habitaciones de algunos empleados de la cárcel. Todas y
cada una de las partes del edificio, dentro y fuera, arriba y abajo,
ofrecían repugnante aspecto de incuria, descuido y degradación.

La ignominia de la cárcel empezaba desde la puerta. En la esquina del
edificio se veían multitud de inscripciones terroríficas e indecentes. A
conveniente altura, una de esas manos de artista que tanto abundan en
España había pintado una horca de la cual pendía un cura, y debajo se
leía \emph{Tamajón}. En la misma puerta otro artista había trazado una
especie de cuadro de ánimas donde varios curas recibían tizonazos de los
demonios, y más lejos varios milicianos nacionales, caracterizados en la
pintura tan sólo por el morrión, asaban un cerdo que llevaba el nombre
de \emph{Vinuesa}. En el portal repetíanse las horcas y además otra
pintura ingeniosa. Un grotesco y ventrudo muñeco, que tenía en la panza
el consabido letrero, abría la boca. Como si esta fuera la de un horno,
varios milicianos o figurillas de morrioncete metían por ella con sendas
palas un objeto en que se leía \emph{Constitución}. Por debajo una
escritura infernal rezaba el \emph{Trágala}, \emph{perro}, \emph{tú
servilón}.

Vinuesa estaba en un calabozo del piso bajo. En la puerta negra habían
trazado con tiza la horca y el ahorcado, repetidas formulillas, como
\emph{Muera el traidor}, y una cuarteta que decía:

\small
\newlength\mlenc
\settowidth\mlenc{\quad ¡Considera, alma piadosa,}
\begin{center}
\parbox{\mlenc}{\quad ¡Considera, alma piadosa,                      \\
                en esta nona estación,                               \\
                el árbol de que colgaron                             \\
                al cura de Tamajón!}                                 \\
\end{center}
\normalsize

Dentro del calabozo no reinaba oscuridad profunda. Veíase al infeliz reo
arrojado en el suelo sobre un jergón inmundo. Era un hombre viejo,
aunque entero, de cuerpo pequeño y que debió de ser fornido; pero la
larga prisión habíale extenuado considerablemente. Su pelo entrecano; su
barba blanca, muy crecida por no haberse afeitado durante el encierro;
su rostro en que se pintaban resignación y amargura, dábanle aspecto
venerable que sin duda no tenía cuando andaba suelto por la Villa, o
haciendo planes en su casa de la inmediata calle de San Pedro Mártir.
Vestía sotana suelta, raída y llena de jirones, y un gorro negro de
punto, calado hasta más abajo de las orejas, le cubría la cabeza. Cuando
no estaba echado sobre el miserable jergón, se ponía a pasear de un
ángulo a otro o se sentaba en la única silla, apoyando los brazos sobre
una mesa negra, y la cabeza en los brazos para dormir un poco. En la
mesa negra estaba pintada también con tiza la horca y un diablillo que
tiraba de los pies del ahorcado. En las paredes se leían varias estrofas
de las más indecentes del Lairón. Pero al desgraciado preso no le
mortificaba tanto leerlas como oírlas, y este era su principal tormento.

Todos los chulillos que pasaban de vuelta para el Lavapiés a la
madrugada; todos los rondadores guitarristas que iban a recorrer las
calles; todos los grupos de vagos que regresaban de los clubs o de las
logias; todos los patriotas que salían de las tabernas a hora avanzada,
y los chiquillos al salir de la escuela por las tardes o al ausentarse
de ella para ir de huelga o pedrea al Mundo-Nuevo, hacían escala al pie
de la reja del calabozo de Vinuesa; así es que este oía constantemente
durante diez y ocho horas de las veinticuatro del día, los famosos
versos:

\small
\newlength\mlend
\settowidth\mlend{\quad Dicen que vienen los rusos}
\begin{center}
\parbox{\mlend}{\quad Dicen que vienen los rusos                    \\
                por las ventas de Alcorcón.                         \\
                \null \qquad \textit{Lairón, lairón.}               \\
                Y los rusos que venían                              \\
                eran seras de carbón                                \\
                \null \qquad \textit{Lairón, lairón}.}              \\
\end{center}
\normalsize

Estas eran las estrofas comunes, pues las picarescas e indecentes, en
que se atribuían al \emph{cura de Tamajón} las mayores atrocidades y
desvergüenzas, no pueden copiarse. El populacho veía en Vinuesa un
galanteador de muchachas, corruptor de doncellas, tercero, mancebista y
cuanto abominable y ruin puede imaginarse. Nada de esto es verdad. Su
único delito había sido el plan que conocemos; pero si hubiera faltado a
las leyes morales con perversidad e indecencia, habría purgado sus
culpas con el infierno expiatorio que tenía en la prisión. Era este un
lúgubre ventanillo cuadrado y pequeño, con una cruz de hierro en el
vano. Por allí entraba la voz terrible del populacho cantando infames
coplas, amenazando e insultando sin cesar al pobre reo. Vinuesa
aborrecía el nefando agujero por donde le entraba la luz y la ira de la
nación vengativa; y por verle tapado, aunque le dejase a oscuras, diera
lo restante de su vida y la esperanza de libertad. Si lograba conciliar
el sueño, no dejaba de ver aquel boquete horrible, que en su mente
febril representaba como el ojo y la boca de la inmunda canalla, que sin
cesar le vigilaba y le escupía.

Gil de la Cuadra estaba encerrado en un calabozo de otra crujía, y no
gozaba de la preeminencia de vistas a la calle. En su encierro había
bastante claridad, y tenía mejores muebles que Vinuesa, entre ellos una
cama en alto. También su puerta se ornaba con inscripciones; pero en lo
interior no las había. Mortificábanle principalmente los gritos, cantos
y disputas de los milicianos nacionales, que tenían su cuerpo de guardia
en el zaguán, y que alborotaban en el patio mucho más de lo conveniente.

Bastante después del encierro sintiose atacado de dolores en las
articulaciones de las piernas, y no dudó que su reumatismo
constitucional le iba a hacer una nueva visita. Guardó cama,
resignándose al suplicio de sus dolores con paciencia cristiana, y tuvo
varias alternativas de alivio o recrudescencia. A falta de auxilios
médicos, disfrutó de los cuidados de un calabocero algo piadoso, que por
haber padecido del mismo mal, no sólo poseía recetas y cierta ciencia
práctica, sino también una compasión hacia todos los reumáticos.

De esta manera transcurrieron muchos días. Lo que más hondamente
perturbaba la naturaleza moral y física del ex-oidor era la
incomunicación y con esta la negra tristeza en que vivía, si aquello era
vivir. Solo, febril, contemplando perpetuamente su situación, midiendo
sin cesar la considerable distancia que le separaba de su hija, pasaba
las largas horas del encierro, y veía la lenta serie de noches y días,
marchando como las ruedas de una máquina de tormento. A ratos oraba, a
ratos derramaba amargas lágrimas; por breves momentos recibía consuelo
de su propia imaginación, representándose la libertad y la paz de su
casa; pero estas bellas sombras pasaban pronto, y el calabozo le ponía
delante sus cuatro paredes inalterables. Conocido el estado de su ánimo,
lleno de amargura, se comprenderá cuáles serían su asombro y emoción al
ver que un día se abrió la puerta del calabozo, que entró un hombre, y
que en aquel hombre reconoció, después de congojosas dudas, la persona
auténtica de Salvador Monsalud.

Este corrió a abrazarle y Gil de la Cuadra se desmayó de alegría.

---¡Mi hija, mi hija!\ldots---murmuró cuando recobraba el uso de la
palabra.---¿Ha muerto?, ¿vive?

---¡Ánimo, Sr.~Gil!---gritó Monsalud.---Pronto verá usted a su hija, que
está buena como nunca, y muy contenta al saber que pronto estará usted
libre.

---¡Yo libre!---exclamó el anciano abrazando a su amigo.

---Todavía no; pero pronto será.

---¿Y Anatolio?

---No ha venido aún.

Siguió haciendo preguntas, menudeándolas con tanta prisa que casi no
daba tiempo a la contestación, y al fin se ocupó de su causa que había
dejado para lo último. Monsalud, en breves términos, le explicó, si no
todo, gran parte de lo que había hecho, así como las circunstancias de
su presencia en la cárcel y el destino que desempeñaba.

---Tengo la seguridad---dijo,---de que conseguiré un objeto en el cual
he empleado tanta actividad, tanta fuerza, tanta paciencia. La santidad
de la obra emprendida, que es el cumplimiento de una de las primeras
leyes cristianas, me hace creer que esta vez, como otras, mi trabajo no
será estéril. He sufrido contrariedades, amigo mío, contrariedades
graves; pero al mismo tiempo he empezado a conocer uno de los mayores
goces que puede sentir el hombre y que hasta ahora\ldots{}

---No había usted conocido.

---Al menos en tan alto grado.

---El goce incomparable de hacer bien a un semejante---dijo Cuadra con
voz balbuciente por la emoción.

---Ese, sí, y el de poder dar forma al agradecimiento expresándolo en
hechos.

---¡Oh!, sí, también es un goce inaudito.

---Y tranquilizar la conciencia.

---Es verdad.

---Porque el recuerdo de las grandes faltas---añadió Monsalud,---no se
atenúa sino con la práctica constante de buenas acciones.

---También, también.

---Todo me anuncia que esta vez mi afán no tendrá, como otras veces, un
éxito desdichado. El corazón mío, que es la desconfianza misma, me está
diciendo ahora: «triunfamos, triunfamos de seguro». Será usted libre,
amigo mío, y lo será pronto. Sólo le recomiendo a usted un poco de
paciencia. Consuélese usted con saber que me tiene muy cerca, y que
estoy discurriendo los medios de rematar nuestra obra.

Gil de la Cuadra, arrojándose en brazos de su protector, lloró como un
niño.

\hypertarget{xxii}{%
\chapter*{XXII}\label{xxii}}
\addcontentsline{toc}{chapter}{XXII}

Mientras esto ocurría, todo Madrid se alarmaba con una estupenda
noticia. Por todos los barrios, por todos los clubs, por todos los
círculos corría una noticia, que muchos suponían increíble, por lo
disparatada, y otros aceptaban con resignación como una nueva prueba de
los desaciertos y traiciones del Ministerio. El fiscal de la causa
formada contra Vinuesa no pedía para este más que diez años de presidio.
El pueblo irritado, a quien habían hecho creer que la muerte del
arcediano no era bastante castigo para las culpas de este, vio en los
diez años de presidio una pena tan suave, que más que pena le parecía
recompensa. De los demás conspiradores absolutistas nada se decía aún;
mas era probable que recibirían en pago de sus infamias algunos años de
encierro, es decir, confites.

No es preciso indicar que en todo Madrid, y principalmente en los
barrios bajos era un Evangelio la opinión de que \emph{había corrido
mucho dinero} para absolver a los malhechores, y los más listos decían:

---¿Pues qué?, el Rey no podía dejar perecer a sus amigos.

En esto se equivocaban, porque Fernando se distinguía de todos los
malvados por un funesto sistema de abandonar cobardemente a cuantos le
habían servido, y aun gozarse de un modo incalificable en la desgracia
de ellos, como lo prueban, entre otras muchas cosas, las célebres
palabras que pronunció ante los guardias fugitivos y vencidos el 7 de
Julio. La verdadera causa de la lenidad relativa del fiscal y más tarde
del juez, fue que el Ministerio y los masones habían llegado a
comprender cuán bárbara y soez era la excitación vengativa del
populacho, a pesar de haberla excitado ellos mismos en Febrero y Marzo,
y quisieron rendir homenaje a la humanidad y la justicia, evitando un
sacrificio inútil. Hemos llamado lenidad a la pena anunciada, porque con
respecto al furioso ardor de la canalla lo parecía; pero en rigor de
justicia era una atrocidad, que sólo tiene disculpa en las infames
transacciones a que obligan los yerros políticos.

En los \emph{Comuneros} la noticia fue chispa arrojada a la mina. La
fortaleza reventó y una explosión de salvajismo, de barbarie, de odio y
necedad atronó la \emph{Plaza de armas}. Los honrados y los inocentes,
que no eran los menos bajo el estandarte de Padilla, hacían coro a los
malvados, por la solidaridad que entre todos reinaba. Eran los primeros
envueltos en el torbellino, y sin saberlo, estaban tan locos como los
demás, mejor dicho, los honrados y los inocentes eran los verdaderos
locos, porque los perversos conservaban bajo la borrachera de venganza
su nefanda razón. Pero en realidad, la noticia de la blandura del juez,
más les agradaba que les afligía. Servíales de pretexto para poner en
ejercicio su ideal de barbaridades, atropellos y desafueros, y de
admirable tema para gritar contra el Gobierno, llenándoles de befa y
escarnio. Acogieron, pues, el suceso con el frenesí del beodo a quien
dan aguardiente, y se hartaron de furia, de exaltación política,
poniéndose como demonios en la sesión que celebraron la noche de la
noticia.

Romero Alpuente, a quien respetaban, no pudo presidir la sesión, porque
le fue imposible sofocar el tumulto. Regato emitía con su habitual tono
de importancia las opiniones más furibundas. Mejía sudaba gritando, y
con el rostro encendido, gesticulaba sin poder conseguir que le oyeran.
Pelumbres daba golpes en los bancos con un bastón semejante a la lava de
Hércules. D. Patricio, renunciando a ser oído por toda la Asamblea,
pronunciaba, ora frases áticas, ora apóstrofes demostenianos en un
pequeño grupo que se formó a su lado. En suma, la \emph{Plaza de Armas}
más que guarnición regular, parecía un ejército indisciplinado, un
manicomio insurrecto o un infierno en que fuese ley la libertad
individual para hacer diabluras. Cada cual pedía una cosa distinta, y es
incomprensible que no se rompieran la cabeza unos a otros, único medio y
fórmula de conciliar todas las opiniones.

Era que comúnmente la Asamblea en pleno no resolvía nunca nada, siendo
más bien doctrinales, digámoslo así, sus sesiones que ejecutivas. La
alta dirección de la \emph{Comunería} estaba, como la de los masones, en
un pequeño consejo, en cuyo seno ha llegado la hora de que nos
introduzcamos osadamente. Hemos presentado en otro libro la camarilla de
Palacio\footnote{Veánse las \emph{Memorias de un cortesano de} 1815.}.
Tócales ahora su vez a las camarillas populares, poderes igualmente
misteriosos y perturbadores, y la dificultad de nuestro trabajo aumenta,
porque las camarillas eran dos: la del populacho o de los exaltados, y
la de los constitucionales o moderados. Procedamos con método.

\emph{Camarilla del populacho.}---No tenía local fijo. Reuníase algunas
veces en un departamento reservado del café de Lorencini; otras, en el
mismo local de la Asamblea o en casa de Regato. La reunión de ella que
nosotros vamos a presenciar no fue celebrada en ninguno de estos
parajes, sino en una taberna de la calle de la Estrella. De los veinte
diputados comuneros no asistió ninguno; de los periodistas, sólo Mejía;
de los que tenían cargos oficiales en la Asamblea de Padilla, sólo
Regato; de los viejos, sólo D. Patricio Sarmiento; pero no faltaba ni
uno siquiera de los amigos de Timoteo Pelumbres, ni tampoco la pandilla
de milicianos nacionales, en la cual alzaba el gallo con altanera
superioridad Pujitos. Sumaban entre todos once personas, y para poder
discutir con más libertad, Regato mandó al tabernero que cerrase, luego
que todos estuvieron dentro, y cuando el vino empezó a hacer su oficio
para que las lenguas pudiesen desempeñar mejor el suyo.

---Queridos compañeros---dijo Regato,---estamos perdidos.

Esta frase hábil produjo la sensación apetecida.

---Perdidos, porque el Gobierno nos va a meter el diente, y los hombres
gordos de nuestro partido se esconden en su casa llenos de miedo.

---Romero Alpuente---dijo uno,---tiembla como una gallina mojada.

---Desde que se ha dicho que el Gobierno va a pegar, nuestros diputados
ya están buscando vendas.

---Está visto que para reclutar gente valerosa---dijo Regato, a quien
agradaba mucho la veneración con que era oído,---no hay que contar con
la gente de lengua y pluma. ¡Pobre pueblo, siempre sudando por gobernar
como manda la ley de Dios, y siempre engañado por tanto pillo! Está
visto que mientras el pueblo no diga: «pues quiero y esto ha de ser», y
lo haga como lo dice, no tendremos libertad.

---Pero cuando el pueblo quiere portarse como quien es---manifestó
Pelumbres,---vienen los \emph{futraques}, llenos de jabón y pomada, y
sacan los catecismos de la política para decirnos cosas lelas y de mil
flores\ldots{} con lo cual se acaba todo, y en buenas palabras resulta
que somos unos zopencos y ellos unos Salomones. Nosotros trabajamos y
ellos comen.

---Señores---repitió Regato dando un suspiro,---estamos perdidos. El
Gobierno, viendo que no servimos para nada, (y no me vuelvo atrás\ldots)
que no servimos para nada, va a pegar, pero a pegar muy fuerte.

Breve silencio siguió a estas palabras.

---Los palos serán para el que los aguante, que yo\ldots{}

---Los palos serán para todos---afirmó Regato en el tono de la mayor
competencia.---Yo sé de buena tinta lo que trama el Gobierno; lo sé
todo, y pues venimos aquí para ver cómo nos defendemos, lo voy a decir.

---El Gobierno va a cerrar los cafés.

---Y a reformar la Milicia Nacional de modo que no entren sino los que
él quiera.

---Y a corregir la Constitución.

---Y a poner dos Congresos: uno como el que está, y otro de clérigos,
obispos, generales, marqueses, camaristas y toda la recua de
alabarderos, \emph{persas} y serviles.

---Y a suprimir todos los periódicos---indicó Pujitos, dando a entender
de este modo sus aficiones literarias.

---Y a mandar a Riego a Filipinas.

---Todo eso y mucho más hará el Gobierno---dijo Regato;---pero como a
quien más aborrece es a los buenos patriotas, empezará su obra
acogotando a los buenos patriotas, que somos nosotros.

---Nosotros---repitieron algunos.

---Y pasando la mano por el lomo a los serviles, que serán los
mandarines de mañana. ¿Qué significa la libertad de Vinuesa?

---¿La libertad?

---La libertad, sí. Para los bobos, eso de los diez años de presidio
significa\ldots{} diez años de presidio; pero para nosotros, que somos
tan listos y vemos un mosquito en la punta de una torre, esa pena no es
más que la absolución del cura.

---Es lo mismo que yo pensaba.

---Le sacan de la cárcel; hacen la pamema de llevarle a Ceuta; métenle
en cualquier convento, donde habrá abundancia de buenas magras, pollos
con tomate, gran trago de vino y muchachas bonitas; dicen luego que se
ha escapado, y al poco tiempo, indulto. Tras el indulto viene la
canonjía y tras la canonjía la mitra.

---Pues estamos bien---dijo uno con impaciencia, golpeando el suelo con
su bastón.---Protesto.

---Protesto yo también---exclamó Pelumbres.

---Si la Sociedad de los \emph{Comuneros}, que empezó con tan buen pie,
no saca ahora la cabeza, ¿para qué sirve?

---Para nada, Sánchez, para nada---repuso un hombre que era tratante en
cueros.---Dende que oí discursos y vi papeles y \emph{toma la palabra,
daca la palabra,} se me cayeron las alas del corazón\ldots{} ¡botijos!,
yo no sirvo para esto.

---Es muy posible que el Gobierno tenga la alevosa intención de indultar
a Vinuesa y aun darle una mitra---dijo con gravedad un individuo de
aspecto decente, furibundo patriota cándido que tenía dos tiendas y un
buen nombre que no hace al caso;---yo creo cuanto ha dicho el amigo
Regato, porque el

Gobierno es en la superficie liberal y en el fondo absolutista.

---Si Riego estuviera en Madrid, otro gallo nos cantara, amigos---indicó
Regato.---Yo de mí sé decir que si tuviera dos docenas, dos docenas nada
más de buenos patriotas, intentaría cualquier sublimidad.

---Cualquier hazaña épica, digna de perpetuarse en mármoles---dijo D.
Patricio.---Sr.~Regato, manifieste usted con claridad su pensamiento.
¿Se trata de que Madrid se levante en masa y arroje del gobierno a ese
Ministerio, y convoque otras Cortes, y le caliente las orejas al
\emph{Rey neto}?

---Eso es difícil hoy; pero no lo será dentro de seis meses, cuando
estemos mejor organizados y se multipliquen las \emph{Casas fuertes} de
los regimientos y se reciba el dinero que nos han prometido de América.
Contentémonos ahora con dar una prueba de nuestro mucho poder, de lo que
somos y lo que valemos, para que tiemble el cobarde tirano y nos tengan
miedo los mandarines.

---Ved aquí, amigos míos---dijo Sarmiento,---cómo admirablemente
concuerda con mi opinión la del Sr.~Regato. Siempre he sostenido la
necesidad de elevar la voz para que nos oigan, de alzarnos sobre las
puntas de los pies para que nos vean, de presentarnos en todas partes
para que nos toquen, mientras llega la hora sublime de los bofetones.

---Yo no entiendo de estas máquinas sutiles---manifestó, con la
ingenuidad de la barbarie, el llamado Sánchez, que era miliciano y había
sido primero cortador de carne y después empleado en cárceles.---Yo lo
que sé es que si conviene dar porrazo se dé porrazo. No hay más que dos
políticas: dar y recibir.

---En lenguaje sencillo---dijo Mejía,---ha expresado Sánchez la idea de
que mientras no se puede realizar una insurrección que dé la victoria al
pueblo, se hagan manifestaciones patrióticas con objeto de que se nos
considere como un elemento importante, capaz de cualquier cosa en el
Gobierno o en la oposición.

---A eso iba---indicó Regato con acento magistral.---En pocas palabras,
señores; el Gobierno dice blanco, pues nosotros decimos negro; el
Gobierno quiere coles, nosotros lechugas; el Gobierno dice \emph{por
aquí no se va}, nosotros decimos, \emph{por ahí iremos}.

---El Gobierno dice, \emph{no más clubs}, nosotros respondemos
\emph{vengan clubs}.

---El Gobierno quiere poca Milicia, nosotros mucha Milicia.

---El Gobierno perdona a los absolutistas, pues condenémosles nosotros.

---Condenémosles, caballeros---gritó el tratante en
corambres.---¡Botijos! Si nosotros no hacemos la justicia, ¿quién la va
a hacer?

Dando golpecitos en la mesa con el fondo del vaso, después de beberse el
contenido, entonó esta canción:

\small
\newlength\mlene
\settowidth\mlene{\quad Ay le-lé, que toma que toma,}
\begin{center}
\parbox{\mlene}{\quad Ay le-lé, que toma que toma,                 \\
                ay le-lé, que daca que daca,                       \\
                ya no bastan las razones,                          \\
                apelemos a la estaca.}                             \\
\end{center}
\normalsize

---El ciudadano D. Bruno ha tocado el punto más delicado de la política
actual ---dijo Regato.---El pueblo, señores, no debe consentir la
impunidad de quien ha trabajado y trabaja aún en contra del pueblo.

---¡Botijos!\ldots{} no.

---De ninguna manera.

---Consentirlo sería gravísimo desacierto---afirmó Sarmiento.

---Como me llamo Pelumbres, tan cierto es que todo el día he estado
pensando en que debíamos hacer justicia, porque podemos y debemos
hacerla. Y si el pueblo no es soberano para esto, ¿para qué lo es?

---A fe de Mejía, sostengo que cuando los jueces son inmorales y
corrompidos, el pueblo no tiene más remedio que echársela de juez.

---Pues con una palabra basta---afirmó el tratante en pellejos.

---Es preciso sacar a Vinuesa de la cárcel antes que le indulten.

---Y ahorcarle---dijo Sánchez, apretándose su propia garganta.

---En la plazuela de la Cebada.

---En la plaza de Palacio, delante del balcón de Su Majestad---gruñó
Pelumbres.

---Admirable y sensata idea---dijo Regato;---pero me parece
irrealizable. No es preciso que se lleven las cosas a ese extremo de
perfección.

---No puedo aconsejar tranquilo la muerte de un hombre---afirmó
Sarmiento con gravedad;---pero hay sacrificios necesarios,
indispensables, y el cura de Tamajón debe morir. También hay en la
cárcel de la Corona un dichoso Gil de la Cuadra, ex-vecino mío, que es
uno de los servilones más furibundos, y un conspirador terrible.

---Gil de la Cuadra---dijo Regato haciendo memoria.---¡Ah!, ya. Le
protege Salvador Monsalud, después de haberle enamorado a su mujer, como
me consta. Váyase lo uno por lo otro.

\emph{---El traidor} Monsalud se dirá de aquí en adelante---indicó
Pelumbres.---Ese canalla, después de entrar en nuestra sociedad ha
admitido un destino del Gobierno.

---En la cárcel de la Corona precisamente---indicó Mejía.---No lo
hubiera creído. Puesto de confianza, señores. Aquí hay gato encerrado.

---Tengo a Monsalud por una persona decente---dijo D. Patricio.---Es
amigo mío y no le creo capaz de servir a los masones. Le he oído hablar
pestes de esos señores.

---Sea lo que fuere---dijo Sánchez,---ello es que antes de meter
semejantes tipos en nuestra sociedad, debiéramos pensarlo mucho.

---Es justa la censura, aunque confieso que yo le presenté---dijo
Regato;---pero no hay motivo para desconfiar de tal joven. Tengo motivos
para creer que puedo dominarle en un momento dado. Ese hombre será mío
cuando yo quiera. En vez de importarnos que esté empleado en la cárcel,
debemos felicitarnos de ello. Sacaremos partido de esta circunstancia.

---¡Re-botijos!\ldots{} ¡Si está en mi lugar y en el puesto de que me
echaron hace dos meses esos mamones!\ldots{} ¿pues no ha de importarme?
Es un caballerito a quien tengo atravesado aquí.

---Dejemos esta cuestión mezquina, señores, y volvamos a lo
principal---dijo Regato.---¿Hay aquí gente de valor?

---Basta y sobra; pero si se quiere cosa mayor, con dar la voz en
ciertos barrios se tendrá toda la gente que se quiera.

---Sr.~D. Bruno, ¿se puede ir a donde se quiera?

---Al cabo del mundo. Digan hora y lugar y allá estaremos todos. No
saldrá tan mal como la noche de los embajadores del Ruso y el Turco.

---Mañana\ldots{} mañana\ldots---dijo Regato meditando.---¿Cuál será la
mejor hora?

---Por la noche.

---No, por el día.

---A las doce del día---gritó el más decente de todos.---No se trata de
ninguna traición, sino de una obra de justicia.

---¡Excelente idea! A las doce del día.

\emph{---Coram populo}---murmuró Sarmiento.

---¡Botijos!, a las doce en punto.

---Y ahora---dijo Regato levantándose,---a prepararse. La cosa puede ser
sencilla si el Gobierno deja a la Milicia en la guardia de la cárcel.
Pero si pone tropa\ldots{}

---Si se atreve a poner tropa, entonces\ldots{}

---Que ponga tropa---gritó Pelumbres dando un puñetazo,---y se hará
justicia a la tropa.

Eso es, justicia a la tropa.

Porque no es más que justicia.

---Esta noche hay otra vez Asamblea, señores---dijo Regato con
misterio.---Mucho cuidado con los caballeros comuneros de corbatín
almidonado y palabrejas cultas. Dirán, como esta noche, que estamos
locos.

---¿Se guardará secreto?

---Hasta donde se pueda; pero hay que reclutar gente, mucha gente.

---¡A la \emph{Fortaleza}, a la \emph{Fortaleza}!

---En la \emph{Fortaleza} hay espías y traidores que todo se lo cuentan
al Gobierno.

---Si el Gobierno lo sabe, mejor---vociferó Pelumbres.---¿Qué apostamos
a que voy a Palacio y se lo digo yo mesmo al Rey?

Una carcajada general acogió estas palabras.

---Las cosas claras. Se va a hacer justicia. Yo lo digo a todo el que me
quiera oír. ¡Muera Tamajón!

---¡Muera Tamajón!---repitieron todos menos Regato.

Este con voz apagada y razones conciliatorias quiso aplacar a sus
amigos; pero estaban muy encariñados con la idea emitida por el dos
veces gato, para dejársela quitar. Hay que pensarlo mucho antes de
arrojar la piltrafa a esta especie de carnívoros; pero una vez arrojada,
el que aspire a quitársela se expone a recibir un mordisco o arañazo.
Así lo comprendió el fundador de la Comunería. Cuando los individuos de
su alto Consejo salieron a la calle rumiando el sangriento manjar que
les había puesto en la boca, el cobarde Regato se asustó un poco; pero
aún tenía seguridades de no ser sospechoso, y entre Pelumbres y D. Bruno
marchó resueltamente a la Asamblea, que aún estaba abierta.

\hypertarget{xxiii}{%
\chapter*{XXIII}\label{xxiii}}
\addcontentsline{toc}{chapter}{XXIII}

Poco después de este suceso las \emph{Plazas fuertes} y \emph{Salas de
armas} encerraban un partido en ebullición. Pasada la media noche la
mayor parte de los comuneros sabían que estaba acordada para el día
siguiente la muerte de Vinuesa. A la madrugada, sabíanlo también los
masones por su bien servido espionaje, y conmovido el \emph{Grande
Oriente} ante amenaza tan audaz, deliberó con calor y afán tan
importante asunto. Lo que allí se dijo verase a continuación.

\emph{Camarilla constitucional.} Reuníase casi siempre en el Grande
Oriente, con asistencia de muchos hombres que se tenían por lumbreras,
de otros que realmente lo eran y de muchos que si carecían de soberbia o
de mérito, cobraban buenos sueldos en las oficinas de Reino. En la
Masonería había, según los datos más verosímiles, cincuenta y dos
diputados. De los ministros, la mitad por lo menos cargaban el mandil.
Pocos eran entonces los hombres notables, por su talento oratorio o por
su pluma, que no doblasen la cerviz ante el misterio eleusiaco, y muchos
que después han figurado en los partidos reaccionarios adoraron la
Acacia. Tal fue el atractivo del Orden masónico, que aun se dice
trataron con él clérigos no apóstatas y un general de franciscos que
después fue arzobispo\footnote{Fr.~Cirilo de Alameda desmintió de un
  modo enérgico la aseveración de Galiano.}. Para que nada faltase, los
del Arte-Real vieron en las logias a un Infante, que recibió el nombre
de \emph{Dracón}, con la risible particularidad de que le llamaban
\emph{Bracón}. Un general muy célebre era designado \emph{Bruto II}.
Puede dudarse que el mismo Fernando VII \emph{recibiese salario}
masónico; pero no que los nombres más ilustres y respetables del
presente siglo, los nombres de Argüelles, Calatrava, Quintana, San
Miguel, Flores Estrada, Galiano y otros figuraron en las listas de
Maestros, siendo probable que todos ellos fueran \emph{Sublimes
perfectos}.

La camarilla, en la hora que nos es permitido asistir a ella, estaba
formada por seis individuos nada más, cuyos nombres, a excepción del de
Campos, deben mantenerse en secreto. Si en el trascurso de la relación
son conocidos, enhorabuena; pero no se culpe al novelador de haber
manoseado nombres pertenecientes a personas de distinto valor, pero
todas respetables, algunas de las cuales han respirado hasta hace
poco\ldots{} y quizás haya alguna que respire todavía.

Los de la camarilla se reunían en la logia, pero allí estaban
familiarmente y sin ceremonias de rito, como clérigos en la sacristía.
De los seis, cuatro eran diputados; y de estos, dos habían sido
ministros y uno lo era en aquellos días. De los dos restantes, uno casi
no era masón, hallándose en la categoría de \emph{durmiente}, y el otro
era Campos. Atención.

Tiene la palabra un joven de treinta y tres años, alto, elegante, fino,
airoso. Sus modales y su vestido eran como su estilo, la corrección
misma. Su rostro morenísimo y su gran boca dábanle aspecto de fealdad;
pero tenía la belleza de la expresión y un claro sello de hidalguía y
caballerosidad que cautivaba. Sus ojos eran negros y vivísimos, llenos
de esa luz particular que indica poderosa erección de la fantasía; sus
cabellos alborotados y fuertes, algo parecidos a los de Chateaubriand,
rodeaban una espaciosa y limpia y celeste frente, emblema del
privilegiado artista. Era su voz grave y persuasiva, y si su estilo
carecía de arrebato, tenía en cambio la serenidad más simpática y un
acento que subyugaba oídos y corazones.

---Nosotros---dijo señalando a su amigo que junto a él estaba,---estamos
decididos a no asociar nuestro nombre a los errores que se están
cometiendo. Amamos la libertad con delirio; pero aborrecemos los excesos
del populacho y la ignominiosa licencia. Antes que empujar a la Nación
por este carril que la precipitará en el abismo, nos retiraremos de la
política, perderemos toda influencia, perderemos nuestro propio
prestigio, y entonces la vergüenza de haber contribuido a este desorden
nos servirá de expiación. No se nos oculta que el absolutismo volverá, y
quizás pronto, si a tiempo no se pone mano en reparar el Reino que se
desquicia; y el absolutismo vendrá porque las instituciones vigentes no
ofrecen condiciones de vida saludable y duradera, porque carecen de
fuerza para contener en límites razonables la iniciativa popular y son
incapaces de fundar nada sólido. Que el Gobierno, sabedor de la inicua
amenaza de los exaltados, evite que se consume un horrendo delito; haga
entender a esa gente que su destino y misión no es todavía ni será en
mucho tiempo dirigir la cosa pública; establezca el imperio de la razón,
de la calma, del buen sentido, y entonces variaremos de opinión.
Mientras esto no suceda, la división será completa, y si hoy permanece
oculta por nuestra prudencia, mañana trascenderá a las Cortes, y de las
Cortes a todo el país.

---Y se formará el partido \emph{anillero} o de los \emph{amigos de la
Constitución}---dijo un viejo alto y flaco, nervioso y lleno de
vivacidad, que respondía entre masones al nombre de \emph{Coriolano}, y
era célebre por un folleto contra los absolutistas y varios escritos de
Economía política.---Esta nueva escuela será funesta. Tendremos al fin
tantos partidos como hombres, y no habrá un solo individuo que se
resigne a pensar como los demás.

\emph{---La Sociedad de los amigos de la Constitución}---dijo el
compañero del primer orador que junto a él se sentaba,---responde a la
necesidad imperiosa de establecer un término medio entre las antiguas
leyes, que viven encarnadas en el país, y los principios liberales. ¿Por
qué no hemos de decirlo? Yo, por lo menos, tengo mi ideal en la
\emph{Carta} francesa, con las dos Cámaras y el voto absoluto.

Oyose un murmullo de desaprobación.

---Condenemos igualmente---dijo con gravedad el de los cabellos
alborotados y la boca grande,---toda clase de reuniones como esta, que o
sirven para fomentar el jacobinismo y ofrecer un secreto peligroso a las
intrigas y a las ambiciones, o no sirven para nada.

---Estamos disputando sobre si nos hemos de dividir más todavía,
mientras una cuestión palpitante, fundada en una alarma quizás falsa,
reclama nuestra atención. Este asunto no tiene espera. Nos está
llamando, y nosotros le volvemos la espalda para discutir sobre si
debemos ponernos un anillo en el dedo o un triangulillo de latón en el
ojal.

El que esto dijo era un hombre de más de cuarenta años, moreno como el
anterior, de facciones bastas y gruesos labios. Su cuerpo era fuerte y
algo pesado; carecía de soltura, gracia y flexibilidad; pero en cambio
parecía poseedor de una gran energía. Lástima que esta energía,
circunscrita al entendimiento y al estro poético, no trascendiese a la
voluntad.

Completaban su persona cabeza admirable, abultada y lobulosa; ojos
grandes y hermosos; una frente a la cual no faltaba sino el laurel para
ser olímpica; expresión grave y tono sentencioso en la voz. Allí dentro
le llamaban \emph{Pelayo}.

---Es verdad, es verdad---dijeron los demás.---A la cuestión.

---Los comuneros han decidido sacrificar a D. Matías Vinuesa---manifestó
Campos, que parecía secretario de la Junta.

---Causa horror el ver que estas atrocidades se cometan; pero causa más
horror aún que se anuncien---afirmó el que oímos al principio de la
sesión.

---Yo no lo creo---dijo el poeta.---Los que se ocupan en propagar
alarmas han escogido esta para el día de mañana. Reconozco que el pueblo
está irritado\ldots{}

---Con razón---manifestó \emph{Coriolano}.---La sentencia del juez es
capaz de sublevar al pueblo más generoso. ¿Por qué se vocifera tanto
contra el populacho, cuando sus excesos no son más que el rechazo,
digámoslo así, de las osadías de los absolutistas? No, no está el mal en
la canalla, que es honrada y generosa: no morirá la libertad en manos
del pueblo, sino en manos de los que quieren establecer una transacción
imposible con el despotismo.

\emph{Coriolano}, que se había expresado con energía, miró a los dos
\emph{anilleros}. Estos callaban, aunque uno de ellos era gran retórico.

---No disculpo ni disculparé a los exaltados que protestan contra la
sentencia del juez---dijo \emph{Pelayo} con calor,---pero téngase
presente que ha tiempo quedan impunes los mayores atentados y crímenes
de los absolutistas. Dicen que Vinuesa es tonto; yo no lo creo. Su plan
indica maquiavelismo, y por lo menos las intenciones de este clérigo han
sido perversas. Ganar y corromper la tropa, sublevar al pueblo,
sorprender a los principales diputados y a las primeras autoridades,
sacrificarlas inmediatamente a la seguridad y a la venganza del partido
conspirador y alzar sobre la sangre de aquellas víctimas el pendón de la
tiranía y de la intolerancia; estos son los proyectos contenidos en los
atroces papeles de Vinuesa. Convicto y aun confeso el miserable preso,
no debe librarse de la suerte rigurosa a que se exponen siempre los que
traman semejantes atentados contra la existencia de un Gobierno
establecido. El juez que ha despachado esta causa ha dicho públicamente
que cualquiera de los cargos que obraban contra el reo era capital, y
que por consecuencia era imposible salvarle. ¿A qué este cambio
repentino? ¿Por qué con tales antecedentes, Vinuesa no ha sido condenado
más que a diez años de presidio? Semejante condescendencia ha llamado
justamente la atención pública. Hasta se asegura que la Audiencia en vez
de agravar la pena la suavizará más. Dícese que han mediado presentes a
los cuales la integridad del juez ha resistido con nobleza y con honor;
pero que después han intervenido ciertos recados imperiosos de Palacio,
a cuyas fulminantes amenazas no ha podido sustraerse el magistrado,
haciéndole blandear desgraciadamente en su fallo\footnote{Este párrafo
  no es del narrador: es de las \emph{Cartas a Lord Holland}.}.

---Siempre han de achacarse todos los yerros a la incorregible
\emph{mano oculta}---dijo con desabrimiento el retórico.

---¡Siempre se han de achacar al populacho!---exclamó colérico el que
respondía al nombre de \emph{Coriolano---}.La plebe es causa de todo. La
Corte y el Rey no hacen más que rezar. Con tan admirable sistema de
crítica, resulta infaliblemente que la Constitución es detestable y que
debe convertirse en Carta.

---El populacho y la Corte---afirmó el retórico,---son igualmente
culpables. Pero si se encomienda al primero el castigo de la última,
esta vencerá.

---Eso es lo que no sabemos---repuso con inquietud y cierta excitación
el economista.---Por de pronto, tenemos que, según lo que acaba de decir
nuestro discreto amigo, la irritación del pueblo contra Vinuesa y contra
el juez Arias está justificada.

---Braman de cólera los genios impacientes---sostuvo \emph{Pelayo---} al
contemplar semejante impunidad, y hasta los más templados prevén y
lloran las tristes consecuencias que necesariamente ha de
producir\ldots{} Pero no puedo creer que un partido popular haya
acordado fría y villanamente el sacrificio del reo. Tanta bajeza es
inverosímil.

---Es cierta---dijo Campos, que hasta entonces, reconociendo su
inferioridad, había permanecido mudo.---La Asamblea comunera es un
volcán que vomita sangre.

---Pero ¿no queda duda de que han acordado eso?

---No queda duda. Lo sé por los espías que tengo allí.

---Si el Gobierno se hace cómplice de iniquidad tan grande---dijo con
honrada convicción el de los alborotados cabellos,---merece la
execración del género humano.

Uno que hasta entonces no había pronunciado palabra adelantó su cuerpo
hacia la mesa, tirando de la silla, y habló de este modo:

---No puedo callar después de lo que he oído. Se quiere que el
Ministerio lo hago todo, y nadie le ayuda, nadie, señores, cuando tiene
que defenderse contra la oposición de moderados y exaltados, y contra
las conspiraciones de absolutistas y comuneros, que se dan la mano para
trastornar al país. Pero el Gobierno no merecerá la execración del
género humano. ¿Acaso es él quien ha alentado las conspiraciones de los
serviles? Si ha habido cohecho en el asunto de la causa de Vinuesa, la
venalidad estaba consumada antes del 4 de Marzo en que entramos
nosotros. No podemos estar mudando jueces todos los días.

---No se trata de mudar jueces; se trata de impedir que una gavilla de
asesinos deshonre la revolución.

---¡Patrañas! Señores, es preciso acostumbrarse a no ver asesinos en
todas partes.

El que esto decía era un hombre casi anciano, masón, bastante listo y de
mucha práctica en los negocios administrativos. ¿Por qué ocultar su
nombre, que por sí se vela bastante con su propia oscuridad? Era don
Mateo Valdemoro, ministro de la Gobernación. En la hora de la madrugada
en que le vemos, quedábale sólo un día de poltrona.

---Yo creo que hay por lo menos exageración---dijo \emph{Pelayo}.

---Aunque sea exageración, deben tomarse precauciones---indicó Campos.

---Pero, señores, es ridículo que por una alarma necia, llenemos las
calles de artillería---indicó el ministro, creyendo que emitía una idea
feliz.---Parecería una provocación, y lo que no es más que una alarma
insignificante, podría trocarse en formidable motín. Nada me mortifica
tanto como la idea, muy generalizada, de que el Gobierno simpatiza con
Vinuesa, con el Abuelo y con los demás absolutistas presos.

---¿Entonces el plan del Gobierno es cruzarse de brazos y dejar
hacer?---preguntó con severidad el literato.

---El Gobierno castigará los desmanes.

---¿Qué desmanes?

---Los que se cometan; pero no hará alarde de despotismo, no provocará
al pueblo.

---Porque le tiene miedo.

---No tiene miedo, sino prudencia. La excitación que existe contra
Vinuesa es natural y lógica. Si acuchillamos al pueblo, porque no
simpatiza con los absolutistas, pasaremos por serviles, y nuestro lema
es Constitución.

---Yo sigo creyendo que no habrá nada---dijo Pelayo, hombre que en su
gran talento, tenía la más patriarcal buena fe.---Repito que el pueblo
es bueno.

---Si no le instigaran los tunantes\ldots{}

---Es más---añadió el ministro.---Si acuchillamos al pueblo, daremos un
gustazo a la Corte, Vinuesa estará libre dentro de dos meses, y las
cárceles llenas de liberales.

---Pues ahorquen ustedes a Vinuesa---dijo con la mayor viveza el
retórico.---Esto sería lógico. Lo absurdo es absolverle y permitir las
horribles venganzas del populacho.

---Siempre el populacho\ldots{} es decir, el gato---indicó
\emph{Coriolano}.

---Si ahorcamos a Vinuesa, exacerbaremos a los serviles y a la
Corte---dijo el ministro en tono de perspicacia.---Prudencia por un lado
y por otro, es lo que conviene. ¿No es sistema de ustedes contemporizar
con todos?

El de los erizados pelos, es decir el retórico o el literato, a quien
esta pregunta se dirigía, estuvo un momento sin saber qué contestar.

---Sí: contemporizar---repuso al fin,---establecer un equilibrio
perfecto, dando la mano a unos y a otros; pero no a los infames, no a
los asesinos.

---Estamos juzgando un suceso que no ha pasado todavía ni pasará
probablemente ---dijo Pelayo.---¿A qué hablar de asesinos? Yo defiendo y
defenderé siempre al pueblo. Si alguna vez asesina, hácelo con el puñal
que le entregan los de arriba.

---Sea de oro, sea de hierro, lo que importa es que no haya
puñal---objetó el retórico.---En una palabra, señores, estamos reunidos
para acordar si se debe impulsar al Gobierno a tomar una medida
enérgica.

---¡Una provocación!\ldots{} Yo opino como el ministro---manifestó
Pelayo.---El pueblo es bueno, es generoso; pero no debe ser provocado.

---Pues preparémonos a que sea nuestro dueño---dijo el que había
demostrado más seso.---Señores---añadió levantándose,---mi compañero y
yo nos retiramos para no volver más aquí.

El viejo economista tiró al retórico de los faldones de su levita,
diciéndole con buen humor:

---Señor cartista: no nos deje usted tan despiadadamente. Somos amigos y
zanjaremos nuestras diferencias de familia. Discutamos.

---Me parece que se ha discutido bastante. ¿No ha llegado aún la ocasión
de hacer algo?

Aquel hombre que tan bien se expresaba, demostrando tener en su espíritu
el instinto de la eficacia política, era de voluntad flaca, como los
demás. La sensatez de sus ideas era un fenómeno comprendido dentro de la
serena esfera de las aptitudes literarias, y al expresarse con tanta
cordura, hablaba su talento, no esa facultad prodigiosa en que se
confunden perspicacia y acción, conformando al hombre político. La misma
perplejidad que tanto combatía le contaminó cuando fue ministro. Amaba
la \emph{Carta}; pero cuando pudo ocuparse de ella con éxito, pensaba
demasiado en la de Horacio a los Pisones.

---Todo puede arreglarse---dijo Pelayo.---Por sí o por no, y aunque hay
en esto mucho de ponderación, creo que se debe quitar la guardia de
milicianos que está en la cárcel de la Corona, y reemplazarla con tropa
de línea.

---Eso me parece una necesidad imperiosa---añadió Campos, atreviéndose,
contra su costumbre, a algo más que callar y tomar lo que le dieran.

---Al menos eso probaría cierta prudencia en el Gobierno---dijo el de la
Carta deteniéndose, mas sin volver a sentarse.

---No: la verdadera prudencia---objetó Valdemoro,---consiste en no poner
ni quitar ninguna guardia, porque eso sería origen de sospechas,
hablillas, escándalos y seguramente de disturbios graves.

---Adiós, señores---dijo el simpático y cortés joven de treinta y tres
años.

---Mudar la guardia me parece una provocación---repitió el ministro
consultando fríamente el rostro de los tres que a su lado quedaban.

Ninguno dijo nada.

---Si se hace con maña y habilidad---dijo Pelayo,---quizás no.

---Señores---manifestó el ministro con la inquietud propia del que se ve
abrumado de responsabilidad.---Es muy fácil resolver todas esas
cuestiones fuera del Gobierno, y cuando uno se mete tranquilamente en su
casa sin dar cuenta a Dios ni al Diablo de lo que hace. Ustedes hablan,
como los libros, un lenguaje discreto; pero la práctica, señores, la
práctica es cosa muy distinta. ¡Mudar una guardia! Parece la cosa más
sencilla del mundo dicho así, como si se tratara de mudarse una camisa;
pero los que estamos dentro del Gobierno vemos las cosas de su tamaño.
Repito que mudar mañana la guardia es pegar fuego a una hoguera. El
Gobierno trabajará; el Gobierno tiene algunas influencias en las clases
populares; aún puede contar con algunos comuneros que le sirven\ldots{}
No pasará nada, respondo de que no pasará nada.

---Mi compañero y yo---dijo el retórico dispuesto a retirarse
definitivamente,---apreciamos la buena voluntad del Gobierno; creemos
que sus intenciones no pueden ser mejores; pero no podemos seguir
asintiendo en esta junta secreta a los actos de debilidad y a la
indeterminación que caracteriza a la política presente. En las Cortes
evitaremos todo lo posible la escisión, pero nuestra conciencia nos
impide continuar aquí. Está probado que la Sociedad a que hemos
pertenecido estorba toda política formal, y es un aliciente para las
ambiciones, para los disturbios populares, y aun para las sediciones del
ejército. Hace tiempo que deseamos la ruptura; hoy se nos presenta una
ocasión y la aprovechamos. Gobiernen ustedes en armonía misteriosa con
los manejos de la Corte, porque las dos políticas contrarias que bajo
tierra y en la oscuridad funcionan luchando, se acuerdan en una cosa, en
hacer polvo y ruinas de la grandeza y poderío del Reino. Inspiren
ustedes al Gobierno y a las Cortes, dominándoles por medio de la
amenazadora extensión de estas Sociedades, y haciéndose pagar su
protección con los destinos, las fajas, las mitras, las cruces que aquí
se reparten. Yo renuncio a los beneficios y a la responsabilidad de esta
labor oscura y funesta. Adiós, amigos míos; la diferencia de opiniones
no entibia la amistad de toda la vida, la amistad de Cádiz en los días
de gloria, la amistad del Peñón de la Gomera en los días terribles.
¡Quiera Dios que no volvamos a abrazarnos en los presidios de África!

Dicho esto se retiraron. ¡Ay! Desgraciadamente para España, en aquellos
hombres no había más que talento y honradez; el talento de pensar
discretamente y la honradez que consiste en no engañar a nadie.
Faltábales esa inspiración vigorosa de la voluntad, que es la potente
fuerza creadora de los grandes actos. Los que salían, a pesar de su
sensato hablar, eran tan niños como los que se quedaban en el
\emph{Grande Oriente}. Entre todos juntos y fundiéndolos a todos, a
pesar de la aptitud versificante y poética de algunos, no se habría
podido obtener el brazo izquierdo de un Bonaparte, ni de un Cisneros, ni
de un Washington, ni siquiera de un Cromwell o un Robespierre. ¡Extraña
ineptitud ocasionada por la servidumbre! En la uña del dedo meñique de
una mujer, Isabel la Católica, había más energía política, más potencia
gobernadora que en todos los poetas, economistas, oradores, periodistas,
abogados y retóricos españoles del siglo XIX.

¿Qué resolvió el \emph{Grande Oriente}, después de la escisión? Cosas
graves. Mudar algunos mandos militares, negar dos canonjías, recomendar
a los pueblos la elección de dos diputados masones, adjudicar tres
subastas, escribir las bases de una transacción contra los comuneros,
leer algunas cartas que hablaban de conspiración, enterarse de las
confidencias hechas por empleados de Palacio, subvencionar un periódico,
adjudicar trece destinos a otros tantos masones, dar una pensión a la
viuda de un perseguido \emph{por defender el Sistema}, echar tierra
sobre un expediente de contrabando, etc.

¿Cuál de las dos camarillas es más responsable ante la historia, la del
populacho o la de los hombres leídos? No es fácil contestar. La primera,
en medio de su barbarie, había resuelto algo en el asunto del día; la
segunda, a pesar de su ilustración, no había resuelto nada.

\hypertarget{xxiv}{%
\chapter*{XXIV}\label{xxiv}}
\addcontentsline{toc}{chapter}{XXIV}

Salvador conoció desde la noche del 3 al 4 el infame proyecto de sus
compañeros de caballería. Si no pudo injerirse en la camarilla, asistió
a la Fortaleza. Oía y callaba, esperando utilizar las circunstancias; y
como había adquirido y fomentado buenas relaciones con comuneros de
todas clases, creía seguro salir adelante con su buen propósito. El plan
de hacer justicia en la persona de Vinuesa le pareció irrealizable,
porque contaba con la energía de las autoridades. Sintió impulsos de
poner en conocimiento de Campos algunas preciosas noticias y datos
adquiridos en la Asamblea, para que aquel las comunicase al Gobierno;
pero su natural honrado y leal se sublevaba contra la delación.

En la mañana del 4 entró en la celda de Gil de la Cuadra, y le dijo:

---Ánimo, señor reo; esta noche saldremos de aquí. Tengo todo preparado.

El anciano, de rodillas, apoyando su cuerpo en el lecho, cruzó las manos
y se puso a rezar fervorosamente.

Poco después Salvador atravesaba el patio de la cárcel, cuando se sintió
llamar. A su lado vio una cara entre burlona y suspicaz, unos taimados
ojos verdosos que gatunamente le miraban, una mano blanca que con
suavidad le agarraba el brazo. Era el Sr.~Regato. Vestía el uniforme de
capitán de la Milicia.

---Amiguito---le dijo,---tenemos que echar un párrafo. Subamos.

Instaláronse solos en una pieza del piso alto, y D. José Manuel habló de
este modo:

---Tengo el corazón oprimido, amigo Salvador. Ya sabe usted que el
pueblo está furioso\ldots{} y con razón, con muchísima razón. El
Gobierno se empeña en perdonar a Vinuesa, regalándole más tarde una
mitra, y el pueblo, que después de todo es soberano, se empeña en que
\emph{Tamajón} debe ser ahorcado. ¿Qué tal? Aquí tiene usted dos reyes
que se desafían sobre el cuerpo de un pobre sacerdote.

---No creo posible que esos hombres feroces consigan su objeto\ldots{}
Tal ignominia no pasará en España. Lo espero así para honor de esta
Nación.

---¡Oh!, no conoce usted los arranques del pueblo español. La resolución
de los comuneros, nuestros amigos, es definitiva. Ya he tratado de
contenerles, porque no me gusta el derramamiento de sangre; pero me ha
sido imposible. Intentarán hacer justicia.

---Pero no lo conseguirán. El Gobierno es malo; pero está compuesto de
hombres honrados.

---El Gobierno se cruzará de brazos, amigo mío---dijo Regato, poniendo
gran interés en aquel diálogo.---He visto a Campos al amanecer y me ha
dicho que el \emph{Grande Oriente} reprueba la justiciada del pueblo,
pero que no hace nada.

---Dicen que se quitará la guardia de milicianos.

---Error; no se quitará guardia ninguna. El Gobierno arde en
sentimientos humanitarios; pero no quiere hacer frente al oleaje
popular, por temor de ser arrastrado. Teme que se le acuse de servil;
teme las murmuraciones y se ruboriza si le dicen que protege al
absolutismo.

El asombro no dejó hablar a Monsalud durante breve rato.

---Eso no puede ser---exclamó al fin pálido de ira.---¡Tal infamia no
cabe en corazones españoles!

---El Gobierno no hará nada. Quizás algunos de sus individuos se
aprestarían a la resistencia si supieran lo que va a pasar, pero no lo
saben. Los masones se lavan las manos como Pilatos; han cogido miedo a
la comunería. En verdad que somos temibles.

---Lo que usted me cuenta, Sr.~Regato---dijo Salvador levantándose con
inquietud,---aparece una pesadilla horrible. Según usted, es muy posible
que esa canalla abominable trate hoy de invadir este edificio, sin que
el Gobierno se lo impida.

---¡Es verdaderamente espantoso!---exclamó Regato afectando
sensibilidad;---pero me parece que podrá evitarse una desgracia\ldots{}
Compadezco con toda mi alma a ese pobre D. Matías. ¿No es verdad que es
una lástima que le maten así tan brutalmente?

---No; no puede ser. Esto se quedará en amenaza ridícula.

---Que no es amenaza ridícula digo\ldots---afirmó Regato acercando más
su asiento al de Monsalud y pasándole la mano por el hombro.---Mire
usted; a mí se me ha ocurrido que podríamos salvar al pobre arcediano.

---¿Cómo?\ldots---preguntó vivamente Monsalud con el interés que le
inspiraban siempre las buenas obras.

---Le asombrará a usted que me inspire lástima ese desgraciado. Yo soy
así, más liberal hoy que ayer, y mañana más que hoy; pero bien está la
sangre en las venas donde Dios la ha puesto, ¿eh?

Monsalud, recordando lo que había oído a Campos respecto al sospechoso
liberalismo de Regato y algunas noticias que él mismo había adquirido,
se explicó fácilmente la compasión del comunero.

---Yo no soy amigo suyo, ni lo fui nunca---prosiguió D. José Manuel
recogiéndose dentro de su reserva como el caracol en su casa.---Los
demonios le lleven. Lo que quiero decir es que pudiéndose evitar la
muerte de un semejante, debe evitarse.

---Parece difícil y sin embargo es sencillo. Cálmese el furor de la
canalla; póngase una buena guardia en el edificio, y todo está
concluido.

---Ninguna de esas dos cosas puede hacerse.

---Pues entonces\ldots{}

---Usted no carece de talento---dijo Regato sonriendo,---y sin embargo
no comprende mi idea. Siga aquí la guardia de milicianos\ldots{}
Supongamos que viene eso que usted llama populacho\ldots{}

---Y que los milicianos, recordando que son hombres de honor, españoles
y cristianos, defienden la entrada.

---No\ldots{} supongamos que no la defienden.

---Entonces entra la canalla.

---Eso es, entra\ldots{}

---Abre el calabozo.

---Abre el calabozo\ldots{} y no encuentra a Vinuesa.

---¡Ah!, ya\ldots{} que se escape\ldots{}

---O que se esconda.

---Pero sus enemigos le buscarán.

---Que le busquen. Con tal que no le encuentren\ldots{}

---Pero ya sabe usted que cuando la ferocidad popular pide una víctima,
si no se le da\ldots{}

---Sacrifica al primero que encuentra.

---Es posible que la falta de Vinuesa la pague otro preso quizás más
inocente que él\ldots{} No, no me conviene ese plan.

---¿Y qué nos importa que la falta de Vinuesa la pague otro?

Monsalud miró a Regato con tanta severidad, que el dos veces gato
entornó sus párpados para mirar al suelo.

---¡Ah!, ya comprendo---dijo afectando buen humor.---Usted no quiere que
le toquen a su Gil de la Cuadra, que es, entre paréntesis, el más malo
de todos y el que merecería cualquier castigo.

---Es verdad que le protejo---dijo Salvador.

---Como que se ha metido usted en esta inmundicia sólo por salvarle.

---También es verdad.

---Como que fue usted conmigo a los comuneros sólo con el fin de hacerse
amigos entre la gente exaltada.

---También es cierto. Ese conocimiento tan hábil de mi conducta y de mis
intenciones me mueve a declarar que poseo del mismo modo parte de los
secretos de una persona a quien yo conozco.

---Con tal que no se refiera usted a las infames calumnias que dicen
contra mí los masones\ldots{}

---Yo no me refiero a calumnias. Usted ha desempeñado su misión
incitando al pueblo a lanzarse en una vía de atrocidades sangrientas.

---Calumnia.

---Usted cumple también su misión, procurando que después del atentado
quede vivo el arcediano; y con tal que el pueblo consume su bestial
proyecto y tenga una víctima\ldots{} poco importa lo demás.

---Yo no quiero que haya víctimas---dijo Regato comprendiendo que era
mejor hablar con franqueza.---Lo que quiero es que Vinuesa no corra
peligro, y que si ha de haber sacrificio, recaiga en la cabeza de
algunos de tantos pillos como llenan esta cárcel y la de Villa. Contaba
con eso y cuento todavía.

---¿Y qué papel debo yo desempeñar en esto?---preguntó Monsalud con
cierta perplejidad.---Porque usted me habla en el tono del que solicita
ayuda.

---Exactamente. El alcaide de la cárcel es hombre con quien no se puede
contar. Usted que ha venido aquí por una intriga; usted que ha venido
aquí con el exclusivo objeto de salvar a un hombre, es quien puede hacer
esta buena obra.

---¿Cómo?---preguntó el joven deseando saber hasta dónde iba el
diabólico entendimiento del agente secreto de Su Majestad.

---Aprovechando la borrachera que tomará hoy al medio día, según su
santa costumbre, el Sr.~Alcaide\ldots{}

---¿Para poner en libertad a Vinuesa?

---Eso no puede ser, porque los milicianos no lo permitirían. Soy listo
y comprendo que si fuera posible este modo de escapar, ya lo habría
usted intentado en favor de Gil.

---Seguramente.

---Lo que yo quiero es que mude usted a Vinuesa de calabozo.

---Le buscarán.

---No le buscarán, si se pone otro en su lugar.

---Eso es entregar un hombre a los asesinos.

Regato no supo qué contestar. Estaba impaciente y nervioso, y agitábase
en su silla tomando diferentes posiciones a cada minuto.

---Hombre de Dios---gritó al fin.---Me sorprenden esos escrúpulos. ¿No
hay en la cárcel un Barrabás? Que muera Barrabás y que se salve Jesús.
Concedo con muchísimo gusto que Gil de la Cuadra no sea el sustituto.

---Esa farsa infame no es propia de mí---contestó el joven,---si el
populacho quiere una víctima, no seré yo quien fríamente se la entregue,
como el leonero que escoge la res más gorda para darla a las fieras con
que se gana la vida.

---Sr.~D. Rígido---dijo Regato sin poder disimular su enfado,---maldito
si le sientan a usted esos humos de juez severo. ¿A qué tanta nimiedad y
sutileza de abogado para un asunto tan sencillo? Usted ha empleado toda
clase de recursos para sacar de aquí al que con más justicia está preso.

---Usted juzga mal a mi amigo---repuso Monsalud con serenidad,---y es
extraño porque le conoce bien. No aparece complicado más que por unas
cartas que se hallaron entre los papeles de Vinuesa, y el juez debe de
haber comprendido que apenas merece castigo, pues sólo le condena a
cuatro años de presidio, pena relativamente leve en estos tiempos.

---Nada de eso hace al caso---dijo Regato como hombre afanado que se
decide a marchar derechamente hacia su objeto.---Usted creerá tal vez
que yo no correspondería a su buena voluntad con otra buena voluntad, a
su beneficio con otro beneficio.

Diciendo esto, el dos veces gato se llevó la mano a un cinto, y
desliándolo hizo sonar su contenido, un metal precioso que hace
enloquecer a los hombres. Monsalud sintió un impulso de ira y crispando
los dedos miró el cuello del agente de Su Majestad. Pero la razón no le
abandonaba, y calculó que era muy prudente contenerse para imaginar
algún ardid que sin comprometerle, le librara de las enfadosas
sugestiones de aquel hombre.

---Guarde usted su dinero, Sr.~Regato---dijo con serenidad.---Yo no soy
Pelumbres.

Regato no dijo nada y puso el cinto sobre la mesa.

---Este soberbio no cede con cualquier bicoca---pensó.---Será preciso
hacer un sacrificio, un verdadero sacrificio.

---Yo creí---indicó Salvador disimulando su ira con una apariencia
festiva,---que ya no le quedaban a usted más ochentines de los que el
Gobierno dio a la Casa Real.

---Son onzas de oro---dijo Regato con naturalidad.---Ya sé que usted me
dirá mil lindezas y pedanterías. No parece sino que es un crimen aceptar
obsequios en pago de un servicio leal. Bueno, señor mío, usted se lo
pierde. Viva usted de sus rentas, viva de sus fincas, ya que donosamente
rechaza lo que le cae\ldots{}

Levantose en seguida y dando varios pasos en diferente sentido, se
detuvo ante el joven, le puso la mano en la cabeza y se la movió con
gesto entre cariñoso y amonestador.

---Y si no---añadió,---no hay nada de lo dicho. Por eso no hemos de
reñir. Cada uno tiene su conciencia como se la hizo Dios. Hay escrúpulos
respetables. Yo no censuro que haya personas así\ldots{} tan atiesadas.
Lo que siento es que se va usted a ver en un mal paso, caballerito. Si
yo le he propuesto lo que ha oído, es por encargo de varios amigos, y
ellos no son como yo, mansos y pacíficos y que con todo se conforman,
sino muy fieros y vengativos. Capaces son de darle un disgusto a mi
señor D. Rígido\ldots{} ¿Qué cree usted?---prosiguió poniéndosele
delante y clavando en él sus ojos cuya pupila brillaba con dorados y
verdes reflejos.---Ya anoche estaban mis amigos muy incomodados con
usted, llamábanle traidor por haber aceptado un destino de esa canalla
masónica.

Monsalud seguía meditando.

---Y en rigor\ldots---añadió el agente de Su Majestad,---la conducta de
usted no ha podido ser más sospechosa. Anoche tuve que platicar mucho
para defenderle a usted\ldots{} «Es un traidor», decían. «Pues si no nos
sirve en su destino de carcelero, haciendo lo que le mandemos, lo pasará
mal\ldots» En fin, como son unos bárbaros, no es de extrañar que digan
barbaridades. Yo me miraría muy bien antes de enemistarme con ellos.

El otro seguía meditando.

---Yo se lo digo a usted con franqueza---continuó Regato animándose al
ver la perplejidad del joven,---porque somos amigos, porque tengo
particulares simpatías con usted, conociendo como conozco sus méritos,
su buen corazón y mucho entendimiento. Tenga usted muy presente mi
advertencia, pero muy presente. Si se resiste a ayudarme, no salga usted
solo por las noches, ni vuelva a poner los pies en la Asamblea ni en
sitio alguno donde nos reunamos. Además, los antecedentes políticos de
usted no son tales que pueda el caballerito estar tranquilo, si alguien
se propone hacerle daño.

---No creo tener enemigos---dijo casi maquinalmente el joven.

---Téngalos o no, usted es un hombre que no ha dejado de cometer errores
en su vida.

Salvador le miró con tristeza.

---Y entre ellos se cuenta---continuó Regato,---el haber tenido
relaciones con Amézaga, el poseedor de los secretos del Rey en Valencey.

---¡Yo!\ldots---dijo Monsalud lleno de estupor.

---No me lo negará usted a mí. Amézaga, que se cortó el pescuezo con una
navaja de afeitar antes que se lo retorciera el verdugo, concluyó como
debía concluir. Usted que le ayudó en la publicidad de los célebres
secretos, no fue objeto de persecuciones ni aun de sospechas, porque
supo esconderse; pero ¡ay, insigne joven!, usted no podrá librarse de
una causa el día en que cualquier mal intencionado quiera hacerle
daño\ldots{} Usted tuvo correspondencia con Amézaga\ldots{}

La cara atónita de Monsalud estaba diciendo:---Es verdad.

---Amézaga le escribió a usted varias cartas que le comprometen, pero de
una manera\ldots{} La causa está abierta. Ya sabemos que este es uno de
los asuntos en que Su Majestad no perdona. Se trata de sus chicoleos en
Valencey, de sus diabluras con los Bonapartes\ldots{} en fin, esto es
grave, y no hay Gobierno, por patriotero que sea, que no apoye a nuestro
Rey.

---Eso es historia antigua---dijo Salvador con desdén.

---Antigua, sí; yo no he visto las cartas de Amézaga dándole
instrucciones a usted y a otros conspiradores para publicar las
aventurillas de Su Majestad; pero el amigo mío que las posee, me ha
dicho que son terribles. Con la mitad de aquello se sube al cadalso en
todos tiempos.

Salvador sentía viva agitación.

---En el año 19, usted conspiraba; usted se vio obligado a esconderse
hoy aquí, mañana allí, para burlar a la policía. En una de estas
mudanzas un amigo mío se apoderó de un paquete de cartas que tenía mi
Sr.~D. Salvador en la gaveta de su mesa. Según me ha dicho, las había
políticas, amorosas, familiares, de todas clases.

---Es verdad que perdí unas cartas; ¿pero qué\ldots?

---Que el poseedor de ellas las guarda como oro en paño. Ni siquiera a
mí me las ha querido mostrar. ¿Sabe usted quién es? Alonso Sánchez, que
fue de la policía y ahora está cesante y como cesante desesperado. Posee
una admirable colección de papeles curiosos\ldots{} Es amigo mío, muy
amigo mío.

Monsalud no contestó. Regato, al decir lo que antecede, apretó el brazo
contra su cuerpo, complaciéndose en sentir bajo el uniforme el contacto
de un cuerpo semejante en tamaño y dureza a un paquete de papeles. Había
mentido como un bellaco. Las cartas firmadas por Amézaga y dirigidas a
Monsalud en Julio del 14 las tenía él, juntamente con otras de dudoso
valor político por ser esquelas de amores o de familia. Habíalas
recibido del agente de policía y las guardaba, como otros muchos tesoros
epistolares, esperando que llegase la ocasión de utilizarlas. El astuto
intrigante daba gran importancia a todo papel que en su mano por
cualquier evento caía, y los tenía clasificados por autores con una
escrupulosidad cariñosa, semejante al celo de los anticuarios y
bibliófilos.

Aquella mañana antes de dirigirse a las cárceles de la Corona, abrió una
arqueta que encerraba numerosos paquetes, parecidos a expedientes, y
después de recorrerlos brevemente con la vista, sacó uno que decía:
\emph{Amézaga}, \emph{Salvador Monsalud}. Guardolo en un profundo
bolsillo interior con que había dotado a su casaca de miliciano, para
que el uniforme, según decía festivamente, no fuera prenda inútil.

---Sr.~Regato---dijo Monsalud.---Todo eso de los papeles de Amézaga me
tiene sin cuidado en lo referente a lo que usted me propone hoy. Pero me
gustaría recobrarlos, ¿por qué he de decir otra cosa?

---¡Bribón!---dijo Regato para sí, oprimiendo dulcemente el bulto de
papel.---Como no cedas ni a las onzas, ni a las amenazas, te venceré con
esto.

\hypertarget{xxv}{%
\chapter*{XXV}\label{xxv}}
\addcontentsline{toc}{chapter}{XXV}

Ninguna importancia dio Monsalud a tal incidente. Fijábase ante todo en
la amenaza de concitar contra él el odio de los Pelumbres y comparsa.
Esto le pareció un verdadero percance, porque Regato en tal especie de
guerra era omnipotente. Considerando la maldad de aquel hombre, vio un
peligro real y cercano, comprendió que no eran palabras vanas las
referentes a la brutalidad vengativa de los amigos del agente de Su
Majestad. Su mente se llenó súbitamente de las ideas evocadas por el
peligro, y pensó en los medios de librarse del que con una mano ofrecía
oro y con otra porrazos.

---Este tunante---pensó Monsalud,---no me perdonará. No soy quien soy,
si dejo a este reptil en disposición de morderme.

Cuando esta idea cruzó por su mente, tuvo otra felicísima: seguir
aparentando perplejidad para que Regato le creyese inclinado a una
inteligencia.

---Mucho lo piensa---dijo para sí D. José Manuel.---Su indecisión es
buena señal. No se enfurece, no grita, no dice una palabra de su honor.
Sacaré el dinero para que viéndole\ldots{} pues\ldots{}

---Déjeme usted pensar un rato lo que debo hacer---dejo Monsalud.

Conservando una seriedad ficticia, Regato empezó a contar dinero sobre
la mesa.

---No se trata de ningún desafuero---dijo,---sino de un servicio. Mi
objeto sólo es que Vinuesa no muera, y que la irritación del pueblo pase
sobre él como pasan las olas por encima de una roca sin conmoverla. Si
el pueblo registra demasiado los calabozos y quiere hacer alguna
atrocidad en cabeza absolutista, lo más acertado me parece sacar a
Vinuesa de su encierro, esconderle en las bohardillas\ldots{} y nada
más. El Alcaide es un borracho y un fanático. No me atrevo a hablarle
porque estamos reñidos desde hace tiempo. Ni él me traga a mí ni yo a
él, ¿entiende usted? Va para un año que no pongo los pies en esta casa y
no conozco a nadie en ella. Pero usted puede hacerlo todo. Los
milicianos que están de guardia no es fácil que se enteren.

---¡Oh!, sí, es muy fácil---dijo Monsalud.

---Pide mucho---pensó Regato,---habrá que hacer un sacrificio mayor.

¡Ah!, tunante---pensó Monsalud mirándole fijamente pero sin dejar
conocer su idea;---tú has creído jugar conmigo, y yo, aunque no soy
agente de Su Majestad, ni dispongo de fuerza alguna, ni de grandes
caudales, te voy a sentar la mano de tal modo que has de acordarte de mí
toda tu vida.

La sonrisa del triunfo presente o anunciado por el corazón alteró el
semblante pálido y serio de Salvador; pero Regato, sin advertir nada,
continuaba manoseando las peluconas.

---Te juro, miserable---prosiguió Monsalud, pensándolo,---que el lazo
que voy a armarte y en el cual vas a caer como un pajarillo inocente, se
deja atrás a tus diabólicos ardides. Cuenta, cuenta dinerito.

---¿Lo ha pensado usted?---preguntó Regato.

---Hombre, sí que lo he pensado\ldots{} ¡Qué demonios! Este es un país
donde las personas honradas no pueden conservar su honradez. No hay
medio de vivir; todo cuesta un ojo de la cara.

---Tiene apuros\ldots---pensó Regato.---Cayó. La historia de siempre.

---Por el momento---dijo Salvador,---guarde usted ese dinero. Puede
pasar alguien, oír su seductor sonido y entonces\ldots{} las
sospechas\ldots{}

---Está bien, muy bien---manifestó el comunero miliciano encerrando las
onzas en el cinto.

---Y ahora discurramos lo que se ha de hacer.

---Es muy sencillo, sacarle del calabozo sin que lo vea nadie, y subirle
a las bohardillas. Salga usted a ver si ya el Sr.~Alcaide está durmiendo
la mona. A los demás empleados de la cárcel se les puede dar
algo\ldots{} Eso a juicio de usted.

Monsalud empezó a dar paseos por la habitación. El plan que rápidamente
había concebido para dar una severa lección y un castigo muy duro al
agente presentósele muy difícil de realizar.

---Atarle aquí, ponerle una mordaza y subirle a las
bohardillas---pensó,---es muy aventurado. Gritará\ldots{} Da la maldita
casualidad de que no hay un solo calabozo vacío. ¿Pero no habrá algún
calabozo vacío?\ldots{} El 17 se ocupó ayer\ldots{} el 14 no se
desocupará hasta mañana.

Siguió meditando.

---No debe perderse el tiempo---dijo súbitamente Regato.---Entremos
ambos en el encierro de Vinuesa. Son las tres y media. El Alcaide duerme
la siesta. Hable usted con los calaboceros que puedan estorbar. Los
milicianos están en el cuerpo de guardia, y si hay algunos en el patio,
se les convidará a todos a café. Mande usted traer copas y café,
diciéndoles que es hoy su cumpleaños.

Monsalud se echó a reír.

---No está mal cumpleaños el que a ti te espera---pensó.

Ya tenía un nuevo plan.

---Espéreme usted aquí---dijo.---Voy a dar una vuelta por la cárcel.
Veré si duerme el Alcaide, diré dos palabras a los calaboceros, aunque
se me figura que no serán necesarias tantas precauciones. La prisión de
Vinuesa está bajo la escalera, y no será preciso pasarle por el patio,
¿entiende usted?

---Entiendo\ldots{} ¡Oh!, las cosas se presentan bien---dijo
Regato.---En fin, vaya usted\ldots{} No olvidarse de las copas. Con los
milicianos no se puede contar sino engañándoles, lo cual es facilísimo.
Dígales usted que se han recibido noticias de que viene Riego con su
ejército, con veinte ejércitos como los de Jerjes, a conquistar Madrid.
Yo no bajo, porque se me pegarían, no dejándome respirar.

Monsalud salió de la pieza, recorrió la cárcel, habló brevemente con el
Alcaide que en aquel momento se disponía a dormir la siesta. Este,
recomendándole mucha vigilancia, le dijo:

---Me parece que no tendremos la jarana que se anunció. Alarmas, alarmas
de los desocupados. No se ha visto ahora un solo grupo sospechoso en
toda la calle, y me parece que tendremos un día tranquilo. Además, la
Milicia no toleraría ningún desmán. Está decidida a que nadie traspase
el umbral de la cárcel.

Pasado algún tiempo después que el Alcaide se encerró en su cuarto,
Salvador convidó a los milicianos, siguiendo las advertencias de su
sobornador, y dio luego varias órdenes a los dos calaboceros que estaban
a la sazón en la casa, enviándoles a puntos de donde no pudiesen volver
antes de un cuarto de hora. Con estas ligeras precauciones había
seguridad completa, como se verá ahora mismo.

Bajo la escalera de la cárcel, en el oscuro hueco que formaba el primer
tramo, había una puerta pequeña y poco visible. Era la puerta del
calabozo en que estaba Gil de la Cuadra. Aquella prisión era la única en
la cual se podía entrar sin atravesar el patio y las crujías bajas del
edificio. Monsalud tomó un pedazo de tiza, y en la puertecilla dibujó
groseramente una horca con su correspondiente ahorcado, cuidando de
poner debajo \emph{Tamajón}. En seguida subió: de un cuarto oscuro
destinado a trastos sacó dos objetos que guardó cuidadosamente,
dirigiéndose al punto en busca de Regato. Pocos momentos después ambos
estaban frente a la puerta del calabozo.

---¿Con que aquí está ese desgraciado?---dijo el agente de Su
Majestad.---Sí, ya veo la célebre horca y los letreros.

Monsalud abrió, y entraron. Al principio la oscuridad no les permitió
ver objeto alguno.

---Sr.~D. Matías---dijo Regato adelantando en las tinieblas.

---¿Quién es?---murmuró Gil de la Cuadra.

---Sr.~Vinuesa\ldots{}

Monsalud cerró por dentro.

Pasó un rato antes de que el agente conociese el engaño.

---¿Qué es esto?---gritó.---Engaño, traición\ldots{} ¡Salvador!

---Engaño, traición---repitió este.

---Infame, abre pronto, o te ahogo---exclamó el gato, ciego de ira y
amenazando con las crispadas zarpas el cuello del joven. Haciendo un
movimiento rápido, echó mano a la espada.

Monsalud levantó el brazo derecho y descargó sobre el agente una
bofetada olímpica, una de esas bofetadas supremas y decisivas, que
recuerdan la quijada de asno de que se servía Sansón. Regato cayó al
suelo. En pocos segundos Salvador le amordazó.

---Ahora---le dijo,---desnúdate\ldots{} ¡pronto!

Nunca el agente se había parecido tanto a un gato. Arañó al joven, y
falto de habla, bufaba sordamente.

---Desnúdate pronto, o te aplasto, reptil. Necesito tu uniforme de
miliciano.

Gil de la Cuadra miraba con estupor aquella escena.

---Necesito tu uniforme.

Monsalud tiraba de las mangas, desabrochaba los botones. En poco tiempo
el morrión, los pantalones, la casaca y la espada de Regato, fueron
arrojados al rincón opuesto. Inmediatamente el joven sacó una larga
cuerda y con mucho trabajo, porque el gato se defendía rabiosamente, le
ató con tal fuerza que no podía moverse. Las argollas que había en la
pared de la prisión sirvieron para sujetar al nuevo preso, que hubo de
quedar adherido, clavado al muro como un murciélago.

---Sr.~Gil---dijo Monsalud imperiosamente,---póngase usted ese vestido
de miliciano. Pronto será de noche. ¡A la calle!

Gil de la Cuadra no apartaba los ojos del triste espectáculo que tenía
delante.

---Pronto\ldots{} ¡el uniforme!---repitió Monsalud.---Saldrá usted ahora
y le ocultaré en mi cuarto hasta que sea de noche\ldots{} Pronto.

Gil de la Cuadra obedeció, y en silencio empezó a vestirse.

Hubo una pausa de silencio profundo. Pero luego sintiose un rumor que
crecía, crecía, y de rumor se trocó en mugido sordo, confusas palabras
de gente, gritos, pasos, puertas que se cerraban. Sonaron varios tiros.

Monsalud, después de asegurar con toda su fuerza la cuerda que ataba a
Regato, salió lleno de zozobra del encierro.

\hypertarget{xxvi}{%
\chapter*{XXVI}\label{xxvi}}
\addcontentsline{toc}{chapter}{XXVI}

Poco después del medio día una horda de caníbales se reunía en la Puerta
del Sol, mejor dicho, se diseminaba, marchándose cada animal por su
lado, después de acordar juntarse por la tarde en el mismo sitio. Así lo
hicieron, y las autoridades miraban aquello como se mira una fiesta.
Después de las cuatro los grupos volvieron a invadir la Puerta del Sol.
Había en ellos una frialdad solemne y lúgubre, como de quien no fía nada
al acaso ni a la pasión, sino al cálculo y a la consigna. La autoridad
seguía no viendo nada, o negligente o cómplice o imbécil que las tres
cosas pueden ser. Los grupos susurraban, y por un momento vacilaron;
pero al cabo de cierto tiempo dirigiéronse por la calle de Carretas y
las de Barrionuevo y la Merced, a la cárcel de la Corona. Llenose la
calle de la Cabeza en su mayor parte. Destacábase al frente de uno de
los grupos el ciudadano Pelumbres, arengando como una bestia que hubiese
aprendido durante corto tiempo y por arte milagroso, el lenguaje de los
hombres. Casi todos llevaban armas menos él.

Considerando que su persona no estaba completa, pidió una navaja; mas
como nadie se hallase dispuesto a tal generosidad, dirigió su mirada de
buitre a todas partes. Hacia la calle de San Pedro Mártir estaban
construyendo una casa. Pelumbres se acercó a la empalizada; vio algunas
piedras de granito a medio labrar y encima de ellas un gran martillo.

---Para el sastre la aguja---dijo,---la lezna para el zapatero; el
cuerno, para el toro, y para el herrero el martillo.

Cuando se dirigió con su arma al hombro a la esquina de la calle de
Lavapiés, sus compañeros rompían a hachazos la puerta de la cárcel. Los
milicianos, no queriendo sostener una lucha contraria, según su
criterio, al progreso, ni tampoco entregarse sin resistencia, habían
asegurado la puerta con un solo cerrojo, y en el zaguán se disponían
intrépidos a descargar sus armas\ldots{} al aire.

La puerta no se resistió mucho. Lo que empezaron los hachazos, dos
docenas de coces lo concluyeron. Disparáronse al aire varios fusiles de
milicianos, la turba penetró en el patio de la cárcel, rápida como un
brazo de agua, rugiente y soez. Hay un grado de ferocidad que la
Naturaleza no presenta en ninguna especie de animales; sólo se ve en el
hombre, único ser capaz de reunir a la barbarie del hecho las ignominias
y brutalidades de la palabra. Viendo a los hombres en ciertas ocasiones
de delirio, no se puede menos de considerar a la hiena como un animal
caritativo.

El calabozo de Vinuesa era bastante conocido de casi todos los que
entraron. Cómo lo abrieron no se sabe. La turba que en la calle era
gruesa, se afiló para entrar en la cárcel. Para penetrar por una
puertecilla estrecha tuvo que aguzarse más. Parecía una serpiente de
largo cuerpo y cabeza estrecha, introduciendo su boca por una hendidura.
El cuerpo se agrandaba en el patio; enroscándose salía a la calle, daba
varias vueltas por las inmediatas, y la cola, parte en extremo sensible
y movible, culebreaba en la plazoleta de Relatores. La cola se componía
de mujeres. Cuando Vinuesa vio que entraban en su calabozo aquellos
hombres terribles, comprendió que su fin era inminente. Poniéndose de
rodillas y cruzando las manos, gritó:

---¡Perdón, perdón!

El calabozo retumbaba con las imprecaciones. Viose en el aire un círculo
rápido y espantoso trazado por un pedazo de hierro adherido al extremo
de un palo, que blandían manos vigorosas. El martillo describió primero
un círculo en vano, después otro\ldots{} y la cabeza del infeliz reo
recibió el mortal golpe. Siguiole otro no menos fuerte y después diez
navajas se cebaron en el cuerpo palpitante.

~

Lavaban los asesinos el martillo en la fuente de la calle de Relatores,
cuando el Gobierno resolvió desplegar la mayor energía. ¡Qué sería de
esta Nación si la Providencia no le deparase en ocasiones críticas el
tutelar beneficio de su Gobierno! La noticia del crimen corrió por
Madrid, y la villa, que es y ha sido siempre una villa honrada, se
estremeció de espanto y piedad. El Gobierno se estremecía también, y
declaraba con patriótico celo que no descansaría hasta castigar a los
culpables. Para que nadie tuviera duda de su gran entendimiento y
perspicacia política, mandó que inmediatamente se pusiera fuerza del
ejército en el edificio, y por si alguien tenía dudas todavía de su
diligente y paternal actividad, ordenó que al instante, sin pérdida de
un momento, \emph{se instruyesen las oportunas diligencias}. Quejarse de
un Gobierno así es quejarse de vicio.

\hypertarget{xxvii}{%
\chapter*{XXVII}\label{xxvii}}
\addcontentsline{toc}{chapter}{XXVII}

Cuando Gil de la Cuadra y Regato se quedaron solos, siguieron oyendo
aquel rumor de voces que resonaba en el patio de la cárcel. Durante más
de un cuarto de hora el estrépito fue grande. Gil de la Cuadra,
comprendiendo que el populacho había invadido el edificio, se puso de
rodillas, y cruzando las manos, rezó en voz alta.

El otro desgraciado se hinchaba y gruñía. De su rostro congestionado
afluía copioso sudor. Trataba de romper sus ligaduras y de escupir su
mordaza; pero unas y otra habían sido puestas por buena mano. Por
último, después de repetidos esfuerzos, de su boca pudo salir una voz,
más que voz, silbido, que decía:---¡Piedad, piedad!

Gil de la Cuadra se acercó a él y limpiole el sudor de la frente. Las
miradas de Regato eran tan expresivas pidiendo compasión; las
contracciones de su cara tan violentas, que el primer preso no pudo
resistir el estímulo de sus sentimientos compasivos, y le quitó la
mordaza.

---¡Ah\ldots{} gracias, gracias!---exclamó el agente de Su Majestad,
aspirando con delicia el aire fétido de la prisión.---Aire, aire\ldots{}
me ahogo aquí.

---Pero con esto concluyen mis complacencias---dijo Cuadra.---No le
quitaré a usted la cuerda; eso no.

---Toque usted mi cintura---murmuró Regato.---¿Qué suena en ese cinto?
Dinero. Todo eso y la libertad\ldots{} pero suélteme usted.

---No puedo.

---¡Y el populacho ha entrado en la cárcel! ¿Ha sentido usted, Sr.~Gil?

---Sí, me pareció que entraba en el patio una ola del mar\ldots{} Ahora
parece que ha cesado el rumor. Se alejan.

---Se alejan, sí. Pero aún se sienten voces. Ese malvado volverá a
entrar aquí\ldots{} ¡Favor, pueblo!\ldots{} ¡Pueblo mío, favor!

Los gritos de Regato no traspasaban los muros de la prisión.

---Sr.~Gil---exclamó con acento de desesperación:---saque usted mi
espada y máteme. Un hombre de mi temple no puede soportar este suplicio.

---Calma, calma, Sr.~D. José Manuel---dijo Cuadra poniendo la mano sobre
la cabeza del agente.---Yo suplicaré a mi amigo que no le haga a usted
daño alguno\ldots{} Pero tarda, tarda.

---¡Su amigo!, ¿pues no tiene la vileza de llamarle su amigo?---dijo
Regato poniéndose tan encendido como cuando tenía la mordaza.

---Mi amigo, mi protector, mi salvador\ldots{} pues si él no existiera,
¿qué sería de mí?\ldots{} pero tarda, ¿no es verdad que tarda?

---¡Estúpido viejo!---gritó Regato fuera de sí,---ten vergüenza, y
córtate la mano antes que estrechar con ella la mano de ese
hombre\ldots{}

---¡Yo!\ldots{} En mi corazón no existe ya ni puede existir el odio. Y
si existiera, para ese joven no tendría sino amor, una admiración
respetuosa, un afecto paternal.

---Es verdad que hay cariños muy singulares---dijo Regato sonriendo con
infernal malicia.---Yo conocí a un sujeto que sacaba a paseo, llevándole
a cuestas, al cortejo de su mujer.

Gil de la Cuadra creyó que Regato sufría enajenación mental. Lleno de
compasión se acercó a él.

---Vendrá pronto---le dijo.---Yo intercederé por usted\ldots{} pero
tarda, ¿no es verdad que tarda? Ahora apenas se oye ruido.

---Intercederá usted---añadió Regato con afán de perversidad.---Y si le
pide algo en cambio, le dará usted su mujer\ldots{} no, porque murió;
pero aún tiene usted una hija. Sin embargo, como él la tiene en su casa,
se habrá cobrado por adelantado.

---Sr.~Regato---dijo Cuadra con severidad.---El lenguaje de usted es
propio de un loco.

---¡Imbécil, imbécil!, el de usted es propio de un ciego\ldots{} ¡Pobre
doña Pepita! Era una excelente señora, y tan guapa\ldots{} seguramente
si no hubiera dado con un esposo tan crédulo como usted\ldots{}

---Sr.~Regato---exclamó Cuadra con enojo.---Le digo a usted que se
calle.

---No digo más sino que aquella señora era una buena pieza.

---La desastrosa situación de usted me impide contestar a esa insolencia
como se merece.

---¿De veras cree usted que la hermosa dama era un modelo de virtudes?

---Sí, canalla, sí lo creo---gritó trémulo de ira Gil de la Cuadra,
llevando su vacilante mano a la espada.

---Pues mis noticias son que pecó varias veces. Dígalo Salvador Monsalud
que fue su cortejo\ldots{} ¡Oh, Dios mío! Estoy preso, estoy
atado\ldots{} pero en mi horrible situación me das armas; me das este
veneno que escupo y con el cual mato.

---¡Miserable!\ldots{}

Gil de la Cuadra corrió hacia él y le oprimió el cuello.

---Ahógame, necio---gruñó Regato,---ahógame. Mi último suspiro será para
echarte en cara tu vilipendio. Ese hombre, ese amigo mío\ldots{}

---¡Qué dices!\ldots{}

---Te burló, te burló. En Francia, todos los españoles lo sabían menos
tú\ldots{}

Gil de la Cuadra vacilaba. Una idea cruzó como un relámpago por su
cerebro; una idea confusamente mezclada con recuerdos, palabras,
coincidencias, detalles.

---El majadero no lo cree---dijo Regato, ya libre de las manos que le
apretaban el cuello.---Voy a darle pruebas para que calle.

---¡Pruebas! Usted está loco. Cállese usted. Esto es una farsa\ldots{}
¡Pero ese hombre no viene, Santo Dios!

---Pruebas, sí. Ponga usted la mano sobre el costado derecho, en la
pechera del uniforme mío que tiene puesto. ¿Qué hay en ese bolsillo?

---Un bulto, una cartera.

---Un paquete. Sáquelo usted.

---Ya está. Cartas\ldots{}

---Lea usted\ldots{}

---¿Qué esto? Una carta firmada \emph{Amézaga}.

---Siga usted, hojee usted ese precioso libro. Tras esa joya vendrá
otra.

Gil de la Cuadra, acercándose al ventanillo por donde entraba una débil
luz, recorría una tras otra y con ardiente curiosidad las cartas.

---A prisa, a prisa. Pase usted todas las primeras. ¿Qué viene ahora?

---Una lista con varios nombres.

---Adelante\ldots{} ¿Y ahora?

---Una\ldots{}

Gil de la Cuadra calló de improviso. El corazón saltole en el pecho.
Quedose frío, mudo, atónito, lleno de espanto, como el que se ve en el
borde del abismo y comprende en veloz juicio que no hay más remedio que
caer.

---¡Ah!---dijo Regato.---El imbécil ha puesto al fin la mano sobre el
delito de su esposa. Es tan bruto que necesita tocarlo para
comprenderlo.

Gil de la Cuadra seguía leyendo.

---¿Qué dice la carta?---añadió el agente.---Tras esa vienen otras
muchas. Yo he pasado buenos ratos leyéndolas. ¡Cómo palpita en ellas la
pasión! ¡Qué vehemente ardor!\ldots{} Y los dos amantes disimulaban
bien\ldots{} ¡Cuántas precauciones para engañar al bobillo! ¡Se
encuentran en esas cartas traiciones inauditas, alevosías de él y de
ella! La señora parecía más apasionada que\ldots{} nuestro amigo.

Gil de la Cuadra seguía leyendo. De repente se desplomó. Un ay de dolor,
una exclamación aguda y penetrante, parecida a las que exhalan los que
sufren repentina muerte, salió de sus labios. Cayó al suelo. Su mano
estrujaba un papel.

---El incrédulo parece convencido\ldots{} ¡Miserable viejo, ahí tienes a
tu Providencia, ahí tienes a tu Salvador, ahí tienes a tu amigo
querido!\ldots{} ¡Le has entregado a tu hija!

Cuando esta última palabra resonó en la prisión, estremeciose el cuerpo
del anciano herido en su alma. Irguiendo la cabeza, abrió los ojos,
diose furibundo golpe en la frente con la palma de la mano, y repitió:

---¡Mi hija!

Un instante después Gil de la Cuadra estaba sentado en el suelo con los
ojos fijos, el cuerpo encorvado, los labios entreabiertos, atónito,
lelo\ldots{}

Abriose la puerta. Monsalud entró.

\hypertarget{xxviii}{%
\chapter*{XXVIII}\label{xxviii}}
\addcontentsline{toc}{chapter}{XXVIII}

---Vamos, Sr.~Gil---dijo.---Vamos al punto.

Nadie contestó. El joven aguardó un instante. Traía una luz.

---¡Ah!---exclamó viendo que Regato continuaba en su sitio.---Pasará
usted aquí la noche, hasta que haya un alma compasiva que le saque. Han
asesinado a Vinuesa. Dicen que habrá esta noche nueva visita a los
calabozos.

Regato no contestó nada. Monsalud se dirigió a Gil de la Cuadra.

---Vamos---le dijo.---¿Por qué se arroja usted al suelo en el momento de
salir?

Extendió el brazo para alzarle; pero el anciano, rechazándolo con
fuerza. Él solo se levantó.

---Vamos fuera---repitió Monsalud.---Llegó el momento\ldots{}
¡libertad!\ldots{}

---De ti, de tu mano---exclamó Gil de la Cuadra con profunda ira,---no
la quiero.

Salvador, estupefacto y espantado, no supo qué decir.

---Vamos---exclamó al fin.

---No quiero.

---Salgamos.

---¡Contigo jamás!

---¿Qué dice usted?\ldots{} amigo\ldots{} por favor.

---¡Miserable, apártate de mí!---gritó Cuadra dirigiendo a su libertador
una mirada en que se reconcentraba todo el desprecio de que es capaz un
alma.---Me manchas, me ofendes, me repugnas.

---¡Qué locura! Vamos pronto---dijo Salvador tomándole por un
brazo.---Piense usted en su hija que espera.

---¡Mi hija, mi pobre Solita!---exclamó el anciano cubriendo con ambas
manos su rostro.

Este recuerdo, estas ideas produjeron conmoción profunda en su ánimo.

De súbito el instinto de libertad surgió poderoso en su alma. Corrió
hacia la puerta y salió. Monsalud fue tras él.

---Déjame, no me toques, malvado\ldots{} ¡Te desprecio, te aborrezco, me
causas horror!

Salvador se detuvo. Su conciencia había dado un grito espantoso.

---No me has salvado, no me has salvado, no; es mentira---murmuró Gil de
la Cuadra.---Tú no puedes haber hecho una buena acción. Déjame, déjame.
No quiero verte más.

Estaban en el patio de la cárcel.

Era el momento en que los soldados enviados por el Gobierno ocupaban el
edificio, arrojando de allí a los milicianos.

Gil de la Cuadra, huyendo de Monsalud que corría tras él, cayó al suelo.
El joven se le acercó. Le habían ocurrido no sabemos qué palabras que le
parecieron convincentes. Acercose un soldado, y golpeando con el pie a
Gil de la Cuadra, dijo:

---Un miliciano borracho. A la calle pronto.

El anciano no podía moverse. Monsalud tomándolo en brazos, le sacó fuera
de la cárcel.

---¡Déjame, déjame, maldito!

Quiso andar, quiso huir, pero le faltaban las fuerzas. Monsalud le
sostenía, y así llegaron hasta la plazuela de Lavapiés, donde aguardaba
un coche. Salvador cargó de nuevo al anciano y lo entró en él. Solita le
recibió en sus brazos.

---Entra tú también, hermano.

Gil de la Cuadra había perdido el conocimiento; pero seguía diciendo:
---¡Maldito, maldito\ldots!

---Yo no---repuso Salvador.---Adiós, hermana, ya sabes dónde has de ir.

---Pero tú\ldots{} Entra de una vez.

---No, adiós; jamás volveremos a vernos\ldots{} Adiós.

Cuando el coche partió hacia las afueras de Madrid, Monsalud, dirigiose
hacia el interior de la villa. Más de una vez se detuvo ante cualquier
esquina en la actitud desesperada de un hombre que ha decidido
estrellarse la cabeza contra las paredes. Andaba sin dirección fija y
pasaba de una calle a otra. En una de las vueltas estuvo a punto de ser
atropellado por una carroza que entraba en el ancho pórtico de histórico
palacio. Era la carroza del marqués de Falfán de los Godos, y conducía a
los que ya eran marido y mujer. En la frente de esta no se había secado
aún el agua bendita que tomara antes de salir de la parroquia.

\flushright{Madrid, Junio de 1876.}

~

\bigskip
\bigskip
\begin{center}
\textsc{Fin de el Grande Oriente}
\end{center}

\end{document}
