\PassOptionsToPackage{unicode=true}{hyperref} % options for packages loaded elsewhere
\PassOptionsToPackage{hyphens}{url}
%
\documentclass[oneside,10pt,spanish,]{extbook} % cjns1989 - 27112019 - added the oneside option: so that the text jumps left & right when reading on a tablet/ereader
\usepackage{lmodern}
\usepackage{amssymb,amsmath}
\usepackage{ifxetex,ifluatex}
\usepackage{fixltx2e} % provides \textsubscript
\ifnum 0\ifxetex 1\fi\ifluatex 1\fi=0 % if pdftex
  \usepackage[T1]{fontenc}
  \usepackage[utf8]{inputenc}
  \usepackage{textcomp} % provides euro and other symbols
\else % if luatex or xelatex
  \usepackage{unicode-math}
  \defaultfontfeatures{Ligatures=TeX,Scale=MatchLowercase}
%   \setmainfont[]{EBGaramond-Regular}
    \setmainfont[Numbers={OldStyle,Proportional}]{EBGaramond-Regular}      % cjns1989 - 20191129 - old style numbers 
\fi
% use upquote if available, for straight quotes in verbatim environments
\IfFileExists{upquote.sty}{\usepackage{upquote}}{}
% use microtype if available
\IfFileExists{microtype.sty}{%
\usepackage[]{microtype}
\UseMicrotypeSet[protrusion]{basicmath} % disable protrusion for tt fonts
}{}
\usepackage{hyperref}
\hypersetup{
            pdftitle={LOS CIEN MIL HIJOS DE SAN LUIS},
            pdfauthor={Benito Pérez Galdós},
            pdfborder={0 0 0},
            breaklinks=true}
\urlstyle{same}  % don't use monospace font for urls
\usepackage[papersize={4.80 in, 6.40  in},left=.5 in,right=.5 in]{geometry}
\setlength{\emergencystretch}{3em}  % prevent overfull lines
\providecommand{\tightlist}{%
  \setlength{\itemsep}{0pt}\setlength{\parskip}{0pt}}
\setcounter{secnumdepth}{0}

% set default figure placement to htbp
\makeatletter
\def\fps@figure{htbp}
\makeatother

\usepackage{ragged2e}
\usepackage{epigraph}
\renewcommand{\textflush}{flushepinormal}

\usepackage{indentfirst}

\usepackage{fancyhdr}
\pagestyle{fancy}
\fancyhf{}
\fancyhead[R]{\thepage}
\renewcommand{\headrulewidth}{0pt}
\usepackage{quoting}
\usepackage{ragged2e}

\newlength\mylen
\settowidth\mylen{...................}

\usepackage{stackengine}
\usepackage{graphicx}
\def\asterism{\par\vspace{1em}{\centering\scalebox{.9}{%
  \stackon[-0.6pt]{\bfseries*~*}{\bfseries*}}\par}\vspace{.8em}\par}

 \usepackage{titlesec}
 \titleformat{\chapter}[display]
  {\normalfont\bfseries\filcenter}{}{0pt}{\Large}
 \titleformat{\section}[display]
  {\normalfont\bfseries\filcenter}{}{0pt}{\Large}
 \titleformat{\subsection}[display]
  {\normalfont\bfseries\filcenter}{}{0pt}{\Large}

\setcounter{secnumdepth}{1}
\ifnum 0\ifxetex 1\fi\ifluatex 1\fi=0 % if pdftex
  \usepackage[shorthands=off,main=spanish]{babel}
\else
  % load polyglossia as late as possible as it *could* call bidi if RTL lang (e.g. Hebrew or Arabic)
%   \usepackage{polyglossia}
%   \setmainlanguage[]{spanish}
%   \usepackage[french]{babel} % cjns1989 - 1.43 version of polyglossia on this system does not allow disabling the autospacing feature
\fi

\title{LOS CIEN MIL HIJOS DE SAN LUIS}
\author{Benito Pérez Galdós}
\date{}

\begin{document}
\maketitle

Para la composición de este libro cuenta el autor con materiales muy
preciosos. Además de las noticias verbales, que casi son el principal
fundamento de la presente obra, posee un manuscrito que le ayudará
admirablemente en la narración de la parte o tratado que lleva por
título \emph{Los cien mil hijos de San Luis}. El tal manuscrito es
hechura de una señora, por cuya razón bien se comprende que será dos
veces interesante, y lo sería más aún si estuviese completo. ¡Lástima
grande que la negligencia de los primeros poseedores de él dejara perder
una de las partes más curiosas y necesarias que lo componen! Sólo dos
fragmentos, sin enlace entre sí, llegaron a nuestras manos. Hemos hecho
toda suerte de laboriosas indagaciones para allegar lo que falta, pero
inútilmente, lo que en verdad es muy lamentable, porque nos veremos
obligados a llenar con relatos de nuestra propia cosecha el gran vacío
que entre ambas piezas del manuscrito femenil resulta.

Este tiene la forma de Memorias. Su primer fragmento lleva por epígrafe
\emph{De Madrid a Urgel}, y empieza así:

\clearpage

\hypertarget{i}{%
\chapter{I}\label{i}}

En Bayona, donde busqué refugio tranquilo al separarme de mi esposo,
conocí al general Eguía\footnote{Puede verse el retrato de este
  personaje en las \emph{Memorias de un Cortesano de} 1815.}. Iba a
visitarme con frecuencia, y como era tan indiscreto y vanidoso, me
revelaba sus planes de conspiración, regocijándose en mi sorpresa y
riendo conmigo del gran chubasco que amenazaba a los franc-masones. Por
él supe en el verano del 21 que Su Majestad, nuestro católico Rey D.
Fernando (Q. D. G.), anhelando deshacerse de los revolucionarios por
cualquier medio y a toda costa, tenía dos comisionados en Francia, los
cuales eran:

1.º El mismo general D. Francisco Eguía, cuya alta misión era promover
desde la frontera el levantamiento de partidas realistas.

2.º D. José Morejón, oficial de la secretaría de la Guerra y después
secretario reservado de Su Majestad, con ejercicio de decretos, el cual
tenía el encargo de gestionar en París con el Gobierno francés los
medios de arrancar a España el cauterio de la Constitución gaditana,
sustituyéndole con una cataplasma anodina hecha en la misma farmacia de
donde salió la Carta de Luis XVIII.

Yo alababa estas cosas por no reñir con el anciano general, que era muy
galante y atento conmigo; pero en mi interior deploraba, como amante muy
fiel del régimen absoluto, que cosas tan graves se emprendieran por la
mediación de personas de tan dudoso valer. No conocía yo en aquellos
tiempos a Morejón; pero mis noticias eran que no había sido inventor de
la pólvora. En cuanto a Eguía, debo decir con mi franqueza habitual que
era uno de los hombres más pobres de ingenio que en mi vida he visto.

Aún gastaba la coleta que le hizo tan famoso en 1814, y con la coleta el
mismo humor atrabiliario, despótico, voluble y regañón. Pero en Bayona
no infundía miedo como en Madrid, y de él se reían todos. No es
exagerado cuanto se ha dicho de la astuta pastelera que llegó a
dominarle. Yo la conocí, y puedo atestiguar que el agente de nuestro
egregio Soberano comprometía lamentablemente su dignidad y aun la
dignidad de la Corona, poniendo en manos de aquella infame mujer
negocios tan delicados. Ella asistía la tal a las conferencias,
administraba gran parte de los fondos, se entendía directamente con los
partidarios que un día y otro pasaban la frontera, y parecía en todo ser
ella misma la organizadora del levantamiento y el principal apoderado de
nuestro querido Rey.

Después de esto he vivido muchas veces en Bayona y he visto la
vergonzosa conducta de algunos españoles que sin cesar conspiran en
aquel pueblo, verdadera antesala de nuestras revoluciones; pero nunca he
visto degradación y torpeza semejantes a las del tiempo de Eguía. Yo
escribía entonces a D. Víctor Sáez, residente en Madrid, y le decía:
«Felicite usted a los franc-masones, porque mientras la salvación de Su
Majestad siga confiada a las manos que por aquí tocan el pandero, ellos
están de enhorabuena.»

En el invierno del mismo año se realizaron las predicciones que yo, por
no poder darle consejos, había hecho al mismo Eguía, y fue que habiendo
convocado de orden del Rey a otros personajes absolutistas para trabajar
en comunidad, se desavinieron de tal modo, que aquello, más que junta
parecía la dispersión de las gentes. Cada cual pensaba de distinto modo,
y ninguno cedía en su terca opinión. A esta variedad en los pareceres y
terquedad para sostenerlos llamo yo enjaezar los entendimientos a la
calesera, es decir, a la española. El marqués de Mataflorida\footnote{Conocido
  por \emph{D. Buenaventura} en las \emph{Memorias de un Cortesano} y en
  *La segunda casaca.}, proponía el establecimiento del absolutismo
puro; Balmaseda, comisionado por el Gobierno francés para tratar este
asunto, también estaba por lo despótico, aunque no en grado tan furioso;
Morejón se abrazaba a la Carta francesa; Eguía sostenía el veto absoluto
y las dos Cámaras a pesar de no saber lo que eran una cosa y otra, y
Saldaña, nombrado como una especie de quinto en discordia, no se
resolvía ni por la tiranía entera ni por la tiranía a media miel.

Entretanto el Gobierno francés concedió a Eguía algunos millones, de los
cuales podría dar cuenta si viviese la hermosa pastelera. Dios me
perdone el mal juicio; pero casi podría jurar que de aquel dinero, sólo
algunas sumas insignificantes pasaron a manos de los pobres guerrilleros
tan bravos como desinteresados, que desnudos, descalzos y hambrientos,
levantaban el glorioso estandarte de la fe y de la monarquía en las
montañas de Navarra o de Cataluña.

Las bajezas, la ineptitud y el despilfarro de los comisionados secretos
de Su Majestad, no cesaron hasta que apareció en Bayona, también con
poderes reales, el gran pájaro de cuenta llamado D. Antonio Ugarte, a
quien no vacilo en designar como el hombre más listo de su época.

Yo le había tratado en Madrid el año 19. Él me estimaba en gran manera,
y, como Eguía, me visitaba a menudo; pero sin revelarme imprudentemente
sus planes. Desde que se encargó de manejar la conspiración, seguíala yo
con marcado interés, segura de su éxito, aunque sin sospechar que le
prestaría mi concurso activo en término muy breve. Un día Ugarte me
dijo:

---No se encuentra un solo hombre que sirva para asuntos delicados.
Todos son indiscretos, soplones y venales. ¿Ve usted lo que trabajo aquí
por orden de Su Majestad? Pues es nada en comparación de lo que me dan
que hacer las intrigas y torpezas de mis propios colegas de
conspiración. No me fío de ninguno, y en el día de hoy, teniendo que
enviar a Madrid un mensaje muy importante, estoy, como Diógenes,
buscando un hombre sin poder encontrarlo.

---Pues busque usted bien, Sr.~D. Antonio---le respondí,---y quizás
encuentre una mujer.

Ugarte no daba crédito a mi determinación; pero tanto le encarecí mis
deseos de ser útil a la causa del Rey y de la Religión, que al fin
convino en fiarme sus secretos.

---Efectivamente, Jenara---me dijo,---una dama podrá desempeñar mejor
que cualquier hombre tan delicado encargo si reúne a la belleza y
gallarda compostura de su persona un valor a toda prueba.

En seguida me reveló que en Madrid se preparaba un esfuerzo político, es
decir, un pronunciamiento, en el cual tomaría parte la Guardia real con
toda la tropa de línea que se pudiese comprometer; pero añadió que
desconfiaba del éxito si no se hacían con mucho pulso los trabajos,
tratando de combinar el movimiento cortesano con una ruidosa algarada de
las partidas del Norte. Discurriendo sobre este negocio, me mostró su
grandísima perspicacia y colosal ingenio para conspirar, y después me
instruyó prolijamente de lo que yo debía hacer en Madrid, del arte con
que debía tratar a cada una de las personas para quienes llevaba
delicados mensajes, con otras muchas particularidades que no son de este
momento. Casi toda mi comisión era enteramente confidencial y personal,
quiero decir que el conspirador me entregó muy poco papel escrito; pero,
en cambio, me repitió varias veces sus instrucciones para que,
reteniéndolas en la memoria, obrase con desembarazo y seguridad en las
difíciles ocasiones que me aguardaban.

Partí para Madrid en Febrero del 22.

\hypertarget{ii}{%
\chapter{II}\label{ii}}

Emprendí estos manejos con entusiasmo y con placer; con entusiasmo
porque adoraba en aquellos días la causa de la Iglesia y el Trono, con
placer porque la ociosidad entristecía mis días en Bayona. La soledad de
mi existencia me abrumaba tanto como el peso de las desgracias que a
otros afligen y que yo no conocía aún. Con separarme de mi esposo, cuyo
salvaje carácter y feroz suspicacia me hubieran quitado la vida, adquirí
libertad suma y un sosiego que después de saboreado por algún tiempo,
llegó a ser para mí algo fastidioso. Poseía bienes de fortuna
suficientes para no inquietarme de las materialidades de la vida; de
modo que mi ociosidad era absoluta. Me refiero a la holganza del
espíritu que es la más penosa, pues la de las manos, yo, que no carezco
de habilidades, jamás la he conocido.

A estos motivos de tristeza debo añadir el gran vacío de mi corazón, que
estaba ha tiempo como casa deshabitada, lleno tan sólo de sombras y de
ecos. Después de la muerte de mi abuelo, ningún afecto de familia podía
interesarme, pues los Baraonas que subsistían, o eran muy lejanos
parientes o no me querían bien. De mi infelicísimo casamiento sólo saqué
amarguras y pesadumbres, y para que todo fuese maldito en aquella unión,
no tuve hijos. Sin duda Dios no quería que en el mundo quedase memoria
de tan grande error.

Fácilmente se comprenderá que en tal situación de espíritu me gustaría
lanzarme a esas ocupaciones febriles que han sido siempre el principal
gozo de mi vida. Ninguna cosa llana y natural ha cautivado jamás mi
corazón, ni me embelesó, como a otros, lo que llaman dulce corriente de
la vida. Antes bien yo la quiero tortuosa y rápida, que me ofrezca
sorpresas a cada instante y aun peligros; que se interne por pasos
misteriosos, después de los cuales deslumbre más la claridad del día;
que caiga como el Piedra en cataratas llenas de ruido y colores, o se
oculte como el Guadiana, sin que nadie sepa dónde ha ido.

Yo sentía además en mi alma la atracción de la Corte, no pudiendo
descifrar claramente cuál objeto o persona me llamaban en ella, ni
explicarme las anticipadas emociones que por el camino sentía mi
corazón, como el derrochador que principia a gastar su fortuna antes de
heredada. Mi fantasía enviaba delante de sí, en el camino de Madrid,
maravillosos sueños e infinitos goces del alma, peligros vencidos y
amables ideales realizados. Caminando de este modo y con los fines que
llevaba, iba yo por mi propio y verdadero camino.

Desde que llegué me puse en comunicación con los personajes para quienes
llevaba cartas o recados verbales. Tuve noticias de la rebelión de los
Guardias que se preparaba; hice lo que Ugarte me había mandado en sus
minuciosas instrucciones, y hallé ocasión de advertir el mucho
atolondramiento y ningún concierto con que eran llevados en Madrid los
arduos trámites de la conspiración.

Lo mejor y más importante de mi comisión estaba en Palacio, adonde me
llevó D. Víctor Sáez, confesor de Su Majestad. Muchos deseos tenía yo de
ver de cerca y conocer por mí misma al Rey de España y toda su real
familia, y entonces quedó satisfecho mi anhelo. Hice un rápido estudio
de todos los habitantes de Palacio, particularmente de las mujeres, la
Reina Amalia, D.ª Francisca, esposa de don Carlos, y D.ª Carlota, del
Infante D. Francisco. La segunda me pareció desde luego mujer a
propósito para revolver toda la Corte. De los hombres, D. Carlos me
pareció muy sesudo, dotado de cierto fondo de honradez preciosísima, con
lo cual compensaba su escasez de luces, y a Fernando le diputé por muy
astuto y conocedor de los hombres, apto para engañarles a todos, si bien
privado del valor necesario para sacar partido de las flaquezas ajenas.
La Reina pasaba su vida rezando y desmayándose; pero la varonil D.ª
Francisca de Braganza ponía su alma entera en las cosas políticas, y
llena de ambición, trataba de ser el brazo derecho de la Corte. D.ª
Carlota, que entonces estaba embarazada del que luego fue Rey consorte,
tampoco se dormía en esto.

Los palaciegos, tan aborrecidos entonces por la muchedumbre
constitucional, Infantado, Montijo, Sarriá y demás aristócratas, no
servían en realidad de gran cosa. Sus planes, faltos de seso y
travesura, tenían por objeto algo en que se destacase con preferencia la
personalidad de ellos mismos. Ninguno valía para maldita la cosa, y así
nada se habría perdido con quitarles toda participación en la conjura.
Los individuos de la Congregación Apostólica, que era una especie de
masonería absolutista, tampoco hacían nada de provecho, como no fuera
allegar plebe y disponer de la gente fanática para un momento propicio.
En los jefes de la Guardia había más presunción que verdadera aptitud
para un golpe difícil, y el clero se precipitaba gritando en los
púlpitos, cuando la situación requería prudencia y habilidad sumas. Los
liberales masones o comuneros vendidos al absolutismo y que al
pronunciar sus discursos violentos se entusiasmaban por cuenta de este,
estaban muy mal dirigidos, porque con su exageración ponían diariamente
en guardia a los constitucionales de buena fe. He examinado uno por uno
los elementos que formaban la conspiración absolutista del año 22 para
que cuando la refiera se explique en cierto modo el lamentable aborto y
total ruina de ella.

{\textsc{Nota del autor.}}~ \emph{A continuación refiere la señora los
sucesos del 7 de Julio. Aunque su narración es superior a la nuestra,
principalmente a causa de la graciosa sencillez y verdad con que toda
ella está hecha, la suprimimos por no repetir, ni aun mejorándolo, lo
que ya apareció en otro volumen.}

\hypertarget{iii}{%
\chapter{III}\label{iii}}

Después de los aciagos días de Julio, mi situación que hasta entonces
había sido franca y segura, fue comprometidísima. No es fácil dar una
idea de la presteza con que se ocultaron todos aquellos hombres que
pocos días antes conspiraban descaradamente. Desaparecieron como caterva
de menudos ratoncillos, cuando los sorprende en sus audaces rapiñas el
hombre sin poder perseguirlos, ni aun conocer los agujeros por donde se
han metido. A mí me maravillaba que D. Víctor Sáez, hombre de una
obesidad respetable, pudiese estar escondido sin que al punto se
descubriese su guarida. Los palaciegos se filtraron también, y los que
no estaban muy evidentemente comprometidos, como por ejemplo, Pipaón,
dieron vivas a la Constitución vencedora, uniéndose a los liberales.

Tuve además la desgracia de perder varios papeles en casa de un pobre
maestro de escuela donde nos reuníamos, y esto me causó gran zozobra;
pero al fin los encontré no sin trabajo, exponiéndome a los mayores
peligros. La seguridad de mi persona corrió también no poco riesgo, y en
los días 9 y 10 de Julio no tuve un instante de respiro, pues por
milagro no me arrastraron a la cárcel los milicianos, borrachos de vino
y de patriotería. Gracias a Dios, vino en mi amparo un joven paisano y
antiguo amigo mío, el cual, en otras ocasiones, había ejercido en mi
vida influencia muy decisiva, semejante a la de las estrellas en la
antigua cábala de los astrólogos.

Pasados los primeros días pude introducirme en Palacio a pesar de la
formidable y espesa muralla liberalesca que lo defendía. Encontré a Su
Majestad lleno de consternación y amargura, principalmente por verse
obligado a poner semblante lisonjero a sus enemigos y aun a darles
abrazos, lo cual era muy del gusto de ellos, en su mayoría gente
inocentona y crédula. No me agradaba ver en nuestro Soberano tan poco
corazón; pero si en él hubiera concordado el valor con las travesuras y
agudezas del entendimiento, ningún tirano antiguo ni moderno le habría
igualado. Su desaliento y desesperación no le impidieron que se
enamorase de mí, porque en todas las ocasiones de su vida, bajo las
distintas máscaras que se quitaba y se ponía, aparecía siempre el
sátiro.

Temerosa de ciertas brutalidades, quise huir. Brindeme entonces a
desempeñar una comisión difícil, para lo cual Fernando no se fiaba de
ningún mensajero; y aunque él no quiso que yo me encargase de ella,
porque no me alejara de la Corte, tanto insté y con tales muestras de
verdad prometí volver, que se me dieron los pasaportes.

El mes anterior había salido para Francia D. José Villar Frontín, uno de
los intrigantes más sutiles del año 14, aunque como salido de la
academia del cuarto del Infante D. Antonio, no era hombre de gran
iniciativa, sino muy plegadizo y servicial en bajas urdimbres. Llevaba
órdenes para que el marqués de Mataflorida formase una Regencia
absolutista en cualquier punto de la frontera conquistado por los
guerrilleros. Estas instrucciones eran conformes al plan del Gobierno
francés, que deseaba la introducción de la Carta en España y un
absolutismo templado; pero Fernando, que hacía tantos papeles a la vez,
deseaba que sus comisionados, afectando ser partidarios de la Carta,
trabajasen por el absolutismo limpio. Esto exigía frecuentes
rectificaciones en los despachos que se enviaban y avisos
contradictorios, trabajo no escaso para quien había de ocultar de sus
ministros todos estos y aun otros inverosímiles líos.

Yo me comprometí a hacer entender a Mataflorida y a Ugarte lo que se
quería, transmitiéndoles verbalmente algunas preciosas ideas del
Monarca, que no podían fiarse al papel, ni a signo ni cifra alguna. Ya
por aquellos días se supo que la Seo de Urgel había sido ganada al
Gobierno por el bravo Trapense, y se esperaba que en la agreste plaza se
constituyera la salvadora Regencia. A la Seo, pues, debía yo dirigirme.

La partida y el viaje no eran problemas fáciles. Esto me preocupó
durante algunos días, y traté de sobornar, para que me acompañase, al
amigo de quien antes he hablado. A él no le faltaban en verdad ganas de
ir conmigo al extremo del mundo; pero le contenía el amor de su madre
anciana. Mucho luché para decidirle, empleando razonamientos y
seducciones diversas; mas a pesar de la propensión de su carácter a
ciertas locuras y del considerable prestigio que yo empezaba a ejercer
sobre él, se resistía tenazmente, alegando motivos poderosos, cuya
fuerza no me era desconocida. Al fin tanto pudo una mujer llorando, que
él abandonó todo, su madre y su casa, aunque por poco tiempo y con la
sana intención de volver cuando me dejase en parajes donde no existiese
peligro alguno. El infeliz presagiaba sin duda su desdichada suerte en
aquella expedición, porque luchó grandemente consigo mismo para
decidirse, y hasta el último momento estuvo vacilante.

Aquel hombre había sido enemigo mío, o más propiamente, de mi esposo.
Desde la niñez nos conocimos; fue mi novio en la edad en que se tiene
novio. Sucesos lamentables que me afligen al venir a la memoria,
caprichos y vanidades mías me separaron de él, yo creí que para siempre;
pero Dios lo dispuso de otro modo. Durante mucho tiempo estuve creyendo
que le odiaba; pero el sentimiento que en mí había era más que rencor
una antipatía arbitraria y voluntariosa. Por causa de ella, siempre le
tenía en la memoria y en el pensamiento. Circunstancias funestas le
pusieron en contacto conmigo diferentes veces, y siempre que ocurría
algo grave en la vida de él o en la mía tropezábamos providencialmente
el uno con el otro, como si el alma de cada cual viéndose en peligro
pidiese auxilio a su compañera.

En mí se verificó una crisis singular. Por razones que no son de este
sitio, yo llegué a aborrecer todo lo que mi esposo amaba y a amar todo
lo que él aborrecía. Al mismo tiempo mi antiguo novio mostraba hacia mí
sentimientos tan vivos de menosprecio y desdén, que esto inclinó mi
corazón a estimarle. Yo soy así, y me parece que no soy el único
ejemplar. Desde la ocasión en que le arranqué de las furibundas manos de
mi marido no debí de ser tampoco para él muy aborrecible.

Cuando nos encontramos en Madrid, y desde que hablamos un poco, caímos
en la cuenta de que ambos estábamos muy solos. Y no sólo había semejanza
en nuestra soledad, sino en nuestros caracteres, principal origen quizás
de aquella. Hicimos propósito de echar a la espalda aquel trágico
aborrecimiento que antes nos teníamos, el cual se fundaba en veleidades
y caprichosas monomanías del espíritu, y no tardamos mucho tiempo en
conseguirlo. Ambos reconocimos las grandes y ya irremediables
equivocaciones de nuestra primera juventud, y nos maravillábamos de
hallar tan extraordinaria fraternidad en nuestras almas. ¡Ser de este
modo, haber nacido el uno para el otro, y sin embargo haber estado
dándonos golpes en las tinieblas durante tanto tiempo! ¡Qué fatalidad!
Hasta parece que no somos responsables de ciertas faltas, y que estas,
por lo que tienen de placentero, pueden tolerarse como compensación de
pasados dolores y de un error deplorable y fatal, dependiente de
voluntades sobrehumanas.

Pero no: no quiero eximirme de la responsabilidad de mi culpa y de haber
faltado claramente, impulsada por móviles irresistibles, a la ley de
Dios. No: nada me disculpa; ni las atrocidades de mi marido, ni la
espantosa soledad en que yo estaba, ni los mil escollos de la vida en la
Corte, ni las grandes seducciones morales y físicas de mi paisano y
dulce compañero de la niñez. Reconozco mi falta, y atenta sólo a que
este papel reciba un escrupuloso retrato de mi conciencia y de mis
acciones, la escribo aquí, venciendo la vergüenza que confesión tan
penosa me causa.

Salimos de Madrid en una hermosa noche de Julio. Cuando dejamos de oír
el rugido de la Milicia victoriosa, me pareció que entraba en el cielo.
Íbamos cómodamente en una silla de postas con buenos caballos y un hábil
mayoral de Palacio. Yo había tomado un nombre supuesto, diciéndome
marquesa de Berceo y él era nada menos que mi esposo, una especie de
marqués de Berceo. Mucho nos reímos con esta invención, que a cada paso
daba lugar a picantes comentarios y agudezas. No recuerdo días más
placenteros que los de aquel viaje.

¡Cuántas veces bajamos del coche para andar largos trechos a pie,
recreándonos en la hermosura de las incomparables noches de Castilla!
¡Cómo se agrandaba todo ante nuestros ojos, principalmente las cosas
inmateriales! Nos parecía que aquella dulce vagancia no acabaría nunca,
y que los días venideros serían siempre como aquel cielo que veíamos,
dilatados, serenos y sin nubes. En tales horas o hablábamos poco o
vertíamos el alma del uno en la del otro alternativamente por medio de
observaciones y preguntas acordes con el hermoso espectáculo que veíamos
fuera y dentro de nosotros, pues de mi alma puede decirse que estaba tan
llena de estrellas como el firmamento.

Han pasado muchos años: entonces tenía yo veintisiete, y ahora\ldots{}
no lo quiero decir por no espantarme; pero creo que he traspasado el
medio siglo\footnote{Según nuestras noticias, la señora escribió esas
  Memorias durante la guerra civil del 48.}. Entonces mis cabellos eran
de oro, ahora son de plata, sin que ni una sola hebra de ellos conserve
su primitivo color. Mis ojos tenían el brillo que es reflejo de la
inteligencia despierta y de los sentimientos bullidores; ahora no son
más que dos empañadas cuentas azules, de las cuales se escapa alguna vez
fugitivo rayo. Mi cara entonces respiraba alegría, salud, y el alma
rielaba sobre mis facciones como la luz sobre la superficie de las
temblorosas aguas; ahora es una máscara que me sirve para disimular los
pensamientos y que a muchos deja ver todavía huellas claras de la gran
hermosura que hubo en ella. Entonces era muy hermosa; ahora soy una
\emph{vieja que debió haber sido guapa}, aunque, si he de creer a don
Toribio, el canónigo de Tortosa, todavía puedo volver loco a cualquiera.
En suma; todo ha pasado, mudándose considerablemente, e infinitas
personas han pasado a ser recuerdos. Lo que siempre está lo mismo es mi
país, que no deja de luchar un momento por la misma causa y con las
mismas armas, y si no con las mismas personas, con los mismos tipos de
guerreros y políticos. Mi país sigue siempre a la calesera.

Pues bien: en todo el tiempo transcurrido entre estas dos épocas, no he
visto pasar días como aquellos. Fueron de los pocos que tiene cada
mortal como un regalo del cielo para toda la existencia, y que en vano
se aguardan después, porque no vuelven. Estos aguinaldos de la vida no
se reciben más que una vez. Salvador era menos feliz que yo, a causa de
los deberes y las afecciones que había dejado atrás. Yo procuraba
hacerle olvidar todo lo que no fuese nosotros mismos; mas resultaba esto
muy difícil, por ser él menos dueño de sus acciones que yo, y aun, si se
quiere, menos egoísta. Íbamos de pueblo en pueblo, sin apresurarnos ni
detenernos mucho. Aquel vivir entre todo el mundo y al mismo tiempo sin
testigo, era mi mayor delicia. Los diversos pueblos por donde pasábamos
no tenían sin duda noticia de la felicidad de los marqueses de Berceo,
pues si la tuvieran, no creo que nos dejaran seguir sin quitarnos algo
de ella.

\hypertarget{iv}{%
\chapter{IV}\label{iv}}

Gracias a nuestro dinero y a nuestro buen porte podíamos disfrutar de
todas las comodidades posibles en las posadas. El calor nos obligaba a
detenernos durante el día, caminando por las noches, y ni en Castilla ni
en Aragón tuvimos ningún mal encuentro, como recelábamos, con
milicianos, ladrones o espías del Gobierno.

Más allá de Zaragoza empezamos a temer que nos salieran al paso las
tropas de Torrijos o de Manso. Por eso en vez de tomar directamente el
camino de Cataluña subimos hacia Huesca. Salvador, cuya antipatía a los
facciosos y guerrilleros era violentísima, se mostró disgustado al
considerarse cerca de ellos. Entonces tuve un momento de súbita
tristeza, oyéndole decir:

---Cuando lleguemos a un lugar seguro o estés entre tus amigos, me
volveré a Madrid.

Yo deseaba que no llegasen ni el lugar seguro ni tampoco mis amigos.
Pero aunque mi tristeza fue grande desde aquel instante, apoderándose de
mi corazón como un presagio de desventuras, estaba muy lejos de
sospechar el espantoso golpe que nos amenazaba, consecuencia
providencial de nuestra falta y de mi criminal ligereza. ¡Ay!, piensa el
malo que sus alegrías han de ser perpetuas, y la misma grata corriente
de ellas le lleva ciego a lo que yo llamo la sucursal del infierno en la
tierra, que es la desgracia y el anticipado castigo de los delitos.

De Huesca nos dirigimos a Barbastro, siguiendo por un detestable camino
hasta Benabarre, donde entramos al anochecer. Detuvieron nuestro coche
algunos hombres, y al verles, exclamé:

---Los guerrilleros. Ya estamos en casa.

Salvador mostró gran disgusto, y cuando fuimos interrogados, dio algunas
contestaciones que debieron de sonar muy mal en los oídos de los
soldados de la fe. Yo tenía confianza en mi gente y la seguridad de no
ser detenida; pero no fue posible evitar ciertas molestias. Nos hicieron
bajar del coche antes de llegar a la posada y presentarnos a un rústico
capitán que estaba en la venta del camino bebiendo vino juntamente con
otro guerrillero, al modo de frailazo, armado de pistolas y con dos o
tres individuos de malísima catadura.

Sus maneras no eran en verdad nada corteses, a pesar de defender causa
tan sagrada como es la del Altar y el Trono; pero con dos o tres
palabras dichas enérgicamente y en tono de dignidad, me hice respetar al
punto. Yo mostraba al que parecía jefe mis papeles, cuando observé que
uno de los hombres allí presentes miraba a mi compañero de viaje con
expresión poco tranquilizadora. Llegose a él, y poniéndole la mano en el
hombro le dijo con brutal modo y expresión de venganza:

---¿Me conoces? ¿Sabes quién soy?

---Sí---le respondió Monsalud, pálido y colérico.---Ya sé que eres un
hombre vil; tu nombre es Regato.

El desconocido se abalanzó en ademán hostil hacia mi amigo, pero este
supo recibirle con tanta valentía, que le hizo rodar por el suelo,
bañado el rostro en sangre. Quedeme sin aliento al ver la furia de
aquella gente ante el mal trato dado a uno de los suyos. Milagro de Dios
fue que no pereciésemos allí; pero el capitán parecía hombre prudente, y
haciendo salir de la venta al agraviado, nos notificó que estábamos
presos hasta que el jefe decidiera lo que se había de hacer con
nosotros.

Afectando serenidad le dije que mirara bien lo que hacía, por ser yo
persona de gran poder en la frontera y en Palacio; pero encogiéndose de
hombros, tan sólo me permitió después de largas discusiones hablar al
que ellos llamaban coronel. Salí desalada de la venta, dejando en ella
la mitad de mi alma, pues allí quedó guardado por dos hombres mi
ultrajado amigo, y me presenté al coronel, que era un capuchino de
Cervera. Acababa de despachar un bodrio y dos azumbres que le habían
puesto para que cenase, y su paternidad, después del pienso, no tenía al
parecer la cabeza muy serena. Sin embargo, no me trató mal. Díjome que
el Sr. Regato le había informado ya de quién era mi acompañante, y que
en vista de sus antecedentes y circunstancias, no podía ser puesto en
libertad. Púseme furiosa; yo me creí capaz de destrozar sólo con mis
uñas a aquel tremendo fraile coronel cuyas barbas y salvaje apostura
ponían miedo en el corazón más esforzado. Sin miramiento alguno le
increpé, diciéndole cuantas atrocidades me vinieron a la boca y
amenazándole con pedir su cabeza al Rey; pero ni aun así logré ablandar
aquella roca en figura de bestia. Oyome el bárbaro con paciencia, sin
duda por ser más fraile que guerrero, y resumió sus resoluciones
diciéndome:

---Usted, señora, puede ir libremente a donde le acomode; pero ese
hombre no me sale de aquí.

¡Ay!, si yo hubiera tenido a mis órdenes diez hombres armados habría
atacado al batallón, cuadrilla o lo que fuera, segura de destrozarlo,
que tanto puede el furor de una hembra ofendida. Volví a la venta,
resuelta a sacar de ella a Salvador con mis propias manos, desafiando
las armas de sus guardianes; pero cuando entré, mi compañero de viaje,
mi adorado amigo, mi pobre marqués de Berceo, había desaparecido. Le
llamé con la voz ronca de tanto gritar; le llamé con toda mi alma, pero
no me respondió. Una mujer andrajosa, que parecía tan salvaje y feroz
como los hombres que en aquel pueblo vi, salió conmigo al camino y
señalando a un punto en la oscuridad del espacio negro, dijo sordamente:

---Allí.

Y mirando hacia donde su dedo me indicaba, vi unas grandes sombras que
parecían murallones almenados y como ruinas hendidos. Pregunté qué sitio
era aquel y la desconocida me contestó:

---El castillo.

La mujer llevando una cesta con provisiones, marchó en dirección del
castillo. Yo la seguí. No tardamos en llegar, y por una poterna
desvencijada que se abría en la muralla, después de pasado el foso sin
agua, penetramos en un patio lleno de escombros y de yerba.

---¡Aquí, aquí le han encerrado!---exclamé mirando a todos lados como
quien ha perdido el juicio.

La mujer se detuvo ante mí, y señalando el suelo dijo con voz muy
lúgubre:

---¡Abajo!

Yo creí volverme loca. Los ojos de la horrible persona que me daba tan
tremendas noticias brillaban con claridad verdosa, como los de animal
felino. Quise seguirla cuando subió la escalerilla que conducía a las
habitaciones practicables entre tanta ruina; pero un centinela me echó
fuera brutalmente, amenazándome con arrojarme al foso si no me retiraba
\emph{más pronto que la vista}. Estas fueron sus propias palabras.

Corrí hacia el pueblo, resuelta a ver de nuevo al coronel capuchino de
Cervera. Pero tanta agitación agotó al fin mis fuerzas, y tuve que
sentarme en una gran piedra del camino, fatigada y abatida, porque a mi
primera furia sustituyó una aflicción profundísima que me hizo llorar.
No recuerdo haber derramado nunca más lágrimas en menos tiempo. Al fin,
sobreponiéndome a mi dolor, seguí adelante, jurando no continuar el
viaje sin llevar en mi compañía al infeliz cuanto adorado amigo de mi
niñez. Desperté al capuchino, que ya roncaba, el cual de muy mal
talante, repitió su fiera sentencia, diciendo:

---Usted, señora, puede continuar su viaje; pero el otro no saldrá de
aquí sin orden superior. Yo sé lo que me digo. ¡Pisto!, que ya me canso
de sermonear. Vaya usted con Dios y déjenos en paz.

Despreciando su barbarie, insistí y amenacé, y al cabo me dio algunas
esperanzas con estas palabras:

---El jefe de nuestra partida acaba de llegar. Háblele usted a él, y si
consiente\ldots{}

---¿Quién es el jefe?

---D. Saturnino Albuín---me contestó.

Al oír este nombre vi el cielo abierto. Yo había conocido en Bayona al
célebre \emph{Manco}, y recordé que aunque muy bárbaro, hacía alarde de
generosidad e hidalguía en todas las ocasiones que se le presentaban. No
quise detenerme ni un instante, y al punto me informé de que D.
Saturnino estaba en una casa situada junto al camino a la salida del
pueblo en dirección a Tremp. Desde la plaza se veían dos lucecillas en
las ventanas de la vivienda. Corrí allá guiada por la simpática claridad
de aquellas luces semejantes a dos ojos y que eran para mí fanales de
esperanza. Llegué sin aliento, agitada por la fatiga y un dulce presagio
de buen éxito que me llenaba el corazón.

El centinela me dijo que no se podía pasar; pero apelando a mis
bolsillos, pasé. En la escalera, en el pasillo alto, fui repetidas veces
detenida; pero con el mismo talismán abríame paso.

---Ahí está---me dijo un hombre señalando una puerta detrás de la cual
se oían alteradas voces en disputa. Sin reparar más que en mi afán
empujé la puerta y entré.

Albuín, que estaba en pie, se volvió al sentir el ruido de la puerta, y
me interrogó con sus ojos, que expresaban sorpresa y cólera por mi
brusca entrada. Otro guerrillero estaba junto a la mesa con los codos
sobre ella, encendiendo un cigarro en la luz del velón de cobre que
alumbraba la estancia.

---¿Qué se le ofrece a usted, señora?---me dijo Albuín moviendo con
gesto de impaciencia su única mano.

Yo no había dado cuatro pasos dentro de la habitación, cuando observé
que más allá de la mesa había otro hombre, apoltronado en un sillón, con
los pies extendidos sobre una banqueta, inclinada la cabeza sobre el
hombro y durmiendo tranquilamente con ese sueño del guerrillero cansado
que acaba de recorrer dos provincias y marear a dos ejércitos. Al verle
¡Santo Dios!, me quedé yerta, muda, como estatua; no pude pronunciar una
palabra, ni dar un paso, ni respirar, ni huir, ni gritar. El terror me
arrancó súbitamente del pensamiento mis angustias de aquella noche.

Aquel hombre era mi marido.

---¿Qué se le ofrece a usted, señora?---volvió a preguntarme el
\emph{Manco}.

Pasado el primer instante de terror, en mí no hubo otra idea que la idea
de huir, de desaparecer, de desvanecerme como el humo o como la palabra
vana que se lleva el viento.

---Pero, ¿qué se le ofrece a usted, demonio?---repitió el guerrillero.

---¡Nada!---contesté, y a toda prisa salí de la habitación.

Yo creo que ni un relámpago corre como yo corrí fuera de la casa. No
veía más que el camino, y mi veloz carrera nunca me parecía bastante
apresurada para llegar al centro del pueblo donde había dejado mi coche.

A lo lejos, detrás de mí, sentí voces burlonas que decían:

---¡La mujer loca, la mujer loca!

Eran los bárbaros a quienes yo había dado tanto dinero para que me
dejasen pasar. A cada instante volvía la cabeza por ver si mi marido
venía corriendo detrás de mí.

Llegué medio muerta a donde estaba mi coche, y tirando del brazo del
cochero para que despertase, grité:

---¡Francisco, Francisco, vuela, vuela fuera de este horrible pueblo!

Y me metí en el coche.

---¿Adónde vamos, señora?---me preguntó el pobre hombre sacudiendo la
pereza.

---¿Estás sordo? Te he dicho que vueles\ldots{} ¿Hablo yo en griego?,
que vueles, hombre. Mata los caballos, pero ponme a muchas leguas de
aquí.

---¿A dónde vamos, señora? ¿Hacia la Seo?

---Hacia el infierno si quieres, con tal que me saques de aquí.

Mi coche partió a escape, y siguiendo el camino en dirección a Tremp,
pasé junto a la malhadada casa donde había visto a mi esposo. Entonces
los bárbaros reunidos junto a la puerta me aclamaron otra vez, arrojando
algunas piedras a mi coche. Su grito era:

---¡La mujer loca, la mujer loca!

En efecto, lo estaba. ¡Ah! ¡Benabarre, Benabarre, maldito seas! En ti
acabó mi felicidad; en las espinas de tu camino dejé clavado mi corazón
chorreando sangre. Fuiste mi calvario y la piedra resbaladiza de mal
agüero donde caí para siempre, cuando más orgullosa marchaba. Fuiste el
tajo donde el cielo puso mi cabeza para asegurar el golpe de su
cuchilla; pero con ser obra del cielo mi castigo, ¡te odio, execrable
pueblo de bandidos! ¡Sepulcro de mi edad feliz, no puedo verte sin
espanto, y mientras tenga lengua, te maldeciré!

\hypertarget{v}{%
\chapter{V}\label{v}}

Llegué a la Seo el 14 de Agosto. ¡Qué viaje el de Benabarre a la Seo! Si
antes todo se adaptaba al lisonjero estado de mi alma, después todos los
caballos eran malos, todos los caminos intransitables, todas las posadas
insufribles, todos los días calorosos, y las noches todas tristes como
los pensamientos del desterrado. Mi alma sin consuelo, mientras más
gente veía, más sola se encontraba. Mi pensamiento no podía apartarse de
aquel lugar siniestro donde habían quedado mi amor y mi suplicio, mi
falta y mi conciencia, representados cada una en un hombre.

Casi antes de desempeñar mi comisión traté de ocuparme de salvar al
infeliz que había quedado cautivo en Benabarre; pero Mataflorida me dijo
sonriendo:

---Luego, luego, mi querida señora, trataremos de ese asunto. Infórmeme
usted de lo que trae, pues no hay tiempo que perder. Hoy mismo
constituiremos la Regencia.

Más de dos horas estuvimos departiendo. Él, como hombre muy ambicioso y
que gustaba de ser el primero en todo, recibió con gusto las
instrucciones reservadísimas que le daban gran superioridad entre sus
compañeros de Regencia. Eran estos el barón de Eroles y don Jaime Creux,
arzobispo de Tarragona, ambos, lo mismo que Mataflorida, de clase
humildísima, sacados de su oscuridad por los tiempos revolucionarios, lo
cual no era un argumento muy fuerte en pro del absolutismo. Una Regencia
destinada a restablecer el Trono y el Altar, debió constituirse con
gente de raza. Pero la edad revuelta que corríamos los exigía de otro
modo, y hasta el absolutismo alistaba su gente en la plebe. Este hecho,
que ya venía observándose desde el siglo pasado, lo expresaba Luis XV
diciendo que la nobleza necesitaba estercolarse para ser fecundada.

De los tres regentes, el más simpático era Mataflorida y también el de
más entendimiento; el más tolerante Eroles, y el más malo y antipático,
D. Jaime Creux. No puede decirse de estos hombres que habían marchado
con lentitud en sus brillantes carreras. Eroles era estudiante en 1808 y
en 1816 teniente general. El otro de clérigo oscuro pasó a obispo, en
premio de su traición en las Cortes del año 14.

Yo no tenía mi espíritu en disposición de atender a las ceremonias con
que quisieron celebrar los triunviros el establecimiento de la Regencia.
Después de publicar su célebre manifiesto, proclamaron solemnemente al
Monarca, \emph{restituyéndole a la plenitud de sus derechos}, según
decíamos entonces. Levantóse en la plaza de la Seo un tablado, sobre el
que un sacristán vestido de rey de armas gritó: «¡España por Fernando
VII!» y luego dieron al viento una bandera en la cual las monjas habían
bordado una cruz y aquellas palabras latinas que quieren decir:
\emph{por este signo vencerás}. Los altos castillos que coronan los
montes en cuyo centro está sepultada la Seo hicieron salvas, y aquello
en verdad parecía una proclamación en toda regla.

Después de la ceremonia política hubo jubileo por las calles y rogativa
pública, a que concurrió el obispo con todo el clero armado y el cabildo
sin armas. Era un espectáculo edificante y al mismo tiempo horroroso.
Daba idea de la inmensa fuerza que tenían en nuestro país las dos clases
reunidas, clero y plebe; pero los frailes armados de pistolas y los
guerrilleros con vela en la mano, el general con crucifijo y el
arcediano con espuelas, movían a risa y a odio juntamente. El ejército
de la fe, uniformado sólo con el gorro catalán habría parecido un
ejército de pavos, si no estuviera bien probado su indomable valor.

Yo veía aquella procesión chabacana, horrible parodia del levantamiento
nacional de 1808, y aquellas espantosas figuras de curas confundidas con
guerreros, como se ven las ficciones horrendas de una pesadilla. Tal
espectáculo era excesivamente desagradable a mi espíritu, y la bulla del
pueblo me ponía los nervios en el más lastimoso desorden. Semejante
Carnaval en Urgel, que es sin disputa el pueblo más feo de todo el
mundo, era para enfermar y aun enloquecer a cualquiera. Mi privilegiada
naturaleza me salvó.

Y pasaban días sin que me fuera posible hacer nada de provecho por mi
amado prisionero de Benabarre. Obtenía, sí, promesas y aun órdenes de la
Regencia; pero como no podía trasladarme yo misma al lugar del
conflicto, era muy difícil que tuviesen cumplimiento. Antes me dejara
morir que encaminarme a paraje alguno donde hubiese probabilidades de
encontrar la persona o siquiera las huellas de mi esposo; y según mis
averiguaciones, este no había abandonado el bajo Aragón.

Al fin supe que mi cara mitad, uniéndose a Jeps dels Estanys, había
pasado a la alta Cataluña. Llena de esperanza entonces corrí a
Benabarre, cargada de órdenes de Mataflorida y del mismo Eroles que
acababa de ponerse a la cabeza de la insurrección catalana. Ningún
obstáculo podían oponerme ya los guerrilleros; mas por mi desgracia,
cuando llegué al funesto pueblo de Aragón ni un solo partidario del
realismo quedaba en su recinto; el castillo había sido volado, y el
mísero cautivo, según me dijeron, trasladado a otro punto.

---¿Vivo?---pregunté.

---Vivo y cargado de cadenas---me contestó la misma mujer de aquella
horrenda noche de Agosto.---Se iba muriendo por el camino; pero le daban
comida y bebida para que no acabase de padecer.

No tuve tiempo para entregarme a inútiles lamentaciones, porque corrió
por todo el pueblo esta horrible voz: \emph{¡los liberales!}, \emph{¡que
vienen los liberales!} y tuve que huir. Con mucho trabajo y gastando
bastante dinero pude escapar a Francia por Canfranc.

{\textsc{Nota del autor.}}~~\emph{Aquí concluye el primer fragmento de
las curiosas Memorias.}

\emph{Como el segundo se refiere a sucesos ocurridos en la primavera del
22, resultando una interrupción de siete meses, nos vemos en la
necesidad de llenar tan lamentable vacío con relaciones propias, que
abreviaremos todo lo posible para que no se echen de menos por mucho
tiempo las aventuras de la dama viajera, contadas por ella misma.}

\hypertarget{vi}{%
\chapter{VI}\label{vi}}

La primera determinación del Gobierno popular que sucedió al de Martínez
de la Rosa, después de las jornadas de Julio, fue nombrar general del
ejército del Norte al rayo de las guerrillas, al Napoleón navarro, D.
Francisco Espoz y Mina. En medio de su atolondramiento, los siete
Ministros, a quienes la Corte llamaba los Siete niños de Écija, no
carecían de iniciativa y de cierta arrogancia emprendedora que por algún
tiempo les permitió sostenerse en el poder con prestigio. El
nombramiento de Mina y aquella orden que le dieron de hacer tabla rasa
de las provincias rebeldes no pudieron ser más acertados.

El gran guerrillero no necesitaba muy vivas excitaciones para sentar su
pesada mano a los pueblos. Navarros y catalanes le conocían. Pero antaño
había hecho la guerra con ellos, y ahora debía hacerla contra ellos, lo
cual era muy distinto. Antes se batía contra tropas regulares y ahora
con ellas perseguía las partidas. Bien se ve que el coloso de las
guerrillas estaba fuera de su natural esfera y asiento. Iba a hacer el
papel del enemigo durante la guerra de la Independencia.

A pesar de esta desventaja empezó con muy buen pie su campaña. No podía
decirse propiamente que había partidas en el Norte, sino que todo el
Norte desde Gerona hasta Guipúzcoa, y desde el Pirineo hasta las
inmediaciones del Ebro, ardía con horrible llamarada absolutista.
Quesada, a cuyo lado despuntaba un precoz muchacho llamado
Zumalacárregui, dominaba en Navarra, juntamente con Guergué y D. Santos
Ladrón; Albuín y Cuevillas y Merino, asolaban la tierra de Burgos;
Capapé, el Aragón; Jeps dels Estanys, el Trapense, Romagosa y Caragol, a
Cataluña, donde el barón de Eroles trataba de formar un ejército regular
con las desperdigadas gavillas de la fe. Muchos frailes del país,
empezando por los aguerridos capuchinos de Cervera que habían escapado
del furor de las tropas liberales, y concluyendo por los monjes de
Poblet que tanto trabajaron en la conspiración, formaban en las filas
del Manco, o de Capapé o de Misas.

Mina tomó el mando de las tropas de Cataluña, y al poco tiempo el
aspecto de la campaña principió a mudarse favorablemente a nuestras
armas. En 24 de Octubre, después de obligar a los facciosos a levantar
el sitio de Cervera, arrasó a Castellfollit, poniendo sobre sus ruinas
el célebre cartel que decía: «Aquí existió Castellfollit. Pueblos, tomad
ejemplo, y no deis abrigo a los enemigos de la patria.»

En Noviembre tomó a Balaguer. En el mismo mes obligó a muchos facciosos
a pasar la frontera en presencia del cordón sanitario con que nos
amenazaban los franceses. En 20 de Enero, uno de los suyos, el brigadier
Rotten, jefe de la cuarta división del ejército de Cataluña, hacía
sufrir a San Llorens de Morunys el tremendo castigo de que había sido
víctima Castellfollit, diciendo a las tropas en la orden del día: «La
villa esencialmente rebelde llamada San Llorens de Morunys será borrada
del mapa.»

Aquel destructor de ciudades señalaba a cada regimiento las calles que
debía saquear antes de dar principio a la operación de borrar del mapa.
No de otra manera procedió Hoche en la Vendée; pero este sistema de
borrar del mapa es algo expuesto, sobre todo en España.

El 8 de Diciembre puso Mina sitio a la Seo de Urgel, mientras Rotten iba
convenciendo a los rebeldes catalanes con las suaves razones que
indicamos, y en uno de los pueblos demolidos y arrasados, precisamente
en aquel mismo San Llorens de Morunys, llamado también Piteus, ocurrió
un suceso digno de mencionarse y que causó maravilla y emoción muy viva
en toda la tropa.

Fue de la manera siguiente: Para que el saqueo se hiciera con orden,
Rotten dispuso que el batallón de Murcia trabajase en las calles de
Arañas y Balldelfred; el de Canarias, en las calles de Frecsures y
Segories; el de Córdoba, en la de Ferronised y Ascervalds, dejando los
arrabales para el destacamento de la Constitución y la caballería. Lo
mismo en la orden de saqueo que en la de incendio, que le siguió, fueron
exceptuadas doce casas que pertenecían a otros tantos patriotas.

El regimiento de Córdoba funcionaba en la calle de Ferronised, entre la
consternación de los aterrados habitantes, cuando unos soldados
descubrieron un hondo sótano o mazmorra, y registrándolo, por si en él
había provisiones almacenadas para los facciosos, vieron a un hombre
aherrojado, o más propiamente dicho, un cadáver viviente, cuya miserable
postración y estado les causaron espanto. No vacilaron en prestarle
auxilio cristianamente sacándole de allí en hombros, después de quitarle
con no poco trabajo las cadenas; y cuando el cautivo vio la luz se
desmayó, pronunciando incoherentes palabras, que más bien expresaban
demencia que alegría.

Rodeáronle todos, siendo objeto de gran curiosidad por parte de
oficiales y soldados, que no cesaban de denostar a los facciosos por la
crueldad usada con aquel infeliz. Este parecía haber permanecido bajo
tierra mucho tiempo, según estaba de lívido y exangüe, y sin duda, era
víctima del furor de las hordas absolutistas, y más que criminal
castigado por sus delitos, un buen patriota condenado por su amor a la
Constitución.

Un capitán ayudante de Rotten, llamado D. Rafael Seudoquis, se interesó
vivamente por el cautivo, y después de mandar que se le diera toda clase
de socorros, le apremió para que hablase. El hombre sacado del fondo de
la tierra parecía joven, a pesar de lo que le abrumaba su padecer, y se
sorprendió muy agradablemente de ver los uniformes de la tropa. Las
primeras palabras que pronunció fueron:

---¿En dónde están?

---¿Los facciosos?---dijo Seudoquis riendo.---Me parece que no les
veremos en mucho tiempo, según la prisa que llevan\ldots{} Ahora, buen
amigo, díganos cómo se llama usted y quién es.

El cautivo hacía esfuerzos para recordar.

---¿En qué año estamos?---preguntó al fin mirando a todos con
extraviados ojos.

---En el de 1823, que parece será el peor año del siglo, según como
empieza.

---¿Y en qué mes?

---En Enero y a 15, día de San Pablo ermitaño. Si usted recuerda cuándo
le empaquetaron puede hacer la cuenta del tiempo que ha estado en
conserva.

---He estado preso---dijo el hombre después de una larga pausa,---seis
meses y algunos días.

---Pues no es mucho, otros han estado más. No le habrán tratado a usted
muy bien: eso es lo malo; pero descuide usted, que ahora las van a pagar
todas juntas. El pueblo será incendiado y arrasado.

---¡Incendiado y arrasado!---exclamó el cautivo con pena.---¡Qué lástima
que no sea Benabarre!

---Sin duda, el cautiverio de usted---dijo Seudoquis, intimando más con
el desgraciado,---empezó en ese horrible pueblo aragonés.

---Sí señor, de allí me trajeron a Tremp y de Tremp a Masbrú y de Masbrú
aquí.

---¡Oh!, ¡buen viaje ha sido! ¡Y seis meses de encierro, bajo el poder
de esa canalla! No sé cómo no le fusilaron a usted seiscientas veces.

---Eran demasiado inhumanos para hacerlo.

Lleváronle fuera del pueblo en una camilla y a presencia del brigadier,
que le interrogó. Desde el cuartel general vio las llamas que devoraban
San Llorens, y entonces dijo:

---Arde lo inocente, las guaridas y los perversos lobos están en el
monte.

El bravo y generoso Seudoquis fue encargado por el brigadier de
vestirle, pues los andrajos que cubrían el cuerpo del cautivo se caían a
pedazos. Al día siguiente de su maravillosa redención, hallose muy
repuesto por la influencia del aire sano y de los alimentos que le
dieron, y aunque le era imposible dar un paso, podía hablar sin
acongojarse como el primer día por falta de aliento.

---¿Qué ha pasado en todo este tiempo?---preguntó con voz débil y
temblorosa al que continuamente le daba pruebas de generosidad e
interés.---¿Sigue reinando Fernando VII?

---Hombre, sí, todavía le tenemos encima---dijo Seudoquis atizando la
hoguera, alrededor de la cual vivaqueaban juntamente con el cautivo
cuatro o cinco oficiales.---Gotosillo sigue nuestro hombre; pero aún nos
está embromando y nos embromará por mucho tiempo.

---¿Y la Constitución, subsiste?

---También está gotosa, o mejor dicho, acatarrada. Me parece que de esta
fecha enterramos a la señora.

---¿Y hay Cortes?

---Cortes y recortes. Pero me parece que pronto no quedarán más que los
de los sastres.

---Y qué, ¿hay revolución en España?

---Nada: estamos en una balsa de aceite.

---¿Qué Ministerio tenemos?

---El de los Siete niños de Écija. ¿Pues qué, vamos a estar mudando de
niños todos los días?

---¿Y ha vuelto la Milicia a sacudir el polvo a la Guardia Real?

---Ahora nos ocupamos todos en cazar frailes y guerrilleros, siempre que
ellos no nos cacen a nosotros.

---¿Y Riego?

---Ha ido a Andalucía.

---¿Hay agitación allá?

---Lo que hay es mucha sangre vertida en todas partes.

---Revolución completa. ¿Dónde hay partidas?

---Pregunte usted que dónde hay españoles.

---Toda Cataluña parece estar en armas contra el Gobierno.

---Y casi todo Aragón y Navarra y Vizcaya y Burgos y León y mucha parte
de Guadalajara, Cuenca, Ávila, Toledo, Cáceres. Hay facciones hasta en
Andalucía, que es como decir que hasta las ranas han criado pelo.

---¡Qué horrible sueño el mío---dijo lúgubremente el cautivo,---y qué
triste despertar!

---Esto es un volcán, amigo mío.

---¿Pero qué quieren?

---Confites. Piden Inquisición y cadenas.

---¿Y quién los dirige?

---El Rey y en su real nombre la Regencia de Urgel.

---Una Regencia\ldots{}

---Que tiene su Gobierno regular, sus embajadores en las Cortes de
Europa y ha contratado hace poco un gran empréstito. ¡Si no hay país
ninguno como este! Espanta el ver cómo falta dinero para todo menos para
conspirar.

---¿Y qué hace el Gobierno?

---¿Qué ha de hacer? Boberías. Trasladar los curas de una parroquia a
otra, declarar vacantes las sillas de los obispos que están en la
facción, fomentar las sociedades patrióticas, suprimir los conventos que
están en despoblado y otras grandes medidas salvadoras.

---¿No ha cerrado el Gobierno las sociedades patrióticas?

---Ha abierto la Landaburiana, para que los liberales tengan una buena
plazuela donde insultarse.

---¿Siguen los discursos?

---Sí; pero abundan más los cachetes.

---¿Y qué generales mandan los ejércitos de operaciones?

---Aquí Mina, en Castilla la Nueva O'Daly, Quiroga en Galicia, en Aragón
Torrijos.

---¿Y vencen?

---Cuando pueden.

---Es una delicia lo que encuentro a mi vuelta del otro mundo.

---Si casi era mejor que se hubiese usted quedado por allá. Así al menos
no sufriría la vergüenza de la intervención extranjera.

---¿Intervención?

---¡Y se asusta! ¿Pues hay nada más natural? Según parece, allá por el
mundo civilizado corre el rumor de que esto que aquí pasa es un
escándalo.

---Sí que lo es.

---Los Reyes temen que a sus Naciones respectivas les entre este
maleficio de las Constituciones, de las sociedades Landaburianas, de las
partidas de la Fe, de los frailes con pistolas, y nos van a quitar todos
estos motivos de distracción. Lejos del mundo ha estado usted, y muy
dentro de tierra cuando no han llegado a sus oídos las célebres notas.

---¿Qué notas?

---El re mi fa de las Potencias. Las notas han sido tres, todas muy
desafinadas, y las potencias que las han dado, tres también como las del
alma: Rusia, Prusia y Austria.

---¿Y qué pedían?

---No puedo decírselo a usted claramente porque los embajadores no me
las han leído; pero si sé que la contestación del Gobierno español ha
sido retumbante y guerrera como un redoble de tambor.

---Es decir que desafía a Europa.

---Sí señor, la desafiamos. Ahora se recuerda mucho la guerra de la
Independencia; pero yo digo, como Cervantes, que \emph{nunca segundas
partes fueron buenas}.

---¿De modo que tendremos otra vez extranjeros?

---Franceses. Ahí tiene usted en lo que ha venido a parar el ejército de
observación. Entre el cordón sanitario y el de San Francisco, nos van a
dar que hacer\ldots{} Digo\ldots{} y los diputados el día en que
aprobaron la contestación a las notas fueron aclamados por el pueblo. Yo
estaba en Madrid esa noche, y como vivo frente al coronel San Miguel,
las murgas no me dejaron dormir en toda la noche. Por todas partes no se
oyen más que mueras a la Santa Alianza, a las Potencias del Norte, a
Francia y a la Regencia de Urgel. Ahora se dice también como entonces
«dejarles que se internen;» pero la tropa no está muy entusiasmada que
digamos. Con todo, si entran los interventores no les recibiremos con
las manos en los bolsillos.

---Tremendos días vienen---dijo el cautivo.---Si los absolutistas
vencen, no podremos vivir aquí. O ellos o nosotros. Hay que
exterminarles para que no nos exterminen.

---Diga usted que si hubiera muchos brigadieres Rotten, pronto se
acababa esa casta maligna. Fusilamos realistas por docenas, sin
distinción de sexo ni edad, ni formalidades de juicio\ldots{} ¡Ay del
que cae en nuestras manos! Nuestro brigadier dice que no hay otro
remedio, ni entienden más razón que el arcabuzazo. Ayer hicimos catorce
prisioneros en San Llorens. Hay de toda casta de gentes: mujeres,
hombres, dos clérigos, un jesuita que usa gafas, un escribano de setenta
años, una mujer pública, dos guerrilleros inválidos; en fin, un
muestrario completo. El jefe les ha sentenciado ya; pero como esto no se
puede decir así, se hace la comedia de enviarles a la cárcel de Solsona,
y por el camino cuando viene la noche y se llega a un sitio
conveniente\ldots{} \emph{pim, pam}, se les despacha en un santiamén, y
a otra.

---Si no me engaño---dijo el cautivo,---aquellos paisanos que por allí
se ven, son los prisioneros de San Llorens.

En una loma cercana, a distancia de dos tiros de fusil se veía un grupo
de personas, custodiadas por la tropa. Parecía un rebaño que se había
detenido a sestear.

---Cabalmente---dijo Seudoquis,---aquellos son. Dentro de una hora se
pondrán en camino para la eternidad. ¡Y están tan tranquilos!\ldots{}
Como que no han probado aún las recetas del brigadier Rotten\ldots{}

---Ojo por ojo y diente por diente---dijo el cautivo contemplando el
grupo de prisioneros.---¡Ah, gran canalla!, no se entierran hombres
impunemente durante seis meses, no se baila encima de su sepultura para
atormentarle, no se les insulta por la reja, no se les arroja saliva e
inmundicia, sin sentir más tarde o más temprano la mano justiciera que
baja del cielo.

Después callaron todos. No se oía más que el rasgueo de la pluma con que
uno de los oficiales escribía, teniendo el papel sobre una cartera y
esta sobre sus rodillas. Cuando hubo concluido, el cautivo rogó que se
le diese lo necesario para escribir una carta a su madre, anunciándole
que vivía, pues, según dijo, en todo el tiempo de su ya concluida
cautividad no había podido dar noticia de su existencia a los que le
amaban.

---¿Vivirán como yo---dijo tristemente,---o afligidos por mi
desaparición habrán muerto?

---Dispénseme usted---manifestó Seudoquis,---pero a medida que hablamos,
me ha parecido reconocer en usted a una persona con quien hace algunos
años tuve relaciones.

---Sí, Sr.~Seudoquis---dijo el cautivo sonriendo.---El mismo soy.
Conspiramos juntos el año 19 y a principios del año 20.

---Señor Monsalud---exclamó el oficial abrazándole,---buen hallazgo
hemos hecho sacándole a usted de aquella mazmorra. ¡Ya se ve! ¿Cómo
podría conocerle, si está usted hecho un esqueleto?\ldots{} Además en
estos tiempos se olvida pronto. ¡He visto tanta gente desde aquellos
felices días!\ldots{} porque eran felices, sí. Aunque sea entre
peligros, el conspirar es siempre muy agradable, sobre todo si se tiene
fe.

---Entonces tenía yo mucha fe.

---¡Ah! Y yo también. Me hubiera dejado descuartizar por la libertad.

---¡Con qué afán trabajábamos!

---Sí; ¡con qué afán!

---¡Nos parecía que de nuestras manos iba a salir acabada y completa la
más liberal y al mismo tiempo la más feliz Nación de la tierra!

---Sí, ¡qué ilusiones!\ldots{} Si no estoy trascordado, también nos
hallamos juntos en la logia de la calle de las Tres Cruces.

---Sí; allí estuve yo algún tiempo. En aquello nunca tuve mucha fe.

---Yo sí; pero la he perdido completamente. Vea usted en qué han venido
a parar aquellas detestables misas masónicas.

---Nunca tuve ilusiones respecto a la Orden de la \emph{Viuda}.

---Pues nosotros---dijo Seudoquis riendo,---tuvimos hasta hace poco en
el regimiento nuestra caverna de Adorinam. Pero apenas funcionaba ya.
¡Cuánta ruina, amigo mío!\ldots{} ¡Cómo se ha desmoronado aquel
fantástico edificio que levantamos!\ldots{} Yo he sido de los que con
más gana, con más convicción y hasta con verdadera ferocidad han
gritado: \emph{¡Constitución o muerte!} Hábleme usted con franqueza,
Salvador, ¿tiene usted fe?

---Ninguna---repuso el cautivo,---pero tengo odio, y por el odio que
siento contra mis carceleros, estoy dispuesto a todo, a morir matando
facciosos, si el general Mina quiere hacerme un hueco entre sus
soldados.

---Pues yo---manifestó Seudoquis con frialdad,---no tengo fe; tampoco
tengo odio muy vivo; pero el deber militar suplirá en mí la falta de
estas dos poderosas fuerzas guerreras. Pienso batirme con lealtad y
llevar la bandera de la Constitución hasta donde se pueda.

---Eso no basta---dijo Monsalud moviendo la cabeza.---Para este
conflicto nacional se necesita algo más\ldots{} En fin, Dios dirá.

Y empezó a escribir a su madre.

\hypertarget{vii}{%
\chapter{VII}\label{vii}}

Después de dar noticia de su estupenda liberación, exponiendo con
brevedad los padecimientos del largo cautiverio que había sufrido,
escribió las frases más cariñosas y una patética declaración de
arrepentimiento por su desnaturalizada conducta y la impía fuga que tan
duramente había castigado Dios. Manifestando después su falta de
recursos y que más que un viaje a Madrid le convenía su permanencia en
el ejército de Cataluña, rogaba a su madre que vendiese cuanto había en
la casa, y juntamente con Solita, se trasladase a la Puebla de Arganzón,
donde pasaría a verlas, pidiendo una licencia. Concluía indicando la
dirección que debía darse a las cartas de respuesta, y pedía que esta
fuera inmediata para calmar la incertidumbre y afán de su alma.

Aquella misma tarde habló con el brigadier Rotten, el cual era un hombre
muy rudo y fiero, bastante parecido en genio y modos a don Carlos
España. Aconsejole este que viera al general Mina, en cuyo ejército
había varias partidas de contraguerrilleros, organizadas
disciplinariamente; añadió que él (el brigadier Rotten) se había
propuesto hacer la guerra de exterminio, quemando, arrasando y
fusilando, en la seguridad de que la supresión de la humanidad traería
infaliblemente el fin del absolutismo, y concluyó diciendo que pasaba a
la provincia de Tarragona con todas las fuerzas de su mando, excepción
hecha del batallón de Murcia, que le había sido reclamado por el general
en jefe para reforzar el sitio de la Seo. Monsalud, sin vacilar en su
elección, optó por seguir a los de Murcia que iban hacia la Seo.

Salió, pues, Murcia al día siguiente muy temprano en dirección a
Castellar, llevando el triste encargo de conducir a los catorce
prisioneros de San Llorens de Morunys. Seudoquis no ocultó a Salvador su
disgusto por comisión tan execrable; pero ni él ni sus compañeros podían
desobedecer al bárbaro Rotten. Púsose en marcha el regimiento, que más
bien parecía cortejo fúnebre, y en uno de sus últimos carros iba
Monsalud, viendo delante de sí a los infelices cautivos atraillados,
algunos medio desnudos, y todos abatidos y llorosos por su miserable
destino, aunque no se creían condenados a muerte, sino tan sólo a
denigrante esclavitud.

Camino más triste no se había visto jamás. Lleno de fango el suelo,
cargada de neblina la atmósfera, y enfriada por un remusguillo helado
que del Pirineo descendía, todo era tristeza fuera y dentro del alma de
los soldados. No se oían ni las canciones alegres con que estos suelen
hacer menos pesadas las largas marchas, ni los diálogos picantes, ni más
que el lúgubre compás de los pasos en el cieno y el crujir de los lentos
carros y los suspiros de los acongojados prisioneros. El día se acabó
muy pronto a causa de la niebla que, al modo de envidia, lo empañaba; y
al llegar a un ángulo del camino, en cierto sitio llamado \emph{Los tres
Roures} (los tres robles), el regimiento se detuvo. Tomaba aliento,
porque lo que iba a hacer era grave.

Salvador sintió un súbito impulso en su alma cristiana. Eran los
sentimientos de humanidad que se sobreponían al odio pasajero y al
recuerdo de tantas penas. Cuando vio que la horrible sentencia iba a
cumplirse, hundió la cabeza sepultándola entre los sacos y mantas que
llenaban el carro, y oró en silencio. Los ayes lastimeros y los tiros
que pusieron fin a los ayes, le hicieron estremecer y sacudirse, como si
resonaran en la cavidad de su propio corazón. Cuando todo quedó en
lúgubre silencio, alzando su angustiada cabeza, dijo así:

---¡Qué cobarde soy! El estado de mi cuerpo, que parece de vidrio, me
hace débil y pusilánime como una mujer\ldots{} No debo tenerles lástima,
porque me sepultaron durante seis meses, porque bailaron sobre mi
calabozo y me injuriaron y escupieron, porque ni aun tuvieron la caridad
de darme muerte, sino por el contrario, me dejaban vivir para
mortificarme más.

El regimiento siguió adelante, y al pasar junto al lugar de la
carnicería, Salvador sintió renacer su congoja.

---Es preciso ser hombre---pensó.---La guerra es guerra, y exige estas
crueldades. Es preciso ser verdugo que víctima. O ellos o nosotros.

Seudoquis se acercó entonces para informarse de su estado de salud.
Estaba el buen capitán tan pálido como los muertos, y su mano, ardiente
y nerviosa temblaba como la del asesino que acaba de arrojar el arma
para no ser descubierto.

---¿Qué dice usted, amigo mío?---le preguntó Salvador.

---Digo---repuso el militar tristemente,---que la Constitución será
vencida.

\hypertarget{viii}{%
\chapter{VIII}\label{viii}}

Hasta el 25 de Enero no llegaron a Canyellas donde Mina tenía su cuartel
general, frente a la Seo de Urgel. Habían pasado más de sesenta días
desde que puso sitio a la plaza, y aunque la Regencia se había puesto en
salvo llevándose el dinero y los papeles, los testarudos catalanes y
aragoneses se sostenían fieramente en la población, en los castillos y
en la formidable ciudadela.

Mina, hombre de mucha impaciencia, tenía en aquellos días un humor de
mil demonios. Sus soldados estaban medio desnudos, sin ningún abrigo y
con menos ardor guerrero que hambre. A los cuarenta y seis cañones que
guarnecían las fortalezas de la Seo, el héroe navarro no podía oponer ni
una sola pieza de artillería. El país en que operaba era tan pobre y
desolado, que no había medios de que sobre él, como es costumbre,
vivieran las tropas. Por carecer estas de todo, hasta carecían de
fanatismo, y el grito de \emph{Constitución o muerte} hacía ya muy poco
efecto. Era como los cumplimientos, que todo el mundo los dice y nadie
cree en ellos. Un invierno frío y crudo completaba la situación,
derramando nieves, escarchas, hielos y lluvia sobre los sitiadores, no
menos desabrigados que aburridos.

Delante de la miserable casilla que le servía de alojamiento solía
pasearse D. Francisco por las tardes con las manos en los bolsillos de
su capote, y pisando fuerte para que entraran en calor las entumecidas
piernas. Era hombre de cuarenta y dos años, recio y avellanado, de
semblante rudo, en que se pintaba una gran energía, y todo su aspecto
revelaba al guerreador castellano, más ágil que forzudo. En sus ojos,
sombreados por cejas muy espesas, brillaba la astuta mirada del
guerrillero que sabe organizar las emboscadas y las dispersiones. Tenía
cortas patillas, que empezaban a emblanquecer, y una piel bronca; las
mandíbulas, así como la parte inferior de la cara, muy pronunciadas; la
cabeza cabelluda y no como la de Napoleón, sino piriforme y amelonada a
lo guerrillero. No carecía de cierta zandunga su especial modo de
sonreír, y su hablar era como su estilo, conciso y claro, si bien no muy
elegante; pero si no escribía como Julio César, solía guerrear como él.

No le educaron sus mayores sino los menores de su familia, y tuvo por
maestro a su sobrino, un seminarista calaverón que empezó su carrera
persiguiendo franceses y la acabó fusilado en América. Se hizo general
como otros muchos, y con mejores motivos que la mayor parte, educándose
en la guerra de la Independencia, sirviendo bien y con lealtad, ganando
cada grado con veinte batallas y defendiendo una idea política con
perseverancia y buena fe. Su destreza militar era extraordinaria, y fue
sin disputa el primero entre los caudillos de partidas, pues tenía la
osadía de Merino, el brutal arrojo del Empecinado, la astucia de Albuín
y la ligereza del Royo. Sus crueldades, de que tanto se ha hablado, no
salían, como las de Rotten, de las perversidades de un corazón duro,
sino de los cálculos de su activo cerebro, y constituían un plan como
cualquier otro plan de guerra. Supo hacerse amar de los suyos hasta el
delirio, y también sojuzgar a los que se le rebelaron como el Malcarado.

Poseía el genio navarro en toda su grandeza, siendo guerrero en cuerpo y
alma, no muy amante de la disciplina, caminante audaz, cazador de
hombres, enemigo de la lisonja, valiente por amor a la gloria, terco y
caprichudo en los combates. Ganó batallas que equivalían a romper una
muralla con la cabeza, y fueron obras maestras de la terquedad, que a
veces sustituye al genio. En sus crueldades jamás cometió viles
represalias, ni se ensañó, como otros, en criaturas débiles. Peleando
contra Zumalacárregui, ambos caudillos cambiaron cartas muy tiernas a
propósito de una niña de quince meses que el guipuzcoano tenía en poder
del navarro. Fuera de la guerra, era hombre cortés y fino, desmintiendo
así la humildad de su origen, al contrario de otros muchos, como D. Juan
Martín, por ejemplo, que, aun siendo general, nunca dejó de ser
carbonero.

Salvador Monsalud había conocido a Mina en 1813, durante la
conspiración, y después en Madrid. Su amistad no era íntima, pero sí
cordial y sincera. Oyó el general con mucho interés el relato de las
desgracias del pobre cautivo de San Llorens, y a cada nueva crueldad que
este refería, soltaba el otro alguna enérgica invectiva contra los
facciosos.

---Ya tendrá usted ocasión de vengarse, si persiste en su buen propósito
de ingresar en mi ejército---le dijo, estrechándole la mano.---Yo tengo
aquí varias partidas de contraguerrilleros, compuestas de gentes del
país y de compatriotas míos que me ayudan como pueden. Desde luego le
doy a usted el mando de una compañía; ¿acepta usted?

---Acepto---repuso Salvador.---Nunca fue grande mi afición a la carrera
militar; pero ahora me seduce la idea de hacer todo el daño posible a
mis infames verdugos, no asesinándolos, sino venciéndolos\ldots{} Este
es el sentimiento de que han nacido todas las guerras. Además yo no
tengo nada que hacer en Madrid. El duque del Parque no se acordará ya de
mí y habrá puesto a otro en mi lugar. He rogado a mi madre que venda
todo y se traslade a la Puebla con mi hermana. No quiero Corte por
ahora. Las circunstancias, y una inclinación irresistible que hay dentro
de mí desde que me sacaron de aquel horrible sepulcro, me impulsan a ser
guerrillero.

---Eso no es más que vocación de general---dijo Mina riendo.

Después convidó a Monsalud a su frugal mesa, y hablaron largo rato de la
campaña y del sitio emprendido, que según las predicciones del general,
tocaba ya a su fin.

---Si para el día de la Candelaria no he entrado en esa cueva de
ladrones ---dijo,---rompo mi bastón de mando\ldots{} Daría todos mis
grados por podérselo romper en las costillas a Mataflorida.

---O al arzobispo de Creux.

---Ese se pone siempre fuera de tiro. Ya marchó a Francia por miedo a la
chamusquina que les espera. ¡Ah! Sr.~Monsalud, si no es usted hombre de
corazón, no venga con nosotros. Cuando entremos en la Seo, no pienso
perdonar ni a las moscas. El Trapense, al tomar esta plaza, pasó a
cuchillo la guarnición. Yo pienso hacer lo mismo.

---¿A qué cuerpo me destina mi general?

---A la contraguerrilla del \emph{Cojo de Lumbier}. Es un puñado de
valientes que vale todo el oro del mundo.

---¿En dónde está?

---Hacia Fornals, vigilando siempre la Ciudadela. Los contraguerrilleros
del Cojo han jurado morir todos o entrar en la Ciudadela antes de la
Candelaria. Me inspiran tal confianza, que les he dicho: «no tenéis que
poneros delante de mí sino para decirme que la Ciudadela es nuestra.»

---Entrarán, entraremos de seguro---dijo Monsalud con entusiasmo.

---Y ya les he leído muy bien la cartilla---añadió Mina.---Ya les he
cantado muy claro que no tienen que hacerme prisioneros. No doy cuartel
a nadie, absolutamente a nadie. Esa turba de sacristantes y salteadores
no merece ninguna consideración militar.

---Es decir\ldots{}

---Que me haréis el favor de pasarme a cuchillo a toda esa gavilla de
tunantes\ldots{} Amigo mío, la experiencia me ha demostrado que esta
guerra no se sofoca sino con la ley del exterminio llevada a su último
extremo.

Salvador, oyendo esto, se estremeció, y por largo rato no pudo apartar
de su pensamiento la lúgubre fase que tomaba la guerra desde que él
imaginó poner su mano en ella.

Mina encargó al novel guerrillero que procurara restablecerse dándose la
mejor vida posible en el campamento, pues tiempo había de sobra para
entrar en lucha, si continuaba la guerra, como era creíble en vista del
estado del país y de los amagos de intervención. Otros amigos, además
del general, encontró Salvador en Canyellas y pueblos inmediatos;
relaciones hechas la mayor parte en la conspiración y fomentadas después
en las logias y en los cafés patrióticos.

\hypertarget{ix}{%
\chapter{IX}\label{ix}}

La Seo de Urgel está situada en la confluencia de dos ríos que allí son
torrentes: el Segre, originario de Puigcerdá, y el Balira, un bullicioso
y atronador joven enviado a España por la República de Andorra. Enormes
montañas la cercan por todas partes y tres gargantas estrechas le dan
entrada por caminos que entonces sólo eran a propósito para la segura
planta del mulo. Sobre la misma villa se eleva la Ciudadela; más al
Norte el {\textsc{Castillo}}; entre estas dos fortalezas, el escarpado
arrabal de Castel-Ciudad, y en dirección a Andorra la torre de Solsona.
La imponente altura de estas posiciones hace muy difícil su expugnación,
es preciso andar a gatas para llegar hasta ellas.

El 29 Mina dispuso que se atacara a Castel-Ciudad. El éxito fue
desgraciado; pero el 1.º de Febrero, operando simultáneamente todas las
tropas contra Castel-Ciudad, Solsona y el Castillo, se logró poner
avanzadas en puntos cuya conquista hacía muy peligrosa la resistencia de
los sitiados. Por último, el día 3 de Febrero, a las doce de la mañana,
las contraguerrillas del \emph{Cojo} y el regimiento de Murcia
penetraban en la Ciudadela, defendida por seiscientos hombres al mando
de Romagosa.

Aunque no se hallaba totalmente restablecido, Salvador Monsalud volvía
tan rápidamente a su estado normal, que creyó de su deber darse de alta
en los críticos días 1.º y 2.º de Febrero. Además de que se sentía
regularmente ágil y fuerte, le mortificaba la idea de que se le
supusiera más encariñado con la convalecencia que con las balas. Tomó,
pues, el mando de su compañía de contraguerrillas, a las órdenes del
valiente \emph{Cojo de Lumbier}, y fue de los primeros que tuvieron la
gloria de penetrar en la Ciudadela. Sin saber cómo, sintiose dominado
por la rabiosa exaltación guerrera que animaba a su gente. Vio los
raudales de sangre y oyó los salvajes gritos, todo ello muy acorde con
su excitado espíritu.

Cuando la turba vencedora cayó como una venganza celeste sobre los
vencidos, sintió, sí, pasajero temblor; pero sobreponiéndose a sus
sentimientos, recordó las instrucciones de Mina y supo transmitir las
órdenes de degüello, con tanta firmeza como el cirujano que ordena la
amputación. Vio pasar a cuchillo a más de doscientos hombres en la
Ciudadela y no pestañeó; pero no pudo vencer una tristeza más honda que
todas las tristezas imaginables, cuando Seudoquis, acercándose a él
sobre charcos de sangre y entre los destrozados cuerpos palpitantes, le
dijo con la misma expresión lúgubre de la tarde de los tres Roures:

---Me confirmo en mi idea, amigo Monsalud. La Constitución será vencida.

Al día siguiente bajó a la villa de la Seo, que le pareció un sepulcro
del cual se acabara de sacar el cuerpo putrefacto. Su estrechez lóbrega
y húmeda, así como su suciedad hacían pensar en los gusanos insaciables,
y no se podía entrar en ella con ánimo sereno. Como oyera decir que en
los claustros de la catedral, convertidos en hospital, había no pocas
personas de Madrid, se dirigió allá creyendo encontrar algún amigo de
los muchos y diversos que tenía. Grande era el número de heridos y
enfermos; mas no vio ningún semblante conocido. En el palacio arzobispal
estaban sólo los enfermos de más categoría. Dirigiose allá y apenas
había dado algunos pasos en la primera sala, cuando se sintió llamado
enérgicamente.

Miró y dos nombres sonaron.

---¡Salvador!

---¡Pipaón!

Los dos amigos de la niñez, los dos colegas de la conspiración del 19,
los dos hermanos, aunque no bien avenidos de la logia de las Tres
Cruces, se abrazaron con cariño. El buen Bragas, que poco antes, viendo
malparada la causa constitucional, había corrido a la Seo a ponerse a
las órdenes de la Regencia, cual hombre previsor, padecía de un
persistente reúma que le impidió absolutamente huir a la aproximación de
las tropas liberales. Confiaba el pobrecito en las infinitas trazas de
su sutilísimo ingenio para conseguir que no se le causara daño, y como
tuvo siempre por norte hacerse amigos, aunque fuera en el infierno, muy
mal habían de venir las cosas para que no saliese alguno entre los
soldados de Mina. A pesar de todo, estuvo con el alma en un hilo hasta
que vio aparecer la figura por demás simpática de su antiguo camarada, y
entonces no pudiendo contener la alegría, le llamó y después de
estrecharle en sus brazos con la frenética alegría del condenado que
logra salvarse, le dijo:

---¡Qué bonita campaña habéis hecho!\ldots{} Habéis tomado la Seo como
quien coge un nido de pájaros\ldots{} Si he de ser franco contigo, me
alegro\ldots{} no se podía vivir aquí con esa canalla de
Regencia\ldots{} Yo vine por cuenta del Gobierno constitucional a
vigilar\ldots{} ya tú me entiendes; y me marchaba, cuando\ldots{} ¡Qué
desgraciado soy! Pero supongo que no me harán daño alguno, ¿eh?\ldots{}
¿Tienes influencia con Mina?\ldots{} Dile que podré ponerle en autos de
algunas picardías que proyectan los Regentes. Te juro que diera no sé
qué por ver colgado de la torre al arzobispo.

Monsalud después de tranquilizarle pidiole noticias de Madrid y de su
familia.

Pipaón permaneció indeciso breve rato, y después añadió con su habitual
ligereza de lenguaje:

---¿Pero dónde te has metido? ¿Te secuestraron los facciosos? Ya me lo
suponía, y así lo dije a tu pobre madre cuando estuvo en mi casa a
preguntarme por ti. La buena señora no tenía consuelo. Se comprende. ¡No
saber de ti en tanto tiempo!\ldots{}

¿Vive mi madre?---preguntó Salvador.---¿Está buena?

---Hace algunos días que falto de Madrid y no te puedo contestar---dijo
Bragas mascullando las palabras,---pero si recibieses alguna mala
noticia no debes sorprenderte. Tu ausencia durante tantos meses y la
horrible incertidumbre en que ha vivido tu buena madre, no son
ciertamente garantías de larga vida para ella.

---Pipaón, por Dios---dijo Monsalud con amargura,---tú me ocultas algo;
tú, por caridad no quieres decirme todo lo que sabes. ¿Vive mi madre?

---No puedo afirmar que sí ni que no.

---¿Cuándo la has visto?

---Hace cuatro meses.

---¿Y entonces estaba buena?

---Así, así\ldots{}

---¿Y Sola estaba buena?

---Así, así. Las dos parecían tan apesadumbradas, que daba pena verlas.

---¿Seguían viviendo en el Prado, donde yo las dejé?

---No, volvieron a la calle de Coloreros\ldots{} Comprendo tu ansiedad.
Si no hubiera huido con la Regencia una persona que se toma interés por
ti, que te nombra con frecuencia, y que hace poco ha llegado de
Madrid\ldots{}

---¿Quién?

---Jenara.

---¿Ha estado aquí?\ldots{} No me dices nada que no me abrume, Pipaón.

---Marchó con el arzobispo y Mataflorida. ¡Qué guapa está! Y conspira
que es un primor. Sólo ella se atrevería a meterse en Madrid, llevando
mensajes de esta gente de la frontera, como hizo en la primavera pasada,
y volver locos a los Ministros y a la camarilla\ldots{} Pero te has
puesto pálido al oír su nombre\ldots{} Ya, ya sé que os queréis bien.
Ella misma ha dejado comprender ciertas cosas\ldots{} ¡Cuánto ha
padecido por arrancar de la facción a un hombre secuestrado en
Benabarre! Ese hombre eres tú. Bien claro me lo ha dado a entender ella
con sus suspiros siempre que te nombraba, y tú con esa palidez teatral
que tienes desde que hablamos de ella. Amiguito, bien, bravo; mozas de
tal calidad bien valen seis meses de prisión. A doce me condenaría yo
por haber gustado esa miel hiblea.

Y prorrumpió en alegres risas, sin que el otro participase de su
jovialidad. Reclinado en la cama del enfermo, la cabeza apoyada en la
mano, Monsalud parecía la imagen de la meditación. Después de larga
pausa, volvió a anudar el hilo del interrumpido coloquio, diciendo:

---¿Conque ha estado aquí hace poco?

---Sí; ¿ves esta cinta encarnada que tengo en el brazo?\ldots{} Ella me
la puso para sujetarme la manga que me molestaba. Si quieres este
recuerdo suyo te lo puedo ceder en cambio de la protección que me
dispensas ahora.

Salvador miró la cinta, pero no hizo movimiento alguno para tomarla, ni
dijo nada sobre aquel amoroso tema.

---¿Y dices que hizo esfuerzos por rescatarme?---preguntó.

---Sí\ldots{} ¡pobre mujer! Se me figura que te amó grandemente; pero
acá para entre los dos, no creo que la primera virtud de Jenara sea la
constancia\ldots{} Si tanto empeño tenía por salvarte, ¿por qué no te
salvó, siendo, como era, amiga de Mataflorida, del arzobispo y del
barón? Con tomar una orden de la Regencia y dirigirse al interior del
país dominado por los arcángeles de la fe\ldots{} Pero no había quien la
decidiera a dar este paso, y antes que meterse entre guerrilleros, me
dijo una vez que prefería morir.

---Y ¿crees tú que ella podría darme noticias de mi familia?

---Se me figura que sí---dijo Pipaón poniendo semblante compungido.---Yo
le oí ciertas cosas\ldots{} No será malo, querido amigo, que te
dispongas a recibir alguna mala noticia.

---Dímela de una vez, y no me atormentes con tus medias
palabras---manifestó Salvador lleno de ansiedad.

---De este mundo miserable---añadió Bragas con una gravedad que no le
sentaba bien,---¿qué puede esperarse más que penas?

---¡Ya lo sé! Jamás he esperado otra cosa.

---Pues bien\ldots{} Yo supongo que tú eres un hombre valiente\ldots{}
¿Para qué andar con rodeos y palabrillas?

---Es verdad.

---Si al fin había de suceder; si al fin habías de apurar este cáliz de
amargura\ldots{} ¡Ah, mi querido amigo, siento ser mensajero de esta
tristísima nueva!

---¡Oh, Dios mío, lo comprendo todo!\ldots---exclamó Salvador ocultando
su rostro entre las temblorosas manos.

---¡Tu madre ha muerto!---dijo Pipaón.

---¡Oh, bien me lo decía el corazón!---balbució el huérfano traspasado
de dolor.---¡Madre querida!, ¡yo te he matado!

Durante largo rato estuvo llorando amargamente.

\emph{Creyendo ahora conveniente el autor no trabajar más por cuenta
propia, vuelve a utilizar el manuscrito de la señora en su segunda
pieza, que concuerda cronológicamente con el punto en que se ha
suspendido la anterior relación}.

\emph{Los lectores perdonarán esta larga incrustación ripiosa, tan
inferior a lo escrito por la hermosa mano y pensado por el agudo
entendimiento de la señora. Pero como la seguridad del edificio de esta
historia lo hacía necesario, el autor ha metido su tosco ladrillo entre
el fino mármol de la gentil dama alavesa. El segundo fragmento lleva por
título}: {\textsc{De París a Cádiz,}} \emph{y a la letra dice así:}

\hypertarget{x}{%
\chapter{X}\label{x}}

A fines de Diciembre del 22, tuve que huir precipitadamente de la Seo,
que amenazaba el cabecilla Mina. No es fácil salir con pena de la Seo.
Aquel pueblo es horrible, y todo el que vive dentro de él se siente
amortajado. Mataflorida salió antes que nadie, trémulo y lleno de
zozobra. No podré olvidar nunca la figura del arzobispo, montando a
mujeriegas en un mulo, apoyando una mano en el arzón delantero y otra en
el de atrás, y con la canaleja sujeta con un pañuelo para que no se la
arrancase el fuerte viento que soplaba. Es sensible que no pueda una
dejar de reírse en circunstancias tristes y luctuosas, y que a veces las
personas más dignas de veneración por su estado religioso, exciten la
hilaridad. Conozco que es pecado y lo confieso; pero ello es que yo no
podía tener la risa.

Nos reunimos todos en Tolosa de Francia. Yo resolví entonces no
mezclarme más en asuntos de la Regencia. Jamás he visto un desconcierto
semejante. Muchos españoles emigrados, viendo cercana la intervención
(precipitada por las altaneras contestaciones de San Miguel), temblaban
ante la idea de que se estableciese un absolutismo fanático y vengador,
y suspiraban por una transacción, interpretando el pensamiento de Luis
XVIII. Pero no había quien apease a Mataflorida de su borrica, o sea de
su idea de restablecer las cosas \emph{en el propio ser y estado que
tuvieron} desde el 10 de Mayo de 1814 hasta el 7 de Marzo de 1820.
Balmaseda le apoyaba, y D. Jaime Creux (el gran jinete de quien antes he
hablado) era partidario también del absolutismo puro y sin mancha alguna
de Cámaras ni camarines; pero el barón de Eroles y Eguía se oponían
furiosamente a esta salutífera idea de sus compañeros.

Mi amigo, el general de la coleta (ya separado de la pastelera de
Bayona) quería destituir a la Regencia y prender a Mataflorida y al
arzobispo. Mataflorida, fuerte con las instrucciones reservadísimas de
Su Majestad, que yo y otros emisarios le habíamos traído, seguía en sus
trece. La Junta de Cataluña, los apostólicos de Galicia, la Junta de
Navarra, los obispos emigrados enviaban representaciones a Luis XVIII
para que reconociese a la Regencia de Urgel, mientras la Regencia misma,
echándosela de soberana, enviaba una especie de plenipotenciarios de
figurón a los Soberanos de Europa.

Nada de esto hizo efecto, y la Corte de Francia, conforme con Eguía y el
barón de Eroles, puso a la Regencia cara de hereje. Por desgracia para
la causa real Ugarte había sido quitado de la escena política, y todo el
negocio, como puede suponerse, andaba en manos muy ineptas. Allí era de
ver la rabia de Mataflorida, que alegaba en su favor las órdenes
terminantes del Rey; pero nada de esto valía, porque los otros también
mostraban cartas y mandatos reales. Fernando jugaba con todos los dados
a la vez. ¿Su voluntad quién podía saberla?

Entre tanto todo se volvía recados misteriosos de Tolosa a París y a
Madrid y a Verona. Eguía se carteaba con el duque de Montmorency,
ministro de Estado en Francia, y Mataflorida con Chateaubriand. Cuando
este sustituyó a Montmorency en el Ministerio, nuestro marqués vio el
cielo abierto, por ser el vizconde de los que con más ahínco habían
sostenido en Verona la necesidad de volver del revés las instituciones
españolas. Necesitando negociar con él y no queriendo apartarse de la
frontera de España por temor a las intrigas de Eguía y del barón de
Eroles, me rogó que le sirviese de mensajero, a lo que accedí gustosa,
porque me agradaban, ¿a qué negarlo?, aquellos graciosos manejos de la
diplomacia menuda, y el continuo zarandeo y el trabar relaciones con
personajes eminentes, Príncipes y hasta soberanos reinantes. Yo, dicho
sea sin perjuicio de la modestia, había mostrado regular destreza para
tales tratos, así como para componer hábilmente una intriga; y el hábito
de ocuparme en ello había despertado en mí lo que puede llamarse el amor
al arte. Mi belleza, y cierta magia que, según dicen, tuve, contribuían
no poco entonces al éxito de lo que yo nombraba plenipotencias de
abanico.

Tomé, pues, mis credenciales y partí para París con mi doncella y dos
criados excelentes que me proporcionó Mataflorida. Estaba en mis
glorias. Felizmente yo hablaba el francés con bastante soltura, y tenía
en tan alto grado la facultad de adaptación, que a medida que pasaba de
Tolosa a Agen, de Agen a Poitiers, de Poitiers a Tours y a París,
parecíame que me iba volviendo francesa en maneras, en traje, en figura
y hasta en el modo de pensar.

Llegué a la gran ciudad ya muy adelantado Febrero. Tomé habitación en la
calle del Bac, y después de destinar dos días a recorrer las tiendas del
Palais Royal y a entablar algunas relaciones con modistas y joyeros,
pedí una audiencia al señor Ministro de Negocios Exteriores. Él, que ya
tenía noticia de mi llegada, enviome uno de sus secretarios, dignándose
al mismo tiempo ofrecerme un billete para presenciar la apertura de las
tareas legislativas en el Louvre.

Mucho me holgué de esto, y dispúseme a asistir a tan brillante
ceremonia, en la cual debía leer su discurso el Rey Luis XVIII y
presentarse de corte todos los grandes dignatarios de aquella fastuosa
Monarquía. Confieso que jamás he visto ceremonia que más me
impresionase. ¡Qué solemnidad, qué grandeza y lujo! El puesto en que me
colocaron los ujieres no era el más cómodo; pero vi perfectamente todo,
y la admiración y arrobamiento de mi espíritu no me permitían atender a
las molestias.

La presencia del anciano Rey me causó la sensación más viva. Aclamáronle
ruidosamente cuando apareció en el gran salón, y en realidad, inspiraba
afecto y entusiasmo. Bien puede decirse que pocos reyes han existido más
simpáticos ni más dignos de ser amados. Luis XVIII tomó asiento en un
trono sombreado con rico dosel de terciopelo carmesí. Los altos
dignatarios se colocaron en pie en los escaños alfombrados. No se verá
en parte alguna nada más grave ni más suntuoso ni más imponente.

Su Majestad Cristianísima empezó a leer. ¡Qué voz tan dulce, qué acento
tan patético! A cada párrafo era interrumpido por vivas exclamaciones.
Yo lloraba y atendía con toda mi alma. Se me grabaron profundamente en
la memoria aquellas célebres palabras: «He mandado retirar mi embajador.
Cien mil franceses, mandados por un Príncipe de mi familia, por aquel a
quien mi corazón se complace en llamar hijo, están a punto de marchar
invocando al Dios de San Luis para conservar el trono de España a un
descendiente de Enrique IV, para librar a aquel hermoso reino de su
ruina y reconciliarlo con Europa.»

Ruidosos y entusiastas vítores manifestaron cuánto entusiasmaba a todos
los franceses allí presentes la intervención. Yo, aunque española,
comprendía la justicia y necesidad de esta medida. Así es que dije para
mí, pensando en mis paisanos:

---Ahora veréis, brutos, cómo os harán andar derechos.»

Pero el bondadoso Luis XVIII siguió diciendo cosas altamente patrióticas
sólo bajo el punto de vista francés, y ya aquello no me gustaba tanto;
porque, en fin, empecé a comprender que nos trataban como a un hato de
carneros. He sido siempre de una volubilidad extraordinaria en mis
ideas, las cuales varían al compás de los sentimientos que agitan
hondamente mi alma. Así es que de pronto, y sin saber cómo se enfrió un
poco mi entusiasmo; y cuando Luis dijo con altanero acento y entre
atronadores aplausos aquello de \emph{Somos franceses, señores}, sentí
oprimido mi corazón; sentí que corría por mis venas rápido fuego, y
pensando en la intervención, dije para mí:

---No hay que echar mucha facha todavía, amiguitos. \emph{Somos
españoles, señores}.

Pero no puedo negar que la pompa de aquella Corte, la seriedad y
grandeza de aquella Asamblea, acorde con su Rey, y existente con él sin
estorbarse el uno a la otra, hicieron grande impresión en mi espíritu.
Me acordaba de las discordias infecundas de mi país, y entonces sentía
pena.

---Allá---pensé,---tenemos demasiadas Cortes para el Rey y demasiado Rey
para las Cortes.

El día siguiente, 1.º de Marzo, era el señalado por Chateaubriand para
recibirme. Yo tenía vivísimos deseos de verle, por dos motivos: por mi
comisión y porque había leído la \emph{Atala} poco antes, hallando en su
lectura profundo deleite. No sé por qué me figuraba al vizconde como una
especie de \emph{triste Chactas}, de tal modo que no podía pensar en él
sin traer a la memoria la célebre canción.

Pero todo cambió cuando entré en el Ministerio y en el despacho del
célebre escritor que llenaba el mundo con su nombre y había divulgado la
manía de los bosques de América el sentimentalismo católico y las
tristezas quejumbrosas a lo René. Vestía de gran uniforme. Su semblante
pálido y hermoso no tenía más defecto que el estudiado desorden de los
cabellos, que asemejaban su cabeza a una de esas testas de aldeano en
cuya selvática espesura jamás ha entrado el peine. En sus ojos había un
mirar tan vivo y penetrante, que me obligaba a bajar los míos. Estaba
bastante decaído, aunque su edad no pasara entonces de los cincuenta y
dos años. Su exquisita urbanidad era algo finchada y fría. Sonreía
ligeramente y pocas veces, contrayendo los casi imperceptibles pliegues
de su boca de mármol; pero fruncía con frecuencia el ceño, como una maña
adquirida por la costumbre de creer que cuanto veía era inferior a la
majestad de su persona.

Pareciome que la presencia de la diplomática española le había causado
sorpresa. Sin duda creía ver en mí una \emph{maja} de esas que, conforme
él dice en uno de sus libros, se alimentan con una bellota, una aceituna
o un higo. Debió admirarle mi intachable vestido francés y la falta de
aquella gravedad española que consiste, según ellos, en hablar
campanudamente y con altanería. En sus miradas creí sorprender una
observación algo impropia de hombre tan fino. Pareciome que miraba si
había yo llevado el rosario para rezar en su presencia, o alguna
guitarra para tocar y cantar mientras durase el largo plazo de la
antesala. En sus primeras palabras advertí marcado deseo de llevarme al
terreno literario, porque empezó hablando de lo mucho que admiraba a mi
país y del Romancero del Cid, asunto que no vino muy de molde en aquella
ocasión.

Yo, viéndole en tan buen terreno, y considerando cuánto debía agradarle
la lisonja, me afirmé en el terreno literario y le hablé de su universal
fama, así como del gran eco de Chateaubriand por todo el orbe. Él me
contestó con frases de modestia tan ingeniosas y bien perfiladas, que la
misma modestia no las hubiera conocido por suyas. Preguntome si había
leído el \emph{Genio del Cristianismo}, y le contesté al punto que sí y
que me entusiasmaba, aunque la verdad es que hasta entonces no había ni
siquiera hojeado tal libro; mas recordando algunos pasajes de los
\emph{Mártires}, le hablé de esta obra y de la gran impresión que en mí
produjera. Él pareció maravillado de que una dama española supiera leer,
y me dirigió varias galanterías del más delicado gusto. Por mi belleza y
mis gracias materiales, yo no debía de ser de palo para el vizconde.
Después supe que con cincuenta y dos años a la espalda aún se creía
bastante joven para el galanteo, y amaba a cierta artista inglesa con el
furor de un colegial.

\hypertarget{xi}{%
\chapter{XI}\label{xi}}

Entrando de lleno en nuestro asunto, el \emph{triste Chactas} me dijo:

---Ya oiría usted ayer el discurso de Su Majestad. La guerra es
inevitable. Yo la creo conveniente para las dos Naciones, y he tenido el
honor de sostener esta opinión en el Congreso de Verona y en el
Ministerio, contra muchos hombres eminentes que la juzgaban peligrosa.
En cuanto a la cuestión principal, que es la clase de Gobierno que debe
darse a España, no creo en la posibilidad de sostener el absolutismo
puro. Esto es un absurdo, aun en España, y las luces del siglo lo
rechazan.

Yo le hice una pintura todo lo fiel que me fue posible del estado de
nuestras costumbres y de las clases sociales en nuestro país, así como
de los personajes eminentes que en él había, haciendo notar de paso,
conforme a mi propósito, que un solo hombre grande existía en toda la
redondez de las Españas. Este hombre era el marqués de Mataflorida.

---Reconozco las altas dotes del señor Marqués---me dijo Chateaubriand
con finísima sonrisa.---Pero la conducta de la Regencia de Urgel ha sido
poco prudente. Su manifiesto del 15 de Agosto y sus propósitos de
conservar el absolutismo puro no pueden hallar eco en la Europa
civilizada.

Yo dije entonces, usando las frases más delicadas, que no era fácil
juzgar de los sucesos de Urgel por lo que afirmaran hombres tan
corrompidos como Eguía y el barón de Eroles, a los cuales, con buenas
palabras, puse de oro y azul. Concluí mi perorata afirmando que la
voluntad de Fernando era favorable a los planes de Mataflorida.

---Para nosotros---dijo,---no hay otra expresión de la voluntad del Rey
de España, que la contenida en la carta que Su Majestad Católica dirigió
a nuestro Soberano.

El pícaro me iba batiendo en todos mis atrincheramientos y me
desconcertó completamente cuando me dijo:

---El Gobierno francés ha acordado nombrar una Junta provisional en la
frontera, hasta que las tropas francesas entren en España.

---¿Y la Regencia?

---La Regencia dejará de existir; mejor dicho, ha dejado de existir ya.

---Pero Fernando no le ha retirado sus poderes, antes bien, se los
confirma secretamente un día y otro.

Al oír esto el insigne escritor y diplomático no contestó nada. Conocí
que se veía en la alternativa de desmentir mi aserto o de hablar mal de
Fernando, y que como hombre de intachable cortesía no quería hacer lo
primero, ni como Ministro de un Borbón lo segundo. Viéndole suspenso
insistí, y entonces me dijo:

---Indudablemente aquí hay algo que ahora no se puede comprender; pero
que andando el tiempo se ha de ver con claridad.

Después, deseando mostrarme el más filantrópico interés por la ventura
de nuestro país, afirmó que él había trabajado porque se declarara la
guerra, sosteniendo para esto penosas luchas con Mr.~de Villéle y sus
demás colegas; que la resistencia de Inglaterra y de Wellington habían
exigido de su parte grandes esfuerzos y constancia, y por último, que
aún necesitaba de no poca energía para vencer la oposición a la guerra
que las Cámaras mostrarían desde el primer día de sus sesiones.

---Muchos---añadió \emph{Chactas},---me consideran loco. Otros me tienen
lástima. Algunos, y entre ellos los envidiosos, preguntan si podré yo
conseguir lo que no fue dado a Napoleón. Pero yo fío al tiempo la
consagración de este gran hecho, tan necesario a la seguridad del orden
y la justicia en los pueblos de Occidente.

Habló también de las sociedades secretas y de los carbonarios, a quienes
parecía tener muchísimo miedo; y yo empecé a comprender que el objeto de
la intervención no era poner paz entre nosotros, ni hacernos felices, ni
aun siquiera consolidar el vacilante trono de un Borbón, sino aterrar a
los revolucionarios franceses e italianos que bullían sin cesar en los
tenebrosos fondos de la sociedad francesa, jamás reposada ni tranquila.

Prometió contestar a Mataflorida, mas sin mostrarse muy entusiasta de
las altas prendas de mi amigo, ni indicar nada que trascendiese a
propósitos de acceder a su petición. Bajo sus frases corteses yo creía
descubrir cierto menosprecio de los individuos de la Regencia, y aun de
todos los que mangoneaban en la conspiración. De un solo español me
habló con acento que indicaba respeto y casi admiración, de Martínez de
la Rosa. Atribuí esto a mera simpatía del poeta.

Despedime de él, deplorando el mal éxito de mi embajada, y aquí fue
donde se deshizo en cumplidos, buscando y hallando en su fina habilidad
cortesana ocasión para deslizar dos o tres galanterías con discretos
elogios de mi hermosura y del país \emph{donde florece el naranjo}. Me
había tomado por andaluza y yo le dejé en esta creencia.

A los dos días fue a pagarme la visita a mi alojamiento de la calle del
Bac, y en su breve entrevista me pareció que huía de mencionar los
oscuros asuntos de la siempre oscura España. En los días sucesivos
visité a otras personas, entre ellas al Ministro de lo Interior, Mr.~de
Corbiere, y a algunos señores del partido del conde de Artois, como el
príncipe de Polignac y Mr.~de la Bourdonnais. También tuve ocasión de
tratar a dos o tres viejas aristócratas del barrio de San Germán,
ardientes partidarias de la guerra de España y no muy bien quistas con
el Rey filósofo y tolerante que gobernaba a la Francia, convaleciente
aún de la Revolución y del Imperio. De mis conversaciones con toda
aquella gente pude sacar en limpio el siguiente juicio, que creo seguro
y verdadero. Las personas influyentes de la Restauración deseaban para
Francia una Monarquía templada y constitucional fundada en el orden, y
para España el absolutismo puro. Con tal que en Francia hubiera
tolerancia y filosofía, no les importaba que en España tuviéramos
frailes e inquisición. Todo iría bien, siempre que en ninguna de las dos
Naciones hubiese franc-masones, carbonarios y demagogos.

Tenían de nuestro país una idea muy falsa. Cuando Chateaubriand, que era
el genio de la Restauración, decía de España: \emph{allí el matar es
cosa natural, ya sea por amor, ya sea por odio}, puede juzgarse lo que
pensarían todas aquellas personas que no supieron escribir el
\emph{Genio del Cristianismo}. Nos consideraban como un pueblo heroico y
salvaje, dominado por pasiones violentas y por un fanatismo religioso
semejante al del antiguo Egipto.

La princesa de la Tremouille se asombraba de que yo supiera escribir, y
me presentó en su tertulia como un objeto curioso, aunque sin dar a
conocer ningún sentimiento ni idea que me mortificasen. Yo creo que ni
uno solo de sus amigos dejó de enamorarse de mí, ilusionados con la idea
de mi sentimentalismo andaluz y de mi gravedad calderoniana, y de la
mezcla que suponían en mí de maja y de gran señora, de Dulcinea y de
gitana. El más rendido se suponía expuesto a morir asesinado por mí en
un arrebato de celos, pues tal idea tenían de las españolas, que en cada
una de ellas se habían de hallar comprendidas dos personas, a saber: la
cantaora de Sevilla y doña Jimena, la torera que gasta navaja, y la dama
ideal de los romances moriscos. Yo me reía con esto y llevaba adelante
la broma.

Volviendo al asunto de la guerra de España, diré que al salir de París
no tenía duda alguna acerca del pensamiento de los franceses en esta
cuestión. Ellos no hacían la guerra por nuestro bien ni por el de
Fernando. Poco se les importaba que después de vencido el
constitucionalismo, estableciésemos la Carta o el despotismo neto. Allá
nos entenderíamos después con los frailes y los guerrilleros
victoriosos. Su objeto, su bello ideal era aterrar a los revolucionarios
franceses, harto entusiasmados con las demencias de nuestros bobos
liberales, y además dar a la dinastía restaurada el prestigio militar
que no tenía.

El principal enemigo de los Borbones en Francia era el recuerdo de
Bonaparte, y el dejo de aquel dulce licor de la gloria, con cuya
embriaguez se habían enviciado los franceses. Una Monarquía que no daba
batallas de Austerlitz, que no satisfacía de ningún modo el ardor
guerrero de la Nación y que no tocaba el tambor en cualquier parte de
Europa, no podía ser amada de aquel pueblo, en quien la vanidad iguala a
la verdadera grandeza y que tiene tanta presunción como genio. Era
preciso armarla, como decimos en nuestro país; era necesario que la
Restauración tuviera su epopeya chica o grande, aunque esta epopeya
fuese de mentirijillas; era indispensable vencer a alguien, para poder
poner el grito en el cielo y regresar a París con la bambolla de las
conquistas. Dios permitió que el anima vili de este experimento fuésemos
nosotros, y que la desgraciada España, cuya fiereza libró a Europa de
Bonaparte, fuese la víctima escogida para proporcionar a Francia el
desahoguillo marcial que debía poner en olvido a aquel mismo Bonaparte
tan execrado.

Mi viaje a París modificó mucho mis ideas absolutistas en principio, si
bien pensando en España no podía admitir ciertas cosas que en Francia me
parecían bien. Toda la vida me he congratulado de haber visto y hablado
a monsieur de Chateaubriand, el escritor más grande de su tiempo. Aunque
su fama se eclipsó bastante después de la revolución del 30, lo cual
indica que había en su genio mucho tomado a las circunstancias, no puede
negarse que sus obras deleitan y enamoran principalmente por la galanura
de su imaginación y la magia de su estilo; y aún deleitarían más si en
todas ellas no hablase tanto de sí mismo. Tengo muy presente su persona,
por demás agradable, y su rostro simpático y lleno de aquella expresión
sentimental que se puso de moda, haciendo que todos los hombres
pareciesen enamorados y enfermos. Me parece que le estoy mirando, y
ahora como entonces me dan ganas de llevar un peine en el bolsillo y
sacarlo y dárselo diciendo: «Caballero, hágame usted el favor de
peinarse.»

\hypertarget{xii}{%
\chapter{XII}\label{xii}}

Ahora hablemos, ¿por qué no?, de la violentísima pasión que inspiré a un
francés. Era este el conde de Montguyon, coronel del 3.º de húsares. Yo
le había conocido en Tolosa, habiendo tenido la desgracia de que mi
persona hiciera profunda impresión en él, trastornando las tres
potencias de su alma. Era soltero, de treinta y ocho años, bien parecido
y atento y finísimo como todos los franceses. Persiguiome hasta París,
donde me asediaba como esos conquistadores jóvenes e impacientes que han
oído la célebre frase de César y quieren imitarla. Al principio me
mortificaban sus obsequios; le rechazaba hasta con menosprecio y
altanería; pero al fin, sin corresponder a su amor de ninguna manera,
admití la parte superficial de sus galanterías. Esto le dio esperanza;
pero siempre me trataba con el mayor respeto. Deseando, sin duda,
identificarse con las ideas que suponía en mi tierra, se había hecho una
especie de D. Quijote, cuya Dulcinea era yo. A veces me parecía por
demás empalagoso; pero después de muchos meses de indiferencia absoluta,
empecé a estimarle, reconociendo sus nobles prendas. Cuando me disponía
a volver a mi país, se me presentó rebosando alegría, y me dijo:

---Acabo de conseguir que me destinen a la guerra de España. De este
modo consigo tres grandes objetos que interesan igualmente a mi corazón:
guerrear por la Francia, visitar la hermosa tierra de España y estar
cerca de usted.

Él pretendía que me detuviese para partir juntos; pero a esto no accedí,
y me marché dejándole atrás, aunque deseosa ¿a qué negarlo?, de que no
me siguiese a mucha distancia, pues a causa del fastidio de viaje tan
largo, Francia, con ser tan bella, empezaba a aburrirme de lo lindo.

¿Se creerá que yo había olvidado a mi pobre cautivo de Benabarre? ¡Ah!,
no, y hasta el último momento que estuve en la Seo de Urgel me ocupé de
su desgraciada suerte. Cada vez que venía a mi pensamiento la idea de
sus penas, me estremecía de dolor, y toda alegría se disipaba en mi
espíritu. Pero este tiene en sí mismo una energía restauradora, no menos
poderosa que la del cuerpo, y sabe curarse de todos sus males siempre
que le ayude el mejor de los Esculapios, que es el tiempo.

Voltaire, que no por impío y blasfemo dejó de tener mucho talento,
escribió una historieta titulada \emph{Los dos consolados}, en la cual
pone de relieve las admirables curas de aquel charlatán, el único cuyos
específicos son infalibles. Yo he leído esa novelita, así como otras del
célebre escritor sacrílego, y esta debilidad mía, imperdonable quizás en
una dama tan acérrima defensora de la religión, la confieso aquí
contritamente, rogando a mis lectores que no revelen a ningún cura de mi
país tan feo secreto, ocultándolo principalmente al señor canónigo de
Tortosa, mi director espiritual, el cual se enfurecerá si le hablan de
las novelas de Voltaire, aunque a mí me consta que él también las ha
leído.

Pues bien, el tiempo fue cicatrizando mis heridas sin curarlas. Yo
también podía erigir una estatua con la inscripción \emph{A celui qui
console}, pues la ausencia indefinida y los días que pasaban rápidamente
habían calmado aquel insaciable afán de mi alma. En mí reinaba la
tranquilidad, pero no el taciturno y seco olvido; y una aparición
repentina del ser amado podía muy bien en brevísimo instante, destruir
los efectos del tiempo renovando mi mal y aun agravándolo.

Desde París a la frontera no cesaba el movimiento de tropas. Por todas
partes convoyes, cuerpos de ejército y oficiales que iban a incorporarse
a sus regimientos. Francia podía creerse aún en los días del gran
soldado. Hasta Burdeos no tuve noticias ciertas de mi querida Regencia y
de mi ilustre mandatario el marqués de Mataflorida. ¡Ay! La suerte de
este insigne hombre de Estado no podía ser más miserable. Eguía había
triunfado, a pesar de las furiosas protestas del regente de Urgel; y
para colmo de desdicha, como aún quisiera este llevar adelante sus locas
pretensiones, el duque de Angulema le mandó prender juntamente con el
arzobispo, confinándoles a Tours. Así acabaron las glorias de aquellos
dos ambiciosos. Yo llegué a tiempo para verles, y cuando manifesté al
marqués las poco lisonjeras disposiciones del \emph{triste Chactas}, el
atroz Regente, desairado, llamó a Chateaubriand intrigante, enredador,
mal poeta y \emph{franchute}. Esta fue la venganza del coloso.

Bayona era un campamento cuando yo llegué. El número de españoles casi
superaba al de franceses, y en todos reinaba grande alegría. Reanudé
entonces mis buenas relaciones con el barón de Eroles, haciéndole ver
que mi viaje a París había tenido por causa asuntos particulares, y
entre risas y bromas me reconcilié con Eguía, el cual, por razón del
mismo gozo y embobamiento del triunfo, estaba muy dispuesto a perdonar.
En cuanto a las negociaciones, yo no tenía humor de seguir ocupándome de
ellas, y deseaba retirarme a descansar sobre mis laureles diplomáticos,
no sólo porque mi entusiasmo absolutista se había enfriado mucho, sino
porque desde algún tiempo las conspiraciones y los manejos políticos me
causaban hastío. Ya he dicho que siempre fui muy inclinada a la mudanza
en mis ocupaciones. Mi espíritu se aviene poco con la monotonía, y si
hubo un día en que me sedujeron las embajadas, otro llegó en que me
repugnaron. ¡Mágico efecto del tiempo, cuya misión es renovar, creando
las estaciones con los admirables círculos del universo! También el alma
humana ve en sí la alterada sucesión de las primaveras e inviernos en
sus dilataciones y recogimientos.

Yo deseaba entrar en España, y tenía propósito de reanudar las
diligencias para averiguar el paradero de mi cautivo de Benabarre. En
Bayona, una familia francesa legitimista, con quien yo tenía antigua
amistad, me convidó a pasar unos días en su casa de campo inmediata a
Behobia, y unos parientes míos invitáronme a que les acompañase a Irún
un par de semanas. A ambos ofrecimientos accedí, empezando por el de
Behobia, aunque la frontera no me parecía el punto más a propósito para
residir en los momentos en que principiaba la guerra. Pero la gente de
aquel país estaba segura de que Angulema atravesaría fácilmente el
Pirineo, por ser muy adicto al absolutismo todo el país vasco-navarro.

Todavía no había pasado Su Alteza la raya, cuando se rompió el fuego
junto al mismo puente internacional. Los carbonarios extranjeros que
andaban por España, unidos a otros perdidos de nuestro país, habían
formado una legión con objeto de hacer frente a las tropas francesas.
Constaba aquélla de doscientos hombres, tristes desechos de la ley
demagógica de Italia, de Francia y de España; y para seducir a los cien
mil hijos de San Luis, se habían vestido a la usanza imperial, y
ondeando la bandera tricolor, gritaban en la orilla española del
Bidasoa: «¡Viva Napoleón II!»

Su objeto era fascinar a los artilleros franceses con este mágico grito;
mas tuvieron la desdicha de que tales aclamaciones fueran contestadas a
cañonazos, y con sus banderas y sus enormes morriones huyeron a San
Sebastián. Pasma la inocente credulidad de los carbonarios extranjeros y
de los masones españoles. Oí decir en Behobia que los liberales
franceses Lafayette, Manuel, Benjamín, Constant y otros fiaban mucho en
los doscientos legionarios mandados por el republicano emigrado coronel
Fabvier. ¡Qué desvaríos engendra el furor de partido! Corría esto
parejas con la necia confianza del Gobierno español, que, aun después de
declarada la guerra, no había tomado disposiciones de ninguna clase,
hallándose sus tropas sin más recursos ni elementos que el parlerío de
los milicianos y el gárrulo charlatanismo de los clubs.

\hypertarget{xiii}{%
\chapter{XIII}\label{xiii}}

Hacia los primeros días de abril vi pasar a los generales de división
Bourdessoulle, duque de Reggio, y Molitor, que entraron en España por
Behobia. Después pasó Su Alteza el sobrino de Luis XVIII, con todo su
Estado Mayor, en el cual iba Carlos Alberto, príncipe de Carignan. No se
puede imaginar cortejo más lucido. Yo no había visto nada tan magnífico
y deslumbrador, como no fuera la comitiva de José Bonaparte antes de
darse la batalla de Vitoria el año 13, feliz para la causa española,
pero de muy malos recuerdos para mí, porque en él perdí la batalla de mi
juventud, casándome como me casé.

También vi pasar a mi amigo Eguía remozado por la emoción y tan
vanaglorioso del papel que iba a representar que no se le podía
resistir, como no fuera tomando a broma sus bravatas. Iban con él D.
Juan Bautista Erro y Gómez Calderón, aquel a quien el mordaz Gallardo
llamaba \emph{Caldo pútrido}. El barón de Eroles, que con los anteriores
tipos debía formar la Junta al amparo del Gobierno francés, entró por
Cataluña con el mariscal Moncey.

No recibieron a los franceses las bayonetas ni la artillería del
Gobierno constitucional, sino una nube de guerrilleros, que les abrieron
sus fraternales brazos, ofreciéndose a ayudarles en todo y a marchar a
la vanguardia, abriéndoles el camino. Tal apoyo era de grandísimo
beneficio para la causa, porque los partidarios realistas ascendían a
35.000 ¡Ay de los franceses si hubieran tenido en contra a aquella
gente! Pero les tenían a su favor, y esto sólo ¡qué fenómeno!, ponía al
buen Angulema por encima de Napoleón. El absolutismo español no podía
hacer al hijo de San Luis mejor presente que aquellos 35.000 salvajes,
entre los cuales (¡cuánto han variado mis ideas, Dios mío!) tengo el
sentimiento de decir que estaba mi marido. ¡Y yo le había admirado, yo
le había aceptado por esposo diez años antes sólo por ser
guerrillero!\ldots{} Cuando se hacen ciertas cosas, ya que no es posible
que el porvenir se anticipe para avisar el desengaño, debiera caer un
rayo y aniquilarnos.

El conde de España mandaba las partidas de Navarra, Quesada las de las
Provincias Vascongadas y Eroles las de Cataluña. ¡Cómo fraternizaron las
partidas con los franceses, que habían sido origen de su nacimiento en
1808! Era todo lo que me quedaba por ver. Se abrazaban, dando vivas a
San Luis, a San Fernando, a la religión, a los Borbones, al Rey, a la
Virgen María, a San Miguel arcángel y a los Sermos. Infantes. Yo no lo
vi, porque no quise pasar la frontera. Me repugnaban estas cosas, y los
soldados de la fe habían llegado poco a poco a serme muy antipáticos.

Largamente hablé de esto con el conde de Montguyon, que me perseguía
tenazmente, permaneciendo en Behobia todo el tiempo que le fue posible.
Él elogiaba a los guerrilleros, diciendo que, a pesar de sus defectos,
eran tipos de heroísmo y de aquella independencia caballeresca que tanto
había enaltecido el nombre español en otros tiempos. También le seducían
por ser, como los frailes, gente muy pintoresca. Mi Don Quijote era una
especie de artista, y gustaba de hacer monigotes en un libro, dibujando
arcos viejos, mendigos, casuchas, una fila de chopos, carros, lanchas
pescadoras y otras menudencias de que estaba muy envanecido.

Debía ser próximamente el 9 de Abril cuando me trasladé a Irún para
vivir con la familia de Sodupe-Monasterio, gente muy hidalga, más
católica que el Papa, realista hasta el martirio y de afabilísimo trato.
Frecuentaban la casa (que era más bien palacio con hermosos prados y
huerta) todos los españoles que el gran suceso de la intervención traía
y llevaba de una Nación a otra, y muchos oficiales franceses, de cuyas
visitas se holgaban mucho los Sodupe-Monasterio, porque oían hablar sin
cesar de exterminio de liberales, del trono de San Fernando y de nuestra
preciosísima fe católica.

Allí Montguyon no me dejaba a sol ni a sombra, pintándome su amor con
colores tan extremados, que me daba lástima verle y oírle. Su acendrado
y respetuoso galanteo merecía, en efecto, alguna misericordia. Le
permití besar mi mano; pero no pudo arrancarme la promesa de seguirle al
interior de España. Cada vez sentía yo más deseos de quedarme en Irún y
en aquella apacible vivienda, donde, sin que faltara sosiego, había
bastantes elementos para combatir el fastidio. Con esta resolución, mi
D. Quijote, que ya parecía querer dejar de serlo en la pureza de sus
ensueños amorosos, estaba desesperado. Despidiose de mí muy enternecido
y besándome con ardor las manos, voluptuosidad inocente de que nunca se
hartaba. ¡Cuán lejos estaba el llagado amante de que no pasarían dos
horas sin que cambiara diametralmente mi determinación!

Pasó del modo siguiente. Al saber que yo estaba en Irún, fue a visitarme
un individuo, que aún no podía llamarse personaje, y al cual conocí en
Madrid el año anterior, y también el 19. Se llamaba D. Francisco Tadeo
Calomarde, y era de la mejor pasta de servil que podía hallarse por
aquellos tiempos. Hijo del Ministro de Gracia y Justicia, se había
criado en los cartapacios y en el papel de pleitos: los legajos fueron
su cuna y las reales cédulas sus juguetes. Su jurisprudencia llena de
pedantería me inspiraba aversión. Tenía fama de muy adulador de los
poderosos, y según se decía, compró el primer destino con su mano,
casándose con una muchacha muy fea a quien dio malísimos tratos.

Los que le han juzgado tonto se equivocan, porque era listísimo, y su
ingenio, más bien socarrón que brillante, antes agudo que esclarecido,
era maestro en el arte de tratar a las personas y de sacar partido de
todo. Habíase hecho amigo de D. Víctor Sáez, y aun del mismo Rey y del
Infante D. Carlos, por sus bajas lisonjas y lo bien que les servía
siempre que encontraba ocasión para ello.

Entonces tenía cincuenta años, y acababa de salir del encierro
voluntario a que le redujo el régimen liberal. Había ido a la frontera
para llevar no sé qué recados a los señores de la Junta. Me lo dijo, y
como no me importaban ya gran cosa los dimes y diretes de los realistas,
que no por estar tan cerca de la victoria dejaban de andar a la greña,
fijeme poco en ello, y lo he olvidado. Calomarde no era mal parecido ni
carecía de urbanidad, aunque muy hueca y afectada, como la del que la
tiene más bien aprendida que ingénita. La humildad de su origen se
traslucía bastante. Hablamos de los sucesos de Madrid que él había
presenciado y prolijamente me informó de todo.

---Siento que usted no hubiera estado por allá---me dijo;---habría visto
cómo se iba desbaratando el constitucionalismo, sólo con el anuncio de
la intervención. Si no podía ser de otra manera\ldots{} Ahora están que
no les llega la camisa al cuerpo, y en ninguna parte se creen seguros.
Después que ultrajaron a Su Majestad, le han arrastrado a Andalucía con
el dogal al cuello, como el mártir a quien se lleva al sacrificio.

---No tanto, Sr.~D. Tadeo---le dije,---Su Majestad habrá ido como
siempre, en carroza, y mucho será que los mozos de los pueblos no hayan
tirado de ella.

---Eso se deja para la vuelta---indicó Calomarde riendo.---Ahora los
franc-masones han seducido a la plebe, y Su Majestad, por donde quiera
que va, no oye más que denuestos. El 19 de Febrero, cuando se
alborotaron los masones y comuneros porque estos querían sustituir a
aquellos en el Ministerio, los chisperos borrachos y los asesinos del
Rastro daban \emph{mueras} al Rey y a la Reina. Un diputado muy conocido
apareció en la Plaza Mayor mostrando una cuerda con la cual proponía
ahorcar a Su Majestad y arrastrarle después. La canalla penetró hasta la
Cámara real. ¡Escándalo de los escándalos! Parecía que estábamos en
Francia y en los sangrientos días de 1792. El mismo Rey me ha dicho que
los Ministros entraban en la Cámara cantando el himno de Riego.

---¡Oh, no tanto, por Dios!---repetí, ofendida de las exageraciones de
mis amigos.---Poco mal y bien quejado.

---Me parece que usted, con sus viajes a Francia y sus relaciones con
los Ministros del liberal y filósofo Luis XVIII, se nos está volviendo
franc-masona ---dijo D. Tadeo entre bromas y veras.---¿Hay en la
historia desacato comparable con el de obligar al Rey a partir para
Andalucía?

---¡Oh, Dios nos tenga de su mano!\ldots{} ¡qué desacato!, ¡qué
ignominia!\ldots---exclamé, remedando sus aspavientos.---Es preciso
considerar que un Gobierno, cualquiera que sea, está en el caso de
defenderse, si es atacado.

---Según mi modo de ver, un Gobierno de pillos no merece más que el
decreto que ha de mandar a Ceuta a todos sus individuos. ¡Ah, señora
mía, y cómo se ha entibiado el fervor de usted! Bien dicen que los aires
de esa Francia loca son tan nocivos\ldots{}

---Creo lo mismo que creía; pero mi absolutismo se ha civilizado,
mientras el de ustedes continúa en estado salvaje. El mío se viste como
la gente y el de ustedes sigue con taparrabo y plumas. Si el Gobierno de
pillos ha resuelto refugiarse en Andalucía, llevándose a la Corte, ha
sido para no estar bajo la amenaza de los batallones franceses.

---Ha sido---dijo Calomarde riendo brutalmente,---porque sabían que
Madrid no tiene defensa posible; que los ejércitos de Ballesteros y de
La Bisbal son dos fantasmas; que cuatro soldados y un cabo de los del
Serenísimo Sr.~Duque de Angulema, podían cualquier mañanita sorprender a
la Villa y a los \emph{Siete Niños} y al Congreso entero y al
Ayuntamiento soberano y a toda la comunidad masónica y Landaburiana.
Esta es la pura verdad. ¡Y qué bonito espectáculo han dado al mundo! En
presencia de la intervención armada, ¿cómo se preparan esos mentecatos
para conjurar la tormenta? Llamando a las armas a treinta mil hombres y
disponiendo (esto es lo más salado) que con los milicianos que quieran
seguir al Congreso se formen algunos batallones, recibiendo cada
individuo cinco reales diarios. ¡Se salvó la patria, señora!

---El Gobierno---repuse prontamente,---creyó sin duda que los franceses
eran como los Guardias del 7 de Julio, es decir, simples juguetes de
miliciano.

---¡Ya se lo diremos de misas!---dijo frotándose las manos.---Ya pagarán
su alevosía. Sólo por el hecho de obligar a nuestro Soberano a un viaje
que no le agradaba, merecerían todos ellos la muerte.

---Hasta los Reyes están en el caso de hacer alguna vez lo que no les
agrada.

---Incluso viajar con un ataque de gota, ¿eh? ¡Crueles y sanguinarios,
más sanguinarios y crueles que Nerón y Calígula! Ni a un perro vagabundo
de las calles se le trata peor.

---Si el Rey no tenía en aquellos días ataque de gota---repliqué
complaciéndome en contradecirle.---Si estaba bueno y sano. La prueba es
que después de clamorear tanto por su enfermedad, anduvo algunas leguas
a pie el primer día de viaje.

---Bueno, concedo que Su Majestad estaba tan bueno como yo. ¿Y si no
quería partir?

---Que hubiera dicho «no parto.»

---¿Y si le amenazaban?

---Haberles ametrallado.

---¿Y si no tenía metralla?

---Haberse dejado llevar por la fuerza.

---¿Y si le mataban?

---Haberse dejado matar. Todo lo admito menos la cobardía.

---Amiguita, usted se nos ha franc-masoneado---me dijo el astuto
intrigante dando cariñosa palmada en mi mano.---A pesar de esto, siempre
la queremos mucho y la serviremos en lo que podamos. Yo estoy siempre a
las órdenes de usted.

Inflado de vanidad, el amigo del Rey hizo elogios de sí mismo, y después
añadió:

---He tenido el honor de ser indicado para secretario de la Junta que se
va a formar en la frontera.

---¡Oh, amigo mío, doy a usted la enhorabuena!---manifesté sumamente
complacida y deplorando entonces haber estado algo dura con
Calomarde.---No se podía haber pensado en una persona más idónea para
puesto tan delicado.

---¿Se le ofrece a usted algo?---dijo D. Tadeo comprendiendo al punto mi
cuarto de conversión.

---Sí; pero yo acostumbro dirigirme siempre a la cabeza---afirmé
resueltamente.---Ya sabe usted que soy muy amiga del general Eguía,
Presidente de la Junta.

---¡Ah!, entonces\ldots{}

---Sin embargo. No puedo molestar a Su Excelencia con ciertas
menudencias tales como pedir noticias de personas, averiguar alguna
cosilla de poca monta\ldots{}

---Para esto es más propio un secretario tan bien informado como yo de
todos los pormenores de la causa.

---Exactamente. Dígame usted, si lo sabe, en dónde está ahora un pícaro
de mala estofa, que se emplea en bajas cábalas del Rey y tiene por
nombre José Manuel Regato.

---¡Ah! ¡Regato!\ldots{} Debe de andar por Andalucía con la Corte. No es
de mi negociado ese caballero\ldots{} ¿Qué? ¿Hay ganas de sentarle la
mano?

---Por sentarle la derecha daría la izquierda.

---Pocas noticias puedo dar a usted del señor Regato. Tengo con él muy
pocas relaciones. Quizás Pipaón, que conoce a todo el mundo, pueda
indicar dónde se halla y el modo de sentarle, no una mano, sino las dos,
siempre que sea preciso.

---Y Pipaón, ¿dónde está?

---Aquí.

---¡Aquí! ¡Pipaón!\ldots---exclamé con gozo.---Yo le dejé en la Seo muy
enfermo y creí que había caído en poder de Mina.

---En efecto cayó; pero él\ldots{} ya usted le conoce\ldots{} con su
destreza y habilidad parece que encontró por allí amigos que le
favorecieron.

---Quiero verle, quiero verle al punto---dije con la mayor
impaciencia.---Deseo mucho tener noticias de la Seo y de las facciones
de Cataluña.

Y entonces se realizó aquel proverbio que dice: «En nombrando al ruin de
Roma\ldots»

Por la vidriera que daba a la huerta de la casa viose la mofletuda cara
y el pequeño cuerpo de Pipaón, que habiendo tenido noticia de mi
residencia en Irún iba también a verme. Mucho nos alegramos ambos de
hallarnos juntos, y nuestras primeras palabras después de los cordiales
saludos fueron para recordar los tristes días de la Seo, su enfermedad y
mi abatimiento, y luego por el enlace propio de los recuerdos, que van
de lo triste a lo placentero, hablamos del miedo del arzobispo, de las
casacas que usaba Mataflorida y de otras cosas frívolas y chistosas, de
esas que ocurren siempre en los días trágicos y nunca faltan en los
duelos. Después de estos desahogos, Pipaón, tomando aquel tono burlesco
que unas veces le sentaba bien y otras le hacía muy insoportable, me
dijo:

---Le traigo a usted noticias muy buenas de una persona que le interesa,
y con las noticias una cartita.

\hypertarget{xiv}{%
\chapter{XIV}\label{xiv}}

Yo me puse pálida. Comprendí de quién hablaba Pipaón, pero no me atreví
a decir una palabra, por hallarse delante el entrometido y curioso
Calomarde, gran coleccionador de debilidades ajenas. Varié de
conversación, aguardando, para saciar mi afanosa curiosidad, a que D.
Tadeo se marchase; pero el pícaro había conocido en mi semblante la
turbación y ansiedad que me dominaban, y no se quería retirar. Parecía
que le habían clavado en la silla. ¡Ay qué gusto tan grande poder coger
un palo y romperle con él la cabeza!\ldots{} ¡Qué pachorra de hombre!

Quise arrojarle con mi silencio; pero él era tan poco delicado que
conociendo mi mortificación, se arrellanaba en el blando asiento como si
pensara pasar allí el día y la noche. Pipaón con su expresivo semblante
me decía mil cosas, que no podía yo comprender claramente, pero que me
deleitaban como avisos o presentimientos lisonjeros. Llegó un momento en
que los tres nos callamos, y callados estuvimos más de un cuarto de
hora. Calomarde tocaba una especie de paso doble con su bastón en la
pata de la mesa cercana. El grosero y pegajoso cortesano había resuelto
quemarme la sangre u obligarnos a Pipaón y a mí a que hablásemos en su
presencia.

Resistí todo el tiempo que pude. Mi carácter fogoso no puede ir más allá
de cierto grado de paciencia, pasado el cual, estalla y se sobrepone a
todo, atropellando amistades, conveniencias y hasta las leyes de la
caridad. Nunca he podido corregir este defecto, y la estrechez de los
límites de mi paciencia me ha proporcionado en esta vida muchos
disgustos. Forzando la voluntad puedo a veces aguantar más de lo que
permite la extraordinaria fuerza de dilatación de mi espíritu; pero
entonces estallo con más violencia, rompo mis ligaduras a la manera de
Sansón y derribo el templo. Vino por fin el momento en que se me subió
la mostaza a la nariz, como dicen las majas madrileñas, y poniéndome en
pie súbitamente, miré a Calomarde con enojo. Señalándole la puerta,
exclamé:

---Sr.~D. Tadeo, tengo que hablar con Pipaón: le suplico a usted que nos
deje solos.

Debían de ser muy terribles mi expresión y mi gesto, porque Calomarde se
levantó temblando, y con voz turbada me dijo:

---Señora, manos blancas no ofenden.

\emph{¡Manos blancas no ofenden!} Diez años después Calomarde debía
pronunciar esta frase al recibir un desaire más violento que el mío, la
célebre bofetada de la Infanta Carlota, una Princesa que, como yo, tenía
muy limitado el tesoro de su paciencia y estallaba con tempestuosas
cóleras, cuando la bajeza y solapada intriga de los Calomardes se
interponían en su camino.

Pipaón y yo nos quedamos solos. En pocas palabras me refirió que había
visto a Salvador Monsalud sano y salvo en la Seo de Urgel. Al oír esto
el corazón dio un salto dentro de mí como una cosa muerta que torna a la
vida, como un Lázaro que resucita por sobrehumano impulso.

---Mina le salvó en San Llorens de Morunys---me dijo,---y desde que se
restableció se puso a mandar una compañía de contraguerrilleros.

Al decir esto, Pipaón me alargó una carta, que abrí con presteza febril,
queriendo leerla antes de abrirla. Al mismo tiempo, y de una sola ojeada
leí el fin y el principio y el medio. Era la carta pequeña y fría.
Decíame en ella que estaba en libertad y que no pensaba salir en mucho
tiempo del lugar donde estaba fechada, que era Urgel. Sentí mi corazón
inundado de un torrente de sangre glacial al ver que no contenía la
carta expresiones de ardiente cariño.

---¿De modo que sigue en Cataluña?---pregunté a D. Juan.

---No señora. A estas horas va camino de Madrid.

---Pues ¿cómo dice en su carta que no piensa salir de la Seo?

---Esa carta me la dio cuando nos separamos, el día 30 de Marzo, pero
dos días después supe, por nuestro común amigo el capitán Seudoquis, que
Mina había encargado a Salvador que fuese a Madrid a llevar un mensaje
reservadísimo a San Miguel y a otras personas.

---¿De modo que está?\ldots{}

---Sobre Madrid, como se dice en los partes militares.

---Pero eso ¿es cierto?

---Tan cierto como que estoy hablando con una dama hermosa.

---¿Y salió?\ldots{}

---Según mis noticias, el 10 de este mes. No sabía qué camino tomar;
pero, según me dijo Seudoquis, estaba decidido a ir por Zaragoza que es
el más derecho, aunque no el menos peligroso.

---¿Sabe la muerte de su madre?

---Yo le di la mala noticia.

---Pero ¿qué va a hacer ese hombre en Madrid?---dije sintiendo una
tempestad en mi cerebro.---Si allí no hay ya Gobierno ni nada.

---Pero está en Madrid el gran Consejo de la franc-masonería. Mina es de
la Orden de la Acacia, señora. Ahora se trata de que la \emph{Viuda}
haga un esfuerzo supremo.

En mi espíritu notaba yo aquella poderosa fuerza de dilatación de que
antes he hablado. Unas cuantas palabras habían trastornado todo mi ser;
mi pulso latía con violencia; asaltáronme ideas mil, y el ardoroso afán
de movimiento que ha sido siempre una de las fórmulas más patentes de mi
carácter se apoderó de mí. Sin necesidad de que yo le despidiese, dejome
Pipaón, que iba en busca de Eguía para solicitar un puesto en la Junta,
y después de pasada mi turbación, pude sondear aquel revuelto piélago de
mi espíritu y mirar con serenidad lo que en el fondo de él había.

¡Cuán grande había sido mi engaño al creer moribunda la afición aquella
que tantas dulzuras dio a mi alma en el verano del 22! La ausencia
habíala escondido entre las cenizas que diariamente depositan los
sucesos de cada instante, esa multitud de ascuas de la vida que van
pasando sin interrupción y apagándose hora tras hora. Pero aquella ascua
del verano del 22 era demasiado grande y quemadora para pasar y
extinguirse como las demás.

Bastó que oyera pronunciar su nombre, que me le anunciaran vivo para que
se verificase en mí un brusco retroceso a los días de mi felicidad y de
mi desgracia. El tiempo volvió atrás; las figuras veladas perdieron la
sombra que las encubría; las apagadas palabras que sólo eran ya ecos
confusos, volvieron a sonar como cuando eran la música a cuyo compás
danzaba con la embriaguez de la pasión mi alma. ¡Cuánto me había
engañado y qué juicios tan erróneos hacemos de nuestros propios
sentimientos y de todo aquello que está lejos! Nos pasa lo mismo que al
ver las lontananzas de la tierra, cuando confundimos con las vanas y
pasajeras nubes los montes sólidos e inmutables que ninguna fuerza
humana puede arrancar de sus seculares asientos. Fue aquello como una
vuelta, como un ángulo brusco en el camino de la vida. Desde entonces vi
nuevos horizontes, paisaje nuevo, y otra gente y otros caminos. ¡Y yo
había creído poder olvidarle y aun poner en su altar vacío al conde de
Montguyon! ¡Qué delirio!\ldots{} ¡Lo que pueden la ausencia, la
distancia, la ignorancia! El tiempo que me había consolado, hiriome de
nuevo, y un día, un instante marcado en mi vida por cuatro palabras como
cuatro estrellas resplandecientes, había destruido la obra lenta de
tantos meses.

Con la presteza que Dios me ha dado formé mi plan de viaje. Tengo algo
del genio de Napoleón para esto de los grandes movimientos. Para mí la
facultad de trasportar todo el interés de la vida de un punto a otro del
mundo es otra prenda muy principal de mi carácter, y al mismo tiempo una
necesidad a la que muy difícilmente puedo resistir. El destino me ha
presentado siempre los sucesos a propósito para tales juegos de
estrategia sublime.

Aquella misma tarde dispuse todo, y por la noche sorprendí a mi D.
Quijote con la noticia de mi viaje. Aficionada a jugar con los corazones
que caen en mis manos (a excepción de uno solo), como juega el gatito
con el ovillo que rueda por el suelo, dije al conde de Montguyon:

---Me he asustado de la soledad en que voy a quedar después que usted se
marche, y voy a Madrid. De esta manera podré vigilar a cierto caballero
francés por si anda en malos pasos.

Él se puso tan contento, que olvidó aquella noche hablarme de la guerra
y de los laureles que iban a recoger. Parecía un loco hablando de los
alcázares de Granada, de los romances moriscos, de las ricas hembras, de
las boleras, de los frailes que protegían los amores de los grandes, de
las volcánicas pasiones españolas y de las mujeres enamoradas que eran
capaces del martirio o del asesinato. Él se creía héroe de mil aventuras
románticas e interesantes caballerías, tales como se las había imaginado
leyendo obras francesas sobre España. Empleo la palabra
\emph{románticas} porque si bien no estaba en moda todavía, es la más
propia. El romanticismo existía ya, aunque no había sido bautizado.
Excuso decir que Montguyon me juró amor eterno y una fidelidad
inquebrantable como la del Cid por D.ª Jimena.

Yo necesitaba de él para mi viaje, por lo cual me guardé muy bien de
arrancar una sola hoja a la naciente flor de sus ilusiones. Era muy
difícil viajar entonces porque casi todos los vehículos del país habían
sido intervenidos por ambos ejércitos. Montguyon me prometió una silla
de postas. Y cumplió su oferta, poniéndola a mi disposición al día
siguiente.

Con el primer movimiento del ejército francés, coincidió mi marcha sobre
Madrid, como una conquistadora. El estrépito guerrero que en derredor
mío sonara, despertaba en mi mente ideas de Semíramis.

\hypertarget{xv}{%
\chapter{XV}\label{xv}}

Pasé por Vitoria y por la Puebla de Arganzón, como los días felices por
la vida del hombre, a escape. No miraba a ningún lado, por miedo a mis
malos recuerdos, que salían a detenerme.

En los pueblos todos del Norte la intervención vencía sin batallas, y
antes de que asomara el morrión del primer francés de la vanguardia, la
Constitución estaba humillada. Los mozos todos comprendidos en la quinta
ordenada por el Gobierno, se unían a las facciones, y eran muy pocos los
milicianos que se aventuraban a seguir a los liberales. No he visto una
propagación más rápida de las ideas absolutistas. Era aquello como un
incendio que de punta a punta se desarrolla rápidamente y todo lo
devora. En medio de las plazas los frailes predicaban mañana y tarde,
con pretexto de la Cuaresma, presentando a los franceses como enviados
de Dios, y a los liberales como alumnos de Satanás que debían ser
exterminados.

El general Ballesteros mandaba el ejército que debía operar en el Norte
y línea del Ebro para alejar a los franceses. No viendo yo a dicho
ejército por ninguna parte, sino inmensas plagas de partidas, pregunté
por él, y me dijeron en Bribiesca que Ballesteros, convencido de no
poder hacer nada de provecho, se había retirado nada menos que a
Valencia. Movimiento tan disparatado no podía explicarse en
circunstancias normales; pero entonces todo lo que fuera desastres y
yerros del liberalismo tenía explicación.

Al ver cómo crecía en los pueblos la aversión a las Cortes y al
Gobierno, el ejército perdía el entusiasmo. A su paso, como se levanta
polvo del camino, levantábanse nubes de facciosos que al instante eran
soldados aguerridos. Así se explica que el ejército de Ballesteros,
compuesto de diez y seis mil hombres, se retirara sin combatir
emprendiendo la inverosímil marcha a Valencia, donde podía adquirir
algún prestigio derrotando a Sempere, al Locho y al carretero Chambó,
tres nuevos generales o arcángeles guerreros que le habían salido a la
Fe.

En Dueñas me adelanté, dejando atrás a los franceses; tenía tanta prisa
como ellos y menos estorbos en el camino, aunque los suyos no eran
tampoco grandes. ¡Cuánto deseaba yo ver tropas regulares españolas por
alguna parte! En verdad, me daba vergüenza que los hijos de San Luis, a
pesar de que nos traían orden y catolicismo, se internaran en España tan
fácilmente. Con todo mi absolutismo yo habría visto con gusto una
batalla en que aquellos liberales tan aborrecidos dieran una buena tunda
a los que yo llamaba entonces mis aliados. Española antes que todo,
distaba mucho de parecerme a los señores frailes y sacristanes que en
1808 llamaban judíos a los franceses y ahora ministros de Dios.

En Somosierra encontré tropas. Eran las del ejército de La Bisbal,
destinado por las Cortes a cerrar el paso del Guadarrama, amparando de
este modo a Madrid. Mis dudas acerca del éxito de aquella empresa fueron
grandes. Yo conocía a La Bisbal. ¿Cómo no había de conocerle si le
conocía todo el mundo? Fue el que el año 14 se presentó al Rey llevando
dos discursos en el bolsillo, uno en sentido realista y otro en sentido
liberal, para pronunciar el que mejor cuadrase a las circunstancias. Fue
el que en 1820 hizo también el doble papel de ordenancista y de
sedicioso. La inseguridad de sus opiniones había llegado a ser
proverbial. Era hombre altamente penetrado del axioma italiano ma per
\emph{troppo variar natura é bella}. Yo no comprendía en qué estaba
pensando el Gobierno cuando le nombró. Si los Ministros se hubieran
propuesto elegir para mandar el ejército más importante al hombre más a
propósito para perderlo, no habrían elegido a otro que a La Bisbal.

Pasé con tristeza por entre su ejército. Aquellos soldados, capaces del
más grande heroísmo, me inspiraban lástima, porque estaban destinados a
desempeñar un papel irrisorio, como leones a quienes se obliga a bailar.
Sentía yo impulsos de arengarles, diciéndoles: «¡Que os engañan, pobres
muchachos! No dejéis las armas sin combatir. Si os hablan de
capitulación, degollad a vuestros generales.»

En Madrid hallé un abatimiento superior a lo que esperaba. Se hablaba
allí de capitular como de la cosa más natural del mundo. Sólo tenían
entusiasmo algunos infelices que no servían para nada, el cuerpo de
coros de los clubs y de las sociedades secretas, la gente gritona y
también muchos de los que habían tirado del coche de Fernando VII cuando
volvió de Francia el año 14. Los absolutistas creían con razón ganada la
partida y afectaban cierta generosidad magnánima. ¡Pobre gente! Algunos
de estos pajarracos vinieron a visitarme, entre ellos D. Víctor Sáez, y
tuve el gusto de mortificarles asegurándoles que Angulema traía orden de
obsequiarnos con las dos Cámaras y un absolutismo templado, suavísimo
emoliente para nuestra anarquía. Esto ponía a mis buenos amigotes más
furiosos que las bravatas de los liberales, pues aún había liberales con
alma bastante para echar roncas.

Pero yo me ocupaba poco de tales cosas. Mi primer cuidado fue hacer
algunas averiguaciones concernientes a la entrañable política de mi
herido corazón. Felizmente a la casa donde yo vivía, que era honradísimo
albergue de una noble familia alavesa, iba a menudo un tal Campos,
hombre muy intrigante, director de Correos, si no recuerdo mal, gran
maestre de la Orden masónica, o por lo menos principalísimo dignatario
de ella, amigo íntimo de los liberales de más viso y también de algunos
absolutistas, como hombre que sabe el modo de comer a dos carrillos.

Yo le había tratado el año anterior, y charlando juntos, me reía mucho
de los masones, lo cual a él no le enojaba. Entre bromas y veras solía
enterarme de algunas cosas reservadas, porque no era hombre de
extraordinaria discreción ni tampoco de una incorruptibilidad absoluta.
En los días de mi llegada de Irún, que eran los de mediados de Mayo del
23, le pregunté si esperaban los masones algún mensaje reservado de
Mina. Negolo; mas yo, asegurándolo con el mayor descaro y nombrando al
mensajero, le hice confesar que esperaban órdenes de Mina de un día a
otro. Él, lo mismo que su secretario cuyo nombre no recuerdo, me
aseguraron no haber visto todavía en Madrid a Salvador Monsalud ni tener
noticia alguna de él.

---No ha llegado aún---dije.---Mucho tarda.

Sin reparar en nada fui a su casa. Un portero, tan locuaz como pedante,
liberal muy farolón, de aquellos a quienes yo llamo \emph{sepultureros
de la libertad}, porque son los que la han enterrado, me informó de que
el Sr.~Monsalud faltaba de Madrid desde el mes de agosto del año
anterior.

---Puede que la Sra. Dª. Solita sepa algo---me dijo.---Pero no es fácil,
porque anoche lloraba\ldots{} Como no llorase de placer, que también
esto sucede a menudo\ldots{}

---¿De modo que la casa subsiste?---le pregunté.

---Subsiste, sí señora; pero no subsistirá mucho tiempo si el Sr.~D.
Salvador no vuelve del otro mundo.

---Pues qué, ¿ha muerto?

---Así lo creo yo. Pero esa joven sentimental siempre tiene esperanzas,
y cada vez que el sol sale por el horizonte esparciendo sus rayos de
oro\ldots{} ¿me entiende usted?

---Sí; acabe de una vez el Sr.~Sarmiento.

---Quiero decir, que siempre que amanece, lo cual pasa todos los días,
la Sra. Dª. Solita dice: «¡Hoy vendrá!» Tal es la naturaleza humana,
señora, que de todo se cansa menos de esperar. Y yo digo: ¿qué sería del
hombre sin esperanza?\ldots{} Dispénseme la señora; pero si piensa
subir, tengo el sentimiento de no poder acompañarla, porque como mi hijo
es miliciano\ldots{}

---¿Y qué?

---Como es miliciano y el honor le ordena derramar hasta la última gota
de su sangre en defensa de la dulce patria y de la libertad preciosísima
del género humano\ldots{}

---¿Y qué más?---dije complaciéndome en oír las graciosas pedanterías de
aquel hombre.

---Que impulsado por su ardoroso corazón, capaz del heroísmo, y por mi
paternal mandato, ha ido a Cádiz con las Cortes; y como ha ido a Cádiz
con las Cortes y no volverá hasta dejar confundida a la facción y a los
cien mil y quinientos hijos, nietos o tataranietos del calzonazos de
Luis XVIII\ldots{} Por vida de la chilindraina y con cien mil pares de
docenas de chilindrones, que si yo tuviera veinte años menos!\ldots{}
Pues digo que como Lucas ha ido a Cádiz\ldots{} y es un león mi hijo, un
verdadero león\ldots{} resulta que me es forzoso estar al cuidado de la
puerta, ¿me entiende la señora?

---Está bien---le dije riendo.---Puedo subir sola.

Quise darle una limosna, porque su aspecto me pareció muy miserable;
pero la rechazó con dignidad y cierto rubor decoroso, propio de las
grandezas caídas.

Subí a la casa. Mi corazón subía antes que yo.

\hypertarget{xvi}{%
\chapter{XVI}\label{xvi}}

En seguida que llamé salieron a abrir. Se conocía que en la casa reinaba
la impaciencia. Una mujer descorrió con presteza el cerrojo y me rogó
que entrase. Era ella. Yo recordaba haberla visto en alguna parte.

Carecía de verdadera hermosura, pero al reconocerlo así con gozo, no
pude dejar de concederle una atracción singular en toda su persona, un
encanto que habría establecido al instante entre ella y yo profunda
simpatía, si en medio de las dos no existiese, como infranqueable
abismo, la persona de un hombre. Vestía de luto, y la delgadez de su
rostro anunciaba el paso de grandes penas. Cuando me vio alterose tanto
y su turbación fue tan grande, que no podía dirigirme la palabra. Por mi
parte la miré con serenidad y altanería, como de superior a inferior,
haciendo todo lo posible para que ella se creyese muy honrada con mi
visita.

Yo había oído hablar a Salvador con cariño y admiración que me ofendían,
de aquella singular hermana suya que no era tal hermana, ni aun pariente
y que muy bien podía ser otra cosa. Nunca creí en la fraternidad honrada
y cariñosa de que él me había hablado, porque conozco un poco el corazón
del hombre, y admito sólo los sentimientos cardinales y fundamentales, y
no esas mixturas y composiciones sutiles que no sirven más que para
disfrazar alguna pasión ilícita\ldots{} Deseaba conocer por mí misma a
la dichosa hermana tan ponderada por él y ver si tenía fundamento el
secreto odio que mi alma hacia ella sentía. Desde que la vi, a pesar de
que me fue muy patente su inferioridad personal con respecto a la nieta
de mi abuela, me pareció tener delante a una rival temible, más
peligrosa cuanto más humilde en apariencia. Al instante traté de buscar
en ella un defecto grande, de esos que afean espantosamente a la mujer.
Mi ingenioso rencor encontró al punto aquel defecto, y dije en mi
interior.

---Esta muchacha debe de ser una hipocritona. No hay más remedio sino
que lo es.

Mi juicio fue rápido, como la inspiración, como la improvisación. Desde
la puerta a la sala, a donde me condujo, hice mil observaciones, entre
ellas una que no debo pasar en silencio. La casa estaba tan
perfectamente arreglada que no parecía vivienda sin dueño. Todo se
hallaba en su sitio, sin el más ligero desorden, en perfecto estado de
limpieza, descubriéndose en cada cosa el esmero peregrino que anuncia la
mano de una mujer poseedora del genio doméstico. Creeríase que el amo
era esperado de un momento a otro y que todo se acababa de disponer para
agradarle cuando entrara.

Al sentarme reconcentré mis ideas acerca del plan que había formado y le
dije:

---Sé que usted padece mucho por saber el paradero del amo de esta casa,
y como tengo noticias de él, vengo a tranquilizarla.

---¡Oh!, ¡señora!, ¡cuánta bondad!---exclamó con repentina alegría.---De
modo que usted sabe dónde está y por qué no viene\ldots{} ¿Le han vuelto
a coger los facciosos?

---No señora. Está libre y bueno.

---Entonces no tiene perdón de Dios---dijo abatiendo el vuelo de su alma
que tanto se había elevado con las alas de la alegría.---No, no tiene
perdón de Dios.

---¿Usted le ha escrito?

---Muchas veces. Dirijo las cartas al ejército de Mina, con la esperanza
de que alguna llegue a sus manos\ldots{} pero no recibo contestación. Es
una iniquidad de mi hermano. Por poco que se acuerde de mí, por muy
grande que sea su olvido, ¿será tal que no me haya escrito una sola vez?

---Los que están en armas---dije sonriendo,---no se acuerdan de las
pobres mujeres que lloran.

---Yo creo que me ha escrito. Él es muy bueno y me considera mucho. No
es capaz de tenerme en esta incertidumbre por su voluntad.

---¿Pero usted no ha recibido ninguna carta?

---En Febrero vinieron dos; pero después ninguna. Quizás se hayan
perdido.

---Podría ser.

---A veces me figuro que no me escribe porque viene. Todos los días creo
que va a llegar, y desde que siento pasos en la escalera, corro a ver si
es él. Todo lo tengo preparado, y si viene, nada encontrará fuera de su
sitio.

---Sí, ya lo veo. Es usted una alhaja. El pobre Salvador debe de estar
muy satisfecho de su hermana. Él la aprecia a usted mucho. Me lo ha
dicho.

---¡Se lo ha dicho a usted!---exclamó tan vivamente conmovida que casi
estuvo a punto de llorar.

---Me lo ha dicho, sí. Él me cuenta todo. Para mí nunca ha tenido
secretos.

Sola me miró de hito en hito durante un momento, que me pareció
demasiado largo. ¿Qué había en la expresión de su semblante al
contemplar el mío? ¿Envidia? No podía ser otra cosa; pero la apariencia
indicaba más bien una resignación dolorosa. Le habría tenido mucha
lástima si no hubiera estado convencida de que era una hipócrita.

---Muchas veces me ha hablado de usted---proseguí,---elogiándome sus
bellas cualidades para el gobierno de una casa. Vea usted de qué manera
ha venido a encontrarse sola al frente de este hogar vacío,
conservándole tan bien para cuando él vuelva.

---La pobre D.ª Fermina---dijo,---que murió de pesadumbre por la pérdida
de su hijo, me encargó todo al morir, poniendo en mi mano cuanto tenía y
ordenándome que lo guardase y conservase hasta que pareciera Salvador.

---¿Entonces ella no le creía muerto?

---Dudaba. Siempre tenía esperanza---manifestó Solita dando un
suspiro.---Yo le hablaba a todas horas de la vuelta de su hijo, y, la
verdad, siempre tuve esperanza de verle entrar en la casa, porque una
voz secreta de mi corazón me decía que volvería. El día antes de
fallecer D.ª Fermina, escribió una larga carta a su hijo\ldots{}
¡Cuántas lágrimas derramó la pobre! Yo habría dado con gusto mi vida,
porque la infeliz madre viera a su hijo antes de morir. Pero Dios no lo
quiso así.

---¿Y esa carta\ldots?---pregunté deseosa de conocer aquel detalle.

---Esa carta la depositó en mí D.ª Fermina, mandándome que la entregase
a Salvador en su propia mano, si parecía.

---¿Y si no parecía?

---Doña Fermina me mandó que le buscase por todos los medios posibles, y
que si tenía noticias de él y no venía a Madrid, fuese a buscarle aunque
tuviera que ir muy lejos.

---Pero ¿cómo podrá usted emprender esos viajes?, ¡pobrecilla!---exclamé
mostrando una compasión que estaba muy lejos de sentir.

---Eso sería lo de menos. No me faltan ánimos para ponerme en camino, ni
tampoco recursos con que emprender un largo viaje, porque D.ª Fermina me
entregó todos sus ahorros para que los destinase a buscar a su hijo.

---¡Ah!, entonces\ldots{} Y para el caso de no encontrarlo ¿qué dispuso
esa señora?

---Que esperase, y le volviera a buscar después.

---¿Y para el caso de que fuera evidente su muerte?

---Que echase al fuego la carta sin leerla. ¡Ha sido desgraciada suerte
la nuestra!---prosiguió la huérfana con abatimiento.---Un mes después de
haber subido al cielo aquella buena señora, vino la carta de Salvador
anunciando que estaba libre. ¡Ay!, en mi vida he tenido mayor alegría ni
mayor tristeza, juntas tristeza y alegría sin que pudiesen ser
separadas. Yo le contesté diciéndole lo que pasaba y rogándole que
viniese. Desde aquel día le estoy esperando. Han pasado tres meses, y no
ha venido ni me ha escrito.

---Pues ha llegado la ocasión de que usted cumpla la última voluntad de
la pobre señora difunta, partiendo en busca de ese hijo desnaturalizado.

---¡Si no sé dónde está!\ldots{} Un amigo que lee todos los papeles
públicos y sabe por dónde andan los ejércitos, las guerrillas y las
contraguerrillas, me ha dicho que las tropas de Mina se han disuelto.
Otro que vino del Norte, me aseguró que Salvador había emigrado a
Francia. Yo, a pesar de estas noticias, le espero, tengo confianza en
que ha de venir, y he resuelto aguardar lo que resta de mes. Sigo mis
averiguaciones, y si en todo Mayo no ha venido ni me ha escrito, pienso
ponerme en camino y buscarle con la ayuda de Dios.

---Siento quitarle a usted una ilusión---dije adoptando definitivamente
mi diabólico plan, y resolviéndome a ponerlo en ejecución.---Salvador no
vendrá por ahora, no puede venir.

---¿Lo sabe usted de cierto?---me preguntó vivamente turbada y con algo
de incredulidad en sus hermosos ojos.

---¿Duda usted de mí?---dije poniendo en mi semblante esa naturalidad
inefable que es uno de mis más preciosos resortes para expresar lo que
quiero.---Precisamente no he venido a otra cosa que a decirle a usted su
paradero, después de tranquilizarla, por si le creía enfermo o muerto.

---¿Y dónde está?

---Habiendo reñido con Mina por una cuestión de amor propio, pasó a las
contraguerrillas que siguen al general Ballesteros.

---¿Entonces sigue en el Norte?

---No señora. Ya sabe usted que el ejército de Ballesteros se ha
retirado a Valencia.

---A Valencia, sí. Efectivamente, lo oí decir. ¿De modo que Salvador
está en Valencia?

---Sí: y estos informes no son vagos ni fundados en conjeturas, porque
yo misma\ldots{}

Al llegar aquí dí un suspiro afectando cierta emoción. Después acabé así
la frase:

---Yo misma me separé de él en Onteniente el 20 de Abril.

---¿Es cierto, señora, lo que usted me dice?---me preguntó con gran
agitación.

---Sí; pero no creo que haga usted el disparate de ponerse en camino
para Levante ---indiqué con objeto de que no conociera mi verdadera
idea.

---¿Pues qué, vendrá?

---Venir no. No vendrá en mucho tiempo, mayormente si de hoy a mañana
capitula la Corte, y se establece el absolutismo. Yo creo que se verá
obligado a emigrar, embarcándose en cualquier puerto de la costa.

---¡Embarcarse!---exclamó con desaliento.---No señora, no; eso no puede
ser. Corro allá al momento.

Se levantó como si de un vuelo pudiera trasladarse a Valencia.

---¿Y será usted capaz de emprender un viaje tan largo?\ldots{} ¿Tendrá
usted valor?\ldots---manifesté con fingida admiración.

---Yo tengo valor para todo, señora---me respondió.

Después del primer movimiento de credulidad, la vi como abatida y
vacilante. Dudaba.

---Puede usted escribirle---le dije,---con la dirección que yo le dé, y
cuando reciba la contestación de él, ponerse en camino\ldots{} Lo malo
será que en ese tiempo tome la guerra otro aspecto y llegue usted tarde.

---Eso sería terrible. Yo creo que si voy debo ir hoy mismo\ldots{} ¿Y
de él se separó usted el 20 de Abril?

Dudaba todavía. Al llegar a este punto, la voz de la conciencia, que aún
me detenía, fue acallada por mis celos, y no pensé más que en el éxito
completo del plan que me había propuesto. No vacilé más, y pensé en la
carta que me había traído Pipaón.

---Me separé de él el 20 de Abril---afirmé;---pero después de eso,
hallándome en Aranjuez, recibí una carta suya.

Con avidez fijó Solita sus ojos en mí. Por grande que fuera mi
serenidad, mi corazón palpitaba, porque ni aun los criminales más
criminales hacen ciertas cosas sin algo de procesión por dentro.
Confesaré ahora la fealdad toda de mi acción para que se comprenda bien
la importancia de aquella escena y mi perverso papel.

---Si me quisiera mostrar usted la carta de Salvador---me dijo en tono
suplicante,---al menos para saber con fijeza el punto en que se
halla\ldots{}

---No la he traído---repuse con el mayor aplomo;---pero volveré a mi
casa, que está a dos pasos y la traeré, para que tenga usted ese
consuelo y una seguridad que no pueden darle mis palabras.

---¡Oh!, no señora; yo creo\ldots{}

---No\ldots{} estas cosas son delicadas. Al instante traeré a usted la
carta que me escribió y que no está fechada en Onteniente, sino en otro
pueblo del reino de Valencia, pues como usted puede suponer, el ejército
se mueve casi todos los días.

Diciendo esto me levanté. Ella me daba las gracias por mi bondad en
cariñosas y vehementes palabras. Brindose a ir conmigo porque yo no me
molestase en volver; pero esto no me convenía y salí rápidamente.
¡Miserable de mí, y cuánto me cegaba la pasión y aquel detestable afán
de hacer daño a la que aborrecía!\ldots{} Contaré esto con la mayor
brevedad posible, porque me mortifica tan desagradable recuerdo, y en
verdad que si pudiera escribir estas vergonzosas líneas cerrando los
ojos, lo haría para no ver lo que traza mi propia pluma.

\hypertarget{xvii}{%
\chapter{XVII}\label{xvii}}

Corrí a mi casa, tomé la carta de Salvador, y con ese golpe de vista del
genio criminal comprendí que lo previsto por mí momentos antes podía
realizarse fácilmente. La data \emph{Urgel} estaba escrita en letra
ancha y mala. La palabra podía ser variada por una mano hábil, y la mía,
fuerza es decirlo, lo era, aunque nunca hasta entonces se había empleado
en tan infames proezas.

Yo tenía muy presente a un primo mío que había comerciado años antes en
un pueblo de Alicante llamado \emph{Vergel}, en las inmediaciones de
Denia, a orillas del río Bolana. Esta palabra era el puñal del asesinato
proyectado por mí. La tomé con la fiebre del rencor. ¡Qué admirablemente
servía para mi objeto! ¡Qué bien dispuestas estaban sus letras para una
obra satánica! No podía pedirse más, no. Tenía delante de mí una de esas
infernales coincidencias que deciden a los criminales vacilantes, y a
veces hasta a los justos les impulsan a escandalosos y horribles
pecados.

Tomé la pluma, y con mano segura, regocijándome interiormente en la
perfección de mi obra, convertí la palabra Urgel en Vergel. La fecha era
fácil de mudar también. Salvador había puesto Marzo en abreviatura. Yo
convertí el Marzo en Mayo, dejando el día que era el 3, lo mismo que
estaba\ldots{} ¡Oh, cuando no se me cayó la mano entonces, creo que
tendré manos para toda mi vida!

Del texto de la carta podía mostrarse la primera plana, donde decía
entre otras cosas insignificantes: «no pienso en muchos días salir de
este pueblo.»

Corrí allá con mi puñal. Las trágicas figuras antiguas a quienes pintan
alborotadas y arrogantes con un hierro en la mano, no fruncirían el ceño
más fieramente que yo, al blandir mi carta homicida. Subí a la casa.
Sola me esperaba en la puerta. Entramos: me senté al punto porque estaba
muy cansada.

---Vea usted---le dije;---el pueblo donde ahora está es Vergel. He
pasado por él.

Solita devoraba con los ojos la carta.

---Vergel---añadí mostrándole la carta,---está entre Pego y Denia, sobre
un riachuelo que llaman Bolana. Si va usted a Onteniente le será muy
fácil llegar a Vergel.

Ella seguía leyendo.

---Asegura que por ahora no piensa moverse de ese pueblo---dijo
meditabunda.---Mejor; con eso tendré la certeza de encontrarle.

---¿Pero de veras insiste usted en ir?\ldots{} El resto de la carta no
se lo enseño a usted porque no puede interesarle---indiqué, afectando la
mayor naturalidad y guardando mi arma.---No puedo creer que haga usted
la locura de\ldots{}

---Iré, iré---dijo con una resolución briosa que inundó mi alma de los
frenéticos goces del éxito criminal.

Después de manifestar así su propósito, frunció el ceño y me dijo:

---Cuando usted se separó de Salvador, ¿él sabía que venía usted a
Madrid?

---Lo sabía.

---¿Y cómo no le rogó que me viese y me tranquilizara?

---Porque sabe---repuse con dignidad,---que yo no sirvo para hacer las
veces de correo. Si he venido a esta casa, ha sido por\ldots{} se lo
diré a usted con entera franqueza; no quiero fingir móviles que no tuve
al venir aquí, aunque después que nos hemos tratado hayan sido distintas
mis ideas.

Solita atendía a mis palabras como al Evangelio. Yo le tomé una mano y
poniéndome a punto de llorar, me expresé así:

---Señora D.ª Solita; dije a usted al entrar que venía con el simple
objeto de tranquilizarla dándole informes de Salvador.

---Así fue, señora, lo que usted me dijo.

---Pues bien; falté a la verdad: quise encubrir mi verdadero objeto con
una fórmula común. Pero yo no puedo fingir, no puedo ocultar la verdad.
Mi carácter peca de excesivamente franco, natural y expansivo. Mis
pasiones y mis defectos, la verdad toda de mi alma, buena o mala, se me
sale por los ojos y por la palabra cuando más quiero disimular. Usted me
ha inspirado simpatías; usted me ha revelado una pureza de sentimientos
que merece el mayor respeto. Quiero ser como usted, y hablarle con la
noble veracidad que se debe a los verdaderos amigos. ¿No es usted
hermana para él?, pues quiero que lo sea también para mí.

Solita al oír esto se apartó lentamente de mi lado. Noté en ella cierta
aversión contenida por el respeto.

---Querida amiga---proseguí forzando mi arte.---No he venido aquí sino
por un egoísmo que usted no comprenderá tal vez. He venido por ver su
casa, por conocer lo único que guarda Madrid de esa amada persona, este
asilo donde él ha vivido, donde murió su madre, y por el cual parecen
vagar aún sus miradas. Quería yo dar a mis ojos el gusto de ver estos
objetos, estos muebles donde tantas veces se han fijado los ojos
suyos\ldots{} Nada más, ningún otro objeto me trajo aquí. He tenido
además el placer de conocerla a usted, y ahora, deseándole que halle
pronto a su hermano, me retiro.

Levantéme resueltamente. Solita había prorrumpido en amargo llanto.

---¡Oh! ¡Gracias, gracias, señora!---exclamó secando sus lágrimas.---Le
diré que debo a usted este inmenso favor.

---No, no, por Dios---repliqué vivamente.---Ruego a usted que no me
nombre para nada. Vería en mí una debilidad que no quiero confesarle,
mediando, como median en uno y otro, los propósitos de separación
eterna.

---Pues callaré, señora, callaré. ¿De modo que usted no le verá más?

Al decir esto había tanto afán en su mirada, que me causó indignación.
La habría abofeteado, si mi papel no hubiera exigido gran prudencia y
circunspección.

---No señora, no le veré más---le dije fijando más sobre mi semblante la
máscara que se caía.---Después de lo que ha pasado\ldots{} Pero no puedo
revelarle a usted ciertas cosas. Si usted le conoce bien, conocerá su
inconstancia. Yo le he amado con fidelidad y nobleza. Él\ldots{} no
quiero rebajarle delante de una persona que le estima. Adiós, señora,
adiós. ¿Se va usted al fin hoy?

Esto lo dije en pie, estrechando aquella mano que habría deseado ver
cortada.

---Sí señora, iré a buscarle, puesto que él no quiere venir.

---¿Pero se atreve usted, sola, sin compañía, por esos
caminos\ldots?---indiqué deseando que me confirmase su resolución.

---Dios irá conmigo---repuso la hipocritona con el acento de los que
tienen verdadera fe.---El ordinario de Valencia que sale esta noche, era
amigo de D.ª Fermina. Con él iré. Tengo confianza en Dios y estoy segura
de que no me pasará nada\ldots{} Ahora, tomada esta determinación, estoy
más tranquila.

---La felicidad le retoza a usted en el rostro---afirmé con cruel
sarcasmo.---Bien se conoce que es usted feliz. Yo me congratulo de haber
proporcionado a usted un cambio tan dichoso en su espíritu.

Cuando pronuncié estas palabras debió secárseme la lengua, lo confieso.

Poco más hablamos. Hícele ofrecimientos corteses y salí de la casa.
Cuando bajaba la escalera sentí impulsos de volver a subir y llamarla y
decirle: «no crea usted nada de lo que he dicho; soy una embustera;»
pero el egoísmo pudo más que aquel pasajero y débil sentimiento de
rectitud, y seguí bajando. Del mismo modo iba bajando mi alma, escalón
tras escalón, a los abismos de la iniquidad. Razoné como los perversos,
diciéndome que la víctima de mi intriga era una mujer hipócrita y que
las maquinaciones de mal género, tan dignas de censura cuando recaen en
personas inocentes, son más tolerables si recaen en quien las merece y
es capaz de urdirlas peores. Pero estos sofismas no acallaban mi
remordimiento, que empezó a crecer desde que salí de la casa y ha
llegado después, por su mucha grandeza y pesadumbre, a mortificarme en
gran manera.

\hypertarget{xviii}{%
\chapter{XVIII}\label{xviii}}

Verdaderamente mi acción no pudo ser más indigna. ¡Precipitar a una
desamparada e infeliz mujer a resolución tan loca, obligarla por medio
de vil engaño a emprender un viaje largo, dispendioso, arriesgado y
sobre todo inútil!\ldots{} Al mirar esto desde tan distante fecha, me
espanto de mi acción, de mi lengua, y de la horrible travesura y astucia
de mi entendimiento.

En aquellos días la pasión que me dominaba y más que la pasión, el
envidioso afán que me producía la simple sospecha de que alguien me
robase lo que yo juzgaba exclusivamente mío, no me permitieron ver
claramente mi conciencia ni la infamia de la denigrante acción que había
cometido; pero cuando todo se fue enfriando y oscureciendo, he podido
mirarme tal cual era en aquel día, y declaro aquí que, según me veo, no
hay fealdad de demonio del infierno que a la mía se parezca.

¡Y sigue uno viviendo después de hacer tales cosas! ¡Y parece que no ha
pasado nada, y vuelve la felicidad, y aun se da el caso de olvidar
completamente la perversa y villana acción!\ldots{} Yo no vacilo en
escribirla aquí, porque me he propuesto que este papel sea mi
confesonario, y una vez puesta la mano sobre él, no he de ocultar ni lo
bueno ni lo malo. La seguridad de que esto no lo ha de ver nadie hasta
que yo no me encuentre tan lejos de las censuras de este mundo como lo
están los astros de las agitaciones de la tierra, da valor a mi espíritu
para escribir tales cosas. Yo digo: «que todo el mundo escriba con
absoluta verdad su vida entera, y entonces ¡cuánto disminuirá el número
de los que pasan por buenos! Las cuatro quintas partes de las grandes
reputaciones morales no significan otra cosa que \emph{falta de datos}
para conocer a los individuos que se pavonean con ellas fatuamente, como
los cómicos cuando se visten de reyes.»

Aquella tarde torné a pasar por allí, y entablé conversación con
Sarmiento; pero me fue imposible averiguar por él si Solita insistía en
partir.

Yo tenía gran desasosiego hasta no saberlo de cierto, y para salir de mi
incertidumbre quise averiguarlo por mí misma. Soy así: lo que puedo
hacer no lo confío a los demás. Me fatigan las dilaciones y la torpeza
de los que sirven por dinero, y carezco de paciencia para aguardar a que
me vengan a decir lo que yo puedo ver por mis propios ojos. Al llegar la
noche y la hora en que solían partir los coches, sillas de postas y
galeras, mi criada y yo nos vestimos manolescamente, con pañolón y
basquiña, y nos encaminamos al parador del Fúcar, de donde, según mis
noticias, salía el ordinario de Valencia.

No tuve que esperar mucho para satisfacer mi curiosidad. Allí estaba.
Solita partía irremisiblemente. Ya no me quedaba duda. La vi dentro del
coche que salía, y no pude sofocar en mí un sentimiento de profundísima
lástima, forma indirecta que tomaba entonces mi conciencia para
presentarme ante los ojos la imagen de mi crimen. Pero el coche partió;
ella se fue con su engaño y yo me quedé con mi lástima.

No se había extinguido el rumor de las ruedas del carro de Valencia,
cuando sonó más vivo estrépito de ruedas y caballerías. Un gran coche de
colleras entró en el parador. Mi criada y yo nos detuvimos por
curiosidad.

---Es el coche de Alcalá---dijeron a nuestro lado.---Esta noche viene
lleno de gente.

Por una de las portezuelas vi la cara de un hombre. El corazón parecía
hacérseme pedazos. Me volví loca de alegría. No pude contenerme. Era él.
Mis exclamaciones cariñosas le obligaron a bajar del coche, y entonces
me arrojé llorando en sus brazos.

\hypertarget{xix}{%
\chapter{XIX}\label{xix}}

Al día siguiente le aguardaba en mi casa y no fue hasta muy tarde,
cuando ya anochecía. Estaba muy fatigado, triste y abatido. Lo primero
de que me habló fue del vacío que había dejado en su casa la muerte de
su madre, de la partida de su hermana, a quien creía encontrar en
Madrid, y del brevísimo espacio que un perverso destino había puesto
entre la marcha de ella y la llegada de él.

---Castigo de Dios es esto---dijo,---por mi descuido en escribirle y mi
desnaturalizado proceder.

Después pasó de la tristeza a la furia. Yo procuraba arrancarle tan
lúgubres ideas, recordándole nuestro placentero viaje del verano
anterior y la catástrofe de su cautiverio; hacíale mil preguntas sobre
sus padecimientos, emancipación, campaña de Cataluña y toma de la Seo;
pero sólo me contestaba con monosílabos y secamente. Escaso interés
mostraba por las cosas pasadas, y aun yo misma, que era un presente
digno a mi parecer de alguna estima, apenas podía obtener de él atención
insegura y casi forzada. Su pensamiento estaba fijo en la fugitiva
hermana, y mis sutiles zalamerías no podían apartarle de allí. No cesaba
de discurrir sobre los móviles de aquel viaje, y yo, sintiendo revivir y
agitarse en mí lo que siempre tuve de serpiente, estuve a punto de
indicarle que Soledad habría partido arrastrada por algún hombre; pero
en el momento en que desplegaba los labios para sugerir esta idea, me
contuve. Aquella vez había vencido mi conciencia, y hallándome con
fuerzas para las mayores crueldades, no las tuve para la calumnia.

Al fin, creí prudente no decirle una palabra sobre aquella cuestión.

---Bastaba que yo viniese con deseo de verla---dijo hiriendo
violentamente el suelo con el pie,---para que ella huyese de mí. Así son
todas mis cosas. Lo bueno existe mientras yo lo deseo. Pero lo toco, y
adiós.

Estas amargas palabras eran un desaire para mí, y por lo visto yo no
estaba comprendida en el número de las cosas buenas; pero sofoqué mi
resentimiento y seguí escuchándole.

---Desde que el deseo de venganza y mi odio al
absolutismo---añadió,---me inclinaron a tomar las armas, tuve el
presentimiento de que la campaña se echaría a perder, y así ha sido. Ya
tienes a la plaza de Figueras en poder de los franceses; a Mina
vagabundo sin saber qué partido tomar, y todo el ejército desconcertado
y sin esperanza de vencer. ¡Gran milagro habría sido que donde yo estoy
hubiese victorias! Desastres y nada más que desastres. La sombra que yo
echo sobre la tierra, destruye.

---¡Qué necio eres! ¿Crees acaso en las estrellas fatales y en el sino?

---No debiera creer; pero todo me manda que crea\ldots{} Ya ves. Me
envía Mina a Madrid con una comisión en que funda grandes esperanzas, y
desde que llego aquí pierdo las pocas esperanzas que traía, porque no
hallo sino desanimación y flojedad. Al mismo tiempo, la ilusión más
querida de este viaje se ha desvanecido como el humo. Yo tenía una
hermana, más que hermana amiga, con una amistad pura y entrañable que
nadie puede comprender sino ella y yo; una amistad que tiene todo lo
santo de la fraternidad y todo lo bueno del amor, sin las tenebrosas
ansias de este. En mi hermana veía yo todo lo que me queda de familia,
lo único que me resta de hogar; en ella veía a mi madre y una
representación de todos los goces de mi casa, la paz del alma, dichas
muy grandes sin mezcla de martirio alguno. Pues bien: llego y mi casa
está desierta. Jamás pensé en perderla. Ella, el único ser de quien
estaba seguro, vuela también lejos de mí, y se va. ¡Ay, Jenara! ¡No
puedo decirte cuán sola estaba mi casa! Figúrate todo el universo vacío
y sin vida. Ni mi madre, ni Soledad\ldots{} ¡Qué sepulcro, Dios mío! Así
se va quedando mi corazón lo mismo que una gran fosa, todo lleno de
muertos\ldots{} Tú no puedes entender esto, Jenara. En ti todo vive. Tu
carácter hace resucitar las cosas y eres un ser privilegiado para quien
el mundo se dispone siempre del modo más favorable; pero yo\ldots{}

---Cúlpate a ti mismo---le dije,---y no hables del destino. Te quejas de
que tu hermana te haya abandonado, y no recuerdas que has estado mucho
tiempo sin escribirle, sin darle noticias de ti, sin decirle ni
siquiera: «estoy vivo.»

---Es verdad; pero se amparó de mí el estúpido delirio de la guerra. Me
sedujo la idea gloriosa que representaba nuestro ejército al perseguir a
los realistas. Sólo veía lo que estaba delante de mis ojos y dentro de
mí: el enemigo y los torbellinos de mi cerebro, un ideal de gloriosas
victorias que dieran a mi país lo que no tiene. Ya sabes que yo me
equivoco siempre. Lo extraño es que conociendo mi torpeza me empeñe en
andar hacia adelante como los demás hombres, en vez de estarme quieto
como las estatuas\ldots{} Ahora todo lo veo destrozado, caído y hecho
pedazos por mis propias manos, como el que entrando en un cuarto oscuro
y lleno de preciosidades y a ciegas tropieza y lo rompe todo. En
Cataluña, desengaños, en Madrid más desengaños todavía; un gran vacío
del entendimiento y otro más grande del corazón. Parece que la realidad
de mis ideas es un ave que se asusta de mis pasos y levanta el vuelo
cuando me acerco a ella. ¡Maldita persona la mía!

Debía enojarme de tales palabras, porque, según ellas, yo no era nada.
Pero no me mostré ofendida y solamente dije:

---Si al llegar encuentras todo solo y vacío, no es porque las cosas
vuelen antes de tiempo, sino porque tú llegas siempre tarde.

---También es verdad. Llego siempre tarde. Ya ves lo que me ha pasado
ahora---dijo con el mayor desaliento.---Se le antoja al general Mina
enviarme aquí cuando todo está perdido. Pero él no contaba con la
rapidez de este desmoronamiento, no contaba con la retirada de
Ballesteros, sin combatir, ni con la defección de La Bisbal. Mina tiene
la desgracia de creer que todos son valientes y leales como él.

---¿La defección de La Bisbal? De modo que ya\ldots{} No creí que fuera
tan pronto. El conde acostumbra preparar con cierto arte sus
arrepentimientos.

---No se dice públicamente; pero es seguro que ya está en tratos con los
franceses para capitular. Me lo ha dicho Campos, que olfatea los
sucesos. De mañana a pasado el aborrecido estandarte negro ondeará en
Madrid. ¿A qué he venido yo? No parece sino que ha venido a izarlo yo
mismo.

---Pues no hagas caso de los masones, ni de la guerra, ni de la
Constitución---le dije.---¿Para qué te empeñas en cosas imposibles? ¿Por
qué desprecias lo que tienes y buscas fantasmas vanos?

Él me miró comprendiendo mi intención. Su mirada no indicaba desafecto;
pero me era imposible vencer su tristeza. Acompañome a cenar, y mis
alardes de humor festivo, mi cháchara y las delicadas atenciones que con
él tuve no lograron disipar las nubes sombrías que ennegrecían su alma.

También la mía se encapotaba lentamente, cayendo en hondas tristezas,
porque acostumbrada a verse señora de los sentimientos de aquel hombre,
padecía mucho al considerar perdido su amoroso dominio y esa tiranía
dulcísima que al mismo tiempo embelesa al amo y al esclavo.

Pero aún conservaba yo gran parte de mi prestigio. Vencí, aunque sin
poder conseguir la tranquilidad que acompaña a los triunfos completos;
porque descubrí en su complacencia algo de violento y forzado. Parecía
que al corresponder a mi leal cariño, lo hacía más bien por delicadeza y
por deber que por verdadera inclinación. Esto me atormentó toda la
noche, quitándome el sueño. Cuando pude dormir, la imagen de la pobre
huérfana que recorría media España buscando a su hermano, a su amante o
lo que fuera, se me presentó para atormentarme más. ¡Ay!, ¡qué terrible
es una gran falta sin éxito!

La visión de la mujer errante no se quitaba de mi imaginación. Pero yo
entonces, creyéndome menos amada de lo que mi frenética ambición de amor
exigía; pensando que me habían vencido ajenos recuerdos y vaguedades
sentimentales referentes a otra persona, me gozaba con fiera crueldad en
la desolación de la hermana viajera.

---¡Bien---le decía,---corre tras él, corre hoy y mañana y siempre, para
no encontrarle al fin!\ldots{} Muy bien, hipocritona, ¡¡me alegro, me
alegro!!

\hypertarget{xx}{%
\chapter{XX}\label{xx}}

Al día siguiente muy temprano entró Campos en casa. Ya he dicho que este
masón era amigo muy constante de la familia con quien yo vivía, un
matrimonio alavés, de edad madura y sin hijos, extraño por lo general a
las pasiones políticas, aunque la señora, como buena vascongada, se
inclinaba al absolutismo. Campos entró gritando:

---¡Ya nos la ha pegado ese tunante!

Al punto comprendí lo que quería expresar.

---La Bisbal ha capitulado ¿no es eso?---le dije.---¡Qué noticia! Ya lo
suponíamos.

---Pero al menos, señora, al menos\ldots---manifestó Campos con
afán.---Las formas, es preciso guardar ciertas formas\ldots{} Todos
estamos dispuestos a capitular, porque no es posible vivir en lucha con
la general corriente, ni con la Europa entera; pero\ldots{} pero\ldots{}

---¿Y qué ha hecho La Bisbal?

---Dar un manifiesto\ldots{}

---Ya lo suponía: es el hombre de los manifiestos.

---Un manifiesto en que dice que sí y que no, y que tira y afloja, y que
blanco y que negro\ldots{} En fin, un manifiesto de La Bisbal. Después
ha entregado el mando al marqués de Castelldosrius y ha desaparecido. El
ejército está desmoralizado. La mayor parte de los soldados se van a
donde les da la gana, y aquí nos tiene usted, como el 3 de Diciembre de
1808, en poder de los franceses\ldots{} ¿Vamos a ver, qué hace ahora un
hombre honrado como yo? ¿Qué hacen ahora los hombres que no se han
metido en nada, que desde su campo defendieron siempre el orden y las
conveniencias?\ldots{}

Yo hacía esfuerzos para contener la risa. La zozobra del masón en
momentos de tanto apuro y su afán por presentarse como hombre de orden
ofrecían un cuadro tan gracioso como instructivo.

---¿De modo que ya se acabó la Constitución?---dijo la señora de
Saracha, elevando majestuosamente las manos al cielo, como en acción de
gracias.---Pues ahora habrá perdón general. Se reconciliarán todos los
españoles, dándose fraternales abrazos y amparándose bajo el manto
amoroso del Rey.

Yo me eché a reír.

---No es mal perdón el que nos aguarda---dijo Campos con detestable
humor.---¡Bonito manto nos amparará! Ya se ha alborotado la gentuza de
los barrios bajos, y las caras siniestras, las manos negras y rapaces,
los trabucos y las navajas van apareciendo. Nada, nada. Tendremos
escenas de luto y de ignominia, otro 10 de Mayo de 1814.

---¿Será posible? Pues me parece que efectivamente hay algo de alboroto
en la calle---dijo mi amiga asomándose al balcón.

Vivíamos en la calle de Toledo, que es la arteria por donde la
emponzoñada sangre sube al cerebro de la villa de Madrid en los días de
fiebre. Cruzaban la calle gentes del pueblo en actitud poco
tranquilizadora. Al poco rato oímos gritar: «¡viva la religión!»,
«¡vivan la caenas!». Fue aquella la primera vez de mi vida que oí tal
grito, y confieso que me horrorizó.

Campos no quiso asomarse porque le enfurecían los desahogos de la plebe
(mayormente cuando chillaba en contra de los liberales) y seguía
diciendo:

---Veremos cómo tratan ahora a los hombres honrados que han defendido el
orden, que han procurado siempre contener al democratismo y a la
demagogia.

No pude vencer mi natural inclinación a las burlas y le dije:

---Sr.~Campos, no doy cuatro cuartos por su pellejo de usted.

---Ni yo tampoco---me respondió riendo.

Él, en medio de su descontento, esperaba filosóficamente el fin, seguro
de sobrenadar tarde o temprano en el piélago absolutista. Era además
hombre de tanto valor como osadía.

La gente de los barrios bajos siguió alborotando todo el día. Moviose la
tropa para mantener el orden, y el general Zayas, que mandaba en Madrid
y había firmado la capitulación aquella misma mañana con los franceses,
parecía dispuesto a ametrallar sin compasión a la canalla. En gran
zozobra vivíamos todos los vecinos de la Villa, porque se hablaba de
saqueo y de la aproximación de las partidas de Bessières, el infante
aventurero, que defendiendo el despotismo quería lograr lo que no pudo
conseguir combatiendo por la República.

Pero la principal causa de mi inquietud era no ver a mi lado a la
persona que más me interesaba en aquellos días. Le esperé toda la mañana
y toda la tarde, y como a ninguna hora parecía y había hecho promesa de
visitarme, creí que le pasaba algo desagradable. Por la noche no pude
refrenar mi ardorosa impaciencia y volé a su casa. Tampoco estaba en
ella, y el anciano portero y maestro de escuela, armado de fusil en
medio de la portería, furioso y exaltado cual si acabara de escaparse de
un manicomio, me inspiró tanto miedo que no quise esperar allí.

Pasé la noche en un estado de angustia horrible. Corrían rumores de que
al día siguiente habría saqueo, prisiones, muertes y escandalosas
escenas. Se decía que los liberales más señalados eran perseguidos por
las calles como perros rabiosos y apedreadas sus casas. Yo no podía
vivir. Al amanecer del otro día, que era el 20 de Mayo, busqué a
Salvador en diversos puntos, y tampoco le pude encontrar. Antes de
volver a casa vi movimiento de tropas en la Puerta del Sol y me dijeron
que Bessières había aparecido con sus cuadrillas que yo llamaba de
\emph{asesinos de la Fe}, por detrás del Retiro, amenazando entrar en
Madrid. La plebe de los barrios bajos se le había reunido, y como
hambrientos perros, aullaban mirando a la Corte, con ansias de
devorarla. Todo Madrid estaba aterrado, y yo más que nadie, no por el
temor del saqueo, sino por la sospecha de que la persona más cara a mi
corazón hubiera sido víctima del furor de la plebe.

Esperé también todo aquel día. Campos entró a darnos noticias de lo que
pasaba. Oíamos cañonazos lejanos, y a cada instante creíamos ver llegar
y difundirse por las calles a la desenfrenada turba salvaje ebria de
sangre y de pillaje. Pero Dios no quiso que en aquel día triunfaran los
malvados. El general Zayas destrozó a los asesinos de la Fe,
acuchillando a los chisperos y mujerzuelas que graznaban entre ellos. La
plebe aterrada volvió a sus oscuras guaridas, y mucha gente mala huyó a
los campos, aguardando a poder entrar con los franceses. Desde que
supimos el gran peligro a que habíamos estado expuestos los habitantes
de Madrid, todos deseábamos que llegasen de una vez los cien mil hijos
de San Luis, para que estableciendo un Gobierno regular, contuvieran a
la canalla azuzada por los realistas furibundos.

Al fin salí de la angustia que me atormentaba. En la mañana del día 21,
el prófugo, por quien yo había derramado tantas lágrimas, se presentó
delante de mí en estado bastante lastimoso, desencajado y lleno de
contusiones, con los ojos encendidos, seca la boca, cubierta de sudor la
hermosa frente, rotos y llenos de polvo los vestidos.

Al punto comprendí que había sido maltratado por las feroces bestias
populares. No le dije nada, y me apresuré a cuidarle, proporcionándole
alimento y reposo. Él me miraba con extraviados ojos. Apretando los
puños exclamó:

---¿Has visto a la canalla?

Necesitaba sosiego, y por todos los medios procuré tranquilizarle.

---No pienses más en eso---le dije,---y regocíjate ahora en la paz de mi
compañía y en esta dulce soledad en que estamos.

---¡No puedo, no puedo!---exclamó con gran agitación.

Y después repetía:

---¿Has visto a la canalla? ¡Pero qué canalla es la canalla!

Más tarde me contó que se había visto en gran peligro, porque al salir
de un sitio en que estaban reunidas varias personas contrarias al
despotismo, fue acometido, pudiendo salvar a duras penas la vida gracias
a su energía y al coraje con que se defendió.

Su estado febril inspirome bastante ansiedad aquella noche que pasó en
mi casa; pero a la mañana siguiente su prodigiosa naturaleza había
triunfado de la ebullición de la sangre irritada.

---No puedo ir a mi casa---me dijo,---y aun será peligroso que salga a
la calle; pero yo necesito disponer mi viaje.

---¿Vuelves al Norte?

---No; tengo que ir a Sevilla, donde está lo que queda de Gobierno
liberal. No tengo ya ni un resto siquiera de esperanza; pero es preciso
que cumpla fielmente la comisión del general Mina, y vaya hasta las
últimas extremidades, para que me quede al menos el consuelo de haberlo
intentado todo y para que se pueda decir esta verdad terrible: «No hubo
un solo liberal en España que supiera cumplir con su deber.»

---Pues si vas a Andalucía, iré contigo---dije con mucho gozo,
regocijándome ya con la idea de acompañarle y huir de Madrid, pueblo que
tanto alarmaba a mi conciencia.

---El viaje no será fácil---respondió sin demostrar grande entusiasmo
por mi compañía,---mayormente para una señora.

---Para mí todo es fácil.

---No se encontrarán carruajes.

---Como ruede el dinero, rodarán los coches.

---La policía vigilará la salida de los liberales.

---No importa.

Sin pérdida de tiempo empecé mis diligencias para nuestro viaje. Las
dificultades eran grandes. Ningún propietario de coches quería arriesgar
su material y sus caballerías, porque los facciosos se apoderaban de
ellas. No me acobardé, sin embargo, y seguí mis pesquisas. Campos
también deseaba proporcionar a mi amigo fácil escapatoria.

La entrada de los franceses, que se verificó el día 23, me dio alguna
esperanza; mas por desgracia entre las fuerzas de vanguardia no venía el
conde de Montguyon. Vi en cambio muchos guerrilleros del Norte, de fiero
aspecto, y temblé de pavor, deseando entonces más vivamente huir de la
Corte.

¡Y qué desorden en los primeros momentos de aquel día! Por mucha prisa
que se dieron los franceses a establecerse, no lograron impedir mil
excesos.

Hombres cuyo furor había sido pagado corrían por las calles celebrando
entre borracheras el horrible carnaval del despotismo. Rompían a
pedradas los cristales, trazaban cruces en las puertas de las casas
donde vivían liberales, como señal de futuras matanzas; escarnecían a
todo el que no era conocido por su exaltación absolutista; gritaban como
locos, maldiciendo la libertad y la Nación. No escapaban de sus
groserías las personas indiferentes a la política, porque era preciso
haber sido perro de presa del absolutismo para obtener perdón. Algunos
frailes de los que más habían escandalizado en el púlpito con sus
sermones sanguinarios eran llevados en triunfo.

Yo salía de misa de San Isidro, y me vi insultada y seguida por una
turba de mujerzuelas feroces, sólo porque llevaba un lazo verde. El
color verde era ya el color de la ignominia, como emblema del
liberalismo, que tantas veces había escrito sobre él Constitución o
muerte. Vi maltratar a un joven de buen porte, sólo porque usaba bigote,
y desde aquel día el tal adorno de las varoniles caras fue señal de
franc-masonismo y de extranjería filosófica.

Quien vio una vez tales escenas no puede olvidarlas. Mis ideas habían
cambiado mucho desde mi viaje a Francia. Conservando el mismo respeto al
Trono y al Gobierno fuerte, había perdido el entusiasmo realista. Pero
en aquel día tristísimo se desvanecieron en mi cabeza no pocos
fantasmas, y aunque seguí creyendo que uno solo gobierna mejor que
doscientos, el absolutismo popular me inspiró aversión y repugnancia
indecibles.

No había concluido de referir en mi casa el gran peligro que había
corrido por llevar un lazo verde, cuando entró Campos. Traía semblante
muy alegre.

---Ya está resuelta la cuestión de tu viaje---dijo a Salvador.---Esta
noche puedes marchar, si quieres.

---¿Cómo?---preguntamos él y yo.

---De un modo tan sencillo como seguro. El marqués de Falfán de los
Godos\footnote{Veánse \emph{El Grande Oriente}.} había pensado marchar a
Andalucía\ldots{} Como la pobre Andrea está tan delicada\ldots{} En fin,
se han decidido a salir esta noche. Tienen silla de postas propia. Al
punto me he acordado de ti, Falfán de los Godos tiene gusto en llevarte
y se alegra mucho de tu compañía.

---Eso no puede ser---dije vivamente, saliendo al encuentro de aquella
proposición con verdadera furia que trataba de disimular.

---¿Por qué no ha de poder ser, señora mía?---dijo Campos.---En la silla
de postas irán cómoda y seguramente el Marqués, mi sobrina con su hijo,
la doncella y dos criados que seremos nosotros, Salvador y yo.
Perfectísimamente.

El taimado masón se restregaba las manos en señal de regocijo.

---Me parece una excelente idea---dijo Monsalud mirándome.---¿No crees
tú lo mismo?

Yo no contesté nada. Estaba furiosa. Él debió comprender en mis ojos la
tempestad que se había desatado en mi corazón, mas no por conocerlo se
apresuró a conjurarla. Antes bien, ocupose de disponer su viaje con una
calma, con una indiferencia hacia mí que me irritaron más. Mi dignidad
me impedía pedir un puesto en aquel coche que se iba a llevar la mitad
de mi alma. La misma dignidad me impedía recordarle nuestro dulce
propósito de ir juntos. Encerreme breve rato en mi cuarto, para que
nadie conociese la alteración nerviosa que me sacudía, y con los dientes
hice pedazos un pañuelo inocente. Mis ojos secos e inflamados no podían
dar salida a la angustia de mi corazón, derramando una sola lágrima.

Cuando me presenté de nuevo, mi apariencia no podía ser más tranquila.
Afectaba naturalidad y hasta alegría; tanta era la fuerza de mi
disimulo, cuando yo llamaba todas las fuerzas de la voluntad para forjar
la máscara de hierro, bajo la cual escondía mi verdadero semblante,
lleno de luto y consternación. ¡Qué padecimiento tan grande! ¿Cómo no,
si Salvador mismo me había contado toda la historia de sus relaciones
con Andrea Campos, después marquesa de Falfán de los Godos? Yo la había
tratado bastante después de ser marquesa. La admirable hermosura de la
americanilla, representándose en mi imaginación, me la quemaba como un
hierro abrasado.

Tuve valor para verles partir. Vi a la sobrina de Campos subir al coche,
haciéndose la interesante con su languidez de dama enfermita; vi al
viejo Marqués engomado y lustroso, como un muñeco que acaba de salir del
taller de juguetes; vi a Salvador tomando en brazos y besando con el
mayor gusto al niño de la Marquesa\ldots{} no quise ver más. ¡El coche
partió!\ldots{} ¡Se fueron!\ldots{}

\hypertarget{xxi}{%
\chapter{XXI}\label{xxi}}

Se fueron y yo me quedé. Las lágrimas que antes no habían querido salir
de mis ojos brotaron a raudales, abrasándome las mejillas. No podía
dejar de pensar en la hipocritona, que corría por los campos desiertos,
lanzada por mí al interminable viaje de la desesperación; pero lejos de
tenerle lástima, aquel recuerdo avivaba mi hondo furor, haciéndome
exclamar:---¡Me alegro, mil veces me alegro!

¡Cuán grande había sido mi castigo! Para que este fuera más evidente,
fui condenada por Dios al mismo suplicio de viajar buscando a una
persona amada, al martirio indescriptible de correr un día y otro día
como el que huye de su sombra, siempre impaciente, siempre anhelante,
precipitada siempre de la esperanza al desengaño y del desengaño a una
nueva esperanza. Porque sí, yo emprendí también el viaje a Andalucía
tres días después. Estaba en la alternativa de morir de despecho o
correr también. Hubo en mí desde aquel día algo de la maldición
espantosa que pesaba sobre el judío errante, y me sentí como arrastrada
por la fuerza de un huracán.

¡Ay!, el huracán estaba dentro de mí misma, en mi despecho, en mis
celos, en un loco afán de no hallarme lejos de dos personas, cuya imagen
ni un solo instante se apartaba de mi pensamiento. Si mis lectores me
han conocido ya por lo que va contado de mi borrascosa vida,
comprenderán que yo no podía quedarme en Madrid. Mi carácter me lanzaba
fuera, como la pólvora lanza la bala.

Partí\ldots{} Pero antes debo decir cómo pude conseguir los medios para
ello. Mi primer paso fue recurrir a Eguía; mas desde la entrada de los
franceses le habían arrinconado como trasto viejo, y una Regencia fresca
y lozana funcionaba en su lugar. Nombrola Angulema de acuerdo con el
Consejo de Estado, y la componían los duques del Infantado y de
Montemart, el barón de Eroles, el obispo de Osma y don Antonio Gómez
Calderón. Secretario de ella era el venenoso Calomarde, al cual me
dirigí solicitando un pase y licencia para el uso de coche-posta.
Recibiome tan fríamente y con tanta soberbia e hinchazón, que no pude
menos de recordar al Don Soplado del poeta sainetero D. Ramón de la
Cruz.

Le desprecié como merecía y recurrí a don Víctor Sáez, nombrado Ministro
de Estado; pero este me recordó a la rana, cuando quiso parecerse al
buey. Tuvo el mal gusto de echarme en cara mi supuesta conversión al
constitucionalismo y a la Carta francesa, diciendo mil necedades
presuntuosas y aun amenazándome. Su fatuidad, semejante a la del pavo
cuando se sopla y arrastra las alas para meter ruido, me hizo reír en
sus propias barbas. El único que se me mostró algo propicio fue Erro,
hombre honrado y modesto. Pero nada positivo saqué de la flamante
situación, que daba pruebas de su agudeza política volviendo las cosas
\emph{al propio ser y estado que tenían en 7 de Marzo de} 1810,
restableciendo los antiguos Consejos y la Sala de Alcaldes de Casa y
Corte. Era esto volver a los tontillos, al guarda-infante y al pelo
empolvado.

Por mi ventura llegó a Madrid el conde de Montguyon. Le vi; hízome la
centésima declaración de amor y luego con semblante dolorido me dijo:

---Soy muy desgraciado, señora, en no poder estar cerca de vos. Tengo
que partir con el general Bourdesoulle para esa poética región que
llaman la Mancha, idealizada por las aventuras del gran caballero.

Entonces le manifesté que si me proporcionaba los medios de hacer el
viaje, poniendo yo por mi cuenta todos los gastos, le seguiría a aquel
encantado país que hizo célebre el gran caballero. Al oír esto se volvió
todo obsequios, y tres días después tenía yo a mi disposición una silla
de postas con caballos del cuartel general de Bourdesoulle y un pase que
me aseguraba el respeto de las turbas por todo el tránsito que iba a
recorrer.

Salí al fin de Madrid acompañada de mi doncella. Salí como el agua de
una esclusa cuando se le abren las compuertas que la sujetan. Yo no veía
bastante llanura por donde correr; en ningún momento me parecía que
andaba bastante mi coche; enfadábame el cansancio de las mulas, la
pesadez de los mesoneros y la flema del mayoral, que se ponía siempre de
parte de las caballerías en mi febril contienda con el tiempo y la
distancia.

En los pueblos por donde rápidamente pasaba, vi escenas que me causaron
tanta indignación como vergüenza. En Ocaña habían quitado las imágenes
que adornaban el ángulo de algunas calles, poniendo en su lugar el
retrato de Fernando, entre cirios y ramos de flores, y debajo la piadosa
inscripción: «¡Vivan las caenas!» En Tembleque presencié el acto solemne
de arrojar al pilón donde bebían las mulas, a dos o tres liberales y
otros tantos milicianos. En Madridejos tuve miedo, porque una turba que
invadía el camino cantando coplas tan disparatadas como obscenas quiso
detenerme, fundada en que el mayoral había tocado con su látigo el
estandarte realista que llevaba un fraile. Necesité mostrar mucha
serenidad y aun derramar algún dinero para que no me causasen daño; pero
no pude seguir hasta que no llegaron a aquel ilustrado pueblo las
avanzadas de la caballería francesa.

En Puerto Lápice se rompió una ballesta de mi coche, ocasionándome una
detención de dos días. Las horas eran siglos para mí. Me quemaba la
tierra bajo los pies. Yo hubiera deseado poseer la autoridad de una
reina asiática para vencer tantas dificultades, atando a los hombres al
pescante de mi coche. La desproporción enorme entre mi impetuoso anhelo
y los medios materiales de que disponía, me llevaron a un lamentable
estado nervioso que de ningún modo podía calmar. Únicamente logré un
poco de alivio a aquel penoso hervor de mi carácter empleando un medio
bastante pueril, pero que no parecerá muy absurdo a las mujeres que se
me asemejan. Consistía en tomar el látigo del mayoral y ponerme a
descargar furiosos latigazos sobre los robles del camino en Sierra
Morena y sobre los olivos de Andalucía.

En Sierra Morena hallé nuevos obstáculos. Allí había una especie de
ejército español, mandado por una especie de general, que tenía el
encargo de hacer una especie de resistencia a las tropas de
Bourdesoulle. Dios había decidido que no hubiese otro Bailén en la
historia, y los inocentes que creían en un nuevo 19 de Julio de 1808 se
llevaron gran chasco. ¡Parece mentira! Quince años después, los papeles
de aquel drama habían cambiado. Los personajes eran los mismos.
Creeríase que habían resucitado los muertos de la gloriosa época, pero
que al vestirse se habían equivocado de uniforme.

En pocas horas fue desbaratado Plasencia (que así se llamaba el general
que defendía la puerta de Andalucía) y los franceses pisaron el glorioso
campo de las Navas de Tolosa, de Menjíbar y de Bailén. Menos afortunada
yo, fui otra vez detenida; y ahora el conde de Montguyon, a quien
Bourdesoulle mandó situarse en Guarromán, mostró muy poco interés porque
yo siguiera adelante. Con todo, tales artes usé para sacar partido de su
caballería andante, que me libré de él muy lindamente. Por fin, el 6 de
Junio entré en Córdoba, donde no me detuve más que lo preciso.

El 9 por la tarde vi a lo lejos una inmensa mole rojiza que iluminaban
los rayos del moribundo sol. Ante mí se extendían hermosas llanadas de
trigo, como un campo de oro, cuya reverberación amarilla ofendía a los
ojos. Yo no había visto un cielo más alegre, ni un ambiente más
respirable y que más embelesase los sentidos, ni un crepúsculo más
delicioso. La enorme torre que se destacaba a lo lejos sobre apretado
caserío, y entre otras mil torres pequeñas, iba creciendo a medida que
yo me acercaba y parecía venir a mi encuentro con gigantesco paso. La
torre era la Giralda y la ciudad Sevilla.

\hypertarget{xxii}{%
\chapter{XXII}\label{xxii}}

¡Sevilla! ¡De qué manera tan grata hería mi imaginación este nombre!
¡Qué idealismo tan placentero despertaba en mí! No creo que nadie haya
entrado en aquel pueblo con indiferencia, y desde luego aseguro que el
que entre en Sevilla como si entrara en Pinto es un bruto. ¡El Burlador,
D. Pedro el Cruel, Murillo! Bastan estas tres figuras para poblar el
inmenso recinto que es en todas sus partes teatro de la novela y el
drama, lienzo y marco de la pintura. ¡Y hasta las pinturas sagradas son
allí voluptuosas! Para que nada le falte, hasta tiene a Manolito
Gázquez, cuyas hipérboles graciosas han dado la vuelta a España, y
parece que forman la base de la riqueza anecdótica nacional.

En Sevilla la noche y el día se disputan a cuál es más bello; pero
cuando llega el rigor del verano, vence irremisiblemente la noche,
asumiendo todos los encantos de la naturaleza y de la poesía. Para ella
son los delicados aromas de jazmines y rosas; para ella el picante rumor
de las conversaciones amorosas; para ella la dulce tibieza de un
ambiente que recrea y enamora, las quejumbrosas guitarras que expresan
todo aquello a que no pueden alcanzar las lenguas. Cuando yo llegué se
dejaba sentir bastante el calor, sin ser insoportable; pero las noches
eran deliciosas, un paraíso en el cual no se echaba de menos el sol.

Me alojé en una hermosa posada de la calle de Génova, y desde la noche
de mi llegada vi a muchos diputados que moraban allí y a otros que iban
a visitarles. Aquello era un hervidero de gente habladora, una olla
puesta al fuego. Sus agitadas disputas, sus gestos, sus furores
indicaban la gravedad de la situación.

Vivían conmigo Argüelles, Canga Argüelles, Salvato, Flórez Calderón, el
canónigo Villanueva y D. Cayetano Valdés el almirante. Iban a visitar a
estos Galiano, Istúriz, Beltrán de Lis, D. Ángel de Saavedra, después
duque de Rivas, y otros. Con algunos de ellos tenía yo amistad.
Oyéndoles supe que se había descubierto una conspiración tramada por
cierto general inglés llamado Downie, el mismo que había organizado una
partida de combatientes en la guerra de la Independencia. La
conspiración debió de ser muy inocente como todas las modas de aquel
tiempo, y todo en ella fue de sainete, hasta el descubrimiento, hecho
por un cirujano.

Tan sólo descansé en la noche de mi llegada, y el día siguiente, que era
el 10 de Junio, di principio a mis investigaciones, saliendo a hacer
algunas visitas. Al pasar por las calles más principales experimentaba
profunda emoción creyendo ver semblantes conocidos. Yo no sé qué había
en aquella fisonomía de la multitud para turbarme tanto; pero esto pasa
cuando lo que amamos se pierde en las oleadas del gentío, al cual presta
su rostro y su persona toda.

Aprovechando bien el día pude ver a muchas personas y dar con alguna que
me indicó el domicilio de los marqueses de Falfán. Este era el principal
objeto de mis impacientes ansias. Pero en aquel día 10 de Junio,
precursor de una de las fechas más célebres de nuestra historia, nadie
hablaba de otra cosa que de política, de la resistencia del Rey a
trasladarse a Cádiz y del empeño de los Ministros en llevárselo de grado
o por fuerza. Advertí entonces que no era Sevilla población muy liberal,
y que en la contienda entablada, la mayoría de los paisanos de Manolito
Gázquez se ponían de parte del Rey. Por un fenómeno extraño, la
aristocracia aparecía más enemiga del absolutismo que el pueblo; pero
esto no me causaba sorpresa, por haber observado el mismo contrasentido
en Madrid.

No pudiendo refrenar mi impaciencia, aquella misma noche fui a casa del
marqués de Falfán. Las visitas de noche son sumamente agradables en
verano y en aquel país, contribuyendo a ello los frescos patios trocados
en salones de tertulia. Nadie puede, sin haber visto estos agradables
recintos, formar idea de ellos y del hermoso conjunto que presentan las
plantas, la fuente de mármol con su murmurante surtidor, los espejos,
los cuadros al mismo tiempo iluminados por las bujías y por el rayo de
luna que penetra burlando el toldo, la dulce cháchara de las
conversaciones, más dulce a causa del gracioso ceceo bético, y por
último, las lindas andaluzas que alegrarían un cementerio, cuanto más un
patio de Sevilla.

Había pocas personas en casa de Falfán. Encontré a la Marquesa muy
desmejorada y triste en gran manera, lo cual no sé si me causó pena o
alegría. Creo que ambas cosas a la vez. Yo justifiqué mi viaje a
Sevilla, suponiendo asuntos de intereses, y no me atreví a preguntar por
él ni siquiera a nombrarle para que mi afectada indiferencia alejara
todo recelo. Tenía esperanza de verle entrar en el patio cuando menos lo
pensase, y me preparaba para no turbarme en el momento de su aparición.
Cualquier ruido de la puerta me hacía temblar, dándome los escalofríos
propios de la pasión en acecho.

Sin que me esté mal el decirlo, y poniendo la verdad por delante de
todo, aun de la modestia, yo estaba guapísima aquella noche, vestida al
estilo de París con una elegancia superior a cuanto veían mis ojos.
Harto me lo probaban los de los caballeros allí presentes, que no se
apartaban de mí, causando envidia a todas. Como los andaluces no son
cortos de genio, aquella noche recibí galanterías y donaires para el año
entero.

Mi afán consistía en sacar alguna luz, algún dato, alguna noticia, de mi
conversación con la marquesa de Falfán; pero fuese discreción suma o
ignorancia de la hermosa dama, ello es que nada dejó comprender. Hablaba
lo menos posible, y con sus miradas lo mismo que con el sentido de sus
palabras sólo una cosa me decía claramente, es a saber: que me aborrecía
de todo corazón. Yo, maestra consumada, disimulaba mejor que ella.

El marqués de Falfán de los Godos, hablándome de política, me distrajo
de esta batalla que yo daba a la taciturna reserva de Andrea. Las
aficiones que yo había mostrado en Madrid a las cosas públicas me
perdieron entonces, porque el buen señor me atacó con verdadera
ferocidad de charlatanismo, deseando saber mi opinión sobre sucesos y
personas. Mi fastidioso interlocutor era liberal templado, partidario de
un justo medio, muy justamente mediano, y de las dos Cámaras y del veto
absoluto. Había tenido sus repulgos de masón, repetía los dichos de
Martínez de la Rosa y era bastante volteriano en asuntos religiosos.
Defendía al clero como fuerza política; pero se burlaba de los curas,
del Papa y aun del dogma mismo, sin que esto fuera obstáculo para creer
en la conveniencia de que hubiese muchos clérigos, muchos obispos,
muchísimas misas y hasta Inquisición. En suma: las ideas del Marqués
eran el capullo de donde, corriendo días, salió la mariposa del partido
moderado.

Decir cuánto me mareó aquella noche fuera imposible. Tuve que saber
cosas que a la verdad me interesaban poco; por ejemplo: que Calatrava, a
la sazón presidente del Ministerio, no era hombre apropiado a las
circunstancias; que los masones primitivos o descalzos estaban en gran
pugna con los secundarios o calzados y ambos con los comuneros y
carbonarios; que los partidarios de San Miguel trabajaban por echarlo
todo a perder más de lo que estaba, y que cuando ocurrió el cambio de
Ministerio que había llevado al poder a los amigos de Calatrava, se
habían visto cosas muy feas. Exaltándose a medida que entraba en
materia, me dijo que él (el marqués de Falfán de los Godos) habría sido
ministro si hubiera querido, cuando se negó a serlo Flores Estrada; pero
que no quiso meterse en danzas; que él (el propio Marqués) había
previsto los terribles sucesos que ya estaban cerca, y que la ruina del
pobre sistema era ya inminente y segura. Apoyábanle en esto todos los
presentes, mientras yo me aburría a mis anchas oyéndole. Era para morir.

Habiendo dicho uno de los tertulios que Su Majestad se negaría
resueltamente a salir de Sevilla, el Marqués habló así:

---Pues el Gobierno insiste en llevárselo a Cádiz, ¡qué
tontería!\ldots{} y como el Rey insiste en no ir, el Gobierno piensa
declararle loco\ldots{} ¡Loco Su Majestad, señores, el hombre más cuerdo
de toda España, el único español que sabe a dónde va y por dónde ha de
ir!

Luego, dirigiéndose a mí y como quien habla en secreto, me dijo que
Calatrava era un hombre atolondrado; Yandiola, Ministro de Hacienda, una
nulidad, y el de la Guerra, Sánchez Salvador, un insensato.

Yo estaba nerviosa a más no poder. Las palabras se me venían a la boca
para contestarle de este modo:

---¿Y a mí qué me cuenta usted de todo eso señor Marqués? ¿Qué me
importa a mí que Calatrava sea un majadero, Yandiola y Sánchez Salvador
dos majaderos y usted más majadero que todos ellos?

Pero con no poco trabajo me contenía. Obligada a decir algo a causa de
mi pícara reputación, me complacía en contradecirle, de modo que todo lo
que para él era blanco, yo lo veía negro. A cuantos el Marqués denigró
yo les supuse talentos desmedidos. En lo relativo a declarar loco a Su
Majestad, dije que me parecía el acto más cuerdo y acertado del mundo.

---Pero, señora---me dijo el Marqués,---esto equivale a destronar a Su
Majestad, porque si le declaran incapacitado para reinar\ldots{}

---Justamente, señor Marqués---repuse.---Le destronan y luego le vuelven
a entronizar; le quitan y le ponen, según conviene a las circunstancias.
¿Hay cosa más natural? ¿El Rey no abre y cierra las Cortes? Pues las
Cortes abren o cierran al Rey cuando les acomoda.

Tomaron a risa, como lo merecían, mis observaciones; pero no por verme
tan inclinada a las burlas, cejó Falfán en su fastidioso disertar.

Entonces entró el príncipe de Anglona, personaje distinguido de la
fracción de Martínez de la Rosa y el duque del Parque, cuya vista me
causó grande alegría. El Príncipe dijo que al día siguiente habría
sesión muy interesante para discutir lo que debiera hacerse en virtud de
la negativa del Rey a salir de Sevilla. Yo le pedí una papeleta de
tribuna al duque del Parque y ofreció mandármela. Anglona se brindó a
llevarme a Palacio. Formando mi plan para el día siguiente, determiné
ver a Su Majestad y asistir a la sesión de las Cortes, encendiendo de
este modo una vela a San Miguel y otra al diablo.

El duque del Parque, cuando no podían oírlo los demás, me dijo con
malignidad:

---Mi secretario, a quien usted conoce, le llevará mañana la papeleta
para la galería reservada de las Cortes.

Al oír esto parece que se abrieron delante de mí los cielos. Mi alma se
llenó de alegría, que a no ser por el gran disimulo que eché sobre ella,
como se echa hipocresía sobre un pecado, hubiera sido advertida por la
concurrencia. Desde aquel momento todo se transformó a mis ojos. Cuanto
dijo el marqués de Falfán de los Godos lo encontré discreto y agudo y
sus majaderías me parecieron prodigios de ingenio y perspicacia
política. A todo le contesté, desplegando verbosidad abundante como en
mis mejores tiempos de Madrid, emitiendo juicios picarescos y
sentenciosos, juzgando a los personajes con graciosa malevolencia y
retratándoles con breves rasgos de caricatura. Desde aquel momento tuve
lo que me había faltado en toda la noche, ingenio. Respondí a las
galanterías, supe marear a más de cuatro, mortifiqué a la Marquesa,
alegré la reunión. Al retirarme no dejaba más que tristezas y
presentimientos detrás de mí. Yo me llevaba todas las alegrías.

\hypertarget{xxiii}{%
\chapter{XXIII}\label{xxiii}}

Desde muy temprano me levanté, pues poco dormí aquella noche. Las noches
de Sevilla no parece que son, como las de otras partes, para dormir. Son
para soñar en vela\ldots{} Le aguardaba con tanta impaciencia, que a
cada instante salía al balcón, esperando verle entre la multitud que
pasaba por la calle de Génova. De repente me anunciaron una visita. Creí
verle entrar; salí corriendo; pero mi corazón dio un vuelco quedándose
frío y quieto, cual si hubiera tropezado en una pared. Tenía delante al
príncipe de Anglona, un señor muy bueno, un caballero muy simpático, muy
atento, pero cuya presencia me contrariaba extraordinariamente en aquel
instante.

Venía para llevarme al Alcázar.

---Su Majestad---me dijo,---recibe ahora muy temprano. Anoche le
manifesté que estaba usted aquí y me rogó que la llevase a su presencia
hoy mismo.

Yo quise hacer objeciones, pretextando la inusitada hora, pues no habían
dado las once; pero nada me valió. Érame imposible resistir a aquella
majadería insoportable que revestía las formas de la más delicada
atención. Tampoco podía defenderme con dolor de cabeza, vapores u otros
recursos que tenemos para tales trances. Humillé la frente como víctima
expiatoria de las conveniencias sociales, y después de arreglarme me
dispuse a aceptar un puesto en la carroza del Príncipe, no sin dejar
antes a mi criada instrucciones muy prolijas para que detuviera hasta mi
vuelta al que forzosamente había de venir. Partí resuelta a hacer a Su
Majestad visita de médico. En aquella ocasión deploré por primera vez
que existieran Reyes en el mundo.

Poca es la distancia que hay de la calle de Génova al Alcázar. Antes de
las doce estaba yo en la Cámara de Su Majestad y salía gozoso a
saludarme el descendiente de cien Reyes, pegado a su regia nariz. No
parecía nada contento; pero mostró mucho placer en verme, dándome a
besar su mano y rogándome que me sentase a su lado. Tanta bondad que a
cualquiera habría ensoberbecido, a mí me hizo muy poca gracia, y menos
cuando con sus preguntas daba a entender que la visita sería larga.

Fernando quiso saber por mí algunas particularidades de la entrada de
los franceses en Madrid, de la defección de La Bisbal en Somosierra y de
la derrota de Plasencia en Despeñaperros. Yo contesté a todo, cuidando
de la brevedad más que de otra cosa, y fingiéndome ignorante de varios
hechos que sabía perfectamente; pero ninguna de estas estratagemas me
valía, porque Fernando VII, que en el preguntar había sido siempre
absoluto, no se hartaba de oír contar cada paso del ejército francés; y
como además de mis palabras, le recreaba bastante, como he dicho en otra
ocasión, la boca que las decía, de aquí que no llevara camino de saciar
en muchas horas la curiosidad de su entendimiento y la concupiscencia de
sus voraces ojos.

---¡Ay!, ¡qué felices son las repúblicas!---pensé.---Al menos, en ellas
no hay Reyes pesados y preguntones que quieran saber noticias de la
guerra a costa de la felicidad de sus súbditos.

Yo le miraba haciendo esfuerzos heroicos para disimular mi descontento.
Al responderle, decía en mi interior:

---Me alegraría de que te encerraran en una jaula como loco rematado.

Él entonces, sin indicios de conocer mi cansancio, hablome así con
cierto tono de confianza:

---Se empeñan en que me han de llevar a Cádiz, y yo me empeño en no
salir de Sevilla. Veremos si se atreven a llevarme a la fuerza o si yo
cedo al fin.

---No se atreverán, señor.

---Ellos saben---continuó;---que en Cádiz hay una terrible epidemia;
pero eso no les importa. ¡A Cádiz de cabeza! ¿Nada importa, señores
diputados, que yo y toda la real familia nos expongamos a
perecer?\ldots{} Veremos lo que decide el Consejo\ldots{}

---Decidirá lo más conveniente.

---Yo les digo a esos señores: ¿Creen ustedes posible resistir a los
franceses? No.~Pues si al fin se ha de capitular, ¿no es mejor hacerlo
en Sevilla?

---Admirable raciocinio, señor.

---Nada, a Cádiz, a Cádiz, y entretanto ni coches para el viaje, ni
recursos\ldots{}

Parecía mortificado por dos o tres ideas fijas que agitadamente se
sucedían en su mente y se enlazaban formando esa dolorosa serie de
vibrantes círculos cerebrales que, si no producen la locura, la imitan.
Me fue preciso en vista de tanta pesadez, fingirme enferma y pedirle
permiso para retirarme. Él entonces, ¡oh fiero y descomunal tirano!, se
empeñó en que me quedase en el Alcázar, donde se me prepararía
habitación conveniente.

---Te comprendo, déspota---dije para mí sofocando mi cólera.

No había más remedio que ser huraña y descortés, rehusando los obsequios
y tapando mis oídos a preguntillas que empezaban a dejar de ser
políticas. Al retirarme, Su Majestad me dijo:

---No saldré de Sevilla, no saldré\ldots{} Veremos si se atreven.

---No se atreverán, señor---le respondí.---Vuestra Majestad podrá, con
una firme voluntad, desbaratar las maquinaciones de los pérfidos.

Estas vulgaridades palaciegas le agradaban. Le dejé entregado a sus
febriles inquietudes y corrí a calmar las mías. Por el camino iba
contando el tiempo transcurrido, que me parecía largo, como todo lo que
precede a la felicidad que se espera. Llegué a mi casa, subí
precipitadamente, creyendo que él saldría a recibirme con los brazos
abiertos; pero en mis habitaciones hallé un silencio y un vacío
tristísimos\ldots{} No estaba. Mi primer impulso fue de ira contra él
por la audacia inaudita, por la infame crueldad de no estar allí; pero
luego tornáronse contra el Rey mis furores, cuando Mariana, mi fiel
criada, me dijo que el caballero se había cansado de esperar.

---¿Luego ha estado aquí?

---Sí señora; ha estado más de hora y media. No haría diez minutos que
usted había salido, cuando entró\ldots{}

---¿Y no dijo que volvería?

---No dijo nada más sino que tenía que ir a las Cortes.

---Yo también tengo que ir a las Cortes---dije sintiéndome como una
máquina loca que mueve a la vez, con precipitada carrera todas sus
ruedas.---Vamos, vístete, Mariana, que no quiero perder esa gran sesión.

Por no ir sola, yo llevaba siempre conmigo a mi leal criada, vestida de
señora, imitando en esto la usanza francesa de las señoritas de
compañía. Esto era sumamente cómodo para mí, porque me libraba de la
necesidad de admitir en muchos casos la compañía de hombres importunos o
antipáticos. En poco tiempo, haciendo yo de sirviente y Mariana de
señora, quedó vestida, no tan bien que se desconociese su inferioridad
con respecto a mí; pero con suficiente elegancia para poder ir al lado
mío. Muchos la creían hermana soltera o parienta pobre.

\hypertarget{xxiv}{%
\chapter{XXIV}\label{xxiv}}

Fuimos a las Cortes, que estaban en San Hermenegildo, en la calle de La
Palma, frente a San Miguel. Difícil hallamos la entrada a causa de la
mucha gente que llenaba la calle agolpándose en las puertas del edificio
como las apiñadas lapas en la roca. Mujeres menos resueltas que nosotras
habrían vuelto la espalda; pero Mariana y yo sabíamos romper las
cortezas del vulgo y al fin nos abrimos paso, y entrando con desenfado y
pie ligero subimos a la galería. Desde antes de entrar en ella oímos la
voz de un orador que resonaba en medio del más imponente silencio.

Mucho hubimos de bregar para encontrar asiento, pero al fin pidiendo mil
veces perdón y oyendo murmullos de descontento a un lado y otro logramos
acomodarnos. Mi primer cuidado no fue atender a lo que aquel gran orador
decía, cosas sin duda altamente dignas de aplauso; mi primer cuidado fue
registrar con los ojos toda la galería reservada por ver si estaba allí
quien me cautivaba más que los discursos. Pero ni a derecha ni a
izquierda, ni delante ni detrás le vi, con lo cual la gran pieza
oratoria que se estaba pronunciando empezó a serme muy fastidiosa.

---¿Quién habla?---pregunté a una señora vieja que estaba junto a mí.

---Alcalá Galiano, el gran orador---repuso en tono de extrañeza por mi
ignorancia.

---¿Y de qué habla?---pregunté sin temor de que la señora vieja me
creyera cerril.

---¿De qué ha de hablar? Del suceso del día.

La señora volvió el rostro hacia el salón, demostrando más interés por
el discurso que por mis preguntas. Yo no quise molestar más, y traté de
atender también. El orador hablaba de la patria, del inminente peligro
de la patria, y de la salvación de la patria y de la gloria de la
patria. Es el gran tema de todos los oradores, incluso los buenos. No he
conocido a ningún político que no estropeara la palabra patriotismo
hasta dejarla inservible, y en esto se me parecen a los malos poetas,
que al nombrar constantemente en sus versos la inspiración, la lira, el
estro, la musa ardiente, la fantasía, hablan de lo que no conocen.

Alcalá Galiano era tan feo y tan elocuente como Mirabeau. Su figura,
bien poco académica y su cara no semejante a la de Antinoo, se
embellecían con la virtud de un talismán prodigioso, la palabra. Le
pasaba lo contrario que a muchas personas de admirable hermosura, las
cuales se vuelven feas desde que abren la boca. Aquel día, el joven
diputado andaluz había tomado por su cuenta el llevar adelante la hazaña
más revolucionaria que registran nuestros anales.

Los españoles sentían la comezón de destronar algo, y el afán de probar
la embriaguez revolucionaria que sin duda embelesa a los pueblos de
Occidente como a los chinos el opio, y dijeron: «hagamos temblar a los
Reyes, pues que ha llegado la hora de que los reyes tiemblen delante del
pueblo\ldots» Mas era aquí la gente demasiado bondadosa para una
calaverada sangrienta. En otra parte al ver al Rey sistemáticamente
contrario a la Representación nacional, le hubieran cortado la cabeza;
aquí le privaron del uso de la razón temporalmente, diciendo: «Señor,
vuestro deseo de esperar aquí a los franceses nos prueba que estáis
loco. Con arreglo a la Constitución declaramos que sois digno de un
manicomio y de perder la autoridad real. Vámonos a Cádiz, y cuando
estemos allí, os adornaremos de nuevo con vuestra cabal razón, y
seguiremos partiendo un confite como hasta aquí.»

Admirable recurso habría sido este a mi parecer, desde el punto de vista
liberal, teniendo un gran ejército para reforzar el argumento en los
campos de batalla. Sin fuerza, aquel hecho probaba que los diputados
estaban más locos que el Rey, y así se lo dije a Falfán de los Godos.
Con esto se comprende que el Marqués había entrado en la galería,
colocándose detrás de mí. Él ponía mucha más atención que yo al discurso
y aun a los rumores que sonaban arriba y abajo.

---Han llenado de gentuza la tribuna pública---me dijo en voz
baja,---para que aplauda las atrocidades que habla ese hombre.

No sé si era o no gente pagada, pero es lo cierto que a cada párrafo
coruscante, terminado en la salvación de la patria o en el afrentoso
yugo de esta Nación heroica, la galería pública mugía como una tempestad
cercana. ¡Qué rugidos, qué gestos de bárbaro entusiasmo, qué manera de
apostrofar! Algunas señoras tuvieron miedo y se retiraron, lo cual me
agradó en extremo, porque la tribuna se quedó muy holgada.

---¿Piensa usted seguir hasta el fin?---me dijo el marqués de Falfán
endulzando su mirada hasta un extremo empalagoso.

---Estaré algún tiempo más---le dije.---No me he cansado todavía.

Y miraba a diestra y siniestra esperando verle y no viéndole nunca. Los
que me conocen comprenderán mi aburrimiento y pena. No hay tormento peor
que tener ocupada la mente por una idea fija que no puede ser desechada.
Es una espina clavada en el cerebro, una acerada punta que hiere, y que
sin embargo no se puede ni se quiere arrancar. Yo procuraba distraerme
de aquel a manera de dolor agudísimo, charlando con Falfán; pero no
conseguí nada. La locura del Rey, declarada por una votación que iba a
verificarse, la exaltación revolucionaria de los diputados, la
elocuencia fascinadora de Galiano, no bastaban a dar otra dirección a
las fuerzas de mi espíritu.

---¿Y usted qué cree?---me preguntó el Marqués.

---Yo no creo nada---respondí con el mayor hastío.---Si he de hablar con
franqueza, nada de esto me importa gran cosa.

---¡Que declaren loco a Su Majestad!\ldots{}

---Lo mismo que si lo declararan cuerdo\ldots{} Yo soy así\ldots{}
Parece que se cansan ---añadí reparando que se suspendían los discursos.

---Es que ahora va una comisión de las Cortes al Alcázar a intimar al
Rey. Si no se resigna a salir\ldots{}

---¿Habrá más discursos?

---Las Cortes están en sesión permanente. Después vendrá lo más
interesante, lo más dramático; yo no pienso moverme de aquí.

---Su Majestad ha de responder que no sale de Sevilla. Me lo ha dicho
esta mañana, y aunque no tengo gran fe en su palabra, parece que por
esta vez va a cumplir lo que dice.

---Lo mismo creo, señora. En ese caso, las Cortes, después de este
respiro que ahora se dan, están dispuestas a poner en ejecución el
artículo 187 de la Constitución\ldots{}

---¿Y qué dice ese artículo?\ldots{}

En el momento de formular esta pregunta me estremecí toda, y me pasó por
delante de los ojos una claridad relampagueante. Le vi: había entrado en
la tribuna inmediata y volvía sus ojos en todas direcciones, como
buscándome. Desde aquel instante las palabras del Marqués no fueron para
mí sino un zumbido de moscardón\ldots{} Por fin sus ojos se encontraron
con los míos.

---¡Gracias a Dios!---le dije, empleando tan sólo el lenguaje de las
pupilas.

El Marqués seguía hablando. Para que no descubriese mi turbación, ni se
enojase al verme tan distraída, le pregunté de nuevo:

---¿Y qué dice ese artículo?

---Si se lo he explicado a usted---repuso.---Sin duda no me presta
atención. Es usted muy distraída.

---¡Ah!, sí\ldots{} estaba pensando en ese pobre Fernando.

---El mejor procedimiento, a mi modo de ver---manifestó Falfán de los
Godos gravemente,---sería\ldots{}

---¡Que le cortaran la cabeza!---indiqué mostrándome, sin cuidarme de
ello, tan revolucionaria como Robespierre.

---¡Qué cosas tiene usted!---exclamó el Marqués, riendo.

Y siguió hablándome, hablándome, es decir, zumbando como un abejorro.
Pasados diez minutos, creí conveniente dirigirle otra vez la palabra, y
repetí mi preguntilla.

---¿Y qué dice ese artículo?

---Por tercera vez se lo diré a usted.

Entonces me fue forzoso dedicarle un pedacito de atención.

---El artículo 187 dice poco más o menos que cuando se considere a Su
Majestad imposibilitado moralmente para ejercer las funciones del poder
ejecutivo, se nombre una Regencia\ldots{}

---¿Cómo la de Urgel?

---Una Regencia constitucional, señora, que desempeñe aquellas
funciones\ldots{}

---¡Oh!, señor Marqués, en todo soy de la misma opinión de
usted---exclamé con artificiosa admiración.---En pocos hombres he visto
un juicio tan claro para hacerse cargo de los sucesos.

Miré a Salvador. Pareciome que con los expresivos ojos me decía:
«Salgamos.» Y al mismo tiempo salía.

---Yo me retiro, señor Marqués---dije de improviso levantándome.

---Señora: ¡se marcha usted en el momento crítico!---exclamó con asombro
y pena.---Se van a reanudar estas interesantes discusiones. ¡Qué
discursos vamos a oír!

---Estoy fatigada. Hace mucho calor.

---Sin embargo\ldots{}

Mientras en el salón resonaba un rumor sordo como el anuncio de
furibunda tempestad parlamentaria, Mariana y yo nos dispusimos a salir;
pero en el mismo instante, ¡oh contrariedad imprevista!, multitud de
caballeros y señoras entraron en la tribuna. Eran los que habían salido
durante el período de descanso, que regresaban a sus puestos para
disfrutar de la parte dramática de la sesión. Además, numeroso gentío
recién venido se apiñaba en la puerta. No era posible salir.

---Señora---me dijo el Marqués,---ya ve usted que no es fácil la salida.
No pierda usted su asiento. Esto acabará pronto.

No tuve más remedio que quedarme. Caí en mi asiento como un reo en su
banquillo de muerte. Lo que principalmente me apenaba era que entre la
multitud había desaparecido el que bastaba a alegrar o entristecer mi
situación. En la muralla de rostros humanos, ávidos de curiosidad, no
estaba su rostro ni otro ninguno que se le pareciese.

---Sin duda me aguarda fuera---pensé.---¡Qué desesperación! ¡Cuándo
acabará esta farsa!\ldots{}

\hypertarget{xxv}{%
\chapter{XXV}\label{xxv}}

---La comisión que fue con el mensaje a Palacio---dijo el Marqués
alargando su rostro para abarcar con una mirada todo el salón,---ha
vuelto y va a manifestar la respuesta de Su Majestad.

---Que le maten de una vez---indiqué en voz baja.---¿Dice usted, señor
Marqués, que esto acabará pronto?

---Quizás no. Me parece que tendremos para un rato. Cosas tan graves no
se despachan en un credo.

Pensé que se me caía el cielo encima. El profundo silencio que reinó
durante un rato en aquel recinto, obligome a atender brevemente a lo que
abajo pasaba. Un diputado en quien reconocí al almirante Valdés, tomó la
palabra.

Pudimos oír claramente las palabras del marino al decir: «Manifesté a Su
Majestad que su conciencia quedaba salva, pues aunque como hombre podía
errar, como Rey constitucional no tenía responsabilidad alguna; que
escuchase la voz de sus consejeros y de los representantes del pueblo, a
quienes incumbía la salvación de la patria. Su Majestad respondió: He
dicho, y volvió la espalda.

Cuando estas últimas palabras resonaron en el salón, un rumor de olas
agitadas se oyó en las tribunas, olas de patriótico frenesí que fueron
encrespándose y mugiendo poco a poco hasta llegar a un estruendo
intolerable.

---Todos esos que gritan están pagados---me dijo el Marqués.

Entonces miré hacia atrás, pues no podía vencer el hábito adquirido de
explorar a cada instante la muchedumbre, y le vi. Estaba en la postrera
fila: apenas se distinguía su rostro.

---¡Ah!---exclamé para mí con gozo.---¡No me has abandonado! Gracias,
querido amigo.

Advertí que desde el apartado sitio donde se encontraba atendía a la
sesión con toda su alma. Mi pensamiento debía de estar donde estaba el
suyo, y atendí también. Segura de tenerle cerca; segura de que fiel y
cariñoso me aguardaba, pude tranquilamente fijar mi espíritu en aquella
turbulenta parte de la sesión, y en el orador que hablaba. Era otra vez
Galiano. Su discurso que en otra ocasión me hubiera fastidiado, entonces
me pareció elocuente y arrebatador.

¡Qué modo de hablar, qué elegancia de frase, qué fuerza de pensamiento y
de estilo, qué ademán tan vigoroso, qué voz tan conmovedora! Siendo mis
ideas tan contrarias a las suyas entonces, no pude resistir al deseo de
aplaudirle, enojando mucho al Marqués con mi llamarada de entusiasmo.

---¡Oh, señor Marqués!---le dije.---¡Qué lástima que este hombre no
hable mal! ¡Cuánto crecería el prestigio del realismo si sus enemigos
carecieran de talento!\ldots{}

Los argumentos del orador eran incontestables dentro de la situación y
del artículo 187 que intentaban aplicar. «No queriendo Su Majestad,
decía, ponerse en salvo, y pareciendo a primera vista que Su Majestad
quiere ser presa de los enemigos de la patria, Su Majestad no puede
estar en el pleno uso de su razón. Es preciso, pues, considerarle en un
estado de delirio momentáneo, en una especie de letargo pasajero\ldots{}

Estas palabras compendiaban todo el plan de las Cortes. Un Rey
constitucional que quiere entregarse al extranjero está forzosamente
loco. La Nación lo declara así y se pasa sin Rey durante el tiempo que
necesita para obrar con libertad. ¡Singular decapitación aquella! Hay
distintas maneras de cortar la cabeza, y es forzoso confesar que la
adoptada por los liberales españoles tiene cierta grandeza moral y
filosófica digna de admiración. «Antes que arrancar de los hombros una
cabeza que no se puede volver a poner en ellos, dijeron, arranquémosle
el juicio, y tomándonos la autoridad real, la persona jurídica, podremos
devolverlas cuando nos hagan falta.»

Yo miraba a cada rato a mi adorado amigo, y con los ojos le decía:

---¿Qué piensas tú de estos enredos? Luego hablaremos y se ajustarán las
cuentas, caballerito.

No duró mucho el discurso de Galiano, porque aquello era como lo muy
bueno, corto, y habían llegado los momentos en que la economía de
palabras era una gran necesidad. Cuando concluyó, las tribunas
prorrumpieron en locos aplausos. Entre las palmadas, semejantes por su
horrible chasquido a una lluvia de piedras, se oían estas voces: «¡A
nombrar la Regencia! ¡A nombrar la Regencia!»

---Señora---me dijo el Marqués horrorizado,---estamos en la Convención
francesa. Oiga usted esos gritos salvajes, esa coacción bestial de la
gente de las galerías.

---Van a nombrar la Regencia.

---Antes votarán la proposición de Galiano. ¡Atentado sacrílego, señora!
Me parece que asisto a la votación de la muerte de Luis XVI.

---¡Qué exageración!

---Señora---añadió con solemne acento.---Estamos presenciando un
regicidio.

Yo me eché a reír. Falfán, enfureciéndose por el regicidio que se
perpetraba a sus ojos, e increpando en voz baja a la plebe de las
galerías, era soberanamente ridículo.

---Lo que más me indigna---exclamó pálido de ira,---es que no dejen
hablar a los que opinan que Su Majestad no debe ser destronado.

En efecto: con los gritos de ¡fuera!, ¡que se calle!, ¡a votar!,
ahogaban la voz de los pocos que abrazaron la causa del Rey. La
Presidencia y la mayoría, interesadas en que las tribunas gritasen, no
ponían veto a las demostraciones. Veíase al alborotado público agitando
sus cien cabezas y vociferando con sus cien bocas. En la primera fila
los brazos gesticulaban señalando o amenazando, o golpeaban el antepecho
con las bárbaras manos que más bien parecían patas. Muchas señoras de la
tribuna reservada se acobardaron y diose principio al solemne acto de
los desmayos. Esto fue circunstancia feliz, porque la tribuna empezó a
despejarse un poco, haciendo menos difícil la salida.

---Señor Marqués---dije tomando la resolución de marcharme.---Me parece
que es bastante ya.

---¿Se va usted? Si falta lo mejor, señora.

---Para mí lo mejor está fuera. Aquí no se respira. Adiós.

---Que van a votar. Que vamos a ver quiénes son los que se atreven a
sancionar con su nombre este horrible atentado.

---Ahí tiene usted una cosa que a mí no me importa mucho. ¿Qué quiere
usted?, yo soy así. Dormiré muy bien esta noche sin saber los nombres de
los que dicen sí.

---Pues yo no me voy sin saberlo. Quiero ver hasta lo último; quiero ver
remachar los clavos con que la Monarquía acaba de ser crucificada.

---Pues que le aproveche a usted, señor Marqués\ldots{} Veo que ya se
puede salir. Adiós, tantas cosas a la Marquesa. Ya sabe que la quiero.

No hice muy larga la despedida por temor a que tuviese la deplorable
ocurrencia de acompañarme. Salí. ¡Ay!, aquella libertad me supo a
gloria. ¡Con qué placentero desahogo respiraba! Al fin iba a satisfacer
mi deseo, la sed de mis ojos y de mi alma, que ha tiempo no vivían sino
a medias. Desde que salí a los pasillos le vi allá lejos esperándome.
Hízome una seña y ambos procuramos acercarnos el uno al otro, cortando
el apretado gentío que salía. Pero cuando estaba a seis pasos de él,
sentí detrás de mí la áspera voz de Falfán, la cual me hizo el efecto de
un latigazo. Volvime y vi su sonrisa y sus engomados bigotes que yo
creía haber perdido de vista por muchos días.

---Señora, no se me escape usted---me dijo, ofreciéndome su brazo.---He
salido porque la votación no es nominal. Esos pícaros han votado
levantándose de su asiento\ldots{} ¡qué escándalo!\ldots{} ¡Votar así un
acuerdo tan grave!\ldots{} ¡Tienen vergüenza y miedo!\ldots{} ya se
ve\ldots{} Tome usted mi brazo, señora.

La importuna presencia del estafermo me dejó fría. No tuve otro remedio
que apoyar mi mano en su brazo y salir con él. Frente a nosotros vi a
Salvador, que me pareció no menos contrariado que yo.

---Querido Monsalud---le dijo el Marqués,---¿ha visto usted la sesión?
¡Gran escena de teatro! Me parece que correrá sangre.

No recuerdo lo que ambos hablaron mientras bajamos a la calle. Me daban
ganas de desasirme del brazo del Marqués, y empujarle con todas mis
fuerzas para que fuera rodando por la escalera abajo, que era bastante
pendiente. Pero me fue forzoso tener paciencia y esperar, fiando en que
el insoportable intruso nos dejaría solos al llegar a la calle. ¡Vana
ilusión! Sin duda se habían conjurado contra mí todas las potencias
infernales. El marqués de Falfán, empleando su relamido tono, que a mí
me sonaba a esquilón rajado, me dijo:

---Ahora, dígnese usted aceptar mi coche y la llevaré a su casa.

---Si yo no voy a mi casa---repuse vivamente.---Voy a visitar a una
amiga\ldots{} o quizás como ya es tarde y no hace calor, daremos Mariana
y yo un paseo.

---Bien, a donde quiera usted que vaya la acompañaré---dijo el Marqués
con la inexorable resolución de un hado funesto.---Y usted, Salvador, ¿a
dónde va?

---Tengo que ver a un amigo junto a San Telmo.

---Entonces no digo nada. Si va usted en esa dirección no puedo
llevarle. Y usted, Jenara, ¿a dónde quiere que la lleve?

---Mil gracias, un millón de gracias, señor Marqués---repuse.---El
movimiento del coche me marea un poco. Me duele la cabeza y necesito
respirar libremente y hacer algo de ejercicio. Mariana y yo nos iremos a
dar una vuelta por la orilla del río.

Bien sabía yo que el señor Marqués no gustaba de pasear a pie y que en
aquellos días estaba medianamente gotoso. Yo no quería que de ningún
modo sospechase Falfán que Salvador y yo necesitábamos estar solos. Al
indicar yo que iría a pasear por la orilla del río, claramente decía a
mi amado:---Ve allá y espérame, que voy corriendo, luego que me sacuda
este abejón.

Comprendiéndome al instante, por la costumbre que tenía de estudiar sus
lecciones en el hermoso libro de mis ojos, se despidió. Bien claro leí
yo también en los suyos esta respuesta: «Allá te espero: no tardes.»

Luego que nos quedamos solos, el Marqués reiteró sus ofrecimientos.
Parecía que no rodaba en el mundo más carruaje que el suyo según la
oficiosidad con que lo ponía a mi disposición.

---La tarde está hermosa. Deseo pasear un poco a pie, repetí, como quien
ahuyenta una mosca.

---Pues entonces---me contestó estrechándome la mano,---no quiero
alejarme de aquí; aún debe pasar algo importante. A los pies de usted,
señora.

Al fin\ldots{} al fin me soltó aquel gavilán de sus impías
garras\ldots{} Mariana y yo nos dirigimos apresuradamente a la margen
del Guadalquivir.

---¡Ahora si que no te me escapas, amor!---pensaba yo.

\hypertarget{xxvi}{%
\chapter{XXVI}\label{xxvi}}

Cuán largo me pareció el camino. Mariana y yo íbamos con más prisa de la
que a dos señoras como nosotras convenía. Pero aun conociendo que
parecíamos gente de poco más o menos, cuando vi la Torre del Oro, los
palos de los barcos y los árboles que adornan la orilla, avivé más el
paso. No faltaba gente en aquellos deliciosos sitios; mas esto me
importaba poco.

---Vamos hacia San Telmo---dije a Mariana.---Creo que es aquel edificio
que se ve más abajo entre los árboles.

---Aquel es.

---Mira tú hacia la izquierda y yo miraré hacia adelante para que no se
nos escape. Dijo que me esperaría en San Telmo.

---Ya le veo, señora. Allí está.

Mariana le distinguió a regular distancia y yo también le vi. Me
aguardaba puntualmente.

---¡Ah, bribón, ya eres mío!---pensé, deteniendo el paso, segura al fin
de que no se me escaparía.

Él miraba hacia la puerta de Jerez, como si nos aguardara por allí.
Avanzamos Mariana y yo, dando un pequeño rodeo para acercarnos a él por
detrás, y sorprenderle, sacudiéndole el polvo de los hombros con
nuestros abanicos. Yo sonreía.

Distábamos de él unos diez pasos, cuando sentí que me llamaban.

---¡Jenara, Jenara!---oí detrás de mí, sin poder precisar en el primer
instante a quién pertenecía aquella horrible e importuna voz.

Volvime y el coraje me clavó los pies en el suelo. Era el marqués de
Falfán de los Godos, que venía hacia mí sonriendo y cojeando. Tan
confundida estaba que no le pude decir nada ni contestar a sus
empalagosos cumplidos.

---Vaya que ha corrido usted, amiguita---me dijo.---Yo acabo de llegar
en coche\ldots{} Es que en el momento de separarnos se me ocurrió una
cosa\ldots{}

---¿Qué cosa?

---Padecí un gran olvido---dijo relamiéndose.---Dispénseme usted. Como
usted dijo que venía a pasear a este sitio\ldots{}

---¿Y qué?\ldots{} ¿qué?\ldots{} ¿qué?

Según me dijo después Mariana, yo echaba fuego por los ojos.

---Que olvidé ofrecerme a usted para una cosa que, sin duda, le será muy
agradable.

---Señor Marqués, usted se burla de mí.

---¡Burlarme! No, hija mía: al punto que nos separamos, dije para mí:
«¡Qué desatento he sido!» Puesto que va al río, debí brindarme a
acompañarla para ver el vapor y mostrarle ese prodigio de la industria
del hombre.

---¡Usted está loco, sin duda!---afirmé ocultando todo lo posible mi
despecho;---¿qué es eso del vapor? No entiendo una palabra.

---¡El vapor, señora! Es lo que más llama la atención de todo Sevilla en
estos días.

---¿Y qué me importa?---dije bruscamente siguiendo mi camino.

---Dispénseme usted si la he ofendido---añadió el Marqués
siguiéndome;---pero como venía usted a pasear al río, y como yo tengo
entrada libre siempre que quiero en esa prodigiosa máquina, creí que la
complacería a usted apresurándome a mostrársela.

---¿Qué máquina es esa?---le pregunté deteniéndome.

Al decir esto había perdido de vista al imán de mi vida.

---Mire usted hacia allá junto a la Torre del Oro.

Miré, y en efecto vi un buque de forma extraña, con una gran chimenea
que arrojaba negro y espeso humo. Sus palos eran pequeños y sobre el
casco sobresalía una armazón bastante parecida a una balanza.

---¿Qué es eso?---pregunté al Marqués.

---El vapor, una invención maravillosa, señora. Esos ingleses son el
Demonio. Ya sabe usted que hay unas máquinas que llaman de vapor, porque
se mueven por medio de cierto humo blanquecino que va enredando de tubo
en tubo\ldots{}

---Ya sé\ldots{}

---Pues los ingleses han aplicado esta máquina a la navegación, y ahí
tiene usted un barco con ruedas que corre más que el viento y contra el
viento. Esto cambiará la faz del mundo. Yo lo he predicho y no me
equivocaré.

Mirando hacia la máquina prodigiosa, vi a Salvador que se dirigía hacia
la Torre del Oro. Veámoslo de cerca, señor Marqués---dije marchando
hacia allá.---Verdaderamente, ese barco con ruedas es una maravilla.

---Creo que ahora va a dar un par de vueltas por el río, para que lo
vean Sus Altezas Reales que están, si no me engaño, en la Torre del Oro.

---Corramos.

---¡Va toda la gente hacia allá! Descuide usted, podremos entrar, si
usted quiere. El capitán es muy amigo mío y los consignatarios son mis
banqueros.

---¿De quién es esa máquina?

---De una sociedad inglesa. De veras hubiera sentido mucho no
mostrársela a usted esta tarde. Cuando me acordé, faltábame tiempo para
acudir a reparar mi grosería.

---Gracias, señor Marqués.

Dejé de ver entonces la luz de mi vida. Mi corazón se llenó de angustia.

---Yo estaba seguro de agradar a usted---me dijo Falfán.---Es un asombro
ese buque.

---Un asombro, sí: apresuremos el paso.

---Si no se nos ha de marchar.

---¡Que se nos pierde de vista, que se nos va!---exclamé yo sin saber lo
que decía.

---Señora, si está anclado\ldots{} Podemos verlo con toda calma.

Nos acercamos a la Torre del Oro, junto a la cual estaba la nave
maravillosa. Tenía dos ruedas como las de un batán, resguardadas por
grandes cajones de madera pintados de blanco, con chimenea negra y alta
en cuyo centro estaba la máquina, toda grasienta y ahumada como una
cocina de hierro, y el resto no ofrecía nada de particular. De sus
entrañas negras salía una especie de aliento ardoroso y retumbante, cuyo
vaho causaba vértigos. De repente daba unos silbidos tan fuertes que era
preciso taparse los oídos. En verdad aquella máquina infundía miedo. Yo
no lo tuve porque no podía fijar en ella resueltamente la atención.

---¿Se atreve usted a entrar?---me dijo el Marqués.

Yo miré a todos lados y vi reaparecer a mi amor perdido, saliendo de
entre la muchedumbre, como el sol de entre las nubes.

---No señor, yo me mareo sólo de ver un barco---respondí a
Falfán.---Estoy satisfecha con admirar desde fuera esta hermosa
invención, y le doy a usted las gracias.

Yo hubiera dado no sé qué porque el vapor echase a andar hacia la
eternidad llevándose dentro al marqués de Falfán de los Godos.

---¡Oh!---exclamó él,---embarquémonos. Yo le garantizo a usted que no se
marea. Daremos un paseo hasta Aznalfarache. Vea usted cuántas personas
entran.

---Pues yo no me decido. Pero no se prive usted por mí del gusto de
embarcarse. Adentro, señor mío. Yo me voy a mi casa.

---¡Ah!, no consiento yo que usted vaya sola a su casa---dijo con una
galantería cruel que me asesinaba.---Yo la acompañaré.

---Gracias, gracias\ldots{} no necesito compañía.

---Es que yo no puedo permitir\ldots{}

De buena gana habría cogido al Marqués por el pescuezo como se coge a un
pollo destinado a la cazuela, y le hubiera estrangulado con mis propias
manos; ¡tal era mi rabia!

---Al menos---añadió,---ya que lo hemos visto por la popa, vamos a verlo
también por la proa.

Al decir esto el Marqués dirigió sus miradas hacia la Maestranza, y sus
ideas variaron de súbito.

---Vamos: por allí viene mi señora esposa---dijo señalando.---¿La ve
usted? Por último se ha atrevido a salir a paseo, aunque no está bien de
salud.

Miré y vi a la marquesa de Falfán que venía con otra señora. También
ellas, atraídas por la curiosidad, se dirigían hacia la Torre del Oro.

---Aguardemos aquí---me dijo el Marqués sonriendo.---Veremos si pasa sin
notar que estamos aquí.

Andrea y su amiga estaban ya cerca de nosotros, cuando Salvador pasó
junto a ellas, se detuvo, las saludó y continuó andando a su lado. Nos
reunimos los cinco.

---¿También tú vienes a ver el vapor?---exclamó Falfán riendo.---Ya te
dije que era una maravilla. Y usted, Sra. Dª María Antonia, ¿también
viene a ver el vaporcito? Y usted Salvador no quiere ser menos. El que
desee entrar que lo diga, y nos embarcaremos.

---¿Yo?\ldots---dijo la Marquesa después de saludarme.---Tengo miedo.
Dicen que revienta la caldera cuando menos se piensa.

---¿De modo que eso tiene una caldera, como las fábricas de
jabón?---preguntó D.ª María Antonia llevando a sus ojos el lente que
usaba.

---¿Entran ustedes, sí o no?---dijo el Marqués empeñado siempre en
reclutar gente.

---Yo no entraré---repuso la Marquesa con desdén:---me mareo sólo de ver
ese horrible aparato. Además, tengo que hacer.

---¿A dónde vas ahora?---preguntó Falfán de mal talante.

---A las tiendas de la calle de Francos. Ya sabes que necesito comprar
varias cosillas.

---Pero si no has paseado aún\ldots{}

---¿Que no? Sra. D.ª María Antonia, dice que no hemos paseado\ldots{} Si
hace más de hora y media que estamos aquí dando vueltas. Ya nos íbamos
cuando te vimos, y volví atrás para rogarte que nos acompañes.

---¡Yo!---indicó el Marqués con mucho disgusto.---Ya sabes que no me
agrada ir a tiendas.

---Y a mí no me gusta ir sola.

---D.ª María Antonia\ldots{}

---Es señora, y para ir a las tiendas conviene la compañía de un
caballero. Mira, hijito, no te apures por eso, Salvador nos acompañará.

---Con mil amores---dijo mi amigo inclinándose.---Tengo mucho honor en
ello.

Cuando allí mismo no abofeteé a mi amante, a la Marquesa, al Marqués, a
D.ª María Antonia y a mí misma, de seguro queda demostrado que soy una
oveja por lo humilde.

---Sí, amigo Monsalud---manifestó Falfán;---acompáñelas usted, se lo
suplico. Jenara y yo nos embarcaremos.

¡Se marcharon! ¡Ay!, no sé cómo lo escribo. Se marcharon sin que yo les
estrangulase. Dentro de mí había un volcán mal sofocado por mi disimulo.
El Marqués me hablaba sin que yo pudiese responderle, porque estaba
furiosamente absorta y embrutecida por el despecho que llenaba mi alma.

---Nos embarcaremos---me dijo Falfán relamiéndose como un gato a quien
ponen plato de su gusto.

---¡Ah!, señor Marqués---dije de improviso apoderándome de una idea
feliz.---Ahora me acuerdo de una cosa\ldots{} ¡qué memoria la mía!

---¿Qué, señora?

---Que yo también tengo que comprar algunas cosillas. ¿No es verdad,
Mariana?

---¿De modo que va usted\ldots?

---Sí señor, ahora mismo\ldots{} Son cosas que necesito esta misma
noche.

---¿Y hacia dónde piensa dirigirse usted?

---Hacia la calle de las Sierpes\ldots{} o la de Francos. Son las únicas
que conozco.

---Pues la acompañaré a usted.

Hizo señas a su cochero para que acercase el coche.

---Mi mujer---añadió,---se va a enfadar conmigo porque no quise
acompañarla y la acompaño a usted.

No hice caso de sus cumplidos ni de sus excusas.

---Vamos, vamos pronto---dije subiendo al coche.

Este nos dejó en la plaza de San Francisco. Nos dirigimos a las tiendas,
recorrimos varias calles; pero ¡ay!, estábamos dejados de la mano de
Dios. No les encontramos; no les vimos por ninguna parte.

En mi cerebro se fijaba con letras de fuego esta horrible pregunta: «¿a
dónde irían?»

Cuando el Marqués me dejó en mi casa ya avanzada la noche, yo tenía
calentura. Retireme a pensar y a recordar y a formar proyectos para el
día siguiente; pero mi cerebro ardía como una lámpara; no pude dormir;
hablaba a solas sin poder olvidar un solo momento el angustioso tema de
mi vida en aquellos días. Por último, mis nervios se aplacaron un tanto,
y me consolé pensando y hablando de este modo:

---¡Mañana, mañana no se me escapará!

\hypertarget{xxvii}{%
\chapter{XXVII}\label{xxvii}}

Al levantarme con la cabeza llena de brumas, pensé en la extraña ley de
las casualidades que a veces gobiernan la vida. En aquella época creía
yo aún en las casualidades, en la buena o mala suerte y en el destino,
fuerzas misteriosas que ciegamente, según mi modo de ver, causaban
nuestra felicidad o nuestra desgracia. Después han variado mucho mis
ideas y tengo poca fe en el dogma de las casualidades.

Mi cerebro estaba aquella mañana, como he dicho, cargado de neblinas.
Pero el día no podía haber amanecido más hermoso, y para ser 12 de Junio
en Andalucía, no era fuerte el calor. Sevilla sonreía convidando a las
dulces pláticas amorosas, a las divagaciones de la imaginación y a
exhalar con suspiros los aromas del alma que van desprendiéndose y
saliendo, ya gimiendo ya cantando entre vagas sensaciones que son a la
manera de una pena deliciosa.

Pero yo continuaba con mi idea fija y la contrariedad que me
atormentaba. A ratos deteníame a analizar aquel singular estado mío y me
asombraba de verme tan dominada por un vano capricho. Es verdad que yo
le amaba; pero ¿no había sabido consolarme honradamente de su ausencia
después de Benabarre? ¿Por qué en Sevilla ponía tanto empeño en tenerle
a mi lado? ¿Acaso no podía vivir sin él? Meditando en esto, me creía muy
capaz de prescindir de él en la totalidad de la vida; pero en aquel caso
mi corazón había soltado prendas, habíase fatigado mucho, había,
digámoslo así, adelantado imaginariamente gran parte de sus goces; de
modo que padecía horriblemente al verse desairado. Aquel suplicio de
Tántalo a que había estado sujeto, irritábale más, y ya se sabe que las
ambiciones más ardientes son las del corazón, y que en él residen los
caprichos y la terrible ley satánica que ordena desear más aquello que
más resueltamente nos es negado. Así se explica la indecorosa
persecución de un hombre en que yo, sin poder dominarme, estaba
empeñada.

Ordené a Mariana que se preparase para salir conmigo. Mientras yo me
peinaba y vestía, díjome que había oído hablar de la partida de Su
Majestad aquel mismo día y que Sevilla estaba muy alborotada. Poco me
interesaba este tema y le mandé callar; pero después me contó cosas muy
desagradables. En la noche anterior y por la mañana, dos diputados
residentes en la misma casa y que traían entre manos la conquista de mi
criada, le habían hecho con respecto a mí, indicaciones maliciosas.
Según me dijo, eran conocidas y comentadas mis relaciones con el
secretario del duque del Parque. ¡Maldita sociedad! Nada en ella puede
tenerse secreto. Es un sol que todo lo alumbra, y en vano intenta el
amor hallar bajo él un poco de sombra. A donde quiera que se esconda
vendrá a buscarle la impertinente claridad del mundo, de modo que por
mucho que os acurruquéis, a lo mejor os veis inundados por los rayos de
la intrusa linterna que va buscando faltas. El único remedio contra esto
es arrojar mucha, muchísima luz sobre las debilidades ajenas, para que
las propias resulten ligeramente oscurecidas. No sé por qué desde que
Mariana vino a mí con aquellos chismes me figuré que mi difamación
procedía de los labios de la marquesa de Falfán.---¡Ah, bribona!---dije
para mí,---si yo hablara\ldots{}

Las hablillas no me acobardaron. Siendo culpable, hice lo que
corresponde a la inocencia: despreciar las murmuraciones.

Cuando manifesté a Mariana que pensaba ir a buscarle a su propia casa,
hízome algunas observaciones que me desagradaron, sin que por ellas
desistiera yo de mi propósito.

---¿No averiguaste ayer la casa donde vive?

---Sí señora, en la calle del Oeste. Pero usted no repara que en la
misma casa viven también otras personas de Madrid que conocen a la
señora\ldots{}

Ninguna consideración me detenía. Escribí una carta para dejarla en la
casa si no le encontraba, y salimos. Mariana conocía bien Sevilla, y
pronto me llevó a la calle del Oeste, que está hacia la Alameda Vieja
junto a la Inquisición. Salvador no estaba. Dejé mi carta, y corrimos a
casa porque al punto sospeché que mientras yo le buscaba en su vivienda
me buscaba él en la mía. Así me lo decía el corazón impaciente.

---Me aguardará de seguro---pensé.---Ahora, ahora sí que no se me
escapa.

En mi casa no había nadie; pero sí una esquela. Salvador estuvo a
visitarme durante mi ausencia, y no pudiendo esperar, a causa de sus
muchas ocupaciones, dejome también una carta en que así lo manifestaba,
añadiendo entre expresiones cariñosas que por la tarde a las cuatro en
punto me aguardaba en la catedral. Después de indicar la conveniencia de
no volver a mi casa, me suplicaba que no faltase a la cita en la gran
basílica y en su hermoso patio de los naranjos. Tenía preparado un coche
en la puerta de Jerez para irnos de paseo hacia Tablada.

---¡Gracias a Dios!---exclamé.---Esta tarde\ldots{}

Tomando mis precauciones para que nadie me importunase y poder estar
completamente libre en la hora de la cita, consagré algunas al descanso.
Pero la ocasión no era la más a propósito, y a las tres ya estaba yo en
la catedral. Era la hora del coro y los canónigos entraban uno tras otro
por la puerta del Perdón. Algunos se detenían a echar un parrafito en el
patio de los naranjos paseando junto al púlpito de San Vicente Ferrer.

Al verme dentro de la iglesia, la mayor que yo había visto, sentí una
violenta invasión de ideas religiosas en mi espíritu. ¡Maravilloso
efecto del arte que consigue lo que no es dado alcanzar a veces ni aun a
la misma religión! Yo miraba aquel recinto grandioso que me parecía una
representación del universo mundo. Aquel alto firmamento de piedra, así
como las hacinadas palmas que lo sustentan y el eminente tabernáculo,
que es cual una escala de santos que sube hasta Dios, dilataban mi alma
haciéndola divagar por la esfera infinita. La suave oscuridad del templo
hace que brillen más las ventanas, cuyas vidrieras parecen un fantástico
muro de piedras preciosas. Las vagas manchas luminosas de azul y rosa
que las ventanas arrojan sobre el suelo se me figuraban huellas de
ángeles que habían huido al sentir nuestros pasos.

Mi mente se sentía abrumada de ideas. Senteme en un banco porque sentía
la necesidad de meditar. Delante de mis pies, a manera de alfombra de
luces, se extendía la transparencia de una ventana. Alzando los ojos
veía las grandiosas bóvedas. Zumbaba en mis oídos el grave canto del
coro, y a intervalos una chorretada de órgano, cuyas maravillosas
armonías me hacían estremecer de emoción, poniendo mis nervios como
alambres. A poca distancia de mí, a la izquierda, estaba la capilla de
San Antonio toda llena de luces por ser 12 de Junio, víspera del santo,
y de hermosos búcaros con azucenas y rosas. Volviendo ligeramente la
cabeza veía el cuadro de Murillo y su espléndido altar.

Yo pensaba en cosas religiosas; pero mi egoísmo las asociaba al amoroso
afán que me poseía. Pensaba en la santidad de la unión sancionada por la
Iglesia y de los lazos matrimoniales cuando son acertados. Consideraba
lo feliz que hubiera sido yo no equivocándome como equivoqué, en la
elección de marido. También pasó por mi mente, aunque con gran rapidez,
el recuerdo de la infeliz joven a quien con mis engaños precipité en los
azares de un viaje absurdo; pero esto duró poco y además me apresuré a
sofocar tan triste memoria, dirigiendo el pensamiento a otra cosa.

La imagen que tan cerca estaba atrajo mi atención. Aquel santo tan
bueno, tan humilde, tan buen compañero y amigo de los pobres es, según
dicen, el abogado de los amores y de los objetos perdidos. Ocurriome
rezarle y le recé con fervor de labios y aun de corazón, porque en aquel
instante me sentía piadosa. No sólo le pedí como enamorada, sino como
quien busca y no encuentra cosas de gran valor; y mientras más le
rezaba, más me sentía encendida en devoción y llena de esperanza.
Concluí adquiriendo la seguridad de que mi afán se calmaría aquella
misma tarde; y juzgando que mi entrada en la catedral a causa de la cita
era obra providencial, mi alma se alivió, y aquella tensión dolorosa en
que estaba fue cesando poco a poco.

¿Cómo no esperar si aquel santo era tan bueno, tan complaciente que
mereció siempre el amor y la veneración de todos los enamorados? No pude
estar allí todo el tiempo que habría deseado porque me causaba vértigo
el olor de las azucenas y también porque la hora de la cita se acercaba.
Cuando salí al patio y en el momento de pasar bajo el cocodrilo que
simboliza la prudencia, la alta campana de la Giralda dio las cuatro.

No habíamos llegado al púlpito de San Vicente Ferrer, cuando Mariana y
yo nos miramos aterradas. Sentíamos un ruido semejante al de las olas
del mar. Al mismo tiempo mucha gente entraba corriendo en el patio de
los naranjos.

---¡Revolución, señora, revolución!---gritó Mariana temblando.---No
salgamos.

La curiosidad, venciendo el miedo, me llevó con más presteza hacia la
puerta. Vi regular gentío que llenaba todo el sitio llamado Gradas de la
Catedral, y parecía extenderse por delante del palacio arzobispal y la
Lonja hasta el Alcázar. Pero la actitud de la muchedumbre era pacífica y
más parecía de curiosos que de alborotadores. Al punto comprendí que la
salida de la Corte motivaba tal reunión de gente, y se calmaron mis
súbitas inquietudes. Esperaba ver de un momento a otro a la persona por
quien había ido a la catedral, y mis ojos la buscaron entre la multitud.

---Aguardaremos un poco---pensé dando un suspiro.

La muchedumbre se agitó de repente, murmurando. Por entre ella trataba
de abrirse paso un regimiento de caballería que apareció por la calle de
Génova. Entrad la mano en un vaso lleno de agua y esta se desbordará;
introducid un regimiento de caballería en una calle llena de curiosos y
veréis lo que pasa. Por la puerta del Perdón penetró un chorro que
salpicaba dicharachos y apóstrofes andaluces contra la tropa, y tal era
su ímpetu que los que allí estábamos tuvimos que retroceder hasta el
centro del patio. Entonces un sacristán y un hombre forzudo y corpulento
de esos que desempeñan en toda iglesia las bajas funciones del trasporte
de altares, facistoles o bancos, o las altísimas de tocar las campanas y
recorrer el tejado cuando hay goteras, se acercaron a la puerta y
después de arrojar fuera toda la gente que pudieron, cerraron con
estruendo las pesadas maderas. Corrí a protestar contra un encierro que
me parecía muy importuno; mas el sacristán alzando el dedo, arqueando
las cejas y ahuecando la voz como si estuviera en el púlpito, dijo
lacónicamente:

---De orden del señor Deán.

\hypertarget{xxviii}{%
\chapter{XXVIII}\label{xxviii}}

Mucho me irritó la orden del señor Deán, que sin duda no esperaba a una
persona amada, y entré en la iglesia consolándome de aquel percance con
la idea de que en edificio tan vasto no faltarían puertas por donde
salir. Pasamos al otro lado; pero en la puerta que da a la plaza de la
Lonja, otro ratón de iglesia me salió al encuentro después de echar los
pesados cerrojos, y también me dijo:

---De orden del señor Deán.

---¡Malditos sean todos los deanes!---exclamé para mí, dirigiéndome a la
puerta que da a la fachada. Allí, un viejo con gafas, sotana y
sobrepelliz, se restregaba las manos gruñendo estas palabras:

---Ahora, ahora va a ser ella. Señores liberales, nos veremos las caras.

Yo fui derecha a levantar el picaporte; pero también aquella puerta
estaba cerrada y el sacristán viejo al ver mi cólera que no podía
contener, alzó los hombros disculpándose con la orden de la primera
autoridad capitular. El de las gafas añadió:

---Hasta que no pase la gresca no se abrirán las puertas.

---¿Qué gresca?

---La que han armado con la salida del Rey loco. Mi opinión, señora, es
que ahora va a ser ella, porque hay un complot que no lo saben más de
cuatro.

Volvió a restregarse las manos fuertemente, guiñando un ojo.

---¿Y a qué hora sale Su Majestad?

---A las seis, según dicen; pero antes ha de correr la sangre por las
calles de Sevilla como cuando la inundación de hace veinte años, la cual
fue tan atroz, señora, que por poco fondean los barcos dentro de la
catedral.

---¡De modo que estaré encerrada aquí hasta las seis!---exclamé llena de
furor.---Esto no se puede sufrir, es un abuso, un escándalo. Me quejaré
a las autoridades, al Rey.

---El Rey está loco---dijo el viejo con horrible ironía.

---Al Gobierno; me quejaré al Arzobispo. O me dejan salir o gritaré
dentro de la iglesia, reclamando mi derecho.

Discurrí con agitación indecible por la iglesia, nave arriba, nave
abajo, saliendo de una capilla y entrando en otra, pasando del patio al
templo y del templo al patio. Miraba a los negros muros buscando un
resquicio por donde evadirme, y enfurecida contra el autor de orden tan
inicua, me preguntaba para qué existían deanes en el mundo.

Los canónigos dejaban el coro y se reunían en su camarín, marchando de
dos en dos o de tres en tres, charlando sobre los graves sucesos. Los
sochantres y el fagotista se dirigían piporro en mano a la capilla de
música, y los inocentes y graciosos niños de coro, al ser puestos en
libertad iban saltando, con gorjeos y risas, a jugar a la sombra de los
naranjos.

Varias veces en las repetidas vueltas que di por toda la iglesia, pasé
por la capilla de San Antonio. Sin que pueda decir que me dominaban
sentimientos de irreverencia, ello es que mi compungida devoción al
santo había desaparecido. No le miré con aversión, pero sí con cierto
enojo respetuoso, y en mi interior le decía:

---¿Es esto lo que yo tenía derecho a esperar? ¿Qué modo de tratar a los
fieles es este?

Mi egoísmo había llegado al horrible extremo de pedir cuenta a la
Divinidad de los desaires que me hacía. Irritábame contra el Cielo
porque no satisfacía mis caprichos.

Pero, ¡maldita hora!, quien a mí me irritaba verdaderamente era el Deán
tirano que mandaba encerrar a la gente porque se le antojaba. Desde que
le vi salir del coro en compañía del Arcediano, moviéndose muy
lentamente a causa del peso de su descomunal panza, le tuve por un
realistón furibundo, sin que por esto me fuese menos antipático. ¿Por
qué habían cerrado las puertas? Por poner el sagrado recinto a salvo de
una invasión plebeya, e impedir que el bullicio de los vivas y mueras
turbase la santa paz de la casa de Dios. A pesar de su celo no pudo el
señor Deán conseguirlo, y desde el patio oíamos claramente los gritos de
la muchedumbre y el paso de la caballería. La Giralda cantó las cinco,
cantó las seis, y aquella deplorable situación no cambiaba ni las
puertas se abrían, ni se desvanecía el rumor del pueblo. Yo creo que si
aquello se prolonga demasiado, me atrevo a decir dos palabras al buen
canónigo encerrador. Por fin no era yo sola la impaciente: otras muchas
personas, encerradas como yo, se quejaban igualmente, y todos nos
dirigíamos en alarmante grupo al sacristán; pero sin conseguir nada.

---Cuando Su Majestad haya salido de Sevilla---nos respondía,---o se
arma la de San Quintín, o todo quedará tranquilo.

Por fin, después de las siete, la puerta del Perdón se abrió y vimos las
Gradas y la gente que iba y venía sin tumulto. Yo me arrojé a la calle
como se arrojaría en el agua aquel cuyos vestidos ardieran. Miraba a un
lado y otro; me comía con los ojos a cuantos pasaban; caminé
apresuradamente hacia la Lonja y hasta el Alcázar; mi cabeza se movía
sin cesar, dirigiendo la vista a todo semblante humano. ¡Afán
inútil!\ldots{} Yo buscaba y rebuscaba, y mi hombre no aparecía en
ninguna parte\ldots{} Ya se ve\ldots{} ¡Las siete de la tarde! Se
cansaría de aguardarme\ldots{} tendría que hacer\ldots{}

Volví de nuevo a la catedral, recorrila toda, salí, di la vuelta por la
Lonja; pero ¡ay!, si diera la vuelta a toda la tierra, creo que tampoco
le encontrara; ¡tal era la horrible insistencia de mi desgracia! Y sin
embargo, hasta en las baldosas del piso, en el aire y en el sonido,
hallaba no sé qué indicio misterioso de que él me había aguardado allí
largas horas. Esto era para morir.

Después de mucho correr, senteme en un banco de piedra junto a la Lonja.
Tanto me enfadaba la gente que veía regresar del Alcázar y de la puerta
de San Fernando, que si las llamas de furor que abrasaban mi pecho
fueran materiales, de buena gana hubiera vomitado fuego sobre los que
pasaban ante mí. Venían de ver partir al Rey loco. Muchos se lamentaban
de que se tratase de tal suerte al Soberano de Castilla. ¡Menguados!,
¿por qué no tomaban las armas? Sí, ¿por qué no las tomaban? Me habría
gustado ver a todos los habitantes de Sevilla destrozándose unos a
otros.

La Giralda cantó otra hora, no sé cuál, y entonces me decidí a tomar
nueva resolución.

---Vamos a su casa---dije a Mariana.

---Es de noche, señora---repuso.

La infeliz no quería alejarse mucho de la casa. Pero no le contesté y
nos pusimos en camino para la calle del Oeste.

---¿Y si no está?---indicó mi criada.---Porque es muy posible que con
estas cosas\ldots{}

---¿Qué cosas?

---Estas revoluciones, señora.

---Si no hay nada.

---Pues\ldots{} como se han llevado al Rey después de volverle
loco\ldots{} En el patio de la catedral decía uno que tendremos
revolución mañana, cuando se marche el Gobierno; porque el Gobierno se
marchará.

---Déjalo ir: no nos hace falta. Date prisa.

---Pues yo creo que nos llevaremos otro chasco.

---Si no está en su casa le esperaré.

---¿Y si no vuelve hasta muy tarde?

---¡Hasta muy tarde le esperaré!

---¿Y si no vuelve hasta mañana?

---Hasta mañana le esperaré. No me muevo de su casa hasta que le vea.
Ahora, ahora sí que no se me escapa, ¿concibes tú que se me pueda
escapar?

\hypertarget{xxix}{%
\chapter{XXIX}\label{xxix}}

Al decir esto, mi corazón, oprimido por tantos desengaños, se ensanchaba
llenándose otra vez de esperanza, de ese don del cielo que jamás se
agota y que a nadie puede faltar.

---Pues no veo yo muy tranquila esta noche la ciudad de Sevilla---indicó
Mariana.---Si, como dicen, se ha marchado toda la tropa, puede que nos
despertemos mañana en un charco de sangre.

Echeme a reír, burlándome de sus ridículos temores, y seguimos avanzando
con bastante presteza hacia la calle del Oeste. Detúveme antes de llamar
en su casa, para que un breve descanso disimulara mi sofocación y se
amortiguasen las llamaradas de mis mejillas.

---Sentémonos---dije a Mariana,---al amparo de este árbol. Ahora no hay
gran prisa. Ya le tengo cogido. Estoy tranquila. Él ha de venir a su
casa. Ahora, ahora sí que le tengo en mi mano.

Cuando llamamos en la reja que daba entrada al patio, una mujer nos dijo
que el señor Monsalud no estaba en casa.

---Pues tengo que hablarle precisamente esta noche y le esperaré---dije
resueltamente.

Yo no reparaba en conveniencia alguna social. En el estado de mi
espíritu, nada tenía fuerza para contenerme. Importábame ya muy poco que
me vieran, que me conocieran, que me señalasen con el dedo, ni que el
vulgo suspicaz y murmurador me hiciera objeto de burlas y comentarios
deshonrosos.

Al principio vacilaba en dejarme entrar la mujer que me abrió la puerta;
pero tanto insté y con tan arrogante autoridad me expresaba, que al fin
me llevó a una sala baja. Allí estaba un viejecillo, que a la débil
claridad de un velón de cobre, arreglaba baúles y cajas, poniendo en
ellos libros, ropa y papeles. Era un tal Bartolomé Canencia. Él no debía
de conocerme; pero se apresuró a saludarme con extremadas urbanidades.
Cual si comprendiera las ansias que yo padecía aquella noche, me dijo:

---No está en casa, ni puedo asegurar que venga pronto; pero sí que
vendrá. Necesitamos arreglar todo para nuestra partida.

¿Cuándo?

---Mañana. Nos vamos con el Gobierno. ¿Quién se atreverá a quedarse aquí
después que marchen los ministros? Esto es un volcán realista. En cuanto
desaparezca el Gobierno que obstruye el cráter, se agitará con fuego y
vapores vomitando horrores. ¡Pobre Sevilla!, no ha querido oír mis
consejos, los consejos de la experiencia, señora, y hela aquí en poder
del realismo más brutal. Este pueblo, tan célebre por su riqueza y por
su gracia como por sus procesiones, está infestado de curas; y aquí los
curas son ricos. No hay más que decir.

Yo me fastidiaba esta conversación, y así con la mayor habilidad la
desvié de la política haciéndola recaer sobre mi objeto. Canencia
contestó a mis preguntas de una manera categórica.

---Esta tarde salimos juntos---me dijo.---Él se quedó en las Gradas de
la Catedral, donde tenía una cita, y yo seguí hacia el Alcázar para
asistir a la salida de Su Majestad\ldots{} Luego nos encontramos de
nuevo a eso de las siete; parecía disgustado, sin duda porque la cita no
pudo verificarse. Entramos en casa y después él salió para ver a
Calatrava. Díjome que volvería a arreglar su equipaje, y aquí me tiene
usted arreglando el mío, señora, para lo que se le ofrezca mandar. De
modo que si usted desea algo en Cádiz, puede dar sus órdenes con toda
franqueza.

---Yo también pienso ir a Cádiz---repuse.

---¡Usted también! Bueno es que vayan todos---dijo con ironía
maliciosa,---para que se haga con toda solemnidad el entierro de la
Constitución. Allí nació, señora, y allí le pondremos la mortaja; que
todo lo que nace ha de perecer\ldots{} Si se hubieran seguido mis
consejos, señora\ldots; pero los hombres se han dejado enloquecer por la
ambición y la vanidad. Ya no existen aquellos repúblicos austeros,
aquellos filósofos incorruptibles, aquellos sectarios de la honradez más
estricta y de la sabiduría ateniense, hombres que con un pedazo de pan,
un vaso de agua y un buen libro se pasaban la mayor parte de la vida.
Ahora todo es comer a dos carrillos, pedir destinos, figurar\ldots{} en
una palabra, señora, ya no hay virtudes cívicas.

---¿Y es seguro que el Gobierno marcha mañana?---le pregunte para
desviarle de su fastidiosa disertación.

---Segurísimo. No puede ser de otra manera.

---¿Por tierra?

---Por agua, señora. Los ministros y diputados marchan en el vapor.

---¿Y usted y Salvador van también en el vapor?

---Iremos donde podamos, señora, aunque sea en globo por los aires.

Él siguió arreglando sus maletas y yo me abrumé en mis pensamientos. En
la sala había un reloj de cucú con su impertinente pájaro, de esos que
asoman al dar la hora y nos hacen tantas cortesías como campanadas tiene
aquella. Nunca he visto un animalejo que más me enfadase, y cada vez que
aparecía y me saludaba mirándome con sus ojillos negros y cantando el
cucú, sentía ganas de retorcerle el pescuezo para que no me hiciera más
cortesías. El pájaro cantó las nueve y las diez y las once, y con su
insolente movimiento y su desagradable sonido parecía decirme:---¿Qué
tal, señora, se aburre Vd. mucho?

Todo el que ha esperado comprenderá mi agonía. Aquel resbalar del
tiempo, aquella veloz corrida de los minutos que pasan de nuestra frente
a nuestra espalda, amontonándose atrás el tiempo que estaba delante, es
para enloquecer a cualquiera. Cuando no hay un reloj que lleve la cuenta
exacta de la cantidad de esperanza que se desvanece y de la paciencia
que se gasta grano a grano, menos mal; pero cuando hay reloj y este
reloj tiene un pájaro que hace reverencias cada sesenta minutos y dice
cucú, no hay espíritu bastante fuerte para sobreponerse a la pena. Ya
cerca de las doce me decía yo: «¿Si no vendrá?»

Habiendo manifestado mis dudas al viejo Canencia que parecía algo
molesto por la duración de mi visita, me dijo:

---Puede que venga y puede que no venga. Seguramente estará ahora en el
café del Turco o en casa del duque del Parque. Ya es medianoche. Dentro
de unas cuantas horas será de día y\ldots{} ¡en marcha todo el mundo
para Cádiz!

Mariana bostezaba, siendo imitada por Canencia. Yo me sostenía
intrépida, sin sueño ni cansancio, resuelta a estar un año en aquel
sitio, si un año tardaba en venir mi hombre.

---De todas maneras---dije a Canencia,---si se marcha mañana, ha de
venir a arreglar su equipaje.

---Es muy posible, señora---me contestó secamente.---En caso de que
quiera Vd. retirarse, puede con toda confianza dejar el recado verbal
que guste. Yo se lo trasmitiré puntualmente y con la fidelidad de un
verdadero amigo.

---Gracias.

---Le diré que ha estado aquí\ldots{} Aunque usted no me ha dicho su
nombre, yo creo conocer a la persona con quien tengo el honor de hablar,
por haberla visto en Madrid algunas veces\ldots{} ¿No es usted la señora
marquesa de Falfán?

Esta pregunta me hizo estremecer en mi interior, como si un rayo pasara
por mí. Pero dominándome con soberano esfuerzo, repuse gravemente y con
afectada vergüenza:

---Sí señor, soy la marquesa de Falfán. Fiada en la discreción de usted,
me he aventurado a esperar aquí en hora tan impropia.

---Señora, yo soy un sepulcro, y además un amigo fiel de ese excelente
joven, y como le debo muchos beneficios, a la amistad se une la
gratitud. Puede usted con toda libertad confiarme lo que quiera. Es muy
posible que él no pueda verla a usted esta noche. Estará muy ocupado y
sin duda el viaje de mañana trastorna sus planes, porque, si no recuerdo
mal, hoy me dijo que pensaba despedirse de usted, por la noche, en casa
de D.ª María Antonia.

Al oír esto me quedé como mármol y enseguida me llené de ascuas.
Desplegué los labios para preguntar: «¿dónde vive esa D.ª María
Antonia?» pero me contuve a tiempo comprendiendo la gran torpeza que iba
a cometer. Evocando toda mi destreza de cómica, dije:

---Así pensábamos; pero no ha podido ser.

El infame pájaro se asomó a su nicho y burlándose de mí cantó la una. Yo
me ahogaba, porque a mis primeras fatigas se unía desde que habló aquel
hombre, la inmensa sofocación de un despecho volcánico de los celos que
me mataban. En mi cerebro se encajaba una corona de brasas
resplandecientes y mi corazón chorreaba sangre, herido por mil púas
venenosas. Mi afán, mi deseo más vivo era morder a alguien.

Esperé más. Canencia seguía bostezando y Mariana dormitaba. Yo sentía en
mis oídos un zumbido extraño, el zumbido del silencio nocturno que es
como un eco de mares lejanos, y deshaciéndome esperaba. Habría dado mi
vida entera por verle entrar, por poder hablarle a solas un momento,
arrojando sobre él las palabras, la furia, la hiel que se desbordaban en
mí. A ratos balbucía terribles injurias que siendo tan infames, a mí me
parecían rosas.

El vil pajarraco volvió a chancearse conmigo y haciendo la reverencia
más pronunciada y el canto más fuerte, anunció las dos.

---¡Las dos!\ldots{} ¡pronto será de día!---exclamé.

---Fijamente no viene ya, señora. Es que se embarca con los
diputados---dijo Canencia dando a entender con sus bostezos que de buena
gana dormiría un rato.

---¿Y a qué hora se embarcan los diputados?

---Al rayar el día: así se dijo anoche en el salón del Congreso, cuando
se levantó la sesión que ha durado treinta y tres horas.

Estuve largo rato dudando lo que debía hacer. Delante de mi pensamiento
daba vueltas un círculo de fuego que alternativamente, en su lenta
rotación, mostrábame dos preguntas; primera: ¿Y si viene después que yo
me vaya? Segunda: ¿Y si se embarca en el muelle mientras yo estoy aquí?

Yo veía pasar una pregunta, después otra. La segunda sustituía a la
primera y la primera a la segunda en órbita infinita. Ambas tenían igual
claridad, ambas me deslumbraban y me enloquecían de la misma manera. Yo,
que por lo general me decido pronto, entonces dudaba. Cuando la voluntad
se iba inclinando de un lado el pensamiento llamábame del otro, y así
contrabalanceados los dos, ponían a mi alma en estado de terrible
ansiedad. Largo rato permanecí en esta dolorosa incertidumbre. Los
minutos volaban, y acercándose aquel en que era preciso resolver
definitivamente, el silencio mismo llegó a impresionar mi cerebro como
un bramido intolerable, formado por mil voces. Oía el latir de mi
corazón como se oye un secreto que nos dicen al oído; mi sangre ardía, y
por fin, aquella misma palpitación de mi alborotado seno fue como una
voz que hablaba diciéndome: «anda, anda.»

El pájaro, riendo como un demonio burlón, me saludó tres veces con su
cortesía y su infernal cucú. Eran las tres.

---Va a ser de día---dijo Canencia, dejando caer sobre el pecho su
cabeza venerable.

Levanteme. Estaba decidida. Pareciome que D. Bartolomé, al verme
dispuesta a partir, vio el cielo abierto. Despedime de él bruscamente y
salimos.

---¿A dónde vamos, señora?---me dijo Mariana.---¿No es hora de
retirarnos ya a descansar?

---Todavía no.

---¡Señora, señora, por Dios!\ldots{} Está amaneciendo. No hemos cenado,
no hemos dormido\ldots{}

---Calla, imbécil---le dije clavando mis dedos en su brazo.---¡Calla, o
te ahogo!

\hypertarget{xxx}{%
\chapter{XXX}\label{xxx}}

Amanecía, y multitud de hombres de mal aspecto vagaban por la calle.
Veíanse gitanos desarrapados, y muchos guapos de la Macarena y de
Triana. Mi criada tuvo miedo; pero yo no. Repetidas veces nos vimos
obligadas a variar de rumbo para evitar el encuentro de algunos grupos
en que se oía el ronco estruendo de ¡vivan las caenas!, ¡muera la
nación!

Llegamos por fin al río. Ya el día había aclarado bastante, y desde la
puerta de Triana vimos la chimenea del vapor que despedía humo.

---Si esos barcos de nueva invención humean al andar---dije,---el vapor
se marcha ya.

Desde la puerta de Triana a la Torre del Oro se extendía un cordón de
soldados de artillería. En la puerta de Jerez había cañones. Nada de
esto me arredraba, porque mi exaltación me infundía grandes alientos, y
hablando al oficial de artillería logré pasar hasta la orilla, donde
algunas tablas sostenidas sobre pilotes servían de muelle. El vapor
bufaba como animal impaciente que quiere romper sus ligaduras y huir.
Multitud de personas se dirigían al embarcadero. Reconocí a
Canga-Argüelles, a Calatrava, a Beltrán de Lis, a Salvato, a Galiano y a
otros muchos que no eran diputados.

---Él se irá también---pensé.---Vendrá aquí de seguro\ldots{} Pero no,
no creo que se me pueda escapar.

Una idea grandiosa cruzó por mi mente, una de esas ideas napoleónicas
que yo tengo en momentos de gravedad suma. Ocurriome embarcarme también
en el vapor, si le veía partir. No tenía equipaje; ¿pero qué me
importaba? Mariana se quedaría para llevarlo después.

Acerqueme a Calatrava, que se asombró mucho de verme.

---Quiero un puesto en el vapor---le dije.

---¿También usted se marcha\ldots? ¿De modo que\ldots?

---Temo ser perseguida. Estoy muerta de miedo desde ayer. Me han
amenazado con anónimos atroces.

---¿Ha preparado usted su equipaje?

---He preparado lo más preciso: el viaje es corto. Mi criada se queda
para arreglar lo que dejo aquí.

---También nosotros dejamos nuestros equipajes porque no caben en el
vapor. Irán en aquella goleta.

---¿Me hace usted un sitio, sí o no?

---¿Un sitio? Sí señora. Dejando el equipaje\ldots{} El Gobierno ha
fletado el buque. Puede usted venir.

Esto se llama proceder pronto y con energía\ldots{} Pero observé a todos
los que llegaban, y no le vi. A cada instante creía verle aparecer.

---No puede tardar---dije, después que di mis órdenes a Mariana.---Ahora
sí que es mío.

Mariana hacía objeciones muy juiciosas; pero yo a nada atendía. Estaba
ciega, loca.

---¿Y si no se embarca?---me dijo mi criada.---Todavía no ha
venido\ldots{}

---Pero ha de venir\ldots{} A ver si está por ahí el duque del Parque.

Miramos las dos en todos los grupos y no vimos al Duque.

---¿El señor duque del Parque no va a Cádiz?---pregunté a Salvato.

---El señor Duque no se ha atrevido a votar el destronamiento.

---¿Y qué?

---Que los que no votaron no se creen en peligro y seguirán en Sevilla.

---De modo que Su Excelencia\ldots{}

---No tengo noticia de que se embarque con nosotros.

---Venga usted---me dijo Calatrava alargándome la mano para llevarme a
la cubierta del buque.

---Entre usted, amigo, entre usted, que aún tengo que decir algo a mi
criada.

---Parece que vacila usted\ldots{}

---En efecto\ldots{} sí\ldots{} no estoy decidida aún.

No, no podía entrar en aquel horrible bajel que iba a partir, silbando y
espumarajeando, sin llevar al que turbaba mi vida. Yo les vi entrar uno
tras otro, les conté; ni uno solo escapó a mi observación, y ¡él no
estaba! ¡Siempre ausente, siempre lejos de mí, siempre en dirección
diametralmente opuesta a la dirección de mis ideas y de mi apasionada
voluntad! Esto era para enloquecer completamente, y digo completamente,
porque yo estaba ya bastante loca. Mi desvarío insensato aumentaba como
la fiebre galopante del enfermo solicitado por la muerte.

Se embarcaron ¡ay!, vi al horrendo vapor separarse del muelle, vi
moverse las paletas de sus ruedas, machacando y rizando el agua, le oí
silbar y mugir echando humo, hasta que emprendió su marcha majestuosa
río abajo.

No yendo él, no podía causarme aflicción quedarme en tierra. Él estaba
también en Sevilla.

---Ahora---dije,---ahora no es posible que le pierda otra vez. Si tengo
actividad e ingenio, pronto saldré de esta angustiosa situación.

No quise detenerme como el vulgo que se extasiaba contemplando el humo
del vapor que conducía hacia el postrer rincón de España el último resto
del liberalismo. Como aquel humo en los aires, así se desvanecía en el
tiempo la Constitución\ldots{} Pero en mi mente no podían fijarse ni por
un instante estas ideas.

Me era forzoso pensar en otras cosas y en la realidad de mi ya
insoportable desdicha. ¿A dónde debía ir? En los primeros momentos
después del embarque no pude determinarlo, y vagué breve rato por la
ribera, hasta que me obligaron a huir los excesos de la salvaje
muchedumbre, que se precipitó sobre los equipajes de los diputados,
apoderándose de ellos y saqueándolos en presencia de la poca tropa que
había quedado en el muelle.

Al mismo tiempo sentí el clamor de las campanas echadas a vuelo en señal
de que Sevilla había dejado de pertenecer al Gobierno constitucional, y
en cuerpo y alma pertenecía ya al absolutismo. ¡Cambio tan rápido como
espantoso! El pronunciamiento se hizo entre berridos salvajes, en medio
del saqueo y del escándalo, al grito de ¡muera la Nación! La verdad es
que los alborotadores hacían poco daño a las personas; pero sí robaban
cuanto podían. Al entrar por la puerta de Jerez, procuré apartarme lo
más posible de la turbulenta oleada que marchaba hacia el corazón de
Sevilla, con objeto, según oí, de destrozar el salón de sesiones y el
café del Turco, donde se reunían los patriotas.

Lejos de desmayar yo con las muchas contrariedades, el insomnio y el
continuo movimiento, parecía que la misma fatiga me daba prodigiosos
alientos. No sentía el más ligero cansancio, y mi cerebro, como una
llama cada vez más viva, hallábase en ese maravilloso estado de
actividad que es para los poetas, para los criminales y para los que se
ven en peligro la rápida inspiración del momento. Yo sentía en mí un
estro grandioso, avivado por mis contrariadas pasiones, mi rencor y mi
despecho. Tenía la penetrante vista del genio y había llegado a ese
momento sublime en que los más profundos secretos de nuestro destino se
nos muestran con claridad espantosa. Mi pensamiento, como la aguja
magnética de una brújula, señalaba con insistencia la casa del marqués
de Falfán.

---¡Oh, allí, allí\ldots{} he de encontrar la solución de este horrible
problema!

\hypertarget{xxxi}{%
\chapter{XXXI}\label{xxxi}}

Y corriendo hacia la casa, soñaba no ya con las delicias de un encuentro
feliz y de una amable reconciliación, sino con proporcionar a mi alma el
inefable, el celestial, el infinito regocijo de un escándalo, de una
escena, de una de esas venganzas de mujer que son la \emph{Ilíada} del
corazón femenino. No sé si me equivocaré juzgando por mí de todas las
mujeres; pero pienso firmemente que ninguna, por muy tímida que sea,
deja de sentir en momentos dados, y cuando se discuten asuntos del
corazón, el poderoso instinto de la majeza. La maja, digan lo que
quieran, no es más que lo femenino puro. De mí puedo asegurar que en
aquel instante me sentía verdulera.

---Tengo la seguridad---decía,---de que le encontraré allí. El corazón
me lo dice\ldots{} Es precisamente lo que necesito; es la satisfacción
más preciosa y agradable de mi inmenso afán, el desahogo de mi pecho,
semejante a un volcán sin cráter, el consuelo de todas mis penas.
Hablaré, gritaré, vomitaré injurias, ¿qué digo injurias?, verdades. Diré
todo lo que sé; abriré los ojos de un marido crédulo y bonachón;
arrancaré la máscara a una hipócrita; confundiré a un ingrato\ldots{} en
suma, estaré en mi elemento\ldots{} ¡¡Ahora, Santo Dios de las
venganzas, ahora sí que no se me puede escapar!!

Al dirigirme a la plaza de la Magdalena, donde vivía el Marqués, vi a
dos o tres patriotas que eran llevados presos por el pueblo con una
cuerda al cuello. ¡Pobre gente! Entre ellos vi a Canencia, que me
dirigió al pasar una mirada suplicante; pero no hice caso y seguí. Casi
arrastrando a Mariana que apenas podía seguirme de puro cansada y
soñolienta, llegué a casa de Falfán.

En el patio encontré al Marqués, que al punto que me vio asombrose mucho
de la alteración de mi semblante, creyendo que ocurría algún grave
accidente.

---Señora---me dijo ofreciéndome una silla,---no extraño que esa gente
mal educada\ldots{} Se están cometiendo toda clase de excesos en la
desgraciada Sevilla.

---No es eso, no---repuse.---Si no me ha pasado nada.

---Señora, su rostro de usted me indica gran desasosiego y agitación.

---Es verdad---dije,---pero\ldots{}

---Está usted muy intranquila.

---Intranquila no, estoy furiosa.

Después de decir esto y de romper en seis pedazos mi abanico, que ya lo
estaba en cuatro, procuré tomar una actitud aparentemente serena, pues
el caso requería en mí la grave majestad del que condena, no la
atolondrada cólera y pueril turbación del condenado.

---¿Y por qué está usted furiosa?---me preguntó el Marqués,
confundido.---¿En qué puedo servir a usted?

---¡Yo sé que está aquí!!\ldots---dije mirando al Marqués de un modo que
le aterró.

---¿Quién?

---¡Oh!, ¿quién?\ldots{} será preciso que yo hable, que lo diga
todo\ldots{}

---Señora, no comprendo una palabra.

---Llame usted a la señora Marquesa, y quizás ella me comprenda---repuse
con amargo sarcasmo.

---Andrea no está en casa.

Al oír esto sentí un sacudimiento. Nuevo y más doloroso cambio en mis
ideas, en mi voluntad, en mi cólera, en mis planes; nuevo movimiento de
la aguja magnética que brujuleaba en mi corazón, marcándome el derrotero
en medio de la tempestad\ldots{} El Marqués no podía tener interés en
negarme a su esposa. Así lo comprendí al momento, y sin vacilar un
instante, dije:

---¿Ha ido a la casa de D.ª María Antonia?

---Precisamente, allí está---manifestó Falfán en tono de confianza
honrada y tranquila que hubiera cautivado a otra persona más irritada
que yo.---La Sra. D.ª María Antonia se puso anoche mala y mi esposa fue
a acompañarla un ratito. A las diez estaba de vuelta.

---¿A las diez?

---Pero sin duda la Sra. D.ª María Antonia se ha agravado hoy, porque al
rayar el día vinieron a buscar a Andrea y allá está. ¿Encuentra usted en
esto algo de extraño?

---No señor, nada---dije levantándome.---¿Y dónde vive esa D.ª Antonia?

---En la calle que sale a la puerta de Carmona, número 26. ¿Pero se va
usted sin explicarme el motivo de su visita, su agitación\ldots?

---Sí señor, me voy.

---Pero\ldots{}

---Adiós, señor Marqués.

Quiso detenerme; pero rápida como un pájaro fugitivo, le dejé y salí de
la casa.

---A la calle que sale a la puerta de Carmona, número 26---dije a
Mariana que me seguía durmiendo.

---Ahora---decía para mí, en el horroroso vértigo que formaban mis
pensamientos y mi marcha,---ahora sí que de ningún modo se me puede
escapar.

Yo saboreaba de antemano las horribles delicias del escándalo que iba a
dar, de la venganza que tomaría, de las palabras que saldrían de mi
boca, como el humo y la lava de un volcán en erupción. Me deleitaba con
aquella copa de amarguras que se convertía en copa llena de delicioso
licor de la venganza. Había llegado al extremo de recrearme en el veneno
de mi alma y de hallar delicioso el fuego que respiraba. Seguía teniendo
las mismas ganas de morder a alguien, y creo que mi linda boca tan
codiciada, habría sido un áspid, si en carne humana hubiera posado sus
secos labios.

Mariana, que conocía a Sevilla, me llevó hacia la puerta de Carmona, yo
no sé por dónde ni en cuánto tiempo. Había yo perdido la noción de la
distancia y del tiempo. Vi una calle larga y solitaria, con muchas rejas
verdes llenas de tiestos de albahaca. Vi una fila de casas de fachada
blanca iluminadas por el sol y otra línea de casas en la sombra. Yo
buscaba el número 26, cuando sentí pisadas de caballos. Delante de mí,
como a cuarenta pasos, abriose una gran puerta y salieron tres hombres a
caballo. ¡Era él!

Corrí, corrí\ldots{} Iba vestido con el traje popular andaluz, y su
figura era la más hermosa que puede imaginarse. Los otros dos vestían lo
mismo. Caracolearon un instante los corceles delante de la casa, y en
seguida emprendieron precipitadamente la carrera en dirección a la
puerta de Carmona.

Yo corría, corría, y al mismo tiempo gritaba. Mariana, que no había
perdido el juicio, me detuvo enlazando con sus dos brazos mi talle. Mi
furor estalló con un grito salvaje, con una convulsión horrible y este
apóstrofe inexplicable: ---¡Ladrones! ¡Ladrones!

En el mismo momento en que yo rugía de este modo, dos mujeres se
asomaban a la ventana de la casa y saludaban a los jinetes con sus
abanicos. Él miró repetidas veces hacia atrás y saludaba también
sonriendo. Vi brillar el lente de D.ª María Antonia, vi los negros ojos
de Andrea\ldots{} ¡Oh Satanás, Satanás!

Yo seguí hasta ponerme debajo de la ventana; pero esta se cerró. Seguí
corriendo un poco más. Un grupo de hombres feroces apareció por una
boca-calle. Su aspecto infundía pavor; pero yo me adelanté hacia ellos y
señalando a los tres jinetes que huían a escape fuera de la puerta entre
nubes de polvo, grité con toda la fuerza de mis pulmones:

---¡Que se escapan!\ldots{} corred\ldots{} corred tras ellos\ldots{}
¡Que se escapan!\ldots{} los patriotas, los más malos de todos, los
ateos, blasfemos, los republicanos, los masones, los regicidas, los
enemigos del Rey\ldots{} ¡los que querían matarle\ldots! Corred y
cogedles\ldots{} Yo tengo dinero\ldots{} Mil duros al que les
coja\ldots{} ¡En nombre de la religión!\ldots{} ¡En nombre de las
caenas!\ldots{} Vamos, vamos tras ellos\ldots{} ¡Que se escapan!

A medida que hablaba, iba desapareciendo en mi espíritu la noción de lo
externo, y me sentía envuelta en tinieblas o en llamas, no sé en qué; me
sentía caer en un hondo infierno lleno de demonios; sumergirme en abismo
de negro delirio, de fiebre, de sueño o muerte; pues no puedo expresar
bien lo que era aquello.

Perdí el conocimiento.

\hypertarget{xxxii}{%
\chapter{XXXII}\label{xxxii}}

Mi dolorosa enfermedad que me puso al borde del sepulcro duró cuarenta
días, de los cuales no sé cuántos pasé en terrible crisis, sin
conciencia de las cosas, atormentada por la fiebre. Mi sangre enardecida
había descompuesto en tales términos las funciones de mi cerebro, que en
aquellos angustiosos días no vivía con mi vida propia, sino con el mismo
fuego mortífero de la enfermedad. Asistiome uno de los primeros médicos
de Sevilla.

Cuando salí del peligro y hubo esperanzas de que aún podría seguir mi
persona fatigando al mundo con su peso, halleme en tristísimo estado,
sin memoria, sin fuerzas, sin belleza. Mas empecé a recobrar muy
lentamente estos tesoros perdidos, y con ellos volvían mis pasiones y
mis rencores a aposentarse en mi seno, como después de una inundación, y
cuando las aguas se retiran, aparece lentamente la tierra, dibujándose
primero los altos collados, luego las suaves pendientes y por último el
llano. Así, pasada aquella avenida de sangre que envolvió mi pensamiento
en turbias olas venenosas, fue apareciendo poco a poco todo lo existente
antes del 13 de Junio.

Una imagen descollaba sobre todas las que me perseguían, cuando mi
fantasía, como un borracho que recobra la claridad de sus sentidos,
empezó a presentarme lo pasado. Esta imagen era la de la huérfana, a
quien supuse corriendo sin cesar por campos y ciudades, buscando lo que
no había de encontrar. ¿Acaso el tormento de ella no era tan grande o
quizás mayor que el mío? Pero yo no me hacía cargo de esto, y lejos de
sentir lástima de mi víctima, echaba leña a la hoguera de mis rencores,
discurriendo mil defectos y fealdades en el carácter de la hermana de
Salvador, para deducir que sus angustias le estaban muy bien merecidas.
¡Qué desatinos tan horribles pensé con este motivo! Parece mentira que
la exaltación de mi ánimo me llevara hasta los últimos desvaríos, hasta
el sacrilegio y la blasfemia.

---Es muy posible---decía yo,---que mis horribles angustias hayan sido
causadas por las maldiciones de esa mujer. Al verse engañada habrá
pedido a Dios mi castigo, y Dios, no hay duda, hace caso de los
hipócritas\ldots{} ¡Ah, los hipócritas!, ¡perversa raza! Son capaces con
sus fingidas lágrimas de engañar al mismo Dios y compelerle a castigar a
los buenos.

A estos horrorosos pensamientos hijos de una turbada razón, añadía otros
quizás más sacrílegos. Mi enfermedad, que parecía un aviso del cielo, no
me había corregido, antes bien, cuando resucité estaba más intolerante,
más soberbia, y proyectaba nuevos planes para vencer la tenaz
contrariedad de mi destino. Lejos de desconfiar de mis fuerzas y de
acobardarme, tenía fe mayor en ellas y me vanagloriaba, suponiendo una
inmediata victoria.

---Me han ocurrido tantos desastres---decía,---porque he sido una tonta.
Pero ahora\ldots{} ¡Oh!, ahora yo me juro a mí misma que moriré o le he
de atrapar\ldots{} Iré a Cádiz.

Cuando esto decía, finalizaba Julio y la temperatura de Sevilla era
irresistible. El médico me ordenó que buscase en la costa aires más
templados.

Los franceses se habían establecido ya en Sevilla, donde reinaba un
orden perfecto. En toda España, y principalmente en algunos puntos
privilegiados de la tragedia, como Manresa y la Coruña, corría la sangre
a raudales. Los dos furibundos partidos se herían mutuamente con impía
crueldad. Pero los ejércitos de ambas Naciones no habían empeñado
ninguna lucha verdaderamente marcial y grandiosa. El nuestro se
desbandaba como un rebaño sin pastores y el francés iba ocupando las
ciudades desguarnecidas y dominando todo el país sin trabajo y sin
heroísmo, sin sangre y sin gloria. Sus victorias eran ramplonas y
honradas, su proceder dentro de los pueblos, noble y templado. Era aquel
ejército como su jefe, leal y sin genio, un ejército apreciable,
compuesto de cien mil buenos sujetos que no conocían el saqueo, pero
tampoco la gloria. ¡Detestable suerte la de España!\ldots{} ¡Haber hecho
temblar al coloso y sucumbir ante un hijo del conde de Artois, ante un
pobre emigrado de Gante!

¡A Cádiz, a Cádiz! Estas palabras compendiaban todo mi pensamiento en
aquellos días. Empecé a disponer mi viaje con gran prisa, y a principios
de Agosto nada tenía que hacer ya en Sevilla.

Mi belleza recobraba al fin su esplendor. Y no era esto poco triunfo,
porque la verdad es que me había quedado como un espectro. ¡Con cuánto
alborozo veía yo despuntar de día en día la animación, la gracia, la
frescura, la viveza, todos los encantos de mi fisonomía, que iban
mostrándose, como flores que se abren al cariñoso amor del sol! Yo no
cesaba de mirarme al espejo para observar los progresos de mi
restauración, y casi casi estoy por decir que me encontraba más guapa
que antes de mi enfermedad. Perdóneseme este orgullo vano; pero si Dios
me hizo así, si me dio hermosura y gracias, ¿por qué no lo he de decir
para que lo sepan los que no tuvieron la dicha de conocerme?

El conde de Montguyon se me presentó en el momento de partir para Cádiz.
¡Oh, feliz encuentro! Mi D. Quijote, que había sido ascendido a jefe de
brigada, me acompañó en casi todo el camino de Sevilla a la costa,
mostrándose en extremo orgulloso por creer próximo el momento de mi
definitiva conquista, y yo cuidaba no poco de confirmarle en esta
creencia, porque quería tenerle muy dispuesto a servirme en negocios
difíciles. Hablamos también de política y de la Ordenanza de Andújar, en
que Su Alteza recomendaba la mayor templanza a los absolutistas,
habiéndoles disgustado por esto. Pero el tema más agradable a mi
caballero era el amor.

Según se expresaba, su bello ideal estaba a punto de realizarse. El país
ardiente, el territorio pintoresco, la dama hermosa; nada faltaba para
que la leyenda fuese completa. Pero yo, esmerándome en fomentar sus
esperanzas, era sumamente avara de concesiones. Mi ordenanza de Andújar
prescribía también la moderación.

Ya me había yo instalado en el Puerto cuando, apremiada por el Conde, le
revelé la causa de mis ardientes deseos de penetrar en Cádiz.

---Un hombre---le dije,---que antes poseía mi confianza, administrando
los bienes de mi casa; un mayordomo que supo servirme algún tiempo con
lealtad para engañarme después con más seguridad, huyó de Madrid,
robándome gran cantidad de dinero, muchas alhajas de valor y documentos
preciosos. Ese hombre está en Cádiz\ldots{}

---Pero en Cádiz hay tribunales de justicia, hay autoridades\ldots{}

---En Cádiz no hay más que un Gobierno expirante que para prolongar su
vida entre agonías, se rodea de todos los pillos.

---Sin embargo, señora, un ladrón de semejante estofa no puede ser
patrocinado por nadie. Horribles cosas se ven en las guerras civiles;
pero nosotros, nosotros los franceses entraremos en Cádiz.

---Esa es mi esperanza.

---¿No tiene usted valimiento con los Ministros liberales?

---Ninguno. Mi nombre sólo les sonará a proclama realista.

---Entonces\ldots{}

---Cuento con la protección de los jefes del ejército francés.

---Y con los servicios de un leal amigo\ldots{} El objeto principal es
detener al ladrón.

---¡Detenerle y amarrarle y arrastrarle!---exclamé con furor.---Mas
deseo hacer mi justicia a espaldas de los tribunales, porque aborrezco
la curia y los pleitos, aun cuando los gane.

---¡Oh!, eso es muy español. Se trata, pues, de cazar a un hombre; ¿por
ventura eso es fácil todavía?

---Fácil, no.

---Y para una dama\ldots{}

---Pero yo no estoy sola. Tengo servidores leales que sólo esperan una
orden mía para\ldots{}

---Para matar\ldots{}

---No tanto---dije riendo.---Esto le parecerá a usted leyenda, novela,
romance o lo que quiera; pero no, mis propósitos no son tan trágicos
como usted se figura.

---Lo supongo\ldots{} pero siempre serán interesantes\ldots{} ¿Ha dejado
usted criados en Sevilla?

---Uno tengo a mis órdenes. Le he enviado por delante, y ya está en
Cádiz.

---Vigilando\ldots{}

---Acechando.

---Bien: le seguirá de noche, embozado hasta las cejas; espiará sus
acciones, se informará de su método de vida. ¿Y ese criado es fiel?

---Como un perro\ldots{} Examinemos bien mi situación, señor Conde. ¿Se
puede entrar en Cádiz?

---Es muy difícil, señora, sobre todo para los que son sospechosos al
Gobierno liberal.

---¿Y por mar?

---Ya sabe usted que en la bahía tenemos nuestra escuadra.

---¿Cuándo tomarán ustedes la plaza?

---Pronto. Esperamos a que venga Su Alteza para forzar el sitio.

---¿Y podrán escaparse los milicianos y el Gobierno?

---Es difícil saberlo. Ignoramos si habrá capitulación; no sabemos el
grado de resistencia que presentarán los insurgentes.

---¡Oh!---exclamé sin saber lo que decía, obcecada por mis
pasiones.---Ustedes los realistas no sirven para esto. Si Napoleón
estuviera aquí, amigo mío, mañana, mañana mismo, sí señor, mañana, sería
tomada por asalto esa ciudad rebelde y pasados a cuchillo los insensatos
que la defienden.

---Me parece demasiado pronto---dijo Montguyon sonriendo.---En fin,
comprendo la impaciencia de usted.

---Sí, quien ha sido robada, vilmente estafada, no puede aprobar estas
dilaciones que dan fuerza al enemigo. Señor Conde, es preciso entrar en
Cádiz.

---Si de mí dependiera, señora, esta tarde mandaba dar el
asalto---repuso con entusiasmo.---Sorprendería a la guarnición,
encarcelaría a los diputados y a las Cortes y pondría en libertad al
Rey.

---Ya eso no me importa tanto---dije en tono de conquistador.---Yo
entraría al asalto sorprendiendo a la guarnición. Dejaría a los
diputados que hicieran lo que les acomodase, mandaría al Rey a
paseo\ldots{}

---Señora\ldots{}

---Buscaría a mi hombre, revolvería todos los rincones, todos los
escondrijos de Cádiz hasta encontrarle\ldots{} y después que le
hallara\ldots{}

---Después\ldots{}

---Después, señor Conde\ldots{} ¡Oh!, mi sangre se abrasa\ldots{}

---En los divinos ojos de usted, Jenara---me dijo,---brilla el fuego de
la venganza. Parece usted una Medea.

---No me impulsan los celos---dije serenándome.

---Una Judith.

---Ni la idea política.

---Una\ldots{}

---Parezca lo que parezca, señor Conde, ello es preciso entrar en Cádiz.

---Entraremos.

---¿No sirve usted ahora en el Estado Mayor del general Bourmont?

---En él estoy a las órdenes de la que es imán de mi vida---repuso
poniendo los ojos en blanco.

---¿Bourmont será nombrado comandante general de Cádiz, luego que la
plaza se rinda?

---Así se dice.

---¿Hará usted prender a mi mayordomo?\ldots{}

---Le haré fusilar\ldots{}

---¿Me lo entregará usted atado de pies y manos?

---Siempre que no huya antes, sí señora.

---¡Huir! Pues qué, ¿tendrá ese hombre la vileza de huir, de no
esperar?\ldots{}

---El criminal, amiga mía de mi corazón, pone su seguridad ante todo.

---¿No dice usted que hay una especie de escuadra?

---Una escuadra en toda regla.

---¿Pues de qué sirven esos barcos, señor mío---dije de muy mal
talante,---si permiten que se escape\ldots{} ese?

---Quizás no se escape.

---¿De qué sirve la escuadra?---añadí con la más viva
inquietud.---¿Quién es el almirante que la manda? Yo quiero ver a ese
almirante, quiero hablar con él\ldots{}

---Nada más fácil; pero dudo\ldots{}

---Me ocurre que si hay capitulación, será más fácil atraparle\ldots{}

---¿Al almirante?

---No; a\ldots{} a ese.

---Sin duda. En tal caso se quedaría tranquilo en Cádiz, al menos por
unos días.

---Bien, muy bien. Si hay capitulación, arreglo, perdón de vidas y
libertad para todos\ldots{} Señor Conde, aconsejaremos al Príncipe que
capitule\ldots{} ¡pero qué tonterías digo!

---Está patente en su espíritu de usted la obsesión de ese asunto.

---¡Oh!, sí; no puedo pensar en otra cosa. El caso es grave. Si no
consigo apoderarme de ese hombre\ldots{} no sé\ldots{} creo que me
costará la vida.

---Yo también le aborrezco\ldots{} ¡Hombre maldito!\ldots{} Pero le
cogeremos, señora. Me pongo al servicio de este gran propósito con la
sumisión de un esclavo. ¿Acepta usted mi cooperación?

Al decir esto me besaba la mano.

---La acepto, sí, hombre generoso y leal, la acepto con gratitud y
profundo cariño.

Al decir esto, yo ponía en mi semblante una sensibilidad capaz de
conmover a las piedras, y en mis pestañas temblaba una lágrima.

---Y entonces---añadió Montguyon con voz turbada,---cuando nuestro
triunfo sea seguro, ¿podré esperar que el hueco que se me destina en ese
corazón no sea tan pequeño?

---¿Pequeño?

---Si es evidente, por confesión de él mismo, que ya tengo una parte en
sus sublimes afectos, ¿no puedo esperar\ldots?

---¿Una parte? ¡Oh, no!; todo, todo.

El inflamado galán abrió sus brazos para estrecharme en ellos; pero
evadí prontamente aquella prueba de su insensato ardor, y poniéndome
primero seria y después amable, con una especie de enojo gracioso y
virtud tolerante, le dije que ni Zamora ni yo podíamos ser ganadas en
una hora. Al decir esto violentos cañonazos me hicieron estremecer y
corrí al balcón.

---Son los primeros tiros de las baterías que se han armado para atacar
el Trocadero---me dijo el Conde.

---¿Y esas bombas van a Cádiz?---pregunté poniendo inmenso interés en
aquel asunto.

---Van al Trocadero.

---¿Y qué es eso?

---Un fuerte que está en medio de las marismas.

---¿Y allí están\ldots?

---Los liberales.

---¿Muchos?

---Mil y quinientos hombres.

---¿Paisanos?

---Hay muchos paisanos y milicianos.

---¡Oh!, morirá mucha gente.

---Eso es lo que deseamos. Parece que siente usted gran pena por ello.

---La verdad---repuse, ocultando los sentimientos que bruscamente me
asaltaban,---no me gusta que muera gente.

---A excepción de su enemigo.

---Ese\ldots{} ¿pero estará en el Trocadero?

---¡Quién sabe!\ldots{} Está usted aterrada, Jenara.

---¡Oh!, yo quiero ir al Trocadero.

---Señora\ldots{}

---Quiero ir al Trocadero.

---Eso mismo deseamos nosotros---me dijo riendo,---y para conseguirlo,
enviaremos por delante algunos centenares de bombas.

---¿Dónde está el Trocadero?---pregunté corriendo otra vez a la ventana.

---Allí---dijo Montguyon asomándose y alargando el brazo.

Hízome explicaciones y descripciones muy prolijas de la bahía y de los
fuertes; pero bien comprendí que antes que mostrar sus conocimientos,
deseaba estar tan cerca de mí como estaba, aproximando bastante su
cabeza a la mía, y embriagándose con el calor de mi rostro y con el roce
de mis cabellos.

\hypertarget{xxxiii}{%
\chapter{XXXIII}\label{xxxiii}}

¡Qué aparato desplegaron contra aquellas fortalezas que se alzan entre
charcos salubres y que llevan por nombre el Trocadero! Desde que llegó
Su Alteza a mediados de Agosto, no hacían más que disparar bombas y
balas contra los fuertes, esperando abrir brecha en sus gloriosos muros.
¡Figúrese el buen lector mi aburrimiento! Considere con cuánta tristeza
y tedio vería yo pasar día tras día sin más distracción que oír los
disparos y ver por las noches las majestuosas curvas de los proyectiles.
Me consumía en mi casa del Puerto sin tener noticias del interior de
Cádiz, ni esperanzas de poder penetrar en la plaza. Ni parecía aquello
guerra formal y heroica como creía yo que debían ser las guerras y como
las que vi en mi niñez y en tiempo del Imperio. Casi todo el ejército
sitiador estaba con los brazos cruzados: los oficiales paseaban fumando;
los soldados hacían menos pesado el tiempo con bailoteo y cantos.

No debo pasar en silencio que el duque del Infantado que llegó de Madrid
en aquellos días, me llevó a visitar a Su Alteza, nuestro salvador y el
ángel tutelar de la moribunda España por aquellos días. Luis Antonio era
un rubio desabrido, cuyo semblante respiraba honradez y buena fe; pero
la aureola del genio no circundaba su frente. Fuera de aquel sitio,
lejos de aquella deslumbradora posición y con otro nombre, el hijo del
conde de Artois habría sido un joven de buen ver; mas no en tal manera
que por su aspecto descollase entre la muchedumbre. Para hallar en él lo
que realmente le distinguía era preciso que un trato frecuente hiciese
resaltar las perfecciones morales de su alma privilegiada, su lealtad
sin tacha y aquel levantado espíritu caballeresco sin quijotismo que le
hacía tan estimable en la Corte de Francia. Era valiente, humanitario,
cortés, afable, puntual y riguroso en el cumplimiento del deber. Si
estas cualidades no eran suficientes a formar un gran guerrero, ¿qué
importaba? La pericia militar diéronsela sus prácticos generales y
nuestros desaciertos, que fueron el principal estro marcial de la
segunda invasión.

Angulema me recibió con la más fina delicadeza y urbanidad; pero de
todas sus cortesanías la que más me agradó fue la de disponer el asalto
del Trocadero. ---¡Al fin, al fin---exclamaba yo,---será nuestro el
horrible fuerte que nos abrirá las puertas de Cádiz!

El 19 abrieron brecha; pero hasta la noche del 30 no se dio el asalto,
habiéndose guardado secreto sobre esto en los días anteriores, aunque yo
lo supe por el conde de Montguyon, que no me ocultaba nada referente a
las operaciones. ¡Noche terrible la del 30 al 31 de Agosto!, noche que
me pareció día por lo clara y hermosa así como por el estrépito guerrero
que en ella resonara y las acciones heroicas dignas de ser alumbradas
por el sol!\ldots{} Apretado fue el lance del asalto, según oí contar, y
Su Alteza y el príncipe de Carignan, se portaron bravamente combatiendo
como soldados en los sitios más peligrosos. No fue ciertamente el hecho
del Trocadero una de aquellas páginas de epopeya que ilustraron el
Imperio; fue más bien lo que los dramaturgos franceses llaman Succés
d'estime, un éxito que no tiene envidiosos. Pero a la Restauración le
convenía cacarearlo mucho, ciñendo a la inofensiva frente del Duque los
laureles napoleónicos; y se tocó la trompa sobre este tema hasta
reventar, resultando del entusiasmo oficial que no hubo en Francia calle
ni plaza que no llevase el nombre del Trocadero, y hasta el famoso arco
de la Estrella, en cuyas piedras se habían grabado los nombres de
Austerlitz y Wagram, fue durante algún tiempo Arco del Trocadero.

Yo me había trasladado a Puerto Real para estar más cerca. En la mañana
del 31, cuando vi pasar a los prisioneros hechos en los fuertes, me
sentí morir de zozobra. Entre aquellas caras atezadas, a cada instante
creía ver la suya. Estuvieron pasando mucho tiempo, porque eran más de
mil entre militares y paisanos. Creo que les miré uno por uno; y al fin,
cuando ya quedaban pocos, redoblé mi atención. ¡Oh misericordioso Dios,
qué estupendas cosas permites! En la última fila, casi solo, más
abatido, más quemado del sol, más demacrado, con los vestidos más rotos
que los demás, pasó él, ¡él mismo\ldots!, no podía dudarlo, porque le
estaba viendo, viendo, sí, con mis propios ojos arrasados de lágrimas.
Llevaba la mano izquierda en cabestrillo hecho con un andrajo, y su paso
era inseguro y como dolorido, sin duda por tener lleno de contusiones el
cuerpo.

Al verle extendí los brazos y grité con toda la fuerza de mi voz. Mi
enamorada exclamación hizo volver la cabeza a todos los que iban delante
y a los curiosos que le rodeaban. Él, alzando los amortiguados ojos, me
miró con expresión tan triste que sentí partido mi corazón y estuve a
punto de desmayarme. Creo que pronunció algunas palabras; pero no oí
sino un adiós tan lúgubre como campanada funeral, y movió la mano en
ademán de cariñoso saludo, y pasó, desapareciendo con los demás en una
vuelta del camino.

Mi primera intención fue correr tras él; pero en la casa me detuvieron.
Cuando serenamente me hice cargo de la situación, formé mil proyectos;
pero todos los desechaba al punto por descabellados. Pensándolo bien,
comprendí que no era tan difícil conseguir su libertad. Me congratulaba
de que, al cabo de tantas fatigas, el destino me le presentara
prisionero para poder decir con más valor que nunca:---Ahora sí que no
se me puede escapar.

\hypertarget{xxxiv}{%
\chapter{XXXIV}\label{xxxiv}}

Envié recados al conde de Montguyon; pero no se le podía encontrar por
ninguna parte. Unos decían que estaba en el Trocadero, otros que en el
Puerto, otros que había ido a las fragatas con una comisión. Por último,
averigüé con certeza su paradero y le escribí una carta muy cariñosa.
Mas pasó un día, pasaron dos y yo me moría de impaciencia, sin poder ver
al prisionero ni aun saber dónde le habían llevado. El Conde, robando,
al fin, un rato a sus quehaceres, vino a verme el día 4. Yo estaba otra
vez medio loca y no tenía humor para hacer papeles, sino que
espontáneamente dejaba que se desbordasen los sentimientos de mi
corazón.

---¡Oh! Cuánto me alegro de ver a usted---le dije.---Si usted no viene
pronto, señor Conde, me hubiera muerto de pena.

Con estas palabras, que creía dictadas por un vivo interés hacia él, se
puso el noble francés un poco chispo, que así denomino yo al
embobamiento de los hombres enamorados. Se deshizo en galanterías, a las
cuales daba cierto tono de intimidad cargante, y después me dijo:

---Pronto, muy pronto, libertaremos a Su Majestad el Rey de España, y
entraremos en Cádiz. El sol de ese día, señora, ¡cuán alegremente
brillará sobre toda España, y especialmente sobre nuestros corazones!

---Mi estimado amigo---indiqué riendo,---no diga usted tonterías.

Él se quedó cortado.

---Basta de tonterías---añadí,---y óigame usted lo que voy a decirle. Ya
he encontrado al hombre que buscaba\ldots{}

---¿Dónde?\ldots{} ¿cómo?\ldots{} ¿ese malvado?

---No es malvado.

---¿Cómo no? Me dijo usted que le había robado sus alhajas.

---¡No es ese\ldots{} por Dios! ¿Cuándo entenderá usted las cosas al
derecho?

---Siempre que no se me expliquen al revés.

---He encontrado a ese hombre\ldots{} Pero entendámonos. ¿No dije a
usted que había venido delante de mí un fiel criado de mi casa, el cual
entró en Cádiz?\ldots{}

---¡Ah!, sí\ldots{} entró para observar los pasos del ladrón.

---Pues ese fiel criado tiene el defecto de ser algo patriota\ldots{}
¡debilidades humanas!, y como es algo patriota se puso a pelear en el
Trocadero por una causa que no le importaba.

---Ya comprendo, y ha caído prisionero. ¿Le ha visto usted?

---Le vi cuando los prisioneros pasaron por aquí, pero no le he visto
más; y ahora, señor Conde, quiero que usted me le ponga en libertad.

---Señora, si Cádiz se rinde pronto, como creo, y todo se arregla,
espero conseguir lo que usted me pide.

---¡Qué gracia! Para eso no necesito yo de la amistad de un jefe de
brigada---dije con enfado.---Ha de ser antes, mañana mismo.

---¡Oh! Señora, usted somete mi amor a pruebas demasiado fuertes.

---¿Quiere usted que dejemos a un lado el amor---le dije poniéndome muy
seria,---y que hablemos como amigos?

Montguyon palideció.

---¿Esa persona---me dijo,---interesa a usted tanto que no puede esperar
a que concluya la guerra, dando yo mi palabra de que el prisionero será
bien atendido?

---No basta que sea atendido---afirmé con resolución.---No basta nada;
quiero su libertad; quiero atenderle yo misma, cuidarle, curar sus
heridas, tenerle a mi lado, llevarle a sitio seguro\ldots{}

Me expresé, al decir esto, con vehemencia suma, porque me era ya muy
difícil contener mi corazón que iba al galope en busca de las anheladas
soluciones. El Conde me oía con cierto terror.

---¿Tanto interesa a usted---repitió,---tanto interesa a usted\ldots{}
un criado?

---No es criado.

---¿Tal vez un anciano servidor de la casa?

---No es anciano.

---¿Un joven?\ldots{} ¿Supongo que no será el ladrón?

---¿Qué ladrón?

---El ladrón de quien usted me habló\ldots{}

---¡Ah! No me acordaba\ldots{} Ya no me ocupo de eso.

---¿Abandona usted la empresa de detener y castigar a ese miserable?

---La abandono.

---¡Qué inconstancia!

---Yo soy así.

---Pero ese, ese otro\ldots{} ¿interesa a usted tanto?\ldots{}

---Muchísimo.

---¿Es pariente de usted?

---No.~Es compañero de la infancia.

---¿Es militar?

---Paisano, señor Conde---dije con el tono de severa autoridad que sé
emplear cuando me conviene.---Si se empeña usted en ser catecismo,
buscaré otra persona más galante y más generosa que sepa prestar un
servicio, economizando las preguntas.

---Creo tener algún derecho a ello---repuso con gravedad.

---No tiene usted ninguno---afirmé con desenfado,---porque este derecho
yo sola podría darlo, y yo lo niego.

---Entonces, señora---objetó, encubriendo su ira bajo formas
urbanas,---he padecido una equivocación.

---Si cree usted que le amo, sí. La equivocación no puede ser más
completa.

Montguyon se levantó. Sus ojos, en los cuales se leía el furor mezclado
con la dignidad, me dirigieron una mirada, que debía ser la última. Yo
corrí a él y tomándole la mano, le rogué que se sentase a mi lado.

---Usted es un caballero---le dije.---Ningún otro ha merecido más que
usted mi estimación, lo juro. Dios sabe que al decir esto hablo con el
corazón.

---Dios lo sabrá---repuso Montguyon muy afligido;---mas para mí, y de
aquí en adelante, las palabras de usted están escritas en el agua.

---Considere usted las que le diga hoy como si estuvieran grabadas en
bronce. La que confiesa hechos que no le favorecen, ¿no tiene derecho a
ser creída?

---A veces sí. Confiéseme usted que su conducta conmigo no ha sido leal.

---Lo confieso---repliqué bajando los ojos y realmente avergonzada.

---Confiese usted que yo no merecía servir de juguete a una mujer
voluntariosa.

---También es cierto y lo confieso.

---Declare usted que ama a otro.

---¡Oh!, sí, lo declaro con todo mi corazón, y si cien bocas tuviera con
todas lo diría.

El leal caballero se quedó atónito y espantado. Estaba, como ellos
dicen, foudroyé. Durante breve rato no me dijo nada, pero yo comprendí
su martirio y le tenía lástima. ¡Oh, qué mala he sido siempre!

---Ese hombre\ldots---murmuró Montguyon,---ese hombre\ldots{}

---Ahora, reconociéndome culpable, reconociéndome inferior a
usted---dije,---le autorizo para que me abrume a preguntas, si gusta, y
aun para que me eche en cara mi ligereza.

---Ese hombre\ldots---prosiguió el francés.---Perdone usted; pero nada
es más curioso que la desgracia. El amor desairado quiere tener miles de
ojos para sondear las causas de su desdicha. Ese hombre\ldots{} ¿quién
es?

---Un hombre.

---¿De familia ilustre?

---No señor, de origen muy humilde.

---¿Le ama usted hace tiempo?

---Hace mucho tiempo.

---Él\ldots{} ¿la ama a usted?

---No estoy muy segura de ello.

---¡Oh! ¡Qué iniquidad!---exclamó con furor el Conde.---Es un miserable.

---Un ingrato, y es bastante.

---¿Y a pesar de su ingratitud le ama usted?

---Tengo esa debilidad, que no puedo dominar.

---Aborrézcale usted.

---Si fuera fácil\ldots{} Difícil cosa es esa.

---¡Es verdad, difícil cosa!---exclamó Montguyon con tristeza.---¿Y ese
hombre?\ldots{}

---¿Pero hay más preguntas todavía?

---No, ya no más. Me basta lo que sé, y me retiro.

---Se conduce usted como un cualquiera---le dije con verdadero
afecto.---Me abandona usted, precisamente cuando mi sinceridad merece
alguna recompensa. ¿Será posible que cuando yo empiezo a tener
franqueza, deje usted de tener generosidad?

---¡Oh! Señora, toca usted una fibra de mi corazón que siempre responde,
aun cuando la hieran con puñal.

---Sí, sí, amigo mío. Usted es generoso y noble en gran manera. Para que
la diferencia entre los dos sea siempre grande, para que usted sea
siempre un caballero y yo una miserable, págueme usted como pagan en
todas ocasiones las almas elevadas. Pues yo me he portado mal, pórtese
usted bien conmigo. Haga cada cual su papel. Cumpla usted el precepto
que manda volver bien por mal. Así crecerá más a mis ojos; así me
abatiré yo más a los suyos; así su generosidad será mayor y mi culpa más
grande también, y usted tendrá en su vida una página más gloriosa que la
victoria que acaba de alcanzar frente al enemigo.

---Comprendo lo que usted me dice---murmuró el francés, descansando por
breve rato su frente en la palma de la mano.---Yo seré siempre digno de
mi nombre.

---¡Caballero leal antes, ahora y siempre!---exclamé yo.

---Bien, señora---dijo levantándose y alargándome la mano que estreché
cordialmente.---Lo que usted desea de mí es bastante claro.

---Sí.

---Y yo---añadió con manifiesta emoción,---empeño mi palabra de
honor\ldots{}

---¡Oh!, lo esperaba, lo esperaba.

---Doy mi palabra de honor de hacer cuanto esté en mi mano para devolver
a usted la felicidad, entregándole a su amante.

---Gracias, gracias---exclamé derramando lágrimas de admiración y
agradecimiento.

El Conde, saludándome ceremoniosamente, se retiró. De buena gana le
habría dado un abrazo.

\hypertarget{xxxv}{%
\chapter{XXXV}\label{xxxv}}

¡Qué días pasaron! Yo contaba las horas, los minutos, como si de la
duración de ellos dependiese mi vida. Entre españoles y franceses era
opinión corriente que la guerra acabaría pronto, que Cádiz expiraba, que
las Cortes se morían por momentos. Sin embargo, aún resistía el Gobierno
liberal y sus secuaces, como la bestia herida que no quiere soltar su
presa mientras tenga un hálito de existencia. Esta constancia no carecía
de mérito, y lo tendría mayor si se empleara en causa menos perdida.
¡Qué sacrificio tan inútil! No tenían hombres, porque los alistamientos
no producían efecto. No tenían dinero, porque el empréstito que
levantaron en Londres produjo\ldots{} una libra esterlina. Yo creo que
si mi espíritu hubiera estado en disposición de admirar algo, habría
admirado la perseverancia de aquel Gobierno que no pudo encontrar en
toda Europa quien le prestase más de cinco duros.

Mi deseo era que se rindiese todo el mundo, que el Rey y la Nación
arreglasen pronto sus diferencias, aunque las arreglaran devorándose
mutuamente. Yo quería tener el campo libre para el desenlace de mi
campaña amorosa, que veía ya seguro y feliz.

Casi todo Setiembre lo pasaron Angulema y las Cortes en dimes y diretes.
Mil recados atravesaban la bahía en un bote; callaban los cañones para
que hablaran los parlamentarios. Tales comedias me ponían furiosa,
porque no se decidía la suerte de los infelices prisioneros del
Trocadero, que habían sido repartidos entre los Dominicos del Puerto y
la Cartuja de Jerez.

Montguyon me visitó el 12, para informarme de que había visto al
prisionero, cuyo nombre y señas le había dado yo oportunamente.

---Está sumamente abatido y melancólico---me dijo.---Se ha negado a
recibir los auxilios pecuniarios que le ofrecí de parte de usted; pero
se ha mostrado muy agradecido. Al oír que Jenara tenía gran empeño en
conseguir su libertad, pareció muy turbado y conmovido, pronunciando
palabras sueltas cuyo sentido no pude comprender.

---¿Y no desea verme?

---Parece que lo desea ardientemente.

---¡Oh! ¡Estas dilaciones son horribles! ¿Y qué más dijo?

---Cosas tristes y peregrinas. Afirma que desea la libertad para
conseguir por ella el destierro.

---¡El destierro!

---Dice que aborrece a su país y que la idea de emigración le consuela.

---Le conozco, sí\ldots{} Esa idea es suya.

Otras cosas me dijo el Conde; pero se referían al trato que se daba a
los prisioneros y a las excepciones ventajosas que él estableciera en
beneficio de mi amado. ¡Cuánto le agradecí sus delicadezas! Mientras
viva tendré buenos recuerdos de hombre tan caballeroso y humanitario.

Interrumpidos los tratos por la terquedad de las Cortes, tomó de nuevo
la palabra el cañón, y el día 20 fue ganado por los franceses con otro
brioso asalto, el castillo de Santi-Petri. Después de este hecho de
armas, Angulema habló fuerte a los tenaces liberales, pegados como lapas
a la roca constitucional, y les amenazó con pasar a cuchillo a toda la
guarnición de Cádiz, si Fernando VII no era puesto inmediatamente en
libertad. El 26 se sublevó contra la Constitución el batallón de San
Marcial, que guarnecía la batería de Urrutia en la costa; y la armada
francesa, secundando el fuego de las baterías del Trocadero, arrojaba
bombas sobre Cádiz. No era posible mayor resistencia. Era una tenacidad
que empezaba a confundirse con el heroísmo, y la Constitución moría como
había nacido, entre espantosa lluvia de balas, saludada en su triste
ocaso, como en su dramático oriente, por las salvas del ejército
francés.

Por fin llegaba el anhelado día.

---Habrá perdón general---decía yo para mí.---Todos los prisioneros
serán puestos en libertad. Huiremos. ¡Cuán grato es el destierro!
Comeremos los dos el dulce pan de la emigración, lejos de indiscretas
miradas, libres y felices fuera de esta loca patria perturbada donde ni
aun los corazones pueden latir en paz.

Montguyon me trajo el 29 muy malas noticias.

---El Duque ha resuelto poner en libertad a todos los prisioneros de
guerra. Pero\ldots{}

---¿Pero qué?

---Ha dispuesto que sean entregados a las autoridades españolas los
individuos que en Cádiz desempeñaban comisiones políticas.

---¿Él está comprendido?

---Sí señora. Desgraciadamente se tienen de él las peores noticias.
Había recorrido los pueblos alistando gente por orden de Calatrava;
había venido desde Cataluña con órdenes de Mina para realizar asesinatos
de franceses. Había organizado las partidas de gente soez que en el
tránsito de Sevilla a Cádiz insultaron a Su Majestad.

---¡Oh, eso es falso, falso, mil veces falso!---exclamé sin poder
contener mi indignación.

Y en efecto, tales suposiciones eran infames calumnias.

---Ha llegado al Puerto de Santa María---añadió Montguyon,---el Sr.~D.
Víctor Sáez, secretario de Estado, ¿por qué no le ve usted?

---No quiero nada con hombres de ese jaez---repuse con enojo.---Usted me
ha dado su palabra de honor, usted ha empeñado su nombre de caballero, y
con usted solo debo contar. ¡Oh!, señor Conde, si mi prisionero es
entregado a la brutalidad de las autoridades españolas, sedientas hoy de
sangre y de venganza, sospecharé que usted me hace traición.

Palideció el caballero francés. Dirigiéndome una mirada desdeñosa, me
dijo al despedirse:

---Todavía, señora, no sabe usted quién soy yo.

A pesar de mis propósitos determiné visitar a Sáez, porque bueno es
tener amigos aunque sea en el infierno. Vencí mis recientes antipatías,
y tomando un coche me encaminé al Puerto de Santa María. Era el 1.º de
Octubre, día solemne en los fastos españoles.

Hallé al buen canónigo más soplado y presuntuoso que nunca, como todo
aquel que se ve en alturas a donde nunca debió llegar; pero contra lo
que yo esperaba, recibiome afablemente y no me dijo una sola palabra
acerca de mi conversión al absolutismo. Parecía olvidado de estas
pequeñeces, y ocuparse tan sólo, como Jiménez de Cisneros, en los
negocios públicos de ambos mundos.

---Hoy es día placentero, señora, día feliz, entre todos los días
felices de la tierra---me dijo.---Su Majestad D. Fernando, ese ilustre
mártir de los excesos revolucionarios es ya libre.

---¿Ya?

---Hoy nos le entregan. Al fin han comprendido esos locos que su
resistencia les podría costar muy cara, pero muy cara. El Duque tiene
malas moscas.

---Felicitémonos, Sr.~D. Víctor---dije con afectado entusiasmo,---de
esta solución lisonjera. España y el mundo están de enhorabuena. Mas
para que se completara la dicha, convendría que tantas y tan graves
heridas no se ensañasen con la venganza y la crueldad del partido
vencedor, y que un generoso olvido de los errores pasados inaugurase la
venturosa era que empieza hoy.

---Así será, señora---repuso sonriendo de un modo que me pareció algo
hipócrita.---Su Majestad ha dado ayer en Cádiz un manifiesto en que
ofrece perdonar a todo el mundo y no acordarse para nada de los que le
han ofendido. ¡Cuánta magnanimidad! ¡Cuánta nobleza!

---¡Oh!, sí, conducta digna de un descendiente de cien Reyes, digna de
quien da el perdón y del pueblo que la recibe. Si Fernando cumple lo que
promete, será grande entre todos los Reyes de España.

---Lo cumplirá, señora, lo cumplirá.

Aunque no tenía gran confianza en las afirmaciones de Sáez, di crédito a
estos propósitos por creerlos inspiración del duque de Angulema.

Invitome luego a presenciar el desembarco de Su Majestad, a lo que
accedí muy gustosa. Nos trasladamos al muelle, y habiendo sido colocada
por un oficial francés en sitio muy conveniente para ver todo, presencié
aquel acto que debía ser uno de los más notables recodos, uno de los más
bruscos ángulos de la historia de España en el tortuoso siglo presente.

¡Espectáculo conmovedor! La regia falúa, cuyo timón gobernaba el
almirante Valdés, uno de los más gloriosos marinos de Trafalgar, se
acercaba al muelle. En ella venía toda la familia real, la Monarquía
histórica secuestrada por el liberalismo. La conciliación ideada por
cabezas insensatas era imposible, y aquellos regios rehenes que la
Nación había tomado eran devueltos al absolutismo, contra el cual no
podían prevalecer aún los infiernos de la demagogia. En una lancha
volvían del purgatorio constitucional las ánimas angustiadas del Rey y
los Príncipes.

Mientras el victorioso despotismo recobraba sus personas sagradas, allá
lejos sobre la gloriosa peña inundada de luz y ceñida por coronas de
blancas olas, los pobres pensadores desesperados, los utopistas sin
ilusiones, los desengañados patricios lloraban sus errores, y buscando
hospitalidad en naves extranjeras, se disponían a huir para siempre de
la patria a quien no habían podido convencer.

Así acaban los esfuerzos superiores a la energía humana, las luchas
imposibles con monstruos potentes de terribles brazos, y que hunden en
el suelo sus patas para estar más seguros, como hunde sus raíces el
árbol. Tal era la contienda con el absolutismo. Querían vencerle
cortándole las ramas, y él retoñaba con más fuerza. Querían ahogarle, y
regándole daban jugo a sus raíces. ¡A vosotros, oh venideros días del
siglo, tocaba atacarlo en lo hondo, arrancándolo de cuajo!\ldots{} Pero
advierto que estoy hablando la jerga liberal. ¡Qué horror! Verdad es que
escribo veinte años después de aquellos sucesos; que ya soy vieja, y que
a los viejos como a los sabios se les permite mudar de parecer.

Fernando puso el pie en tierra. Dicen que al verse en suelo firme
dirigió a Valdés una mirada terrible, una mirada que era un programa
político, el programa de la venganza. Yo no lo vi; pero debió de ser
cierto, porque me lo dijo quien estaba muy cerca. Lo que sí puedo
asegurar es que Angulema hincando en tierra la rodilla besó la mano al
Rey, que luego se abrazaron todos, que D. Víctor Sáez lloraba como un
simple, y que los vivas y las exclamaciones de entusiasmo me volvieron
loca. Los franceses gritaban, los españoles gritaban también, celebrando
la feliz resurrección de la Monarquía tradicional y la miserable muerte
del impío constitucionalismo. El glorioso imperio de las caenas había
empezado. Ya se podía decir con toda el alma:---¡Viva el Rey absoluto!
¡Muera la Nación!

\hypertarget{xxxvi}{%
\chapter{XXXVI}\label{xxxvi}}

Faltaba la solución mía. Mi corazón estaba como el reo cuya sentencia no
se ha escrito todavía. El 1.º de Octubre por la tarde y el día 2 hice
diligencias sin fruto, no siéndome posible ver a Sáez ni a Montguyon, a
quien envié frecuentes y apremiantes recados. Ninguna noticia pude
adquirir tampoco de los prisioneros. Creo que me hubiera repetido el
ataque cerebral que padecí en Sevilla, si en el momento de mi mayor
desesperación no apareciese mi generoso galán francés a devolverme la
vida. Estaba pálido y parecía muy agitado.

---Vengo de Cádiz---me dijo.---Dispénseme usted si no he podido servirla
más pronto.

---¿Y qué hay?---pregunté con la vida toda en suspenso.

---Deme usted su mano---dijo Montguyon ceremoniosamente.

Se la di y la besó con amor.

---Ahora, señora, todo ha acabado entre nosotros. Mi deber está
cumplido, y mi deber es perdonar, pagando las ofensas con beneficios.

Yo me sentía muy conmovida y no pude decirle nada.

---Ni un momento he dudado de su nobleza e hidalguía---indiqué con
acento de pura verdad.---A veces tropezamos en la vida con el bien y
pasamos sin verlo. Señor Conde, mi gratitud será eterna.

---No quiero gratitud---díjome con mucha tristeza.---Es un sentimiento
que no me gusta recibido, sino dado. Deseo tan sólo un recuerdo bueno y
constante.

---¡Y una amistad entrañable, una estimación profunda!---exclamé
derramando lágrimas.

---Todo está hecho.

---¿Conforme a mi deseo\ldots? ¡Bendito sea el momento en que nos
conocimos!

---Señora, su prisionero de usted está sano y salvo a bordo de la
corbeta Tisbe que parte esta tarde para Gibraltar.

---¿Y cómo?\ldots{}

---Por sus antecedentes debía ser condenado a muerte. Otros menos
criminales subirán al cadalso, si no se escapan a tiempo. Yo le saqué
anoche furtivamente de los Dominicos y le embarqué esta mañana. Ya no
corre peligro alguno. Está bajo la salvaguardia del noble pabellón
inglés.

---¡Oh, gracias, gracias!

---Además del servicio que a usted presto, creo cumplir un deber de
conciencia arrancando una víctima a los feroces Ministros del Rey de
España.

---¿Pues qué---pregunté con asombro,---¿Su Majestad no ha ofrecido en su
Manifiesto de Cádiz perdonar a todo el mundo?

---¡Palabras de Rey prisionero! Las palabras del déspota libre son las
que rigen ahora. Su Majestad ha promulgado otro decreto que es la negra
bandera de las proscripciones, un programa de sangre y exterminio.
Innumerables personas han sido condenadas a muerte.

---Esto es una infamia\ldots{} pero en fin, ¿él está en salvo\ldots?

---En salvo.

---Y sabe que me lo debe a mí\ldots{} sabe que yo\ldots{} ¡Oh!, señor
Conde, no extrañe usted mi egoísmo. Estoy loca de alegría, y puedo
repetir con toda mi alma: «ahora sí que no se me puede escapar.»

---Sabe que a usted lo debe todo, y espera abrazarla pronto.

---¿Cómo?

---Muy fácilmente. Comprendiendo que usted desea ir en su compañía, he
pedido otro pasaporte para D.ª Jenara de Baraona.

---De modo que yo\ldots{}

---Puede embarcarse usted esta tarde antes de las cuatro a bordo de la
Tisbe.

---¿Es verdad lo que oigo?

---Aquí está la orden firmada por el almirante inglés. Me la ha dado
juntamente con las que ponen en salvo a los ex-regentes Císcar y Valdés,
impíamente condenados a muerte por el Rey.

---¡Oh\ldots{} soy feliz, y todo lo debo a usted!\ldots{} ¡Qué admirable
conducta!

Sin poder contenerme, caí de rodillas, y con mis lágrimas bañé las
generosas manos de aquel hombre.

---Así castigo yo---me dijo levantándome.---Prepárese usted. A las tres
y media vengo a buscarla para conducirla a bordo del bote francés que me
han facilitado dos guardias marinos, parientes míos.

El Conde se retiró recomendándome otra vez que estuviera pronta a las
tres y media. Era la una.

Ocupeme con febril presteza de preparar mi viaje. Estaba resuelta a
abandonar todo lo que no nos fuera fácil llevar. Mariana y yo trabajamos
como locas, sin darnos un segundo de reposo.

La felicidad se desbordaba en mi alma. Me reía sola\ldots{} Pero ¡ay!,
una idea triste conturbó de súbito mi mente. Acordeme de la pobre
huérfana viajera, y esto produjo en mi espíritu una detención dolorosa
en su raudo y atrevido vuelo\ldots{} Pero al mismo tiempo sentía que los
rencores huían de mi corazón siendo reemplazados por sentimientos dulces
y expansivos, los únicos dignos de la privilegiada alma de la mujer.

---Perdono a todo el mundo---dije para mí.---Reconozco que hice mal en
engañar a aquella pobre muchacha\ldots{} Todavía le estará
buscando\ldots{} Pero yo también le he buscado, yo también he padecido
horriblemente\ldots{} ¡Oh! ¡Dios mío! Al fin me das respiro, al fin me
das la felicidad que tanto he buscado y que no pude obtener a causa sin
duda de mis atroces faltas\ldots{} La felicidad hace buenos a los malos,
y yo seré buena, seré siempre buena\ldots{} Esta tarde, cuando le vea,
le pediré perdón por lo que hice con su hermana\ldots{} ¡Oh!, ahora me
acuerdo de la marquesa de Falfán y torno a ponerme furiosa\ldots{} No,
eso sí que no puede perdonarse, ¡no!\ldots{} Tendrá que darme cuenta de
su vil conducta\ldots{} Pero al fin le perdonaré. ¡Es tan dulce
perdonar!\ldots{} Bendito sea Dios que nos hace felices para que seamos
buenos.

Esto y otras cosas seguía pensando, sin cesar de trabajar en el arreglo
de mi equipaje. Miraba a todas horas el reloj que era también de cucú,
como el de aquella horrible noche de Sevilla; pero el pájaro de Puerto
Real me era simpático y sus saluditos y su canto regocijaban mi
espíritu.

Dieron las tres. Una mano brutal golpeó mi puerta. No había dado yo la
orden de pasar adelante cuando se presentaron cuatro hombres, dos
paisanos y dos militares. Uno de los paisanos llevaba bastón de policía.
Avanzó hacia mí. ¡Visión horrible!\ldots{} Yo había visto al tal en
alguna parte. ¿Dónde? En Benabarre.

Aquel hombre me dijo groseramente:

---Señora D.ª Jenara de Baraona, dese usted presa.

En el primer instante no contesté, porque la estupefacción me lo
impedía. Después, rugiendo más bien que hablando, exclamé:

---¡Yo presa, yo!\ldots{} ¿Quién lo manda?

---De orden del excelentísimo Sr.~D. Víctor Sáez, Ministro universal de
Su Majestad.

---¡Vil! ¡Tan vil tú como Sáez!---grité.

Yo no era mujer, era una leona.

Al ver que se me acercaron dos soldados y asieron mis brazos con sus
manos de hierro, corrí por la estancia. No buscaba mi salvación en
cobarde fuga; buscaba un cuchillo, un hacha, un arma cualquiera\ldots{}
Comprendía el asesinato. Mi furor no tenía comparación con ningún furor
de hombre. Era furor de mujer. No encontré ninguna arma. ¡Dios vengador!
Si la encontrara, aunque fuera un tenedor, creo que habría matado a los
cuatro. Un candelabro vino a mis manos; tomelo y al instante la cabeza
de uno de ellos se rajó\ldots{} ¡Sangre! ¡Yo quería sangre!

Pero me atenazaron con sus salvajes brazos\ldots{} ¡Presa,
presa!\ldots{} Todos mis afanes, todos mis sentimientos, todos mis
deseos se condensaban en uno solo: tener delante a D. Víctor Sáez para
lanzarme sobre él, y con mis dedos teñidos de sangre, sacarle los ojos.

No pudiendo hundir mis dedos en ajenos ojos, los volví contra los
míos\ldots{} clavelos en mi cabeza, intentando agujerearme el cráneo y
sacarme los sesos. Mi aliento era fuego puro.

Lleváronme\ldots{} ¿qué sé yo a dónde? Por el camino\ldots{} ¡oh Satán
mío!, ¡oh demonio injustamente arrojado del Paraíso!\ldots{} sentí el
disparo de la corbeta inglesa al darse a la vela.

\flushright{Madrid, Febrero de 1877.}

~

\bigskip
\bigskip
\begin{center}
\textsc{Fin de los cien mil hijos de san luis}
\end{center}

\end{document}
