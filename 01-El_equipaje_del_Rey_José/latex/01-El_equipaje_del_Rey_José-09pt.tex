\PassOptionsToPackage{unicode=true}{hyperref} % options for packages loaded elsewhere
\PassOptionsToPackage{hyphens}{url}
%
\documentclass[oneside,9pt,spanish,]{extbook} % cjns1989 - 27112019 - added the oneside option: so that the text jumps left & right when reading on a tablet/ereader
\usepackage{lmodern}
\usepackage{amssymb,amsmath}
\usepackage{ifxetex,ifluatex}
\usepackage{fixltx2e} % provides \textsubscript
\ifnum 0\ifxetex 1\fi\ifluatex 1\fi=0 % if pdftex
  \usepackage[T1]{fontenc}
  \usepackage[utf8]{inputenc}
  \usepackage{textcomp} % provides euro and other symbols
\else % if luatex or xelatex
  \usepackage{unicode-math}
  \defaultfontfeatures{Ligatures=TeX,Scale=MatchLowercase}
%   \setmainfont[]{EBGaramond-Regular}
    \setmainfont[Numbers={OldStyle,Proportional}]{EBGaramond-Regular}      % cjns1989 - 20191129 - old style numbers 
\fi
% use upquote if available, for straight quotes in verbatim environments
\IfFileExists{upquote.sty}{\usepackage{upquote}}{}
% use microtype if available
\IfFileExists{microtype.sty}{%
\usepackage[]{microtype}
\UseMicrotypeSet[protrusion]{basicmath} % disable protrusion for tt fonts
}{}
\usepackage{hyperref}
\hypersetup{
            pdftitle={El equipaje del rey José},
            pdfauthor={Benito Pérez Galdós},
            pdfborder={0 0 0},
            breaklinks=true}
\urlstyle{same}  % don't use monospace font for urls
\usepackage[papersize={4.80 in, 6.40  in},left=.5 in,right=.5 in]{geometry}
\setlength{\emergencystretch}{3em}  % prevent overfull lines
\providecommand{\tightlist}{%
  \setlength{\itemsep}{0pt}\setlength{\parskip}{0pt}}
\setcounter{secnumdepth}{0}

% set default figure placement to htbp
\makeatletter
\def\fps@figure{htbp}
\makeatother

\usepackage{ragged2e}
\usepackage{epigraph}
\renewcommand{\textflush}{flushepinormal}

\usepackage{indentfirst}

\usepackage{fancyhdr}
\pagestyle{fancy}
\fancyhf{}
\fancyhead[R]{\thepage}
\renewcommand{\headrulewidth}{0pt}
\usepackage{quoting}
\usepackage{ragged2e}

\newlength\mylen
\settowidth\mylen{...................}

\usepackage{stackengine}
\usepackage{graphicx}
\def\asterism{\par\vspace{1em}{\centering\scalebox{.9}{%
  \stackon[-0.6pt]{\bfseries*~*}{\bfseries*}}\par}\vspace{.8em}\par}

 \usepackage{titlesec}
 \titleformat{\chapter}[display]
  {\normalfont\bfseries\filcenter}{}{0pt}{\Large}
 \titleformat{\section}[display]
  {\normalfont\bfseries\filcenter}{}{0pt}{\Large}
 \titleformat{\subsection}[display]
  {\normalfont\bfseries\filcenter}{}{0pt}{\Large}

\setcounter{secnumdepth}{1}
\ifnum 0\ifxetex 1\fi\ifluatex 1\fi=0 % if pdftex
  \usepackage[shorthands=off,main=spanish]{babel}
\else
  % load polyglossia as late as possible as it *could* call bidi if RTL lang (e.g. Hebrew or Arabic)
%   \usepackage{polyglossia}
%   \setmainlanguage[]{spanish}
%   \usepackage[french]{babel} % cjns1989 - 1.43 version of polyglossia on this system does not allow disabling the autospacing feature
\fi

\title{El equipaje del rey José}
\author{Benito Pérez Galdós}
\date{}

\begin{document}
\maketitle

\hypertarget{i}{%
\chapter{I}\label{i}}

El 17 de Marzo de 1813 salieron de palacio algunos coches, seguidos de
numerosa escolta, y bajando por Caballerizas a la puerta de San Vicente,
tomaron el camino de la puerta de Hierro.

---Su Majestad intrusa va al Pardo---dijo don Lino Paniagua en uno de
los corrillos que se formaron al pasar los carruajes y la tropa.

---Todavía no es el tiempo de la bellota, señores---repuso otro, que se
preciaba de no abrir la boca sin regalar al mundo alguna frutecilla
picante y sabrosa del árbol de su ingenio.

---Su Majestad se ha convencido de que no engordará en España, y por ese
camino adelante no parará hasta Francia---indicó un tercero, hombre
forzudo y ordinario que respondía al nombre de Mauro Requejo.

---¡A Francia! Todas las mañanas nos saluda la gente con el estribillo
de que se marchan los franceses aburridos y cansados, y por las noches
nos acostamos con la certidumbre de que los franceses no se aburren, ni
se cansan, ni tampoco se van.

---¡Tiene razón el Sr.~D. Lino Paniagua!---exclamó otro personaje que se
distinguía de los demás individuos del grupo por el deslumbrante verdor
de sus anteojos y un extraño modo de reír, más propiamente comparable a
visajes de cuadrumano que a muecas de racional.---¡Tiene razón! Hace
cinco años no se oye más que esto: «Se van sin remedio: ya no pueden
sostenerse un día más: el lord dará buena cuenta de todos ellos dentro
del mes que viene\ldots» Y así corren los meses y los años: la gente
muere, el pan sube, los pleitos merman, el dinero se acaba y los
franceses no se van sino para volver. Cuatro veces hemos visto salir al
Sr.~Pepe y cuatro veces le hemos visto entrar con más bríos. ¿Se
acuerdan Vds. de la batalla de Bailén? Pues todos decían: «Gracias a
Dios que se acabó esto. No ha quedado un francés para simiente de
rábanos». ¡Ay! no pasaron muchos meses, sin que les viéramos otra vez
mandados por el Emperador en persona. Al cabo de cinco años se ha
repetido la fiesta. Diose una batalla en Salamanca y aquí de mis bocas
de oro: «¡Ya se acabó todo!\ldots{} ¡Gracias a Dios!\ldots{} Viva el
lord\ldots» Los franceses salen por un lado y los ingleses entran por
otro. Pero esto parece escenario de un teatro: el lord se va por la
derecha y José se nos cuela por la izquierda\ldots{} Señores, no puedo
olvidar las acotaciones de las comedias, que dicen hace que se va y se
queda\ldots{} A mí que soy perro viejo y tengo sobre mi alma cristiana
cuatro dedos de enjundia de marrullería, no se me emboba con estas
entradas y salidas.

---El Sr.~licenciado Lobo---dijo D. Narciso Pluma que a la sazón se
encontraba también allí,---se halla tan bien en su escribanía de cámara,
que no quisiera le molestase el ruido de las tropas, ni el estrépito de
la guerra. Al fin y al cabo, los destinos dados por Murat no han de ser
eternos.

---Ya os veo venir, embrollones; os entiendo farsantes; os conozco,
trapisondistas---repuso Lobo disimulando su enojo.---¿Quieren hacerme
pasar por afrancesado?\ldots{} Parece que corren vientos anglicanos y
wellingtonianos\ldots{}

---Puede ser.

---Señores, demos una vuelta por los Pozos de Nieve a ver si clarean las
casacas rojas del lado de Fuencarral y Alcobendas.

---¿Por qué no? El ejército aliado parece que viene hacia acá. Pero en
suma, señores ¿a dónde va esta gente? ¿Qué tinajas atraen con su
olorcillo a nuestro intruso mosquito?

---Yo digo que no pasa del Pardo.

---Y yo que antes dejará de catarlo que quitarse el polvo de los zapatos
mientras no llegue a la raya de Francia.

---Por allí viene el reverendo Salmón que nos dirá la verdad, pues este
fraile de la Merced gusta de cucharetear con todo el mundo, y aquí cojo
un vocablo, allá pesco una sílaba, ello es que todo lo sabe.

---Bien venido sea el padre Salmón---dijo Requejo adelantándose a
saludar al venerable mercenario que en la noble compañía del marqués de
Porreño tomaba de la Virgen del Puerto.

---¿Y qué nuevas tienen Vds., señores míos?---preguntó el buen fraile
limpiando el sudor de su rostro, pues según se fatigaba al subir la
empinada cuesta de San Vicente, parecía que se dejaba la mitad de sus
rollizas carnes en el camino.

---Como vuestra Paternidad no nos diga algo\ldots{}

---El aparato de fuerza que lleva el Rey, y la muchedumbre de coches en
que le acompaña toda su servidumbre francesa y española---dijo con
gravedad el marqués de Porreño,---prueban que el viaje será largo.

---Estamos a 17 de Marzo\ldots{} pasado mañana son los días de D. Pepito
---indicó el fraile frotándose las manos.---Quiere celebrarlo en el
Escorial.

---¿En Marzo? Eso es hablar en mojigato---dijo Pluma señalando con
picaresca malignidad a un anciano astroso y taciturno que hasta entonces
no había desplegado sus sibilíticos labios.---El Sr.~Canencia que está
presente le enseñará a Vd. a hablar en jacobino. No se dice Marzo, sino
Ventoso, víspera de Germinal y antevíspera de Floreal.

Todos se rieron a costa del abatido D. Bartolomé Canencia, que habló de
esta manera:

---En mi escuela se atiende a los hechos no a las palabras, factis non
verbis.

---Estamos en Marzo---afirmó Lobo,---pero ahora nos ocupamos de nuestro
Rey postizo, y ya se sabe que está siempre en Vendimiario.

---Veo que será preciso buscar las noticias en otra parte---dijo con
impaciencia Paniagua.---El padre Salmón no está hoy de vena para contar,
y D. Bartolomé Canencia, que conoce todos los pasos de los franceses
como los saltos de las pulgas dentro de su camisa, no nos quiere decir
nada, sin duda por no vender a sus amigos.

---¡Mis amigos, los franceses!---exclamó Canencia turbándose como
jovenzuelo tímido, a quien se descubre un secreto amoroso.---¿Soy acaso
hombre que se entusiasma con las victorias militares de Juan y de Pedro?
¡Batallas! ¡Ejércitos! ¡Napoleón! ¡Lord Wellington! ¡Qué basura! Soy
partidario del género humano, señores. Odio las guerras, destructoras de
la convención social, y aguardo el día de la emancipación de los
pueblos. Sé que me calumnian; sé que algunos se atreven a sostener que
estuve en Salamanca en una sociedad masónica\ldots{} ¿Por ventura estas
mis venerables canas y esta entereza filosófica que debo a mis estudios
son a propósito para degradarse en logias y aquelarres\ldots? Pero basta
que me hayan dado ese miserable destinillo en la contaduría del Noveno
para que se me crea ligado en cuerpo y alma a los Bonapartes, señores, a
los hijos de doña Leticia, que hoy dominan el mundo con la
espada\ldots{} ¡Como si la espada fuera otra cosa que un pedazo de
acero, una herramienta brutal, una lanceta inerte y punzante que sólo
sirve para sangrar a los pueblos!\ldots{} Y entre tanto las
ideas\ldots{} Volved los ojos a todos lados y decidme, ¿dónde están las
ideas?

Las risas impidieron a Canencia seguir adelante en su comenzado
discurso. Salmón le quitó la palabra de la boca, para decir:

---Mala pascua me dé Dios y sea la primera que viniere, si a este D.

Bartolomé no le cambian pronto su plaza de la contaduría del Noveno por
una jaulita en el Nuncio de Toledo\ldots{} En suma nada nos ha dicho del
viaje del rey. Lo que yo aseguro es que ayer nada se sabía en palacio de
tal viaje\ldots{}

---Por allí viene quien nos ha de sacar de dudas---dijo Pluma señalando
hacia Caballerizas.

Todos los del corrillo fijaron la atención en un joven bien parecido, de
rostro alegre y franco que precipitadamente bajaba en dirección a San
Gil. Vestía el uniforme de la guardia española creada por José en Enero
de 1809, y a la cual pertenecían buen número de compatriotas nuestros
con todos o casi todos los suizos y valones de los antiguos cuerpos
extranjeros.

---¡Eh, Salvadorcillo Monsalud, Salvadorcillo Monsalud!---gritó el
licenciado Lobo, llamando al mozo del uniforme.

---Es sobrino de Andrés Monsalud, el que apalearon en Salamanca---indicó
con malicia Requejo.---El Sr.~Canencia puede dar noticia de la batalla
de los Arapiles y de los palos de Babilafuente.

---Señores patriotas, buenos días---dijo el joven guardia acercándose al
corrillo y saludando a todos con festivo semblante.

---¿Qué ocurre, discreto amigo, aunque jurado?---le preguntó Salmón
posando su manto en el hombro del mancebo.---¿A dónde va por esos
caminos el Emperador de las Tinajas?

---A Valladolid---repuso el militar.

---¡A Valladolid!---exclamaron todos.---¡Ya lo presumía yo!

---Por allí están la Nava, Rueda, la Seca, Mojados y demás cepas\ldots{}

---¿Con que a Valladolid?

---No faltarán batallas\ldots---indicó el joven con énfasis.---Napoleón
ha mandado un recado a su hermano, diciéndole que salga a campaña.

---¿Un recadito?

---Y nosotros salimos también\ldots{} Y con nosotros los ministros, y
con los ministros los empleados, y con los empleados\ldots{}

---Con los empleados los empleos---añadió Lobo.---Eso será bueno.

---En palacio están empaquetando a toda prisa cuadros y
alhajas---prosiguió Salvador con alborozo y orgullo, propios de la
juventud al verse portadora de nuevas estupendas.---Ayer embaulamos
juntamente con la batería de cocina una tabla pintorreada que llaman el
Pasmo de Sicilia\ldots{} Nos llevamos hasta los clavos\ldots{} Dentro de
pocos días se van a embargar todos los coches y carros de la villa, y
aún no bastará.

---¡Todos los carros! Pero esta gente nos va a dejar sin un alfiler para
atrabarnos las chorreras.

---¿Acaso vinieron a otra cosa? Pues qué---afirmó Salmón,---¿cree Vd.
que esa gente ha sabido lo que es pan antes de venir a España?

---Y ahora, señores---dijo el militarejo,---harán Vds. bien en marcharse
cada uno a su casa de dos en dos, porque la policía no gusta de ver
grupos en los alrededores de palacio.

Esta advertencia produjo rápidos efectos: deshízose el grupo, y por
parejas se alejaron en direcciones diversas los esclarecidos varones,
marchando cuál a su oficina, cuál a su tienda, este a la escribanía,
aquel al convento, quién a la tertulia de la botica, quién a los
estrados de las damas y a las reuniones de la gente tónica, afanosos
todos de transmitir las noticias recibidas, que de calle en calle, de
sala en sala, y de boca en boca iban desfigurándose y abultándose hasta
el punto de que no las conocería el mismo que las lanzó a los vaivenes y
agitaciones del mundo.

¡Y entonces no había periódicos!

José Bonaparte había salido en efecto para Valladolid, obedeciendo a su
amo y hermano que le mandaba ponerse al frente del ejército, mientras
él, no escarmentado con la desastrosa campaña de la Moscowa, se disponía
a emprender otra nueva en Alemania contra la sexta coalición.

Cuando el coche, pasado el arco de San Vicente, torció a la derecha en
dirección a la Puerta de Hierro, Su Majestad, que hablaba con el general
Jourdan, dejó a este con la palabra en suspenso, y se asomó por la
portezuela para contemplar el real palacio que quedaba detrás, sentado
en los bordes de la villa, con un pie arriba y otro abajo, destacando su
enorme cuerpo blanco sobre las rampas de ladrillo que le sirven de trono
y sobre la verdura de los árboles que le sirven de alfombra. José
Bonaparte dirigió al edificio una mirada en la cual difícilmente podrían
conocerse los sentimientos de su corazón. Aquel abandonado albergue que
veía Su Majestad tras sí, ¿era una mansión risueña, de la cual no podía
alejarse sin pena, o por el contrario, cueva horrorosa en cuyo recinto
no había sino cautiverio y tristeza? ¿Era grata al intruso la idea del
regreso, o se complacía su ánimo con el pensamiento de perder de vista
para siempre la enorme casa blanca y las rojas murallas y el jardín
rastrero entre cuyo follaje levanta el abollado sombrerete de su techo,
la ermita de la Virgen del Puerto?\ldots{}

Napoleón el Chico, después del triste mirar, recostose taciturno en el
fondo del coche, mas no oyeron sus cortesanos ningún suspiro como el que
en parecido caso regaló a la historia Boabdil el de Granada. Reanudose
la conversación entre José y el mariscal Jourdan. Madrid y su palacio y
su polvo y su claro cielo y su aire sutil no fueron ya para el hermano
de Bonaparte más que un recuerdo.

\hypertarget{ii}{%
\chapter{II}\label{ii}}

Salvadorcillo Monsalud era un joven de veintiún años, de estatura
mediana y cuerpo airoso y flexible. Su rostro moreno asemejábase un poco
al semblante convencional con que los pintores representan la
interesante persona de San Juan Evangelista, barbilampiño y un poco
calenturiento, con singular expresión de ansiedad inmensa o de
aspiración insaciable en los grandes ojos negros. Grave seriedad
sentimental se desprendía de su persona, de su voz y de su porte;
cautivaba a todos por su bondad, y a las muchachas por sus modales
corteses y su agraciada delicadeza no adquirida con la educación, pues
había nacido en cuna muy humilde. Era como el Evangelista, algo tímido y
muy circunspecto, lo cual no resultaba útil en este siglo, ni aun cuando
principiaba. Con su traje de guardia española, Monsalud estaba muy
gallardo; pero sin aquel espantable continente marcial que caracteriza a
los militares de afición: era su figura la de un soldado en yema o
campeón verde que aún no se había endurecido al sol de los combates, ni
acorazado con la provocativa soberbia y fanfarronería de una larga vida
de cuarteles.

Este joven tenía por tío a Andrés Monsalud, que vivía en la Cava-Baja, y
por amigo íntimo y confidente a un compatriota llamado Juan Bragas, que
con él viniera poco antes de la Puebla de Arganzón a buscar fortuna.
Había emigrado Salvador por razones que se conocerán en el transcurso de
esta historia, y que no eran ciertamente alegres. Indeciso primero sobre
la carrera a que debía dedicarse, y no sintiéndose con vocación para el
comercio ni para la curia ni para la Iglesia, entrose de rondón por la
puerta del militarismo, ancha y abierta siempre, y que tiene la ventaja
sobre las demás puertas, incluso la Otomana, de llevar rápidamente a
todas partes. Diérale su buena madre al partir una cantidad que podía
parecer considerable en el condado de Treviño, pero que en Madrid era de
esas que se disuelven pronto en la inmensidad de la vida, como grano de
sal en tinaja de agua. Viéndose pues, el joven sin nada blanco ni
amarillo en sus arcas, y no teniendo más tesoro que los sabios consejos
de su insigne tío D. Andrés Monsalud, resolvió aprovecharse de este
caudal, que a todas horas se le vertía en los oídos, ya en forma de
reprimenda, ya con color de amonestación. No por entusiasmo, no por
falta de patriotismo, no por bélico ardor, sino por necesidad, entró
Salvador en uno de los regimientos españoles que servían malamente a
José, y a los cuales llamábamos entonces jurados. Bien pronto le dieron
las charreteras de sargento.

Eran los individuos de estos cuerpos muy aborrecidos y escarnecidos en
Madrid, por servir al enemigo intruso, tirano y ladrón de la patria;
pero Monsalud no se preocupaba de esta falta de estimación, que al
recaer sobre la infame bandera, alcanzaba también a su humilde persona.
Aunque el joven tenía ideas y no pocas, si bien revueltas y confusas y
desordenadas, aún no poseía las que comúnmente se llaman ideas
políticas, es decir, no había llegado, a pesar del vehemente ardor de la
generación de entonces, al convencimiento profundo de que la solución
nacional fuese mejor o peor que la extranjera. No faltaba ciertamente en
su corazón el sentimiento de la patria; pero estaba ahogado por el
precoz desarrollo de otro sentimiento más concreto, más individual, más
propio de su edad y de su temple, el amor. Está escrito, que en ciertos
casos, tal vez siempre, el rostro de una mujer tenga mayores dimensiones
y ocupe dentro del universo más grande espacio que las inmensidades
materiales y morales de la patria. Por esta causa, por este aparente
absurdo, Fernando el Deseado y José Bonaparte eran a los ojos de
Monsalud dos figuras lejanas y pequeñitas, que apenas se parecían en las
nieblas del cerrado horizonte.

Quién era la persona que así llenaba la fantasía y ocupaba las potencias
todas del alma de este joven, sabralo el lector más adelante, cuando con
sus propios ojos la vea y oiga su vocecita y conozca su historia.
Monsalud estaba solo en Madrid, porque realmente, para él los cien mil
habitantes de la capital, no eran nadie, ni su amigo y su tío eran
tampoco gran cosa. La soledad y la distancia habían ahondado el hoyo de
su pensamiento, dentro del cual tristemente se revolvía, escarbando con
ardor por todos lados sin hallar salida, ni respiro, ni luz.

Hemos dicho que tenía un amigo, sí, Juan Bragas, joven nacido como
Monsalud en el lugar de Pipaón, y que poseedor de mayores recursos y
valimiento había resistido a las primeras escaseces de la vida
cortesana, pescando al fin por lo muy pedigüeño y sumiso, una pluma de
ganso en las covachuelas. Juan Bragas era, pues, covachuelista, es
decir, palote árido y enteco en el cual debía injertarse después la
vigorosa rama del funcionario público. Su carácter difería mucho del de
Monsalud, y, sin embargo se juntaban ambos jóvenes con sumo gusto para
charlar y referirse sus respectivas desventuradas aventuras.

Juan Bragas carecía por completo de imaginación y de sensibilidad fina:
pero sabía poner las cosas en su sitio, y tenía el mejor ojo del mundo
para ver todos los objetos en su tamaño real: poseía, en suma, aquel
poderoso instinto aritmético que a ciertas organizaciones, quizás las
más influyentes hoy, les sirve para reducir a cantidad o a tamaño, mejor
dicho, a una forma visible y fácilmente apreciable todos los hechos de
la vida en lo moral y en lo físico. Bragas no se equivocaba nunca: tenía
en sus juicios la infalibilidad de las matemáticas. Monsalud era una
equivocación perpetua: llevaba infiltrado en su naturaleza el error
constante y todas las deslumbradoras mentiras de la poesía.

A pesar de esto, no reñían nunca y se querían de veras. Quizás ha
dispuesto Dios que el mundo se componga de un Monsalud y de un Bragas.
¡Oh admirable armonía y concordia sublime! Las cuerdas del arpa no
exhalarían, no, su armoniosa voz, si no existiera una caja vacía y seca,
una especie de ataúd oscuro que retumbase bajo ellas, y vibrase
agrandando los sones en su desnuda concavidad que podría servir de
despensa.

Cuando Monsalud estaba libre del servicio iba a buscar a Bragas, el cual
limpiaba una tras otra las amarillentas plumas, guardándolas en el cajón
con tanto cuidado como guarda un cirujano sus instrumentos, se quitaba
después los manguitos negros, se desperezaba, y tomando con la diestra
mano el sombrero, y despidiéndose con la zurda de D. Gil Carrascosa,
jefe de la oficina, salía a la calle. Ambos jóvenes dirigían sus pasos
por lugares no muy concurridos, bajando frecuentemente al campo del
Moro, a la Virgen del Puerto, o bien se lanzaban intrépidos a las ondas
de polvo del cerrillo de San Blas o de la vuelta exterior del Retiro.

Un día, que debió de ser allá por los últimos de Mayo de 1813, Bragas y
Monsalud hablaron de esta manera.

---Amigo Juan Bragas, estoy de enhorabuena porque al fin voy a dejar
este maldito pueblo que aborrezco. Los franceses se retiran mañana y yo
con ellos.

---¿A Francia?

---O por el camino de Francia, al menos---añadió Monsalud,---con lo cual
dicho se está que pasaré por la Puebla de Arganzón, nuestra querida
villa. Anímate, Juan\ldots{} Ya me parece que estoy entrando por la
calle real; que me acerco a mi casa sin que mi madre lo sospeche; ya me
parece que llego, empujo la puerta, y me presento dando gritos y
porrazos. A mi madre se le cae la calceta de la mano, corre a echarse en
mis brazos, y la aguja de media que lleva sobre la oreja, se me clava en
la frente\ldots{} El corazón me baila en el pecho, amigo Bragas, cuando
tales cosas pienso.

---De veras te digo que pareces cómico---dijo Bragas riendo.---¡Qué bien
sabes fingir y representar una cosa que no es verdad!

---Y luego---añadió Monsalud,---saldré de mi casa, y paso a paso iré
junto a Nuestra Señora de la Asunción, a cuya plazoleta caen las
ventanas de Generosa, y arrojaré una chinita a los vidrios\ldots{}

---Para que se asome Genara con su pañuelo encarnado sobre los
hombros\ldots{} ¡La pícara qué guapa es!---afirmó Bragas.---Me parece
que la estoy mirando, cuando bailaba contigo en casa del maestro
Rondaña. Salvador, ¿te acuerdas de aquel lunarcito que tiene sobre el
rincón derecho de la boca? ¡Santa Virgen, que rinconcito!

---Para retirarse a él y decir: «ya no quiero más mundo».

---¿Pues y aquel modo de mirar, y aquel reconcomio de ángeles divinos,
cuando se menea, o alza los hombros, o le da a uno las buenas tardes?
Paréceme que la oigo: «Buenas tardes, Braguitas, ¿has visto en las eras
a Salvador Monsalud?»

---¡Ay, amigo!---exclamó el joven soldado dando un suspiro.---¡Cuando
uno piensa que ha tenido todo eso y todo eso ha perdido!\ldots{}

---¡Miren el Juan Lanas! Valiente hombre tenemos aquí---dijo el de la
covachuela mofándose de la sensibilidad un tanto exagerada de su
amigo.---Échate a llorar y ponte flaco y amarillo y echa suspiritos al
aire, por una mujer, por un lunar bien puesto encima de una boquirrita.
Mira, Monsalud, si tú eres necio, yo no lo soy. Ya te lo he dicho varias
veces: las mujeres para un rato y nada más. Mucho de que te quiero y te
adoro; pero después\ldots{} puntapié. Eso de llorar y entristecerse y
decir palabrotas y quererse morir por una de tantas es propio de bobos.

---Tú no sabes lo que es el amor, Juan Bragas---dijo el soldado;---o
mejor dicho, crees que viene a ser algo semejante a un plato de
estofado.

---Ni más ni menos. Un plato de estofado repugna después de haber
comido\ldots{} Por consiguiente, no te acuerdes más de la Generosa, que
a buen seguro ella se acuerda de ti como de las nubes de antaño. Los
paisanos que llegaron el otro día me dijeron que se iba a casar con el
hijo de D. Fernando Garrote, el cual tiene más dinero que pesáis tú y
Generosa juntos.

---¡Con el hijo de D. Fernando Garrote, con Carlitos Garrote!---murmuró
Monsalud palideciendo.---Juan Bragas, si vuelves a decir eso delante de
mí, te cojo y\ldots{} vamos, te cojo y te ahorco de un árbol.

---¡Piedad, señor mío!---dijo Bragas deteniéndose ante su amigo y
haciendo grotescos gestos.---Está Vd. enamorado o lo que es lo mismo,
imbécil, y los imbéciles suelen ser graciosos.

---Bragas, eres una bestia---dijo el soldado.---Para ti no hay más vida
que el forraje que te echan todos los días en casa de tu patrón D. Mauro
Requejo. Siento tener por amigo una bestia; pero en fin eres un buen
muchacho: tu solo defecto es que coceas de vez en cuando.

---Pero jamás he llevado sobre mí la albarda del enamoramiento. Ven acá,
hombre sin seso, ¿de quién estás enamorado? De Generosa. ¿La ves acaso?
¿No está a cien leguas de donde tú estás? ¿No te dijo su abuelo que
jamás casarías con ella por ser tú un triste pelón y tener tus arcas
rasas, lisas y mondas como fondo de mortero de piedra? De modo que estás
queriendo a una sombra, a un imposible, a una ilusión, a una telaraña;
justo, esa es la palabra, a una telaraña.

---Juan---repuso Monsalud,---al oírte me confirmo en que eres un saco de
carne, con dos agujeros que llaman ojos, para ver lo que se le pone
delante, y boca y barriga para comer y llenarse de bazofia todos los
días. Cada hombre tiene su destino en el mundo: el tuyo ya sabemos cuál
es.

---Y el tuyo lo veo yo clarito también: holgazanear, mirar a las
estrellas cuando las hay, taconear por las calles para llamar la
atención de las costureras que pasan, no tener qué comer y ser toda la
vida un señoritico cañihueco y hambrón.

---Pues mira, a veces se me ha ocurrido, amigo Bragas, que yo sería
mucho más feliz si fuese como tú, es decir, un saco con sentidos. Pienso
muchas veces en mi porvenir y digo: «Quién sabe, ¡vive Dios! si esto que
pienso será una mentira, una cosa vana y disparatada». Todos los jóvenes
hacemos nuestros cálculos para lo porvenir, Juan, y los míos son un poco
extraños y fuera de lo común. A mí se me ha puesto en la cabeza que para
levantarse todos los días, comer, dormir la siesta, pasear, cenar y
meterse en la cama, no valía la pena de que hubiésemos nacido. Más vale
ser un puñado de polvo que los vientos se llevan y desparraman por todas
partes. O yo no he de valer nada o he de vivir de otra manera. Soy un
ignorante; sé poco de las cosas del mundo; mas por lo poco que sé,
comprendo que hay muchos trabajos admirables en que el hombre se puede
emplear. Digan lo que quieran, el mundo no marcha bien.

---Pues yo creo que marcha admirablemente---dijo Bragas
riendo.---¿También quieres enmendar la obra de Dios?

---No digo tal: quiero decir que esto no va bien; no sé si me explico.
Si tú tuvieras siquiera un pedazo de alma, tendrías las inquietudes y
los deseos que yo tengo, y estarías enamorado como yo lo estoy. Es un
padecimiento; pero no puedes formarte idea de que se te quita este
padecimiento, sino haciéndote cargo de que estás muerto. Vivir curado
del mal de amores es cosa que la mente no puede concebir, Braguitas.

---Dime, Salvador---indicó el covachuelo con ademán festivo,---¿piensas
seguir así?\ldots{} Te juro que vas a hacer bonitísima carrera. Por ese
camino de los amorosos sufrimientos y del suspirar y escupir sangre se
va a general en poco tiempo.

---¿Y quién te ha dicho que yo quiero ser general en dos
palotadas?\ldots{} Lo que digo es que yo seré alguna cosa que meta
ruido.

---Siendo militar y tambor, en efecto puedes meter mucho ruido.

---Allá lo veremos\ldots{} ¿Y tú qué piensas ser?

---¿Yo? Dificilillo es anunciarlo desde ahora, Sr.~Monsalud; pero no me
quedaré de monago. Sepa usía que en el fondo de mi baúl tengo siete
duros.

---¿Y qué haces que no pones un buen comercio o un segundo Banco de San
Carlos?

---Por poco se empieza. Yo sacaré el pie del lodo, Sr.~Monsalud. Y no me
pidas prestados los siete duros, porque más fácil será que saques un
alma del Infierno que sacar mis soles del fondo del arca donde los
guardo. Como no me he de enamorar, ni siento comezón de echarme
vinagrillo de los Siete Ladrones en el pañuelo, allí se estarán hasta
que vayan otros tantos a hacerles compañía. Conque perdone por Dios,
hermano, que no tenemos suelto.

---Bien sabes que nunca te he pedido nada.

---Pero pudiera ocurrírsete cualquier día, Salvador. Tú vas sacando
malas mañas\ldots{} Ahora que te vas al Norte, asistirás a alguna
batalla\ldots{} Como no faltará algún pueblo que entrar a saco, mucho
ojo, amiguito, y mete mano.

---Descuida, soy buen amigo: si después de una batalla, se reparte botín
y me toca algo, te lo mandaré.

---Hombre, no es mala idea\ldots{} Pero si te tocase alguna herida o
descalabradura, puedes quedarte con ella.

---Oye, Juanillo---replicó vivamente Monsalud,---¿no dices que tu mayor
gusto consistiría en ser ministro del Rey para tener mucho dinero y
hacer mucho bien y llenarte de gloria y morir honrado y bendecido?

---Sí.

---Pues te guardas el dinero, ¿eh?\ldots{} y la gloria, la honra y las
bendiciones me las mandas.

\hypertarget{iii}{%
\chapter{III}\label{iii}}

Así pensando y discutiendo, a veces riñendo y regalándose el uno al otro
palabras un poco fuertes; haciendo luego las paces para prometerse
amistad invariable, dieron nuestros dos amigos la vuelta del Retiro, y
cuando tornaban a Madrid por la calle de Alcalá, vieron que discurría de
arriba abajo mucha gente, y que contraviniendo las disposiciones de la
policía francesa, en todas partes se formaban grupos. Pedíanse las
personas unas a otras las noticias arrebatándoselas de la boca y
comentándolas para soltarlas luego desfiguradas. Cuál aseguraba saber
mucho, cuál ignorándolo todo se hacía repetir hasta tres veces la misma
noticia. Todos los madrileños parecían sorprendidos, y los más, alegres.

Al punto pararon mientes Monsalud y Bragas en aquella estupenda novedad
de los corrillos y de la animación que se repetía, a pesar del gobierno,
siempre que llegaban noticias de alguna batalla. Deseosos de conocer la
verdad de lo que ocurría, husmearon en varios grupos, mas no viendo
caras conocidas en ninguno de ellos, no se atrevieron a meter su
cucharada y se contentaron con algunas palabras sueltas. Pero hacia las
Baronesas, creyó Bragas oír la voz de D. Gil Carrascosa, abate antaño, y
por entonces covachuelista en la misma covachuela del covachuelo
mancebo. Acercáronse y vieron que el licenciado Lobo venía a su
encuentro, juntamente con D. Mauro Requejo y el Sr.~D. Bartolomé
Canencia. Fundiéronse todos en el grupo, a punto que Carrascosa decía:

---Mañana salen de Madrid los franceses. Parece que ahora va de veras,
señores patriotas, y que no volverán más. El Rey José está muy apretado
y no puede pasar, según dicen, de la línea del Ebro. Aquí no quedará un
solo francés, ni un solo jurado, ni un solo polizonte, ni un solo
jacobino. Respira, ¡oh patria!

---La verdad---dijo D. Lino Paniagua, que también era de los
presentes---es que Wellington se ha movido.

---Y como también se ha movido el cuarto ejército que manda
Castaños\ldots{} Parece que quieren cerrar a los franceses el paso de
Burgos y Vitoria.

---¡Admirable plan!---exclamó Lobo.---¡Cerrar el paso! nada más claro.
El cuarto ejército estaba en todas partes como perejil mal sembrado.
Castaños en Extremadura con una división, Porlier y Losada en Galicia
con otra, Morillo en Asturias, Mina en Vizcaya. Lord Wellington que
desde Fregeneda ponía su lente en todo, les ha mandado adelantarse. Uno
viene por aquí, otro por allá, con tan admirable concierto y arte como
las piezas de un reloj que ordenadamente van caminando sin estorbarse
una a otra. El francés que con la cholla cargada de vapores viníferos,
se duerme en Valladolid, en Segovia, en Madrid y en Zaragoza, no ve el
nublado, hasta que le cae encima. Se asusta, llama a Farfulla I en su
ayuda, pero Farfulla I después de la campaña de Rusia no está para
fiestas, y héteme al rey José en campaña. Él había dicho como los
castellanos: «Vino puro y ajo crudo, hacen al hombre agudo\ldots» pero
en buena se ha metido\ldots{} ¡Grandes batallas se preparan! Todo esto,
amigos míos, lo barruntaba yo; se necesita no tener un solo grano de sal
en la mollera para comprender que hallándose el lord en Fregeneda, Longa
y Mina en el Norte, Morillo en Asturias, y Carlos España en el Bierzo,
pues\ldots{} yo lo veo claro como el agua.

---Y yo turbio como el cieno---dijo Canencia, con filosófico
desdén.---¡Una batalla más! Rousseau ha dicho que las verdaderas
batallas son las que ganan la sabiduría contra la ignorancia de la
corrompida humanidad.

No tardó en pasar el padre Salmón, que con el padre Ximénez de Azofra y
el marqués de Porreño, regresaba a su convento, y pegándose al grupo
hizo varias preguntas.

---Eso ya lo sabíamos\ldots{} que se va toda la canalla mañana
temprano\ldots{} ¿Pero y de los ejércitos, qué se dice?

---A mí se me figura---dijo con gravedad el marqués de Porreño,---se me
figura\ldots{} es idea mía\ldots{} puede que me equivoque, pero
juraría\ldots{}

---¿Qué?

---Que el lord se ha movido.

---Eso no tiene duda---repuso Lobo dignándose repetir el plan de campaña
con que poco antes había demostrado su perspicacia estratégica.

Y al poco rato partieron en distintas direcciones. Acompañaron al señor
marqués los dos reverendos, y recibidos por la interesante familia de
este, Salmón exclamó:

---¡Gran bomba, señores! El lord se ha movido.

---¡Y mañana salen de aquí todos los franceses!

---¡Benditos sean los designios de la divina Providencia!---dijo la
hermana del marqués.

---¡Wellington se ha movido!---repitió el mercenario, mirando a diestra
y siniestra por ver si se vislumbraban en el horizonte lejanos signos de
soconusco,---y juntamente con Mina y Morillo viene sobre Madrid.

---¡Jesús! ¡Sobre Madrid!

---Así lo han dicho. Parece que da la vuelta por el Duero, que está como
Vd. sabe en Tordesillas. Y como Castaños pasa de Extremadura a Asturias,
con el sétimo cuerpo, digo, con el octavo o con el duodécimo\ldots{} en
junto unos cuatrocientos mil hombres.

Poco después la hija del marqués de Porreño iba a casa de Sanahúja,
donde ya sabían la noticia, gracias a don Lino Paniagua, y decía:

---Lo menos setecientos mil hombres dicen que trae Vellinton.

Conviene advertir que casi todos los españoles pronunciaban el nombre
del general inglés como acabamos de escribirlo. Algunos lo modificaban
diciendo Velliztón, acentuando la última sílaba, lo mismo que decían
Stapletón Cotón; pero esto no hace al caso, y siga nuestro cuento. El
conde de Rumblar, que a la sazón hallábase en casa de Sanahúja, partió
como un rayo, y en la Puerta del Sol topó con José Marchena, a quien
dijo que José iba sobre Fregeneda y que el duque de Ciudad Rodrigo
estaba en Valladolid\ldots{} Poco después D. Narciso Pluma, que esto
oyera y otras muchas estupendas cosas que había oído poco antes, las
revolvió todas, haciendo la más chistosa ensalada que puede imaginarse,
y entró en casa de Porreño, donde sostuvo que se estaba dando una
batalla junto al Duero entre D. Pablo Morillo con doce mil hombres, y el
rey José con setecientos mil\ldots{}

Repitámoslo, sí. ¡Entonces no había periódicos!

\hypertarget{iv}{%
\chapter{IV}\label{iv}}

Cuando se disolvió el grupo los dos jóvenes siguieron su camino.

---Vamos a casa de mi tío---dijo Monsalud,---a ver qué piensa de estas
cosas. Ya anochece; apretemos el paso\ldots{} ¿No te parece que los
habitantes de la villa están un poco alborotados?

---¡Salen los franceses!\ldots{} ¡Un cambio de gobierno!---murmuró
Bragas intranquilo.---Ahora todos los que han sido empleados durante el
gobierno intruso\ldots{}

---A la calle, amigo. ¡Pues no es poca afrenta la que tienen encima.
Haber servido al intruso!\ldots{} ¡Oh, vilipendio!

---Pero yo soy español, muy español. Detesto a los franceses.

---Ahora que se van es muy cómodo decir eso. Yo, Sr.~Juan, no les tengo
rencor. Con ellos he servido, con ellos voy.

---Entonces dirás: «¡Viva Napoleón!»

---No diré ni que viva ni que muera, porque yo no he de matar ni he de
resucitar a nadie. Me alegraré de que sea rey de España Fernando
VII\ldots{} Ya sabes por qué he servido a José: me moría de hambre y
acepté sus banderas. Tal vez hice mal, pero las juré y tras ellas voy a
donde me lleven. Eso de gritar hoy Bonaparte y mañana Fernando, como
hacen muchos, no entra en mi sistema. Sirvo a José sin entusiasmo; pero
con lealtad.

---¡José, José---exclamó Bragas alzando la voz,---es un borracho! No se
tiene lealtad con los borrachos.

---A ti y a mí nos ha dado de comer. Los dos nos encontrábamos en Madrid
bastante perdidos y derrotados. Mi tío me colocó en el regimiento de
jurados, lo que fue muy fácil, porque nadie quería entrar en él. Tu
colocación parecía más difícil; pero tanto lloraste y gimoteaste ante el
conde de Cabarrús, que el buen señor, considerando que eres hijo de su
criado, diote a roer ese hueso de la covachuela. Para conseguirlo, te
fingiste entusiasmado con el fraternal gobierno de Bonaparte, ¡y qué
memoriales le echabas!\ldots{} ¡cuántas resmas embadurnaste con lamentos
y suspiros!\ldots{} Para que todo no fuera música y palabrillas vanas,
te aplicaste el oficio de dar vítores y palmadas en la calle siempre que
el Rey pasaba, y gritar «¡Mueran los madripáparos!»

---¡Mentira, mentira!---exclamó Juan Bragas, cuyo rubor no podía
distinguirse a causa de la oscuridad de la noche.---¿De dónde has sacado
tales invenciones?

---Verdad, verdad pura, digo yo---continuó Monsalud,---como también lo
es que te daban obra de tres reales por función, quiero decir por cada
carrera detrás del coche de Pepe Botellas, gritando y vitoreándole. Ello
es que si te desgañitaste, ganando aquella ronquera que te puso en
peligro de callar para siempre en la sepultura, en cambio recibiste el
destino que tienes, el cual verdaderamente no es mucho premio para tanto
batir palmas y asordar a la gente con los vivas.

---Salvador, Salvador, mira que me incomodo---dijo Bragas con voz
balbuciente, señal de que le ponía colérico el verídico retrato que su
amigo diestramente trazaba.---Cualquiera que te oiga ¿qué pensará de mí?

---Ahora quieres pasar por hombre formal. Vas muy serio y finchado por
la calle, entras en la covachuela dando taconazos, y cualquiera
supondría que dentro de ese casacón que compraste en el Rastro, va un
Consejero de Indias.

---Si no va todavía, irá con el tiempo, señor mío.

---Y como parece que el Rey José y los franceses y los jurados se
marchan para siempre, quieres hacer olvidar que te colocó el conde
Cabarrús\ldots{} Ahora es preciso empecinarse, señor Juan Bragas, como
se empecinó su merced durante el tiempo en que evacuaron la villa los
franceses y la ocuparon los aliados después de la batalla de los
Arapiles.

---Amigo Monsalud---gruñó el otro,---yo soy dueño de hacer mi santa
voluntad ahora y siempre. Sé donde me aprieta el zapato, y cada uno
tiene su alma en su almario. Tú mismo que ahora te la echas de hombre
recto y puntilloso estás esperando a que los franceses salgan de aquí
para desertar de sus filas y pasarte a los españoles, lo cual es muy
meritorio y por extremo patriótico; que no hay gloria más envidiable que
servir a la patria, ni deshonra que se compare a la de ayudar al enemigo
contra nuestros hermanos. Y ahora que los franceses van de capa caída y
parece que huyen vencidos, el heroísmo consiste en volverles la espalda.

---Eso no lo haré yo---dijo con energía Monsalud,---que cuando entré a
servirles lo hice por mi voluntad.

---Pues no te podrás quitar de encima la nota de traidor---indicó con
energía Bragas,---que traidores son los que sirven al enemigo de la
patria. ¿No te da vergüenza de vestir ese uniforme?

Cuando esto decía, habían entrado en la calle de Toledo y tomaban por la
derecha la embocadura de la Cava-Baja, donde tenía su residencia el Sr.
Monsalud senior, tío de nuestro héroe. Por las noches Salvador solía
hacer parada en casa de su tío, antes de encerrarse en el cuartel, y
acompañábale generalmente Bragas, atraído por un olorcillo de una
regular cena que allí se aderezaba y el reclamo de una animada tertulia.

---Veremos qué piensa mi tío de estas cosas---dijo Monsalud.---Él es un
afrancesado rabioso, y desde que el conde de España le mandó dar de
palos en Salamanca, no cesa de decir que ahorcaría a todos los
empecinados si estuviere en su mano.

No había concluido Monsalud de decir lo antecedente, atravesando la
plazoleta que llaman Puerta Cerrada, aunque no hay allí puerta alguna
abierta ni entornada, como no sean las de las casas, cuando muchas de
las gentes reunidas junto a las tiendas, y el gran número de majos,
chulillos y muchachos desvergonzados que por allí discurrían, fijaron su
atención en los dos jóvenes, y principalmente en el sargento de la
guardia, cuyo uniforme a cien leguas le denunciara como servidor del rey
entrometido.

---Parece que nos miran---dijo Monsalud,---y nos señalan. ¿Llevamos algo
de particular?

---Es que la gente está alborotada\ldots---balbució Bragas, temblando de
miedo.---Llevas uniforme de la guardia jurada\ldots{} Ese traje es muy
aborrecido en Madrid, y con razón, con muchísima razón\ldots{} No creas
que te van a defender tus amigos. Ocupados de su viaje, no se cuidan de
niñerías, y lo mismo les importará que te insulten o que no. Los
franceses desprecian a los traidores que les sirven, como los
despreciamos los españoles.

Iba a contestar Monsalud, cuando de un grupo de holgazanes que sostenía
la esquina de la Cava-Baja salieron voces de a ese, a ese, y luego un
murmullo de risas insolentes. Monsalud se paró en medio de la calle, y
volviéndose a los del grupo les miró cara a cara, esperando que alguno
pasase de las palabras a las obras. En el mismo instante, varias pelotas
de lodo, arrojadas por los chiquillos, se aplastaron en su pecho,
salpicándole la cara.

El populacho es algunas veces sublime, no puede negarse. Tiene horas de
heroísmo, en virtud de extraordinaria y súbita inspiración que de lo
alto recibe; pero fuera de estas horas, muy raras en la historia, el
populacho es bajo, soez, envidioso, cruel y sobre todo cobarde. Todos
los vencidos sufren más o menos la cólera de esta deidad harapienta que
por lo común no sale de sus madrigueras sino cuando el tirano ha caído.
Si no le supo exterminar con su iniciativa y su fuerza, casi siempre se
da el gustazo de rociarle con su fango; y a todas las instituciones o
personas que caen por el esfuerzo de campeones de otra esfera más alta,
el populacho les pone su ignominioso sello de inmundicia. La libertad y
las caenas, a quienes alternativamente aduló, han visto sobre sí en el
momento terrible a la furia inmunda que les escupía. Como la hiena, es
intrépida con los muertos.

Casi desguarnecida Madrid de tropas francesas, pues muchas habían ido
saliendo desde mediados de Mayo; dispuesto todo para marchar las últimas
en la madrugada del siguiente día 27, el enemigo, puesto un pie en el
estribo, no se cuidaba ya de hacer cumplir las reglas de policía. El
estado de la guerra y la comprometida situación de José junto al Ebro,
confirmaban a aquel en su idea de que la ocupación de España iba a tener
fin; mas si estaban indiferentes y aun alegres los franceses, los
españoles comprometidos con ellos, no cabían en su pellejo de puro
azorados y medrosos. A muchos de estos insultó la plebe en diversos
puntos, y algunos aterrados algunos al ver el desamparo en que quedaban,
desertaron para acogerse de nuevo a las banderas de la patria.

Se comprenderá, pues, que la situación de Monsalud frente a los
respetables varones del populacho matritense, no era muy lisonjera.
Ciego de enojo, con el rostro encendido y la voz balbuciente, echó mano
a la empuñadura del sable gritando:

---Al que se me acerque, lo atravieso.

Y capaz era de hacerlo como lo decía, lo cual fue sin duda conocido por
el egregio concurso de la esquina, no habiendo entre todos ellos uno
solo que se destacase del grupo para hacer frente al irritado mancebo.
Viendo este que con ser tantos, no pasaban a vías de hecho, siguió su
camino; pero los disparos de lodo se repitieron de tal modo por la
cohorte infantil, que Monsalud sin hacer uso del arma, corrió tras uno
de aquellos angelitos de arroyo para castigar su desvergüenza. Antes que
llegara a atraparle, lo que no osaron tantos hombres, atreviose a
hacerlo una mujer, la cual cuadrándose marcialmente ante Salvador y
desafiándolo del modo más varonil con ojos, gestos, manos y la cortante
y ponzoñosa lengua, le dijo:

---¡Eh! so estandarte, si toca Vd. al muchacho no tendrá tiempo de
encomendarse a Dios. Si el angelito le roció, es porque puede hacerlo, y
para eso y mucho más le he parido\ldots{} Conque siga adelante; y punto
en boca y manos quietas.

Dada la señal por la matrona, acercáronse valerosos algunos de los
chulos y tomadores que antes dispararan sobre el soldado burlas y
palabrotas; enracimáronse los chiquillos y mujeres en derredor suyo, y
una tempestad de insultos tronó en sus oídos. Aturdido al principio el
mozo, defendiose con empellones y golpes muy bien dirigidos.

---¡Matarle!---gritó una arpía, al sentirse abofeteada por la mano
vigorosa de la víctima.

Y también a su compañero el del casacón.

---A mí, señores ¿pues qué he hecho yo?---dijo Bragas, procurando
echarse fuera del volcán.---Yo no conozco a ese hombre.

---¡Mueran los jurados!

---¿Acaso visto yo ese vergonzoso uniforme?---repitió casi llorando
Braguitas.---Soy un joven honrado, español puro y neto, y jamás he
servido a la basura.

Monsalud, a quien no hostigaba ningún hombre de buenos puños, sino tan
sólo mujerzuelas, chicos y algún cobarde zarramplín, de esos que van a
todas las pendencias a meter ruido, pudo echar mano al sable y apartar
un poco de su persona al indigno enjambre. Repartió de plano con seguro
puño algunos golpes, y sin ser Papa creó gran número de cardenales en
menos que canta un gallo. Algunas personas graves y varios majos
decentes intervinieron en el asunto, aplacando la furia de todos, y
propusieron que se dejase en libertad al guardia, con tal que allí mismo
se quitase el uniforme. Enfurecido y fuera de sí Monsalud, iba a
arremeter contra los amigables componedores, cuando apareció su tío D.
Andrés saliendo de la casa cercana que era donde vivía, y con razones y
tal cual empellón, él y otros que le acompañaban cortaron la pendencia,
obligando al joven a meterse en el portal que cerraron al instante.

Puesto en salvo su sobrino, a quien acabaron de aplacar las personas de
ambos sexos que había en la casa, el Sr.~Monsalud creyó oportuno dirigir
la palabra a los del pueblo, un tanto mohíno por no haber podido vengar
en el renegado las contusiones recibidas.

---No hagan Vds. caso, señores---les dijo con voz oratoria, que en su
vana sonoridad gustaba de oírse a sí misma.---Ese joven es mi sobrino,
un mala cabeza, un insensato que se afilió en el cuerpo de guardias
jurados, sin saber lo que se hacía. Pero en el fondo de su alma,
señores, mi sobrino es español por los cuatro costados y aborrece a los
pérfidos enemigos de la patria. Comprendo, señores, que el pueblo se
ensañe contra los afrancesados: esos viles merecen pronto y ejemplar
castigo. (Señales de aprobación). Pero respetemos la desgracia, señores
y señoras; que demasiado castigo tienen esos viles en su propio
remordimiento y vergüenza. Esta noche es noche de gran regocijo para los
buenos españoles, porque mañana se marchan los pocos borrachos que
quedan en Madrid. España es libre, señoras, caballeros y niños. ¡Viva
España! (Ruidosos aplausos, y tal cual rebuzno y no pocas patadas,
berridos y coces). Yo respondo de que mi sobrino dejará las traidoras
banderas en que ha servido; él es buen patriota, tan buen patriota como
yo, que estoy dispuesto a derramar la última gota de mi sangre, sí, la
última y postrera gota en defensa del Rey y de la Constitución. ¡Viva la
Constitución! (Ibídem)\ldots{} Y si alguna vez he vivido entre
franceses, no lo hice por amistad hacia ellos, como dicen mis enemigos,
sino que les seguí y me metí industriosamente entre sus filas para
averiguar sus planes y espiar sus acciones e informar de todo a nuestros
queridos, a nuestros queridísimos generales\ldots{} ¡Ah! ¿Queréis más
pruebas? Pues allá van las pruebas. Os ruego que contestéis a mis
preguntas. ¿Quién soy yo, señores? Yo soy un mártir del patriotismo.
Consagré mi vida al servicio de la patria, y hallándome cerca de
Salamanca, en un pueblo de cuyo nombre no quiero acordarme, los
franceses me apalearon. ¿Y por qué, señores? Porque con mi espionaje
puse todos sus secretos estratégicos al servicio de lord Wellington.
Pues qué, ¿creéis que sin mí se hubiera ganado la batalla de los
Arapiles? (Estupor). Aún tengo sobre mi cuerpo cien cardenales que con
su noble púrpura manifiestan mi heroísmo. Luego vine a Madrid a gozar
del espectáculo de este gran pueblo, ebrio de gozo por su libertad, y en
Agosto del año pasado juramos la Constitución en presencia del general
inglés. ¡Oh día solemne! ¡Oh época feliz! Si se empañó tan diáfana
claridad con el regreso de los franceses, mañana se desgarrará el velo
tenebroso de la invasión, mañana se marcharán otra vez para siempre,
para siempre, señores, con su séquito inmundo de traidores y jurados y
afrancesados. Ved cómo tiemblan, cómo se esconden de vuestras
patrióticas miradas, cómo su vergüenza les hace bajar la cabeza ante la
majestad de nuestro puro españolismo sin mancha. Enorgullezcámonos,
señores, de no haber servido jamás a los franceses, de no habernos
contaminado jamás con viles masones y filosofastros, y digamos con el
ángel: Ave María\ldots{} Cada cual a su casa que es hora de acostarse.
¡Viva la Constitución y el lord y Fernando VII! (Tumulto y
extraordinaria sensación, acompañada de sonoros bramidos y vocablos que
no lleva en sus blancas páginas el Diccionario por miedo a ruborizarse).

\hypertarget{v}{%
\chapter{V}\label{v}}

Salvador subió tristemente la escalera de la casa acompañado de varias
personas que atraídas del ruido y del temor bajaron, y en la meseta
donde se abría la puerta del domicilio de su señor tío, recibiole,
candil en mano, la esposa de este, que le dijo así:

---No podía ser otra cosa que una barrabasada del sobrino de mi marido.
¡Todo sea por Dios! Este chico tiene la cabeza a las once y está podrido
de ella. ¿Te han herido?

---El pueblo de Madrid aborrece este uniforme---gritó Bragas que detrás
a poca distancia subía,---y no le falta razón.

---Sólo a este loco se le ocurre sacar el sable porque le echaron un
poco de fango---dijo la señora de Monsalud alumbrando para que pasasen
todos a la sala.

Componían aquella noche la tertulia, doña Ambrosia de los Linos y sus
dos hijas, una de las cuales, casada poco antes, vivía en el piso
tercero del mismo edificio. Ambas eran bastante lindas, principalmente
la soltera, que cautivaba por su frescura, por sus vivarachos ojos, por
sus rosados carrillos, marcados aquí y allí con vagabundos lunares, y
por su gracia en el mirar y la flexible ligereza de su cuerpo, tanto más
admirable, cuanto que la muchacha era algo más que medianamente gordita,
prometiendo en diversos parajes de su persona, que igualaría con los
años a su enorme mamá. También estaba allí D. Mauro Requejo que solía ir
todas las noches, por ser pariente de la señora de Monsalud, y no tardó
en presentarse don Gil Carrascosa.

La señora de Monsalud era una mujer de presencia no vulgar ni
desagradable, pero muy gastada y decaída por causas que ignoramos.
Durante un matrimonio estéril, que ya contaba trece años, marido y mujer
no habían ofrecido al mundo un modelo perfecto de concordia. Repetidas
veces se separaron para volverse a juntar; repetidas veces crujieron los
palos de las inválidas sillas, y volaron por el aire los platos
desportillados, instrumentos unas y otros de la ciega cólera homicida de
ambos consortes. Andrés Monsalud era hombre de mala conducta, fatuo,
desarreglado, trapisondista, embrollón, aventurero. Serafinita pecaba de
caprichosa, holgazana, embustera, y tenía más vanidad que una princesa,
gustando mucho de emperifollarse, y sobre todo de aparentar posición y
suponer posibles muy superiores a lo que en realidad tenían ella y su
marido, pues reunida la fortuna inmueble de entrambos, allá se iba con
la nada.

Por último, después de la tragedia de Babilafuente, Serafinita logró
atraer a su marido y poner casa en Madrid, y de la noche a la mañana por
mediación generosa de un caballero francés dieron a Andrés un regular
destino en la Visita de Propios, con lo cual uno y otro estaban tan
huecos, que de allí, a tratar a Dios de tú, apenas había el canto de una
peseta. Su morada, no obstante, era humildísima, porque el sueldo no
rayaba, ciertamente, en Potosí; mas Serafinita se esmeraba en aumentar
con mil artificiosas combinaciones el lustre y aparato de su casa.

---Puedes respirar tranquilo, sobrino---dijo la señora con
bondad.---Descansa y se te dará un vaso de agua para matar el susto.

---No quiero agua---repuso bruscamente el joven, paseándose de largo a
largo por la sala.---Tengo que marcharme.

---¡Marcharse!---exclamaron a dúo y con desconsuelo las dos niñas de
doña Ambrosia.

---Este joven gusta de pendencias y de derramar sangre---añadió
esta.---¡Cómo se conoce que los franceses le crían a sus pechos!

---Pero al menos---dijo Serafinita,---¿te quitarás el uniforme?

---Sí, hablad de eso a este babieca---indicó Juan Bragas, que había ido
a fondear junto a la más pequeña de las fragatitas de doña
Ambrosia.---Es muy gabacho este caballero. Los pocos españoles
extraviados que sirven en las banderas de José, están a estas horas con
los ojos y el corazón vueltos hacia la madre patria afligida; pero este
mi D. Quijote botellesco, dice que su honor le obliga a no abandonar a
la canalla.

---Hace cosa de seis meses---afirmó Serafinita,---habría sido gran
locura mostrar siquiera un adarme de españolismo; pero hoy es distinto.
Los franceses van de capa caída y buen tonto será quien se embarque con
ellos.

---¡Oh, sí, será un idiota!---dijo doña Ambrosia,---aunque lo mejor
habría sido no servirles nunca.

---Las circunstancias---añadió Serafinita,---obligan a los hombres a
sofocar algunas veces su natural impulso y fogosidad patriótica. Ahí
está mi marido, que no le hay más español en toda la tierra del
garbanzo, y sin embargo viose arrastrado a cierto compadrazgo con los
franceses, y aun anduvo con masones y revoltosos, malquisto de todo el
mundo. Pero de algo valen los consejos de una mujer prudente. Yo le
traje al buen camino, y como mi familia, que no es ninguna familia de
tres por un cuarto, ha tenido siempre relaciones con altos personajes,
fácil me fue amarrar a mi esposo al pesebre de la Visita de Propios.
Diole la plaza un ministro francés; ¿pero tenemos la culpa de que haya
sido francés quien primero echó de ver nuestros méritos, o si se quiere,
los de mi marido, para todo lo que sea cosa de aritmética en cualquiera
oficina?

---Si recibimos un pequeño favor de esa canalla---gritó con vehemencia
Bragas,---diéronnos lo nuestro y nada tenemos que agradecerles.
Españoles somos, y ahora váyanse con dos mil demonios.

---Lo que hay en esto---dijo D. Mauro Requejo, que sombríamente había
permanecido en un rincón de la sala, sin hablar hasta entonces,---es que
para dar sus destinos a los señores Monsalud y Bragas, fue preciso
quitárselos a otros, que pecando de empecinados, mortificaban con
cuchufletas y versitos a los franceses.

---¡Nadie hay más empecinado que yo!---exclamó con furioso arranque de
entusiasmo Juan Bragas, saltando en medio de la sala, con gran regocijo
de las niñas de doña Ambrosia.---¡Viva D. Juan Martín Díez!

---¡Viva, viva mil años!---repitió Andrés Monsalud, presentándose en la
sala, con semblante reposado y satisfecho, sin duda por la vanagloria
que el reciente discurso callejero había dejado en su ánimo.---¡De buena
has escapado, sobrinillo! ¡Exponerse a las iras del pueblo
español!\ldots{} Vamos, te perdono; yo también he sido calavera, yo
también he sido revoltoso y provocativo y\ldots{}

---Afrancesado---indicó con malicia doña Ambrosia.---No hay que
echársela ahora de apóstol Santiago.

---Un poquillo---repuso Monsalud con turbación.---Pero de arrepentidos
se hacen los santos. La prueba de mi sinceridad la tengo hoy en la
confianza de mis amigos. Hanme comisionado esta tarde para preparar los
festejos\ldots{}

---¿Para cuando entre D. Carlos España?---preguntó la de los Linos.

---Para cuando entre D. Juan Martín o lord Wellington\ldots{} Un arco de
triunfo, ¿qué les parece a Vds.? En mi oficina hemos resuelto componer
unos versos, y ver si se hace un carrito.

---Ya nos cayó que hacer, amigas mías---dijo con júbilo
Serafinita.---Desde mañana pondremos manos a la obra, porque las
guirnaldas de rabo de cometa no son cosa que se despache en tres días.

---Y luego mucho de banderitas y escarapelas---dijo una de las
muchachas.

---Y será preciso que doce o catorce doncellas tiernas se vistan de
ninfas para ir delante del carro cantando el Velintón.

---Y como haya alegoría vestiremos a mi sobrino de dios Marte---indicó
Monsalud.

El joven soldado dirigió a su tío una mirada de desprecio.

---Estará saladísimo---dijo doña Ambrosia.---Mi esposo y padre de estas
dos niñas hizo de Marte cuando la jura del otro Rey, y era una gloria el
verle con todo su hermoso cuerpo medio desnudo y un chafarote en la
mano\ldots{} ¡Oh! ustedes no alcanzaron a ver tanta preciosidad.

D. Gil Carrascosa, entrando apresurado en la estancia, saludó a todos
con amable cortesanía, especialmente a las niñas.

---¿Pues qué---dijo,---todavía está nuestro mozalbete metido dentro de
la indigna librea francesa? A estas horas casi todos los españoles que
servían a José han desertado. Acabo de ver a dos que se escondieron esta
mañana.

---¡Han desertado!---repitió el coro de mujeres.

---Fuera esa casaca, sobrino---gritó Monsalud dirigiendo al hijo de su
hermana imperiosa mirada.---¡Ay! acuérdate de tu madre, a quien no nos
atrevimos a dar parte de tu afrancesamiento\ldots{} Si lo llega a saber,
se morirá de pena.

---Te esconderemos aquí---dijo Serafinita,---aunque no habrá peligro,
pues ellos tienen bastante que hacer para ocuparse de ti.

---En esta casa no---afirmó con aplomo el tío.---Los vándalos conocen el
rabioso españolismo mío, y de seguro vendrían a buscarle aquí,
acusándome de haberle impulsado a la deserción.

---Pues se puede esconder en mi casa---dijo la mayor de las Linas, que
era la casada y tenía su nido en el tercer piso.

---Eso es, que se esconda arriba---repitió con extraordinaria vehemencia
la soltera, contemplando al joven Monsalud de tal modo que parecía
envolverle con su mirada como en amorosa y blanda nube protectora.

---Sí, en el tercero.

---Yo le cederé mi cuarto y mi cama, y dormiré con mi hermana---añadió
la doncella en un segundo arranque de generosidad.

---Francamente, Dominguita, tu esposo está fuera y no me gusta ver a dos
muchachas solas en la casa con el dios Marte---objetó doña Ambrosia.

---Pues al sotabanco. Hablaremos al Sr.~Pujitos para que le ceda un
rincón.

---Conque, sobrino, vete despojando de tu uniforme.

El soldado, a quien tal proposición ofendía en lo más delicado de su
alma, y que estaba a la sazón irritado por la escena de la calle, y
además por el impertinente charlar de su tía, contestó con ardor:

---Antes me quitaré el pellejo que el uniforme. Me lo puse por mi
voluntad, lo tendré mientras exista el ejército a que pertenezco y la
bandera que juramos.

---¿Eres francés?

---No sé lo que soy---repuso con desdén.

---¿Harás armas contra tus paisanos?

---No; pero tampoco abandonaré cobardemente a los que me han dado de
comer.

Monsalud tío rompió en estrepitosas risas, acompañado por Bragas,
Requejo y Carrascosa.

---Pero, sobrino de todos los demonios, ¿no tienes en mí la norma de tu
conducta?

---Si yo le imitara a Vd. en esto---dijo el joven temblando de
indignación---no tendría idea del honor, ni una chispa de vergüenza en
mi alma, ni en mi corazón el sentimiento del deber, ni sería digno de
que me mirasen los hombres. Adiós. Me voy para siempre de esta casa y de
Madrid.

El soldado salió resueltamente. Un poco atontado el tío, bastante
aturdida su esposa, no pronunciaron una sola palabra para detenerle.

---Ese muchacho es un insolente---dijo al fin la señora de la casa.

---¡Pobrecito!---murmuró el oficial de la Visita de Propios.

---¡Él se lo pierde!---indicó majestuosamente Serafinita.---Ahora que
mandan los españoles he de conseguir para ti una buena vara, Andresito.
Serás corregidor de Alcalá, de Ocaña o de Tarancón. Yo había calculado
que Salvadorcillo nos acompañaría con un buen momio.

---No se puede sacar partido de ese muchacho.

La niña soltera de doña Ambrosia había llevado el pañuelo a sus
picarescos ojos, de súbito humedecidos por ignorada causa.

---¡Pobrecito!---exclamó con zozobra.---Se ha marchado solo. Está
expuesto a que le insulten otra vez en la calle. Le darán golpes, le
arrojarán lodo, manchándole la frente, el cabello, la boca, los ojos,
¡ay! los ojos, el uniforme\ldots{}

---Esto parte el corazón. ¡Pobre muchacho!---exclamó la
casada.---Alguien debía salir con él.

---¡Qué falta de caridad dejarle salir solito! Si yo fuera
hombre\ldots{}

---La verdad es que puede sucederle alguna cosa mala---dijo Serafinita
dando un suspiro.

---Usted que es su amigo---exclamó con ira la doncella volviéndose a
Juan Bragas que a su lado estaba,---¿por qué no salió con él para
ampararle en caso de un atropello?

---¿Amigo?---dijo con desdén el covachuelo.---No tanto. Conocido y nada
más\ldots{} Nos hablamos alguna vez, paseamos juntos, pero\ldots{}

---Es Vd. un mal amigo---gritó la muchacha con voz
temblorosa.---¡Dejarle partir sin compañía!\ldots{} Esto se llama
deslealtad, cobardía.

Juan Bragas se echó a reír.

---Pero\ldots{}

---Haga Vd. el favor de no volver a dirigirme la palabra en toda la
noche, ni volver a mirarme en su vida, ni estar donde yo esté, ni
respirar donde yo respiro, ni ponerse donde yo le vea, ni\ldots{}

La tertulia fue triste, tristísima. Los hombres viendo que no podían
alegrar el ánimo de las dos muchachas, ni el de la señora de la casa, ni
sacarles palabras que no fuesen lúgubres como un funeral, pegaron la
hebra con doña Ambrosia, y dándole a la lengua sin descanso por espacio
de dos horas, azotaron a medio mundo con la piel arrancada al otro
medio.

\hypertarget{vi}{%
\chapter{VI}\label{vi}}

En la mañana del día que siguió a estos sucesos salieron los pocos
franceses que quedaban en Madrid. Les mandaba el general Hugo y llevaban
consigo convoy tan inmenso, que al verlo creeríase que en la capital de
la monarquía no quedaba un alfiler. Desde muchos días antes habían sido
embargados cuantos coches y carros y calesas rodaban por las calles de
la villa, y casi toda la servidumbre se ocupaba en el embalaje de las
diversas riquezas que José y los suyos se habían apropiado. Estos
señores hacían buena presa donde quiera que ponían la mano y no eran
nada melindrosos ni encogidos para esto del incautarse. Murat despojó la
casa de Godoy y el real palacio, y José mandó traer de Toledo, de
Valladolid y del Escorial cuanto pudiese ser transportado; esta última
circunstancia salvó las piedras del edificio.

Luego que estuvo reunida cantidad fabulosa de cuadros, estatuas, joyas
de camarín y sacristía, dejando a las Vírgenes y Santas sin un anillo
que ponerse, establecieron cuatro depósitos en Madrid, los cuales fueron
el Rosario, San Felipe, doña María de Aragón y San Francisco. Una
comisión separó lo sublime de lo bueno, y no siendo fácil llevarlo todo,
dispusieron atropelladamente lo primero en cajas, mezclando lo sagrado
con lo profano, es decir, las bellas artes con los enseres de la casa y
cocina del Rey José y diversos adminículos que este para diferentes
fines usaba. Muebles, porcelanas, vajillas, armas, añadiéronse al botín.
Considerando que aun después de tanto despojo quedaba en España alguna
cosa de todo punto inútil, según ellos, a la ignorancia castellana,
echaron mano a las colecciones mineralógicas del gabinete de Historia
Natural y embaularon también los depósitos de ingenieros y de artillería
y el hidrográfico. De Simancas cargaron con lo más curioso que allí
había. Aquella gente, hasta la historia nos quiso quitar.

Una caja en que holgaba un poco el tocador de José (así lo cuenta un
testigo ocular) fue rellena con los pedruscos y los minerales de la
Historia Natural. Entre una masa enorme de cartas geográficas iba
Nuestra Señora del Pez; y la Perla anidó con una montura fina recamada
de plata y oro. Se gastó un monte de claros, y por algunos días las
iglesias que servían de depósitos y las galerías del real palacio
resonaban cual si en ellas trabajase un regimiento de cíclopes. La tabla
del Pasmo, que ya se hallaba en pésimo estado, acabose de rajar, y la
pintura con las sacudidas y golpes se cuarteaba que era una bendición.
¡Oh divino Jesús! ¡No padeciste más en el Gólgota!

Completaban el convoy las cajas de guerra llenas de dinero en buen oro y
buena plata antigua, de aquello que ya no se ve, y seducía entonces con
su brillo los ojos de los extranjeros y con su noble son los oídos de
todos. No se habían descuidado los franceses en reunir dinero, como
gente allegadora y económica, ni menos en llevárselo; que si para
limpiar de vicios a la capital hubieran usado de tanta diligencia como
para limpiarla de onzas, fuera esta villa un paraíso en la tierra. Con
el ejército iban los muchos particulares comprometidos que quisieron
seguirles, y entre los carros de oficio, gran número de vehículos con
equipajes de empleados altos y bajos. Ofrecían estos desgraciados
individuos espectáculo lastimoso. Si algunos llevaban consigo buen
acopio de víveres y ropa, otros no cargaban más que lo puesto, y todos
lloraban el hogar abandonado, la paz perdida, el honor en duda,
lamentándose del gran compromiso en que se veían. Algunos hacían de
tripas corazón, prometiéndoselas muy felices en las próximas batallas;
pero los más miraban sin engañarse la realidad del molesto viaje y
después la emigración, el general desprecio y la pérdida de la hacienda.

Desfilaron los carros por el camino de Segovia, pues Hugo quería pasar
la sierra por Guadarrama, y aquella culebra rastrera formada por
interminable fila de vehículos, que de lejos parecían vértebras
articuladas, desapareció en la noche del 27 de Mayo, dejando a Madrid en
poder de los guerrilleros que al instante lo ocuparon y tras ellos las
autoridades españolas. De esta manera y con este despojo la capital de
España dejó para siempre de ser francesa.

No seguiremos al general Hugo y su convoy en todo su viaje hasta que en
los campos de Vitoria perdieron los franceses gran parte de lo mucho que
habían cogido. Bastantes apurillos pasó en Cuéllar y en Tudela de Duero;
pero al fin logró unirse al grueso del ejército francés en Valladolid.

Reunidos todos, la continua amenaza de las divisiones aliadas les hizo
muy penoso el camino desde Valladolid a Burgos. Aquí no pudieron
resistir mucho tiempo, y sin gran prisa se dirigieron a Vitoria por
Miranda confiados en que Wellington no les molestaría del lado allá del
Ebro; pero tan admirable combinación de movimientos había hecho el
inglés que cuando los franceses pasaron el gran río, lo pasaban también
los aliados por diferentes puntos, y ambos enemigos se encontraban
frente a frente en las montañas de Álava y Vizcaya. Apretó Bonaparte el
paso juntando a los suyos para que desperdigados aquí y allí no fueran
batidos al pormenor, y el 19 de Junio llegó a la Puebla de Arganzón,
donde es fuerza que quitemos la vista del Rey y de su ejército para
fijarla en una sola persona, que por ahora y mientras vengan sucesos
estupendos en la esfera de la historia, ha de llevar en estas líneas la
preferencia.

¡Y por qué no! ¡Por qué hemos de ver la historia en los bárbaros
fusilazos de algunos millares de hombres que se mueven como máquinas a
impulsos de una ambición superior, y no hemos de verla en las ideas y en
los sentimientos de ese joven oscuro! ¡Si en la historia no hubiera más
que batallas; si sus únicos actores fueran las celebridades personales,
cuán pequeña sería! Está en el vivir lento y casi siempre doloroso de la
sociedad, en lo que hacen todos y en lo que hace cada uno. En ella nada
es indigno de la narración, así como en la naturaleza no es menos digno
de estudio el olvidado insecto que la inconmensurable arquitectura de
los mundos.

Los libros que forman la capa papirácea de este siglo, como dijo un
sabio, nos vuelven locos con su mucho hablar de los grandes hombres, de
si hicieron esto o lo otro, o dijeron tal o cual cosa. Sabemos por ellos
las acciones culminantes, que siempre son batallas, carnicerías
horrendas, o empalagosos cuentos de reyes y dinastías, que preocupan al
mundo con sus riñas o con sus casamientos; y entretanto la vida interna
permanece oscura, olvidada, sepultada. Reposa la sociedad en el inmenso
osario sin letreros ni cruces ni signo alguno: de las personas no hay
memoria, y sólo tienen estatuas y cenotafios los vanos
personajes\ldots{} Pero la posteridad quiere registrarlo todo; excava,
revuelve, escudriña, interroga los olvidados huesos sin nombre; no se
contenta con saber de memoria todas las picardías de los inmortales
desde César hasta Napoleón; y deseando ahondar lo pasado quiere hacer
revivir ante sí a otros grandes actores del drama de la vida, a aquellos
para quienes todas las lenguas tienen un vago nombre, y la nuestra llama
Fulano y Mengano.

\hypertarget{vii}{%
\chapter{VII}\label{vii}}

Olvídese la importuna digresión, y sepan los que en ello tuvieren
interés, que antes que el ejército de José pasase el Ebro, llegaron a la
Puebla de Arganzón las tropas de una división que custodiaba parte del
convoy. Fue esto, si no mienten las noticias que con pretensiones de
verídicas se me han dado, hacia el 16 ó 18 de Junio. El gran convoy
venía detrás. Los carros del pequeño detuviéronse en el camino a las
inmediaciones del pueblo, y las tropas repartiéronse por las casas y
caseríos para allegar víveres. En las inmediaciones de la villa veíanse
grandes masas de soldados: aquí artillería, allá columnas que iban de un
lado para otro; en lo más apartado la impedimenta, y largas filas de
vehículos, que después de breve descanso debían seguir adelante.

La Puebla de Arganzón, como lugar campestre, había dejado las ociosas
plumas, y aunque de por sí no fuese aquella villa madrugadora,
despertola el rumor de tanta tropa y de los tambores sin cesar batidos,
confundiendo su ronco son con el cantar de los gallos que en todos los
corrales entonaban su alegre grito de alerta. Veíase a los honrados
habitantes salir de sus casas y juntarse en corrillos. Los ancianos
preguntaban si se había ganado ya la batalla y advertidos de que no,
quejábanse de la mucha tardanza en arremeter, propia de los tiempos
nuevos, asegurando que en otra ocasión ya estaría todo despachado y el
asunto resuelto. Las mujeres corrían de casa en casa pidiéndose
provisiones para esconderlas, pues los franceses que en número tan
considerable rodeaban el pueblo reclamarían pronto lo que no se habían
llevado los guerrilleros el día anterior.

En las tabernas los taberneros no tenían manos para tanto despacho y muy
alborozados escanciaban a los franceses, pues en esto del vender y ganar
dinero no hay naciones: ellos quisieran tener un Océano de aguardiente y
vino, que junto con algunas pipas de linfa del Zadorra les hubiera hecho
millonarios en un par de años de guerra.

Un joven sargento avanzaba solo por las calles de la Puebla, evitando al
parecer la compañía de sus camaradas franceses, y más aún la vista de
los habitantes de la villa. Así es que cuando veía un grupo en la puerta
de una casa se apartaba tomando distinto camino.

---¿No es aquella la cara de Salvadorcillo Monsalud, el hijo de la
señora Fermina la de Pipaón?---decía una mujer viéndole pasar.

---Parece que es aquélla su cara; pero no su cuerpo; que es cuerpo y
uniforme de francés el que ha pasado.

---Adelantadas estáis---decía un tercero.---¿Pero no sabéis que
Salvadorcillo Monsalud, engañifado por su tío, ha sentado plaza en la
guardia del rey José?

---Cierto es, aunque no lo participó a su madre por vergüenza; y cuando
la señora Fermina lo supo, estuvo llorando tres días, y aún no lo quería
creer, siendo tal su pesadumbre por esta traición de Salvador, que la
buena mujer dice que más quería verlo muerto que sirviendo a los
franceses.

---Y tiene razón. ¿Mas para qué dejó que el muchacho fuese a Madrid
donde todo es corruptela y picardía?---dijo un personaje a quien todos
oían con respeto, y que era, si nuestras noticias no son falsas, el
boticario del lugar.---Pero esto pasa a todos los muchachos que no
tienen padre, o mejor, a aquellos que han nacido del pecado y de unión
nefanda, como ese diablillo de Salvador Monsalud, que no se sabe de qué
tronco vino, ni de cuál cepa sacó doña Fermina este mal sarmiento.

El jurado se detuvo ante una casa de aspecto humilde, en cuya puerta no
se veía persona alguna. Miró a las ventanas, y las vio cerradas. Un
gallo cantaba dentro, y dos o tres gallinas salieron a la calle
sacudiendo sus plumas y picoteando el suelo, no tardando en aparecer
tras ellas el gallardo esposo. Poco después un gato asomó por la puerta
entreabierta y se detuvo sobre el umbral, relamiéndose con placentera
satisfacción los largos bigotes. El joven contempló un instante con
interés profundo a aquellos seres, y se acercó para entrar, desalojando
al gato, que asustado corrió hacia dentro. Las gallinas y el gallo,
sobresaltándose también y cambiando algunas cacareadas frases, huyeron
por la calle adelante.

Monsalud se asomó por el hueco de la entornada puerta. La emoción de su
alma era tan viva que le temblaban las manos al ponerlas sobre las
viejas tablas y los mohosos clavos; apenas podía sostenerse en pie a
causa del desmayo de su cuerpo y de la flojedad nerviosa que
experimentaba. Miró hacia dentro: veíase un patio pequeño y en el fondo
una habitación oscura dentro de la cual se distinguían los maderos de un
telar. Monsalud contempló durante un rato aquel humilde interior, y
copiosas lágrimas se agolparon a sus ojos.

De repente una mujer de edad madura apareció en la habitación del telar,
moviendo los trastos de un lado para otro y barriendo después. Volvíase
de vez en cuando hacia un sitio donde debía de estar otra persona con
quien hablaba, a juzgar por sus gestos expresivos. Junto a la mujer
apareció luego un perro, que saltando y enredando entre sus pies la
estorbaba en su faena, recibiendo un ligero escobazo que lo decidió a
salir al patio.

Salvador, que se había detenido en la puerta para gozar en silencio y a
solas por un instante del inefable sentimiento que llenaba su alma y
para regocijar su imaginación con la idea del contento que su madre
recibiría al verle, no pudo por más tiempo refrenar su impaciencia y
empujó suavemente la puerta.

---No me espera---dijo para sí oprimiéndose el corazón que parecía
querer saltársele del pecho.---¡La pobrecita se sorprenderá y se
alegrará tanto\ldots! Este momento vale por todas las pesadumbres que ha
padecido durante mi ausencia.

La puerta rechinó, y el perro fue saltando y gruñendo amorosamente al
encuentro de Salvador. Este se precipitó en el interior de la casa. Doña
Fermina mirando hacia el patio muy sobresaltada, vio al joven que hacia
ella corría con los brazos abiertos, diciendo: «¡Madre, madre, aquí
estoy!» La buena mujer abalanzose a recibirle con expresión de frenético
contento; mas al tocarle con sus manos y al verle casi en sus brazos, su
semblante se alteró de súbito, lanzó una exclamación de espanto, y
cerrando los ojos y echando la cabeza atrás, cual si descargase sobre
ella el rayo de instantánea muerte, cayó sin sentido al suelo. Sus
labios contraídos apenas pronunciaron esta frase, empezada con ardiente
cariño y concluida con terror:

---¡Hijo mío!\ldots{} ¡¡francés!!

\hypertarget{viii}{%
\chapter{VIII}\label{viii}}

El militar, aturdido por tan inesperado como funesto accidente y no
comprendiendo bien lo que había oído, creyó que la excesiva alegría la
había desconcertado; mas antes de acudir a los remedios que el paroxismo
reclamaba, hincose en tierra, y besando y abrazando a su madre, la llamó
con los nombres más tiernos y afectuosos, seguro de que su voz la
despertaría. Salvador no había visto aún a otra mujer que en la estancia
estaba: era una vieja flaca y amarillenta, de ojos ardientes y vivos
como ascuas, descarnadas y picudas manos, una de las cuales oprimía el
puño de un bastón negro, mientras la otra se alzaba acompasadamente a la
altura de la cara, para servir de signo visible y movible a su extraño
lenguaje. No la vio Monsalud hasta que se acercó a él, y poniéndole los
cinco amarillos palitroques de su mano sobre la pechera del uniforme, le
dijo con terrible ironía:

---Acábala de matar, verdugo, acaba de matar a tu santa y buena madre.

Salvador miró a la vieja, y aunque de antiguo la conocía, su triste
aspecto y la áspera y desapacible voz produjéronle impresión muy
extraña, especie de frío intenso y doloroso en el corazón, cual si con
una aguja se lo atravesasen, erizamiento nervioso y acritud en los
dientes, como lo que se siente al contacto de las cosas agrias y
heladas.

---Por Dios, doña Perpetua, dígame Vd. ¿qué tiene mi madre?---exclamó el
joven.---¿Está mala?

---¿Eres tú la causa y lo preguntas?---añadió la vieja, poniendo su mano
sobre la frente de la desmayada.

Luego paseando sus dedos por la pechera del levitón de Salvador, y
tentando la botonadura adornada con águilas, y metiéndolos después entre
la lana del sombrero y deslizándolos por las carrilleras de cobre, dijo:

---¡Traes sobre ti esta infernal vestimenta francesa, y preguntas lo que
tiene tu madre! ¡Pobre Ferminita! ¡Se resistía a creer tan grande
infamia en el hijo que llevó en sus entrañas y crió a sus pechos! ¡Pedía
a Dios fervorosamente que no fuese verdad lo que le habían dicho; su
alma se consumía en hondas tristezas, y sin consuelo pasaba las noches
llorando tanta afrenta! La muerte del hijo que perece en los campos de
batalla destroza el corazón, pero no afrenta; la traición del hijo
desvergonzado que comete la infamia de pasarse al enemigo, es el más
vivo de los dolores de una madre española.

---Usted está loca, madre Perpetua---dijo Monsalud rechazando a la vieja
con desdén.---Mi madre es una mujer sencilla: ya comprendo todo. Vd. y
el cura le han trastornado el juicio con eso de traiciones y afrentas.
Honrado soy. Mi buena madre no me aborrecerá por lo que he hecho.

---¡Monstruo!---gritó la vieja agitando el palo.---Huye de aquí. Vete
con esos herejes que te han catequizado: vete con Satanás que es tu amo;
vete al negro infierno que es tu casa. Deja a esta santa mártir que ya
te ha llorado como perdido para siempre. No eres su hijo: tú no puedes
haber nacido en esta casa, ni en este honrado país\ldots{} Vete, vete,
hereje, judío; mas ¿qué digo? ¡francés!

El apostrofado miró a la vieja; mas sin acobardarse siguió esta
vituperándole con la firmeza y el aplomo de quien tiene la seguridad de
ser respetada. Vestía doña Perpetua el traje de las antiguas dueñas, con
toca blanca rizada y limpia, manto y saya negros, pendiente de la
cintura un luengo rosario y del pecho cruz de madera sencilla. A pesar
de los muchos años, su talle era derecho y apenas se encorvaba un poco
al andar. Indudablemente había en el aquilino perfil de la vieja cierta
energía majestuosa que hacía recordar, a quien las hubiese visto, las
rigurosas y ceñudas sibilas creadas por la inspiración artística.
Acartonada y seca no tenía la repugnante escualidez con que nos pintan a
las brujas. Expresábase con vigor y hasta con elocuencia, y su voz
retumbaba en los oídos como una campana de mucho uso, mas no rota
todavía.

Para que nuestros lectores no carezcan de todas las noticias necesarias
respecto a tan singular tipo, les diremos que la madre doña Perpetua
tenía cien años cabales, no hallándose ciertamente en proporción su
acabamiento con su mucha edad, que a la vista no parecía exceder de los
setenta. Era una doncella secular nacida en la Puebla de Arganzón a poco
de establecerse en España Felipe V, y que nunca había salido de aquel
pueblo. Dedicose desde su juventud a obras piadosas, mas sin aficionarse
al claustro: gustaba de la independencia y de andar de casa en casa
comadreando, y trayendo y llevando noticias, dichos e ideas, libando
aquí y melificando allá cual las abejas. Así creció y fue echando días y
años como el siglo, y pasaron ante ella tres generaciones de pueblos y
tres generaciones de reyes y veinte guerras, y ella pasó de un siglo a
otro como quien atraviesa una puerta para pasar de la sala a la alcoba.

Su vida austera y los buenos consejos que daba para reconciliar
matrimonios y dirimir contiendas y transigir desavenencias y acomodar
caracteres, juntamente con su buena manderecha para establecer la
concordia en todas partes, diéronle gran reputación en la villa.
Respetábanla mucho, y cuando abría la boca, conticuere omnes. Como era
tan larga su vida y había visto tanto bueno y tanto malo y tenía mucha
experiencia de las cosas físicas y morales, tomábanla todos por
consejera. Sabía curar males de varias clases, y conocía mil salutíferas
hierbas y untos, además de toda la farmacopea casera, mezclando en
hórrido caos la medicina y la religión, lo terapéutico y lo
supersticioso. Enciclopedia del alma y del cuerpo, reunía todo el saber
y todo el sentir de su país en aquella época.

Rezaba por todos los muertos y reía por todos los nacidos. No había
bautizo, ni duelo, ni boda a que no asistiese, disfrutando de lo mejor
del festín, cuando lo había. Sabía contar especies diversas de cuentos
interesantes, algunos heroicos, muchos de pícaros tahúres y guapos, y
los más de devoción o de brujerías, males de ojo, miedos y otras cosas
divertidas que embobaban a los chicos y a las mujeres. Ningún asunto
doméstico o social o religioso tenía para ella secretos, y era la
ciencia suma en teología de aldea, en economía al pormenor, en culinaria
y en filosofía burda.

Doña Fermina a los pocos minutos, comenzó a querer volver de su síncope.
La vieja había traído agua en una escudilla y le rociaba el rostro
diciendo:

---Ya vuelve en sí; aunque para ver lo que tiene delante, más valiera
que sus ojos no se abrieran jamás a la luz. Vete, te digo, tu madre te
llora muerto; no turbes la paz de su alma poniéndotele delante en esa
forma aborrecible.

Monsalud sin escuchar a doña Perpetua, alzaba a su madre del suelo y
cuidadosamente la sentó en su sillón. Sosteniendo con sus manos la
cabeza de la infeliz mujer, le decía:

---Madre, soy yo, soy Salvador, el mismo de siempre, el hijo querido.
¿Por qué se ha asustado Vd. al verme? El vestido no hace al hombre.

Doña Fermina, viendo el rostro de su hijo cerca de sí, le dio mil besos
amorosos; mas después apartó la cara y extendió los brazos para rechazar
al joven.

---¡Mi hijo\ldots{} francés!\ldots---repitió con el mismo tono de
angustia y terror\ldots---¡Ese traje!\ldots{} ¡Era verdad!

---¡Y el muy bribón se empeña en seguir aquí atormentándote,
Ferminita!---exclamó con desabrimiento la vieja.---¿Hase visto
desvergüenza semejante?

---¿Qué delito he cometido?---dijo Monsalud con viva congoja estrechando
entre las suyas las heladas manos de su madre, y de rodillas ante
ella.---¿Qué habré yo hecho para que Vd. se desmaye, madre, cuando me
ve, y esta buena mujer me manda huir?

---¿Qué has hecho?---repitió la madre con estupor.---¡Te has pasado a
los franceses, estás maldito de Dios y de los hombres, tocado de herejía
y perdida para siempre tu alma y contaminada yo también por haberte
parido y criado!

---¡Qué horribles palabras y qué espantosa idea!---exclamó el joven
procurando reír, pero con el alma destrozada de vergüenza y
dolor.---¿Tantos males ocasiona este capote que llevo? ¡Oh! madre
querida, yo conocí que hacía mal, yo resistí, conociendo que era una
falta servir a los enemigos de mi patria; pero me moría de hambre, y
además mi tío tenía mucho empeño en que yo sirviera a los franceses. Una
vez dado este paso, ya no puedo volver atrás, porque el honor me prohíbe
vender a los que me han dado un pedazo de pan para vivir y una espada
para que los defienda. Si por esto he perdido el amor de mi madre, de la
única persona que en el mundo me ha querido, de la que me dio la vida,
de aquella a quien he consagrado siempre la mía, será porque algunos
malintencionados habrán emponzoñado su alma con bajos sentimientos.

---No, yo te amo siempre---dijo doña Fermina, no pudiendo resistir el
ansia vivísima de besar a su hijo y regar con ardientes lágrimas sus
mejillas, aunque doña Perpetua extendía a menudo entre los dos sus manos
de cartón;---yo siempre te quiero, pero he hecho juramento ante Dios de
no admitirte bajo este techo ni darte mi bendición, ni llamarte hijo, si
no abjuras tus errores y maldices tus banderas infernales y reniegas de
ese vil Rey y tornas a la patria y al deber\ldots{} Mi conciencia me
exigió este juramento y lo he prestado por consejo de respetables
personas a quienes debo consuelos tiernísimos en esta última y tan
amarga desventura que ha caído sobre mí.

El joven, cubriendo con ambas manos su rostro, lloró; mas de súbito
estalló una violenta indignación en su alma, y apartándose de las dos
mujeres, púsose en el centro de la pieza.

---Mi honor---gritó con voz alterada y resuelta,---me impide desertar;
pero si pierdo el amor de mi madre, y se me arroja de mi casa porque no
quiero ser desleal y perjuro, no quiero vivir. Aquí tengo una
espada---añadió desenvainándola,---y no me falta valor para atravesarme
con ella el corazón.

Doña Fermina se arrojó llorando en brazos de su hijo. La mujer secular
permanecía silenciosa, fría, clavada en su silla, contemplando la
patética escena como una estatua de cartón que dentro de su pasta
encolada tuviera un alma observadora. Sus ojos negros clavábanse en el
joven con fijeza aterradora.

En aquel instante entró un nuevo personaje. Era un anciano fornido y
alto, de rostro sanguíneo, duro y tosco, mas no desagradable por cierto,
mirar franco y campechano que le animaba y hasta le embellecía. Su
cabeza calva, apenas se exornaba económicamente con un cerquillo de
blancos pelos esporádicos sobre las sienes y en el occipucio y en cuanto
a su cuerpo era bravío, imponente, recio, como de varón hecho a las
intemperies, a las luchas con hombres y elementos. Vestía negro traje
talar, llevado con desenvoltura y abierto por delante para poder
introducir fácilmente las manos en el bolsillo o cuadrarlas en la
cintura, como frecuentemente lo hacía aquel hombre, dueño de dos manos
enormes, velludas, que sabían llevar el arado, la espada y la hostia.
Era D. Aparicio Respaldiza, cura de la Puebla de Arganzón.

Mirando al mancebo, más bien con lástima que con rencor, le dijo:

---Ya sabía que estabas aquí, desgraciado. Te hacíamos muerto, muerto
con la muerte de la deshonra que deja el cuerpo vivo. El alma se va y
queda la vergüenza.

Luego acercándose a doña Fermina, que deshecha en lágrimas, recibía
consuelos y caricias de la beata, le dijo:

---¡Señora Fermina, valor!\ldots{} El sentimiento materno es el más
fuerte de todos. No trate usted de vencerlo: al contrario, desahogue su
pecho, llore hasta mañana. Este hijo muerto no es quizás perdido para
siempre, y puede resucitar, si se abraza a la cruz de la patria. Yo seré
el primero que le reciba en mis brazos.

---Y yo---repitió la beata sin que se mostrase en la engrudada máscara
de su rostro, compasión, ni alegría, ni sentimiento alguno.---Yo también
le abriré mis brazos.

---Hijo mío---dijo doña Fermina poniéndose de rodillas ante Salvador y
cruzando las manos,---vuelve en ti; deja esos hábitos infernales,
abandona a los que te han seducido, torna a la patria y recibirás la
bendición de tu madre y el amor que siempre te he tenido y te tengo a
pesar de tu horrible pecado. Hazlo por Jesucristo crucificado, por la
religión que te enseñé, por el agua que en el bautismo recibiste, por el
pan eucarístico que has recibido en tu cuerpo; hazlo, por mí, por mi
honor y buen nombre, que para siempre he perdido en este pueblo, por mi
tranquilidad que no recobraré sin ti; hazlo por el señor cura de nuestra
aldea que te enseñó los mandamientos y la doctrina y la lectura y la
escritura y el latín, con lo poco que sabes; hazlo por la santa doña
Perpetua que nos da tan buenos consejos y más de una vez te ha
entretenido contándote tan bellas historias; hazlo, en fin, por todos
los que te aman en esta villa y en el lugar de Pipaón, donde no sé si
por ventura o eterna desdicha mía naciste.

Monsalud, enternecido por voz tan elocuente que agitaba hasta lo más
hondo su alma, como la tempestad el Océano, se había sentado en un
escabel y con los codos en las rodillas y la cabeza encajada entre las
palmas de las manos, lloraba en silencio. El témpano colosal y
endurecido de su entereza se desleía poco a poco.

---Y lo que es ahora---dijo el cura para favorecer el deshielo,---los
franceses van a ser destrozados. ¡Pobrecitos de los que se unan a ellos!

---Bueno---dijo Salvador alzando de repente la cabeza;---déjenme que lo
piense. Eso no se puede decidir en un momento: los que estamos
acostumbrados a cumplir con nuestro deber, y a obedecer a nuestros
superiores\ldots{}

---No hay ningún superior que tenga sobre ti más autoridad que tu
madre---dijo el cura paseándose por la habitación, con las manos a la
espalda;---tu madre, personificación viva de la patria, que a todos sus
hijos gobierna y dirige.

Doña Fermina corrió a abrazar a su hijo, besándole cariñosamente en la
frente y en las mejillas.

---Querido niño mío---le dijo,---veo que estos dos excelentes amigos te
van convenciendo. Dejarás a esos perros franceses, devolviéndome la
tranquilidad y poniéndome en paz con mi conciencia y con Dios. Siéntate,
descansa; te esconderemos para que no puedan verte los vecinos con ese
endiablado uniforme\ldots{}

---Es una imprudencia que le tengas en tu casa mientras de todo en todo
no se convierta---dijo la santa con severidad.

---¿Y qué importa?---repuso doña Fermina ofendida de la intolerancia de
su consejera.---Mi hijo está arrepentido. El pobrecito estará hambriento
y fatigado. Lo primero es que tenga salud.

---Puede quedarse---afirmó el cura, menos celoso que la
beata.---Salvador es un buen muchacho\ldots{} ha dicho que lo
pensaría\ldots{} Tiene buen natural y mucha inteligencia\ldots{} y sobre
todo, el deber le ordena servir a la patria. Aquí donde me ves---añadió
deteniéndose en medio de la estancia en actitud marcial,---estoy
disponiéndome para salir por ahí con otros amigos\ldots{} Ya sabes que
mi puntería es la mejor de toda la tierra de Álava. Hemos decidido
organizar una partidilla, para auxiliar a las de Longa. ¿Qué te parece
mi proyecto? ¡Oh, admirable! Los hombres se deben a su patria, y es
preciso que nosotros, los que estamos en cierta jerarquía demos el
ejemplo a los demás\ldots{} La ocasión es solemne, y ningún español
puede permanecer en su casa: Wellington está cerca y es preciso
ayudarle. ¿Qué tal? ¿Te animas? Yo no espero sino a que venga de
Peñacerrada D. Fernando Garrote, que es hombre muy entendido en guerras,
para partir con él\ldots{} Serás un buen escopetero, Salvador.

---Siéntate, hijo---indicó la madre, observando que el joven no se
entusiasmaba excesivamente con el bélico ardor de Respaldiza.---Voy a
aderezar algo de comida. Estarás muerto.

---No tengo ganas de comer---respondió el mozo, profundamente abstraído.

La madre le miró con desconsuelo, viendo sin duda en su abatimiento
pensativo la señal de nuevas vacilaciones.

---He dicho que lo pensaría, ¿no es eso?---murmuró Monsalud sin pensar
en comer.---Pues bien, lo pensaré\ldots{} déjenme pensarlo todo el
día\ldots{} Es cosa grave\ldots{} El convoy que he custodiado y que
lleva el general Maucune, sale ahora mismo; pero yo no saldré hasta
mañana con el convoy grande.

La madre y los dos amigos permanecieron mudos, y sin pestañear le
observaron. Luego abrazó el hijo a la madre, y sonriendo dijo:

---Volveré más tarde.

Cuando salió de la habitación, la vieja se expresó así:

---¡Perdido, perdido para siempre!

Más optimista y generoso el cura, tranquilizó a la afligida madre,
diciendo:

---Es nuestro.

\hypertarget{ix}{%
\chapter{IX}\label{ix}}

Para mayor claridad de sucesos que han de venir, Dios mediante, no
estará de más referir algunos antecedentes relativos a las principales
personas de esta historia. Era doña Fermina natural de Pipaón y rama del
tronco de una honradísima e hidalga familia; mas Dios quiso que en ella
y su hermano tuviese fin el lustre de su casa, pues quedando huérfanos
en edad temprana, mientras él derrochaba en Madrid toda la fortuna
paterna, sufrió ella una desgracia irreparable que por siempre la
condenó a la oscuridad y a la vergüenza, con lo cual acabó para el
mundo, y en el olvido quedaron las nobles prendas de su alma y superior
mérito.

Una herencia de poquísimo valor y un pleito enfadoso la obligaron a
establecerse en la Puebla en 1811. Vivía allí con modestia y muy
retirada; pero la trataban algunas personas, y entre ellas asiduamente
doña Perpetua y el cura, que bien pronto ejercieron en su ánimo grande
influencia, convidándoles a ello la gran sencillez y bondad de la
piadosa mujer. Doña Fermina no era vieja aún; pero habíala desfigurado
la negra tristeza que en todos tiempos llenaba su alma, y finalmente el
pesar por la ausencia de su hijo. Los amores de este con cierta joven de
la villa, y sus cuestiones y disputas con otro muchacho, hijo de
acomodados padres, obligaron a doña Fermina a enviarle a Madrid, donde
hizo lo que ya sabemos, y se entregó en cuerpo y alma a los franceses.

Después de la conferencia antes referida, salió Monsalud a la calle, y
vagó por las principales del lugar, tan ocupado por sus pensamientos que
a nada atendía, ni paró la atención en la mucha gente que le miraba. Su
entereza había sido muy quebrantada por la lastimosa escena de la
mañana, y la deserción que antes le parecía un hecho deshonroso, contra
el cual a voces protestara su pura conciencia, se le representaba al fin
no sólo como natural, sino como en alto grado laudable y meritorio. El
grande amor que a su madre tenía, y el prestigio de las dos
religiosísimas personas de que se ha hecho mención, habían trastornado
sus ideas, abierto nuevas vías a su pensamiento, y cambiado el modo de
ver las cosas de la vida y especialmente de la guerra.

---Es indudable---dijo para sí,---que el deber que hacia mi patria tengo
anula todos los demás deberes\ldots{} Al nacer contraje con mi patria el
compromiso tácito de defenderla, y este compromiso anula también todos
los juramentos posteriores\ldots{} Váyanse los franceses con doscientos
mil demonios\ldots{} Pero una conciencia honrada ¿puede consentir el
abandono traidor de los que nos han hecho un beneficio, y el hacer armas
contra ellos, aunque sea en las filas de la patria? No, en caso de
desertar renunciaré a mis grados militares, romperé mis charreteras y
dejando a los franceses, me retiraré a mi casa resuelto a no volver a
tomar un fusil en la mano.

Así discurría, balanceando su voluntad de un lado para otro, pero
inclinándose más del lado de la deserción. Al fin sus pensamientos
tomaron vuelo por distintos espacios, y puso en olvido a franceses y
españoles: en aquel mar agitado de sus ideas sobrenadó lo que sobrenada
siempre, y todo lo demás se fue al fondo. Mirando las verdes copas de
unos árboles que se elevaban sobre los tapiales viejos de una huerta
entre irregulares tejados, dijo hablando consigo mismo:

---¿Estás ahí, Genara? Todo sigue lo mismo, árboles, casa, cielo y
tierra, el aire y el sol, y lo mismo también mi corazón, que antes
dejará de latir que de quererte.

Los redobles de tambor que sonaron en las inmediaciones del pueblo le
obligaron a seguir adelante.

---Como la división no se pone en marcha hasta mañana
temprano---dijo---tengo tiempo de pensar lo que debo hacer; vamos al
campamento y esta noche\ldots{} Esta noche veré a Genara aunque me sea
preciso degollar a su madrastra y ahorcar a su abuelo.

Pensándolo así, fue al campamento llamado por su obligación; mas nada le
ocurrió en él digno de contarse, por lo cual apresuramos la narración,
acortando el día y transportando a nuestros lectores a la apacible y
oscura noche, cuando Monsalud dirigiose solo y con el alma llena de
ansiedades entre dulces y dolorosas, a aquellos mismos tapiales de
tierra que por la mañana vimos, descollando sobre ellos la frondosa
arboleda de una huerta. Llegó el joven y reconocidos los contornos para
ver si alguien le observaba, cerciorado al fin de que en las callejas
contiguas no había curiosos ni rondadores, tomó una piedrecilla y la
arrojó contra la única ventana de la casa que a la huerta daba. Luego
articuló hábilmente unos silbidos que parecían el canto de un pájaro
nocturno; mas ninguna señal de la casa contestó a su extraña música
hasta la tercera repetición.

Abriose al fin la ventana, pero no conociendo Salvador la persona que en
el oscuro hueco apareciera y receloso de que fuera el suspicaz abuelo o
la vigilante madrastra, calló y ocultose en las densas sombras que
proyectaban las cercanas paredes. Poco después creyó sentir pasos en la
huerta y el tenue ruido de las matas que se rozaban unas con otras,
apartándose para dar paso a un vestido. Acercose entonces muy quedito a
la empalizada que tapaba la entrada de la huerta, y que en sus tablas
carcomidas tenía grietas, agujeros y hendiduras suficientes para dar
paso libre a la palabra durante la noche y aun a la vista durante el
día. El joven conocía aquellos viejos maderos, la disposición de sus
huequecillos y claros como se conoce el traje que se ha usado muchos
años. Al pegarse a ellos su corazón más que su oído le dio a entender
que por dentro suspiraba una persona.

---Generosa---dijo aplicando los labios a una juntura por donde
difícilmente podía pasar un dedo.

---Salvador---repuso desde el contrario lado una dulce y conmovida voz
como gemido del viento entre las hojas.---¿Eres tú?

---Aquí estoy, siempre tuyo, siempre queriéndote, muriéndome, Genara,
por ti ---dijo Monsalud oprimiendo su cuerpo contra las frías y duras
tablas.---Dime si me has olvidado, si quieres a otro. Genara, estás aquí
y no puedo verte. ¡Maldita noche!\ldots{} ¿Me has olvidado? ¿Me quieres
todavía?

---Sí---repuso desde dentro la dulce voz,---te quiero. ¿Por qué has
estado tanto tiempo sin escribirme? ¡Cuánto me has hecho llorar!

---Genara---exclamó el joven apoyando su frente abrasada sobre la
madera,---mete tus deditos por esta rendija de la derecha.

Dos blancos dedos aparecieron por la rendija, moviéndose como dos
culebritas. Monsalud, después de imprimir en ellos amorosos besos los
estrujó entre sus manos, hasta que la muchacha los retiró diciendo:

---Me lastimas, Salvador.

---Genara, soy muy desgraciado, soy el más infeliz de los hombres.
Déjame que te vea, pues viéndote, aunque sea un momento, me será menos
penosa la vida.

---¿Por qué eres desgraciado?

---¿Por qué\ldots?---repuso el joven vacilando,---porque no te veo,
porque tu abuelo y tu madrastra no quieren que seas para mí\ldots{}
Genara, por Dios, rompamos estas tablas.

---¿Estás loco? Deja las tablas como están y hablemos. Aún no sé si
podré estar aquí mucho tiempo.

---¿Los de tu casa duermen?

---Sí; pero mi abuelo tiene el sueño muy ligero, y como todos hemos de
madrugar mañana para ir a Vitoria, se ha acostado vestido, y al menor
ruido, Salvador, saldría como un león.

---¿Te vas a Vitoria?

---Sí, el abuelo teme que los franceses destruyan esta villa. Allá
estamos más seguros\ldots{} ¿Irás tú por allá?

---Tal vez.

---Pero no me has dicho las causas de tu desgracia. Yo también soy
desgraciada. Tengo un pesar que me destroza el alma. ¿Sabes por qué?
Porque te quiero, Salvador---dijo la muchacha con acento
quejumbroso,---porque te quiero mucho, porque desde hace dos años, desde
que tú y tu madre vinisteis a estableceros en esta villa, te estoy
queriendo.

---¿Lloras, Genara?---preguntó Monsalud, oyendo los sollozos de su
amiga.

---Sí, lloro\ldots{} Pero de ti depende que me muera de dolor o que sea
muy feliz. Respóndeme.

---¿A qué?

---Salvador, Salvador de mi alma, en la Puebla se ha dicho que te habías
pasado a los franceses. Hoy mismo dijo mi abuelo que estabas entre los
vándalos que llegaron anoche. Yo no he querido creerlo, se me ha
resistido creerlo: dime si es verdad, dime si te has pasado a los
franceses; y si es cierto, Salvador, no volverás a oír una palabra de mi
boca, ni me verás. Genara ha muerto para ti. Genara te aborrece.

Monsalud se quedó yerto y frío y sin habla. Helado sudor corría por su
frente.

---Genara---dijo haciendo un esfuerzo para traer la palabra de su
agitado corazón a sus trémulos labios,---¿por qué has de tomar tan a
pechos\ldots?

---Contéstame pronto---repitió la voz.

El joven vaciló un momento y después dijo:

---Pues bien; es mentira.

---¡Salvador, has dicho que es mentira!---exclamó Genara alzando la
voz.---¡Bendita sea tu boca! ¡Bendita sea tu alma! Todo mentira;
invenciones de la gente, envidia también de tus buenas prendas.

---Invenciones, envidia---repitió sordamente Salvador.

---Pues tú me lo dices, lo creo---dijo la muchacha.---Nunca me has dicho
sino la verdad. No sé de dónde ha sacado la gente tal noticiota. Dijeron
que te habían visto hoy por el pueblo, vestido con un uniforme verde y
un sombrero de piel.

Monsalud calló.

---Hace un momento, Salvador mío, me quedé dormida; soñé primero con tu
uniforme verde y tu sombrero de piel, adornado con un águila dorada. ¡Me
causabas horror! A pesar de tanto como te he querido, viéndote de aquel
modo me parecías el más horrible, el más espantoso de los hombres.

Salvador sentía en su garganta un cerco de hierro que le ahogaba. Era la
gola con la insignia imperial. Bajando hasta su pecho le mordía el
corazón, y el águila majestuosa que exornaba su frente no le hubiera
quemado el cerebro con más violencia, si fuera una llama. El desgraciado
joven sentía en su interior una ansiedad semejante a la agonía que
precede a la muerte.

---Pero después---prosiguió la joven,---tuve otro sueño mejor. Soñé que
lo de pasarte a los franceses era mentira, como has dicho, soñé que
volvías a la Puebla vestido de paisano, pobre, pero con honra; que
volvías después de haber estado combatiendo con los franceses en las
filas de Longa, de Pastor o de Mina\ldots{} ¿Estás de paisano? Cuéntame
lo que has hecho durante ausencia tan larga.

---Todo te lo contaré. Pero dime; si yo hubiera cometido la infamia, la
deslealtad, la alevosía de servir a los franceses, ¿es cierto que
habrías aborrecido al pobre Salvador que lo mismo te quiere hoy que
ayer?

---No me lo digas---contestó la joven.---¿Por qué se quiere a las
personas? ¿Por el rostro? No lo creas. Se quiere a las personas por las
prendas del alma, por el valor, por la honradez, por la generosidad, por
la lealtad, por la dignidad, por la nobleza.

Monsalud no oía estas palabras. Sentíalas en su corazón como saetas que
se lo atravesaban de parte a parte.

---El que en una guerra como esta---continuó la joven,---da de lado a
sus hermanos que están matándose por echar a los franceses; el que ayuda
a los enemigos, a esa caterva de herejes, ladrones y borrachos, es un
traidor cobarde, un ser despreciable, un Judas. Los perros de España
merecen más consideración que el que tal vileza comete. Si tú la
cometieras, Salvador, no sólo te aborrecería, sino que me mataría la
vergüenza de haberte querido.

Monsalud apuró con resignación este cáliz de amargura. Las palabras de
la vehemente muchacha, juntamente con el recuerdo de la escena ocurrida
en la casa materna, le hicieron comprender la inmensidad del sentimiento
patrio. Todo lo que en él había de violentamente salvaje desaparecía
ante la grandeza de su lógica. Contra aquello ¿qué podían José ni
Napoleón con todos sus ejércitos? Sobre aquel sentimiento, sobre aquel
odio de las muchachas a todo el que no fuera patriota, descansaba la
inmortalidad nacional, como una montaña sobre sus bases de granito.
Monsalud lo vio todo, vio aquel gigante cruel y sublime, salvaje pero
grandioso, y se inclinó ante él abrumado, vencido, resignado,
comprendiendo su propia miseria y la magnitud aterradora de lo que tenía
delante.

---Genara---dijo con voz conmovida,---mete tus deditos por esta rendija.
Me muero de dolor; soy el más desgraciado de los hombres.

---¿Por qué?---dijo Genara poniendo su alma en las yemas de los dedos y
echándola a la calle.---Yo estoy contenta\ldots{} ¿Pero Salvador, qué es
esto que toco? Un botón de metal, y otro, y otro. ¿Tienes uniforme?

---Me compré un chaquetón en Valladolid, cuando venía para acá---repuso
turbado el militar.---Así se usan hoy.

---Salvador, ahora que te has movido, ha sonado contra el suelo una cosa
de hierro. Parece un sable.

---¿Pues no te dije que lo tenía? Sí, me lo dieron unos guerrilleros en
Nájera.

---¿Has estado con los guerrilleros?---preguntó la joven con
entusiasmo.---¡Y no me lo habías dicho! ¡Oh, con los guerrilleros!
¡Bendígalos Dios!\ldots{} Salvador, entra tu mano por este agujero
grande que hay más arriba\ldots{} ¿Con que has estado con los
guerrilleros?

La mano de Monsalud pasó de la calle al jardín, y el joven sintió sobre
ella los labios de la joven, quemándole como ascuas, que se le metían
por las venas adentro hasta el mismo corazón.

---Salvadorcillo---dijo la joven, acariciando la mano de su
amigo,---¿esta mano ha matado muchos franceses?

A Monsalud, después del anterior fuego, se le heló la sangre en las
venas, al oír esto.

---Siempre que oigo contar hazañas de guerrilleros---prosiguió
Genara---me acuerdo de ti. A todos me los figuro como tú, y me parece
que nadie puede ganarte en valentía. Sueño con las sangrientas batallas
en que perecen muchos franceses. ¡Ay! si yo fuera hombre, no quedaría
con vida ni uno solo de esos perros. Cuando voy a la iglesia y oigo al
cura contarnos en el púlpito las ventajas de los guerrilleros; cuando
vienen a casa los amigos de mi abuelo y hablan de las batallas ganadas
por Longa y Mina, no puedo apartar de ti mi pensamiento. Me moriría de
felicidad si oyera tu nombre entre tantas maravillas de valor. Los
buenos soldados de España se me representan como San Miguel, ángeles
armados y hermosos que destrozan al dragón. ¿Eres tú de esos, Salvador;
eres tú un San Miguel?---añadía con exaltación admirable.---Dime que sí,
y te querré más todavía. Dime que has matado muchos enemigos, que has
defendido a España contra esos borrachos del infierno, dime que te has
bañado en su sangre maldita y machacado sus horribles cabezas, y te
querré más que a mi vida, te querré como a Dios\ldots{} Nosotros somos
Dios, Salvador; nosotros los españoles somos Dios y ellos el demonio,
nosotros el Cielo y ellos el Infierno. Así lo dicen el cura y mi abuelo,
y tienen mucha razón.

---¡Mucha razón!---repitió Monsalud por decir algo.---Genara, tu
exaltación me conmueve. Ahora veo que hay otra religión además de la que
está en el catecismo, la religión de la patria. Los hombres la practican
y las mujeres la sienten. Si la fe en Dios mueve las montañas, la fe de
esa otra religión también las mueve. Con ella el heroísmo y el martirio
son cosas fáciles\ldots{} Genara, yo te juro ante Dios que nos está
mirando desde lo más alto del Cielo, que haré todo lo posible para
elevarme como tú hasta el último grado en la fe de la madre España. Mis
proezas no han sido hasta ahora muy grandes; pero aún hay franceses en
la tierra. Soy joven, fuerte, robusto: soy soldado de la patria. Morir
por ella y morir por tu amor me parece lo mismo. Genara de mi alma,
quiereme mucho.

---Salvador mío, ese es el lenguaje que me gusta oírte---dijo la
muchacha.---Estamos en guerra. Todo hombre que no sea guerrero hoy no
merece más que desprecio. ¿Te gusta a ti la guerra, Salvador? Di por
Dios que sí, dímelo.

---Extraordinariamente, Genara. El corazón que no palpita por estas tres
cosas, Dios, la mujer amada y la victoria, no es corazón de español ni
de hombre.

Sintiose el suave estallido de algunas tablas. Genara sacudía la
empalizada.

---¿Qué haces?---le preguntó Monsalud.---Esto se mueve.

---Salvador, amigo querido de toda mi vida---dijo con pasión la
muchacha---¡Malditas sean estas tablas que nos separan! Empuja un poco
de ese lado.

---Se romperán, Genara. Esto no es tan fuerte como parece---indicó el
joven con terror.

---Quiero verte---añadió Genara con voz que se ahogaba entre sollozos y
suspiros.---Hace tanto tiempo que no te veo\ldots{} y si ahora te
vuelves con los guerrilleros, y tu arrojo te causa la muerte en una
acción\ldots{} no te veré más\ldots{} ¡Ay! estas condenadas tablas no
ceden.

---No---repuso el mancebo tranquilizándose.

---Oye---dijo la doncella con exaltación,---si es tan grande tu empeño
por entrar y verme, no es menor el mío. Nada más triste que hablar y no
poderse ver las caras. ¿Estás pálido, Salvador, estás tostado del
sol?\ldots{} Oye lo que me ocurre. Mi abuelo tiene la llave de esta
puerta sobre la mesa de su cuarto. Ahora duerme\ldots{} puedo entrar de
puntillas y cogerla. No sentirá nada\ldots{} Aquí está el candado,
hijito\ldots{} Se abrirá fácilmente\ldots{} ¿Conque voy por la llave?

\hypertarget{x}{%
\chapter{X}\label{x}}

---Detente---dijo Monsalud, a quien causaba rubor y angustia la idea de
que al abrirse la puerta, descubriera Genara por su traje el engaño de
su patriotismo y la verdad de su afrancesamiento.---Detente, Generosa, y
reflexiona un momento sobre lo que vas a hacer\ldots{} Te quiero más que
a mi vida; te quiero no por egoísmo, sino con verdadero amor que pone
por encima de todo el bien de la persona amada. No necesito llave para
abrir esta puerta del cielo, Genara: basta un esfuerzo para echarla a
tierra; pero no la romperé, no, porque mi propia estimación y sobre todo
la tuya me lo prohíbe.

---Dices bien, yo estoy loca---murmuró la muchacha.---Acércate; que
sienta yo tu respiración pasando por estas rendijas, Salvador mío. ¿No
te marcharás todavía?

Monsalud, fatigado de la farsa que estaba representando y que repugnaba
a la dignidad y lealtad de su alma generosa, mas sin deseos de ponerle
fin alejándose de la dulce criatura amada, quiso variar de conversación,
entablándola sobre un asunto que no tuviera relación con la guerra, ni
con los franceses, ni con los guerrilleros.

---Niña mía---dijo,---se me había olvidado un asunto del cual pensé
hablarte.

---¿Cuál?

---Durante este tiempo en que no nos hemos visto, he tenido celos,
muchos celos. En Madrid me dijeron que querías al hijo de D. Fernando
Garrote. Recordarás, que cuando éramos novios, él te hacía la corte, que
Garrote y yo nos mirábamos con muy malos ojos, que por haber reñido
primero de palabra y después de obra, tuve que salir de la Puebla
jurándole enemistad eterna. Si después de esto, has tenido la debilidad,
no digo de quererle, porque esto me parece imposible, sino de admitir
sus galanteos, buscaré a ese fatuo y donde quiera que le encuentre, le
mataré.

Contra lo que Monsalud esperaba, Genara no se escandalizó de lo que
acababa de oír ni menos contestó a los agravios del mancebo con mimos y
lloros, según costumbre tan antigua como el mundo. Oyó él tras los
maderos una risita que no le hizo feliz, y después estas palabras.

---¡Qué tonto eres! No hagas caso de eso. Cierto es que Carlos Garrote
me hace la corte y quiere casarse conmigo. Me envía regalitos, ramos de
flores, va a misa a la misma hora que yo, y algunas veces viene con sus
amigos a desgañitarse bajo las rejas de esta casa, acompañado de
guitarras y bandurrias.

---Genara, Genara, me estás destrozando el corazón---exclamó el mancebo
con fuego.---¿Por qué te ríes?

---Me río de él. Y no es mal muchacho, Salvador---continuó
Genara.---Tiene buen porte, muy bueno, sí, y también excelentes
cualidades, sólo que no es amable ni delicado como tú, sino brusco,
serio, y\ldots{}

---Y fatuo y vanidoso y soplado---interrumpió Monsalud.---Veo que no te
disgusta mi enemigo.

---Ni me gusta, ni me disgusta---dijo la doncella, aplicando su boquita
a las hendiduras para que se oyese mejor lo que decía.---Si no le
quiero, tampoco desconozco sus buenas cualidades, especialmente el valor
grande y temerario que ha mostrado en esta guerra. ¿Qué crees tú? Carlos
Navarro, el hijo de D. Fernando Garrote, es la admiración de esta villa
y el honor de todo el país de Álava. Ha corrido por esos mundos con
Longa y Pastor, y todos dicen que no han visto mozo de más arrojo y
bravura. ¿Pues y su tino para la guerra? ¿Y su ciencia militar que nadie
le ha enseñado? Todo lo sabe, y es al modo de los grandes capitanes, que
en un abrir y cerrar de ojos aprenden por completo el arte de pelear. Mi
abuelo asegura, que de Carlos Navarro a Alejandro el Grande va menos que
el canto de un duro. Hace meses, cuando entró en la Puebla después de
haber derrotado a los franceses, todos los habitantes de esta villa
salimos, como en procesión, a vitorearle. ¡Qué día, Salvador! Yo me
acordaba de ti y hubiera querido que estuvieses aquí para ver tanto
entusiasmo. Yo no cabía en mí de puro confusa y exaltada y alegre. No sé
lo que pasaba en mi alma cuando vi a Carlos Navarro en su caballo blanco
entrar triunfalmente cubierto de guirnaldas de flores, con la espada en
la mano y el orgullo de la victoria en los ojos; ¡ay, Salvador! me eché
a llorar.

---¡Te echaste a llorar!---dijo Monsalud con un volcán de celos dentro
del pecho.---No lo digas delante de mí. Eso es un insulto,
Genara\ldots{} me estás matando.

Sin añadir más palabras, golpeó con tanta violencia las tablas, que la
débil empalizada vaciló. Ocupado por el dolor y los celos, que entre
confusiones mil agitaban su alma. Monsalud no advirtió que en el extremo
de la calleja donde tan descuidadamente departía con su tormento, había
aparecido un hombre; que aquel hombre se había acercado con cautela y
puéstose inmóvil y vigilante como a dos varas de la amorosa conferencia.
Cuando la empalizada crujió al recibir los golpes de fuera, dio algunos
pasos más hacia adelante el que parecía fantasma, y entonces le vio
nuestro celoso joven.

Ambos se miraron sin hablar nada, hasta que el desconocido rompió el
silencio, diciendo con voz grave:

---¿Qué hace Vd. aquí?

---Lo que quiero---repuso Monsalud reconociendo al instante la voz de
Carlos Navarro, hijo único del célebre y hasta ahora no conocido D.
Fernando Garrote.---Siga Vd. su camino, que no me creo obligado a
informarle de mi conducta, señor entrometido.

---Ahora veremos quién desfila---dijo el otro sin perder la calma.---Me
parece que tengo enfrente a Salvadorcillo Monsalud, el cual marchó a
Madrid a servir a los franceses.

---El mismo soy---exclamó el militar con brío,---¿qué quieres de mí,
Carlos Navarro?\ldots{} Supongo que traerás una espada.

---No.

---¿Navaja?

---Tampoco. Vengo sin armas. Si las trajera, no las deshonraría
midiéndolas con las de un miserable traidor, con las de un vendido a los
franceses.

---¡Navarro! Llevo un uniforme que no es el tuyo---exclamó Salvador con
violento coraje.---No lo desprecies. El corazón que va dentro de él no
ha cometido ninguna acción villana. Lo mismo puedo matarte con una
espada española que con un sable francés.

---¡Vendido!\ldots{} deja libre la calle. No reñiré contigo. Cuando me
encuentro con un traidor, escupo y paso.

---¡Miserable, cobarde, salteador de caminos!---gritó Monsalud sintiendo
culebrear el rayo dentro de sus venas.---Defiéndete, si no quieres que
aquí mismo te atraviese y envíe al infierno tu alma perversa.

Monsalud desenvainó el sable. Navarro no hizo movimiento alguno hostil,
pero echando atrás el embozo de su capa negra, alargó la mano sin otra
arma que una linterna. El espacio que separaba a los dos enemigos se
inundó de luz.

En el mismo instante la empalizada, que poco antes se estremecía
sacudida con violencia por un hombre, cedió por completo a los esfuerzos
de una mujer, y abierta al fin, dio paso a Genara, que pálida como la
muerte, fue derecha a ponerse entre los dos jóvenes. Alargando sus
brazos podía tocar el pecho del uno y del otro. Lo primero en que se
fijaron sus ojos fue en la gallarda persona del renegado, cuyo brillante
uniforme reflejaba la luz de la linterna en los relucientes botones de
cobre, en el águila, carrilleras, gola y cartera. Genara dio un grito
agudísimo, miró a uno y otro galán alternativamente toda acongojada y
confusa, como quien no cree lo que ven sus ojos y tocan las propias
manos. Monsalud que resuelta y ciegamente iba ya contra su enemigo,
detúvose al ver interpuesta a la hermosa joven.

---Este es Monsalud---exclamó ella con perplejidad
indescriptible.---Navarro, ¿es este Monsalud?

---Por el uniforme francés se le conoce---respondió el guerrillero.

---¡Francés, francés!---gritó la doncella.---¡Tú francés\ldots{}
embustero además de traidor!

---Sí, francés, francés---rugió Salvador;---francés, traidor y embustero
y todo lo que quieras; pero vete de aquí y déjame solo con ese hombre.

---¡Virgen María! ¡Señor mío Jesucristo! Asísteme en este trance
---murmuró la joven.

Después entró corriendo en el jardín, y desde la empalizada y con voz
clara, argentina, sonora, penetrante y exaltada, con voz que no puede
definirse, como no puede definirse la pasión extraña que la inspiraba,
gritó:

---¡Navarro, mátale, mátale sin piedad!

\hypertarget{xi}{%
\chapter{XI}\label{xi}}

---Mátale---repitió alejándose la voz, al mismo tiempo dulce y
guerrera---mátale por traidor y embustero.

Monsalud al oírla, sintió en su corazón frío de muerte; sintiose
cobarde, zumbó en su cerebro la sangre inflamada; su brazo era un
estropajo inerte que apenas podía mover el sable, aquel hierro, trocado
en caña inútil por la súbita congoja del alma\ldots{} El universo entero
se le había caído encima.

---No tengo armas---dijo Navarro sin dar un paso hacia adelante ni hacia
atrás y soltando la linterna.---Puesto que no puedo ni quiero batirme
contigo en lid de caballeros, asesíname, francés; ese es tu oficio.
Asesina al guerrillero de Andía y la Borunda.

La serenidad grave y un poco petulante de aquel hombre, el mirar fijo de
sus ojos, su hermosa estatura, la capa que de los hombros le caía hasta
los pies, dándole el aspecto de una estatua negra, trastornaron a
Monsalud más de lo que estaba. ¿Por qué no decirlo? Tenía miedo, un
pavor, semejante al que infunde la superstición. Todo cuanto veía
parecíale sobrenatural, obra del demonio, obra de Dios tal vez.
Sobreponiéndose a su espanto, dijo:

---Es mentira, la traes bajo tu capa. ¿Tienes miedo?

Con esta pregunta pensó sacarle de su fría impasibilidad; mas el otro
sonriendo con desdén, replicó:

---Salvador, guarda ese chisme y vete con los tuyos.

---Mátale, mátale por traidor y embustero---gritó más lejos, desde la
casa y junto a la puerta que daba al jardín la voz divina y furiosa de
Genara.

Un hecho es este cuyo tenebroso misterio no penetrará jamás con
exactitud el observador; pero es indudable que la pasión amorosa
confundida con el arrebatado sentimiento patriótico que en el alma de la
mujer produce fenómenos extraordinarios, durante las grandes guerras de
raza, está sujeta a veleidades casi increíbles. El fanatismo de Genara
hizo de ella en la ocasión crítica que narramos un ser espantoso; pero
¿es posible pronunciar la última palabra sobre la vengativa saña de su
alma exaltada, sin deslindar lo que de sublime y de perverso había en
los sentimientos que precedieron a la explosión tremenda? La pavorosa
figura bella y terrible, que pedía la muerte de un hombre, pocos minutos
antes amado, encaja muy bien dentro del tétrico cuadro de la época, en
la cual las pasiones humanas exacerbadas y desatadas arrastraban a los
hechos más heroicos y a los mayores delirios. Había en Genara una
entereza romana que de ningún modo podía ser completamente odiosa, y en
sus odios lo mismo que en sus amores no se quedaba nunca a medias.

---Tiene razón---dijo de súbito Monsalud arrojando el arma.---Yo soy el
que debe morir. ¡Navarro, ahí tienes mi sable! Haz el gusto a Genara.

Navarro recogió el sable y entregándolo a su rival le habló así:

---Te he dicho que te marches a tu campamento. Ni una palabra más. No
gusto de conversación.

En el mismo instante sonaron dentro de la casa voces de alarma.

---¡A ese!, ¡al francés!\ldots{} ¡al renegado!---gritaban voces
distintas.

Y viéronse luces y abriéronse puertas y aparecieron algunos hombres y
mujeres con palos y escopetas.

---¡Al pozo con él!---gritó uno.

---¡Ahorcarle!\ldots{} venga la cuerda---gritó otro.

---Meterle en el horno---vociferó un tercero.

De las casas vecinas salieron algunas personas más, y otros aparecieron
por la calleja, de tal modo y con tanta presteza que Monsalud se vio
amenazado por una ruidosa caterva de personas de todas clases.

---¡Muerte al francés!---gritaban.

Recobrando su ánimo se apercibió para defenderse.

La voz de Genara repitió a lo lejos con estridente aullido que parecía
proceder de la garganta de un ángel de exterminio, flotante en el negro
espacio sobre el lugar de la escena, las siguientes palabras:

---¡Por traidor y embustero!

Hubiéralo pasado muy mal, perdiendo seguramente la vida el pobre jurado,
si su propio rival no le defendiese de aquella turba rabiosa, apartando
a unos, haciendo callar a otros y repartiendo a diestra y siniestra gran
cantidad de porrazos.

---Nosotros no asesinamos---gritó.---Dejen libre a este pobre hombre que
se va a su campamento.

Pero ya que no podían acabar con él, siguieron azuzándole con la soez
valentía del número. Protector y protegido, sin dejar por eso de ser
encarnizados enemigos, caminaron largo trecho, abriéndose paso con
dificultad. Gracias a la hora tardía y oscuridad de aquellos lugares, no
acudió más gente al alboroto, que si acudiera, mal lo habría pasado el
del uniforme francés a pesar de hallarse tan cerca sus amigos.
Felizmente para Salvador, a medida que avanzaban, disminuía la molesta
chusma, hasta que al fin y después de andar largo trecho hacia una de
las puertas de la villa, donde se distinguían las fogatas y se escuchaba
el rumor de las fuerzas acampadas, la ruin turba quedó reducida a media
docena de hombres. Navarro les aplacaba y despedía uno por uno, logrando
al cabo quedarse solo con la víctima. Más abrumaba a Monsalud la nobleza
que demostrara en la referida ocasión su enemigo que los insultos con
que le vituperó poco antes.

---Estamos solos---dijo cuando llegaron a la plazoleta inmediata a la
puerta que da paso al puente del Zadorra.---Navarro, agradezco tu
generosidad. Quieres matarme en buena lid, y no has permitido que me
asesinen esos bárbaros. Solos estamos. ¿Es cierto que no traes armas?

---Ya lo he dicho---replicó el otro.

---Lo creo; eres valiente y sé que no las ocultarías por cobardía.
¿Insistes en no batirte conmigo? No me he pasado a los franceses: antes
de servirles, yo no había tomado las armas por ninguna causa. Mi destino
lo ha querido así; pero no estoy deshonrado. Mi desgracia, mi abandono,
mi pobreza lleváronme a las filas del enemigo, y la deshonra consistiría
en abandonarlas durante el peligro\ldots{} Ve, pues, en busca de tus
armas; aquí te espero.

---No quiero---repuso Navarro, con sequedad.---Ya te he dicho que sigas
tu camino.

Y luego con expresión de orgullo que Monsalud no acertaba a explicarse,
añadió:

---Soy guerrillero.

Dijo esto, como si dijera: «Soy Dios».

---Bien, ¿y qué más da que seas guerrillero? Eso prueba que eres
valiente---epuso el otro con aflicción.

---¿Sabes lo que haré si te vuelvo a encontrar junto a las tapias de la
casa de Genara, o si la miras, o si hablas de ella en público, siquiera
digas solamente que la has conocido?

---¿Qué?

---Cortarte las orejas\ldots{} Conque adiós.

Dicho esto volvió la espalda y se alejó tranquilamente, dejando a
Salvador perplejo y dudoso entre aceptar aquel inopinado desenlace de la
contienda o arremeter tras su enemigo para herirle. Una ira loca sucedió
a las dolorosas dudas, y siguiendo a Carlos gritó con toda la fuerza de
sus pulmones:

---¡Navarro, eres un cobarde!

El guerrillero volvió atrás y con provocativa flema le dijo:

---Como están cerca tus amigos; como se les ve desde aquí y podrían
venir al menor ruido, te has vuelto tan bravo, que si te vieran los
gatos de la vecindad, temblarían de miedo.

---Navarro---exclamó Monsalud con frenético coraje,---toma mi sable.
Espérame un instante, un instante no más, mientas voy a que un amigo me
preste el suyo. Entonces me podrás decir lo que te acomode y yo morir o
cerrarte para siempre esa boca insolente.

---Salvador---gritó Navarro comenzando a perder la enfática serenidad
que mostraba,---no me provoques con tus ladridos\ldots{} Te he perdonado
y me insultas, te desprecio y me sigues. Tanto me buscarás, que al fin
has de encontrarme.

Con rápido movimiento se desembozó, dejando en tierra la linterna.

---No tienes tú la culpa---dijo,---sino quien sabiendo lo que eres, baja
de noche a hablar contigo por la reja de la huerta. Genara no te conocía
sin duda o la engañaste con torpes embustes e infames artes.

---Dime todo eso con una espada, con una pistola, con tu sangre,
malvado---exclamó Monsalud rugiendo de ira,---y te contestaré lo que
mereces.

---Pues sea---gritó Carlos, y en el mismo momento oyose sonar el
chasquido del resorte de una navaja, cuya larga hoja brilló en la
oscuridad.

---Yo también traigo la mía---exclamó con júbilo Monsalud, arrojando el
sable.---Navarro, defiéndete.

Envolvían en el siniestro brazo el uno su capote y el otro su capa,
cuando se oyeron pisadas y luego voces alegres que por un callejón
cercano se acercaban.

---Son franceses---dijo Navarro, pateando con furia.

---¿Franceses? ¿Y qué importa?---exclamó Salvador.---Seguirán su camino.
Adelante pues.

---Traidor---gritó el guerrillero,---me has traído a donde están tus
amigos.

---Vamos adonde quieras, elige sitio---repuso el jurado apresurándose a
partir.

Apenas dieron algunos pasos en la dirección que indicara Navarro
marchando delante, cuando se vieron detenidos por media docena de
franceses, borrachos todos como cubas, los cuales reconociendo al punto
a Monsalud, le rodearon, y con gritos y vociferaciones del peor gusto le
saludaron.

---Dejadme, dejadme solo, amigos---dijo este.

---¿Quién es este bravo mozo?---gritó un francés dirigiéndose a Navarro.

---¡Ah! ¿tenéis pendencia?

---Echad mano al paisano y llevémosle al cuerpo de guardia---dijo un
francés.

---Al que le toque---vociferó Monsalud resguardando con su cuerpo el de
su enemigo,---le mataré como a un perro.

---¡Oh! ¡qué bríos!---gruñó otro francés.

---Vaya, basta de disputas---chilló un tercero,---y vénganse los dos a
la taberna con nosotros.

---Tenemos que hacer en otra parte\ldots{} Sigan ustedes
adelante\ldots{}

---Están desafiados\ldots{} Ved las navajas.

Ambos contendientes cerraron y guardaron las armas.

---¿Desafío?---dijo uno que tenía la charretera de sargento.---Ahora
mismo van a ir los dos al cuerpo de guardia. ¿Con que desafío? A fe de
Jean-Jean que no consiento tal cosa.

---¡A la taberna, a la taberna!

Apareció entonces otro grupo de franceses que se unió al primero.

---Vamos, ven acá farsante---gritó Jean-Jean asiendo a Monsalud por el
brazo y tratando de llevárselo consigo.

---Señor espantajo---indicó un jurado amenazando a Navarro,---o toca Vd.
tablas ahora mismo, o le pondremos a la sombra.

Navarro callaba, sofocando su coraje; pero acariciaba la navaja,
dispuesto a atravesar al primero que osase ponerle la mano encima.

Salvador, desasiéndose con gran trabajo de los que entorpecían sus
movimientos, se acercó a Navarro, y comprendiendo que la situación de
este no era muy satisfactoria, dijo en voz alta:

---Señores, déjenme hablar dos palabras a solas con este amigo, y
después nos iremos juntos a la taberna.

---Si me dan tiempo para ir a buscar a dos de mis amigos, a dos nada
más---le dijo Navarro en voz baja,---daré cuenta de ti y de esos
borrachos.

---Carlos---repuso Monsalud,---ponte en salvo. Nada podemos hacer por
esta noche. Estos majaderos no nos dejarán solos.

Trémulo de coraje, el guerrillero no contestó nada.

---Señala sitio y hora para mañana, para pasado mañana, para cuando
quieras.

---El sitio y la hora en que nos volvamos a encontrar---respondió Carlos
echando fuego por los negros ojos.

---El sitio y la hora en que nos volvamos a encontrar---repitió Monsalud
con febril resolución.---Por la noche y por Dios que la hizo juro que
así será.

---Me voy---dijo Navarro con sarcasmo.---Tus amigos te han salvado esta
noche\ldots{} Ahora, cuando yo vuelva la espalda, azúzalos contra mí.

Sin más palabras ni hechos, Navarro se internó a buen paso por una
oscura y solitaria calle, y como algunos de los franceses allí
presentes, quisieran ir tras él, púsose Monsalud entre ambas esquinas de
la angosta vía y con determinación firmísima dijo a sus camaradas:

---El que quiera seguirle tiene que pasar sobre mi cuerpo.

\hypertarget{xii}{%
\chapter{XII}\label{xii}}

Cuando Jean-Jean y comparsa se empeñaban en llevar a Salvador a la
taberna, este iba en tal estado de sombrío estupor y excitación mental
que a las palabras de sus amigos, respondía tan sólo:

---¡Él guerrillero, yo francés!\ldots{} ¡Yo francés, él
guerrillero!\ldots{} ¡Él blanco, yo negro!\ldots{} ¡Él cielo, yo tierra!
¡Si ese hombre fuera Dios, yo quisiera ser el demonio!

A poco de entrar en la taberna, y antes que lograran hacerle tomar nada,
escapose fuera y se dirigió a su casa en lastimoso estado moral y
físico, con la razón delirante, el cuerpo flojo y desmayado como el de
un beodo, hablando sordamente consigo mismo a veces, y a ratos
profiriendo gritos que alarmaban al vecindario. Cuando entró en su casa,
hallábanse en ella, a pesar de lo avanzado de la noche, doña Perpetua y
el cura, acompañando ambos a doña Fermina. En el centro de la pieza
había una mesa puesta con no poco aparato de vasos y platos,
desplegándose allí gallardamente todo el lujo de la casa como para una
fiesta. Las viandas que sobre ella estaban, habían dejando de humear,
enfriadas ya por el largo plazo de espera, y las quijadas de la santa
como las del cura se abrían bostezando de apetito y sueño.

---Hijo mío, ¡cuánto nos has hecho esperar! Son las once dadas---dijo
doña Fermina, abrazándole.---Pero tú tienes algo, estás amarillo como un
muerto. ¿Qué dices ahí entre dientes?

---¡Guerrillero él! ¡francés yo!---murmuró Salvador dejándose caer en
una silla.

---Espera, te ayudaré a que te quites el uniforme---dijo la madre.---¿Se
han marchado ya los franceses?

---Salvador---dijo en tono agrio el cura, observando al sargento con
severidad.---Un joven de tus cualidades no debe estar en las tabernas
hasta hora tan avanzada.

Y como Monsalud no contestase a la advertencia, sino riendo a la manera
que ríen los locos, el presbítero añadió, levantándose de su asiento:

---¡Salvador, estás borracho! ¡Qué terribles hábitos se adquieren en el
ejército!

---¡Y entre franceses!---añadió la beata.---El Rey les da buen ejemplo
para que sean un modelo de sobriedad.

---Ya se te pasará---dijo doña Fermina con maternal
benevolencia.---Hijo, ¿quieres dormir?

---Sí, dormir; quiero dormir---repuso con gozo recostándose en un arca.

---Toma primero un bocado, muchacho.

---Sí, tengo hambre---exclamó el jurado abalanzándose a la comida y
engullendo descortésmente sin consideración a los demás convidados.

Mas al instante apartó el plato con repugnancia.

---No tengo gana---dijo entre dientes.

El cura se paseaba por la habitación agitado y colérico.

---Los malos hábitos adquiridos no se olvidan en un día---afirmó doña
Perpetua, echando al viento la voz por el registro más agridulce.---Esta
mañana lo dije y ahora lo repito. Fermina, haz cuenta que no tienes
hijo.

Doña Fermina rompió a llorar, y como interrogase cariñosamente al
desgraciado joven acerca de sus propósitos y de la enmienda que por la
mañana prometiera, este dijo:

---¡Guerrillero él, yo francés, francés toda la vida!

---Salvador---gritó el cura con enojo y fiereza.---Te creí traidor por
inexperiencia, mas no vicioso ni degradado\ldots{} Esta mañana me
causabas lástima, ahora me causas horror.

---El pobrecito no sabe lo que se dice, señor cura---añadió la
atribulada madre.---Esos pícaros lo han llevado a la cantina, y\ldots{}
por fuerza le han obligado a beber. Pero es un alma de Dios mi hijo.
Esta mañana nos prometió dejar para siempre esas aborrecidas banderas, y
lo hará, ¿pues no lo ha de hacer\ldots? ¿Te quedarás aquí esta noche?
Suelta el uniforme y duerme.

Oyéronse entonces lejanos toques de clarín. Callaron todos sobrecogidos
por el son guerrero que parecía venir del campamento francés y Monsalud
escuchaba con aparente júbilo. De pronto levantose y gesticulando como
un insensato, y con desesperados gritos, gritó de esta manera:

---¡Viva Napoleón! ¡Viva el amo del mundo! ¡Viva Francia! ¡Mueran los
guerrilleros!

---Esto no se puede tolerar---exclamó el cura bramando de ira y echando
mano al respaldo de la silla que más cerca tenía.---Traidor infame y
deslenguado blasfemo, sal de aquí al momento.

---¿Qué has dicho, hijo?---balbució entre angustiosos sollozos doña
Fermina temblando como un niño.---Tú, tú, ¿pues no eres\ldots?

---¡Afrancesado, francés hasta morir!---repuso el joven con enérgico
brío.---¡Francés hasta morir!

---Señor cura, señor cura---dijo la madre con tanto espanto como
dolor,---ríñalo Vd.

---Buen caso hago yo de los curas---repuso Salvador mirando con
desprecio al venerable Respaldiza.---Son los corruptores del linaje
humano, como dicen Jean-Jean y Plobertin, que presenciaron la revolución
francesa.

Doña Fermina ocultó el rostro entre las manos.

---Señor cura guerrillero---añadió el joven con insolente
sarcasmo,---cuidado no le cojamos a Vd. por esos trigos\ldots{} En mi
regimiento no hay piedad para los clérigos armados\ldots{} ¡Se les coge,
se les desnuda, se les ahorca!\ldots{}

Doña Perpetua se levantó de su asiento como una estatua que de súbito
cobra vida para aterrar a los hombres.

---¡Miren la embaucadora!---gritó Monsalud remedando de un modo grotesco
los ademanes de la santa mujer.---Vendré a rescatar a mi madre de las
garras del demonio, para llevármela a Francia.

La beata y el cura le señalaron la puerta sin proferir una palabra.

---¡Guerrillero él, yo francés!---repitió el joven, no con palabras,
sino con aullidos.---Madre, adiós, adiós\ldots{} Escribiré desde
Francia.

Tropezando, haciendo gestos amenazadores y articulando gritos y bravatas
poco inteligibles, pero horripilantes como la risa de los locos, salió
de la estancia y de la casa, mientras cura y beata auxiliaban a la
infeliz madre, que había perdido el conocimiento.

\hypertarget{xiii}{%
\chapter{XIII}\label{xiii}}

El buen orden de esta historia pide que ahora dejemos a Monsalud para
que vaya solo o acompañado a donde mejor le plazca y su triste destino
le lleve, y que volvamos los ojos y dirijamos nuestros pasos hacia
Carlos Navarro, quien por lo que hasta ahora de él vimos, parece ha de
ser personaje de historia y digno de ser conocido más de cerca.

Singular era este hombre, y más singular aún su padre D. Fernando
Navarro, vulgarmente conocido en la Puebla con el remoquete de D.
Fernando Garrote, que de sus mayores pasó a él sin que se pueda saber
por qué. Aseguraban los ancianos de la villa, que siendo todos los
Navarros, desde las generaciones más remotas, hombres muy fuertes, y a
más de fuertes, algo pegones y amigos de dominar a los débiles y de
machacar sobre los humildes, debieron de recibir por estas cualidades el
sobrenombre citado, que les caía a maravilla. Los últimos vástagos de
esta dinastía garrotil, son los que presentaremos ahora, eligiendo para
ello el momento en que, desocupada momentáneamente la Puebla por los
franceses, quiso D. Fernando poner en efecto su pensamiento de ir a las
partidas con Respaldiza, apretándole a ello la falta que él pensaba
hacía en el ejército su tardanza, según eran los agravios que pensaba
vengar, proezas que acometer, y cabezas que descalabrar.

D. Fernando vivía desde algún tiempo en una casa de campo hacia
Peñacerrada, donde había puesto fin a sus viajes y correrías, porque los
achaques y dolores en la trabajada osamenta eran ya obstáculo a su
fantasía siempre ardiente y a su corazón valeroso. Triste y solitario y
aburrido dejaba pasar sus días en la vasta vivienda, aun en lo más crudo
de la guerra, hasta que por capricho o voluntariedad impropia ya de sus
años, resolvió variar de conducta. Para hacer los preparativos de
marcha, trasladose el 18 de Junio a la Puebla, donde tenía la casa
solar, residencia habitual de su juventud y edad madura hasta los
últimos años. Allí vivía de ordinario su hijo, y un pariente pobre que
le administraba el mayorazgo, consistente en tierras de pan, algunas
viñas y mucho monte en el término de Treviño.

Allí le tenemos, allí está nuestro gran don Fernando en una sala baja,
sentado en ancho sillón de vaqueta con las piernas extendidas sobre un
banquillo. Ocúpase en limpiar la hoja de una luenga espada de taza, hoja
toledana y grandes gavilanes retorcidos. Frente a él, acurrucada en una
silla baja está la que ya conocemos, incomparable y seráfica doña
Perpetua, observando con atención prolija al insigne varón.

Era D. Fernando Navarro, o si se quiere don Fernando Garrote un hombre
de más de sesenta años, de elevada estatura y bien proporcionadas
carnes, ni gordo ni flaco, arrogante a pesar de su avanzada edad, de
frente despejada, ojos vivos, los brazos y las piernas vigorosas, aunque
ya nada listos a causa del mucho cansancio, ancha la espalda, curva y
airosa la nariz, blancas y pobladas las cejas, así como el cabello, la
piel rugosa y con largos bigotes retorcidos entrecanos, que eran
singular adorno de su fisonomía en aquellos tiempos en que todo el mundo
se rapaba el rostro. Tenía este hombre la apariencia de un veterano de
los antiguos tercios, héroe de las batallas de San Quintín y de las
Gravelinas, conquistador de medio mundo y saqueador del otro medio desde
Roma hasta Maestrich. Uníase a su belleza varonil y majestuosa cierta
expresioncilla insolente y de perdona-vidas, y parecía satisfecho de la
superioridad que Dios le había dado sobre el resto de los mortales.
Observando su vanaglorioso ademán y porte guerrero, viéndole tan
convencido de que la humanidad existía para que él probara sobre ella la
fuerza de sus puños, se comprendía bien el apodo de Garrote que
recibiera del vulgo. Lleváronlo sin ofenderse sus antepasados, que
también fueron tremebundos, y el D. Fernando respondía al mote y a veces
firmaba con él.

Durante su juventud Navarro había guerreado bastantes años, primero en
la campaña contra Portugal hacia 1762, después en el bloqueo de
Gibraltar en 1779, y aún se asegura que por dar desahogo a su grande
afición militar tuvo sus amagos y vislumbres de bandolerismo, en tiempo
de paz, lo cual es muy propio de españoles; pero esto debe acogerse con
prudente desconfianza, y la honra de tan insigne varón nos obliga a no
asegurar de un modo terminante lo del latrocinio, consignándolo tan sólo
como un simple rumor.

Lo que sí no deja duda, por constar en papel sellado dentro de los
mismos archivos de la audiencia de Pamplona, es que el gran Navarro
entretuvo sus ocios y dio alimento a su arrebatada actividad y ardiente
fantasía, introduciendo por los Alduides tejidos de hilo y algodón, en
lo que según su entender no se ofendía a Dios, siendo claro como el agua
que ni en el Decálogo, ni en el Nuevo Testamento, ni en ningún catecismo
se dice nada contra el contrabando. Hacía esto nuestro adalid más que
por propio lucro, por ayudar a los amigos, por favorecer a unos cuantos
pobrecitos que vivían de ello, por armar camorra con los empleados del
fisco y por dar palos. Esto era para Garrote fuente de delicias físicas
y morales sin término.

Al llegar aquí, y cuando después de enumerar casi todas las cualidades
de hombre tan eminente, me encuentro enfrente de la más importante, no
puedo menos de alzar los ojos al cielo, cruzar las manos y decir:
«¡Bendito sea Dios, que en una sola pieza puso tantas y tan admirables
prendas del alma y del cuerpo!» Ello era que D. Fernando Navarro, luego
que heredó el mayorazguillo, y además algunos pingües dineros que le
dejaron dos tíos suyos venidos de las Indias, retirose a la Puebla y
allí se hizo un D. Juan Tenorio. Su arrogante figura, su garbo para
vestir y su mucho gracejo para hablar, su gran experiencia del mundo y
diestra habilidad para engañar, proporcionáronle adelantamientos
fabulosos en la carrera.

Siendo al mismo tiempo muy liberal y dadivoso, así de dinero como de
palos, encontraba abiertos casi todos los caminos, y bien pronto todo el
condado de Treviño, toda Álava y aun parte de la Rioja, llenáronse de
víctimas en distintas edades y estados. Algunos disgustos experimentó en
diversas ocasiones; mas como era Garrote la persona más poderosa en la
villa, y casi casi en la comarca, como tenía la llave dorada, y aun se
habló de que iba a recibir la merced de un título de Castilla, todo se
quedó en palabras y en dos o tres porrazos. Un fraile francisco quiso
con amonestaciones convertirle, librando de azote tan fiero a los
habitantes de la baja Álava y Rioja alavesa, mas por una singularidad
digna de ser mencionada en la historia, los villanos todos,
especialmente los más humildes se pusieron de parte de D. Fernando,
hasta que el bendito fraile se cansó, y resolvió que lo mejor era rezar
por las agraviadas.

Lo que no puede pasarse en silencio, es que hacia el fin de su carrera
D. Fernando se casó, animándole a ello su propio interés y el de una
familia de Navarra que con la suya estaba genealógicamente entroncada.
Antes, mucho antes del matrimonio, había nacido un varón, que fue
reconocido con solemnidad. Sacó Carlitos, con el cariz y la figura de su
padre, muchas de las prendas de su alma, y singularmente el valor y la
generosidad, y creció el niño en la holganza, dedicándose a ejercicios
de fuerza, con descuido de la inteligencia, aunque la tenía
privilegiada. No mostró como el progenitor, afición al galanteo frívolo,
y durante algunos años huía de las faldas como del demonio: tanto que
creyeron iba derechito por el camino de la iglesia, mas de pronto
resultó muy apasionado y tierno, y verificose radical transformación en
sus hábitos, y más que todo en su pensamiento. En el transcurso de esta
fiel historia irán saliendo muchas cosas que ahora no conviene
anticipar, y que completarán el conocimiento de este benemérito joven,
primero mojigato, guerrillero después, y adornado siempre de estupendas
cualidades.

Ahora lo que importa referir, es que en 1812 tomó el gusto Carlitos a
las partidas, enamorándose de tal modo de aquella errante, gloriosa y
popular vida, que a vuelta de pocos meses era uno de los más bravos e
inteligentes soldados del bravísimo Longa, siendo tantas sus hazañas que
en la Puebla de Arganzón gozaba de más fama que en Macedonia el Grande
Alejandro. No está de más decir, que entre las causas que determinaron a
D. Fernando a meter su cucharada en el negocio de la guerra, no fue la
menor cierta comenzoncilla, o por ponerlo más claro, cierta envidia del
gran renombre de su hijo, y tenía la certidumbre de que con sólo echarse
al campo eclipsaría con un solo arranque las proezas de todos los
fusileros de Longa, Mina y Pastor.

Conocidas así las personas, refiramos ahora lo que hablaron doña
Perpetua y el Sr.~Garrote, mientras este, esperando a su hijo, al cura
Respaldiza y demás personas que debían acompañarle, se ocupaba en
limpiar el moho a varios trebejos, resto de su alborotada mocedad.

---Reflexione Vd., Sr.~Garrote---dijo la vieja, apoyando las manos en el
palo y la barba en las manos,---sobre lo que tantas veces le he dicho y
ahora le repito. Un hombre lleno de pecados, que ha sido el escándalo de
un siglo y el Satanás de esta honrada villa, debe ocuparse en arreglar
sus largas cuentas con Dios para no presentarse a él desprevenido, con
el libro de las deudas de su conciencia tan embrollado y lleno de
borrones.

---Cuando vuelva de la guerra, viejecita---repuso D. Fernando
cariñosamente y con cierto respeto,---te prometo reconciliarme y poner
el mayor arreglo en mi libro.

---¡De la guerra!---exclamó la vieja moviendo la cabeza---¡y quién sabe
si esos pobres huesos molidos volverán como salen! ¡Semejante estafermo
no puede mantenerse sobre el caballo, y habla de matar franceses y de
ganar batallas! ¡Alabado sea el Señor! ¿No vale más que el Sr.~Garrote
se esté quietecito en su casa? Yo le vendré a hacer compañía, y nos
regocijaremos hablando de los benditos tiempos pasados y de la ruindad
de los presentes, así como de la supina perversidad de los que han de
venir, trayendo seguramente el fin y ruina total del mundo.

---Viejecita---repuso D. Fernando,---en sesenta años que he vivido no he
sentido gusto semejante al que ahora llena mi alma por la empresa que
voy a acometer\ldots{} Ya, ya verán una mano pesada para el
sable\ldots{} Seguramente los franceses tienen ya noticia de que me
preparo\ldots{}

---Si se preparara Vd. para una buena, larga y devota confesión que
fuera una limpia general de su alma, mejor sería\ldots---dijo la santa
mujer.

---Hay muchos medios de limpiar el alma y dejarla como un
espejo---afirmó triunfante Garrote, esgrimiendo la espada y dando dos o
tres tajos en el aire,---muchas maneras, y de esto hablan los Santos
Padres, según creo, madrita; y si no hablan es porque se les quedó en el
tintero.

---No conozco más medio que el arrepentimiento.

---Verdad es que yo he pecado bastante---dijo el héroe;---pero ha sido
sin mala intención. Reconozco que he ofendido a Dios; pero si después de
la ofensa, le sirvo, ¿el servicio no quita la ofensa?

La mujer del siglo miró con estupor al anciano, sin contestarle.

---Yo pequé---continuó este,---pero he aquí que la gran contienda entre
Dios y el demonio es llevada a los campos de batalla; he aquí que yo,
hombre un poco ligero de cascos, pero cristiano viejo y con una fe como
un templo, saco la espada y digo: «Señor, si mucho te ofendí, ahora te
consagro mi vida y voy a morir en defensa de tu Iglesia o a matar a
todos tus enemigos». Este acto, señora doña Perpetua, esta abnegación
mía por la causa de Dios, ¿no bastan a limpiarme, cual si echaran mi
alma en lejía?

---Según y cómo---respondió la anciana, confusa ante un problema nuevo
para ella, cuya solución no podía dar en definitiva.---Ejemplos hay de
guerreros insignes que han ido a ocupar lugar preferente en el Cielo,
sólo por una buena batallita ganada contra herejes; pero no se dice que
tuvieran muchos pecados, ni que estuviesen impenitentes.

---¿Y qué más penitencia que la muerte en defensa de Cristo?---exclamó
el guerrero sintiéndose con más fuerza que su antagonista.---¡Morir,
derramar uno su sangre por una causa, por una idea, por la religión, por
Dios\ldots!

---¡Oh! sí, es verdad, sí, sí---dijo la vieja abrumada por esta lógica.

---¿Nuestro Señor Jesucristo no nos dio el ejemplo? ¿No redimió a todo
el género humano, y muriendo nos limpió la gran mancha original, sin
dejar rastro de ella?

Al decir esto, el Sr.~Garrote frotaba con verdadero frenesí la hoja de
acero, como si la herrumbre que tenía fuera la de su propia alma, y
aquel orín el inveterado orín de su propia conciencia.

---Es verdad---gruñó la vieja.---Vaya el señor D. Fernando a la guerra,
si bien no estaría de más una confesión general y algún acto de
reparación para tranquilizar el alma de quien yo me sé, de un ángel de
Dios, Sr.~D. Fernando\ldots{}

La beata fijó en Garrote sus penetrantes ojos negros, y Navarro frunció
ligeramente el ceño, demostrando que aquel tratado de los ángeles de
Dios no era muy de su agrado. Pero la santa mujer, hecha de muy antiguo
a reprender sin rebozo las faltas ajenas y a sentenciar en materia de
pecados con tanto aplomo como el Papa desde la silla del Pescador, no
hizo caso del avinagrado gesto de D. Fernando, y dijo:

---Sr.~Lucifer, de todas las excelentes muchachas que Vd. perdió para
siempre, una sola existe en la Puebla de Arganzón; mas tan quebrantada
por los disgustos y la vergüenza de su desgracia, que es difícil conocer
en su abatido y ya viejo rostro a la hermosa hija de don Pablo el
Riojano.

---Bueno, bueno---dijo Garrote frotando con más fuerza:---¿y qué tengo
yo que ver con esa mujer?

---¡Conciencia empedernida! ¡Hombre sin entrañas! ¿No la perdió Vd. para
siempre? En Pipaón hace veintidós años todo el mundo sabía que D.
Fernando Garrote tenía amores con la niña del Riojano y se corrió la voz
de que se iban a casar. Desde entonces ha pasado mucho tiempo. Vino doña
Fermina a la Puebla hace dos años traída por su mezquina herencia, y el
enfadoso pleito que la dejara sin camisa que ponerse. Pocos la tratan
aquí, y en cuanto a sus tristes antecedentes, sólo yo, por confidencia
que me ha hecho correspondiendo a mis cristianos consejos, sé que esta
venerable y modesta mujer es la doncella engañada hace más de veinte
años en Pipaón, y que Salvadorcillo Monsalud es de la propia carne, de
la misma sangre y de los mismísimos huesos de este tenebrario que tengo
delante.

---¡Cuánto sabe la madre!---dijo D. Fernando, frotando el arma hasta
desollarse los dedos.---Supe que Ferminilla había venido a la Puebla
hace dos años trayendo consigo a un muchacho revoltoso; pero como casi
todo el tiempo vivo en Peñacerrada, a ninguno de ellos he visto\ldots{}
y a la verdad, no son muchas las ganas\ldots{}

---Pues yo la veo todos los días. Yo la acompaño y consuelo de la amarga
tristeza que aún hoy sus desdichas y su atroz pecado le causan. Cuando
llegó aquí, picome la curiosidad. Viéndola tan piadosa, tan santa y
ejemplar, pues es mujer que no sale de su casa más que para ir a la
iglesia, solicité su amistad; conocí que era un alma abatida y que
necesitaba de mí. ¿Qué habría sido de ella sin mis consejos? Se los di,
pues; mi conversación le agradó en extremo, y abriome su corazón
confiándome todo y especialmente la tristeza de su desgracia, cuyo autor
fue este señoritico precioso.

---Bien: ¿y qué?---dijo Navarro esforzándose en aparecer risueño, y
dejando a un lado la espada que estaba más limpia que alma de
bienaventurado.---Yo, la verdad, lo hice sin mala intención.

---¡Sin mala intención!---exclamó la beata con enojado semblante.---Sin
mala intención dicen que se rebeló Luzbel contra Dios. Esa buena mujer
es la criatura más desgraciada que existe en el mundo, y aunque
seguramente Dios la ha perdonado por su grande arrepentimiento y
continuo llorar, ella jamás se consuela, y ahora con la reciente
desgracia del hijo que idolatraba, parece que va a entregar su alma al
Señor.

---Pues qué, ¿ha muerto su hijo?---preguntó Garrote con vivo interés.

---Se ha pasado a los franceses, lo cual es peor que morir. Se ha pasado
a los franceses, que es como morir el alma y seguir viviendo el cuerpo
para afrenta de la familia y de la nación\ldots{} Anoche mismo\ldots{}

---¡Y dices que es hijo mío!---exclamó don Fernando con rabia, dando
fuerte patada en el suelo.---No, madrita: ese muchacho no tiene mi
sangre\ldots{} Es mentira, ¡viven los cielos!

Iba a seguir protestando, cuando le interrumpió de súbito la presencia
de su hijo Carlos, que acababa de entrar.

\hypertarget{xiv}{%
\chapter{XIV}\label{xiv}}

Carlitos era bastante parecido a su padre, salvo algunas diferencias; se
le asemejaba en la tez morena, en los cabellos asimismo negros, en la
arrogancia del cuerpo y talle y en cierta expresión de nobleza que en
toda su persona gallardamente se mostraba. Diferenciábase en la
estructura de las cejas que en el mozo eran juntas, y en la seriedad
invariable y algo torva que tenía en los grandes ojos. Con respeto
adelantose el joven hacia su padre, cuya mano besó, repitiendo la misma
señal de veneración y cortesía en las arrugadas extremidades de la
vieja. D. Fernando contemplaba a su hijo con el arrobamiento de un
artista satisfecho y enfatuado ante la belleza de su obra maestra.

---¿Nos vamos ya?---le preguntó.

---Dentro de una hora---repuso el joven.---Difícil es que nos unamos a
la partida de Longa que está en Munguía con los ingleses; pero nos
uniremos a los que están hacia Miranda con el general Morillo. Para no
tropezar con los franceses daremos la vuelta por Uralde y Burgueta,
tomando el camino real en Armiñón. No hay nada que temer por ese lado.

D. Fernando se levantó para desperezarse, lo cual hizo como un león
viejo, no sin que crujieran sus choquezuelas y sus articulaciones todas.
Después dio algunos pasos por la habitación como para probar la
elasticidad de sus miembros.

---Esta máquina sirve todavía---dijo.

Y luego dio fuertes voces llamando a sus criados.

---¡El caballo!\ldots{} ¡ensillar el caballo!

Doña Perpetua, firme siempre en la perpetuidad de su desaprobación,
movía la cabeza en señal de duda respecto a la eficacia de aquella
máquina para hacer algo de provecho, y si no con la boca, con los ojos
reprendió a don Fernando por su atrevida aventura.

Al punto comenzó Garrote su atavío marcial, sepultando sus pies en
antiguas botas de cuero fino. Forrose después en un chaleco grueso y se
fajó con una interminable banda de seda que le dio muchas vueltas en
torno a la cintura, y sobre esto se puso un uniforme blanco de los
antiguos regimientos distinguidos, el cual aunque viejo y fuera de moda,
estaba servible. La cabeza la adornó con un deforme sombrero procedente
de las campañas del décimo octavo siglo y que recordaba al general
O'Reilly. A pesar de la notoria ancianidad de dichas prendas, tal era la
histórica figura del insigne Navarro, que con ellas no resultaba
ridículo.

Al vestirse parecía que se remozaba; la alegría brillaba en sus ojos;
decía mil bufonadas graciosas, y con fatuidad chispeante se presentaba a
sí mismo como modelo de apuestos militares, deprimiendo a la afeminada
juventud del día. En mitad de esta escena entró el cura hecho un arsenal
ambulante, según venía de armado y municionado, y celebró con palmadas y
vítores los preparativos de su amigo, mostrando los suyos y volviéndose
de todos lados para que le vieran.

---¡A matar franceses!---gritó el presbítero.---¡A matar franceses y
afrancesados, para gloria de la nación y triunfo de la fe!

---Señores---dijo Garrote con hueca voz y un poco del tonillo pedantesco
de los oradores modernos,---toda mi vida la he consagrado al servicio
del Rey, de la patria, de la religión\ldots{}

La beata frunciendo el ceño, miró a don Fernando con expresión de burla.

---No, de la religión no---añadió Navarro con modestia,---quiero decir
que no he prestado a la religión servicios directos; pero siempre he
sido piadoso, buen cristiano y temeroso de Dios\ldots{} Alguno que otro
pecadillo que anda suelto por ahí no es para darse de cabezadas, ¿no es
verdad, señor cura?

---Sí hombre, sí---exclamó el padre de almas con risa
campechana.---Contra una juventud algo ligera viene una vejez heroica en
servicio de Dios.

¡En servicio de Dios! A eso iba---prosiguió Garrote acompañando sus
palabras con una enérgica acción del dedo índice.---Quería decir que
siempre fui ferviente cristiano y una vez reventé a palos a dos
contrabandistas porque hablaron mal de la santidad de Pío VI. Señores,
en mis campañas gloriosas, o por mejor decir, en toda mi vida, he tenido
por norte la honra del Rey, la honra de la nación y sobre todos los
nortes y sures, el norte de la religión que es mi guía, mi faro, mi luz
del cielo.

---Si este D. Fernando no hace ahora un par de heroicidades estupendas
que dejen atrás la antigüedad de Aníbales y Césares---exclamó con
entusiasmo el cura,---me dejo quitar el hábito que visto y las licencias
del sagrado orden que practico.

---Pues bien, señores---siguió el héroe,---¿a qué han venido aquí los
franceses? A quitarnos nuestro Rey, a quitarnos nuestra patria y a
quitarnos, ¡oh crimen nefando! nuestra santa religión. Ved a España
entera cómo se levanta en contra de esa canalla y en pro de tan caros
objetos. Ved a España, vedme a mí, que un poco tarde, pero a tiempo
todavía, me decido a echar una cana al aire.

---¡Una cana al aire!---repitió doña Perpetua rascándose.---Si D.
Fernando no las deja todas en el campo de batalla, será milagro del
Cielo.

---Hay un mal grave, señores, un mal terrible, al cual es preciso
combatir---continuó Garrote sin hacer caso de la vieja.---¿Qué mal es
este? Que los franceses han traído acá la idea de cambiar nuestras
costumbres, de echar por tierra todas las prácticas del gobierno de
estos reinos, de mudar nuestra vida, haciéndonos a todos franceses,
descreídos, afeminados, badulaques, tontos de capirote y eunucos. ¿Y qué
ha sucedido? que mientras la mayor parte de los españoles se echaban al
campo para extirpar toda la maleza galaica y sahumar con el vapor de la
guerra el país infestado de franceses, unos pocos de los nuestros han
admitido aquella mudanza. ¡Abominables tiempos, señores! Ved cómo hay en
Madrid una casta de miserables sabandijos a quien llaman afrancesados,
que son los que visten a la francesa, comen a la francesa y piensan a la
francesa. Para ellos no hay España, y todos los que guerreamos por la
patria somos necios y locos. Pero todavía existe una canalla peor que la
canalla afrancesada, pues éstos al menos son malvados descubiertos y los
otros hipócritas infames. ¿Sabéis a quién me refiero? pues os lo diré.
Hablo de los que en Cádiz han hecho lo que llaman la Constitución y los
que no se ocupan sino de nuevas leyes y nuevos principios y otras
gansadas de que yo me reiría, si no viera que este torrente
constitucional trae mucha agua turbia y hace espantoso ruido, por
arrastrar en su seno piedras y cadáveres y fango. ¿Queréis pruebas? Pues
oídlas. Estos hombres se fingen muy patriotas y aparentan odiar al
francés, pero en realidad le aman. ¡Ah! Pasad la vista por sus
abominables gacetas. ¿Las habéis leído? Decís que no. Pues yo las he
leído y sé que respiran odio a los patriotas, al Rey y a la sacrosanta
religión. Son los discípulos de Voltaire, que van por el mundo
predicando la nueva de Satanás.

El cura al oír esto sintió que las lagrimas se agolpaban a sus ojos.
Eran lágrimas de admiración. Estaba pálido, mas no de envidia, aunque
reconocía que él jamás había dicho en sus sermones cosas tan bellas.

---Pues bien, señores---añadió Navarro,---hoy voy a combatir contra los
franceses y mañana contra los afrancesados que son peores, y después
contra los llamados liberales que son pésimos; y si yo no pudiere o si
Dios se sirve llamarme a sí sobre el campo de batalla, aquí está mi
hijo, a quien entregaré mi espada y que ya tiene mi espíritu.

---Dios que vela por España---dijo el cura con acento solemne,---nos
conservará a nuestro buen amigo y volveremos todos cubiertos de
laureles.

---Los laureles---dijo la beata,---no caen mal sobre una frente serena
que puede alzarse ante el tribunal de Dios sin los rubores del pecado.
Sr.~D. Fernando, ponga sus cinco sentidos en lo que le he dicho, y no
entregue su cuerpo al plomo enemigo sin descargar su alma del peso de
tantas y tan negras culpas. El cuerpo que sirve de vaso a un alma limpia
es respetado por la muerte; no así el que es saco de inmundicias. No hay
contra el plomo y las bayonetas mejor coraza que una buena y general
confesión.

---Viejecita---repuso D. Fernando sonriendo,---como el cura va conmigo a
la guerra, echaremos un párrafo por esos caminos y entre batalla y
batalla me iré descargando de todos mis pecados y él absolviéndome, todo
esto al compás de nuestras caballerías.

---Cabal, cabal---exclamó el presbítero.---Por mucha que sea la faena,
no falta un ratito para meter la mano en la conciencia y sacar algunos
puñados de maleza.

---Y para los soldados, voto al chápiro---dijo D. Fernando golpeando el
suelo con la contera de la espada,---ha de haber un poquito de manga
ancha. Ya se ve: siempre en campaña al sol y al frío, comiendo poco y
bebiendo menos, sin otro regalo que mil trabajos, y teniendo por cama el
suelo, por descanso la fatiga, por almuerzo la pólvora y por cena la
metralla\ldots{} ¡Oh! los que así vivimos no podemos ser mirados como
los demás, ¿no es verdad, señor cura?

---Verdad, verdad\ldots{} ¡Con que en marcha!\ldots{} ¿No se te olvida
nada Respaldiza? ---dijo el cura preguntándose a sí mismo y tentándose
el cuerpo.---No, nada se te olvida, curita\ldots{} la pólvora, las
balas, el frasquito de aguardiente, las lonjas de jamón\ldots{} el
chocolate crudo\ldots{} el tabaco\ldots{}

A todas estas iba llegando gente, amigos del insigne Garrote.

Llegó la hora de la partida y los expedicionarios oprimían los lomos de
sus respectivas caballerías. La salida de la casa fue una verdadera
ovación. D. Fernando, seguido de su hijo, del cura y de los demás
guerrilleros, rompió por entre la multitud que le vitoreaba aclamándole
padre de la patria y héroe de la Puebla. En aquel instante nadie se
acordaba de las fechorías de D. Fernando Garrote, que había sido siempre
popular, muy popular, lo mismo por sus generosidades que por sus
atrevimientos. En España los audaces de buena cepa, aunque sean bandidos
o Tenorios, son siempre queridos y admirados del pueblo, que lo perdona
todo, a excepción de la cobardía y la avaricia.

Luego que se encontró fuera de la villa y en pleno campo la pequeña
partida, compuesta de una docena de hombres, Carlos, indicando la
dirección de Treviño, que debían tomar por las montañas, se puso a
vanguardia con otro amigo, para explorar el camino y ver si se
distinguían fuerzas francesas. En tanto D. Fernando y el cura,
quedándose solos atrás emparejaron sus cabalgaduras, que perezosamente
iban al paso, y entablaron el curiosísimo diálogo, que se verá a
continuación.

\hypertarget{xv}{%
\chapter{XV}\label{xv}}

---Señor cura---dijo Garrote,---ahora que nos encontramos solos, quiero
que conversemos un poco sobre un asunto que me está escociendo por
dentro.

---Ya le entiendo a Vd. amigo mío, Vd. es de parecer que en vez de
unirnos a la partida de Longa, marchemos solos al encuentro de los
franceses.

---No es nada de eso, Sr.~D. Aparicio, lo que me preocupa.

---Ese fusil que lleva Vd.---añadió el cura,---es un arma de príncipes;
en cambio esa espada no sirve sino para degollar palominos. Por el
contrario, mi sable vale un imperio, y esta escopeta no lo es más que en
el nombre. Hagamos, pues, un cambalache: darele a usted el sable, pues
la principal habilidad de Vd. consiste en el tajo, mientras que siendo
mi fuerte la puntería, cogeré por lo tanto su fusil.

---No es eso tampoco lo que tenía que hablar.

---Usted tiene muy cansada la vista y no puede hacer la puntería.

---Que no es eso---repitió Garrote con enfado.

---¿Pues qué, hombre de Dios?

---Un caso de conciencia.

---¿Esas tenemos?---dijo el cura riendo.---Esta mañana estuve una hora
en el confesonario sin que nadie se me acercara, y ahora que monto a
caballo\ldots{}

---No pierde el sacerdote el Sacramento por ir a horcajadas.

---Jamás he visto que el ilustre Garrote se confesara; ¿y ahora que va a
la guerra le entran esos escrúpulos? ¿Hay algún pecado nuevo? Pero no sé
por qué recuerdo ahora\ldots{} Esa maldita Perpetua\ldots{}

No, los antiguos. Por lo mismo que voy a la guerra, siento un vivo deseo
de reconciliarme con Dios\ldots{} Aunque hombres como yo no mueren a dos
tirones, quién sabe si por artes del enemigo me cogerá una bala\ldots{}

---Y adiós alma\ldots{} Nada, nada---dijo el cura,---aun los hombres más
bravos deben venir a estas fiestas con el alma preparada\ldots{} Aquí
donde Vd. me ve, voy como un angelito de Dios\ldots{} Me podrían
enterrar con corona de rosas como a los niños.

---Vamos a ver. Si los pecados se perdonan con el arrepentimiento y la
penitencia, los míos ya los puedo dar por idos. Estoy arrepentido de los
males que he causado, y ahora que soy viejo y nada puedo, he caído en la
cuenta de que hice mal, muy mal. En cuanto a la penitencia, ¿no es
suficiente esta que yo mismo me impongo de dejar la tranquilidad y
bienestar que disfrutaba en mi casa de Peñacerrada, para echarme al
campo en busca de las privaciones, de las hambres, de las heridas, de
los fríos, de los calores y quizás quizás de la muerte? Y todo esto no
por una causa cualquiera, sino por la causa de Dios, de la religión y su
santa Iglesia primero, y del Rey y de España después.

---Mi parecer es---dijo el cura sonriendo y tentando de nuevo sus
bolsillos y la alforja para ver si se le olvidaba algo,---que con lo
hecho por Vd., con su arrepentimiento primero y el sacrificio de su
bienestar después, hay para irse derecho al cielo.

D. Fernando respiró con desahogo, y muy vivamente añadió:

---Si ofendí a Dios con mis calaveradas, ahora le sirvo con mi heroísmo:
¿no es verdad? Váyase lo uno por lo otro. Jamás cometí acción ninguna
indigna de un caballero\ldots{} pues\ldots{} ya me entiende Vd\ldots{}
porque hay pecados de pecados.

---Es evidente\ldots{} Pero si el arrepentimiento y la penitencia
limpian el alma, no está de más un poco de palique con el cura\ldots{}

---Ya, la confesión.

---La humillación del alma ante Dios, y aquello de reconocer verbalmente
sus faltas y avergonzarse de ellas delante del sacerdote\ldots{}

---Por hablar no quedará---dijo Garrote,---pero es lástima que esto no
lo hiciéramos despacito en el pueblo en vez de hacerlo a caballo por
estos andurriales.

El cura rompió a reír.

---¡Qué singulares cosas tiene D. Fernando Garrote!---exclamó avivando
el paso de la cabalgadura.---Esta noche cuando lleguemos a cualquier
mesón\ldots{} ¿Pero está Vd. triste, señor Navarro; a qué viene tanto
mirar al suelo y ese gesto de ajusticiado?

---Amigo D. Aparicio---repuso el guerrero,---no puedo apartar de mi
pensamiento la idea de que me coja una bala.

---Los bravos no mueren\ldots{}

---Si el caso llega---añadió el guerrillero muy preocupado y
entristecido---no moriré sin decir antes a voz en grito ante Dios y los
hombres que siempre fuí católico, apostólico, romano y defensor de la
santa Iglesia, cuyos dogmas creo desde el primero hasta el último.

---Bien, eso es lo principal\ldots{} Ahora señor Garrote, déme Vd. su
fusil---dijo el cura con vivísimo interés, mirando a un punto lejano
hacia la izquierda.---¿No le parece que se distingue por allí el morrión
de un francés?

---No puede ser, hombre.

---Será algún rezagado. Anoche pasó por aquí el ejército enemigo.

---Pues como iba diciendo---prosiguió Garrote ensimismado y algo
sombrío,---toda mi vida he sido católico, apostólico, romano\ldots{}
Jamás he robado a nadie el valor de un real. No he levantado falsos
testimonios, y si dije alguna mentirilla leve, fue sin hacer daño a
nadie, o por galanteo, pues\ldots{} cosas de mujeres. Si he jurado en
falso ha sido en asunto de amores. Honré a mis padres mientras vivieron;
no he matado a nadie, ni\ldots{}

---Ni deseado la mujer ajena---dijo el cura interrumpiéndole con risas.

---¡Alto, alto! que ahí está el busilis---gritó D. Fernando.

---¿Qué, qué es lo que está?---dijo Respaldiza mirando con zozobra a un
lado y otro.

---Nada, hombre, no hay que asustarse, lo principal de mis pecados,
digo\ldots{}

---Creí que había divisado Vd. algún destacamento enemigo. Pero ¿por
dónde vamos, amigo Garrote?

---Vamos bien; adelante---dijo Navarro, tan sólo preocupado de su
conciencia.

Iban por un terreno bastante solitario y compuesto de cerros que se
sucedían unos a otros, elevándose cada vez más. De trecho en trecho,
hallábanse pequeñas llanadas.

---Ya se sabe qué clase de pecados son los míos---continuó Garrote sin
poder apartar el pensamiento de aquella idea.---No son en verdad de los
que más afean al hombre; y en el mundo vemos que mientras se niega el
agua y el fuego al asesino, al galanteador no sólo no se le niega nada,
sino que todo el mundo le admira, le señala, y con su amistad se honran
tontos y discretos, buenos y malos.

---Así es en efecto---dijo Respaldiza,---lo cual no quita que el
galantear sea pecado, porque es el desenfreno del más feo y torpe vicio,
y con él se injuria a la familia, al mundo y a Dios.

---Por más que me diga el señor cura, no puedo creer que el galanteo sea
vicio tan inmundo como el robar, el calumniar y blasfemar. Al hacer
cocos a una doncella o mujer casada, parece como que se tributa cierto
holocausto al Señor por las maravillas que puso en el alma y en el
cuerpo. El espíritu pone de manifiesto lo que encierra de más noble, y
la materia\ldots{}

---Tate, tate, Sr.~D. Fernando---dijo entre risas Respaldiza.---Al
querer confesarse está usted haciendo la apología de sus pecados, y
revistiéndolos con las mentirosas formas de una fantasía voluptuosa. Es
una singularísima manera e arrepentirse\ldots{} Vaya un polvito---añadió
sacando la tabaquera.

---No, no, ya estoy arrepentido, Sr.~D. Aparicio. Ya estoy arrepentido
de todo---afirmó Garrote con decisión.---No sirvo ya para maldita la
cosa. ¡Quién me había de decir en aquellos tiempos, cuando todo el mundo
me parecía pequeño para mis aventuras, que se me había de acabar la
vigorosa energía\ldots!

---Punto, punto final, amigo mío---dijo el cura mirando a la izquierda.

---Iba a decir que ahora aborrezco todo aquello, y que lo
deploro\ldots{} Pero me pasa una cosa singular, amigo, y es que me
arrepiento, pero no estoy tranquilo. El corazón me baila en el pecho, y
siento en mí no sé qué comezón y zozobra.

El bravo cura se irguió de repente alzándose sobre los estribos, y gritó
con ansiedad:

---Sr.~D. Fernando, el fusil, venga el fusil, ¡por todos los santos!

---¿Qué hay? ¿Viene algún destacamento francés?---preguntó el guerrero
mirando al mismo punto hacia el cual se dirigían los atónitos ojos del
presbítero.

---¡Un morrión! Por allí va el morrión de un francés.

---¿El morrión solo?

---Bajo el morrión ha de ir una cabeza, y bajo la cabeza un cuerpo, sólo
que va por aquel camino hondo y no se ve más que el cimborrio\ldots{}
Ese fusil, Sr.~D. Fernando ¡por amor de Dios!

---Ya, ya lo veo---dijo Garrote, poniéndose la palma de la mano sobre
los ojos en forma de visera.---Pero es un hombre solo, un pobre soldado
rezagado, quizás un prisionero fugitivo. ¿Qué hacemos?

---¡Bonita pregunta! Matarle. Un enemigo menos tendrá España.

---Pero si no me engaño---dijo D. Fernando mirando a todos lados con
cierta inquietud,---nos hemos perdido. ¿En dónde están mi hijo y los
demás amigos?

---Delante van. Ese fusil, Sr.~D. Fernando: veremos si el cura de la
Puebla desmiente la fama de ser el mejor tirador de todo el condado, y
aun de toda Álava.

---Amigo, ¿por dónde vamos?---repitió Navarro deteniendo el
caballo.---Con esta conversación de mis pecados y de la bondad de Dios,
que todos me los perdona, nos hemos distraído y sin saber cómo, nos
hallamos separados de los demás de la partida.

---¿Cómo es eso? ¡Gran geógrafo tenemos aquí!---exclamó el cura.---¿Pues
no es este el camino de Uralde?

---No, con mil demonios; aquellas casas que a lo lejos se parecen son
las primeras de Añastro. Carlos y la compañía se han ido camino derecho
a Uralde, y nosotros ¡ahora caigo en ello, con cien mil pares de
Satanases! nos equivocamos en la encrucijada donde está la venta de
Martín.

---Adelante---dijo el cura con resolución.---Buscaremos un atajo por
aquí a la izquierda\ldots{} ¿Hay miedo, Sr.~D. Fernando? Lo mismo da ir
por Uralde que por Añastro. Usted tiene la culpa, pues charla que
charla\ldots{}

---No hagamos calaveradas---dijo Garrote bastante intranquilo.---Casi
estamos en país enemigo. A lo mejor saldrá de detrás de una mata un
puñado de franceses.

---Aquel que allí está no se me escapa---dijo el cura, observando
siempre el morrión que por el camino hondo se movía.---¿Nos vamos a por
él?

---¡Dos contra uno!---exclamó con desdén D. Fernando.---Esta heroicidad
no es de las mías.

---¿Pero si ese uno se convierte en seis dentro de un rato? ¿Quién sabe
lo que habrá detrás de aquella colina?

---Pues vamos a él---dijo D. Fernando dirigiendo su caballo por un
sembrado y hacia el punto donde el formidable morrión aparecía.---Esta
guerra en detalle es la que a mí me enamora, y la verdad es que hecha
con inteligencia, no hay ejército invasor que a ella resista.

---¡El fusil, ese fusilito, por amor de Dios y de María Santísima!

---¡Ahí va!\ldots{} ¡que Dios esté en la chispa, en la pólvora y en la
bala!

Galoparon buen trecho por el sembrado, y de pronto, como liebre que
levantan perros, viose salir del camino hondo un soldado francés, el
cual azorado y temeroso al ver sobre sí dos tan disformes jinetes echó a
correr con ligerísimos pies, mirando hacia atrás a cada instante para
ver si era perseguido.

---Alto ahí, amiguito---gritó el cura,---que no te salvarás aunque
tengas mejores piernas que Mercurio, el de los alados talones\ldots{}
¡Alto!

---Ríndete y nada te haremos por ser dos contra uno---gritó D. Fernando
llevándose la mano al sombrero, que con el fuerte viento se le
tambaleaba sobre el cráneo.---Date, tunantuelo, que somos generosos y
caballeros.

---¡Borracho, ladrón! Ríndete o te tiendo\ldots{}

Aunque muy velozmente corría el francés, al poco rato pusiéronse los
caballos a medio tiro; disparó D. Aparicio su fusil, hiriendo al
fugitivo con tan fatal acierto en mitad de la espalda, que después de
dar algunos pasos vacilantes cayó al suelo.

---¡Qué ojo! ¡Sr.~Garrote! Por Santa Lucía bendita. ¡Qué
puntería!---exclamó con júbilo Respaldiza.---Yo mismo me admiro, yo
mismo me alabo, yo mismo me hago mi apoteosis, porque soy en esto del
tirar una de las más grandes maravillas de la Creación.

---La verdad es que como cacería esto ha sido admirable---repuso
Garrote,---pero como acción de guerra no se puede poner al lado de las
de Wellington. Ese pobre muchacho lo pasa mal.

Llegaron al sitio donde el francés se revolvía en su sangre profiriendo
injurias y blasfemias contra sus perseguidores.

---Arriba muchacho, eso no es nada---dijo Navarro, cuya generosidad,
como hemos dicho, se mostraba en todas ocasiones .---Dinos dónde está el
destacamento a que perteneces y te perdonamos la vida.

---El destacamento---repitió el cura.---Sí; para huir de él.

---O para atacarle si es poca gente. Usted con su puntería y yo con mis
puños\ldots{}

A esta bravata siguió un rato de silencio, porque el pobre francés
herido, se había desmayado. Mirábanse Garrote y D. Aparicio sin saber
qué partido tomar, cuando sintiose a lo lejos ruido de caballos, y como
alzaran a un mismo tiempo la vista cura y seglar, vieron que hacia ellos
se dirigía por el camino hondo hasta una docena de franchutes a caballo.
Púsose más pálido que la cera de su iglesia el buen Respaldiza, y D.
Fernando, a pesar de su garrotesca bravura, frunció el majestuoso ceño.
El primer impulso del tirador fue huir, más detúvole su amigo, bien
porque creyera imposible la fuga, bien porque la impavidez de su alma
atrevida gozase en la temerosa aproximación del peligro.

---¡El sable, el sable!---gritó tomando el arma de su amigo, a quien
entregó la espada vieja.

La mano del cura temblaba.

---Hemos cometido una acción villana asesinando a un hombre---exclamó
con solemne acento Garrote;---Dios nos castiga. Ahora\ldots{} pelear
como buenos españoles y morir como caballeros cristianos.

---¿Qué hacemos?

---¿Qué hemos de hacer? ¡A ellos! Dios sea con nosotros.

No hubo muchos ni variados lances en aquel suceso, porque en el espacio
de pocos minutos, los enemigos se acercaron a nuestros dos héroes,
diciéndoles en castellano que se rindieran.

---Son españoles.

---Afrancesados\ldots{} mala gente\ldots---murmuró D. Aparicio.

---¡Que me rinda yo!---gritó Navarro esgrimiendo el sable.---Ahora
sabréis, canallas, traidores, cómo acostumbra a hacer sus rendiciones D.
Fernando Garrote el de la Puebla. Si he de morir, moriré matando.

Y sin más dimes ni diretes, comenzó a descargar sablazos sobre los que
más cerca tenía. En tanto Respaldiza, viendo a su amigó enredado con los
franceses, quiso ponerse en salvo, pero se lo impidieron, y en un
santiamén fueron ambos desarmados. Garrote había descalabrado a uno y
herido levemente a otro, recibiendo en cambio dos pistoletazos, que por
fortuna sólo hicieron estragos en el alto sombrero. Gritó, vociferó,
injurió en nombre de Dios, del Rey y de España; pero al cabo, ambos
fueron conducidos prisioneros sobre sus mismas cabalgaduras, y muy bien
vigilados por los doce dragones, que se pusieron en marcha después de
recoger al herido.

Así acabó la grande, la memorable expedición de D. Fernando Garrote y el
reverendo beneficiado de la Puebla. Mientras esto sucedía, Carlos
Navarro y la compañía buscaban inútilmente a los dos viejos guerreros en
el camino de Uralde.

\hypertarget{xvi}{%
\chapter{XVI}\label{xvi}}

Silenciosamente, y abrumados de amargura y desesperación, marchaban los
dos prisioneros el uno tras el otro: los caballos que montaban no
parecían menos tristes que sus amos, a juzgar por la lentitud de su paso
y la inclinación de la cabeza. Los españoles y franceses que les habían
cogido y les custodiaban iban, charlando en una y otra lengua
mezcladamente, y uno de ellos dijo:

---A estos tunantes no les perdonará el general Gazan\ldots{} han
asesinado a un francés, y ya sabemos con qué moneda se pagan estas
deudas.

---El uno de ellos parece cura.

---Y el otro parece sacristán.

D. Fernando Garrote se puso lívido al oír que se le llamaba sacristán, y
después se le encendió hasta la raíz del cabello el pálido rostro. Si
hubiera tenido armas, habría castigado en el acto tanta insolencia en
menos que se dicen castañas. Respaldiza, durante el camino, sintiéndose
sediento, pidió que le dejaran beber de un arroyo cercano.

---Tiempo hay de beber. En Aríñez no falta agua, padrito. Y si no, tome
un buche de la del bautismo, que como cura debe de tener tan a la
mano\ldots{} Beberá antes que le despachen.

---¡Despacharme!---exclamó D. Aparicio con acento compungido.---¿Qué es
eso de despachar?

Garrote, colérico por la cobardía que mostraba su amigo, le miro con
ojos fieros.

---¡Que nos despachen!---dijo.---¿Qué mayor gloria para buenos españoles
que morir a manos de estos tunantes?

---Cierre el pico el vejete sacristán---gritó un jurado,---o no
aguardamos a llegar al cuartel general.

---¡Traidor! Tu persona es para mí tan despreciable como la de un vil
esclavo, y tus palabras como los ladridos de un perro---exclamó con
admirable entereza Navarro.---Si quieres darme la muerte aquí mismo,
dámela. Ni porque me mates he de aborrecerte más, ni porque me dejes
vivo he de estimarte. Soy un hombre leal que sirve a su patria, y tú un
cobarde desleal que sirve al enemigo.

En aquel mismo instante se acabara la vida y con la vida las hazañas de
D. Fernando Garrote, si el sargento que mandaba la tropa no impusiera
silencio a todos, mandándoles seguir adelante.

Después de tres horas largas y penosas de camino, llegaron a Aríñez, y
los dos prisioneros fueron presentados a un coronel. Las tropas
francesas entre las cuales se encontraban, pertenecían a la división del
general Gazan. Caía la tarde y los soldados se preparaban a pasar la
noche lo mejor posible: encendíanse las cocinas de campaña, y en torno a
las casas de labor se veían alegres corrillos. Los caballos bebían en
una gran acequia que de un punto a otro atravesaba el pueblo, y los
oficiales organizaban sus meriendas al aire libre.

D. Fernando Garrote se quedó sin alma cuando se vio entre aquella gente.
Deseaba morirse, o que la tierra se abriese para tragársele, o que
reventase a su lado el más poderoso de los cañones franceses. Lleváronle
de Herodes a Pilatos durante largo rato de la tardecita, cual si no
supiesen qué hacer de él, y unos le tenían lástima, otros le miraban con
desdén o con ira. Pero el que excitaba más sentimientos de enojo era D.
Aparicio, por ser muy aborrecidos entre los extranjeros los curas
armados; así es que después que le concedieron el apagar la rabiosa sed
en la misma acequia donde hociqueaban los caballos, echáronle una cuerda
al cuello, sin miramiento alguno a las órdenes sacerdotales.

No fueron tan crueles con Garrote, quizás porque mostraba mucha dignidad
en su infortunio y no hacía aspaviento ni exhalaba femeniles quejas como
su compañero. Lleváronles a los dos a un gran patio, contiguo a una casa
grande y vieja, el cual parecía servir de taller de herrería y
carretería, porque en él había varios soldados artífices trabajando, y
allí podían discurrir libremente los dos prisioneros; mas no escaparse,
porque un centinela guardaba la puerta.

Respaldiza, despavorido y medio muerto de terror, echose al suelo para
llorar su desventura. Navarro se paseaba de largo a largo, sin hablar a
su amigo ni a nadie. En las bardas de aquel corral que caían a poniente
había unas rejas por donde se veía la carretera de Vitoria. No cesaban
de pasar por ella carros cargados de cajas y arcones de diversos
tamaños, los cuales venían del lado de la Puebla, y se detenían,
acomodándose en el estrecho camino para dar descanso a las caballerías.
También había multitud de galeras y sillas de posta, donde iban las
familias españolas que abandonaban la corte con los franceses. El ruido
y el tumulto de aquella parte del camino donde se habían reunido y
amalgamaban tantos vehículos y caballos, eran espantosos. Unida esta
algazara con los martillazos de los que trabajaban sobre el yunque
dentro del patio, formábase una música infernal que hubiera vuelto loco
a D. Fernando Garrote si el cerebro de este pudiera descomponerse por
otra causa que por el espantoso hervir de las ideas.

Paseábase el esclarecido varón con la barba clavada en el pecho y las
manos dentro de los bolsillos: su espíritu después de vagar un buen
espacio por las dulces regiones del pensamiento religioso, se irritó de
repente y la idea del suicidio se le puso delante siniestra y halagüeña
a la vez, aterrándole y consolándole. Miró Navarro a los que machacaban
hierro sobre el yunque y consideró que le harían merced en dejarle poner
su vieja cabeza entre ambos hierros. Después fijó su atención en las
diversas herramientas que pendían del techo de un tingladillo donde
estaban la fragua y el fuelle; pero no creyó posible apoderarse de
ellas, ni menos usarlas contra su vida sin ser inmediatamente visto y
atajado. Volviendo al inquieto pasear, puso la atención en un pozo que
en mitad del patio había, y al punto hizo resolución de arrojarse en él
de cabeza; pero tardaba mucho en decidirse a ello, y observaba de
soslayo la soga y polea. Acercose al brocal para mirar al fondo y vio
allá abajo su imagen temblorosa y desfigurada dentro de un círculo
luminoso. En esta contemplación se detenía, cuando un francés le arrancó
de allí, señalándole la fragua.

---Camarada---le dijo en mal español con sonrisa burlona,---allí hacen
falta vuestros servicios.

Un español joven, moreno y agraciado acercose en tanto al cura, que no
se apartaba de su rincón y con acento de chacota le dijo:

---¿Qué bueno por aquí, Sr.~Respaldiza? Parece que la expedición no ha
salido bien.

---¡Ay Salvadorcillo de mi alma!---exclamó el cura con mucha
congoja.---Al verte, me parece que veo un ángel del cielo\ldots{} Dime
¿nos matarán?\ldots{} ¿Intercederás por nosotros? Yo te ruego que
olvides las palabrillas coléricas que se cruzaron entre nosotros anoche
en casa de tu madre. Yo suelo gastar esas bromitas\ldots{}

---Olvidadas están, señor cura; pero me parece que nada puedo hacer por
Vds. ¿Quién es el compañero?

---Allí lo tienes junto al pozo, D. Fernando Garrote, el primer
caballero de toda la comarca.

---Le hubiera conocido---dijo Monsalud observándole,---nada más que por
la semejanza que tiene con su hijo Carlos.

Y acercándose a Navarro, que en aquel instante disputaba con el francés,
tomó nuestro joven una expresioncilla bastante insolente, y habló de
este modo al infeliz anciano:

---Sr.~D. Fernando, aquí dicen que vaya Vd. a menear el fuelle, y yo
creo que este honroso oficio nadie puede desempeñarlo donde hay un señor
de la llave dorada.

Miró Garrote al atrevido soldado con tanta ira, que los ojos parecían
saltársele del casco.

---Mozuelo sin honor ni vergüenza---exclamó con dignidad y
altanería,---¿piensas que un hombre como yo ha venido aquí para oír tus
necedades ni menos para obedecerte? Estos miserables exterminarán a la
gente honrada; pero no la deshonrarán.

---¡Al fuelle! ¡al fuelle!---gritaron varias voces, y con más fuerza que
ninguna la del mozo que hasta entonces había movido sin descanso la
enfadosa máquina.

---¡Soplad vosotros, canallas!---gritó Navarro, echando inmediatamente
mano al lugar donde debía estar el puño de la espada.

---No hay que apurarse por tan poca cosa---dijo de improviso el cura
levantándose del suelo y acudiendo oficiosamente al lugar de la
disputa.---Si es preciso que alguien sople, yo soplaré, que lo haré muy
bien, caballeritos, y bueno es un poco de ejercicio a estas horas.

Deseando congraciarse con sus verdugos, Respaldiza cuya poquedad de
ánimo y corazón pequeño se habían mostrado ya, se prestaba a todo.

---¿Qué más da?---decía entre dientes.---Más padeció Jesús por nosotros.
A él le pusieron atado a una columna y le abofetearon y escupieron.
Movamos el fuelle, herreros de Satanás. Si vuestros cuerpos estuvieran
dentro del fuego, ¡con qué ganas soplaría!

Metió la mano en la argolla y tirando de la cadena infló el depósito de
viento. El caño de la fragua resonó con ardiente resoplido, como la
respiración de un cíclope, y las moribundas ascuas revivieron lanzando
llamas rojizas. Al compás del canto de los herreros, tiraba de la cadena
el cura, afectando en su semblante cristiano humildad; pero lleno de
cólera y más que de cólera de miedo.

La noche sin luna oscurecía el cielo y la tierra; pero no cesaba el
espantoso ruido dentro y fuera del patio.

La roja claridad de la fragua iluminó los diversos grupos, y D.
Fernando, que tenía en su alma todas las oscuridades de la tristeza y
todas las llamas de la desesperación, no pudo pensar en echarse al pozo,
porque los franceses lo cerraron.

A ratos le causaba profunda pena ver la degradación y falta de dignidad
de su compañero de desgracia, el cual seguía en su tarea, y aun sonreía
ante los soeces herreros con mengua de su honor y de la jerarquía
sacerdotal. Por fin cesó el trabajo; entraron varios soldados españoles
y dos o tres renegados, trayendo un par de zaques de vino, a cuya vista
se regocijaron todos, disponiéndose a dejarlos vacíos. En el mismo
instante llegó Monsalud con algunos soldados, y ordenando a los
prisioneros que le siguiesen entró con ellos en el piso bajo de la casa
contigua, que lo era de labor y estaba destinada en su parte alta a
alojamiento de oficiales. Sin decirles cosa alguna, encerró a cada uno
en una pieza baja, separadas ambas por un tabique ruinoso, y sin puerta
que las comunicara. Luego que D. Fernando entró en lo que parecía
mazmorra, echose en el desnudo piso sin mirar al que le había encerrado.
Este arrojó un pan en el suelo, y como cayese a regular distancia del
prisionero, el sargento empujó la hogaza con la punta del pie, diciendo:

---Ahí tiene Vd. para pasar la noche. Estoy de guardia hasta las doce y
me han encargado la custodia de los dos prisioneros. Traeré también agua
y algo de carne, si hay.

---No necesito nada---dijo Garrote sin mirarle.---Yo no como tu pan.

Incorporándose, dio tan fuerte puntapié a la libreta que la lanzó al
otro extremo de la pieza.

---Mal genio tiene Vd.---dijo el joven con lástima.---Hay que llevarlo
con paciencia. El coronel me ha mandado que después de encerrar e
incomunicar a Vd. y a su compañero les notifique\ldots{}

---Ya lo sé\ldots{} que seremos arcabuceados\ldots{}

---A la madrugada. El general no quiere carnicerías; pero el jueves
cogió Mina a diez franceses y a todos los degolló.

---Hizo bien---dijo D. Fernando;---y es lástima que no te cogiera
también a ti, español renegado a lo que pareces\ldots{} Si Dios me
sacara de esta cárcel y recobrase yo mi libertad y mis armas a ningún
afrancesado perdonaría.

---Amigo---dijo el joven,---la situación en que Vd. se halla no es la
más propia para vituperar la conducta de los demás y poner cual no digan
dueñas a los que, por razones que Vd. ignora, servimos a los franceses.

---Mi situación no me espanta---repuso el viejo con gravedad.---Moriré
por la patria, por la religión, y Dios me acogerá en su seno. La muerte
que me espera no la cambiaría por cien vidas como la tuya, infeliz
joven, por esa vida deshonrada en flor.

El mozo guardó silencio.

---¿Quién te engañó? ¿Quién te sedujo? ¿Sabes lo que es servir al
enemigo y hacer causa común con los verdugos de la patria?

---Hablador es el viejo---dijo Salvador un poco enojado.---Hará Vd. bien
en descansar y en tranquilizarse, Sr.~Navarro. Adiós.

---¿Cómo sabes mi nombre?

---Me lo dijo Respaldiza. Conozco mucho al cura de la Puebla de
Arganzón, donde he vivido dos años.

---¿Cómo te llamas?

---Salvador Monsalud\ldots{} yo soy de Pipaón.

El anciano dio un suspiro profundo echando hacia atrás la cabeza, que al
chocar bruscamente contra el tabique produjo un triste y hueco sonido
como el de un cántaro que está a punto de romperse.

---Adiós---dijo el joven con la mayor indiferencia.---Volveré después a
traer a Vds. alguna cosa. Me da lástima de los que van a morir aunque se
lo tengan muy merecido\ldots{} ¿Conque agua? Si hubiera carne\ldots{}
Veremos.

\hypertarget{xvii}{%
\chapter{XVII}\label{xvii}}

El estado moral de D. Fernando Garrote fue, desde que se quedó solo, el
más espantoso que imaginarse puede. La imagen y la idea de la muerte que
poco antes ocuparan por completo su espíritu, huyeron como accidentes
fútiles y pasajeros, indignos del pensamiento. Toda su vida pasada, sus
culpas, sus glorias se le pusieron delante juntamente con el infeliz
joven cuyo nombre acababa de saber. Veía tan claro el designio de Dios,
que hasta con los ojos del cuerpo estaba viendo al mismo Dios delante de
sí, grave, ceñudo, majestuoso y admirablemente sobrenatural y divino. D.
Fernando sintió el terror más vivo que un alma humana puede sentir,
miedo semejante tan sólo a los terrores bíblicos que sobrecogían al
pueblo elegido, cuando entre rayos y truenos sonaba la voz que había
mandado a la luz que se hiciera, y a la tierra separarse de las aguas.

El anciano se prosternó en tierra y apoyando contra las frías baldosas
su ardiente cabeza, dijo en voz alta:

---¡Señor, Señor, lo merezco! ¡He sido un malvado! ¡Cúmplase tu
voluntad! ¡Justicia terrible, pero justicia al fin! ¡Digna de mi vida es
esta última hora que has dispuesto para mí!

Después siguió balbuciendo en voz baja oraciones piadosas y vehementes
hasta que su alma se fue tranquilizando poco a poco y las terribles
majestuosas facciones del semblante de Dios, que delante creía ver, se
amansaron. El pobre anciano respiró y levantándose del suelo fue
tentando las paredes hasta el rincón más próximo, donde se acurrucó,
cruzando las piernas y los brazos, y entre estos escondiendo la cabeza,
de tal modo que parecía un ovillo. En tal postura, solo, sin movimiento,
profundamente abstraído y encerrado dentro de sí mismo, como el gusano
en su capullo, dijo el soliloquio siguiente, examen sincero de sus
muchas culpas:

---«Consagré mi juventud al vicio. Obediente a la ley de Dios tan sólo
en lo superficial y externo, falté a todos los deberes cristianos. Iba
todos los días a misa y rezaba el rosario, ambos actos sin devoción y
por pura rutina, pues en misa no atendía más que a las mujeres que
poblaban la iglesia. Llamándome buen católico, y defendiendo de palabra
y aun de obra la religión siempre que se ofrecía, mi conducta no dejaba
de ser execrable. ¿De qué valía a mi alma el ser presidente por derecho
hereditario de la sagrada congregación de Esclavos de Cristo, ni hermano
mayor de la Virgen de la Asunción, y guardián de su camarín, cuyas
llaves se han conservado siempre en las arcas de mi familia, con el
derecho de vestir la imagen en las grandes fiestas?\ldots{} ¡Ay! He sido
un perverso que se ha burlado de todas las leyes divinas y humanas.
Amonestome un buen religioso francisco; pero me burlé de sus palabras
atendiendo más que a él a los que me adulaban fomentando con viles
alabanzas mi disolución.

»Diome el cielo fortuna, sin duda por probarme en el empleo que de ella
haría, y más valiera que me criara Dios pobre y desnudo, para que así mi
natural vicioso se encaminase a la virtud, y con las abstinencias se
educara firme y valerosa mi alma. Mas yo empleé mi hacienda en
deslumbrar con engañosos oropeles la inocencia, en seducir con mentidas
promesas a honradas familias, en corromper dueñas y criadas. Hice del
honor mercadería que con el oro se compra y se vende, y de la paz y
buena fama de las familias, un juego caprichoso. El demonio, mi aliado y
en realidad mi Dios, sugeríame a cada instante artificios nuevos para
derrocar la honestidad y vencer la resistencia, que la templanza y el
recato ofrecían a mis abominables apetitos. Todo lo atropellé; pisoteé
los sentimientos más puros como pisotean los cerdos las flores de un
jardín, sin comprender su belleza.

»Dios me tocaba a veces el corazón, dándome ratos de profunda tristeza
en los cuales mi conciencia aclarándose ante mí con prodigiosa luz, me
ponía delante la fealdad horrenda de mi conducta; mas estos momentos que
coincidían siempre con mi cansancio, eran breves como los relámpagos en
la noche oscura, y mi alma envilecida dejaba el arrepentimiento para la
vejez. Mi memoria con ser portentosa, no puede recordar uno por uno
todos los desafueros que cometí, los planes execrables que realicé, ni
las víctimas todas de mi salvaje descomedimiento. Pero en estos momentos
terribles en que mi conciencia a la vista de un hombre se ha abierto de
súbito como una sima llena de horrores, y se me ha presentado Dios con
el semblante de la justicia, aprestándose a juzgarme sin misericordia
porque no la merezco, uno solo de mis crímenes se me ofrece visible y
claro entre los demás, porque a todos los compendia, y con su magnitud
oscurece a los otros.

»La ejemplar persona sacrificada vive, al parecer para mi castigo. ¡Ay!
A muchas seduje, a muchas atropellé; pero con ninguna fue el engaño tan
torpe y miserable como con esta. Cuanto puede hacer un hombre para
disimular su vil intención, yo lo hice; cuanto puede inventarse para
aparecer bueno sin serlo y apasionado sin estarlo, mi entendimiento,
fecundo siempre para el mal, lo inventó con pasmoso ingenio. Burleme
después de la desgraciada joven a quien sacrifiqué y yo mismo aplaudí su
deshonra en reunión de inicuos amigos y calaveras. Llevado de no sé qué
perversos instintos, que desde entonces han sido causa en mí de
espantosos remordimientos, llegué hasta a suponer en aquella infeliz
faltas que no había cometido, y torpezas y tratos con otros hombres que
jamás se acercaron a ella. ¡Escupir el cadáver de la víctima que se
acaba de inmolar, no es tan vil como lo que yo hice! ¡Ay! ¿Por qué no
taladró mi lengua un hierro encendido como esos que he visto esta tarde
en la fragua del patio? ¡Oh, Dios mío! ¿Por qué no quedé paralítico,
ciego y mudo, sin sentido para la maldad, y sólo con pensamiento para
meditar en mi merecida ruina y pensar en mi salvación?

»Nació un niño a quien pusieron por nombre Salvador. Me lo dijeron y lo
oí como si oyera decir: `La vaca del vecino ha parido un ternero'. Ya no
volví a Pipaón desde que proyecté casarme con otra mujer. Olvidado de mí
aventura, llegué sin embargo a entender que la hermosa hija de D. Pablo
el Riojano había quedado en la miseria. Nada hice por ella; poco a poco
fue envolviéndose en nubes de misterio lo sucedido y la madre y el hijo
no existieron para mí. Hace tres años dijéronme que un joven llamado
Salvador Monsalud había aparecido en la Puebla en compañía de su madre,
mujer melancólica, piadosa y enferma. Sentí cierta aflicción
inexplicable, pero nada hice. El amor de mi hijo legítimo me ocupaba por
entero. Hace poco, y aún hoy mismo, doña Perpetua me ha recordado la
antigua y casi olvidada deuda; mas preocupado con mis preparativos de
guerra y soñando con gloriosas hazañas, apenas detuve el pensamiento en
los dos desgraciados seres que tan cerca estaban de mí\ldots{}

»Ha tiempo, sin embargo, que el arrepentimiento trabaja en mi alma,
labrándose en ella un hueco con lentitud, pero con constancia. He vuelto
los ojos a Dios aunque de soslayo, y a fuerza de pensar en mis culpas y
en la justicia divina, he llegado a considerar que el mejor desagravio
que a Dios podía ofrecer era sacrificarle los últimos días de mi vida,
combatiendo por la fe verdadera contra los herejes y renegados. En mi
necio orgullo no he comprendido hasta ahora que Dios no podía aceptarme
como diligente servidor, ni menos premiar mi arrojo. Clara, como la luz
del sol al medio del día, veo ahora su mano llevándome al destino y fin
deplorable que merecía; veo su lógico designio, obra de la perpetua
justicia, en los sucesos de esta tarde, y más que en otra cosa alguna,
en la presencia de ese joven, de ese ejemplo vivo de mis crímenes, de
esa venganza humana y celeste, de ese malaventurado hijo mío, que con la
frialdad de los verdugos y la crueldad de un enemigo vencedor se me ha
puesto delante para anunciar la muerte que merezco. ¡Oh! merezco más,
mucho más, Señor, merezco vivir después de lo que he visto.

»Las facciones de este muchacho han producido en mí incomprensible
turbación; su nombre, pronunciado por él mismo, ha caído sobre mí como
un rayo celeste. Ya sé cómo suenan las trompetas del Juicio. Dios mío,
estoy humillado, vencido y me arrastro por el suelo como un insecto
miserable, buscando tu pie soberano para que me aplaste. Me creo indigno
hasta de mirar la luz del día que criaste lo mismo para los buenos que
para los malos. Señor, la muerte que me aguarda no será bastante cruel
para lo que yo merezco. Un hombre que lleva mi sangre y debiera llevar
mi nombre, me custodia en esta mazmorra hasta que llegue el instante de
la muerte; y él mismo, si se lo mandan\ldots»

D. Fernando no se atrevió a continuar la frase, que no era dicha sino
pensada, y aun así la sofocó cortando el vuelo de su pensamiento,
suspendiendo la fórmula oscura del lenguaje con que discurrimos a solas
y en silencio; pero no pudo cortar, ni atajar, ni detener la idea que
surcó por su cerebro como un relámpago. Espantado de ella, se afirmó con
ambas manos las abrasadas sienes, sacudiéndose a un lado y otro la
cabeza. Si quisiera arrancársela y arrojarla lejos de sí, como un
despojo inútil, no lo hiciera de otra manera.

Oyó una voz alegre que cantaba y al mismo tiempo abrieron la puerta.
Monsalud entró alumbrándose con una linterna, y traía además una botella
de vino.

---Sr.~D. Fernando---dijo desde la puerta,---aquí le traigo esto para
que entone el cuerpo y le ayude a pasar los malos ratos de esta noche.

\hypertarget{xviii}{%
\chapter{XVIII}\label{xviii}}

Salvador adelantó con paso inseguro, dirigiendo la luz de la linterna a
todos los lados de la estancia.

---¿En dónde se ha metido Vd.?---dijo riendo a carcajadas como quien ha
perdido el equilibrio de sus facultades.---¡Ah! Está Vd. en el
rincón\ldots{} ¡qué postura! De ese modo piden los ciegos en los
caminos.

D. Fernando Garrote, ante aquellas burlas, sintió que su sangre se
trocaba en hielo.

---Entre esta gente---dijo con mucha aflicción,---¿es costumbre burlarse
de los desgraciados que van a morir?

---Perdóneme Vd.---añadió el joven luchando con el extravío de sus
sentidos.---No sé lo que digo\ldots{} esos pícaros hicieron propósito de
embriagarme, y si no me levanto pronto\ldots{}

---Vicio muy feo es el de la embriaguez---afirmó Garrote.---Un joven
valiente y noble como tú, ¿será capaz de degradarse, abusando del
vino?\ldots{}

---No, no señor---repuso Salvador, en quien la vergüenza pudo por un
momento más que la turbación de su mente.---Nunca he sido borracho, pero
de poco tiempo a esta parte me dan tales tristezas y se me acongoja el
alma de tal modo a consecuencia de mis desgracias, que algunas
veces\ldots{}

---¡Pobre muchacho!---dijo el guerrero, acercándose a Monsalud, que,
puesta en el suelo la linterna y la botella, se había sentado junto a
ellas.---Me parece que como joven inexperto y sin fundamento, no te
vendría mal recibir algunos consejos, y voy a dártelos.

---Pues toca la casualidad de que yo no he venido a recibir consejos,
sino a acompañar a Vd. un tantico y traerle algo confortativo, porque
siempre me da mucha compasión de ver a un hombre condenado a morir por
cosas de guerra, y aunque este hombre sea mi enemigo, sí, mi enemigo por
varias causas, siempre procuro que sus últimas horas no sean muy
tristes. Conque guárdese Vd. los consejos y beba vino, si gusta.

---No beberé---repuso D. Femando;---pero pues dices que vienes a hacerme
compañía, acepto el obsequio de un poco de conversación.

---¿De qué vamos a hablar?

---De ti.

---¡De mí!---exclamó Salvador, otra vez atacado de la nerviosa hilaridad
que tanto disgustara a Garrote.---¡Bonito asunto! Tanto vale hablar del
infierno.

---Al verte entre franceses, joven, apuesto, y con esa expresión de
nobleza que tiene tu persona\ldots{}

---¡Oh qué lisonjero está el buen hombre!---dijo Monsalud.---Amiguito,
no me adule Vd., pues aunque compasivo no me vendo por alabanzas.

---Al verte así---continuó Garrote,---he pensado que sólo seducido y
engañado ha podido un joven de tanto mérito entrar al servicio del Rey
José y de los enemigos de la patria y de la religión.

---Ni seducido, ni engañado, sino por mi propio gusto y libre
voluntad---respondió el mancebo con firmeza.

---¡Y por tus venas corre sangre española! ¿No aborreces a esos herejes,
asesinos y ladrones, de cuyos crímenes horrendos eres cómplice, sin
duda, por inocencia?

---No les aborrezco, sino que les estimo.

D. Femando cruzó las manos y elevó los ojos al cielo.

---Les estimo---prosiguió Monsalud,---porque ellos me ampararon cuando
de todos era abandonado; diéronme de comer cuando me moría de hambre, y
me pusieron este uniforme que han llevado los primeros soldados del
mundo y los vencedores de toda Europa.

Garrote se estremeció de espanto, y un abatimiento angustioso sucedió a
su anterior excitación.

---¿Pero tan pobre estabas y tan desamparado de todo el mundo, que
necesitases venderte a los franceses para vivir?

---Pobre y desamparado, sí, porque mi madre había perdido la poca
hacienda heredada, y no teníamos sobre qué caernos muertos. Yo fui a
Madrid, y un tío que allí tengo, me metió en un regimiento de la guardia
jurada.

---Pero tu deber es pelear por la patria. ¿No ves a toda la nación en
masa sublevada contra esos viles? ¿No ves el desprecio y el odio que
inspiran? Observa bien que entre los pocos españoles que sirven en las
filas francesas, no hay uno solo que sea persona honrada.

---¡Calumnia! Los hay muy buenos y yo no me tengo por ladrón,
Sr.~Garrote ---dijo Monsalud enojándose un poco.---Y punto en boca sobre
esa materia.

---Poco a poco, joven, no he querido ofenderte---repuso Navarro con
tanta humildad y timidez como un chico de escuela.---Te diré cuál ha
sido mi intento. Al verte, sentí profundas simpatías hacia ti, y tanto
me entristeció ver a un joven de mérito en la vil condición de
afrancesado y en la torpe esclavitud de esa canalla, que me atreví a
esperar que los consejos y la autoridad de este infeliz anciano, próximo
a morir, tendrían alguna fuerza para desviarte de ese infame camino, ¿Me
equivocaré, Salvador?---añadió con expresión muy afectuosa.---¿Será
posible que tu buen corazón y clara inteligencia no respondan a esta
cariñosa súplica mía, a este deseo de que te conviertas y dejes a tus
viles amos y vuelvas a la santa fe de la patria en que todos los buenos
españoles vivimos y morimos?

Monsalud miró a D. Fernando por breve espacio, de hito en hito, y
después rompió a reír con estrépito y descaro. El insigne Garrote no
pudo contemplar por mucho tiempo aquella faz burlona, porque tuvo que
esconder la suya entre las palmas de la mano, para ocultar el llanto.

---No ha sido malo el sermón, padrito---dijo el mozo.---¿Y Vd. qué
pedazo de pan se lleva a la boca con que yo sea afrancesado o deje de
serlo? A fe que me divierto oyéndole. ¡Buen modo de disponerse a una
buena muerte! A ver, padrito---añadió llenando un vaso de los dos que
había traído,---echemos un trago a la salud del gran Napoleón I,
Emperador de los franceses y señor de todo el mundo.

---No---dijo D. Fernando rechazando el vaso,---no puedo creer que digas
tales disparates formalmente. Eres joven, has bebido más de lo regular,
y no sabes lo que sale de tu boca\ldots{} Comprendo bien la causa
principal de tu falta. Te sentías con ardor guerrero, heredado, sin
duda, del que te dio el ser y la vida, y como los franceses tienen buena
labia para deslumbrar a los jóvenes hablándoles de las grandezas del
Imperio y de sus fabulosas batallas de Italia y Alemania, caíste en la
trampa. ¡Qué necedad! La más arrebatada fantasía no puede soñar triunfos
tan grandes como los que hemos alcanzado nosotros en esta guerra contra
los decantados ejércitos de Napoleón. Nuestras batallas de Bailén, de la
Albuera, de Tamames, de Talavera, y las defensas gloriosísimas de
Zaragoza, Gerona y Tarragona, no tienen igual ni aun en los fastos de la
antigüedad heroica. Y si estos hechos no fuesen aún de suficiente
magnitud para lo que ambiciona tu grande espíritu, ahí tienes
diseminadas por toda la redondez de España, esas inimitables partidas de
guerrilleros, los más bravos, los más atrevidos, los más generosos y
leales hombres de la tierra, los verdaderos libertadores de la patria,
los que al fin rescatarán a nuestro adorado Fernando, los que devolverán
a la sagrada religión su esplendor y a Dios su reino predilecto.

Antes que concluyera, Monsalud había empezado a reír. Tomó las
elocuentes amonestaciones del anciano como materia de placenteras
burlas, y resuelto a contrariarle en todo por convicción, le dijo:

---No me hable Vd. de los guerrilleros, que si hay en la tierra plebe
inmunda digna del presidio, ellos son. Compónense las partidas de los
asesinos, ladrones y contrabandistas de cada lugar, con más los
holgazanes, que son casi todos. Hacen la guerra, por robar, no por echar
de aquí a los franceses, y si algún día se acabaran estas misas, el Rey
Fernando tendría que colgarlos a todos para poder reinar en paz.

D. Fernando exhaló hondísimo suspiro; mas no desesperanzado todavía de
tocar alguna fibra sensible en el corazón del mancebo, le habló así:

---Aunque los guerrilleros fueran como dices, que no son sino lo
contrario, no podrías justificar tu conducta. A todos has hecho
traición, Salvador, a lo divino y a lo humano; has hecho traición a la
patria, a los españoles que son tus hermanos; has hecho traición a tu
madre, que sin duda es española también y enemiga de nuestros enemigos;
has hecho traición al Rey, bajo cuyo amparo nacimos y en cuya veneranda
persona se representa nuestro hogar y el sol que nos alumbra, y
principalmente has hecho traición a Dios, cuya fe, más pura y fuerte en
la nación española que en ninguna otra, han venido a destruir los
franceses, introduciendo aquí, con la herejía, mil costumbres y
prácticas nuevas que no conducen sino al pecado.

---Dios\ldots{} ¡Buen caso hago yo de Dios!---exclamó el mancebo con un
cinismo que llevó a su último extremo los temores de D. Fernando.---¡Qué
atrasada está la gente por aquí!\ldots{} No hay ninguno que haya leído a
Voltaire, como lo he leído yo en todas las paradas del viaje desde que
salí de Madrid.

---¡Desgraciado!---exclamó el anciano poniendo sus manos sobre los
hombros del joven.---¿Qué estás diciendo?

---¡Dios! Una palabrota y nada más. Si lo hay, que lo dudo mucho, estará
allá arriba acariciándose la barba blanca y sin meterse en nuestros
asuntos. Dígolo, porque muchas veces lo llamé y\ldots{} ¿me oyó Vd.?
Pues él tampoco.

---¡Desgraciado!---repitió el anciano.---¡Mil veces más desgraciado que
si cayeras para siempre traspasado por las bayonetas de tus viles
amigos! ¿No crees en Dios omnipotente, justo y misericordioso? ¿No crees
en la Santísima Trinidad? ¿No crees en la Encarnación del hijo de Dios,
ni en su pasión y muerte por redimirnos del pecado?

---¡Oh cuánta monserga y cuánto embrollo!---repuso Monsalud
riendo.---¡La Trinidad! Tres que son uno y uno que viene a ser tres.
Bonito lío han armado\ldots{} Jesucristo no era más que un buen
predicador y tan hombre como yo. Y de la llamada Virgen María ¿qué puedo
decir sino que\ldots?

---Calla, calla, blasfemo infame---gritó con encendida cólera D.
Fernando, poniendo su mano en la boca del descomedido muchacho.---Tú no
eres, no puedes ser lo que yo creí.

---¿Qué hombre ilustrado cree hoy semejantes paparruchas? Todo eso lo
han inventado los frailes para engañar y dominar al pueblo, embobándolo
con pantomimas ridículas y prácticas necias. ¡Los frailes!---añadió con
cierta petulancia.---¿Hay casta de cerdos más inmunda en todo el orbe?
Yo digo que hasta que no ahorquen al último Papa con las tripas del
último fraile, no habrá paz en el mundo. Ellos son los que promueven las
guerras, los que hacen estúpidos a los Reyes; ellos son los que han
levantado a la nación española, no por religiosidad, sino porque saben
que el deseo de Napoleón es quitarles sus inmensas y mal empleadas
riquezas para dárselas a los pobres.

---No, no---repetía D. Fernando con vehemencia, contemplando a Salvador
con atónita atención;---no eres tú lo que yo creí, no eres tú quien yo
creí, no, mil veces no, voto a\ldots{} Afrancesado, traidor a la patria,
desleal con el Rey, irreligioso, blasfemo, no te falta sino ser mal hijo
para que eternamente estés separado de mí.

---¡Mal hijo! Si lo soy no es culpa mía---dijo el mancebo bebiendo el
vino que había escanciado para el Sr.~Garrote.---Mi madre es una
excelente mujer; pero muy sencilla e inocente, y se ha dejado dominar
por D.ª Perpetua y por los frailes de la Puebla. Empeñose en que
abandonara mis banderas; negueme a ello, echome de su casa, yo salí, se
desmayó\ldots{} Las mujeres no atienden más que a su capricho; son
vanas, frívolas, superficiales, mojigatas, y le aburren a uno con sus
rezos\ldots{} No hagamos caso de tales simplezas y bebamos, Sr.~D.
Fernando. Otro traguito.

---Tu madre---dijo D. Fernando,---es, según tengo entendido, una santa y
honrada mujer, de sanos principios.

---Pues sus principios no son los míos, ni lo serán nunca. Ella adora
las atrocidades de los salvajes guerrilleros, y yo las aborrezco; ella
se mira en Fernando VII, y yo lo tengo por un principillo corrompido y
voluntarioso; ella detesta a los afrancesados, y yo les tengo por muy
buenos patriotas, porque quieren regenerar a España con las ideas de
Napoleón; ella no puede ver a los que han hecho la Constitución de Cádiz
ni a los que se llaman liberales, y yo les admiro por creerlos
inclinados a echarse en nuestros brazos\ldots{}

---¡Perdido, perdido para siempre!---exclamó D. Fernando con inmensa
angustia.---¡Sin honor, sin principios, sin patriotismo, sin religión,
sin lazo alguno con la sociedad, ni con España, ni con la familia, ni
con Dios\ldots! ¡Oh qué aflicción, qué castigo, Dios mío!

---Puesto que Vd. no quiere probarlo---dijo el sargento, echando otro
medio cuartillo,---me lo beberé yo. Luego dormiré seis horas y así se
olvidan ciertas cosas, cosas terribles Sr.~D. Fernando, que atormentan
noche y día.

---Dios te tocará en el corazón, infeliz joven---dijo Navarro,---y hará
penetrar un rayo de su divina luz en tu oscuro entendimiento, y te
reconciliarás con España, con Dios, con tu madre y\ldots{} conmigo.

---¿Reconciliarme yo?---dijo el joven severamente dejando a un lado el
vaso vacío.---Yo no me reconciliaré jamás; eché los dados. Me voy a
Francia; consagraré mi vida a trabajar contra esta fementida patria que
aborrezco.

---Justamente despreciado por los hombres y maldecido por Dios, tu vida
será un infierno y tu muerte horrorosa y desesperada como la mía.
Mírame, en mí tienes un ejemplo de cómo castiga Dios en la última hora a
los que han olvidado su doctrina. Sin ser blasfemo ni traidor, como tú,
yo he sido muy pecador. He vivido largo tiempo con vida placentera y
feliz; pero en esta postrera noche de mi vida, me considero el más
desgraciado de los hombres, no seguramente por la muerte que me amenaza
y que merezco y deseo, pues los españoles debemos morir como caballeros
y como cristianos. Uno de los más amargos motivos de pena para mí, es
verte insensible a mis ruegos, degradado, envilecido, verte en el camino
de tu total mengua y perdición, sin poder remediarlo; verte en ese
estado de locura y embriaguez, aferrado a la maldad. Si respondieras,
aunque sólo fuese con eco muy débil, a mis sentimientos y a mis ideas,
si no me parecieses, como me pareces, un verdadero monstruo, esta
pasajera amistad que nos une podría ser un sentimiento más grande,
Salvador, mucho más grande y hermoso para ti y para mí.

Monsalud le miró con sorpresa.

---He sentido vivísima inclinación hacia ti---continuó el anciano.---En
esta soledad en que me encuentro, ausente de los míos, con un pie dentro
del sepulcro y la eternidad llamando a mi alma, tú podrías ser consuelo
inefable de este anciano moribundo, recibiendo en cambio de mí lo que
jamás has tenido, ni esperas tener.

Monsalud se levantó y con súbita cólera apostrofó al anciano en estos
términos.

---Viejo astuto, ¿quieres engañarme con lisonjas y gatuperios para que
te deje escapar? Yo no soy como los guerrilleros, que se venden por
dinero. Su señoría de la llave dorada no conoce con qué clase de
personas está tratando. ¡Pues no es poco sabihondo el viejecito!\ldots{}

---¡Miserable!---exclamó D. Fernando, sin poder contener su cólera y
levantándose también.---Veo que en ti no puede caber ningún sentimiento
generoso. ¡Mereces la abyección en que vives! Márchate, quiero estar
solo.

---¡Si será preciso ponerle algunas arrobas de hierro en los pies al D.
Quijote de la Puebla!---dijo Monsalud dando algunos pasos, con escasa
seguridad\ldots---Parece que se tambalea el piso\ldots{} Adiós, hasta
después. Tengo que hacer.

D. Fernando fue de aquí para allí con inmensa agitación. Hizo por último
el espanto lugar en él una violenta y súbita cólera, que se manifestara
en sus gestos y voces de un modo que asombró más a Salvador.

---¡No eres tú, tú no eres, no!---exclamó con atronadora voz.---¡Me he
equivocado! Dios se está burlando de mí\ldots{} es un castigo; ¡pero qué
castigo, Dios mío!

Sin comprender aquellas palabras, Salvador se detuvo ante el agitado
anciano. La generosidad de su noble corazón eclipsada por falsas ideas,
y la turbación física en que se hallaba, inspirole algunas palabras
consoladoras para el anciano; mas un hecho trivial le desvió de aquel
buen camino, separando a uno y otro personaje más de lo que estaban. En
la versatilidad de sus juicios, Salvador achacó las incoherentes
palabras de Garrote a extenuación y debilidad mental ocasionada por la
falta de sustento y el pavor de la próxima muerte. Pensándolo así, echó
en el vaso cuanto en la botella restaba, y con intención compasiva, le
dijo:

---¡Vaya, pelillos a la mar! Sr.~Garrote\ldots{} beba Vd. y le caerá
bien\ldots{} Luego llevaré otro gaudeamos al señor cura.

---Quita allá---contestó D. Fernando, apartándose con horror del
joven.---Tú no eres quien yo creí\ldots{} Tú eres de casta de borrachos
y traidores.

Recibió Salvador con paciencia el insulto, y empinando el codo, dijo:

---Puesto que Vd. no lo quiere, no se desperdiciará tan buen vino. Se lo
quitamos a unos arrieros que venían de la Nava.

La cabeza de Monsalud, que era de muy poca resistencia para la bebida, a
causa de su antigua sobriedad, luego que su cuerpo recibió aquel
trasiego, se desorganizó completamente; se oscurecieron sus facultades,
desmayose su cuerpo, entrole de improviso la innoble estupidez y el
repugnante cinismo de que había dado ya algunas pruebas en la
conferencia con su padre, y perdió su carácter, su generosidad, su buen
juicio, su discreción, perdiolo todo, para no ser más que un vulgar
soldado.

---Sr.~Garrote\ldots---dijo tambaleándose,---adiós\ldots{} Parece que se
mueve el piso\ldots{} ¿por qué baila Vd.?\ldots{}

---Vete, vete, déjame solo---replicó D. Fernando sin mirarle.

---¡Bonito fin han tenido las campañas del padre Respaldiza y del
Sr.~Navarro!---exclamó lanzando una carcajada de imbecilidad que retumbó
en la estancia como un eco infernal.---¡Bonito fin!\ldots{} ¡Échese su
merced a guerrillero!\ldots{} ¡Quién lo había de decir\ldots{} aquí está
el primer caballero del condado, el de la llave dorada, el gran D.
Fernando Garrote, que quiso derrotar él solo los ejércitos de
Napoleón!\ldots{} ¿Por qué no trajo consigo a Carlitos para que le
sacara del paso?\ldots{} Me hubiera gustado ver a todo el hato de
salteadores de caminos, distribuidos en estas cámaras reales, esperando
la orden del coronel\ldots{} ¡Adiós, señor D. Fernando Quijote,
adiós\ldots{} buen viaje!\ldots{}

D. Fernando se acercó a Salvador, y asiéndole el brazo y apretándole con
tanta fuerza como si su mano fuese una tenaza de hierro, le dijo
sombríamente:

---Salvador, cuando me saquen de este calabozo haz fuego sobre mí: mi
destino es ese, mi castigo no será el castigo que merezco, si no sucede
así. ¡Dios lo quiere!

---¿Fuego yo?---repuso el joven con sonrisa de demente.---Yo me
voy\ldots{} Salgo de guardia ahora\ldots{} Entrará otro\ldots{} No
quiero matar\ldots{} me da mucho temblor y me pongo malo.

Lucharon por breve rato en la acongojada alma del guerrero sentimientos
diversos. Luego sintió que las lágrimas brotaban de sus ojos; una
aflicción horrible le abrumaba. Apartose del joven, corrió luego hacia
él, mas su aspecto, su habla, su embriaguez le llenaron de espanto.

---Mi muerte---exclamó,---por las circunstancias espantosas que la
rodean, no se parece a ninguna otra muerte. Creo que toda la naturaleza
se desquicia en derredor mío y que en medio del cataclismo general, vivo
muriendo. Me parece que la muerte del malvado, como la del justo entre
los justos, no puede verificarse sino entre tinieblas horrorosas y
confusión del cielo con la tierra. ¿Es de noche? ¿Es de día? ¿Eres un
ángel o un demonio?\ldots{} Huye de aquí, monstruo mío\ldots{} No sé lo
que siente mi alma al verte y al oírte\ldots{} ¿Esto es vida o qué es
esto? ¡Dios poderoso, acoge mi alma,\ldots{} y basta, basta ya de
suplicio!

El Sr.~Garrote se arrojó al suelo. Monsalud a causa del vino, no vio en
todo aquello más que demencia y miedo. Hasta que no se halló fuera y
recibió en el rostro el fresco de la noche no se aclararon sus juicios,
ni pudo conocer que había estado inconveniente, cruel y\ldots{} grosero.

\hypertarget{xix}{%
\chapter{XIX}\label{xix}}

Cuando se quedó solo, elevó D. Fernando de nuevo su pensamiento a Dios.
Adquirió con esto cierta tranquilidad, cierto reposo emanado de la
profunda convicción de su inmensa desgracia, y aceptando aquella
amargura se engrandecía a sus propios ojos. La fogosidad de su
imaginación llevábale a compararse con los colosos de infortunio, pero
superándolos a todos: tan pronto recordaba a Job de la antigüedad
hebraica, como a Edipo, de los tiempos heroicos, y hasta en sus
coloquios, en sus alegatos ora tiernos, ora coléricos con la divinidad
se les parecía.

Después de un instante de estupor contemplativo sintió anhelo vivísimo
de comunicar a alguien la congoja de su alma, y se acordó de su amigo
Respaldiza, cuya voz había oído poco antes al través del tabique sin
hacerle caso. La endeble pared consistía en un armazón de maderas y
adobes, cubierta a trechos de viejísimo yeso, que formaba en sus
irregulares claros y fajas al modo de un fantástico mapa. Por diversas
partes, y principalmente junto al suelo, había muchos agujeros por donde
podían pasar el ruido y la claridad, pero no objeto alguno de más grueso
que un dedo. Golpeó don Fernando el tabique, diciendo:

---Sr.~D. Aparicio, Sr.~Respaldiza, ¿está usted ahí?

El cura contestó desde la otra parte:

---Sí, Sr.~D. Fernando, aquí estoy más muerto que vivo. ¿Con quién
hablaba Vd.?\ldots{} ¿Hay esperanzas de salvación? Me parece que trataba
Vd. con Salvadorcillo Monsalud\ldots{} Es mal sujeto, no hay que fiarse
mucho de él.

---Amigo Respaldiza---dijo Garrote sentándose en el suelo y apoyando su
rostro en la pared, junto a un sitio donde menudeaban las
grietas.---Acérquese Vd. a este sitio donde me encuentro, y óigame.
Tengo que hablarle.

---Ya estoy\ldots{} ¿Hay esperanzas de escapatoria?

---No hay que pensar en escaparse, señor cura. Nuestra muerte es
inevitable.

---¡Oh! ¡Dios mío Jesucristo!---exclamó Respaldiza con voz desfigurada
por la aflicción y el llanto.---¿Qué hemos hecho para tan triste
fin?\ldots{} ¿Pero no será posible intentar?\ldots{} Echemos abajo este
tabique; juntémonos, y entre los dos ejecutaremos algo ingenioso para
salir de aquí.

---Es difícil. Por mi parte no intentaré nada para salvar esta miserable
vida, que es para mí el más horroroso peso. ¡Somos muy pecadores!

---Yo no tanto\ldots{} ¿pero es posible que no logremos\ldots? ¡Oh!
Desde aquí siento los aullidos de esos lobos carniceros, de esos
demonios del infierno que nos guardan. Están borrachos, y parece como
que bailan y juegan.

---No nos ocupemos de nuestros enemigos, y pensemos en la salvación de
nuestras almas---dijo con unción D. Fernando.---Sr.~Respaldiza, Vd. es
sacerdote.

---Sí, sacerdote soy---repuso con desesperación el clérigo,---y como
sacerdote, digo que esto es una gran picardía, una gran infamia, un
asesinato horrendo. ¡Ya se las verán con Dios!

---Vd. es sacerdote---añadió D. Fernando,---y un buen sacerdote,
piadoso, instruido, aunque ahora caigo en que no cuadraba muy bien a su
estado tener tan buena puntería; pero sea lo que quiera, Vd. es un
hombre de bien, y un sacerdote cristiano, a cuyas manos baja Dios en el
santo oficio de la Misa.

---Sí, sí.

---Pues bien, siendo Vd. sacerdote y yo pecador, quiero confesarme en
esta hora suprema; quiero confesarme, sí, después de treinta y tantos
años de impenitencia.

Prolongado silencio anunció el estupor del sacerdote.

---¿No me contesta Vd.?---preguntó impaciente Navarro.

---¡Confesarse!\ldots{} Linda ocasión ha escogido usted\ldots{} Sobre
que todavía puede ser que nos indulten.

---No hay que esperar tal cosa. Seamos dignos de nosotros mismos, y
muramos como caballeros cristianos.

---¡Morir, morir!---repitió angustiosamente el cura.

Retembló el tabique con sordo estampido. La cabeza de Respaldiza había
chocado violentamente contra él.

---Sr.~D. Aparicio---dijo D. Fernando después de una pausa,---he visto a
Dios.

---¿A Dios?\ldots{} ¿Dónde, amigo mío, dónde?

---Aquí, aquí mismo en este oscuro calabozo. He visto pasar ante mí
también mi vida entera, y me han ocurrido cosas que espantarán a Vd. en
cuanto se las refiera.

---¡Es singular! ¡Ver a Dios y no pedirle que nos sacara de
aquí!\ldots{} ¡Ah! Vd. tiene razón, seamos piadosos y buenos cristianos
en esta hora suprema, único medio de que Dios nos favorezca. Chillar y
jurar con desesperación en estos trances no es propio del espíritu
cristiano. Recemos, Sr.~D. Fernando, oremos humildemente con toda la
compostura y devoción posibles. No se me olvidó el rosario; aquí está.
Pidamos a Dios de todo corazón que\ldots{}

---Antes conferenciemos un poco---dijo Garrote,---pues no sólo tengo que
revelar a Vd. secretos muy graves, sino pedirle consejo y parecer sobre
algún punto delicado de mi conciencia.

---Ya soy todo oídos.

---Bien sabe Vd., venerable amigo, que he sido gran pecador, un hombre
disoluto, despreocupado, vicioso, un libertino. Verdad es que jamás me
separé de la Iglesia; pero esto no atenúa mis grandes faltas, ¿no es
verdad?

---Verdad. Respecto a sus escándalos, amigo Garrote, muchos y grandes
han sido en la Puebla. He oído contar horrores; mas nunca me atreví a
reprenderle, por ser Vd. un excelente sujeto y haber tenido conmigo
delicadas deferencias. Tratándose de los más humildes feligreses de mi
parroquia, sí me atrevía yo a reprenderles sus vicios; pero a un señorón
como usted\ldots{}

---La ley de Dios es igual para todos\ldots{} Pero vamos adelante.
Muchos desafueros cometí, muchas honras atropellé, muchas desdichas
causé, y no hubo casa donde yo pusiese mi planta maldita, que al
instante no se inficionase con la corrupción y deshonra que llevaba
conmigo.

En este tono y con verdadera humildad cristiana prosiguió D. Fernando
refiriendo sus culpas, sin detenerse en los casos particulares, hasta
que llegando al punto capital de su confesión, dijo lo que sigue:

---Pero la más grave de mis faltas, por el cúmulo de circunstancias
denigrantes que en ella hubo, fue la deshonra de una doncella de Pipaón,
a quien engañé valiéndome de pérfidas astucias impropias de un
caballero, sí, pérfidas astucias y torpísimas artes que voy a enumerar
una por una, aunque al referirlas, la lengua parece que se me abrasa y
el rubor que enciende mi cara es como si una llama la envolviera toda.

Respiró con ansia, y luego refirió lamentables escenas y acontecimientos
que omitiremos por no ser de indudable interés para esta historia. Con
los ojos cerrados, apoyada la calenturienta sien contra el tabique,
entreabierta la boca, la mano izquierda en el suelo para apoyarse y la
derecha sobre el corazón, iba contando D. Fernando sus execrables
ardides, y soltaba las palabras una a una, cual si su arrepentida
conciencia se recrease en las torpezas que echaba afuera, para quedarse
pura y limpia. Cuando concluyó aquel capítulo bochornoso, oyose la débil
voz de Respaldiza que decía:

---Horroroso, infame, execrable es todo eso; pero el arrepentimiento es
sincero, y si grandes son las culpas de los hombres, mucho mayor es la
misericordia de Dios.

---Nació un niño---dijo D. Fernando, cuya alma se iba sublimando a
medida que adelantaba la confesión,---y aquí vienen nuevas infamias
mías, pues sabiendo que la madre y el hijo estaban en la miseria, no me
cuidé de socorrerlos. Un día pasé por Pipaón y enseñáronme al muchacho
que estaba jugando en las eras. Tenía los zapatos rotos y todo su
vestido hecho pedazos. Causome su vista cierta aflicción pasajera; pero
nada más: salí de Pipaón aquella misma tarde, y no me volví a acordar de
ellos. Por último, después de más de veinte años de olvido, he aquí lo
que sucede\ldots{} Salgo en busca de fabulosas hazañas, y a los pocos
pasos mis ilusiones se disipan como el humo\ldots{} ¡la mano de
Dios!\ldots{} Me traen aquí prisionero, y sin más lances me destinan a
morir y me encierran en este calabozo\ldots{} ¡la mano de Dios!\ldots{}
Luego se presenta un joven, le hago algunas preguntas, me dice su nombre
que es el de Salvador Monsalud, y en él reconozco a mi hijo\ldots{} ¡por
tercera vez la mano de Dios!\ldots{}

---¡Salvador Monsalud!---exclamó el cura alzando las manos.---¡Ese
perdido, ese afrancesado, ese traidorcillo borracho!\ldots{}

---El mismo, el mismo---dijo Garrote:---es un monstruo, es como el
crimen que le engendró, y Dios me lo ha puesto delante para hacerme
conocer la horrible magnitud de mis culpas, como un ejemplar vivo del
pecado que engendró el pecado.

---Conozco a la pobre doña Fermina, y ahora me explico algunas frases
oscuras que sorprendí algunas veces\ldots{} ya\ldots{} Es una excelente
mujer; pero Salvador es un muchacho arrebatado y sin discreción, ni
prudencia, ni honor, ni respeto a los mayores, sin amor a la patria, ni
religiosidad, ni sentimiento alguno que le recomiende. ¡Bendito sea
Dios, y que cosas hace! ¡Descender, salir de un caballero tan cumplido
como Vd., de un noble señor, algo libertino, sí, pero ilustre y
generoso, esa bestiezuela desleal, ese muchacho sin pudor ni
honor!\ldots{} ¡Bien dice Vd. que ha sido para castigo!\ldots{} ¿Está
Vd. seguro de\ldots?

---Hijo mío es: mi vida abominable no podía dar otro fruto. Es hermoso
de cuerpo; pero su alma es horrible. Si por favor especial del Cielo yo
viviera, la idea de haber dado el ser a criatura tan execrable, sería
para mí causa de constante horror.

---¡Oh, sí\ldots{} le conozco!\ldots{} Diré a Vd. amigo mío. Antes de
marchar a Madrid, Salvadorcillo no era mal muchacho, aunque muy
casquivano y distraído; pero después que se juntó con su tío y renegó,
hase vuelto el más despreciable muñeco que puede verse.

---La vergüenza que me causa el ser padre de un renegado
envilecido---dijo D. Fernando,---de un joven, cuyas absurdas ideas son
tales que parece que habla Satanás por su boca, es uno de los mayores
tormentos de esta última noche de mi vida\ldots{} Varias veces tuve las
palabras en la lengua para revelarle los lazos que a mí le unían; pero
enmudecí, porque todo lo que de noble y honrado existe en mi alma se
sublevaba contra el fatal parentesco, y aquí, Sr.~D. Aparicio de mi
alma, entra el grave punto de conciencia que quería consultar con Vd.
después de mi confesión.

---Sepámoslo\ldots{} pero se me figura que aumenta la algazara de esos
borrachos. Parece que se acercan a las puertas de este edificio, y
aúllan junto a ellas como una manada de lobos carniceros.

---La cuestión es ésta---dijo Garrote sin hacer caso del terror de su
amigo.---Dadas las deplorables circunstancias del carácter de Salvador,
sus infames ideas, su irreligiosidad, su traición, su envilecimiento,
¿debo revelarle que es mi hijo?

Calló Respaldiza largo rato, y al fin, repetida la pregunta por D.
Fernando, contestó:

---Según y conforme\ldots{} Perverso es el niño, e indigno por todos
conceptos de tener por padre a un caballero ilustre y tan patriota como
el Sr.~D. Fernando, en quien algunas faltas, hijas de la flaca condición
humana, no disminuyen sus altas prendas: despreciable es el muchacho,
digo; pero por malo que le supongamos, y aunque su herejía y
envilecimiento hayan secado en él el manantial de todos los sentimientos
generosos, es imposible que al ver a su padre en esta mazmorra,
acompañado de un infeliz amigo, no imagine alguna bellaquería o
travesura para ponerlos a ambos en libertad.

Garrote dio un suspiro, cambiando de postura, por serle insoportable la
que desde el principio del diálogo tenía.

---Yo pregunto con mi conciencia y Vd. contesta con su egoísmo\ldots{}
Monsalud no puede salvarnos\ldots{} además, yo no quiero salvarme, ¡no,
mil veces! yo deseo la muerte.

---¿No puede salvarnos?---preguntó el cura con desconsuelo.

---No, porque sus compañeros no se lo consentirían, y además ha dejado
hace un rato de ser nuestro carcelero, y en este momento, quizás esté
con su regimiento camino de Vitoria.

---¡Oh qué desgraciada suerte!\ldots{} ¡Me parece que esos condenados
nos quieren asesinar!\ldots{} ¿Oye Vd. sus infames carcajadas?

---Las oigo, sí, pero no las escucho\ldots{} El parecer de Vd. es lo que
me preocupa y lo aguardo con impaciencia.

---Por todos los santos, si no ha de ver más a Salvador, ¿para qué ha de
quebrarse los cascos por saber lo que más conviene decirle?

---Únicamente pido a Vd. consejo---dijo Navarro con impaciencia,---sobre
mi conducta pasada. Es decir, ¿hice bien o hice mal en callar el secreto
dejando a ese desgraciado en la orfandad lastimosa que a mi juicio
merece?

---Bien, bien, admirablemente hecho---repuso el clérigo con
cansancio.---El infame mozuelo que se ha vendido a nuestros enemigos,
que abandonó a su madre, que se burló descaradamente de mí, amenazándome
con ahorcarme, no tiene derecho a ser hijo de alguien, no, ni menos a
enfatuarse con descender del nobilísimo tronco de los Navarros.

---Pero revelarle todo habría sido grande humillación, habría sido
ponerme al nivel de su bajeza, de su herejía, de su villanía, y por
tanto habría sido también expiación de mis culpas, y nuevo purgatorio
añadido al que merezco y necesito.

---No tanto, no tanto---afirmó el cura.---Bastante ha padecido Vd. en
descargo de sus pecados. Revelar a Salvador la nobleza de la sangre que
por sus venas corre, sería en cierto modo santificar sus errores, y
conviene que siga abandonado a su triste destino. Allá se las entenderá
con Dios. El deber de Vd. consiste en perdonarle y pedir a Dios que
ilumine al perverso mancebo.

---Pecador fuí, pecador soy---dijo D. Fernando elevando al cielo los
ojos y cruzando las manos,---pero he conservado los sentimientos
fundamentales, el amor de Dios y el honor\ldots{} Aborrezco todo lo que
Dios aborrece, y amo todo lo que Él ama\ldots{} ¡Oh señor mío
Jesucristo, tú que me ves en esta última hora renegado por el
arrepentimiento y la penitencia, no quieres, no puedes querer que ese
miserable lleve mi nombre; tú no puedes querer que en su detestable vida
asocie su infamia a mi apellido, y ya que no me deshonró en vida con su
traición, me deshonre muerto! ¡La traición! Sólo al pronunciar esta
palabra, tiemblan mis carnes, y mi alma entrevé un infierno de
vergüenza, más espantoso que el de las llamas que abrasan el cuerpo. ¡La
traición! ¡Pasarse al enemigo, ser bandido como él, ateo como él, ladrón
como él, borracho como él! ¡Ah! Todos los crímenes, incluso los que yo
he cometido, me parecen faltas veniales comparadas con esta. Quédese,
pues, ese malaventurado hijo mío en la oscuridad de su nacimiento, que
será perpetua y profunda, como las tinieblas que envuelven su alma. Él
ha querido ser espúreo, espúreo será. Si la naturaleza nos hizo proceder
el uno del otro, entre un renegado por convicción y un caballero
español, entre un insensato ateo y un cristiano piadoso, entre un
jacobino de esta nueva raza execrable, condenada por Dios, y un hombre
recto, vasallo humilde de su Rey, no debe, no puede haber parentesco.

Dijo esto D. Fernando Garrote en alta voz, al modo de oración, y tan
creído estaba de que Dios, a quien tal discurso dirigía, aprobaba sus
sentimientos y su rigurosa intolerancia, que se quedó muy tranquilo,
meditando sobre las profundidades del ancho abismo abierto entre él y su
abandonado hijo.

---¿No les oye Vd.?---gritó de pronto Respaldiza, golpeando el
tabique.---Han vuelto a acercarse a la puerta de este cuarto y gritan y
juran. ¡Parece que se alejan! ¿Oye Vd., señor D. Fernando?

---Y si por favor especial de Dios---repuso Garrote, indiferente al
pánico de su compañero de desgracia, y mortificado por punzantes
dudas,---ese infeliz muchacho al verse honrado por ni nombre, se
enmendara de sus extravíos\ldots{}

---¡Enmendarse!---exclamó el cura.---Haríalo hipócritamente por
engañarle a Vd. si vivía\ldots{}

---Es verdad, es verdad, no puede ser---añadió D. Fernando.---Los que
nos han puesto el infame mote de serviles, los que insultan a los
valientes guerrilleros, llamándoles ladrones de caminos y asesinos, los
que en sus inmundas gacetas hacen befa de las cosas santas y de los
ministros de Dios, y parodian a los franceses, imitando su lenguaje, sus
costumbres, sus ideas, esos no pueden ser nuestros hijos, ni nuestros
hermanos, ni nuestros primos, ni nada que con nosotros se roce y enlace,
no pueden de ningún modo nacer de nosotros\ldots{} Esa gente no es
gente, esos españoles no son españoles. Entre ellos y nosotros, lucha
eterna.

---Para poner motes se pintan solos---dijo el cura, dejando caer una
gota de humor festivo en la amarga copa del aflictivo diálogo que uno y
otro bebían.---A nosotros nos llaman lechuzos, y a la Santa Inquisición
la llaman Chicharronismo. No puede darse desvergüenza igual. Por eso es
cosa corriente en el país, que a los guerrilleros de estas montañas les
queda mucho que hacer, después de acabar con los vándalos de fuera.

No lo oyó D. Fernando, porque se había arrastrado a gatas hasta el
centro de la pieza y allí puesto de hinojos, con los brazos alzados y la
mirada fija en el techo, entabló nuevo coloquio con la Divinidad, en
estos términos:

---Señor que me has criado, que me has conducido a este fatal término,
mi castigo ha sido grande, pero merecido\ldots{} ¡Oh! si volviera a
nacer, no saldría jamás del camino de la justicia y del deber\ldots{} Me
has puesto delante el monstruo engendrado por mis pecados, me lo has
puesto delante para que vea qué horribles frutos deja en el mundo la
depravación. Para tormento y horrorosa penitencia mía, el dulce regocijo
que la naturaleza debía infundirme en presencia de este joven, se ha
trocado en vergüenza, en aborrecimiento, en horror. ¿No es bastante
pena, Dios mío?\ldots{} Cumplo con mis deberes de cristiano resignándome
a morir, y sufriendo el bochorno que mi parentesco con tal monstruo me
produce; cumplo con mis deberes de caballero y de español, repudiando a
ese hijo precito y apartándole de mí y de mi memoria para siempre. ¿Es
de tu agrado esta conducta, Dios mío? Mi conciencia está tranquila, y
muero en ti, fiando en que mis pecados serán perdonados, y mi conducta
como cristiano, como caballero y como español aprobada en tu supremo
tribunal.

¿Qué respondió Dios a esto? Pronto lo sabremos.

D. Fernando se humilló en el suelo y dijo para sí:

---¡Virgen santa! ¿por qué me empeño en estar tranquilo y no lo estoy?

Respaldiza le llamó, diciéndole con voz angustiosa:

---Sr.~D. Fernando de mi alma, ¿no les oye usted? Parece que quieren
echar abajo la puerta de este cuarto. Chillan, chillan y
vociferan\ldots{} Sin duda quieren asesinarme; Sr.~D. Fernando, por amor
de Dios, ampáreme Vd.

En efecto, oíase violento rumor de golpes y porrazos. D. Fernando, que
hasta entonces no había tenido miedo a la muerte, sintió escalofríos en
todo su cuerpo, y el corazón le palpitó con vivísima inquietud.

---No, no estoy tranquilo---dijo para sí.---¡Si permitirá Dios que tenga
miedo en esta hora tremenda!\ldots{} Conciencia mía, ¿estás tranquila?

---Esos salvajes quieren penetrar aquí para ensañarse en mi cuerpo
miserable ---gritó entre sollozos el cura.---¡Señor mío Jesucristo,
piedad! ¡Piedad, santa Virgen de la Asunción, señora y patrona mía!

---Esto es horroroso---exclamó D. Fernando corriendo de un lado para
otro en la oscura pieza.---Que nos fusilen\ldots{} pero que no nos
arrastren, ni nos destrocen, ni nos escupan, ni nos insulten\ldots{}
¡Piedad, misericordia!

Los gritos de la salvaje turba que graznaba en la puerta del calabozo,
donde viviendo aún moría de terror el desgraciado D. Aparicio
Respaldiza, aumentaban de rato en rato, y al fin era tanto el ruido que
D. Fernando no pudo oír los lamentos de su infeliz amigo. Oyó sí que la
puerta se rompía; conoció que multitud de soldados franceses y algunos
españoles entraban en tropel, rugiendo como bestias coléricas;
comprendió que se abalanzaban sobre el pobre sacerdote y oyó estas
palabras en claro y soez castellano:

---Cortarle las orejas.

Después llegaron a sus oídos agudísimos ayes y clamores de la infeliz
víctima; sintió que la llevaban fuera atropelladamente y la fúnebre y
horrenda procesión se, presentó a su fantasía con formas tan espantosas,
que tuvo miedo, un miedo indescriptible, inmenso, y cayó de rodillas,
clamando:

---Señor mío Jesucristo, ¿todavía más?

Parecía que una voz contestaba desde lo alto:

---Sí, más todavía.

\hypertarget{xx}{%
\chapter{XX}\label{xx}}

Luego que Monsalud saliera de la prisión, se serenó un tanto; mas por
algún tiempo estuvieron aún sus entendederas en lastimoso eclipse. No
era de aquellos a quienes la bebida impulsa a desaforados disparates de
palabra y obra, sino que por el contrario en aquella su embriaguez
primera, después de algunos minutos de estúpida animación, sintiose
amodorrado y con tristeza tan congojosa, que el cielo parecía habérsele
puesto sobre los hombros. Sus amigos españoles renegados y franceses
bebían y jugaban a los naipes, reunidos en alegres grupos dentro de la
sala que servía de cuerpo de guardia y también en el patio. Los del
convoy, paisanos y militares, habían ido allí atraídos por el olor de
los riojanos pellejos; pero como se acercara la hora de partir y el
descanso de bestias y hombres había sido grande, se disponían a seguir
adelante.

Salvador advirtió que algunos jurados y cazadores franceses,
soliviantados por el vino, hacían tan infernal ruido como si todo el
ejército de José estuviese bailando dentro de una sola pieza. Mareado y
aturdido, anhelando silencio y reposo, Monsalud huyó de su compañía y
fue al patio, donde algunos paisanos graves y sargentos con ínfulas de
coroneles, dirigiendo en pomposas espirales hacia el limpio cielo, cual
si quisieran empañarlo, el humo de sus pipas, hacían cálculos sobre la
campaña emprendida y los acontecimientos que se aguardaban para el día
siguiente.

---Salvador---dijo un francés, asiendo a nuestro amigo por un botón de
su uniforme,---¿has oído algo?

---¿De qué?---preguntó Monsalud dejándose caer sobre un banco y cerrando
los ojos.

---De la campaña. Toda la división está en movimiento. ¿No oyes las
cajas al otro lado del Zadorra?

---Sí, ya las oigo.

---Buena hora has escogido para dormir---añadió el francés intentando
poner en pie al aturdido joven.---Arriba, muchacho, que nos vamos.

---¿A dónde?

---A Vitoria con el convoy grande.

---¡Con el convoy grande!---repitió Salvador alargando los brazos cual
si quisiera alcanzar el cielo con ellos.---¿Pues no ha salido ya?

---¡Bestia! El vino te ha puesto el entendimiento del revés. Salieron
los carros que llevó consigo el general Maucune.

---¿Y nosotros salimos ya o estamos aún aquí?---preguntó
Salvador.---Juro a Vd. Sr.~Jean-Jean, que no lo sé.

---Te lo explicaré a puñetazos---repuso el formidable dragón.

Zumbido lejano atrajo entonces la atención de todos.

---¡Un tiro de cañón!---exclamaron unos.

---¿Hacia qué parte?

---Juro que es hacia Subijana.

---Hacia la Puebla.

Monsalud participando de la general curiosidad, trató de sacudir el
pesado sopor que embargaba sus sentidos.

¡Una batalla!\ldots{} ¿pues qué hora es?

---Quizás las avanzadas estén reconociendo alguna posición\ldots{}
Señores, mañana 22 será un día de sangre, lo dice Plobertin, que ha
visto el sol de muchos días de batalla.

---Es desgracia que nosotros no podamos asistir a la gran acción que se
prepara, Sr.~Jean-Jean---dijo Salvador,---y que a hombres de tal temple
les destinen a custodiar cofres y estuches.

---¡Oh, joven Epaminondas!---repuso con socarronería el astuto
dragón.---No envidies a los que se han de cubrir de gloria en el día de
mañana. Soldado viejo soy, y te juro que mientras más cruces gano para
mí y más tierras conquisto para nuestro Emperador, más anhelo la paz.
Marchemos tras los cofres y por el camino. Seamos galantes con las
señoras que van en el convoy, recomendándonos a ellas como soldados de
Friedland y de Essling, y glorifiquemos a la Francia y bendigamos a
Napoleón\ldots{} por no habernos llevado a la campaña de Rusia.

Reinaba cierta inquietud entre la tropa que no había perdido el sentido
con la embriaguez. Por otra parte, varios paisanos y bagajeros y unos
cuantos soldados franceses de la peor especie, se habían cogido del
brazo y recorrían parte del camino en burlesca procesión, gritando y
cantando: algunos de ellos, que apenas podían tenerse en pie, eran
llevados en vilo por sus compañeros. Luego que berrearon a sus anchas,
insultando a las infelices señoras que aguardaban junto a sus coches la
partida del convoy, tomaron al patio, y acercándose a la puerta que daba
entrada a las habitaciones de los presos, la golpearon de tal modo con
patadas y puñetazos, que a ser débil se quebrantara al instante hecha
menudas piezas. La turba embriagada quería que le entregaran a los dos
infelices prisioneros para anticipar el castigo impuesto por la
superioridad militar.

---¿Pero aquí no manda nadie?---dijo el francés que respondía al nombre
de Plobertin.---Esta canalla hará una atrocidad si la dejan.

---¡Que nos entreguen al cura, al cura!---gritaba la turba furiosa.---Al
cura y al sacristán.

Y golpeaba la puerta, que a fuerza de porrazos comenzaba a resentirse.

---Aquí viene el capitán---dijo Jean-Jean.---Mandará dar veinte palos a
los borrachos, y hará cumplir la sentencia.

Un capitán francés reprendió a los revoltosos su estúpida crueldad,
amenazándoles con fuerte castigo; pero aquel, como los demás oficiales
alojados allí, estaba en gran zozobra por causa más grave que las
travesuras de algunos soldados ebrios, y regresó al lado de sus
compañeros, dejando tras sí el tumulto el tumulto que de nuevo estallara
con más fuerza.

---Vámonos por no ver esto---dijo Plobertin.---Parece que algunos carros
se han puesto ya en marcha\ldots{}

---Nosotros formamos a retaguardia---dijo Monsalud,---hay tiempo
todavía.

---La gentuza vuelve a las andadas---indicó Jean-Jean.---La puerta no
resistirá mucho tiempo más: no es esa la Zaragoza de las puertas.

---¡Que las paguen todas juntas!---afirmó otro individuo del respetable
cuerpo de dragones.---Ese cura y ese sacristán son guerrilleros, que es
como decir salteadores de caminos. Pues qué ¿les hemos de tratar con
mimo, después que ellos han asesinado a centenares de hombres
pertenecientes, como quien no dice nada, a la nación francesa?

---¡A la nación francesa!---repitió el zapador Plobertin encendiendo su
pipa.---La nación francesa pide venganza\ldots{} La verdad es que el
cura y el sacristán no merecen mis simpatías.

---Pues yo---dijo Monsalud con resolución,---si encontrase quien se
decidiera, arremetería contra esa chusma y les haría entrar en razón.

---Joven Temístocles---exclamó Jean-Jean,---menos fuego. ¿Pueden tus
paisanos colgar de los árboles racimos de franceses, descuartizarlos,
meterlos en los pozos y asarlos en los hornos, y nosotros no podemos ni
siquiera desorejar a uno de tus desalmados curas y monagos?

---El honor de la Francia---dijo Plobertin,---pide que se les fusile al
momento.

---Pero sin martirizarlos vergonzosamente---añadió con viveza
Monsalud.---Si el Rey lo sabe, castigará a los que le están deshonrando
con esta algarada salvaje.

---En esto de mortificar a los guerrilleros y curas con
pistolas---afirmó Jean-Jean---yo digo como nuestro glorioso rey Luis XV
de la antigua dinastía: Laissez faire, laissez passer. Con que a
caballo, Sr.~Monsalud, que marcha el convoy.

La confusión y el alboroto iban en aumento, y no había autoridad que
mandase, ni voz alguna que contuviese a los desalmados. Fueron y
vinieron algunos oficiales, pero sin desplegar la energía que el caso
requería, porque acostumbrados a considerar a los guerrilleros como
bestias malignas, toleraban los desmanes de la embriagada soldadesca, o
al menos no se cuidaban de atajar una brutalidad que creían justificada
por la salvaje fiereza de los partidarios.

La puerta cedió al fin, y los gritadores se precipitaron por ella dentro
del edificio. Encontrábase primero frente a la puerta principal otra más
pequeña que era la que daba ingreso a la celda del cura, y que por ser
endeble, fue brevemente echada al suelo de una patada. Pocos momentos
después, el infeliz D. Aparicio Respaldiza salía empujado y arrastrado
por la soldadesca, mutilado el rostro, cubierto de sangre, abofeteado,
injuriado, escupido. Medio muerto de espanto, encomendaba el desgraciado
su alma al Señor, y en aquel momento angustioso, aquel hombre no exento
de faltas, aunque tampoco perverso, mal sacerdote, sin duda, pero antes
por error y falsas ideas que por maldad, si tuvo la flaqueza de pedir
misericordia a sus viles verdugos, luego que se vio arrastrado
irremisiblemente al suplicio sin vislumbrar remedio, les perdonó a todos
y supo morir como cristiano.

Llevole la turba a un campo cercano donde algunos robustos árboles
convidaban a aquellos cafres a colgar del alto ramaje el cuerpo del
infeliz enemigo vencido e indefenso, y mientras se consumaba el
sacrificio, se regocijaban con la idea de repetir la función en la
persona de aquel a quien llamaban el sacristán, a pesar de que su
aspecto no indicaba tan humilde oficio.

Monsalud, que desde el patio presenciaba la feroz escena, baldón del
humano linaje, mas no por eso rara en aquella guerra que tanto tenía de
heroica como de salvaje, sentía en su alma violentísimo coraje y
vergüenza. Al ver que llevaban al suplicio, ya mutilado y moribundo, al
infeliz Respaldiza, acordose del otro preso; un vago sentimiento agitó
su pecho, sintió algo semejante a dulce recuerdo o a esos misteriosos
rumores del corazón, que a veces gimen en los oídos de nuestra alma, sin
que entendamos claramente lo que quieren decirnos. Inquieto y dominado
por profunda aflicción, que no acertaba a explicarse, dirigiose a la
rota puerta del edificio. Allí estaba el sargento poco antes encargado
de la custodia de los prisioneros, y en compañía de dos o tres bárbaros
como él contemplaba estúpidamente, con las manos juntas atrás y su pipa
en la boca, el fúnebre via crucis del cura hacia el monte cercano.

---¡Bestia!---le dijo enérgicamente Monsalud.---¿De ese modo guardas a
los prisioneros?

El sargento soltó la carcajada de la insensibilidad aumentada por el
vino, y alzando los hombros, repuso:

---¿Y qué?\ldots{} ¿No les habían de matar de madrugada?\ldots{} ¿Dónde
están los oficiales? Si ellos no cumplen con su deber, ¿qué puedo hacer
yo?

---¡Miserable!---gritó el joven con furia.---Si esos verdugos se
hubieran empeñado en romper esa puerta antes de las doce, hora en que
salí de guardia, me habrían cortado a mí las orejas antes de tocar el
pelo de la ropa a los prisioneros\ldots{} Déjame entrar; queda ahí
dentro un infeliz, que no morirá como mueren los cerdos.

El sargento y los suyos hicieron como que querían defender la puerta.

---¡Atrás!---gritó Monsalud.---Dame la llave de la prisión del
sacristán.

Briosamente arrebató la llave de manos del carcelero.

---Monsalud---dijo el sargento fingiendo la entereza de un hombre de
bien---¿quieres salvar a ese hombre? Está más loco que D. Quijote, y a
todos los que entran a verle les llama hijos para que le pongan en
libertad.

---¡Estúpido farsante!---repuso el joven.---¿Te atreves a darme
lecciones de disciplina, de honor y de obediencia, tú que has faltado a
todas las leyes de la Ordenanza y de la humanidad?

---Lo digo---añadió el carcelero echándosela de bravo,---porque para
sacar de aquí al sacristán, pasarás sobre mi cadáver.

---¡Y sobre el mío!---repitieron los otros, algunos de los cuales no se
podían tener de borrachos.

---¡Atrás, a un lado!---vociferó Monsalud abriéndose paso y tomando la
linterna que estaba en el suelo.---No puedo salvar a ese hombre, porque
el general le ha condenado a morir; pero mientras yo aliente, canallas
cobardes, un caballero honrado y decente no morirá, ya lo he dicho, como
mueren los cerdos. Los infames vuelven; no hay tiempo que perder.
Adentro.

Abrió con mano firme la puerta del aposento en que gemía D. Fernando
Garrote. El infeliz anciano, al sentir que sacaban arrastrado a su
compañero, después de mutilarle, había sentido como antes dijimos, un
terror violentísimo que dio al traste con toda su entereza y varonil
grandeza de ánimo. Extraviose su razón, dio voces, y cuando entró el
sargento le habló como si fuera Salvador. Levantose del suelo en que
yacía y como un loco corrió de un muro a otro buscando salida, y se
aporreó las manos contra ellos, cual si a puñetazos pudiese horadarlos.
La unción religiosa huyó de su mente; huyeron la resignación, la
paciencia, la cristiana humildad, dejando tan sólo el impetuoso
instinto. Gritaba con desesperación:

---Jesús divino; ¡sólo tú sabes padecer, sólo tú sabes morir! Soy hombre
y acepto la muerte; pero no el tormento, no la vergüenza, no el
martirio, no las manos ni la saliva de la soez plebe en mi rostro, ni la
ignominiosa cuerda en mi cuello, ni el vil filo de sus navajas en mi
piel\ldots{} ¡Piedad, misericordia, Dios mío! ¡No tengo valor! Soy una
mujer, un pobre niño\ldots{}

Con febril ansiedad, y aunque sabía que ninguna arma llevaba sobre sí,
registró todos sus bolsillos y ropas, buscando un corta-plumas, una
aguja, un alfiler con que darse la muerte.

---¡Nada, nada!---exclamó con desesperación.---Dios poderoso, ¿tan malo,
tan perverso he sido?\ldots{}

En aquel instante una claridad rojiza deslumbró sus ojos, y en medio de
ella, como el ángel de una aparición divina, vio D. Fernando Garrote a
Salvador Monsalud. Sorprendido por aquella imagen que en el momento de
la más abrumadora angustia se le presentaba, don Fernando cayó de
rodillas.

---¡Eres tú, Salvador, hijo mío querido, eres tú!---exclamó desahogando
con efusión su alma.---Vienes a salvarme\ldots{} sí, sí. Tengo miedo,
Dios me abandona y no me permite morir con la dulce y tranquila muerte
del buen cristiano.

---He tenido lástima---dijo Salvador con voz balbuciente,---y he
venido\ldots{}

---¡A salvarme!\ldots{} ¡Oh, justicia! ¡oh, lección divina!---gritó
vertiendo amargas lágrimas don Fernando Garrote.---¡Has sido tú más
generoso que yo! Sí, más generoso, querido hijo mío\ldots{} Bien me
decía el corazón, que mi conducta era egoísta y mezquina. Salvador, por
orgullo, por preocupaciones más fuertes para mí que la razón, por
egoísmo, te oculté un secreto, cuya confesión debía ser para mí una
deuda sagrada.

Salvador no comprendía nada, y pensando tan sólo en el objeto de su
visita, dijo:

---Pronto llegarán: aún puede Vd\ldots{}

---He sido un miserable, he sido un egoísta, las ideas adquiridas en las
disputas de los hombres, las he sobrepuesto a los sentimientos más
dulces de mi corazón, a mi conciencia y a mis deberes. Salvador, este
miserable que ves aquí a tus pies, humillado y envilecido, es el que te
ha dado la vida, es tu propio padre, que por su mala suerte y su
indisculpable apatía, no ha tenido hasta ahora la dicha de conocerte.

El semblante de Salvador, atónito primero, expresó después la más
desconsoladora incredulidad. Una sonrisa, impropia ciertamente del lugar
y de la ocasión, vagó por sus labios; pero recobrando al punto su
seriedad, y movido a gran compasión por el triste estado mental que en
el anciano suponía, le dijo con frialdad:

---Sr.~Garrote, yo no tengo padre.

Estas palabras atravesaron como una espada de hielo el corazón del
desgraciado Navarro.

---En nombre de tu santa y buena madre, en nombre de Dios---dijo,---en
nombre de Dios, no me desmientas\ldots{} He sido un infame egoísta, he
sido un necio lleno de orgullo hasta en esta ocasión tristísima, pues
hace un momento me horrorizaba la idea de llamar hijo a un traidor
renegado. Dios me ha castigado por esto; pero siempre misericordioso
conmigo, te me ha puesto delante en mi última hora, para que mi
confesión sea completa. ¡Bendito sea Dios!

---Desgraciado loco---dijo Monsalud, contemplando al reo con impasible
calma y profunda lástima, tan extraño a los sentimientos que este
expresaba, como si fueran de otro mundo.---Comprendo que en situación
tan aflictiva, trate de seducir a sus carceleros, llamándoles sus hijos.
Todo es inútil conmigo, porque no he venido aquí a librarle a Vd. de la
muerte.

---¡No me cree!---rugió D. Fernando arrojándose en el suelo.---Dios mío,
Dios justiciero que así prolongas mi castigo, ¿más todavía?

Una voz del cielo pareció responder:

---Sí, todavía más.

---Viendo que era inevitable para Vd. un fin tan horrible como el del
pobre Respaldiza---dijo Salvador llevando la mano al cinto donde tenía
las pistolas,---y suponiéndole hombre de valor, he creído que era
caritativo proporcionarle un medio de evitar la ignominia de martirio
tan bárbaro.

D. Fernando se levantó de súbito. Parecía un esqueleto con vida y con
toda la vida en los ojos. En aquel instante oyéronse los desaforados
gritos de la turba que volvía. Estremeciose el anciano; dominado
nuevamente por un terror congojoso, aparentó luego serenidad heroica, y
contemplando al mancebo con altanería, exclamó:

---Un hombre de honor, un caballero como yo, no morirá a manos de viles
sicarios; un hombre como yo, no será sacrificado salvajemente por tus
bárbaros amigos. He cumplido contigo y con mi conciencia. No contaba con
mi desgraciado destino ni con tu incredulidad\ldots{} Que Dios me
perdone lo que voy a hacer. Salvador, dame un arma cualquiera, y adiós.

Con la seguridad de quien ve realizado su pensamiento, Monsalud entregó
una pistola a D. Fernando Garrote, diciéndole:

---Eso mismo pensaba yo\ldots{} Un hombre de honor, un caballero
decente\ldots{} Que Dios le ampare a Vd.

D. Fernando irguió con altivez la majestuosa frente, miró a su hijo con
calma desdeñosa, le miró mucho durante un rato relativamente largo, y
luego con voz trémula y solemne en la cual había cierto sensible acento
de pesadumbre mezclado de sarcasmo, habló de esta manera:

---Salvador, gracias, gracias\ldots{} Que Dios te ampare y te perdone.
Adiós.

---Adiós---dijo Monsalud desde la puerta saliendo rápidamente.

Cuando la brutal soldadesca entró atropelladamente en donde estaba el
bravo guerrero, halló su cadáver caliente y tembloroso sobre el suelo,
la sien partida y destrozado el cráneo. Su mano palpitante asía con
rabioso vigor el arma.

\hypertarget{xxi}{%
\chapter{XXI}\label{xxi}}

¡Cuántos habrá que al leer estas escenas que acabo de referir, las
hallarán excesivamente trágicas y tal vez exagerada la terrible pugna
que en ella aparece entre los lazos de la naturaleza y las especiales
condiciones en que los sucesos históricos y las ideas políticas ponen a
los hombres! Yo aseguro a los que tal piensen, que cuanto he contado es
ciertísimo y que en el lamentable fin de D. Fernando Garrote no he
quitado ni puesto cosa alguna que se aparte de la rigurosa verdad de los
acontecimientos. Vivió el citado Garrote en los mismos años que le
presento, y fueron su carácter y sus costumbres y sus ideas tales como
he tenido el honor de pintarlas, salvo la diferencia que entre el
artificio de la narración y la verdad misma existe y existirá siempre
mientras haya letras en el mundo. Cierta fue también su malograda
expedición con el cura Respaldiza, y evidente su desastroso cautiverio y
fin horrendo, aunque no le cupo peor suerte que a otros muchos, quier
españoles, quier franceses, víctimas entonces del furor de las
desenfrenadas pasiones.

En cuanto a las circunstancias verdaderamente terribles que acompañaron
al último aliento de aquel desgraciado varón, no son tales que deban
causar espanto a la gente de estos días, la cual viviendo como vive en
el fragor de guerra civil, ha presenciado en los tiempos presentes todos
los furores del odio humano entre seres de una misma sangre y de una
misma familia; ha visto rotos todos los vínculos en que principalmente
apoya su conjunto admirable la sociedad cristiana. ¡Oh! si en el santo
polvo a que se reducen la carne y los huesos de tantos hombres
arrastrados a la muerte por el fanatismo y las pasiones políticas,
quedase un resto de vida, ¡cuántas íntimas reconciliaciones, cuántos
tiernos reconocimientos, cuántos perdones no calentarían el seno helado
de la honda fosa, donde el insensato cuerpo nacional ha arrojado parte
de sus miembros, como si le estorbasen para vivir! Y si la eterna vida
disipa las nieblas que oscurecen aquí el pensar de los hombres, ¡cuántos
seres habrá que en la desolación de la impenitencia y en su solitario
vagar por la desconocida esfera, maldecirán la mano corporal con que
hirieron el uno al hijo, el otro al hermano! La actual guerra civil, por
sus cruentos horrores, por los terribles casos de lucha entre parientes
que ha ofrecido, y aun por el fanatismo de las mujeres, que en algunos
lugares han afilado sonriendo el puñal de los hombres, presenta cuadros,
cuyas encendidas y cercanas tintas palidecerán, tal vez, los que
reproduce los narradores de cosas de antaño. El primer lance de este
gran drama español, que todavía se está representando a tiros, es lo que
me ha tocado referir en este, que más que libro, es el prefacio de un
libro. Sí; al mismo tiempo que expiraba la gran lucha internacional,
daba sus primeros vagidos la guerra civil; del majestuoso seno
ensangrentado y destrozado de la una, salió la otra, cual si de él
naciera. Como Hércules, empezó a hacer atrocidades desde la cuna.

Púsose en marcha el largo convoy bastante después de media noche. Todo
el camino real, desde las últimas casas de Aríñez hasta Gomecha, estaba
ocupado. ¡Con cuánta ansiedad veían que España se iba quedando atrás,
las infortunadas familias que buscaban un refugio en Francia!

---Si podemos llegar a Vitoria---decía Jean-Jean que iba a caballo junto
a Monsalud en la retaguardia,---estamos en salvo. Allá se las entiendan
el Rey y el mariscal Jourdan con Wellington y Hill. ¡Gran batalla
tendremos hoy!\ldots{} Pero créeme: daría una de mis manos por no verla.

---Han dado orden de marchar más a prisa, señor Jean-Jean---dijo
Salvador.---La cosa apremia. Vd. da una mano por no ver esta batalla y
yo daría las dos por verla.

---¡Oh, joven Bayardo, caballero sin miedo y sin mancilla! ¿Sabes lo que
es una batalla? Un engaño, chico, una farsa. Los generales embaucan a
los pobres soldados, les hablan de la gloria, les arrastran a la
barbarie, les hacen morir y luego la gloria es para ellos. Pónense a
mirar la batalla desde una altura lejana a donde no lleguen las balas, y
echando el anteojo a un lado y otro, hacen creer a los tontos que están
observando distancias y calculando movimientos. Así como los nigromantes
hablan de estrellas, ciclos, conjuros para engañar a los necios, los
generales hablan de paralelas, ángulos, cuñas, etc\ldots{} y hacen
garabatos en un papel\ldots{} ¡Oh, yo he medido la Europa con el compás
de mis piernas; yo he escupido mi saliva en el Austria y en la Rusia, y
sé lo que es una batalla! Después que los unos han destrozado a los
otros a fuerza de brazo, porque aquí todo se hace a fuerza de brazo, el
general recorre a caballo el campo de batalla, y con sonrisa hipócrita
da gracias a los soldados; manda que se asista a los heridos, y los
cirujanos empiezan a trabajar en la carne como los ebanistas en madera.
Enterramos a los muertos, damos una muleta a los cojos y una venda a los
ciegos: Nuestros nombres no se escriben en ningún monumento ni nadie los
sabe, ni los pronuncia más boca que la de nuestros compañeros. No así el
general que se pone un calvario en el pecho, y se echa a cuestas un
título como una casa, de tal modo que si hoy derrotásemos a los ingleses
y españoles en cualquiera de estos sitios que atrás dejamos, no faltaría
un general que se llamase mañana duque de Subijana de Álava, o Príncipe
del Zadorra. Luego viene la historia, con sus palabrotas retumbantes y
entre tanta farsa caen unos reyes para subir otros sin que el pueblo
sepa por qué, y los políticos hacen su agosto chupándose la sangre de la
nación, que es lo que a la postre resulta de todo esto.

Iba a contestarle Salvador, cuando una sonora y fresca voz de mujer
gritó:

---Sr.~Monsalud, Sr.~Monsalud, ¡gracias a Dios que se le ve a Vd.! ¡Qué
prisa tiene el caballerito para dar cuenta de los encargos que
recibe!\ldots{} ¡Oh, qué prisa, sí!

Monsalud, a pesar de la oscuridad, distinguió perfectamente un rostro
femenino que por la portezuela de un coche asomaba, acompañado de una
mano con quiroteca, cuyos dedos pajizos se movían saludando de una
manera apremiante y afectuosa.

---Perdone Vd. señora doña Pepita---dijo el militar acercando su caballo
al vehículo.---Hace dos días que no la veo a Vd. por ninguna parte. ¿Y
el señor oidor cómo sigue?

Un rostro acartonado y marchito, en cuya superficie brillaban con chispa
mortecina dos tristes y ya muy viejos ojuelos, apareció un momento en la
portezuela, y una voz fatigada resonó diciendo estas palabras, que
parecían una especie de limosna oral:

---Buenos días tenga el señor sargento Monsalud.

Y desapareció luego dentro del coche.

---¿Apostamos---dijo la dama sonriendo,---a que no me compró Vd. en la
Puebla los polvos a la marichala que le encargué, ni las pastillas de
malvavisco?

---Señora, ya sospechaba yo---repuso el joven,---que en la Puebla no
habría cosas tan finas.

---¡Ah, tunante!---exclamó ella, amenazando festivamente al joven con su
descomunal abanico cerrado, que esgrimía como si fuese una
espada.---Disculpas\ldots{} Y hablando de otra cosa, ¿cuándo llegaremos
a Francia?

---Pronto, señora. Si hay batalla al romper el día, como dicen, nosotros
habremos ganado de aquí a esa hora mucho terreno, y nadie nos estorbará
el paso.

El oidor dejose ver de nuevo. Era un varón de años, flaco e indolente,
enfermo tal vez, y parecía muy aburrido del largo viaje.

---¡Batalla al romper el día!---dijo frunciendo el ceño.---Me parece que
principia a despuntar la aurora. ¿Y hacia dónde es esa batalla?

---Hacia ninguna parte, hombre---repuso con desdén y superioridad doña
Pepita.---Tu gran miedo te hace ver batallas en las puntas de los dedos.
¡Qué aburrimiento! No se puede ir contigo a ninguna parte\ldots{}
Recuéstate en el coche y calla, o me enojaré.

---¡Todo sea por Dios!---murmuró el oidor sepultándose en el coche.

---No se descuide Vd. en avisarme todo lo que ocurra---dijo la dama
alzando la voz, cuando por uno de los movimientos tan propios de una
marcha, el coche se alejó bastante de los jinetes.

Monsalud la saludó con una sonrisa, mientras Jean-Jean le decía:

---Si esa señora doña Pepita tan garbosa, con su grueso lunar velludo en
la barba, sus buenas carnes, sus ojos negros, su cara un tanto
arrebolada y sus quirotecas amarillas, me hubiese mirado a mí desde la
portezuela, apuntándome con su abanico y haciéndome preguntas diversas
desde que salimos de Valladolid, a estas horas, joven guerrero, ya nos
trataríamos de tú, y todos mis compañeros envidiarían al sargento
Jean-Jean. Verdad que yo soy hombre muy circunspecto y no he querido
decirle una sola palabra, además de que no es de caballeros quitarle su
conquista a un camarada; que si llego a hablar con ella y echo mis
visuales y disparo los tiros de mi galantería, y trazo mis paralelas, y
lanzo los escuadrones, y enfilo las piezas, y pongo el sitio en regla,
Monsalud, en dos horas es mía la plaza; en dos horas hago yo lo que a ti
te costará dos meses\ldots{} ¿pero en qué piensas? ¿estás mirando las
estrellas que desaparecen?\ldots{} Salvador, Salvador, despierta, que
estoy hablando, está hablándote todo un Jean-Jean.

Profundamente abstraído y meditabundo, Monsalud había olvidado a doña
Pepita, al oidor y a Jean-Jean. Poco después de este ligero incidente,
la claridad del día empezó a derramarse por tierra y cielo, bañándolo
todo con las dulces y frescas tintas de la mañana. El sereno firmamento
parecía suspendido sobre la frente del mortal para presidir y proteger
su alegre vida, sublimada por el trabajo, por la virtud, por inocentes y
castos amores. El campo estaba impregnado de la grata y placentera
atmósfera que por el aliento penetra hasta nuestro corazón inundándolo
de felicidad, o si se puede decir, aromatizándolo, pues parece que
balsámicas esencias penetran hasta lo más hondo de nuestro ser,
sacudiendo los sentidos y despertando el alma con el estímulo de vagas
emociones. Las altas montañas y los verdes prados se aclaraban, disipada
la niebla que los cubría, mostrando su lozano verdor, compuesto de mil y
mil hojuelas húmedas, que tiritaban al roce del pasajero viento. Poco
después los rayos del sol se introducían por todas partes, en el seno de
las nubes, entre el follaje de los árboles, en los infinitos
huequecillos de los arbustos y las piedras, en la profunda masa
cristalina de las aguas del río. Todo tomó color, y con el color la
grandiosa existencia del día. ¡Ah! si queréis conservar la dulce paz en
vuestra alma cerrad los oídos\ldots{} Estrepitosos cañonazos resonaron a
lo lejos y el convoy entero, como si obedeciera una orden, se detuvo.

Por algún tiempo no se oyó en todo el espacio ocupado por tantos carros
y hombres, el más ligero rumor; pero no tardó en producirse de un
extremo a otro discordante algarabía.

---Dicen que no se puede pasar de Gomarra\ldots{} Los ingleses están
atacando a la Puebla\ldots{} También hay batalla por Subijana\ldots{} y
en Avechuco\ldots{} y en Crispiniana.

Estas frases, se repetían, pasando de boca en boca y dando ocasión a
multitud de preguntas que no eran nunca bien contestadas. Las respuestas
aumentaban la confusión.

---¡Patarata!---exclamaba un jurado de los más vehementes el cual había
aprendido pronto la fanfarronería francesa;---el general Clausel, que
está en la Puebla, les enseñará lo que pueden tres ingleses contra un
solo francés. ¿Y qué nos puede importar la Puebla si queda atrás?
Adelante.

Pero los carros y coches no obedecieron la enfática orden del bravo
dragón, permaneciendo tan quietos cual si los clavaran en el suelo. El
día había aclarado completamente, permitiendo ver la palidez y la
extrema ansiedad de todos los semblantes\ldots{} De pronto una voz
pavorosa recorrió de un extremo a otro la línea del convoy, repitiendo:

---No se puede pasar. Crispiniana ha sido atacada, y los ingleses y los
guerrilleros han aparecido por Gomarra\ldots{}

La configuración del camino por donde intentaba marchar el convoy era la
más a propósito para infundir miedo a los viajeros. Altos cerros a un
lado y otro formaban un estrecho callejón tortuoso, por cuyo fondo el
camino y el Zadorra culebreaban estorbándose a cada paso. Frecuentemente
pasaba el uno por encima del otro, cediéndole ora la derecha ora la
izquierda. Aunque en la noche antes se habían tomado todas las
precauciones para el paso del convoy ocupando las alturas, aquel
repetido cañoneo que se oía más arriba, ponía en gran inquietud a todos,
y recelaban que las fuerzas destacadas se hubieran visto en la necesidad
de acudir en socorro de los de Crispiniana o Gomecha\ldots{} Por fin,
después de una hora de ansiedad, moviose la larga procesión entre gritos
de alegría. Los mulos, los caballos, los bueyes y los hombres dieron
algunos pasos; después se volvieron a parar. Parecía una comitiva de
entierro cuando el carro fúnebre se atasca.

Pero transcurrido otro rato de ansiedades, de angustiosas preguntas y de
mal humoradas respuestas, el dragón de mil patas marchó de nuevo con
bastante prisa.

---¿Qué hay?\ldots{} Sr.~Monsalud, una palabra por amor de Dios---dijo
la oidora echando fuera del coche su ostentoso lunar, su franca sonrisa,
su rostro todo, no pequeño ni falto de gracias por cierto, su abanico y
sus quirotecas.---Cuénteme Vd. lo que ocurre.

---Cuéntenoslo Vd.---añadió el oidor asomándose también tras de su
consorte.

---No hay nada que temer---dijo deteniéndose el jinete, que regresaba de
la vanguardia del convoy.---Camino franco hasta Vitoria.

---Nos hemos detenido, señora---indicó Jean-Jean, metiéndose donde no le
llamaban,---porque la vanguardia ha estado reconociendo el camino.

---La batalla está empeñada por aquí, a mano izquierda---dijo Monsalud
extendiendo el brazo en la dirección indicada,---y se ha roto el fuego
por tres puntos distintos.

---Por tres puntos distintos, señora---añadió el intruso
Jean-Jean.---Quizás pasemos por sitios peligrosos. Si gusta la señora
oidora, la acompañaré a la portezuela para preservarla de cualquier
accidente.

---No, gracias, retírese Vd.---repuso la dama con
desdén.---Sr.~Monsalud, ¿se marcha Vd. tan pronto? ¿Perderán esa
batalla? ¿La perderemos? ¡Ay, no me diga Vd. que sí!\ldots{} Engáñeme
Vd. por favor.

---¡Qué se ha de perder!---vociferó el francés.

---Señor sargento---dijo el oidor,---no se separe Vd. de nosotros. Mi
mujer tiene un miedo espantoso.

---¡Oh, sí!---murmuró la dama.

---Si por desgracia nuestra nos viésemos en peligro\ldots{}

---No, no se separe Vd. de nosotros, señor Monsalud---dijo doña
Pepita.---Mi marido cobra alientos viéndole a Vd. tan cerca\ldots{}
podría ocurrir algún accidente funesto; que nos viésemos envueltos,
comprometidos\ldots{} ¡Cómo retumban los cañonazos en estas
montañas!\ldots{} Por Dios, Sr.~Monsalud, distráigame Vd., cuénteme
cosas agradables para que con la conversación entretengamos y engañemos
el miedo; hablemos de asuntos risueños, placenteros, tiernos y dulces,
de esos que regocijan el espíritu y matan el hastío. Hágame Vd. olvidar
que a dos pasos de nosotros se está dando una batalla\ldots{} quiero
estar alegre y reír\ldots{} quiero olvidar y engañarme.

Engáñeme Vd\ldots{} ¡Oh, sí! dígame Vd. que no tema,
tranquilíceme\ldots{} Pero no oigo lo que usted me dice Vd. ¡Oh! no tema
usted alzar la voz. Mi marido no oirá nada: es un poco sordo.

\hypertarget{xxii}{%
\chapter{XXII}\label{xxii}}

La batalla en que doña Pepita no quería pensar y en la cual nosotros no
fijaremos tampoco mucho la atención, fue del modo siguiente.

Ya sabemos la dirección y traza del camino real de Miranda a Vitoria,
que va a orillas del Zadorra, rozando al pasar los lindes del condado de
Treviño. Hállanse en este camino los lugares de la Puebla, Aríñez,
Crispiniana y Gomecha, y después de deslizarse entre altos riscos,
penetra holgadamente en el llano de Vitoria. Ocupaban los franceses la
orilla izquierda del Zadorra. Otro afluente del Ebro, el Bayas, y otro
camino, el de Vitoria a Bilbao, servía de base al ejército aliado, que
se extendía desde Murguía hasta cerca de Subijana de Álava. Dueños los
franceses del camino de Burgos a Vitoria, tenían segura la retirada, así
como los pasos del río, y una posición excelente en las alturas que
rodean a la Puebla. Este camino, estos puentes y estas alturas, eran lo
que en la mañana del 21 empezaron a disputarles las tropas inglesas,
portuguesas y españolas por diversos puntos y con rapidez y energía
extraordinarias. El inglés Hill y el bravo español D. Pedro Morillo,
atacaron la Puebla y sus riscos eminentes, coronados por una fortaleza
feudal de antiguo llamada El Castillo; el general Graham, con el
guerrillero Longa, atacaron la derecha enemiga en el camino de Bilbao
por Avechuco, y después por Gomarra menor. Conquistados felizmente estos
puntos extremos y altos, fueron atacados todos los pasos intermedios del
Zadorra, el llamado Tres Puentes, Crispiniana y Gomecha. Hubo en estos
ataques alternativas sangrientas de fortuna y adversidad, porque los
franceses los reconquistaban a medias después de perderlos, hasta que
definitivamente los poseían los aliados. Mientras estas luchas horribles
ensangrentaban el Zadorra, hacia el Norte se daba la verdadera estocada
de muerte, con el movimiento de avance del general Graham y del
guerrillero Longa que cortaron al enemigo el camino de Francia. Sin otra
salida que el de Pamplona, precipitose por él todo el ejército, con José
a la cabeza; mas si los hombres que aún tenían piernas pudieron escapar,
no gozaron igual suerte la artillería ni la impedimenta que se atascaron
en el camino, como los ratones con morrión al querer huir después de la
batalla con las comadrejas.

Tal fue en breves términos la de los aliados con los franceses en las
inmediaciones de Vitoria, acción que tuvo, como todas las obras
maestras, una gran sencillez. Si la he descrito a grandes rasgos, no ha
sido porque en ella encontrase menos interés ni menos elementos para la
narración que en otras funciones de guerra, a cuyo relato di
anteriormente, si no gran interés, atención considerable. Me mueve a
hacerlo así, el propósito de variar la materia de estos libros, dando en
el presente la preferencia a una curiosa fase de aquella campaña y de
aquella guerra, cual fue la suerte del más rico botín que un ejército
invasor se ha llevado consigo al abandonar el país expoliado.

En todas las batallas hay un interés subalterno que apenas menciona con
desdén la historia, y que consiste en las vicisitudes de aquel fondo
positivo de toda contienda entre los hombres; en todas ofrece gran
interés el drama oscuro que se desarrolla dentro de la alforja grande o
pequeña que los ejércitos llevan a la grupa. Mientras los generales se
calientan los sesos haciendo cálculos tácticos, y mientras truena la
artillería y se destrozan las falanges, allá en la cola del ejército,
una ciudad portátil, llevada por mercaderes ambulantes, tiembla por su
destino. Las tiendas, los bagajes, las cocinas, las cantinas, los
equipajes, los coches, los botiquines, las camillas representan la vida
y la muerte. Son la suprema necesidad y el supremo peligro de la
batalla. Sin esto no se puede vencer, y con esto no se puede huir.

Todo el interés de la batalla de Vitoria estuvo en la impedimenta. Hacia
aquellos cofres tendiéronse anhelantes las manos crispadas de vencedores
y vencidos. Podía decirse que aquel convoy era el resumen de la guerra,
y que los franceses al perderlo, perdían la tierra trabajosamente
conquistada; al verlo tan grande, tan custodiado, creerían también, que
no pudiendo dominar a España, se la llevaban en cajas, dejando el mapa
vacío.

Y a pesar de la ruda batalla empeñada a la izquierda, el pesado equipaje
seguía adelante, avivando el paso todo lo posible. Era una tortuga
impaciente y azorada que ansiaba resbalar como culebra, y parecía que la
zozobra y anhelo de los que en ella llevaban sus intereses, impulsaban
la pesada armazón. Durante cuatro horas largas, no ocurrió detención
alguna; pero a medida que se acercaban a Vitoria arreciaba el tiroteo,
hasta que llegaron a un punto en que divisaron claramente y a corta
distancia las columnas en movimiento y las baterías escupiendo fuego.
Allí dieron las ruedas su última vuelta, y los caballos su último paso,
y los cocheros su último grito, y el afligido corazón de los viajeros el
último latido de esperanza. Todo acabó: había sonado la terrible
sentencia. No se podía pasar.

---Sr.~Monsalud, eso que me contaba Vd.---dijo poco antes de la
detención la oidora,---es tan inverosímil, que si Vd. no lo afirmara
como lo afirma, lo dudaría\ldots{} ¿Ella misma gritaba que le matasen a
Vd.?\ldots{} ¿Pero qué es esto? Nos paramos otra vez.

---Otra vez, señora\ldots{}

---Y ahora será para siempre---vociferó Jean-Jean.---¡La batalla está
perdida!

---¡Perdida!---exclamó doña Pepita, a punto que el oidor sacaba la
cabeza pidiendo informes.

---¿Dicen que se gana la batalla?

---No; que se pierde---repuso la dama.---No seas impertinente, ni me
estrujes el cabriolé\ldots{} Por Dios, Sr.~Monsalud, ¿nos abandona
Vd\ldots? ¡Qué insoportable ruido! Parece que suenan mil truenos a la
vez\ldots{} Salvador, deme Vd. la mano, a ver si me infunde
valor\ldots{} ¡Por Dios, la mano!

---Una dama valerosa como Vd. no se asustará porque perdamos una batalla
---replicó el joven, alargando su mano.---Ya ganaremos otra.

---La ganaremos, sí, ganaremos una hermosa batalla---dijo Pepita
recobrando sus frescos colores.---¡Cuán cansada estoy de la estrechez
del coche!\ldots{} Quisiera salir un momento, un momentito. ¿Nos
detendremos mucho aquí?

---Per secula seculorum---gruñó detrás del coche Jean-Jean\ldots---Esto
se acabó.

---¡Qué confusión por todas partes!---exclamó Pepita.---Mi marido llora,
Sr.~Monsalud; es demasiado pusilánime. Supongo que no nos harán
nada\ldots{} ¿Será preciso huir?\ldots{} ¡Oh! huir, y ¿cómo?

---En el coche no es posible.

---Pero sí en un caballo, ¡ay! en la grupa de un caballo\ldots{} ¡Dios
mío, cómo gritan! Pues qué, ¿se ha perdido toda esperanza?

El oidor exhibió nuevamente su fisonomía, en la cual una palidez
cadavérica anunciaba el miedo causado por la peor noticia que un oidor
ha podido oír en el mundo.

---¡Pie a tierra todo el mundo!---gritó una voz estentórea.---Las ruedas
no pueden seguir\ldots{}

---Aún hay zapatos y herraduras---clamó Jean-Jean\ldots{}

Casi todos los jinetes echaron pie a tierra, y muchos viajeros
arrojáronse fuera de los coches, despavoridos y aterrados. El concierto
de imprecaciones y lastimosas quejas, excedía a todo encarecimiento.

---Salgamos también---dijo Pepita, llevando el pañuelo a sus ojos para
enjugar una lágrima.---Pero me es imposible andar\ldots{} Sr.~Monsalud,
me desmayaré sin remedio\ldots{} No se separe Vd. ni un momento de mí.

El oidor salió del coche y perezosamente estiró el acecinado y árido
cuerpo para devolverle su posición y forma prístina, semejante a la que
tienen los mortales, cuando no han pasado ocho horas dentro de un coche.
No lo consiguió fácilmente el respetable varón, cuya figura, después que
a sus anchas se desperezó y dejó caer los brazos y echó sobre las
piernas el liviano peso del cuerpo, se asemejaba mucho a un gran
paraguas cerrado.

---¡Esto es horrible, espantoso!---clamaba la dama.---¿Y a dónde vamos?
¿Qué se hace? ¿Qué nos pasa? ¿Hay esperanza de seguir? ¿Nos quedamos
aquí?\ldots{} ¿Retrocedemos?\ldots{} ¿Tomaremos un bocado?\ldots{} ¿Nos
cogerán los ingleses?\ldots{} ¿Pues y nuestro dinero?\ldots{} ¡Oh,
Sr.~Monsalud de mi alma, Vd. que es tan bueno y tan generoso, sálveme
Vd.!

---No es tan desesperada nuestra situación---repuso el joven, notando
que el cuerpo de doña Pepita, al buscar en su brazo indolente apoyo, no
era un cuerpo de sílfide, de fantástica forma e imaginaria pesadumbre.

---¡Qué espantoso es esto!\ldots---añadió la dama.---¡Los hombres gritan
y blasfeman!\ldots{} ¡Las mujeres lloran!\ldots{} ¡Qué
desolación!\ldots{} Sr.~Monsalud, andemos un poquito para
desentumecernos\ldots{} Todos lloran la hacienda perdida\ldots{} ¿pues y
nosotros? ¡traemos tanta plata, tantas alhajas!\ldots{} ¡Yo también
lloro, Dios mío!\ldots{} ¿Será posible que nos cojan esos perros
ingleses?\ldots{} Adelante; vamos por aquí\ldots{} Busquemos a alguien
que nos de buenas noticias\ldots{} no pueden ir las cosas tan mal como
dicen\ldots{} ¡Oh, los ingleses! ¡Cogerla a una los ingleses!\ldots{}
pero no, mil veces no, esclarecido joven, Vd. me defenderá hasta
morir\ldots{} Me horripilo de pensar que un inglés pondrá la mano sobre
mí\ldots{} Sigamos más allá\ldots{} ¿No habrá nadie que diga: «la
batalla se ha ganado»?\ldots{} ¿Pero dónde estamos? ¿Dónde está mi
marido? ¡Se ha perdido!\ldots{} ¡Lo hemos dejado atrás! ¡Urbanito,
Urbanito!

---El señor oidor habrá ido en busca del jefe para saber la verdad de
todo.

---¡Oh, qué horroroso aspecto ofrecen estas pobres gentes!\ldots{} Vea
Vd. en aquella pobre mujer que abraza llorando a sus niños\ldots{} Estos
otros no hablan más que de huir\ldots{} ¡Jesús crucificado! ¿a dónde
iremos nosotros?\ldots{} Será preciso abandonarlo todo\ldots{} ¡Aquí
están diciendo que no hay esperanza!\ldots{} Allí gritan «sálvese el que
pueda». Mire Vd. a esos sacando atropelladamente su ropa de las arcas.
Será preciso llevarlo todo a cuestas\ldots{} ¡Oh! ¿aquellos que por allí
vienen, no son los heridos de la batalla?\ldots{} ¡Malditos
ingleses!\ldots{} Por piedad, Monsalud, no me abandone Vd\ldots{} Es
imposible huir en coche\ldots{} yo no sé montar a caballo\ldots{} ¿podré
ir a la grupa?\ldots{} ¡Qué desolación!\ldots{} Vamos por aquí\ldots{}
los gritos, las blasfemias, los juramentos de esos hombres desesperados
que parecen demonios, me hacen temblar, y me pongo mala\ldots{} Por
aquí\ldots{} Qué bullicio, qué algarabía\ldots{} ¿Y mis alhajas, y mis
encajes, y mis ropas?\ldots{} Corramos allá, corramos\ldots{} Mas no veo
a mi marido por ninguna parte. ¡Urbanito, Urbanito!

---Vamos por aquí\ldots{} En estos casos es triste llevar consigo el
valor de un alfiler. Pobre y desvalido yo, lo mismo tengo vencedor que
vencido.

---¡Qué felicidad!---continuó la dama, que por no encontrarse bien en
ninguna parte, quería estar al mismo tiempo en todas.---Así quisiera ser
yo; libre como el aire, y con la galana pobreza de los pájaros que no
tienen más que un vestido, y a donde quiera que van, llevan todo su
ajuar consigo\ldots{} Huyamos de este sitio. Los llantos de esas mujeres
me hacen llorar también a mí\ldots{} Aquéllos dicen que los ingleses nos
sorprenderán aquí\ldots{} ¡esto es espantoso! ¡Los ingleses, los
guerrilleros!\ldots{} Me parece que muchas personas han emprendido la
fuga por el llano adelante\ldots{} ¿No ve Vd.? Llevan un lío a las
espaldas, y los zapatos en la mano para correr mejor\ldots{} Observe Vd.
a aquel infeliz que se da de cabezadas contra un cañón\ldots{} estos de
aquí hablan de quitarse ellos mismos la vida\ldots{} Por Dios, si forman
Vds. de nuevo, no me abandone Vd\ldots{} deserte Vd. si es preciso,
deserte Vd.. Si me veo sola, me moriré de pavor\ldots{} ¡Yo que pensaba
ir a Francia y regresar a Madrid para el otoño!\ldots{} En medio de mis
desgracias, he tenido la sin igual ventura de conocerle a Vd., de
encontrar a un joven tan leal como modesto que está dispuesto a
ampararme contra esos vándalos de ingleses\ldots{} Estos pobres jurados
y míseros lacayos del Rey José, hablan de morir matando o abrirse paso
por entre los vencedores\ldots{} Les será imposible, ¿no es verdad? Por
Dios, no se abra Vd. paso, no se abra usted paso y quédese aquí\ldots{}
más vale rendirse\ldots{} ríndase Vd.; nos rendiremos los dos\ldots{}
vamos, vamos pronto\ldots{} no puedo ver tanta desolación\ldots{}
escondámonos en algún sitio\ldots{} ¿Ve usted a mi esposo?\ldots{}
Busquémosle\ldots{} es capaz de dejarse dominar por la desesperación, y
hará alguna locura\ldots{} ¿En dónde dejamos nuestro coche?\ldots{} A
prisa, a prisa, Sr.~Monsalud, sosténgame Vd. si me caigo; creo que me
caeré, sí\ldots{} me caigo sin remedio\ldots{} ¡Dios mío! ¿No le parece
a Vd. que me voy a caer?

\hypertarget{xxiii}{%
\chapter{XXIII}\label{xxiii}}

Pero no se cayó. Corrieron Monsalud y Pepita por entre la revuelta masa
de gente y vehículos, espantados una y otro del triste espectáculo que
el detenido convoy ofrecía, y antes que refiramos lo que resultó de su
improvisada amistad y de las extrañas vicisitudes del viaje, es de todo
punto indispensable advertir, que esta gallarda dama del lunar, cuyas
quirotecas tendremos ocasión de ver más adelante en el escenario de
otras historias, pertenecía a la familia de Sanahúja, no siendo ella
misma desconocida para nuestros lectores, pues algún incidente de sus
verdes abriles tuvo cabida en otro libro. Enteramente nuevo para mí y
para los que me leen, es el oidor; pero recientemente han llegado a
estas manos documentos y apuntes, cuyo interés me mueve a asegurar una
poderosa intervención de este personaje en las páginas que leerá el que
las leyere. Por ahora, sólo corresponde decir, que en aquel tumulto de
lágrimas y blasfemias, de desesperación y hondo desaliento, el jurado y
doña Pepa buscaban a Urbanito por todas partes, sin que Urbanito
pareciese.

Entretanto un suceso importante y decisivo llevó al último extremo el
terror de los infelices empleados, bagajeros y conductores; y fue que
por el llano adelante aparecieron varias columnas francesas marchando en
desorden y con precipitación. Aparecieron luego caballos a escape,
cubiertos de espumoso sudor, anhelantes y como poseídos de insensata
cólera, y después muchos heridos transportados en camillas o en
palanquines, o simplemente cargados entre dos por los hombros y los
pies. Tras esto sintiose el rodar estrepitoso de algunos cañones.

---¡Paso, paso a la artillería!---gritó una voz que parecía un huracán.

Los carros que obstruían el camino procuraron abrir calle; pero si lo
consiguieron en un pequeño trecho, después los cañones tuvieron que
hacer alto. Juraban los artilleros y votaban los carreteros. Los de
infantería, desparramándose a un lado y otro del camino, siguieron
adelante. La velocidad adquirida en los primeros momentos de la
retirada, era tal, que no podían contenerse, y miraban hacia atrás
creyendo sentir en sus espaldas las herraduras de la caballería inglesa.

Los heridos fueron depositados en tierra y cuando el furor de las armas
había cesado para ellos, sacaron las suyas los cirujanos. Con la
presteza inconcebible que ponen en sus operaciones los médicos de los
ejércitos, se atendió a todos ellos. Vendajes, emplastos, amputaciones,
cuantos remiendos se aplican a la persona humana después de una batalla,
fueron aplicados sobre el suelo y al aire libre. Corría la sangre sobre
las camillas y por la tierra; pero los lastimeros ayes de los infelices
que habían sido mutilados por el cañón y la fusilería, no eran más que
un accidente superficial en aquel tumulto de tan diversos ruidos
compuesto, en aquella atmósfera de pánico que se extendía por todo el
camino hasta más alla de Vitoria. Era de ver la frialdad de los
cirujanos disponiendo se cortase un brazo o pierna, haciendo brillar a
la luz del sol el fúnebre esplendor de sus instrumentos, para no dar
tiempo a la víctima ni aun a quejarse de su malhadada suerte. En aquella
carpintería de carne humana, no había consuelos morales ni físicos para
el infeliz paciente, ni narcóticos, ni atenuantes, sino la crueldad
fría, desnuda, impasible de la ciencia quirúrgica, que como su parienta
la ciencia militar, no repara en la carne y sangre de los hombres para
ir a su fin.

Conforme los curaban mal o bien, les iban transportando a otro lugar o a
los carros que habían de llevarlos a paraje más seguro; pero llegaron
tantos, que los cirujanos no pudieron atenderles, aunque tenían las
mejores manos del mundo. Arrojados de aquí para allí, clamaban al cielo;
pero el cielo debía de estar ocupado en otra cosa, porque no les hacía
caso.

Por otro lado ocurrían parecidas escenas, porque si el ejército de Gazan
emprendió su retirada por el lado de Berrosteguieta, cerca de donde
estaba el convoy, los de Erlon y Reille lo hicieron más allá de Vitoria;
así es que en una extensión de más de dos leguas se ofrecía el
espectáculo de los soldados furiosos abriéndose camino por entre un
dédalo de carros y cureñas y furgones y ambulancias y coches de viaje, y
cirujanos ocupados, y heridos que no podían moverse.

Aunque en todo el camino reinaba gran confusión, pudo oírse y
generalizarse la orden de que la retirada no se emprendiera por el
camino de Francia, sino por el de Salvatierra y Pamplona. Esto parecía
una salvación, y muchos vehículos y casi toda la artillería se
dirigieron allá; pero la mala estrella de los franceses en aquel día
quiso que el camino de Salvatierra estuviese lleno de zanjas y
cortaduras hechas por los guerrilleros de Mina y Longa poco antes para
molestar a Foy y L'Abbé, por cuyo motivo ninguna rueda pudo pasar más
allá de Harrazo. En el camino de Francia seis o siete coches de lujo
seguidos de otros carros con equipajes y gran repuesto de víveres finos,
pugnaban por retroceder hacia Vitoria para tomar la vía de Salvatierra;
pero no les fue posible abrirse paso. Eran los carruajes de José y su
comitiva, que dispuestos a la cabecera del convoy para emprender la
retirada hacia el Norte, habían tropezado con las tropas de Graham y
Longa.

Hacia las tres de la tarde la irrupción de soldados en retirada aumentó
de una manera horrorosa. Hambrientos y abrasados de sed, se abalanzaban
a las cajas de víveres y a las cantinas arrebatando entre aullidos
siniestros todo lo que hallaban al alcance de sus manos. Agotado todo,
las tropas se apoderaban de los víveres de los particulares, penetrando
brutalmente en los coches para arrancar el pedazo de pan de las manos de
un niño o de una mujer. No pudiendo seguir el camino saltaban los setos
y se esparcían por los sembrados en varias direcciones, siguiendo todas
las veredas con tal que llegasen a parajes lejos del malhadado Zadorra.

Pero cuando el tumulto y el delirante estrépito y el barullo llegaron a
su colmo, fue cuando aparecieron, procedentes del campo de batalla,
veinte o treinta piezas de artillería, furiosas, ardientes, impetuosas,
no hallando ante sí bastante camino para volar; arrastradas por caballos
locos, verdaderos dragones, cuyo resoplido quemaba y que parecían llevar
en sus venas todo el fuego que inflamara los aires durante la batalla.
Aquellas máquinas, simulacro de las ignotas fuerzas que en el cielo
producen el trueno y el rayo, huían para no caer en manos del enemigo.
Los artilleros, semejantes a fabulosos aurigas, herían los caballos con
el látigo primero, y después con los sables, para precipitarlos en
delirante carrera. Todo lo atropellaban ante sí por salvarse. Si un
grupo de heridos o de familias desvalidas se interponía en su camino,
las ciegas máquinas compuestas de cureña, cañón, artilleros y caballos,
pasaban por encima de los cuerpos humanos, como el brutal dios de la
India. Las ruedas, lanzadas en furioso torbellino exterminador, dejaban
hondos surcos en el suelo aplastando todo lo que se les ponía por
delante, la yerba y el hombre.

Un chirrido de metales que juegan y chocan entre sí, de cadenas que se
rozan, de ejes que vibran, de llantas que trepidan, de clavos que
saltan, de tornillos que se aflojan, de cacharros de metralla que suenan
unos contra otros como los cascabeles de un bufón, se mezclaba a los
indescriptibles rumores de las balas que iban moviéndose dentro de las
cajas, tocando infernal música al compás de la marcha; se mezclaba el
golpear de los escobillones, cuyos mangos batían contra el maderaje de
la cureña; al chasquido de cien látigos que culebreaban en el aire
estallando como cohetes; a los gritos de los que querían imprimir a
aquellas máquinas fugitivas el rencor, la angustia y el pánico de sus
inflamados corazones.

Tras aquellas piezas vinieron otras. Calientes aún sus bocas vueltas
hacia atrás, parecía que exhalaban con los últimos vapores de la pólvora
y el último mugido del disparo, sorda imprecación. Treinta, sesenta,
cien cañones huían desesperados: al verlos y al oírlos, creeríase que el
trueno, tomando la odiosa forma de gigantesco pólipo de hierro, se
arrastraba por la tierra. Las peñas de los montes desgajándose y cayendo
sobre el llano y saltando en desesperado juego y carrera infernal por
arte del demonio, no hubieran causado más espanto. Mientras la
infantería continuaba en el fuego, dando tiempo a que el cuartel general
y los cañones se pusiesen en salvo, estos ocuparon todos los huecos que
quedaban en el camino y algunos destrozando cuanto hallaron al paso,
pudieron ponerse en primera línea. Los demás, aprisionados al fin entre
millares de ruedas de pesados bagajes y enormes fardos, se atascaron en
el camino, agolpándose unos contra otros.

Entre esta aglomeración de obstáculos producida por tanta maquinaria
inútil, las infortunadas familias afrancesadas y los conductores del
convoy formaban grupos aflictivos, parte en el camino, parte en los
sembrados, y entre lágrimas y lamentos se consultaban sobre la
determinación que debían tomar en tan extremado conflicto. Unos creían
conveniente abandonarlo todo y huir para salvar lo más importante, que
era entonces, como siempre, la vida; otros aseguraban que por nada del
mundo abandonarían su fortuna. Muchos, encontrando una solución
salvadora en medio del general azoramiento, habían echado a tierra los
baúles y abriéndolos sacaban de ellos lo más valioso, llenándose los
bolsillos y haciendo líos con lo de poco peso. Hombres y mujeres,
soldados y paisanos se consultaban, se movían de aquí para allí,
repartiéndose lo que habían de llevar, aconsejándose unos a otros,
animando los valerosos a los débiles, ayudándose en lo que podían. De
pronto se oyeron en la parte del camino, más allá de Vitoria, las
tremendas voces de «¡paso, paso!»

Algunos caballos de la guardia se esforzaban en cortar el apretado
gentío, y se precipitaban rechinando aguijoneados por la espuela. Viendo
los jinetes que era imposible abrir paso, esgrimieron los sables y
descargando furibundos tajos a diestro y siniestro sobre soldados,
paisanos y mujeres, gritaron:

---¡Paso, paso al Rey!\ldots{} ¡Paso al Rey!

La multitud gimió azotada con látigo de acero, y prorrumpió en
imprecaciones contra José.

---¡Paso al Rey!---repetían los de la guardia.

Exasperados por la resistencia, redoblaron su furor, y cargando sin
piedad, aquí machacaban una cabeza, allí hundían un pecho.
Arremolinándose a un lado y otro y aplastándose contra los coches, la
turba se desgajó y en su angustioso seno pudo abrirse un surco; por una
calle de maldiciones y de odio y de sed de venganza, pasó a caballo un
hombre pálido, con el negro y abundante cabello en desorden, fruncido el
ceño, trémulas las manos. Era José que no había podido salvar sus
coches, y huía a uña de caballo por donde Dios le encaminase, llevando
en su alma todas las congojas de sus cinco años de fúnebre reinado.

Los que le abrían paso, lograron encontrar salida al campo libre a la
derecha del camino. Seguido del general Jourdan, que se había olvidado
el bastón, y de otros generales que olvidaron el sombrero, y aun de
otros que no se acordaban del honor, corrió por allí José lanzando su
caballo a todo escape, aterrado, jadeante, sin serenidad, como el
asesino que acaba de cometer un gran crimen y huye de su perseguidor a
conciencia.

Poco después de este suceso, llegó el momento supremo de aflicción para
los del convoy, para los artilleros, los infantes y todos los que no
podían ponerse en salvo.

Una voz, cien voces gritaron con ronca desesperación:

---¡Los ingleses\ldots{} los guerrilleros!

Allá lejos, hacia Vitoria, entre las columnas de infantería que se
acercaban con el mayor orden posible, viose una multitud de jinetes.
Brillaban en alto los sables, y los veloces caballos avanzaban con
rapidez extraordinaria. Ya no quedaba más recurso que huir abandonándolo
todo. ¡Horrible determinación! Viose a los artilleros desenganchar los
atalajes; viose a los carreteros disponiéndose a salvar sus caballerías.
Las cureñas y cajas y los furgones y las ambulancias y los coches y
carromatos quedaron en un instante libres de correajes y cuerdas. Todo
lo que tenía pies se puso en marcha. Aquello era un río de gente y
caballos, atropellándose unos a otros en violenta confusión a la
desbandada. Ciento cincuenta cañones, doscientos carros de municiones y
los innumerables equipajes y vehículos particulares quedaron
abandonados. Sobre un solo caballo se enracimaban hombres y mujeres,
empujándose para descargar el peso de aquellas tablas de salvación. El
que lograba apoderarse de un caballo defendía la grupa a puñetazos y a
tiros. No había piedad, no había prójimo: reinaba el egoísmo en su
brutalidad instintiva, y se luchaba por el caballo como en los
naufragios por el bote. El que caía, caía.

Apartados del camino, junto a un montón de cajas y bagajes, se
encontraban tres personas que ya conocemos.

---No, no puede Vd. huir---decía la dama deteniendo enérgicamente al
joven y haciendo violenta presa en sus dos brazos.---¡Qué felonía!
¡dejarme sola!\ldots{} ¡mi pobre marido no podrá defenderme!\ldots{}
¡Oh! llora como una mujer y se arrastra por el suelo, pidiendo a Dios
misericordia, sin poner nada de su parte para conjurar este gran
peligro.

---¡Señora, señora!\ldots{} ¡los ingleses! ¡los guerrilleros!

---Sí\ldots{} ya los veo\ldots{} es preciso huir\ldots{} ¿pero cómo? No
hay un solo caballo.

---Corramos en busca del mío---exclamó el joven.---Lo rescataré a
sablazos\ldots{} Aún es tiempo.

---No\ldots{} mi esposo no puede moverse\ldots{} ¿A dónde va
Vd.?\ldots{} Me quedo sola, Virgen de las Angustias, enteramente
sola\ldots{} Quédese Vd., por Dios\ldots{}

---Mi uniforme de jurado me pierde. No viviré ni un segundo después que
me vean.

Con febril presteza e iluminada por súbita idea, abalanzose la dama
hacia el joven; arrojó en tierra el sombrero de este, desabotonó su
levita con dedos más ligeros que el pensamiento, arrancó el uniforme
como si fuera un pañuelo puesto sobre los hombros, arrancó el tahalí, la
gola, el cinturón, la cartera y en un instante no quedó sobre el cuerpo
del infeliz renegado ni una sola prenda que indicara su filiación. Él la
ayudaba con igual rapidez. Aquellas cuatro manos trabajaban en el
desnudar y en el vestir, cual si fueran cuarenta, y sin descansar
arrojaban en tierra las prendas quitadas, sacando otras de los cofres
para cubrir el transformado cuerpo; ataban las cintas, prendían los
botones, abrían un hoyo en el suelo para sepultar las nefandas
insignias, y lo cubrían con tierra. Las cuatro manos realizaron su obra
en pocos minutos, y el renegado desapareció, dejando en su lugar a un
joven que podía pasar por oidor en la sala de Mil y Quinientas. Luego
las mismas cuatro manos trataron de levantar del suelo al infeliz
Urbanito, que ya se creía comido por los ingleses.

\hypertarget{xxiv}{%
\chapter{XXIV}\label{xxiv}}

Los ingleses llegaron despiadados, horribles, hambrientos de matanza y
de botín, como hombres que habían estado luchando todo el día por ambas
cosas. Precipitáronse entre la multitud, mas como no podían avanzar a
causa de los entorpecimientos del camino, les fue difícil perseguir a
los fugitivos, y toda la saña recayó sobre los que no habían podido
escapar.

El botín era el más magnífico, el más rico y grande sin duda que en
batalla alguna ha podido quedar a merced de vencedor furioso. Componíase
de todo: en él había armas, material de guerra, víveres, alhajas, dinero
y hermosura. No puede formarse idea de la apasionada codicia, de la
brutal concupiscencia, del vengativo ardor con que los ingleses primero
y los guerrilleros después cayeron sobre el magnífico tesoro abandonado.
La menor resistencia producía la muerte. En poco tiempo todas las cajas
fueron abiertas, todos los tesoros aprehendidos, muchas riquezas
holladas.

Joyas, ropas, telas finísimas, muebles, cuadros, plata labrada, monedas,
víveres de lujo que constituían la despensa ambulante de José, fueron
esparcidos por tierra, y mil manos febriles arrebataban de un lado para
otro los preciosos objetos. Según el genio de cada cual así se iban
derechos los unos al oro, otros a las mujeres, y algunos a destrozar por
puro instinto dañino cuanto veían delante. Entre las desgraciadas
familias que se vieron en tan tremenda hora, hubo algún individuo que se
dio la muerte antes que le pusieran la mano encima los feroces
partidarios. Las señoras imploraban de rodillas piedad para sí y sus
tiernos hijos, siendo muy contadas las que la alcanzaron. El vencedor es
la más brutal e insensata bestia que engendra el mal en las tempestades
humanas. Para esta electricidad furibunda que sabe elegir el sitio donde
cae, no existe pararrayos.

En los primeros momentos, tanto salvaje atropello y brutal codicia
produjeron un tumulto horroroso, en el cual los lamentos de mil y mil
víctimas no permitían oír las voces y mandos militares. En la vasta
extensión del camino, los soldados cometieron todo linaje de excesos,
robando y asesinando. En vano algunos oficiales quisieron proteger a las
infelices familias de paisanos: la soldadesca, aparentando obedecer, tan
sólo cambiaba la escena de sus infames tropelías. Por aquí un soldado
avanzaba en irrisoria apoteosis esgrimiendo el bastón de mando del
general Jourdan, jefe de Estado Mayor del ejército fugitivo; otro
cubríase acullá con el sombrero de José Bonaparte, y un tercero repartía
a sus camaradas las pelucas que en vistosa y variada colección llevaba
en su equipaje otro familiar del pobre Rey intruso.

Atreviose un sujeto de mal genio a descalabrar a cierto inglés, porque
quiso posesionarse de la menor y más hermosa de sus hijas, y este rasgo
de entereza costole la vida, salvándose su esposa, una de sus hijas y
dos niños de corta edad, por milagro del cielo y la intervención
compasiva de otros soldados. En lo de meter mano a los cofres de dinero,
a los bolsones de cuero y a las cajas de guerra que contenían inmensos
caudales, distinguíanse principalmente los aldeanos de los alrededores
de Vitoria y multitud de individuos de equívoca conducta, que de la
misma ciudad habían acudido.

Cuando la tristísima noche empezó a cubrir de oscuridad la fatal escena,
mercaderes al menudeo, trajineros y gentezuela de esa que acude a todos
los desastres para pescar algo, se reunieron allí en gran número. Como
ellos lo querían todo para sí, hubo dimes y diretes y aun porrazos con
los guerrilleros y los ingleses. Sin pedir permiso a Dios ni al diablo,
los aldeanos cargaban sus caballerías de objetos preciosos, como si todo
cuanto allí yacía hubiera sido siempre de su exclusiva propiedad; y
mientras tanto no cesaban de aclamar a Fernando VII como el más grande
de los Reyes, al lord como el más insigne de los generales nacidos y por
nacer y a los guerrilleros como lo más selecto entre las hechuras de
Dios.

Cuando la noche se oscureció más y la vergüenza de tales hechos tuvo un
manto negro con que cubrirse, otros individuos de la peor calaña, se
ocupaban en desnudar a los muertos y en buscar anillos y relojes y dijes
en el cuerpo de los heridos\ldots{} Mil farolitos temblorosos semejantes
a las vagabundas claridades de un cementerio, rebuscaban con su luz
siniestra por aquí y por allí, iluminando semblantes lívidos y
destrozados cuerpos. Por otro lado los que habían recogido gran cantidad
de dinero en duros españoles, se ocupaban en cambiarlos por oro a los
ingleses, los cuales, como buenos mercaderes en toda la extensión del
globo terráqueo, se hacían pagar la guinea a ocho pesos. Había quien
acaparaba todas las ropas, ora sacándolas de los cofres, ora
arrancándolas del cuerpo de vivos y muertos. Porque nada faltase, hasta
hubo quien hizo acopio de la pólvora de los furgones, para venderla
después a los guerrilleros de la Montaña y el Páramo. El vino obtenía
preferencia y primas escandalosas, y toda la carretería y recuas de
Vitoria tuvieron en qué ocuparse. Muchos aldeanos se enriquecieron con
la rapiña de aquella noche, y en Álava y la Rioja existen todavía
familias ricas cuya fortuna proviene de la batalla de Vitoria.

En cambio, si gran parte del gentío de Vitoria y de sus inmediaciones
había acudido allí para recoger los restos del naufragio, muchas
personas llegaban impulsadas por la simple vehemencia personal de la
guerra, para contemplar el odioso imperio derrotado y sus armas
perdidas; para gozar en el mísero castigo de los malos patriotas, y
escupir los avergonzados semblantes de los traidores. Cuentan que
algunos renegados a quienes no fue posible ni huir, ni cambiar de
vestido, recibieron rápida muerte todos juntos en fiera hecatombe, sin
que les valiese la ardiente protesta de abjurar y volver a los amores de
la patria. Una mujer furiosa cayó sobre el grupo que formaban aquellos
infelices al implorar piedad y alzó en su mano vigorosa un puñado de
cabellos. Rugiendo los enseñó a la muchedumbre. Aquella y otras mujeres
de las cercanías que acudieron a vociferar sobre el cadáver de la
Francia vencida, habían mandado a sus hijos a las guerrillas, y algunas
de ellas los habían perdido. Bravas como guerreras y resentidas como
leonas, cobraban de tal manera sus deudas de sangre.

En la oscuridad de la noche los chillidos de las mujeres semejaban la
algazara de pájaros rapaces picoteando aquí y allá, batiendo las
fúnebres alas, destrozando con la inquieta garra. Sin callar un momento,
algunas ayudaban a los hombres en el despojo, examinaban una tela,
ponderando su finura, recogían herramientas abandonadas, sin dejar de
responder con agudos vivas a todo lo que berreaban sus hermanos, sus
padres o sus hijos.

Dos o tres de estas matronas discutían el modo de conducir cierta
cantina ambulante que se habían apropiado, cuando se les acercó una
afligida dama que parecía ser de las del convoy. Era hermosa aunque la
palidez y susto le disimulaban su belleza. En su cabellera abundante y
en su vestido no había más que desorden, un desorden de naufragio que
daba más interés a su abatida persona; y con sus manos sin quirotecas se
apretaba contra el pecho un chal, no bien puesto y sin duda arrebujado
con precipitación al salir de su escondite.

---Señoras---dijo acercándose con timidez a las que tomaban el tiento al
tonelete de la cantina,---si tienen Vds. corazón, si son Vds. mujeres, y
tienen hijos, padres, esposos, denme un poco de agua para unos
pobrecitos que se mueren de sed allí donde están los arcones grandes.

---Miren la pazpuerca---gritó una de las del grupo, que era tabernera en
el barrio de Villasuso en Vitoria.---Teniendo, como tendrá, todo lo que
ha robado, viene a pedirnos limosna.

---Yo no he robado nada, señora---repuso la dolorida envolviéndose en el
chal con todo el empeño que el pudor y el fresco de la noche exigían de
consuno.---A mí sí que me han quitado cuantas alhajas y dinero tenía;
pero no me quejo, ni acuso a nadie.

---Ladrón que roba a ladrón\ldots{}

---Por una casualidad nos hemos encontrado mi marido, mi hermano y yo en
este funesto lance---prosiguió la dama,---porque ninguno de los tres
somos, ni hemos sido jamás, afrancesados. Españoles rancios somos los
tres; íbamos a Francia (adonde mi marido llevaba una comunicación
secreta de la Regencia para el rey Fernando) y quiso nuestra infeliz
suerte que nos juntásemos aquí con el malhadado convoy que ayer
pereció\ldots{} y nos tomaron por familia de empleados traidores\ldots{}
Pero no he sido yo tampoco de las peor tratadas (porque al punto me
conocieron los oficiales ingleses, muchos de los cuales han frecuentado
mi casa en Madrid) y he podido conservar alguna ropa\ldots{} Otras
pobrecitas señoras están allí envueltas en una sábana. ¿No les da a Vds.
lástima? ¿No me favorecerán con un poco de agua y si es posible un poco
de comida para mi esposo, secretario del virrey del Perú, y para mi
hermano el veedor que era en Zaragoza cuando la célebre defensa?

Las tres alavesas se miraron como consultándose sobre lo que habían de
hacer.

---La verdad es---dijo una con ínfulas de autoridad sobre las
otras---que si no miente la señora en lo que ha dicho y hubo casualidad,
bien se le puede dar lo que pide.

---¿La vamos a creer por lo que diga?---exclamó otra.

---No pido más que agua, señoras caritativas, agua por amor de Dios.

---Él la ampare.

---Bien poco es lo que pide---dijo la tercera que hasta entonces
callara.---Y pues pasó ya el laberinto, hagamos una obra de
misericordia. Aquí donde me veis, yo, que tuve alma para arrastrar a un
jurado desde el camino hasta el árbol donde le ahorcaron, me muero de
pena oyendo a esta señora\ldots{} Allá va el agua\ldots{} y
aguardiente\ldots{} y estas cortezas de pan\ldots{} y estas sardinas
rancias\ldots{} y tres pares de guindas\ldots{} y una pata de gallina
fiambre, que estaba en el botiquín del Rey.

La dolorida iba recogiendo lo que la mujer indicaba al tiempo de
dárselo, y corrió a donde aguardaban muertos de hambre y de sed el
secretario del virrey del Perú y el veedor de Zaragoza.

\hypertarget{xxv}{%
\chapter{XXV}\label{xxv}}

Tras la triste noche, apareció el día triste también, y empañado con
densas neblinas. Mientras gran parte del ejército victorioso perseguía
al francés por el camino de Salvatierra, el lugar donde pereció el
convoy se trocaba en un campo de feria. En todas partes se hacían tratos
y cambios, según los negocios de cada uno. Los ingleses concretaban
todas sus operaciones al numerario, despreciando las especies. La
joyería había desaparecido como por encanto, sin que se supiese quiénes
fueron los acaparadores de tan estimable artículo. En plata labrada aún
quedaban algunas existencias por la mañana, y como entre ellas no
escaseaban las obras de arte ni en el ejército inglés los anticuarios,
hubo pieza que valió a sus primitivos tomadores guinea sobre guinea.

Pero la gran mayoría de los objetos, especialmente los que eran de fácil
transporte, desaparecieron en la noche. No se han visto manos más
listas, ni mayor diligencia en hombres y mujeres para hacer la mudanza.
Por fortuna para las artes, la parte del convoy que contenía los grandes
cuadros, pudo ser salvada por haber salido de la Puebla con el general
Maucune doce horas antes que los demás. Perdiéronse por entonces para
España tan incomparables tesoros; mas no se perdieron para el arte,
siendo en verdad providencial que se salvasen, y que restaurado alguno
de ellos, volviesen todos acá tres años después.

Ya entrado el día, muchos vecinos acomodados en Vitoria salieron para
ver el campo de batalla y el lugar del convoy, que principalmente
despertaba la curiosidad. Viéronse llegar frailes de distintas órdenes,
canónigos de la colegiata, señores muy graves acompañados de damiselas
sensibles, jóvenes currutacos, viejos verdes y maduras matronas, todos
medio locos de entusiasmo por la gran victoria alcanzada. Iban de ceca
en meca sonriendo ante los estragos y haciéndose señalar por los
aldeanos los lugares que fueron teatro de acontecimientos más trágicos
durante la batalla. El campo del convoy, ya convertido en feria, fue por
su proximidad a Vitoria más visitado, y a cada momento llegaban a él
alegres parejas, familias, tríos de canónigo, fraile y regidor, con más
algunas damas sueltas, es decir, que no iban con nadie. Ninguno se
retiraba sin llevar algún recuerdo, pareciéndose en esto a los modernos
ingleses, o a los que llaman touristas, y los cascos de granada, las
balas de fusil y hasta los botones de los uniformes de renegado pasaron
a ser joyas históricas, destinadas a vincularse en el patrimonio de las
familias. Aún existen en Vitoria muchos de estos pedacitos del gran
desastre.

Diose orden de enterrar los cadáveres que en el llano del convoy había,
no siendo tan fácil los del vasto campo de batalla por ser en número de
cuatro mil, juntas las pérdidas de unos y otros, pasando de diez mil los
heridos. Mortificó a los curiosos el espectáculo de tanto hombre muerto,
siquier fueran franceses y renegados; y muchos ofrecieron la cooperación
de sus manos para echar tierra dentro de los hoyos que se tragaban tanta
juventud desgraciada en vida y en muerte, los amores de innumerables
madres, tanta y tan robusta vida nutrida en los pacíficos hogares para
la paz y la felicidad.

Entre los curiosos que de Vitoria habían venido era de notar un anciano
de mucha edad y poca andadura, con el cuerpo inclinado hacia adelante,
la cabeza temblorosa, verdes espejuelos ante los ojos y apoyada la una
mano en grueso bastón de nudos, mientras con la otra cogía el brazo de
una linda joven rubia. Iban los dos por el camino adelante observando
todo con curiosidad suma, siendo ella la que primeramente con sus
vivísimos volubles ojos veía los objetos y los señalaba después a la
tardía atención del viejo. Él se regocijaba con la vista de tanto cañón
tomado, de tanta riqueza rescatada, y a cada nueva sorpresa se
desvanecía en apologéticos comentarios de la destreza de lord
Wellington, encomiando, sobre todo el providente designio del Altísimo,
que como padre y ordenador de las victorias, nos había dado aquella tan
completa y admirable.

---La causa de Dios triunfa y triunfará mientras haya soldados
cristianos en el mundo---decía el abuelo a su linda nieta.---A estos
desastres horrorosos son conducidos los que han intentado alevemente
apropiarse nuestro suelo, y mudar nuestras costumbres, haciéndonos de
fieles piadosos, herejes corrompidos, de leales y pacíficos,
revolucionarios y jacobinos.

---¡Ah, pobres muchachos!---exclamó la nieta y apartando con horror la
vista de unos infelices cuerpos de jurados que eran conducidos a la
sepultura.---Son renegados, papaíto, tienen uniforme verde, sombrero de
piel con águila dorada, una cartera en la cintura con águila, y muchos
botoncitos\ldots{} también con águila.

---Sí, verás águilas por todas partes. Esos hoyos se llenarán de ellas,
y la honda tierra no podrá guardar en su seno tantas insignias
imperiales. A eso está destinado el poder de Bonaparte. Europa no tiene
bastante tierra para sepultar el inmenso cadáver\ldots{} En cuanto a los
infelices jurados, son los que menos lástima me inspiran. Oye bien lo
que te digo, hija mía, oye la voz de un anciano patriota, español y
cristiano: además del infierno que existe para toda clase de pecadores,
ha de haber uno con tormentos extraordinarios de inapreciable horror
para los que hacen traición a su patria y a sus banderas.

---¡Otro infierno!---exclamó la muchacha con espanto, a pesar de que
diariamente oía parecidos conceptos.

---¡Otro! Allá en lo profundo los condenados ordinarios no han de querer
habitar con los renegados y traidores---dijo el hombre decrépito,
silabeando enérgicamente con sus gruesos labios.---Los renegados venden
a sus hermanos, entregan la patria al enemigo para que este la despoje y
la deshonre a su antojo extirpando en ella la fe religiosa, faro del
mundo y único consuelo de las buenas almas. El traidor en esta guerra,
donde se discuten las dos cosas más sagradas, es decir, el Rey y la
religión; el traidor en esta guerra, digo, es el más vil instrumento de
Satanás. Sólo le igualan en maldad los que yo llamo traidores y
renegados en el campo de la ley, o para que me entiendas mejor, los que
por favorecer hipócritamente a Bonaparte, introducen en España
caprichosas leyes a estilo jacobino, y constituciones que son lazos
tendidos a los pueblos por la herejía, por la licencia, por el
democratismo, por la soberbia de los pequeños que quieren parecerse a
los grandes, gritando y metiendo bulla\ldots{} Pero Dios está con
nosotros, hija mía. Dios es español.

---¡Dios es español!

Dios, sí---añadió el viejo golpeando violentamente el suelo con su
nudoso bastón,---y ya ves ahí los golpes de su mano protectora. Creo que
mediante la bondad divina y la espada del arcángel guerrero, el mal que
aparece en nuestra leal y sumisa España no tomará grandes proporciones.
Abriranse muchos hoyos como ese, y esas bocas de la tierra española se
tragarán a sus perversos hijos.

---¡Ay!---gritó la muchacha, temblando y agarrándose fuertemente al
brazo de su abuelo.---Pero no es nada\ldots{} nada, papaíto.

---¿Tienes miedo?

---No\ldots---dijo la joven, reponiéndose de su sobresalto y
turbación---es que\ldots{} no sé por qué me he estremecido toda y he
sentido frío en el corazón al ver\ldots{}

---¿Qué has visto?---preguntó el viejo deteniéndose.

---Todavía no han enterrado aquellas águilas, papaíto, aquellas águilas
que brillan en los sombreros peludos, en las golas, y en las carteras, y
en los botones\ldots{} Sus alas abiertas, sus picos corvos, sus garras
que aprietan un haz de rayos\ldots{}

---¿Qué?

---Me dan miedo.

---¡Eres tonta! Adelante\ldots{} Pero si no me engaño, ese que hacia
aquí viene es nuestro amigo Carlos Navarro, el hijo de D. Fernando
Garrote\ldots{} Mira tú, a ver si me engaño\ldots{}

Miraba hacia atrás la damita con la fijeza de una curiosidad vivísima.
Su rostro había adquirido marmórea blancura.

---¿Por qué te detienes y miras hacia atrás?---gruñó el viejo sacudiendo
el brazo.---¿Dices que tienes miedo y miras, Genara?\ldots{} Te digo que
observes si ese que se ha detenido junto a aquel cañón es Carlos
Navarro, el hijo del desgraciado D. Fernando Garrote.

---El mismo es---repuso Genara observando.

---Vamos hacia él\ldots{} ¡Pobre muchacho! Quizás no sepa todavía el
desgraciado fin de su padre, asesinado en Aríñez por los vándalos.

Antes que nieta y abuelo llegasen junto a él, Carlos Navarro, que los
vio, corrió a su encuentro. Su semblante estaba alterado por viva
aflicción y algunas lágrimas humedecieron sus ojos cuando tomó para
besarla la mano del decrépito anciano, su amigo.

Vestía Navarro un traje que no era completamente militar, ni tampoco de
paisano. Componíase de una blusa en cuyas mangas, a falta de
charreteras, mostraban la arbitraria graduación del guerrillero, galones
diversos de plata y oro, puestos con arte y aun con cierta elegancia.
Botas y espuelas muy finas eran distintivo de que guerreaba a caballo, y
cubría la cabeza no con los empinados morriones de la época, sino con
una sencilla gorra verde de cuartel, primorosamente bordada de oro. La
sofocación del día anterior y la pesadumbre recientemente recibida
habían dado a su rostro un tinte violáceo y como enfermizo que parecía
aumentar el negror de sus fieros ojos y afilarle la nariz y hacerle más
grande la vasta frente. Había en su cuerpo la indolencia de la victoria
un poco enfatuada; pero aun así, por su alta estatura y airoso porte y
grave semblante era una de las figuras de más atractivo que podían
verse.

---Señor D. Miguel de Baraona---dijo con voz conmovida,---¿ha venido Vd.
desde Vitoria a ver el campo de batalla y el gran convoy ganado?

---Sí---replicó con entusiasmo el anciano encendido su corazón con fuego
juvenil,---he venido a ver vuestros triunfos, vuestra gloria, jóvenes
sublimes, jóvenes admirables, ¡hijos queridos de España y de Dios! Ven
acá---añadió echándole los brazos al cuello,---ven acá y déjame que te
estreche contra mi corazón: abrazándote, creo abrazar a toda la España
valerosa y cristiana. Me rejuvenezco, hijo mío. Que Dios te bendiga, que
Dios te conserve. Tú y los tuyos sois instrumentos de su bondad divina,
sois la imagen humana de su brazo omnipotente. Seguid en vuestra
gloriosa, en vuestra santa tarea de limpiar esta cizaña, que no os
faltará que hacer en algún tiempo, porque el mal se ha desatado en
España y vendrán días de sangre\ldots{} Ya sé por qué estás tan
afligido, hijo mío, ya he sabido por unos jurados prisioneros que fueron
anoche a Vitoria, la inmensa desgracia\ldots{}

---¡Mi padre!\ldots---exclamó Carlos cubriéndose el rostro con las
manos.

---Tu padre, tu excelente padre---dijo Baraona.---D. Fernando Garrote,
el gran caballero cristiano de Treviño, el hombre de ideas sólidas, el
español puro ha sido asesinado por los traidores\ldots{} Lo sé, y he
llorado al patriota y al amigo. También sé que murió el pobre
Respaldiza.

---¡No esperaba esta desgracia!---murmuró con desaliento Navarro secando
sus lágrimas.---Confiaba en Dios; me sentía protegido por la divina
mano, y al ver el heroísmo de mi padre, su firme propósito de pelear por
la patria y por la Iglesia, creía yo que el Señor no podía abandonarle
en manos de los facinerosos.

---¡Oh! ¿Sabemos acaso sus designios profundos?---dijo con buena
entonación Baraona, señalando con su palo el firmamento inundado de
luz.---Hijo mío, oye bien lo que te digo, que es la voz de un patriota y
de un español puro, sin mancha de afrancesamiento. Además del paraíso
que Dios destina a los elegidos, ha de haber otro paraíso mejor para
estos mártires de la patria, para estos defensores de los grandes
principios, para estos que en primera línea han peleado por la esposa de
Jesucristo, para estos a quienes debe la sociedad su fundamento, para tu
virtuoso y santo padre, en fin.

---¡Otro cielo!---murmuró Genara pensativa.

---¡Has perdido a tu padre!---prosiguió Baraona con efusión estrechando
de nuevo al joven entre sus brazos.---En mí tendrás otro desde hoy.

Carlos Navarro se arrojó en los brazos del anciano ocultando en el
hombro de este su rostro inundado de llanto.

---Hace tiempo que tu buen padre me habló de un dulce proyecto que me
agradaba en extremo, Carlos---dijo el viejo mirando alternativamente a
su nieta y al joven guerrillero.---¿Sabes lo que quiero decir? Tú mismo
me has manifestado de una manera indirecta la noble afición que te
inclina hacia mi familia. Carlos, hijo mío, que este día de gloria,
aunque triste para ti, lo sea también de contento para los tres que aquí
estamos.

Genara se puso como una amapola.

Contra lo que Baraona esperaba, Carlos no hizo demostración alguna de
contento. Mirando a Genara con tristes ojos, dijo:

---Genara no me quiere.

---¡Que no! ¡Mal pecado!---gruñó el viejo mirando con asombro a su nieta
que callaba.---Genara, recuerda lo que me dijiste la noche en que
salimos de la Puebla\ldots{} Pero, hijos míos, vosotros os entenderéis.
No es propio de mis canas intervenir como mediador de galanteos. Carlos,
ven con nosotros. Tú tienes cara de no haber comido en tres días; yo y
mi nieta no hemos tomado cosa alguna después del chocolate; pero como
pensamos pasar aquí gran parte del día, trajimos una no despreciable
refacción. Vamos allá\ldots{} ¿En dónde dejamos el coche, Genarilla?
Ya\ldots{} ahí; hacia aquellos olmos. Ven Carlos; allí nos espera el
señor canónigo de la colegiata, D. Blas Arriaga, el capellán de las
monjas de Santa Brígida y mi primo el secretario de la Inquisición.
Despáchate, si tienes algo que decir a tus amigos, acaba pronto, pero no
convides a ninguno, porque nos quedaríamos a media ración\ldots{} La
merienda no es mala; viene alguna carne fiambre y lengua y una pavita.
Las monjas añadieron bollos y limoncillos, y el canónigo trajo lo mejor
de su bodega\ldots{} Pues parece que no y tengo hambre. Este aire del
campo, el regocijo de este día\ldots{} En marcha, en marcha, pues.

Dirigiéronse los tres hacia el lugar donde esperaba el cochecito. En los
lugares más apacibles del vasto campo, veíanse algunas meriendas sobre
la verde yerba, pues los vitorianos hicieron festivo aquel día, tomando
la visita al campo de batalla como una especie de romería, en la cual no
podían faltar ni el buen vino, ni las buenas tajadas, ni la noble
expansión éuskara.

Genara y Carlitos marchaban silenciosos, pero por los tres hablaba D.
Miguel de Baraona, siendo tal su alborozo, que desde lejos empezó a
agitar el palo, llamando con su cascada voz a los tres personajes que
antes mencionara y que vagaban por aquellos contornos. Antes de que
todos los comensales se reunieran, pasaron Baraona y la nieta por el
mismo paraje donde poco antes infundieran a ésta tanto miedo las águilas
de los insepultos jurados.

---¿Otra vez tiemblas?---le dijo el abuelo observando que la muchacha
palidecía.---¡Qué medrosa eres!

---Genara no puede tener miedo a los muertos---afirmó Carlos con
aplomo.---Genara es una mujer valerosa.

---¡Ay, no vayamos por aquí!---exclamó la joven soltando bruscamente el
brazo de su abuelo:---he visto, he visto\ldots{}

---¿Qué has, visto?

---Ya están dentro del hoyo---dijo Baraona acercándose al grupo de gente
que rodeaba la ancha sepultura,---pero falta echar tierra, mucha tierra
encima.

Genara, a pesar de su agitación, en vez de huir, acercose resueltamente
al hoyo, y allí permaneció fija, inmóvil, con la vista clavada en
aquella hondura donde yacían revueltos y en extrañas posturas los
cuerpos arrojados dentro. Observolos a todos y a cada uno con atención
profunda: ni lloraron sus ojos, ni perdió su semblante aquel grave ceño
estatuario que la asemejaba en tal escena a una diosa antigua recibiendo
la ofrenda de sangre humana vertida en aras de su orgullo.

---Abuelo, ya ves cómo no tengo miedo a los muertos---dijo al fin:---¿y
tú?

---Ven, ven acá, tonta, tontísima---gritó el abuelo.

Los que contemplaban el fúnebre espectáculo se descubrieron, y empezó a
caer tierra dentro.

---Dios manda que se rece a los muertos y se perdone a los que nos han
ofendido---dijo gravemente Navarro descubriéndose también al pasar junto
al hoyo y mirando los fúnebres despojos que dentro había;---pero no
puedo mirar sin encono vuestro uniforme. Si tuvisteis parte en la muerte
del mejor de los padres, ¡malditos! que Dios os condene eternamente, y
sean vuestros tormentos superiores a todo lo que puede idear la
imaginación más exaltada.

Dicha esta imprecación, que denotaba las violentas pasiones del alma de
Carlos Garrote, hizo la señal de la cruz y se unió a Baraona que ya
estaba algo distante, junto a su nieta. Cuando llegaron bajo los olmos,
ya el canónigo de la colegiata, el capellán de las monjas y el
secretario de la Inquisición revolvían la cesta de los fiambres.

\hypertarget{xxvi}{%
\chapter{XXVI}\label{xxvi}}

Aquella a quien oímos primero junto a la empalizada de una huerta de la
Puebla de Arganzón, y acabamos de ver y oír ahora mismo al borde de una
sepultura, era una muchachuela bonita, de apariencia delicada y casi
infantil. Recordaba normalmente su fisonomía la de aquellas vírgenes a
quienes figuran los pintores tocando el laúd y a veces el violín en los
místicos conciertos del cielo, entre aperladas nubes que hacen resaltar
el oro de sus cabellos y la beatífica seriedad de sus labios sin
sonrisa, pues el arrobamiento y el canto las ponen graves como doctores.
Genarita o Generosa, a pesar de su belleza original, tenía en ocasiones
un ceño bastante sombrío y un modo de mirar que no indicaba la
diafanidad, o mejor, el perfecto equilibrio de espíritu de un ángel
celeste. Era solemnemente meditabunda, y aunque su semblante era de esos
que en otros caracteres y en la misma edad están siempre mirando a todos
lados, aunque no vean más que el vuelo de las moscas, ella parecía estar
dispuesta a no ocuparse nunca de cosas pequeñas. Las moscas que ella
miraba, no las veían los demás.

La fisonomía engaña casi siempre, y bajo aquel semblante que recordaba a
la espigadora Ruth o a la organista Cecilia, se escondía una culebrita
graciosa que halagaba enroscándose, un carácter vehemente que a la edad
de diez y siete años vivía atormentándose a sí mismo con aspiraciones
locas, con entusiasmos delirantes, con deseos no bien definidos o que
variaban a cada hora. El reptil se mordía a sí propio, por no haber
encontrado todavía en quien cebarse, y con la cola se azotaba la cabeza.
Impresionable hasta un extremo casi inverosímil, lo que a otras
entristecía, a ella la ponía furiosa, lo que a otras daba alegría,
infundía en aquesta una fiebre de júbilo, que necesitaba un pesar para
calmarse. Sus sentimientos siempre en lucha, se manifestaban de
improviso y de una manera torrencial y borrascosa. Cualquier accidente
externo, impresionándola como impresiona el rayo, podía hacerlos cambiar
en un instante.

Sus ideas eran, sin embargo, exclusivas y fijas, ideas asimismo oscuras
y extravagantes sobre la vida y la sociedad, pero arraigadas con
tenacidad extraordinaria. Tenía la terquedad de su abuelo, hombre de
granito, una especie de montaña humana, formada con los seculares
yacimientos del ideal de la autoridad, y que no podía henderse ni
desmoronarse, ni dejar de ser montaña. Carecía Generosa de la fácil
ternura que parece propia de una complexión delicada, y cuando este
dulce sentimiento aparecía en ella, era enteramente superficial y
simulado. Finalmente, no le faltaban dotes de inteligencia, siempre que
no se tocase a las preocupaciones o a las ideas que en su consistencia
geológica eran base de la familia.

Todo esto lo vemos más adelante, porque esta hermosa bestiecita, esta
mujer linda y profunda, este hermoso vaso lleno de tempestades, y que
conteniendo el Océano parece una redoma de peces, ocupará lugar muy
importante en las historias que van a leerse, y a las cuales sirve de
prefacio la siguiente.

Sentados todos, y tendido el mantel, la cesta dio de sí todo lo que
tenía, y empezó la comida.

---Es preciso sobreponerse a la tristeza que esos desagradables sucesos
hayan podido ocasionar a alguno de los presentes---dijo el viejo
Baraona, descuartizando la pava, mientras el capellán de las monjas de
Santa Brígida aplicaba su nariz a la boca de las botellas para ver si
era justa la fama de las bodegas del señor canónigo.

---Basta de melancolías, Carlitos---indicó el secretario de la
Inquisición.---A lo hecho, pecho, y cuando las cosas no tienen
remedio\ldots{}

---Dejadle que se desahogue y llore la muerte del más insigne caballero
de este país---ordenó con énfasis Baraona, partiendo en lonjas la lengua
de vaca, sin dar ni por un momento reposo a la suya,---de aquel modelo
de patricios, de aquel hombre cuyos sanos principios en todo lo relativo
al gobierno de estos reinos, eran admiración y enseñanza de cuantos le
oían.

---Grande y ejemplar varón ha perdido España, no puede dudarse---añadió,
elevando los ojos al cielo, el capellán de Santa Brígida, tranquilizado
ya respecto a los títulos de celebridad de las bodegas de su amigo.---Le
lloraremos toda la vida los que conocimos su caballerosidad y aquella
noble entereza de principios.

---Su muerte---dijo Baraona llenando los platos de los demás,---debe
quedar en la memoria de los buenos hijos de España como un recuerdo
santo. Ha sido el mártir de esta gloriosa fe del patriotismo cristiano,
del patriotismo cristiano, señores, entiéndase bien. Siempre habrá
distancia inconmensurable entre lo que yo llamo el patriotismo cristiano
y esa gárrula palabrería de los que se llaman patriotas en Cádiz y en
Madrid.

---Los que nos llaman serviles, Sr.~D. Miguel---indicó el capellán.

---Tan infame mote---afirmó Baraona frunciendo el ceño y apretando el
puño,---será escrito con sangre en la frente de los que lo inventaron.
¿No es verdad, Carlitos?

Carlos, profundamente abstraído, ni comía ni contestaba sino con ligeras
inclinaciones de cabeza.

---¿Saben cómo les llamo yo?---dijo Baraona con violenta cólera y dando
fuerte golpe en la tierra con la botella que en su mano tenía.---¡Pues
les llamo negros!

---¿Negros?---dijo Genara con súbito arranque de jovialidad que
contrastaba con su anterior tristeza.---Pues sea: beba Vd. señor
capellán, beba Vd. señor canónigo, y Vd. señor secretario.

Y tomando la botella de manos de su abuelo, a todos repartió porción
bastante a humedecer los secos paladares.

---¿Y Vd. no bebe, Generosita?

---¿Yo?\ldots{} Una miaja\ldots{} menos, mucho menos, señor capellán,
con medio dedo me basta---repuso la muchacha levantando el vaso para
impedir que el capellán lo llenase todo como quería.

---Y aún me parece mucho---indicó Baraona.---A ver, Carlos, tu vaso.

---Ahora---dijo la doncella con animado semblante,---alcen Vds. los
vasos y beban a la salud de toda la gente blanca.

Tan entusiástica proposición, dicha con arrebatadora voz, con gran
viveza en los ojos, con una sonrisa celestial que descubrió los blancos
dientecitos de la víbora entre el coral de sus frescos labios, y
acompañada de un gracioso gesto con brazo y mano derechos, produjo
mágico efecto entre los comensales. Gritaron todos, y una aclamación
recorrió aquellos campos de tristeza.

---Las mujeres---dijo Baraona,---tienen el don de expresar las ideas con
gracia incomparable y en forma que las hace inteligibles a todo el
mundo. A la salud de toda la gente blanca, a la salud de la patria libre
de franceses y de ideas francesas, de la religión de nuestros padres, de
nuestras santas y morigeradas costumbres, de nuestra inmutable y siempre
gloriosa España, que desafía a los siglos y sobre la cual pasan y
pasarán los negros innovadores, como hojas de otoño que se lleva el
viento.

---Amén---murmuró el capellán.

---El pobre Carlitos no come---dijo el canónigo.---No debe uno dejarse
dominar por el dolor. Hay que hacer un esfuerzo\ldots{} no debe ser
desatendido el cuerpo. Aquí donde me ven, aunque parece que tengo
apetito no es verdad, y necesito vencerme y luchar conmigo mismo para
pasar cada bocado\ldots{} Me ha ordenado el doctor que coma, y aunque es
para mí un suplicio, lo acepto, porque Dios manda que se conserve la
salud del cuerpo.

---Vamos, otro esfuercito---dijo el capellán de monjas, poniendo un
pedazo de pechuga en el plato, ya dos veces vacío del inapetente
canónigo.

---Carlos, hay que ser juicioso---indicó Baraona.---Genara, te encargo
que no dejes morir de hambre a nuestro heroico guerrillero.

Genara empezó a poner en práctica el encargo, y Carlos dejábase seducir
poco a poco.

---Yo me hago cargo de su tristeza---dijo el secretario de la
Inquisición, a quien los médicos no habían recomendado que hiciese
esfuerzos para comer.---El recuerdo del noble mártir que ha subido al
cielo\ldots{}

---¡Oh, sí!---exclamó Baraona, acudiendo en auxilio del capellán de
monjas, que se había quedado ya sin pechuga y sin lengua.---La imagen
funesta no se apartará de su mente en mucho tiempo, y más vale que sea
así, señores, para que no pierda los bríos ni el indomable furor de
venganza que le impulsa a combatir\ldots{}

---¡Es verdad!

---La muerte de nuestro valiente y caballeroso amigo---continuó el
anciano,---me ha inspirado una idea que voy a comunicar a Vds.

A excepción del capellán de monjas que hacía estudios anatómicos en el
esqueleto de la pava, todos los presentes dieron reposo a los dientes,
para escuchar al respetable patriarca de las montañas alavesas.

---En lo sucesivo, señores---dijo este con grave y profético tono,---y
atendidos los síntomas de discordia civil que presenta España por el
insolente jacobinismo de los negros, los buenos españoles debemos adorar
fervorosamente dos cruces.

---¡Dos cruces!---exclamó Genara.

---¡Dos cruces, sí! La cruz religiosa, aquella en que Dios se dignó
morir para redimirnos del pecado; aquella que desde niños adoramos;
aquella que nos hicieron besar nuestras madres en la cuna, y además esta
otra cruz del sentimiento patrio en la cual ha muerto nuestro buen
amigo, el incomparable, el santo entre los santos guerreros, D. Fernando
Garrote, acompañado del buen cura de la Puebla. Esta cruz que como
instrumento de ignominia han alzado los franceses, los renegados y los
traidores, será para nosotros como la otra, lábaro sagrado que llevará a
la juventud a la gloria. Murió D. Fernando en ella: clavole un clavo la
traición, otro la deslealtad, otro la herejía. Expiró en ella coronado
con las espinas del democratismo, y pusiéronle el Inri de las ideas
jacobinas, que después de todo son las ideas que han traído aquí el
escándalo, y las que aceptaron los afrancesados, y quieren imponernos
los llamados liberales\ldots{} Señores, donde hay mártires, hay
religión; desde que hay cruz, hay fe. Adoremos esa cruz, llevémosla en
nuestro corazón juntamente con la otra, de la cual es como un reflejo;
adorémoslas a las dos, pues las dos deben ser nuestro norte y nuestra
luz. ¡Religión! ¡Patria!---añadió con majestuoso acento, en el cual
vibraba la grave armonía de la inspiración.---¡Sois dos nombres y sin
embargo no sois más que una sola idea, una idea inmutable, eterna, fija
como el mundo, como Dios, del cual todo se deriva! ¡Religión!
¡Patria!\ldots{} ¡Sois dos luces espléndidas, cuyo fulgor no puede
apagarse, ni tampoco cambiar como las chispas de una fiesta de pólvora!
¡Una y otra fe tenéis dogmas eminentes, que la arrogante ciencia del
hombre no puede variar; una y otra fe tenéis la inmutable y permanente
condición del pensamiento divino que os ha creado! Sois lo que sois, y
no podéis ser otra cosa. En vuestro sagrado catecismo la mano audaz del
filósofo no puede hacer la menor variación ni mudar una sola letra.
¡Sois como el firmamento inmenso a donde no puede llegar la mano del
hombre para quitar o poner una sola estrella!

---Bendito sea el insigne patriarca que tales cosas piensa y tales
maravillas dice---exclamó con efusión de sensibilidad y entusiasmo
Carlos Garrote, besando las manos del viejo Baraona.---¡Esas dos cruces,
grabadas están en mi corazón, la una sobre la otra! Me preservaron
contra las armas de los traidores y de los vándalos, y me preservarán
contra toda clase de enemigos.

El capellán de monjas, no pudiendo contener su entusiasmo, abrazó
tiernísimamente a Baraona, y el secretario de la Inquisición abrazó a
Garrote. Aquello era una manifestación general de sentimientos
patrióticos.

---Carlos---dijo Genara al joven guerrillero cuando la borrasca de los
abrazos pasó,---en Vitoria nos dijeron que habías hecho cosas admirables
en la batalla de ayer. Cuéntanos algo de eso.

---Sí, que nos cuente sus heroicidades. También he oído hablar de
ellas---indicó el canónigo.

---Al instante\ldots{} ¡fuera modestia!---exclamó Baraona.

Carlos, por tan distintos ruegos apremiado, trató de vencer su amarga
tristeza, y cediendo principalmente a las súplicas de Genara, que le
cautivaban el alma, empezó a contar varios sucesos del día anterior,
dando la preferencia a los que había presenciado, siendo actor en ellos;
pero al nombrarse a sí propio, lo hacía con gravedad y modestia, no
ensalzando sus propias acciones, sino antes bien rebajándolas para no
aparecer vanidoso. En la relación ponía gran arte, para que se revelara
su mérito sin dejar de ser modesto, y siéndolo, su persona, aparecía en
ellos rodeada de brillante aureola.

Oíanle todos con atención profunda, y Genara con arrobamiento. Fijos sus
ojos en el rostro del guerrillero, parecía que anhelaba leer en él sus
ideas, antes que fueran expuestas por la palabra. El relato fue muy
largo, pero interesante y conmovedor, siendo muy del gusto de todos los
allí presentes, que no perdieron ni una sílaba. El único que no se
mostró excesivamente interesado por las glorias nacionales, fue el
capellán de monjas, que cerrando los ojos con beatífica tranquilidad, se
quedó dormido.

Concluida la patética narración, Baraona habló de retirarse a Vitoria;
pero los demás fueron de opinión que se durmiera la siesta al amparo de
aquella hermosa olmeda, y así lo hicieron los cuatro personajes,
quedándose en vela Genara y Carlos. Largo tiempo transcurrió en
conversación muy íntima y cordial, en la cual parecía haber
confidencias, declaraciones, riñas, arrepentimientos, promesas, y qué sé
yo\ldots{} todos los dulces amargores de un amoroso diálogo. Al fin
despertaron los durmientes, siendo el capellán de monjas el más pesado
para volver en su acuerdo. Caía la tarde y empezaron a recoger todo; mas
aún no se habían levantado, cuando apareció ante ellos una señora de
buena presencia, vestida con heterogéneas ropas, de una manera tan
singular que más parecía tapada que vestida. Su semblante indicaba
zozobra, inanición y reciente llanto. Parecía persona de calidad, y al
punto comprendieron Baraona y sus amigos que era una víctima del día
anterior.

---Señores---dijo,---siendo españoles, no pueden dejar de ser
caritativos\ldots{}

---Así es, en efecto, señora---repuso Baraona.

Y siendo caritativos, ¿tendrán la bondad de darme algo de lo que de su
merienda les ha sobrado?\ldots{} Soy una infeliz víctima del saqueo y
rapiña de anoche, a pesar de no ser afrancesada y encontrarme en el
convoy por casualidad\ldots{}

---Ello podrá ser cierto---dijo el secretario de la Inquisición con
malicia,---pero también podrá no serlo.

---Por casualidad, sí\ldots{} He sufrido el despojo sin culpa---continuó
la afligida dama, llorando.---Soy una persona principal que se ve en la
triste necesidad de pedir limosna para vivir. Allí, tras aquellas cajas
vacías, con las cuales hemos hecho una especie de barraca, está mi
esposo, alcalde de la ciudad de Bailén, cuando la batalla, y mi
amadísimo hermano, seminarista hasta hace poco, y después guerrillero en
las guerrillas del Fraile, hasta que una enfermedad le obligó a
dirigirse a Francia\ldots{}

---¡Oh, señora!---dijo el canónigo,---no es preciso que Vd. nos cuente
la historia completa de sus parientes. Persona principal y decente
parece Vd. Deploramos la casualidad que la ha hecho tan desgracia.
Caritativos somos, y no restos de nuestra comida, sino algo entero que
debe de quedar en la cesta le daremos\ldots{} Genarita, lléveselo Vd.

La dolorida sin poder contener sus lágrimas no cesaba de repetir:

---Gracias, gracias, generoso señor.

---Ya podía esta señora vestirse de otra manera---dijo sonriendo el
capellán al oído del canónigo.---¿No es verdad que tal traje no es
propio para ponerse delante de eclesiásticos?

Genara se levantó para dar a la desconocida cuanto quedaba en la cesta.

---Hija, ve con ella y mira si tienen necesidad de algo de ropa---dijo
Baraona.---Juraría que esa señora ha dicho verdad, y que no es
afrancesada, sino una rancia española\ldots{} Carlos, acompaña a mi
hija.

\hypertarget{xxvii}{%
\chapter{XXVII}\label{xxvii}}

Indudablemente el guerrillero y Genara deseaban pretexto cualquiera para
alejarse un trechito y perder de vista por breve momento al abuelo y
compañeros de mesa. Disimulando su gozo marcharon tras la desconocida;
pero como no tenían prisa de llegar a donde ella iba, la dejaron ir
delante y que se alejase todo lo que quisiera. Principiaba a oscurecer.
Viéndose solos, reanudaron su coloquio con mayor vehemencia al pie de
los olmos, siendo Genara la que con más calor se expresaba. Tomándose
las manos, dejáronse ir vagabundos, abandonados a la dulce corriente que
de sus mismas palabras y de sus propios movimientos se derivaba.

---Genara de mi vida---decía el guerrillero cuando ya llevaban algunos
minutos de paseo, de conversación, de miradas tiernas y de apretones de
manos,---si es cierto lo que me dices, te perdono, y seré para ti lo que
siempre he sido, un esclavo. Día de luto es este para mí, pero si algún
consuelo debo recibir, consistirá en palabras de tu boca, Genara de mi
corazón; mi vida y mi persona te pertenecen. Te adoro desde que te
conocí y te idolatraré hasta la muerte.

---Carlos---repuso la joven con ardor,---si no me crees lo que te he
dicho, me enojaré, me pondré enferma, me consumiré de tristeza, me
moriré de pesadumbre. Carlos, no lo dudes ni un momento. Si bajé aquella
noche a la empalizada de la huerta, fue porque confundí a Salvador
contigo\ldots{} hizo la misma señal\ldots{} No había dicho dos palabras
el traidor, cuando llegaste tú\ldots{} ¿Lo crees, Carlos? Dime que lo
crees, dime que no queda en tu alma una chispa de recelo, y seré la
mujer más feliz de la tierra.

---Bien, Genara---dijo Navarro.---Aunque no fuera verdad, debería
creerlo. ¿Oíste lo que dijo tu abuelo cuando nos encontramos hace poco?
Su deseo era el mismo de mi desgraciado padre, y también el mismo que ha
sido por mucho tiempo y es hoy la más cara, la más dulce, la más risueña
ilusión de mi vida. Dime una palabra y nuestro destino quedará fijado
para siempre, y la noble pasión de mi alma satisfecha, y la elección
suprema de la vida santificada por un leal juramento, ante las miradas
de Dios que desde el cielo nos está mirando y nos bendice. ¿Genara,
quieres ser mi mujer?

Genara contestó arrojándose en los brazos del guerrillero, que la
estrechó en ellos amorosamente. Casi en el mismo instante, ambos jóvenes
hicieron un movimiento de sorpresa y temor. Alguien les miraba; frente a
ellos y a distancia como de cuatro varas estaba una figura delgada y
sombría, un hombre completamente vestido de negro, con la cabeza
descubierta. Después de dar algunos pasos, se detuvo. Tras él veíase una
especie de choza formada por cajas vacías, y en el angosto recinto, de
tal manera formado, clareaba la llama de un hogar y se oían algunas
voces.

---Aquí es---dijo Navarro viendo la barraca.---Entra y da a esas pobres
gentes lo que les traes.

Genara después de dar algunos pasos, lanzó un grito de espanto.

---Navarro, Navarro, defiéndeme---exclamó con angustiosa voz, corriendo
a arrojarse en los brazos del guerrillero y dejando caer en el suelo las
viandas que llevaba.

---¿Quién es, quién va?---dijo Navarro con turbación en el breve momento
que tardó en conocer a la sombría figura que tenía delante.

---Defiéndeme---gritó Genara dando diente con diente;---ese hombre me
quiere matar.

El aparecido no había hecho movimiento alguno. Llegose a él Navarro,
dejando atrás y a regular trecho a la atemorizada joven y le observó con
calma.

---¡Ah!\ldots{} es Monsalud\ldots{} poca cosa, poca cosa\ldots{} No
temas, Genara\ldots{} Esto ni pincha ni corta\ldots{} A fe que no
esperaba verte, Salvador. Creí que habías muerto.

---Hubiera hecho muy mal en morirme---dijo Monsalud,---sin cobrar una
deuda que tengo contigo.

---¿Conmigo?\ldots{} ¡ah, ya!---añadió Navarro flemáticamente.---Cuando
quieras\ldots{} ¿Era para ti para quien pedía esa mujer, llamándote
seminarista y guerrillero del Fraile?

---¿Qué dices?---preguntó Monsalud, ajeno a las jerarquías inventadas
por doña Pepita.

---¡Que eres un farsante, un embustero!---exclamó Navarro perdiendo la
serenidad.

---Sí, un embustero, un farsante---repitió Genara alejándose más.

---Pero observo aquí la mano de Dios---añadió Carlos con
petulancia.---Con tu disfraz y tu cambio de nombre te has ocultado de
todo el ejército, pero no te has ocultado de mí.

---Es verdad---dijo Monsalud con enérgica ira.---Pues aquí me tienes.
Puedes delatarme, denunciarme, llevarme arrastrado por los cabellos a
donde tus salvajes jefes están haciendo cuentas por ver si algún jurado
se escapó de la carnicería de anoche. Yo me salvé; pero ahora te
proporciono ocasión de ganar un elogio, quizás un grado\ldots{} Anda,
llévame; di que me has descubierto, que me has cogido, y quizás te den
un cigarro.

---Si yo fuera tú, te delataría\ldots---dijo Navarro dando un paso hacia
adelante.---Puedes vivir y engañar hasta dentro de un rato\ldots{} Pero
me olvidaba de que te hemos traído de comer.

Navarro, recogiendo del suelo lo que había caído, lo arrojó a los pies
de Monsalud, que no hizo ademán alguno, dando a entender que no recibía
limosna.

---¿Hasta dentro de un rato?---dijo Salvador.---¿Por qué no ahora mismo?

Doña Pepita atraída por las voces, presenciaba la singular escena sin
comprender una palabra; mas no se le ocultaba que allí había peligro
para Monsalud, y llegándose al otro, le dijo con amargura:

---Señor militar, no delate Vd. a mi pobre hermano\ldots{} No, ¿para qué
mentir? no es mi hermano, es mi amigo\ldots{} Es un muchacho honrado y
leal. Ya que escapó, déjele Vd. vivir.

Una figura macilenta y oscura se arrastraba a cuatro pies por el suelo,
semejándose por la oscuridad de la noche a un gran perro de Terranova.
Era el oidor que recogía los restos de la comida.

---¡Yo delatar!---exclamó Navarro.---Señora, esté Vd. tranquila. No
haremos ningún daño a su\ldots{}

---A su amigo---murmuró Genara acercándose al grupo y clavando sus ojos
con ansiedad profunda en el semblante de la desconocida señora.

---No le haremos ningún daño---añadió con ironía Navarro, tomando la
mano de Genara, como para retirarse con ella,---pero el amiguito se
muere de hambre y de miedo: cuídele Vd.

Volvieron la espalda Navarro y Genara. Después de una breve disputa con
doña Pepita, Salvador se separó de esta para seguir a los prometidos
esposos.

---Detengámonos---dijo Navarro a su presunta consorte.---Viene detrás, y
puede herirnos por la espalda.

---¡Pero aquella mujer, aquella mujer!---exclamó Genara apretando los
puños y temblando de ira.---¿La viste? ¿Has oído insolencia igual? ¿Pues
no dijo que era su?\ldots{}

---Su cortejo\ldots{} Salvador es muchacho de muy malas costumbres.

---¡Cuando tal dijo\ldots!---añadió Genara con la exaltación propia de
su carácter en determinadas ocasiones .---¡Oh! Navarro, no tienes
alma\ldots{} ¿por qué no abofeteaste a esa infame mujer?

Baraona y los tres amigos, viendo la tardanza de los dos jóvenes, se
adelantaban a su encuentro.

---Vamos, que es muy tarde. Aprisa, niños\ldots{} ¿qué habláis
ahí?\ldots{} Hombre, ¡como si no tuvieran tiempo de charlar hasta que se
les seque la lengua!\ldots{}

---Aprisita, aprisita---dijo el capellán, arropándose con su
manteo.---La noche está fresca.

---Ya se ve\ldots{} Como ellos están en la flor de su edad y conservan
todo el calor de la vida---murmuró el canónigo con cierta expresión
envidiosa.

Genara y Navarro llegaron al fin.

---¿Qué tienes, hijita?---dijo Baraona advirtiendo mucha palidez y
trastorno en el semblante de su nieta.

---No es nada---replicó Carlos.---Hemos visto escenas muy lastimosas en
la barraca. ¡Cuánta desgracia y miseria en este triste campo, señor
Baraona!

---Sí, lo comprendo; pero la guerra es guerra.

---La guerra tiene que ser guerra, es claro---repitió el capellán.

---Pues claro: ¿qué ha de ser la guerra sino guerra?---murmuró el
canónigo.

---Evidentemente la guerra es y será siempre guerra---añadió el
secretario de la Inquisición.

---Al coche, pronto al coche.

Un vehículo, del cual no se podía decir fijamente si era coche o
catedral, se acercó al sitio donde estaban los amigos.

---Carlos, supongo que no podrás venir con nosotros---indicó Baraona,
subiendo penosamente con el auxilio de un criado.

---Me es imposible.

---¡Ah! no había visto a esa persona que te acompaña, buenas noches,
Sr\ldots---dijo D. Miguel saludando a Monsalud, el cual siguiendo a
Carlos, había quedado a cierta distancia.

---Es un amigo a quien casualmente acabo de encontrar.

---¡Ah! muy señor mío\ldots---dijo Baraona.

---Por muchos años\ldots---gruñó el capellán.

---¡En marcha, en marcha!---exclamó el canónigo.

---Hasta mañana---dijo Navarro a Genara cuando subía y se internaba
dentro de la máquina.---Hasta mañana.

Genara miraba hacia fuera con estupor.

---¿No me contestas? Te he dicho que hasta mañana---añadió Navarro
ofendido de la profunda abstracción de su futura esposa.

---¡Si Dios quiere!---repuso al fin Genara.

Y el monumental coche partió arrastrado por poderosas mulas.

\hypertarget{xxviii}{%
\chapter{XXVIII}\label{xxviii}}

---Ya estamos solos---dijo Navarro a Monsalud.

---Ya estamos solos, y en lugar a propósito---repuso Salvador.---Podemos
alejarnos del camino. La noche está oscura\ldots{}

---¿Qué armas tienes?

---Ninguna. Dame la que quieras.

---Renegado---exclamó Navarro,---estamos en el campo del convoy. Aquí
dejaste tu vestido para ponerte el que llevas, aquí han de estar tus
armas.

---Escondidas bajo tierra---repuso Salvador con desaliento,---pero si me
fuera en ello la vida, no sabría encontrar entre tanta confusión el
sitio donde las pusimos.

---Salvador---gritó el guerrillero con ira,---si de esa manera piensas
evadirte de tu compromiso\ldots{}

---No me insultes, no eches más ignominia sobre mí---dijo Monsalud con
emoción profunda, y antes que colérico, conmovido y sin aliento.---Soy
un desgraciado, el más desgraciado de los hombres. Si no tienes lástima
de mí, guárdame al menos la consideración que merece el
infortunio\ldots{} ¿Me aborreces? ¿Te estorbo? ¿Te soy odioso? ¿Te
molesta que viva? ¿Te mortifica que respire el aire que Dios hizo para
todos? Pues delátame, denúnciame\ldots{} Marcha delante y te seguiré.

---¡Qué miserable cobardía!---exclamó Navarro acompañando sus palabras
de un enérgico gesto.---Si tienes miedo, si quieres renunciar a tu
compromiso, dilo, y no me llames delator.

---Vamos a donde quieras---murmuró Monsalud dando algunos pasos.---Nada
te costará buscarme el arma que más te guste.

---Vamos---repitió Garrote.

Ambos dieron algunos pasos: Navarro, decidido, impetuoso, resuelto;
Salvador, indolente, desmayado\ldots{} Pasaban junto a un árbol próximo
a la cerca del camino, cuando el infeliz renegado apoyó sus brazos en el
tronco y echó la cabeza hacia atrás, diciendo:

---No puedo más\ldots{} me muero\ldots{}

Sus piernas se aflojaron y cayó de rodillas. Ni la energía de su alma,
ni la emoción que en aquel momento sentía, ni la presencia de su enemigo
que renovaba en él odios implacables, podían vencer el desmayo de su
cuerpo, en el cual apenas había entrado algún mezquino alimento durante
cuarenta y ocho horas.

---¿Qué mimos son esos?---preguntó Navarro.

Me muero\ldots---murmuró Salvador.---Si tienes prisa y quieres acabar
pronto, saca tu espada y atraviésame. No puedo vivir; no tengo ánimo
para defenderme.

La extremada palidez y extenuación del desgraciado joven, no se
ocultaron a su enemigo. Navarro comprendió cuán indigno sería provocar a
duelo a un moribundo. Compasivo y generoso, acercose al joven y,
echándole ambos brazos al cuerpo, le levantó.

---Vamos, no has comido hoy---dijo.---Debí empezar por lo primero..,
pues para todo hay tiempo. Ven conmigo.

Monsalud se dejó levantar y conducir maquinalmente, apoyado en el brazo
de su rival. Así anduvieron largo trecho, despaciosamente y sin hablar
palabra. Parecían dos tiernos amigos, dos cariñosos hermanos, de los
cuales el fuerte sostenía y amparaba al débil. Nadie al verlos hubiera
dicho que entre ellos y en torno a ellos, envolviendo sus hermosas
cabezas con fúnebre celaje, flotaba el fantasma horroroso de la guerra
civil. Caía la frente del uno sobre el pecho del otro, se enlazaban sus
manos, se confundían sus alientos; pero no había ni la más mínima
porción de afecto en aquel abrazo de muerte. Quizás el aborrecimiento
mismo impulsaba al fuerte a ser generoso; quizás la propia causa
impulsaba al débil a ser condescendiente.

Llegaron a una gran barraca improvisada con cajas y lienzos, de la cual
salía humo, mucha bulla, y un olor fuertísimo a aceite frito y a
guisotes de campaña. Los dos jóvenes entraron. Soldados y guerrilleros
bebían y comían allí, sin dar reposo a la lengua un solo momento.
Entraban o salían atropelladamente trayendo y llevando víveres y
pellejos de vino.

Monsalud se dejó caer en el suelo, mientras Navarro decía, dirigiéndose
a uno de los más alborotadores:

---Roque, da de comer y de beber a este amigo.

Todos se fijaron en la abatida persona de Monsalud, que parecía
moribundo.

---¿Es jurado?---preguntó uno.

---Es un hermano del cura de Nájera; es mi amigo---repuso Navarro.---Iba
a Francia, cuando tropezó con el convoy y me lo dejaron como lo
veis\ldots{} ¡Eh, Sr.~Soldevilla!---añadió sacudiendo a Salvador por el
brazo,---ahora se pondrá Vd. como nuevo\ldots{} Désele primero un buen
vaso de vino.

---Mejor es un par de tajadas\ldots---indicó un guerrillero que era
riojano y conocía al señor cura de Nájera.---¡Por vida de\ldots! conozco
a todos los Soldevillas de Nájera y de Cameros, y juro que esa cara no
es de ningún Soldevilla de aquella tierra\ldots{} Como que yo conozco
esa cara.

---Y yo también---añadió otro del mismo estambre.

---Y yo.

---Despachaos, pedazos de plomo---gritó Navarro, sentándose
resueltamente al lado de su enemigo, con objeto de evitar cualquier
ofensa que pudiera hacérsele\ldots{}

Para disipar las sospechas de sus camaradas o hacerles entender que
estaba decidido a defender al infeliz jurado, entabló con él familiar
diálogo en esta forma:

---Eso pasará pronto, Sr.~Soldevilla. Buena suerte fue para Vd. tropezar
con un amigo como yo, que le asistiré en cuanto sea menester, y le
protegeré aun a riesgo de mi vida contra todo aquel que intentara
hacerle daño.

---Gracias, muchas gracias---dijo Monsalud, bebiendo con febril ansiedad
en una taza que le presentaron.

---Tengo que comunicar a Vd. una triste noticia, y es que mi excelente
padre, el señor D. Fernando Navarro, amigo de su familia de usted, ha
sido asesinado por los infames renegados.

---¡Asesinado!---repitió sordamente Monsalud, engullendo el pan y las
magras que le dieron.---¡Infeliz suerte!\ldots{} Quizás no moriría de
esa manera.

---Sí; pero los viles que pusieron la mano en aquel hombre insigne no
vivirán mucho tiempo---dijo foscamente Navarro ofreciendo a Monsalud un
vaso de vino.---Revolveré la tierra por encontrarlos, y uno a uno caerán
en mis manos, de las cuales pasarán al infierno.

---¡Al infierno!---balbució Monsalud.---Gracias, gracias, Sr.~Navarro;
voy recobrando la vida. ¡Ah! pero ahora recuerdo\ldots{} oí hablar de su
padre de Vd\ldots{} Sí, antes que cayésemos en poder de los ingleses,
trabé conversación con un joven jurado. Díjome que el Sr.~D. Fernando se
había dado a sí mismo la muerte, por no caer en manos de la vil canalla
que después de sacrificar ignominiosamente a cierto clérigo, le iban a
martirizar a él de la misma manera.

---También me lo han dicho así.

---Y el joven que me habló de este asunto, amigo Navarro, añadió que él
mismo, después de prestar varios servicios al desgraciado don Fernando,
le había suministrado el medio de eximirse, por un acto enérgico, de la
bochornosa muerte que le tenían preparada. Dijo también que el ilustre
señor, vencido de la extenuación y del pánico, perdió en sus últimos
momentos el juicio, cayendo en singulares locuras y manías.

---Tantos detalles no habían llegado a mi noticia---dijo el
guerrillero,---y en cuanto a las palabras de ese renegado que con Vd.
habló, no les doy fe.

---¿Por qué?

---Porque no.

---Es uno que dijo llamarse\ldots{} ¿a ver cómo? ¡Ah! Salvador no sé
cuántos.

---Me lo figuraba\ldots---contestó Navarro con diabólica risa.---Uno de
los que busco\ldots{} y de los que no se me escaparán, a fe mía\ldots{}
Es un reptil que ha querido morderme y que he de aplastar sin remedio.
Traidor renegado, ha hecho migas con los franceses y es uno de los más
crueles sayones que tiene la canalla para atemorizar a las gentes
inofensivas de este país. Embrollón, embustero, farsante y lleno de
fatuidad, atreviose a poner sus ojos en un ángel del cielo a quien
idolatro y que no puede ser sino para mí\ldots{} ¡Oh! nuestra rivalidad
es ya un poco antigua\ldots{} pero se ha recrudecido recientemente,
Sr.~Soldevilla de mi alma, desde que ese miserable ratoncillo que no
merece roer la suela de mis zapatos, se ha atrevido a manchar la buena
fama de la mujer que adoro, engañándola con miserables artes y
obteniendo de ella ciertos favores por el más vil y repugnante
medio\ldots{} Tome Vd. más carne, Sr.~Soldevilla---añadió
presentándosela---tal vez necesite Vd. recobrar todas sus fuerzas para
esta noche\ldots{} Pues sí, como decía, empleando infames medios\ldots{}

---Gracias, gracias, Sr.~Navarro---dijo Salvador rechazando la
carne.---Debe de ser un gran tunante ese joven.

---Como que para hablar con Genara y arrancarle algún honesto favor,
remedaba mi persona y mi voz en la oscuridad de la noche\ldots{}

---No quiero nada más---dijo Monsalud secamente.---Me encuentro bien.

---Poco ha comido Vd\ldots{}

---Lo necesario para afrontar cualquier peligro.

---Pues sí, amigo Soldevilla---añadió Navarro,---perdone Vd. que me haya
exaltado al oírle nombrar persona tan aborrecida para mí. He jurado
matarle, matarle sin piedad, y me parece que mientras él viva me está
robando con su aliento la existencia que Dios me dio para vivir, y el
aire para respirar.

Monsalud, sacudido por viva excitación nerviosa, se levantó del suelo en
que yacía.

---¡Oh! no se levante Vd\ldots{} descanse Vd. más, Sr.~Soldevilla---dijo
Navarro con ironía semejante a la del diablo cuando sonríe a las almas
en el momento de cargar con ellas.---Tome Vd. fuerzas, amigo mío, que
quizás las necesite pronto, sí, muy pronto\ldots{} Si quiere Vd. dormir,
duerma sin cuidado; y por si tuviese recelo de que mis compañeros le
hagan algún daño, esté tranquilo; que no me moveré de su lado hasta que
abra los ojos.

---No quiero dormir---repuso Salvador poniéndose en pie.---Agradezco a
Vd. lo que ha hecho por mí\ldots{} Y ahora que recuerdo, cuando ese
jurado, que antes mencioné, hablaba del trágico fin del Sr.~D. Fernando
Garrote y de su funesta locura, lo hacía con tanta compasión, que
parecía haberse interesado vivamente por él.

---Buen caso haría yo de las hipócritas palabras de ese necio---dijo
Navarro sin disimular su ira.---¡Oh! sólo el oír en su boca el sagrado
nombre de mi padre, me parece un insulto\ldots{} A ver,
Sr.~Soldevilla---añadió tomando el sable de un guerrillero que
dormía,---¿qué le parece a Vd. ese sable?

---Admirable---respondió el jurado pasando el dedo por el filo y
apoyando la punta en el suelo para probar la flexibilidad de la hoja.

---Si no recuerdo mal, me rogó Vd. que le proporcionase un sable.
Quédese Vd. con el que tiene en la mano. Este borracho de Roque es de mi
compañía, y mañana me entenderé con él.

---¡Gracias, gracias!---dijo Monsalud con extraordinaria
animación.---¡Cuántos favores debo a Vd.!

---¿No duerme Vd. un ratito?

---No.

---Es verdad. Tiempo tiene Vd. de dormir---dijo Navarro
levantándose,---sí, de dormir mucho, muchísimo.

Casi todos los guerrilleros que antes había en la barraca, o habían
salido a tocar la guitarra sobre el campo o dormían como troncos.
Monsalud y Navarro salieron. Cuando se hallaban a buen trecho de la
tienda, el renegado dijo a su enemigo.

---¡Navarro, Navarro!\ldots{} Dios que nos mira sabe que no te tengo
miedo\ldots{} Acabas de hacerme un beneficio; mi corazón se oprime al
pensar que puedo darte la muerte\ldots{} Aguarda por Dios, a que te
ofenda de nuevo, aguarda a que esta gratitud se disipe\ldots{} Te
aborrezco; pero un secreto respeto enfría mis rencores, cuando pienso
que nos vamos a batir. A pesar de los horribles insultos que hace poco
me has dirigido, te ruego que esperes, que esperes hasta mañana
siquiera. Creo que debemos esperar.

---Adelante---repuso Navarro con enérgico acento.---No tienes que
agradecerme nada. No te he perdonado, no te perdonaré, si no me
confiesas que fingiste mi persona y mi voz para engañar a Genara.

---¡No lo confesaré porque es mentira!---exclamó Salvador inflamándose.

¡Pues te mataré porque es verdad!---rugió Navarro.---Miserable, ¿piensas
que el hombre que ha hablado a solas con esa mujer puede insultarme
respirando el aire que yo respiro y viendo la luz que yo veo?

---No una, sino muchas veces he hablado con ella---dijo Salvador.

---¡Mientes, bellaco!---gritó Navarro abalanzándose hacia él con el
sable desnudo.---Defiéndete, hijo de nadie, miserable espúreo.

Monsalud sintió que por sus venas corría fuego, que su cerebro era un
volcán. Ciego, loco de ira, se puso en guardia, gritando:

---Defiéndete, salvaje. Mátame; pero antes de hacerlo, sabe que eres un
bandido, y tu Genara una vil mujerzuela.

---Canalla, toma el camino del infierno\ldots{} ¡corre\ldots{}
anda\ldots{} allá vas!

No hablaron ni una palabra más y los aceros chocaron.

Estaban en un sitio solitario, y la noche era oscurísima. Durante breve
rato las dos hojas de acero se rozaron con discorde sonido. De pronto
Navarro dio un grito terrible y cayó al suelo inundado de sangre.

---¡Dios mío!\ldots{} ¡muero!\ldots---exclamó con un rugido en el cual
parecía que echaba el alma.

Y luego con voz expirante añadió:

---¡Padre!\ldots{}

Monsalud hincó una rodilla en tierra y le miró el rostro, sin advertir
que algunos hombres se acercaban.

\flushright{Madrid, Junio-Julio de 1875.}

~

\bigskip \bigskip 

\begin{center} \textsc{Fin de el equipaje del rey José.}
\end{center}

\end{document}
