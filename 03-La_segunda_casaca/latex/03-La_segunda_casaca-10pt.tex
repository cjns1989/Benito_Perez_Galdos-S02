\PassOptionsToPackage{unicode=true}{hyperref} % options for packages loaded elsewhere
\PassOptionsToPackage{hyphens}{url}
%
\documentclass[oneside,10pt,spanish,]{extbook} % cjns1989 - 27112019 - added the oneside option: so that the text jumps left & right when reading on a tablet/ereader
\usepackage{lmodern}
\usepackage{amssymb,amsmath}
\usepackage{ifxetex,ifluatex}
\usepackage{fixltx2e} % provides \textsubscript
\ifnum 0\ifxetex 1\fi\ifluatex 1\fi=0 % if pdftex
  \usepackage[T1]{fontenc}
  \usepackage[utf8]{inputenc}
  \usepackage{textcomp} % provides euro and other symbols
\else % if luatex or xelatex
  \usepackage{unicode-math}
  \defaultfontfeatures{Ligatures=TeX,Scale=MatchLowercase}
%   \setmainfont[]{EBGaramond-Regular}
    \setmainfont[Numbers={OldStyle,Proportional}]{EBGaramond-Regular}      % cjns1989 - 20191129 - old style numbers 
\fi
% use upquote if available, for straight quotes in verbatim environments
\IfFileExists{upquote.sty}{\usepackage{upquote}}{}
% use microtype if available
\IfFileExists{microtype.sty}{%
\usepackage[]{microtype}
\UseMicrotypeSet[protrusion]{basicmath} % disable protrusion for tt fonts
}{}
\usepackage{hyperref}
\hypersetup{
            pdftitle={La segunda casaca},
            pdfauthor={Benito Pérez Galdós},
            pdfborder={0 0 0},
            breaklinks=true}
\urlstyle{same}  % don't use monospace font for urls
\usepackage[papersize={4.80 in, 6.40  in},left=.5 in,right=.5 in]{geometry}
\setlength{\emergencystretch}{3em}  % prevent overfull lines
\providecommand{\tightlist}{%
  \setlength{\itemsep}{0pt}\setlength{\parskip}{0pt}}
\setcounter{secnumdepth}{0}

% set default figure placement to htbp
\makeatletter
\def\fps@figure{htbp}
\makeatother

\usepackage{ragged2e}
\usepackage{epigraph}
\renewcommand{\textflush}{flushepinormal}

\usepackage{indentfirst}

\usepackage{fancyhdr}
\pagestyle{fancy}
\fancyhf{}
\fancyhead[R]{\thepage}
\renewcommand{\headrulewidth}{0pt}
\usepackage{quoting}
\usepackage{ragged2e}

\newlength\mylen
\settowidth\mylen{...................}

\usepackage{stackengine}
\usepackage{graphicx}
\def\asterism{\par\vspace{1em}{\centering\scalebox{.9}{%
  \stackon[-0.6pt]{\bfseries*~*}{\bfseries*}}\par}\vspace{.8em}\par}

 \usepackage{titlesec}
 \titleformat{\chapter}[display]
  {\normalfont\bfseries\filcenter}{}{0pt}{\Large}
 \titleformat{\section}[display]
  {\normalfont\bfseries\filcenter}{}{0pt}{\Large}
 \titleformat{\subsection}[display]
  {\normalfont\bfseries\filcenter}{}{0pt}{\Large}

\setcounter{secnumdepth}{1}
\ifnum 0\ifxetex 1\fi\ifluatex 1\fi=0 % if pdftex
  \usepackage[shorthands=off,main=spanish]{babel}
\else
  % load polyglossia as late as possible as it *could* call bidi if RTL lang (e.g. Hebrew or Arabic)
%   \usepackage{polyglossia}
%   \setmainlanguage[]{spanish}
%   \usepackage[french]{babel} % cjns1989 - 1.43 version of polyglossia on this system does not allow disabling the autospacing feature
\fi

\title{La segunda casaca}
\author{Benito Pérez Galdós}
\date{}

\begin{document}
\maketitle

\hypertarget{i}{%
\chapter{I}\label{i}}

¡Qué infames eran los liberales de mi tiempo! En vez de conformarse a
vivir pacífica y dulcemente gobernados por el paternal absolutismo que
habíamos establecido, no cesaban en sus maquinaciones y viles proyectos,
para derrocar las sabias leyes con que diariamente se atendía al sosiego
del Reino y a hundir a todos los hombres eminentes que describí en la
primera parte de mis \emph{Memorias}.

¡Miserables, bullangueros! ¿Qué volcán os escupió de su pecho sulfúreo,
qué infierno os vomitó, qué hidra venenosa os llevó en sus entrañas? No
os contentabais con aullar en los presidios, clamando contra nosotros y
contra la augusta majestad soberana del mejor de los Reyes, sino que
también, ¡oh, vileza!, agitasteis con nefandas conspiraciones la
Península toda, amenazándonos con un nuevo triunfo de la aborrecida
revolución. Después de insultarnos a todos los que componíamos aquel
admirable conjunto y oligarquía poderosa, para mangonear en lo pequeño y
lo grande, con el Reino en un puño y el Trono en otro, os atrevisteis a
conjuraros con militares descontentos y paisanos inquietos para cambiar
el Gobierno. ¡Trece veces, trece veces alzó su horrible cabeza y clavó
en nosotros sus sanguinolentos ojos el monstruo de la revolución! Trece
veces temblaron nuestras pobres carnes, cubriéndose del sudor de la
congoja y susto que tales tentativas de desorden nos producían. Así es
que, en medio de la privanza y regalo en que vivíamos, se nos podía
ahorcar con un cabello, y al despertar cada mañana, nos preguntábamos si
había llegado ya la hora de bajar del machito.

¡Trece veces, trece conspiraciones! Al ver tal insistencia y la
endemoniada tenacidad de aquella gente, que al pie de los cadalsos donde
expiraba una conjuración, comenzaba a tender los hilos de otra nueva,
cualquiera hubiera creído que el despotismo era la peor cosa del mundo y
que el afligido Reino no se consideraba con vida hasta no sacudírselo de
encima. ¡Embrollones, farsantes, que así desdoraban una institución tan
buena!

No quiero seguir adelante sin contar las abortadas conspiraciones que yo
recuerdo:

1.ª Conspiración para asesinar a Elío y a La Bisbal (1814).---Fue una
intriga misteriosa que unos atribuyeron a los masones y otros a la
corte.

2.ª Conspiración de Cádiz (1814).---Tenía por objeto proclamar la
Constitución del 12 y restablecer en el Trono a Carlos IV, que en sus
buenos tiempos había dado pruebas de muy entendido en aquello del
\emph{reinar y no gobernar}.

3.ª Sublevación de Mina en Navarra (1814).---Abortó a los pocos días.

4.ª Conspiración del \emph{café de Levante} en Madrid (1815).---Andaban
en esto varios afrancesados. Dejáronse coger tontamente, y casi todos
fueron condenados a presidio.

5.ª Conspiración de Porlier en la Coruña (1815).---Esto ya fue un poco
más formal. Frustrose el plan y ahorcaron al \emph{Marquesito}.

6.ª Conspiración de Richard en Madrid (1815).---Fue misteriosa, grave,
atrevida, y la condujeron con destreza sus autores, que eran lo más
perdido de todo el Reino, un comisario de guerra y un sargento de
marina, un soldado y un fraile, diversa gente animada de brutales
deseos. Los angelitos querían asesinar al mejor de los reyes durante su
paseo a las Ventas del Espíritu Santo o en casa de Juana la Naranjera.
La cabeza de Richard estuvo mucho tiempo clavada en un palo en la
carretera de Aragón. Funcionó la horca, y algunos sufrieron un tormento
muy simpático y persuasivo, que se llamaba \emph{los grillos a salto de
trucha}.

7.ª Conspiración del Conde de Montijo en Granada (1816).---\emph{El tío
Pedro} del 19 de Marzo en Aranjuez, había sido después afrancesado en
Bayona, agitador en Cádiz más tarde, y luego absolutista acérrimo en la
Junta de Daroca. Hallándose de capitán general en Granada, dicen que
preparó, ayudado del \emph{Grande Oriente}, las sublevaciones militares
que estallaron más tarde.

8.ª Gran conspiración de Lacy en Cataluña (1817).---Compañías
sublevadas, gritos, entusiasmo, soborno, audacia, traición; y por fin
mucha sangre y un bravo general arcabuceado en Mallorca.

9.ª Conspiración de Torrijos en Alicante (1817).---Proyecto de
alzamiento militar en varias plazas de Levante. La Inquisición se
encargó de castigar a los culpables; pero lo hizo tan mal, que desde
entonces se dijo: \emph{inquisidores y masones todos son unos}.

10. Conspiración de Polo en Madrid (1818).---Se dijo que Polo y sus
amigos deseaban poner en el Trono al venerable Carlos IV. Enviose un
emisario a Roma, y como el solitario Rey no tenía qué comer, no le
pareció mal el proyecto. Militares muy altos anduvieron en estos
enredos, pero descubierto todo, hubo muchas prisiones\ldots{}

11. Conspiración de Vidal en Valencia (1819).---Trama espantosa contra
el tirano Elío. Dios amparó a este y Valencia presenció una horrible
tragedia. La horca y los fusiles la desenlazaron entre lágrimas y
crujido de dientes. En las cárceles no cabían los presos. Para
desahogarlas, fusilaban. La tierra, sedienta, pedía sangre que beber.
Cruzaba los aires pavoroso hálito de odio. Oíanse pasos de gigante. Algo
muy terrible se acercaba.

12. Conspiración del conde de La Bisbal en el Palmar (1819).---Durante
su vida política y militar, el conde encendió siempre una vela al santo
y otra al demonio. En 1814, cuando se dirigía a felicitar al Rey por su
vuelta, llevaba dos discursos escritos, uno en sentido liberal y otro en
sentido absolutista, para espetarle aquel que mejor cuadrase a las
circunstancias. En 1819, después de merendar con los conspiradores de
Cádiz y los oficiales del ejército expedicionario de América, los
arrestó de súbito, haciendo una escena de farsa y bulla, que le valió la
gran cruz de Carlos III. El ejército estaba furioso. Tenía la fiebre
devoradora de la insurrección. Desde Madrid oíamos su resoplido
calenturiento, y temblábamos. En las logias no había más que militares,
infinitas hechuras de aquellos cinco años de guerra, los cuales habían
de emplear en algo su bravura y sus sables. Todo indicaba tormenta.
Cruzaban el negro cielo relámpagos de amenaza. Nos sentíamos en el
cráter de la revolución, y nuestros pies se quemaban. A cada bufido de
la subterránea lava creíamos ver la erupción.

13. Conspiración de los provinciales en Galicia (1819).---Órdenes
falsificadas pusieron sobre las armas las milicias gallegas. ¡Qué
escándalo!\ldots{} ¡hasta las milicias gallegas!\ldots{} Unos echaron la
culpa a los empleados de la Inspección, otros a la Capitanía general de
Galicia. Ello es que hasta los escribientes se creían autorizados para
hacer revoluciones. Cada oficina era un infierno, y un ordenanza
habilidoso, falsificando un sello, ponía con el alma en un hilo al Trono
y al Gobierno. ¡Qué país!

La 14 se verá más adelante.

\hypertarget{ii}{%
\chapter{II}\label{ii}}

¡Qué hombre tan completo era el Sr.~D. Miguel de Baraona! Su gran
patriotismo, su caballerosidad, su fervor religioso, su rectitud, su
entereza, le hacían tan respetable, que era imposible oírle sin
subordinarse con filial sumisión a su voluntad y a su pensamiento.
Merecía muy bien el remoquete de \emph{Patriarca del Zadorra} y yo se lo
daba con frecuencia, para tenerle contento y parecer amable ante él.
Pues ¿y aquella energía moral que desplegaba a los setenta y tantos
años, cuando no podía ni empuñar la espada, ni alzar la voz sin peligro
de estar tosiendo tres horas? Su cuerpo caduco participaba también de
aquel vigor nervioso, más semejante a los tempranos ardores de la
juventud que a las voluntariedades caprichosas de los viejos, y siempre
que se enfadaba o se le contradecía, daba con la trémula mano tan
fuertes bastonazos, que la casa se estremecía.

Otro más celoso por la causa del Rey y por la monarquía absoluta no
nació de madre. En su amor inmenso, en su fervor entusiasta y en su
religiosa devoción por la patria inmutable, no había sutilezas, ni
distingos, ni cabían transacción ni arreglo alguno. Para él la templanza
era traición. Miraba al liberalismo como una especie de horrenda
herejía, más digna aún del fuego que las de Lutero y Calvino. Juntaba la
religión con la política, haciendo de todas las creencias una fe sola o
un solo pecado, y había amalgamado dogmas y opiniones, haciendo un
Evangelio, en el cual Elío no era menos que un apóstol. Comprendía que
el sol se ennegreciera; pero no que sus principios pudieran variar.
Según él, la sociedad estaba perfectamente arreglada tal como entonces
la conocíamos, y constituida por leyes tan inmutables como las del mundo
físico. Discutiendo, no cedía ni una pulgada de su terreno.

---Mis principios---decía,---estos principios que sustento, no son míos,
son de Dios, y no se puede ceder ni un ápice de lo ajeno. La maldad de
los hombres no puede nada contra mis principios. Me vencerá la
violencia; pero no me convencerá el sofisma. La infame revolución podrá
triunfar un día por expreso consentimiento de Dios; pero no porque
triunfe dejará de ser alcázar de pecados fundado sobre la arena de la
traición.

Había venido D. Miguel a la corte a varios asuntos privados y del común.
Era hombre que no se acobardaba ante los desaires de las oficinas; ni
ante la tiesura y desdén de los personajes más envanecidos. Tuvo la
dicha de encontrarme después de dar los primeros pasos en la corte, y
nos entendimos perfectamente. Todo aquello que podía resolverse con
facilidad, fue arreglado entre los dos, sin que jamás frunciéramos el
ceño por palabra ni por peseta de más o de menos. D. Miguel había traído
un bolsón de cuero lleno de onzas de oro, y siempre que echábamos
bendiciones, frotadas las manos con el dorado unto milagroso, se abrían
de par en par las puertas de las oficinas y con ellas el corazón de los
más cerrados covachuelos. Baraona había venido también a estar a la mira
de un pleito de tenuta que no tenía trazas de acabarse en medio siglo.

Acompañaba en Madrid a Baraona su nieta, una tal Jenarita, muy hermosa e
interesante mujer, a quien yo había conocido en mis verdes Abriles en la
Puebla de Arganzón. Era rubia, callada, grave, pensativa, poco franca,
de carácter velado. Su tranquilidad y calma eran como la tenue oscuridad
de los días bochornosos. Ya se sabe que detrás de las nubes está el sol.
¡Aquella hermosura, cuán distinta era de la de mi funesta
Presentacioncita, la risueña asesina, que me ponía ante los ojos las
frescas rosas de su cara para que no viera las aleves manos con que me
empujaba a la muerte! Presentacioncita sin ser hermosa, era lindísima.
Tenía toda la gracia de Dios en sus ojos flecheros, y burlándose de uno,
daba idea de las bromas que deben de gastar los ángeles en el cielo.
Jenara era hermosa como una ideal figura, antes soñada que vista;
hermosa como las creaciones del arte que ha sabido escoger todas las
perfecciones, desechando lo feo. No se burlaba nunca; hablaba
seriamente, como habla la discreción pura, la prudencia suma, la
cortesanía y la urbanidad. Su gracia (pues también la tenía), no era la
desenvoltura picante y alegre de una muchacha juguetona; consistía en lo
que llaman gracia los artistas clásicos, en la perfecta nobleza de los
ademanes y de las palabras, en la armonía sin discrepancias, en el
misterioso ritmo que se desprende de toda la persona y es don rarísimo
acordado a pocos sobre la tierra. Distinguíase además por una expresión
magnífica, tan llena de elegancia como de soberbia. Su fisonomía era
pura, delicada, sin la más ligera incorrección, y su mirar de una
diafanidad celeste. Hermosa hasta no más, se envolvía en una capa de
nieve, bajo la forma de un silencio sistemático, de miradas castas, de
indiferencia hacia la mayor parte de los asuntos y las personas.

En 1815, como dije en la primera parte de mis \emph{Memorias}, vinieron
a Madrid el Sr.~de Baraona y su nieta. Poco después se casó esta con un
joven guerrillero, del cual no puedo menos de ocuparme para disipar las
dudas que acerca de su persona puedan haber corrido. Carlos Navarro,
hijo del nunca bien ponderado D. Fernando Garrote, fue gravemente herido
en un duelo al día siguiente de la batalla de Vitoria. Dejole el fiero
matador sobre el campo, del cual fue al poco rato recogido con más
señales de muerte que de vida, pues la existencia se le iba a borbotones
por la descomunal hendidura que su contrario le había abierto en el
pecho. Largo tiempo estuvo el infeliz héroe suspenso de un hilo sobre el
negro abismo del morir. Los médicos de Vitoria le sentenciaban todos los
días para la mañana del siguiente. Pero la enérgica naturaleza del
enfermo, ayudada por cuidados asiduos, le sostuvieron, hasta que al fin
la aplanada y caída existencia se fue enderezando poco a poco. El
convalecer fue tan largo como la enfermedad, y un año después del
suceso, Carlos Garrote, reconocido coronel del ejército, apenas podía
tener el sable en la mano.

A principios de 1816 vino a Madrid y se casó con Jenara. Vivieron algún
tiempo acompañados de Baraona en la calle de Cosme de Médicis. Pero en
Setiembre del 18, Navarro tuvo precisión de ir a Treviño a asuntos de
interés, y en los días a que me refiero no había vuelto todavía, aunque
le esperaban todas las semanas. No podía haber ocurrido desavenencia en
el matrimonio, porque ambos cónyuges se escribían con frecuencia.
Repetidas veces oí a Carlos renegar de la corte y de los cortesanos,
asegurando que Madrid era para él destierro espantoso más bien que
agradable residencia.

Yo vivía en una hermosa casa de la calle de la Inquisición, esquina a la
Flor Baja, cerca del edificio de la Inquisición de corte y a poca
distancia de los Premostratenses. Mis servicios a determinado prócer
diéronme aquella habitación demasiado grande para un soltero, mas tan
suntuosa, que me acomodé con gusto en ella para aparentar grandeza ante
el vulgo y dar en los hocicos con mi magnificencia a los pobres petates
paisanos míos, que tanto me habían despreciado en mis tiempos de miseria
y nulidad. No me envanecí poco con D. Miguel de Baraona, infanzón y
ricacho alavés, mostrándole mi vivienda; y enamorose tanto de ella mi
venerable paisano, que algunos meses después de la partida de su yerno,
me dijo:

---Pipaón, en esta gran casa vives tú como garbanzo en olla. ¿No te ha
acontecido algún día perderte en sus cuadras y corredores y no poderte
encontrar? En cambio yo estoy muy estrecho en aquella fría y triste casa
de la calle de Cosme de Médicis. ¿Por qué no he de venirme a vivir
contigo mientras llega el día en que, terminado ese maldito pleito,
pueda volverme a la Puebla? Aquí hay espacio para todos, y sin que tú
nos molestes ni molestarte nosotros a ti, podemos acomodamos. Yo pagaré
lo que me corresponda, y si no lo llevas a mal ocuparemos mi nieta y yo
estas hermosas piezas asoleadas que se abren al Mediodía y caen a ese
patio, lindante con el jardín vecino. Aquí estamos muy bien guardados;
por un lado la Inquisición; por otro el Santo Rosario.

Acepté sin vacilar. Lejos de molestarme, me agradaba la compañía, y como
me habían dado la casa sin otro gravamen que algunos censillos y costas
de poco precio, nada más confortativo para mí que sacarle algún jugo,
arrendando una parte de ella. Instalose en seguida Baraona, ocupando una
deliciosa y alegre crujía solana que daba a lugar abierto, y desde la
cual se veían los árboles de un jardín de la vecindad. Yo seguí en las
mismas piezas que antes ocupaba, sin más novedad que la mejor compañía y
algunos gastos menos. Cada cual tenía su servidumbre, y aunque comíamos
juntos contribuíamos separadamente al plato común.

Por las noches, después de la cena, nos reuníamos todos en amena
tertulia, a la cual solía concurrir algún amigo, tal como D. Blas
Arriaga, capellán de monjas, y D. Pedro Retolaza, secretario de la
Inquisición de Logroño, ambos personajes establecidos accidentalmente en
Madrid por motivo de pretensiones y otras cosillas. También nos honraba
alguna vez D. Juan Esteban Lozano de Torres, que era entonces ministro
de Gracia y Justicia, y mi antiguo protector D. Buenaventura, que era ya
marqués.

Allí no se hablaba más que de las conspiraciones descubiertas, de las
que se iban a descubrir y de las que por todas partes descaradamente se
fraguaban. Esta era entonces la comidilla habitual de las gentes en todo
Madrid. Luego que cada cual expresaba su opinión sobre los peligros que
amenazaban a la desdichada monarquía y sobre las probabilidades de que
desapareciese arrastrado por huracanes de traición, pecado y osadía, el
gallardo edificio del gobierno absoluto, se iban retirando los tertulios
y quedábamos solos los de casa, charlando otro ratito, más ocupados de
asuntos domésticos que de la revuelta política. Una noche, luego que
Arriaga y D. Buenaventura se retiraron, Baraona, que había estado harto
pensativo durante todo el tiempo de la tertulia, pronunció, en coloquio
consigo mismo, no sé qué balbucientes expresiones, y golpeando repetidas
veces el brazo del sillón en que se sentaba, se encaró conmigo y me
dijo:

---¡Vive Dios, que si ahora se nos escapa, estos justicias de Madrid
merecerían ser ahorcados al lado de los ladrones a quienes ayudan y
protegen!

Yo le miré interrogándole con los ojos.

---Querido Pipaón---añadió cuando las toses le dieron algún
respiro,---tengo que comunicarte un asunto importante, y espero tu
parecer y con tu parecer, tu ayuda.

---¿Qué ocurre?

---El infame asesino de mi hijo Carlos, del esposo de Jenara, está en
España.

---¡Salvador Monsalud en España!---exclamé.---No lo creo. Por D. Pedro
Ceballos, con quien solía cartearse antes de que este fuera a
Viena\ldots{} (tratos de masonería, Sr.~D. Miguel), por D. Pedro
Ceballos, digo, que es un hermanuco de tomo y lomo, supe hace tiempo que
Salvadorillo seguía en París.

---¡Hace tiempo! No se trata de hace tiempo; se trata de ahora---dijo
con impaciencia.---Es indudable que ese vil trabaja dentro de España en
las tenebrosas conspiraciones que Dios está permitiendo para fines sólo
conocidos de la Sabiduría infinita.

---Puede ser.

---No puede ser, sino que es---dijo repentina y enérgicamente Jenara,
que hasta entonces había permanecido silenciosa.---Yo le he visto.

---¿Le ha visto usted? ¿Luego está en Madrid?

---¡En Madrid, en la corte, en donde está el Trono, el Gobierno, el Rey,
los Consejos, la suprema Justicia!---exclamó Baraona con aquella furia
senil que se desbordaba de su pecho en las contrariedades
graves.---¡Esto es escandaloso!\ldots{} No sé de qué valen las medidas
adoptadas contra los afrancesados\ldots{} ¿Es esto gobierno?\ldots{} ¿es
esto justicia?\ldots{} ¡Ah, Pipaón, aquí están poseídos de necedad! No
persiguen más que a los mentecatos inofensivos y dejan en libertad a los
perversos. ¡Ahorcan a los sargentos y permiten que todos los oficiales
del ejército se vendan a la masonería!

---Monsalud no es oficial del ejército.

---Pero es malo, rematadamente malo, y listo\ldots{} Ahí tienes el
secreto de su impunidad\ldots{} ¡Dios soberano! Ese Rey, esos ministros,
esos consejeros, ¿en qué piensan?

---Descuide usted, Sr.~D. Miguel---dije agitando en mis manos la badila,
después de acariciar la ya moribunda lumbre del brasero.---Si Salvador
está en Madrid, no se escapara.

---Muy pronto lo has dicho\ldots{} Me parece que he de renunciar al más
grande regocijo que ha soñado últimamente mi imaginación desconsolada.
Me moriré sin ver el castigo de un miserable, convicto de los siguientes
crímenes: asesinato, infidencia, herejía, afrancesamiento y traición. La
idea de que ese monstruo naciera en aquella honrada tierra de Álava, que
no ha sabido ser madre sino de hombres eminentes, de caballeros piadosos
y ejemplares campesinos, me enardece la sangre Pipaón amigo. Según todos
los indicios, él dio muerte a nuestro insigne compatriota, a aquel
espejo de la caballería alavesa, e gran D. Fernando Garrote; también
hirió gravemente al hijo de este y mío por los lazos del corazón,
Carlos\ldots{}

---En duelo\ldots---dijo Jenara interrumpiéndole.---Un duelo temerario y
horroroso.

---No fue duelo---afirmó Baraona resueltamente, enojado de la
interrupción.---Aunque Carlos, impulsado por su noble generosidad lo
diga así, y aun sostenga que él le provocó, es mentira, mentira,
mentira\ldots{} Hiriole a traición Monsalud. Cuando el pobre mártir
cayó, apoderáronse del asesino algunos guerrilleros que a la sazón
pasaban. Confesó él mismo su crimen con hipócritas palabras; hizo la
farsa de que deseaba morir conformándose con su destino, y hubiera
perecido, en efecto, al siguiente día, si la diligente protección de una
señora afrancesada no comprara su libertad, primero con ruegos, después
con dádivas; pues todas sus alhajas (que eran muchas y habían sido
ocultadas en el momento de la derrota) las dio por ponerle en salvo. El
criminal se refugió en Francia. Nosotros, deseosos de hacer pronta
justicia, trabajamos porque el Gobierno español lo reclamase al Gobierno
francés; pero nada se pudo conseguir. Allá están tan embobados como
aquí. Respondieron que se ignoraba su paradero. Para averiguarlo,
aprehendimos a la madre del delincuente. Diole tormento la Inquisición
de Logroño, en cuyas cárceles está todavía; pero de los labios de la
infeliz no ha salido una sola palabra que sea luz de nuestra oscuridad,
certeza de nuestra ignorancia. ¡Ah!, Pipaón, mientras no se haga pronta
justicia, mientras no desaparezca este espectáculo de los bribones, que
se pasean impunes por la Península, insultando con sus miradas a la
gente honrada, no tendréis Gobierno firme y respetable. Os ocupáis de
tonterías: de crear cruces, de mudar los ministros todos los meses, de
dictar leyes que no se cumplen. Esto es hacer pajaritas de papel,
mientras el suelo se estremece, mientras la tempestad se prepara y el
volcán ruge. Vendrá la revolución y os encontrará disputando sobre el
color de una venera, o sobre si la Reina está o no está
embarazada\ldots{} En verdad, no sé dónde volveremos nuestras miradas
los partidarios del Gobierno de Cristo, de la verdadera política
cristiana, que tiene por base la justicia. ¡Desgraciado de mí! Cerraré
para siempre los ojos, sin que en la postrera mirada de ellos pueda ver
otra cosa que miseria y debilidades, los buenos patricios olvidados, los
criminales libres, la revolución amenazando o quizás triunfante, los
mayores delitos impunes o quizás premiados, y Salvadorcillo Monsalud
paseándose tranquilo por las calles de Madrid.

Hundió la barba en el pecho y permaneció en silencio largo rato.

---Si está aquí---dije yo, por decir algo,---y mucho lo dudo\ldots{}
pero en fin, si está, es cosa muy fácil averiguar su domicilio y
llevarle a la cárcel. Ya sabe usted que ahora estoy en desgracia y no
puedo nada; pero, sin embargo, intentaré\ldots{}

---Harías la obra más meritoria y más patriótica de tu brillante
carrera, Pipaón---manifestó Baraona con semblante adusto.---Mi nieta y
yo te lo agradeceríamos mucho más que esos mil favores de oficina que
nos hiciste. ¡La justicia! ¡El castigo del crimen, de la traición, de la
herejía, del engaño!\ldots{}

Yo deliro por esto. La justicia sin aplicación no es ni será más que un
ideal vago e inútil. No hay que decir que se encargue Dios de castigar
al criminal, no. Aparte de esto, a nosotros, hombres, nos corresponde no
dar paz a la cuchilla, para que los díscolos aprendan, para que los
buenos teman y los extraviados se corrijan\ldots{} ¿Por ventura habría
llegado a la Tierra de Promisión el pueblo elegido, si Moisés, por orden
de Dios, no hubiera aplicado tremendos y merecidos castigos? ¡Oh! ¡Cuán
hermoso espectáculo dio aquí Su Majestad dictando a poco de su llegada
rigurosas leyes contra los francmasones y liberales! Yo creí que el
pueblo elegido llegaría a la Tierra de Canaán; pero no, ya veo que se
quedará en mitad del camino. Todo es debilidad; las leyes no se cumplen;
cada cual hace lo que más le agrada; son presos los pequeñuelos,
mientras los grandes conspiran; alrededor del Trono alzan su cabeza
enmascarada de sonrisas la traición y la sedición; todos los militares
trabajan sordamente en la masonería. Es esto un constante hervidero de
inquietud, de amenaza, de ambiciones locas que surgen, como los insectos
en el muladar, de la gran escoria del Reino; los magnates se ocupan de
convites y cenas, mientras los masones proyectan comerse a la Nación;
son cogidos algunos criminales conspiradores, y a poco se les suelta;
reina una confabulación espantosa entre los conspiradores y la policía,
entre presos y carceleros, entre alguaciles y alguacilados para taparse
sus respectivas infamias, y hasta la Inquisición, volviéndose tibia y
complaciente, es un cuchillo que se ha hecho alfiler; apenas
pincha\ldots{} Todo es flojedad, enervación, raquitismo, pequeñez. La
Nación que tan enérgica, varonil y potente ha sido contra el extranjero,
es en su vida interior un juego de chiquillos, que juegan en el fango, y
con el fango hacen bolas que se arrojan unos a otros, no para matarse,
sino para mancharse\ldots{} ¡Quiero morirme de una vez, si no he de
vivir más que para ver esto! ¡Los hombres como yo estamos de más en
reuniones de muchachos! El papel de Herodes es difícil, y el de maestro
de escuela, ridículo.

\hypertarget{iii}{%
\chapter{III}\label{iii}}

Dijo, y siguió accionando en silencio durante un rato. Estaba
desasosegado y colérico. La enorme desproporción entre su energía
intelectual y su fuerza física, entre sus ideas y su posición, le ponían
en aquel estado de frenesí, tan semejante a una monomanía furiosa.

---En algunas cosas tiene usted razón, Sr.~D. Miguel---dije.---No se
castiga todo lo que debiera castigarse; pero si ese humor endiablado que
usted tiene se ha de aplacar con la prisión y escarmiento de Salvador
Monsalud, dese usted por curado\ldots{} Hablaremos a Lozano de
Torres\ldots{} aunque sigo en mis trece, y sostengo que ese desgraciado
no está en Madrid. Debe de haber error en esto.

---Está, está en Madrid---afirmó Jenara, clavando en mí sus ojos azules,
cuya serenidad se alteró visiblemente.---Yo le he visto.

Al decir \emph{yo le he visto}, se puso pálida. Su semblante expresaba
más bien miedo que cólera.

---¿Le ha visto usted?---pregunté con incredulidad.

---Hace seis días---dijo poniéndose más pálida aún,---fui a misa a la
iglesia del Rosario, que está aquí cerca. Después de oír misa y de
rezar, me dirigí a la puerta. Estaba oscura la iglesia. Pasaba yo junto
a la entrada de una capilla, cuando sentí más bien que observé la
proximidad de un bulto, de una figura, de un hombre. Llegó hasta mí una
corriente de aire frío, cual si una capa se agitara a mi lado; yo
temblé. Al mismo tiempo, llevadas por aquel aire glacial, sonaron en mis
oídos estas palabras, dichas con marcado tono de burla e ironía: «Adiós,
Generosa\ldots» Me estremecí toda; tropecé en una estera, y ya tocaban
mis rodillas el suelo, cuando una mano me levantó con energía. En el
mismo instante, como levantaron la cortina del cancel de la puerta,
entró alguna luz, y vi a mi lado una cara muy morena, la misma cara.
¡Jesús!

Jenara daba a su relación un interés inmenso. La patética emoción del
drama se pintaba en su semblante.

---Nunca he tenido---añadió,---tan fuerte impresión, no sé si de miedo,
no sé si de ira, no sé si de lástima\ldots{} En término muy breve
experimenté sensaciones diversas, traídas la una por la otra. Temblé,
como si sintiera la mano del Demonio agarrando la mía\ldots{} me pareció
que iba a ser asesinada en aquel mismo instante\ldots{} me pareció que
aquel hombre no era un diablo ni un asesino, sino simplemente un pobre
que me pedía limosna\ldots{} se me representaron uno tras otro los
crímenes de Monsalud, desde su traición a la causa nacional hasta su
duelo con Carlos\ldots{} no vi luego más que desgracia, mendicidad,
hambre\ldots{} ¡y qué cara, Santo Dios!

---¿Le observó usted bien?

---Está más moreno, mucho más moreno que antes. Sus ojos queman; su
boca, al sonreírse con ironía, no sé si sanguinaria o hambrienta,
muestra unos dientes más blancos que el marfil; su aspecto infunde miedo
y dolor. Viste de un modo extraño, anda de prisa, pasa y mira.

---¿Pero le ha visto usted una sola vez?---pregunté, asombrado de tantos
detalles.

Jenara estuvo un rato sin contestar. Luego, mirando al suelo, dijo:

---Una sola vez. Yo corrí para salir de la iglesia. Desde la puerta miré
hacia dentro, y vi que un fraile se le acercó.

---¡Un fraile!\ldots---murmuró sordamente Baraona.---¡Buenos están
también!

---¿Y dice usted que desde ese día no ha vuelto a verle?---pregunté a
Jenara.

Después de vacilar, me contestó:

---No\ldots{} no puedo asegurar que le haya vuelto a ver\ldots{} ni
tampoco que no le haya visto\ldots{}

---¿Cómo es eso?

---Quiero decir que la impresión que en mí produjo aquel encuentro ha
sido tan duradera, que a veces se reproduce ella misma, sin causa
real\ldots{} La imaginación\ldots{}

---Diga usted los nervios. Cuidado con creer en duendes y
apariciones---afirmé riendo.

Después callamos todos, contemplando las menudas ascuas de la copa de
bronce, que mezclándose con la blanca ceniza, lanzaban su último brillo;
existencias que próximas a expirar, dirigían a los vivos su postrer
mirada.

Baraona, Jenara y yo, mirábamos en silencio la moribunda lumbre. Todo
callaba en derredor nuestro. Era la hora en que los espíritus
pusilánimes y los niños suelen tener miedo, y al ir a acostarse
atraviesan corriendo y cantando para ahuyentarlo, los largos pasillos y
las oscuras piezas. Era la hora en que las puertas de algún ventanejo
alto y lejano suelen dar porrazos, estremeciendo la casa y el corazón de
sus habitantes. Era la hora en que el gato trasnochador suele lanzar
lastimeros ayes, que parecen llanto de criaturas o algazara de voladoras
brujas que van por los aires a sus repugnantes asambleas. Era la hora en
que el viento suele ponerse en la boca el tubo de la chimenea, como un
gigante que sopla su bocina, y cantar o decir o refunfuñar alguna
horripilante estrofa, que hiela la sangre en las venas del inquieto
durmiente\ldots{} Los tres nos hallábamos profundamente pensativos,
cuando sonó de improviso en lo interior de la casa inusitado estrépito,
una puerta que se cerró, un mueble que vino al suelo, un golpe, un tiro,
qué sé yo\ldots{} una nada, una tontería, un fútil accidente; pero que
sin duda a causa de la hora y de cierta predisposición de espíritu, nos
estremeció a todos.

---¿Qué es eso?---exclamamos a una vez.

Miré a Jenara. Estaba blanca como el papel, y sus dientes chocaban.

---Es la puerta de mi cuarto que ha dado un golpe. Quedó abierta la
ventana de la calle\ldots---dije yo, tranquilizándome por completo.

Al cabo de un instante me sentaba de nuevo junto al brasero, después de
cerciorarme de la insignificante causa de nuestro pueril miedo. Jenara
seguía temblando; yo me reí, y ella, arropándose en su mantón, dijo:

---Tengo frío.

---Vamos a acostarnos---dijo Baraona levantándose.

Les acompañé a sus habitaciones. Al pasar por la larga galería que las
separaba de las mías y del comedor, observé que Jenara dirigía miradas
inquietas a un lado y otro. La sombra de nuestros cuerpos sobre la pared
atraía sus miradas con más fijeza de lo que una vana sombra merece. Yo
iba tras ellos. Cuando les despedí en la puerta, Jenara me dijo: «Entre
usted». Seguía temblando, y como yo le interpelase sobre aquella
injustificada desazón, no contestaba sino:

---Tengo frío.

Obligome a que registrase su habitación, a que asegurase las puertas,
las cerraduras de las ventanas, y cuando me retiré al fin después de
tranquilizarla respecto a lo innecesario de tales precauciones, echó
llaves y cerrojos por dentro, quedándose acompañada de su criada.

Dirigime a mis habitaciones, sin dar importancia a las voluntariedades
de mi hermosa huéspeda; pero al llegar a mi alcoba y lecho, y cuando me
disponía a acostarme, recibí una sorpresa, una impresión tan fuerte, que
mis carnes temblaron, dieron unos contra otros mis dientes, y me quedé
frío, absorto, mudo, petrificado. Sobre mi lecho y en la misma vuelta de
las sábanas, había un papel escrito. Con trémula mano lo tomé;
recorriéronlo mis ojos en un instante; decía así:

«Infame Bragas: Tú que eres amigo y compinche del Tigre y del Zorro,
podrás conseguir que manden poner en libertad a Fermina Monsalud, presa
y atormentada en la Inquisición de Logroño por supuesto delito de
infidencia. El Elefante trabaja en pro de la mujer inocente. Ha
asegurado que la Culebra, es decir, tú, podrás ayudarle con éxito
seguro.

»Infame Bragas: Si dentro de quince días está libre mi madre, no te
pesará; si no lo estuviere, te acordarás de

\flushright{\textsc{Salvador Monsalud}».}
\justify

\hypertarget{iv}{%
\chapter{IV}\label{iv}}

Juzgad ¡oh amigos!, de mi asombro, de mi anonadamiento. Largo rato
estuve con el papel en las manos sin saber qué partido tomar, sin poder
concretar mis ideas, sin resolverme a dar un paso, ni poder formar un
juicio claro sobre aquel hecho. En mi cerebro bullía el caos. Ocupaba mi
espíritu un miedo horroroso, un miedo cual nunca lo he tenido.

Pasé algún tiempo en dolorosa incertidumbre. Como si tuviera la
conciencia de que mi cuerpo era una masa de apretada aunque suelta
arena, que se iba a desmoronar al menor movimiento; no me atrevía a dar
un paso ni a menear un dedo. Poco a poco fuime recobrando, empecé a
discurrir; me esforcé en atenuar la gravedad del caso, y la curiosidad
se abrió paso en mi espíritu. ¿Quién había traído aquella hoja
amenazadora? El hombre que me escribía, mi camarada antaño, ¿por qué
había ideado tan singular modo de comunicarse conmigo? ¿Era él realmente
o algún chusco desocupado? Y quien quiera que fuese, ¿de qué medios se
había valido para dirigirme tan atroz apercibimiento?

Mi casa no era casa de duendes, aunque muy antigua y grande, propia por
lo tanto para que se pasearan por ella los invisibles habitantes de la
sombra, si el miedo les permitía la entrada. Felizmente yo no creía en
brujerías, ni en chuscadas de duendes, ni en fabulosas correrías de
almas en penas. Ni por un instante pensé en tales puerilidades. Pero al
mismo tiempo yo tenía la seguridad, gracias a un reconocimiento prolijo
que a poco de mi mudanza hice, de que mi casa, con ser de dos puertas,
no tenía comunicaciones novelescas, ni sótanos, ni compuertas, ni
armarios maravillosos, ni escotillones, ni ninguna tramoya de esas que
en el teatro y en los libros dan materia para un sorprendente enredo. No
teniendo, pues, mi casa secreto alguno, era evidente que alguno de los
criados había sido mensajero del extraño mensaje.

Eran tres: el primero, que tenía por nombre Farrancho, servíame de
mandadero, ayuda de cámara y también de amanuense en casos de mucha
urgencia, y era hombre de honradísimos antecedentes, por su cacumen casi
incapaz de Sacramento, pues discurría como una acémila, por su carácter
moral apreciabilísimo al parecer. Jamás le cogí en mentira, ni en hurto,
ni en falta alguna.

La segunda persona de mi servidumbre era una mujer, una venerable
matrona bastante vieja y fea para no incurrir en deslices amorosos,
bastante joven y aseada para servir bien y guisar mejor; Marta por lo
diligente y entendida en cosas domésticas, Magdalena por lo piadosa.
Había servido a monjas durante veinte años, con lo cual dicho se está
que era la prudencia misma, la santidad personificada, la honradez en
efigie. Jamás se ocupó de chismes domésticos, y parecía carecer del uso
de la palabra, como no fuera para emplear ciertas fórmulas piadosas,
pues nunca entraba en mi cuarto sin decir lúgubremente el estribillo
cartujo de \emph{morir tenemos}. Su obediencia era ciega, su solicitud
extremada, su cariño firme y mudo como el de los buenos esclavos, su
arte culinario de plata, su silencio de oro. Hasta su nombre era
admirable de concisión y santidad. Se llamaba Doña Fe.

Había además en la casa otra hembra; pero no me servía a mí (aunque bien
lo quisiera yo), sino a Jenara, de quien era doncella. Paquita, guapa
moza, estaba desde poco antes en casa, y no me eran conocidas las
prendas de su carácter. Parecía excelente muchacha. Mis sospechas
recaían principalmente en ella, después en Farrancho. Doña Fe estaba
libre de toda suposición desfavorable, porque además de tener un
carácter formalísimo, incapaz de toda farsa o enredo, hallábase a la
sazón en cama, molestada de horribles dolores en la cara y oídos.

Después que mentalmente repasé las cualidades de aquel doméstico
triunvirato, recayó mi atención en el asunto principal, en la extraña
hoja que tan a deshora había venido a turbar la tranquilidad de un
hombre de bien, servidor diligente de su Rey y de su patria. Lo más
singular del singularísimo documento era que el autor de él, ya fuese en
realidad Monsalud u otro cualquier pelanduscas de su propio estambre, al
mismo tiempo que solicitaba mi auxilio, me ofrecía su protección, como
parecía indicarlo el \emph{no te pesará}. Pero a renglón seguido me
amenazaba de un modo insolente. El \emph{te acordarás de mí} me ponía en
gran cuidado\ldots{} ¿Sería aquello una farsa ridícula? El que ofrece
protección o castigo es porque tiene poder; y si Monsalud tenía poder,
¿por qué solicitaba mi auxilio?\ldots{} ¿Debía yo despreciar el escrito
o fijar en él toda mi atención?

Pensando en esto, venían a mi memoria recuerdos del ardiente carácter de
mi antiguo amigo; surgía ante los ojos de mi imaginación su figura,
representándomela desmelenada, horrible, teñida de la palidez siniestra
del jacobinismo; volviendo a contemplar el escrito en cuyos caracteres
se conocía la mano de Salvador, y dueño de mi espíritu, el miedo me
sumergía de nuevo en vacilaciones sin fin.

Las palabras del escrito indicaban una resolución firme. Lo que a mis
lectores podrá parecer oscuro y enigmático, para mí no lo era entonces,
por ser común y aun popular el tiznar con viles apodos la persona de
hombres esclarecidos y respetabilísimos, que consagraban su vida al
servicio del Reino. Así el \emph{Zorro}, era D. Juan Esteban Lozano de
Torres, ministro de Gracia y Justicia; el \emph{Tigre}, mi amigo y
protector D. Buenaventura, recientemente convertido en marqués de M***,
y el \emph{Elefante}, D. Ignacio Martínez Villela, consejero de Castilla
y hombre muy metido en Palacio, aunque por entonces corrían voces de que
era masón.

Después de mucho meditar, no repuesto del mortal susto, juzgué que para
requerir a los criados convenía esperar al siguiente día. Acosteme; pero
el sueño huía de mis ojos. No se apartaban de mi mente las anécdotas que
acerca de los masones y su audacia había oído contar últimamente sin
darles importancia; recordé lo que por entonces se decía de connivencias
misteriosas, de sobornos de criados, con otras artimañas atrevidas que
establecían una verdadera mina dentro y debajo de la sociedad.

Yo procuraba determinar algo; pero ninguna resolución definitiva lograba
echar su raíz en mi vacilante y perturbada voluntad. Mi entendimiento
excitado por la vigilia, iba de aquí para allí, entre las revueltas olas
de un mar de ideas, empujado, ya de un lado, ya de otro, sin poder
llegar a ninguna orilla, ni sumergirse en el silencioso y quieto fondo,
que era el dormir y lo que yo más deseaba.

Pero la luz del día ¡bendita sea mil veces!, disipó aquel delirio
caliginoso en que mi pensamiento con angustia se revolvía como un loco
en su jaula. Se me presentó el hecho en proporciones muy pequeñas, y
libre ya del miedo, si no del recelo, tomé dos resoluciones: no hacer
caso del escrito, e interrogar a mis criados para despedir de mi honrado
hogar al delincuente.

Cuando conté el caso a Doña Fe llenose de miedo, trajo al punto de la
iglesia un cantarillo de agua bendita, y roció toda la casa, recitando
exorcismos. La piadosa mujer, hecha un mar de lágrimas al ver el peligro
que mi persona había corrido, me dijo haber visto a Farrancho en la
calle el día anterior, secreteándose con individuos de aspecto tan
revolucionario como heterodoxo, y aunque el tunante protestó y lloró, y
me mojó las manos con la baba de sus hipócritas besos, le despedí. Su
culpabilidad era evidente. Jenara me respondió de la inocencia de su
doncella, y antes hubiera dudado yo de mí propio que de la venerable
matrona a quien tan bien sentaba el nombre de Fe. Baraona quiso
levantarse a deshora del lecho para dar dos palos al infame y desleal
muchacho; pero le contuvimos, y durante un rato Jenara y yo hablamos
vagamente del asunto.

---Yo tampoco he dormido nada en toda la noche---me dijo.

Le pregunté si también había recibido papelito; pero no se dignó
contestarme.

\hypertarget{v}{%
\chapter{V}\label{v}}

El incidente que he referido dejó de preocuparme al siguiente día, y
poco a poco fue olvidado por completo. Salgamos ahora de mi casa y
veamos cómo andaban las cosas públicas en aquellos días, que eran los
últimos de Octubre de 1819, a los once meses de la sangrienta
conspiración de Vidal en Valencia y a los cuatro de los sucesos del
Palmar.

Grandes mudanzas habían ocurrido en la corte desde 1815 a 1819. En tan
breve tiempo Fernando se había casado dos veces, la primera, con Isabel
de Braganza (cuyas bodas concertó en el Brasil Fray Cirilo de Alameda y
Brea, enviado secreto de Su Majestad Católica), la segunda, con María
Amalia de Sajonia, hermosa y desabrida, humilde y bondadosísima, devota
y también algo poetisa. Mientras reinó Isabel, la influencia política de
los criados mermó mucho en Palacio, y este fue lo que debía ser, una
vivienda de Reyes; pero desde Diciembre del 18, en que Dios se llevó de
la tierra a la insigne Princesa, las culebras de la camarilla empezaron
a recobrar su imperio. Sin embargo, ni Alagón ni Chamorro fueron tan
poderosos. Ramírez de Arellano y un tal Villar Frontín, antiguo
escribano del resguardo, eran los que se comían el Reino crudo.

Nueva gente se encontraba en las oficinas, en los Consejos, en Palacio,
y los ministros variaban a menudo; que no es la inconstancia don
peculiar de los poderes constitucionales. En seis años vi bajar y subir
tantos, que casi se pierde la cuenta de ellos. Ceballos se hundió en
Octubre de 1816. D. Tomás Moyano había desaparecido también del
escenario, cayendo en la oscuridad, de donde jamás volvió a salir,
quedando tan sólo, cual muestra de su paternal administración, los mil y
un parientes que en su breve poltronazgo sacó de la miseria y soledad
del campo; D. Francisco Eguía también dejó por algún tiempo al ejército
huérfano de su protección. Hubo un divertido minueto de señores
ministros de la Guerra durante corto plazo, porque a Eguía sucedió
Ballesteros, a Ballesteros el marqués de Campo Sagrado, y al marqués de
Campo Sagrado otra vez el Sr.~Eguía, sin cuya coleta parecía no poder
existir la atribulada Nación. La Marina había perdido a Cisneros, y era
gobernada por Figueroa. Desgraciada andaba la marina en aquellos
tiempos, pues para que su orfandad fuera completa, también perdió en
Abril de 1817 a aquel imponderable terror de los mares, el Infante D.
Antonio Pascual, de quien dijo el poeta:

\small
\newlength\mlena
\settowidth\mlena{Pues falleció el Infante D. Antonio!!!}
\begin{center}
\parbox{\mlena}{\quad ¡Neptuno, Tetis, Céfiro y Favonio,            \\
                Eterno mostrarán llanto abundante,                  \\
                Pues falleció el Infante D. Antonio!!!}             \\
\end{center}
\normalsize

Así terminaba el soneto que al triste suceso dedicó D. Diego Rabadán, el
primero de los poetas de aquel tiempo, Rioja de los líricos y Herrera de
los heroicos, hombre de esclarecido ingenio, gloria de su época, y al
cual la envidiosa posteridad ha tratado injustamente, equiparándolo al
D. Hermógenes de Moratín\ldots{} ¡Como si no fuera la mejor pieza del
mundo aquel célebre soneto en que, para decir que D. Antonio había
muerto de pulmonía, se manifestaba \emph{que el cierzo quiso dar
testimonio de su aridez,}

\small
\newlength\mlenb
\settowidth\mlenb{ arruinando a la España su Almirante!}
\begin{center}
\parbox{\mlenb}{\textit{arruinando a la España su Almirante!}}      \\
\end{center}
\normalsize

No puede darse imagen más hermosa ni entonación más robusta que la de
aquel comienzo:

\small
\newlength\mlenc
\settowidth\mlenc{\quad Ya vencidos de Acuario los rigores}
\begin{center}
\parbox{\mlenc}{\textit{\quad Ya vencidos de Acuario los rigores     \\
                        que aprisionan a líquidos cristales...}}     \\
\end{center}
\normalsize

Pero llevado de mi afición a la poesía y a los buenos poetas de mi
tiempo, me he apartado de lo que estaba tratando, y era, si no recuerdo
mal, los cambios de ministros. D. Felipe González Vallejo, a quien
pusimos en Hacienda, salió como había entrado, es decir, que se lo llevó
un viento cortesano, y el pobrecito con ser tan inocentón y tan para
poco, no se libró del destierro. Entonces era común que a todos los
caídos les recetaran un paseo higiénico para recobrar las fuerzas
gastadas en el servicio de la patria. Sucediole Ibarra, luego López
Araujo, que apenas sabía leer y escribir, y al fin entró el célebre D.
Martín Garay, que más que hombre era una escuela, pues trajo al
Ministerio todo un plan e idea completa para reformar la Hacienda
pública, tarea equivalente a beberse el mar o a ponerse por montera el
Moncayo. Gozaba aquel señor de mucha fama, que aún conserva su nombre;
pero todos los hombres de mi tiempo, desde el Rey y los ministros y el
clero hasta el último zascandil, se pusieron en contra suya, y tuvo que
salir del Ministerio y marcharse con la música y el sistema a otra
parte. Por fortuna no tuvo tiempo de hacer nada de provecho; que si le
dejáramos, capaz hubiera sido de volver la Hacienda del revés, elevando
los ingresos y mermando los gastos. Su sucesor Imas era un bendito.

En Estado, el célebre León Pizarro, amigo y compinche de D. Antonio
Ugarte, no duró mucho tiempo, ni tampoco Irujo, que empezó su carrera
por paje de bolsa de un consejero y la acabó marqués y millonario. El
duque de San Fernando, su sucesor, no fue menos afortunado, porque al
principio de la guerra era soldado raso y en 1818 teniente general,
duque, grande de España y no sé qué más.

En Gracia y Justicia, después del obispo de Michoacán, que fue ministro
veinticuatro horas (¡tanto se emprende en término de un día!) entró y
duraba aún en la época de mi relación, D. Juan Esteban Lozano de Torres,
la gran figura de aquellos tiempos, y no porque la tuviera gallarda ni
aun digna de ser vista, sino porque con su hermosura moral tenía
cautivados a todos, empezando por el Rey. Había sido Lozano de Torres en
su mocedad relojero. No había hecho estudios de ninguna clase, siendo el
primero y el único ministro de Gracia y Justicia lego en jurisprudencia.
Ni siquiera sabía latín, cosa rara y chocante en aquellos tiempos.

La carrera de este benemérito español había sido el comisariato del
ejército. ¡Y qué herejías dijeron de él a propósito de la administración
del hospital militar de la Isla! Con ser tan fuertes, sin embargo, las
especies que acerca del comisario dijo el vulgo, no llegaban, ni con
mucho, a lo que decían los enfermos, un atajo de tunantes que ponían el
grito en el cielo desde que les faltaba caldo. ¡Qué tal fama de
abastecedor y despensero tendría el niño, cuando, destinado a la
Intendencia de Castilla la Vieja, no quiso darle posesión el gran
Wellington, jefe del ejército aliado!

La causa de su elevación a la silla de Gracia y Justicia fue el
desmedido y loco amor que a Fernando tenía, el cual era de tal
naturaleza que raras veces se presentaba ante Su Majestad sin derramar
lágrimas de ternura, y para besarle la real mano hincaba la rodilla en
tierra. Había en el alma de Lozano un sentimiento parecido a la dulce
fibra del misticismo, que le llevaba a la identificación con el objeto
amado, haciéndole partícipe no sólo de las impresiones morales de este,
sino también de sus sensaciones físicas. Cuando Fernando estaba enfermo,
Lozano de Torres se quejaba de la misma dolencia, y si a Su Majestad le
dolía un pie, al punto cojeaba el amigo; tal era la fuerza de simpatía
entre los dos.

Pero cuando el ministro de Gracia y Justicia desplegaba toda la
vehemencia de su alma fervorosa, era cuando la Reina Isabel estaba
embarazada. En cierta ocasión mi hombre celebró en San Isidro por su
cuenta solemne función religiosa y Manifiesto, que había de durar hasta
que Su Majestad saliese de cuidado; y queriendo dar pública muestra de
su amor a la Monarquía, hizo en medio de la iglesia tales aspavientos de
devoción, golpeándose el pecho y desollándose las rodillas ante el
altar, que los fieles no pudieron contener la risa. No quedó sin premio
lealtad tan ardiente\ldots{} ¡pues no faltaba más! Según puede verse en
la Gaceta, Fernando VII dio a Lozano de Torres la gran cruz de Carlos
III, por haber publicado el embarazo de la Reina.

Desde 1815 éramos muy amigos D. Juan Esteban y yo. El pobrecito no
recibía recomendación mía sin que al punto la despachase, y en la
camarilla partíamos un confite, según éramos de tolerantes y
condescendientes el uno con el otro, sin estorbarnos ni quitarnos de la
boca el hueso, como hacían algunos, más semejantes a perros hambrientos
que a cortesanos hartos. Yo no dejaba de prestarle servicios menudos, a
más de los grandes, bien desempeñando ante Su Majestad un papel, entre
Lozano y yo convenido, bien llevándole secretitos y noticias, sabiamente
pescados al vuelo detrás de una cortina.

Conste, ante todo, que yo estaba cesante desde el verano, pues una
cuestión de delicadeza (yo siempre fui muy delicado), obligome a ceder
mi plaza a un sobrino del ministro de Estado; pero se me había ofrecido
el primer puesto que vacase en el Real Consejo. Como la ambición y el
dorado sueño de mi vida eran esta canonjía, la esperaba con viva
ansiedad.

¡Crítico y solemne momento! A fines de Octubre estaba vacante una de las
canonjías del Consejo. Yo tenía derecho a esperar que se cumpliría la
oferta, no sólo por mis méritos personales, que eran muchos, dicho sea
sin modestia, sino porque en repetidas ocasiones y por mediaciones de
ambos sexos, me había prometido la plaza Su Majestad.

Verdad es que las promesas de Fernando eran como los cien pájaros
volando del viejo refrán; ¡pero tenía yo tantos amigos! Como el viajero
que después de larga travesía divisa la ansiada orilla, así estaba yo
cuando divisé la tal vacante. No cabía en mi pellejo de puro angustiado,
inquieto y caviloso. Estudiaba hasta las más insignificantes palabras de
los íntimos de Fernando; atendía a los gestos y a las miradas, porque no
había accidente alguno en que no viese esperanzas de obtener mi
prebenda.

Andaba tan desasosegado que apenas comía. ¡Ay!, si hubieran provisto la
vacante en individuo distinto del que está dentro de esta casaca, me
habría muerto de pena\ldots{} Y verdaderamente, había motivos para que
no estuviese tranquilo, por ser España la tierra de la injusticia y de
la ingratitud. ¿El sin par Colón no murió en el olvido? ¿No acabó sus
días Hernán-Cortés oscurecido en una aldea? ¿Y qué diré de
Cervantes?\ldots{} ¡Vive Dios, que si no me daban la plaza, yo había de
hacer algo sonado; Rey y cortesanos y ministros se habían de acordar de
mí!

Pero últimamente yo tenía en la corte el favor a que me hacían acreedor
mis servicios y adhesión al Monarca. Tocome a mí también un poco de
aquel hálito de desgracia que a tantos había matado y aunque no me
persiguieron ni me desterraron, hallábame en situación bastante
equívoca, ni elevado ni caído, lejos de Palacio, a pesar de que Su
Majestad me enviaba hipócritas recadillos. Yo no podía tragar al
Sr.~Ramírez de Arellano, ni este me tragaba a mí. Supe que se hacían
esfuerzos para desprestigiarme; pero como yo tenía tantos amigos, como
conservaba excelentes relaciones con los hombres más eminentes, no sólo
esperaba defenderme de los que me querían empujar hacia abajo, sino
también recobrar el terreno perdido. Alagón, Ugarte, D. Buenaventura,
Imas, Villela, San Fernando, Lozano de Torres, me tenían en gran aprecio
y me halagaban con fastuosas promesas.

Yo no descansaba. Comprendiendo, como groseramente dice el refrán, que
el que no llora no mama, vivía sobre un pie, de visita en visita, de
conferencia en conferencia, de lamento en lamento, pidiendo a todos, ya
en desnudas ya en artificiosas razones; exponiendo mis méritos, como se
exponían entonces; desacreditando a todo el que estuviese en olor de
candidato; trabajando a lo topo y a lo castor, en la oscuridad y a la
luz del día; armando muchos enredillos y ganando voluntades y levantando
polvaredas de intriga y humaredas de adulación; en fin, practicando todo
lo que un hombre listo practicaba entonces y practica hoy en
circunstancias análogas, que estas viejas mañas son de hoy como ayer, y
primero faltarán garbanzos que Pipaones en España. Oí decir un día que
la vacante se proveería al siguiente. Corrí a ver al Sr.~Lozano en su
despacho del ministerio, y cuando me vio puso cara agridulce, como de
quien sonríe para disimular disgusto. Temblando aguardé mi sentencia.

Lozano de Torres era pequeño y cari-fruncido, con un airoso moñito de
pelo rubio sobre la frente, graciosamente arremolinado. Iba ya para
viejo; sus movimientos eran tardos, sus pasos meditados, y al andar,
colocaba en el suelo con una especie de estudio el blando pie, calzado
con zapato de paño. Poníase ordinariamente muy serio, queriendo de este
modo tomar la máscara de los hombres de saber; pero con los amigos de
confianza, y cuando no se trataban asuntos graves del ramo, era francote
y risueño, mostrando a las claras su alma sencilla y su rústico
entendimiento. Tan declaradamente manifestaba su índole al hablar, que
sólo le faltaba decir: «¡Dios mío, cuán bobo soy!»

Hízome sentar a su lado; ofreciome un polvo, que rehusé; diome después
un cigarrillo, y tras un par de toses, habló de esta manera:

---Querido Pipaón, anoche me habló largamente de usted Su Majestad.
Conviene en la precisión de dar a usted un puesto correspondiente a sus
dilatados\ldots{} a sus dilatados servicios.

---En efecto---repuse;---la última vez que tuve el honor de entrar en la
cámara real Su Majestad me dijo que la plaza vacante del Consejo Real
sería para mí.

El ministro cerró fuertemente un ojo, torciendo con extraño mohín la
boca.

---¿La vacante del Consejo?\ldots---balbuceó.---Sí\ldots{} en efecto; yo
mismo prometí a usted\ldots{} Si de mí solo dependiese; pero\ldots{}

---¿Pero qué\ldots{} pero qué?---dije remedando la perplejidad de
Lozano.---¿Es esto formal? ¿Se puede decir hoy una cosa y mañana otra?
Si se me cree indigno de formar parte de una corporación en la cual han
entrado peluqueros, boticarios y mozos de caballerizas, díganlo de una
vez\ldots{} ¿Por ventura la he pretendido yo?

---No, ya sé que es usted modesto.

---Yo no he pedido la plaza\ldots{} han venido a ofrecérmela, empezando
por el Rey; me han estado pinchando mucho tiempo; me han sacado de mis
casillas\ldots{} Si yo no quiero ser consejero, si no quiero
figurar\ldots{} Por todo el oro del mundo no sacrificaría mi dignidad en
cambio de una posición.

---Vaya, Sr.~de Pipaón, no se amosque por tan poca cosa---dijo el buen
Torres.---¿Por qué no espera usted ocasión más favorable? Siendo usted
quien es, no tardará en ser consejero. Pronto habrá más vacantes.
Aguarde usted unos meses\ldots{} Su Majestad la Reina Doña Amalia estará
embarazada bien pronto. Cuando venga lo que ha de venir, se repartirán
muchas mercedes, sobre todo si es Príncipe\ldots{}

---Señor Ministro---repuse, sin poder contener mi sofocación;---se han
burlado ustedes de mí. Esto no se hace con un hombre que ha prestado
tantos y tan difíciles servicios al Reino, al Rey, a los amigos, a usted
mismo.

---Es verdad, por eso dije que anoche acordamos darle a usted una
recompensa magnífica---afirmó su excelencia melifluamente.

---¿Cuál?

---Puede usted escoger. La Superintendencia de la Moneda en Méjico,
la\ldots{}

---¿Indias, Sr.~Lozano?---exclamé con el mayor desdén.---Ya sabe usted
que no me gusta viajar por mar. Puesto que se me trata de ese modo,
renunciaré a servir en la Administración. Para ir a América y labrarme
en cinco años una fortuna, no necesito que el Gobierno me dé un destino
con visos de destierro.

---Entonces, amiguito\ldots{} Debo advertirle que Su Majestad fue quien
manifestó deseos de que marchase usted a América.

---Es raro---respondí.---La última vez que nos vimos, Su Majestad no me
dio un canastillo de cerezas como a Campo Sagrado, ni un mazo de
cigarros como a Villamil. Yo no pretendí la plaza de consejero; yo no la
quería; yo no di paso alguno para que se me diera; pero me la
ofrecieron: se ha dicho que yo iba a entrar en el Consejo; he recibido
ya las felicitaciones y aun algunos regalos anticipados como previa
acción de gracias por beneficios que no he hecho todavía\ldots{} por
consiguiente, si ahora salimos con que no hay nada, mi situación no
puede ser más grotesca. Mi dignidad, mi honor, indúcenme a no admitir
otro destino que el de Consejero.

---Pues hijo---repuso Lozano, dando un suspiro.---Lo que es eso\ldots{}
La vacante está ya provista.

Y me alargó un papel que tomó de la próxima mesa.

\hypertarget{vi}{%
\chapter{VI}\label{vi}}

---¡Me lo figuraba!---exclamé con indignación, devolviendo la minuta
después de leerla.---El nuevo consejero es el sobrino del marqués de
M***.

¡Bonito nombramiento!

La ira apenas me permitía articular las palabras. Pegajosa saliva
entorpecía mi lengua, y con los crispados dedos arañaba los brazos del
sillón en que me sentaba.

---¡El sobrino del marqués de M***!---repetí.---¡Me lo temía!\ldots{}

---Mañana aparecerá en la \emph{Gaceta}.

---Y mañana sabrá España, ¿qué digo?, sabrá la Europa entera, sí señor,
la Europa entera, cuáles son las prendas, cuáles los antecedentes que se
necesitan aquí para escalar los puestos del Consejo. En primer lugar,
ser jugador, borracho, calavera, no pagar las deudas contraídas, deber
más de tres mil reales en Canosa; y en segundo lugar, no saber más que
un poco de latín, echársela de traductor de Horacio, decir mil
pedanterías a propósito de leyes antiguas, defender malamente algún
pleito de tenuta, criticar en todo, fantasear en la Sala de Alcaldes,
hablar mal de los funcionarios honrados y respetables como usted, y
también tener de brevas a higos algún tratadillo con los masones de
Granada y de Madrid.

D. Juan Esteban alzó los hombros.

---¡Qué personajes, Santo Dios!---proseguí sin que con tanto hablar se
desfogara mi cólera.---Tal sobrino para tal tío\ldots{}

---Silencio---dijo vivamente Lozano.---El marqués de M*** está aquí.

En efecto, sin previo anuncio, porque a causa de su intimidad con el
ministro no lo necesitaba, apareció en el despacho el marqués de M***,
el cual no era otro que aquel famoso personaje a quien puse el nombre de
D. Buenaventura, tapando con esta especie de benevolencia el suyo
propio, para que la posteridad no le mortificase. Fue mi protector, mi
amigo, mi Providencia en los primeros años de mi carrera\footnote{\emph{Memorias
  de un cortesano de} 1815.}. Por esta razón infundíame siempre mucho
respeto, y aunque últimamente solía mostrar cierta envidia de mi rápido
encumbramiento y me molestaba cuanto podía, yo, hombre agradecido, le
ponía generosamente a él como a sus sobrinos, fuera del alcance de mis
artimañas y de mi lengua.

D. Buenaventura, a quien solían llamar el \emph{Tigre} se había hecho
marqués de la manera más sencilla. Nombrado consejero de Hacienda en
1814, hizo en poco tiempo una gran fortuna, comprando fincas que estaban
adjudicadas al crédito público. Por aquellos tiempos, necesitando los
padres de Atocha algún dinerillo para reparar su templo, dioles Fernando
dos títulos de nobleza para que los vendiesen. D. Buenaventura compró en
veinte mil duros el de marqués de M***. Era familiar de la Inquisición,
hombre cruel, y absolutista tan fanático, que se pasaba la vida buscando
masones por todos lados y averiguando picardías de liberales para
contárselas al Rey. Tenía en 1819 gran privanza en Palacio; pero le
hacía sombra Villela, de quien se contaban no sé qué masónicas
liviandades. Conmigo sostenía buenas relaciones, pero a pesar de eso,
solapadamente y sin dejar de halagarme, bebió los vientos para quitarme
la plaza de consejero; y a pesar de lo mucho que me moví, ganome la
partida, como se ha visto.

---¿Se murmura, eh?---dijo amistosamente, después de saludarnos.---Este
diablo de Pipaón no está nunca contento.

---Ya le he dicho que puede esperar mejor ocasión---añadió D. Juan
Esteban, ofreciendo un cigarrillo a su amigo.---Grandes acontecimientos
van a venir\ldots{} Puede que nazca un Príncipe\ldots{}

---Es claro---dijo el marqués, mirándome con sorna.---Pero ¿tú qué
crees? ¿se hacen consejeros a los treinta y seis años? Estos
sietemesinos, apenas dejan el biberón, ya ambicionan los primeros
puestos del Estado\ldots{} ¡qué tiempos, señores!, no sé a dónde vamos a
parar. He aquí un chiquilicuatro a quien saqué de las covachuelas hace
seis años. Le hemos visto subir como la espuma, le hemos ayudado como
buenos amigos, y ahora, ingrato y desconsiderado, todo lo quiere para
sí. Paciencia, amiguito, paciencia y aguardar. Felizmente no estamos en
los tiempos en que el Sr.~Chamorro y Paquito Córdova disponían de los
destinos y sueldos del Reino. Ya los caprichos de una bella no conmueven
la monarquía: ya no caen y se levantan los ministros al compás de la
escoba de los mozos de retrete: estamos en tiempos mejores.

---Las personas han variado, convengo en ello---respondí con
malicia,---pero las cosas no. Entre las ruinas de la antigua camarilla,
eleva su majestuosa frente la negra del Sr.~Villela.

---Silencio---dijo Lozano de Torres.---Le espero de un momento a otro, y
puede venir.

---¿Quién gobierna? ¿Quién aconseja a Su Majestad? ¿Quién empuña el
timón de la nave como generalmente se dice?---proseguí.---Todos sabemos
que si Artieda no tiene el poder que tenía, lo tienen Ramírez de
Arellano y Villar Frontín, pues los ayudas de cámara también caen y se
levantan, como los ministros, aunque sin canastillos de cerezas ni mazos
de cigarros.

---Bueno---dijo D. Buenaventura, riendo.---Sigue tú en la agencia
universal y diplomática de D. Antonio Ugarte. Sigue comprando barcos
rusos y contratando empréstitos. ¿Qué más quieres, pelafustán? ¿Aspiras
también a comprar a los rusos sus barbas, para ponérnoslas a nosotros
después de hacérnoslas pagar?

D. Juan Esteban se reía como un bendito.

---¿Quieres ser consejero?---añadió el marqués.---¿Y para qué? ¿Qué vas
tú a hacer en el Consejo? Sepámoslo. ¿Meditas algún informe luminoso
sobre cualquier materia? ¿Vas a poner en olvido las dotes eminentes de
Jovellanos, Campomanes, D. Arias Mon y demás notabilidades? Para traer y
llevar los recados de D. Antonio Ugarte, para ayudarle en sus negocios,
¿no estás mejor en cualquier oficina que en el Consejo? A pesar de ello,
yo te prometo que te apoyaré decididamente en la primera vacante, ¿qué
más quieres?

---Sé lo que es el Consejo---respondí breve y sentenciosamente;---sé lo
que son las oficinas; todo lo conozco y aprecio en su justo valor, menos
las influencias que imperan hoy, las cuales son de tal naturaleza, que
no sabe uno a qué atenerse.

Me levanté para marcharme. En el mismo instante un portero anunció a D.
Ignacio Martínez de Villela, que no tardó en entrar. Me quedé.

Este venerable señor, uno de los que más trabajaron en 1814 cuando la
persecución de los diputados, era entonces muy influyente en Palacio. Él
y Lozano de Torres y otros que no menciono, formaban a la sazón la
pequeña corte del Monarca, sustituyendo a la antigua, que con gran
trabajo desbancaron y de la cual tuve la gloria de formar parte. Era
Villela, además de corpulento como un elefante, hombre muy vividor, y en
la apariencia grave y respetable, con grandes humos de probo y
justiciero. Oyéndole, parecía que por su boca hablaba el derecho público
y privado. Poseía bastantes conocimientos jurídicos, lo cual le daba
respetabilidad, poniéndole en situación muy favorable; porque desde 1816
y desde la venida de la Reina (que coincidió con el eclipse de nuestra
camarilla), comenzaron a estar en alza los llamados sabios, los
jovellanistas, y los de la escuela de Garay, verificándose un descenso
rápido en el influjo de toda la gente lega y romancista.

Pero la mayor notoriedad del magistrado en cuestión no era su sabiduría,
sino su negra, una tal Doña Inés, ama de llaves y gobernadora de la
casa, de cuya intervención en los negocios públicos se habló durante
mucho tiempo. Habíase captado de tal modo la voluntad de su dueño, que
teniendo este la clave de muchos nombramientos, túvola ella también.
Especialmente las mitras, que se concedían siempre a propuesta del
Consejo, fueron de tal modo monopolizadas por Doña Inés, que esta no
abría la mano sin que saliera de ella un obispo. Había previo convenio y
eclesiástico arreglo antes de que una mitra fuese provista, y era cosa
sabida: ni el más pintado, aunque fuera el mismo San Pedro, empuñaba el
báculo, si antes no se ponía a bien con la tal negra, impetrando y
consiguiendo su soberana gracia. Con este motivo ocurrió más adelante un
suceso curioso que no quiero callar.

Vacó la diócesis de Astorga, y siguiendo los trámites ordinarios, fue
presentado para la silla un sujeto, cuyo nombre no hace al caso. Llevose
el decreto al Rey para que lo firmara, y Fernando, que tenía felicísimas
salidas de aticismo cómico, leyó detenidamente el pliego, sonriendo con
la socarronería que le era habitual. Estaba verdaderamente cargado, como
ahora se dice, de aquella ambición desmedida de la negra de su amigo, y
decidiendo emplear su iniciativa y usar sus prerrogativas con tanta
insolencia usurpadas, no colérico, sino con mucha calma y gravedad, tomó
la pluma y al margen de la propuesta puso estas sencillas palabras, que
constan en un archivo: «Será obispo de Astorga D. X\ldots{} X\ldots.
\emph{y perdone por esta vez Doña Inés}.»

Pues bien, aquel que acababa de entrar en el despacho del venerable
Magistrado era el venerable magistrado, el celoso Juez de 1814, el
Consejero de la Sala de Justicia del Consejo Real, con honores del de la
de Cámara; era el amo de su negra, en fin.

\hypertarget{vii}{%
\chapter{VII}\label{vii}}

---Señores---dijo sin responder a nuestro saludo.---Ocurre una cosa muy
importante. El Sr.~Requena acaba de morir de un ataque de apoplejía
fulminante. ¡Pobre señor, pobre amigo mío! ¡Nos queríamos tanto!\ldots{}
Pero, en fin, puesto que Dios ha querido llamarle a su seno\ldots{} ello
es que con esta muerte hay ya otra vacante en el Consejo.

Yo di un salto en mi sillón.

---¡Una vacante en el Consejo!---repitieron el marqués de M*** y Lozano
de Torres.

---Sí, señores---añadió Villela sentándose;---una vacante en la Sala de
Provincia.

---No podía venir más a propósito---dijo Lozano de Torres mirándome.

---Ahí tienes, Pipaón, ahí tienes\ldots---dijo el marqués de M***.---La
Providencia no abandona jamás a quien confía en ella. He aquí que cae
del cielo una vacante y te toca en la punta de la nariz.

---Poco a poco, señores---dijo el Sr.~Villela de muy mal talante,
mirándome por encima de sus gafas verdes.---No me toquen a esa vacante,
que es para mi primo.

Toda la hiel de mi cuerpo vino a mis labios al oír esto, y era tanto lo
que se me ocurría decir, que no dije nada.

---Tengo promesa de Su Majestad para la primera vacante---añadió
Villela,---y además, amigo Lozano, ¿no hablamos de esto la otra noche?

---Sí, es cierto\ldots---repuso con turbación el ministro;---pero a la
verdad, no sé cómo contentar a todos. Pasan ya de media docena las
personas a quienes Su Majestad ha prometido la primera vacante. Creo que
lo mejor será echar suertes.

---¡Bah!---exclamó Villela con su impaciencia habitual y mirándome de
hito en hito;---¿lo dice usted por Pipaón, que nos está oyendo?
Amiguito, usted es joven aún y puede esperar. En mis tiempos no se
entraba en el Consejo antes de los sesenta años. En los que vivo no he
visto un mozo más favorecido por la fortuna que usted\ldots{} Cuando
mucho se sube, más peligrosa puede ser la caída. Usted se ha encaramado
con excesiva prontitud, y me temo que si no se detiene un tantico, vamos
a ver pronto el batacazo\ldots{} Un polvito, señor marqués; un polvito,
Sr.~Lozano; amigo Pipaón, un polvito.

Describió un lento semicírculo con su caja de rapé, en la cual iban
entrando sucesivamente los dedos de los amigos.

---Sr.~D. Ignacio---repuse yo, aspirando con placer el oloroso
polvo,---admito los consejos de una persona tan autorizada como
usted\ldots{} pero debo hacer una indicación. Jamás pretendí la plaza de
Consejero; pero como se me ha ofrecido repetidas veces y se ha hecho
pública mi pronta entrada en la insigne corporación, sostengo el cuasi
derecho que me da la real promesa.

---¡Oh!\ldots{} usted puede sostener lo que quiera---repuso Villela,
volviendo risueño el rostro y elevando la mano, cuyos dedos sostenían
aún el polvo.---Cada uno es dueño de tener las ilusiones que quiera. Por
eso no hemos de reñir.

---Con perdón del Sr.~Villela---dije yo, inclinándome y poniendo un
freno a mi cólera,---seguiré esperando, que Su Majestad no me ha de
dejar en ridículo.

---Tantas veces han puesto en ridículo a Su Majestad personas que yo
conozco\ldots---indicó el Consejero de la Sala de Justicia, llevándose a
la nariz los dedos y aspirando el tabaco con cierto adormecimiento
voluptuoso en sus ojos ratoniles.

---¡No lo dirá usted por mí!---repuse colérico.

Villela se puso muy encendido.

---Por todos---murmuró.

---Señores, señores, basta de tonterías---dijo el ministro, conociendo
que la cuestión se agriaba un poco.---Basta de pullas. Se procurará
contentar a todos. Esto se acabó.

---Por mi parte, concluido---dijo Villela estirando el cuerpo, arqueando
las cejas, sacudiendo los dedos y tirando de la punta del monumental
pañuelo; para sacarlo del bolsillo.

---Por mi parte, ni empezado siquiera---indiqué yo.

---Háblese de otra cosa---dijo el marqués de M***.

---Hablarán ustedes, porque yo me voy al Consejo---dijo Villela, después
de sonarse con estrépito.

---¿Tan pronto?

---Pero no sin hacer al señor ministro una recomendación. A eso he
venido.

Diciendo esto Villela sacó un papelito.

---Veamos qué es ello.

---Lo primero que pido al Sr.~Lozano de Torres, confiado en que lo
hará---añadió Villela,---es una obra de justicia, es que ponga término a
una iniquidad horrenda, a un atropello impropio de los tiempos que
corren.

---¿Qué?

---En las cárceles de la Inquisición de Logroño---continuó
Villela,---está una pobre mujer anciana, llamada Fermina Monsalud, a la
cual se ha dado tormento para arrancarle declaraciones en la causa que
se sigue a un hijo suyo que vive en Francia. Es mujer piadosísima y a
nadie se le ha ocurrido tacharla de herejía. ¿Por qué ha de pagar esa
inocente las faltas de otro? Si no pueden atar a la rueda al verdadero
criminal ¿por qué se ensañan en la que no ha cometido otra falta que
haberle parido?

---¿Cómo se llama esa señora?---preguntó Lozano, haciendo memoria.---Ese
apellido\ldots{}

---Fermina Monsalud---repuso Villela, guardando el papelito.

---Monsalud\ldots---repitió D. Buenaventura, apoyando la barba en la
mano y haciendo también memoria.

Tuve intenciones de hablar; pero después de un rápido juicio, resolví no
decir una palabra y observar tan sólo.

---Esto es una iniquidad, una brutalidad sin nombre---exclamó Villela,
golpeando el brazo de la silla.---Hablé anoche de ello a Su Majestad y
Su Majestad se escandalizó\ldots{}

El ministro y el Marqués meditaban.

---Pero eso es cosa del Supremo Consejo---observó Lozano de Torres.

---Yo no quiero cuentas con el Supremo Consejo---repuso Villela.---Bien
sabemos todos que este no hace sino lo que le manda el Ministro de
Gracia y Justicia. Haga usted que pongan en libertad a esa pobre mujer,
y cumplirá con la ley de Dios.

---Y con la de los masones---murmuré.

---¿Alguno de los presentes tiene que decir algo en contra de lo que he
manifestado?---preguntó Villela con soberbia.

Nuevamente sentí deseos de hablar; pero el recuerdo de la epístola,
acompañado de cierto miedo, me cortó la voz y callé.

D. Buenaventura no dijo tampoco nada, y seguía meditando.

---Déjeme usted nota---indicó Torres.---Yo veré\ldots{}

El Consejero escribió la nota y la entregó al ministro. Al retirarse,
habló así:

---Tengo gran empeño en ello, Sr.~Lozano, pero grandísimo empeño. Si
consigo arrancar a esa mártir de las garras de los verdugos de Logroño,
me conceptuaré dichoso.

Cuando D. Ignacio Martínez de Villela se fue, alzó de súbito la
meditabunda frente el Sr.~D. Buenaventura, y dando un porrazo con el
bastón, exclamó:

---¡Vive Dios, Sr.~Lozano de Torres, que ya no me queda duda!

D. Juan Esteban reía como un zorro, y graciosamente se atusaba con la
mano derecha el remolino de cabellos rubios que Dios, cual digno
coronamiento de una obra perfecta, había puesto sobre su frente.

---¡Fermina Monsalud!---repitió, leyendo el papel que había dejado
Villela.

---Madre de Salvador Monsalud---dijo el Marqués;---madre del hombre que
anda trayendo y llevando mensajes de los masones; de ese que ha logrado
hasta ahora burlar, con su ingenio peregrino, las pesquisas de la
justicia.

---El mismo---añadió Lozano.---Ese pobre Sr.~Villela\ldots{} Vamos,
parece increíble.

---\emph{Vox populi}, \emph{vox cœli}---repuso el marqués.---Hace tiempo
se viene diciendo que muchos elevados personajes de la corte están en
connivencia con la masonería; hace tiempo se viene diciendo que el
Sr.~Villela\ldots{} Lo que digo: \emph{vox populi}, \emph{vox cœli}.

---Cuando el río suena, agua lleva---afirmó Lozano, que, por no saber
latín, expresaba la misma idea en refrán español.---Para mí hace tiempo
que no es un secreto el francmasonismo de Villela; pero Su Majestad, a
quien D. Ignacio ha sabido embaucar con tanto arte, no consiente que se
le hable de esto, y sostiene que todo lo que se dice de las sociedades
secretas es pura fábula.

---También yo tengo datos para asegurar el francmasonismo del señor
Consejero que acaba de salir---dijo D. Buenaventura.

---Desde que estoy en esta casa---afirmó Lozano,---no ha pasado una
semana sin que haya venido con pretensiones de indulto, de
sobreseimiento o de evasión en favor de algún agitador o revolucionario.

---Y este empeño por que se ponga en libertad a la mamá de ese\ldots{}
Cuando la Inquisición de Logroño le ha dado tormento, ya sabrá por qué
lo ha hecho.

---Pues claro está.

---Salvador Monsalud\ldots{} ¿dónde he oído yo ese nombre?---dijo D.
Buenaventura, procurando recordar e irritado de su fatal memoria.

---Hace días que hablé de él en este mismo sitio---repuso Lozano.---Es
un revoltoso a quien no se ha podido prender nunca.

---Ya\ldots{} si no se puede castigar a nadie---dijo el marqués con
enfado.---Si todos los criminales se escapan, protegidos por estos
señores que afectando servir al trono y a las buenas ideas, son los más
firmes auxiliares de la revolución. No sé cómo Su Majestad protege a tan
pérfidos hipócritas\ldots{} Ya lo he dicho, la serpiente de la anarquía
se agasaja en los mismos cojines del regio solio\ldots{} ¡Y pretende
ahora la nueva vacante del Consejo! Pipaón, o hemos de poder poco, o
será para ti.

Me incliné dando las gracias con lenguaje mudo.

---Es triste lo que está pasando---dijo el ministro.---Prendemos a los
revolucionarios, y los más altos personajes del absolutismo, los más
íntimos amigos del Rey, vienen a implorar que se ponga en libertad.

---Soy familiar de la Santa Inquisición---exclamó con vehemencia el
marqués.---Mi deber es seguir la pista a los criminales. Es preciso
trabajar con pies y manos para que no se nos venga encima la revolución,
¿estamos? Adelante: es urgente desenmascarar a los bribones, poner de
manifiesto las malas artes y la perfidia de los que les protegen.

---Pues señor familiar de la Inquisición---dijo Lozano
sonriendo,---descúbrame usted el paradero de ese Salvador Monsalud;
proporcióneme los medios de cogerle, y yo le respondo de que no se
burlará por más tiempo de los ministros de Su Majestad\ldots{}

---¿Está en Madrid?---preguntó el Marqués.

---Creo que no.

---Está en Madrid---dije yo, rompiendo al fin el silencio.

El Ministro y D. Buenaventura me miraron asombrados.

---No se pasmen ustedes---añadí;---yo no soy masón. Por una casualidad
he sabido que está en la corte ese señor mensajero de los revoltosos.
Hablando con toda franqueza, debo decir que en nuestra primera mocedad
fuimos amigos Salvador Monsalud y yo; pero desde el año 13 no nos hemos
vuelto a ver.

---¿Y cómo sabe usted que está en Madrid?

---Una señora paisana mía, que por desgracia le conoce muy bien, asegura
haberle visto hace días.

---Soy familiar de la Inquisición---repitió gravemente D.
Buenaventura:---y como tal tendría un gozo vivísimo en poder echar mano
a un propagador del jacobinismo y de la herejía\ldots{} ¡Ah, Pipaón, si
tú quisieras ayudarme!\ldots{} ¿Dices que le conociste en tu juventud?

---Somos paisanos.

---¿Y qué tal hombre es?

Me llevé el dedo a la frente para indicar ingenio.

---Sí, debe de ser listo\ldots{} pero un tunante, ¿eh?

---Sirvió al Rey José.

---¡Afrancesado!

---¿Y tú respondes de que está en Madrid?

---Respondo.

---Ha demostrado en las últimas conspiraciones un atrevimiento y una
constancia que confunden---dijo Lozano.

---Vamos, es preciso cogerle aunque no sea sino por dar en los hocicos
al masón vergonzante Sr.~Villela que le protege\ldots---dijo el
marqués.---Pipaón, ¿me ayudas o no?

---Ayudo.

---Soy familiar de la Inquisición; pondré de mi parte cuanto pueda. ¿No
hemos visto a los más insignes hombres de la nobleza, a los Medinacelis
y Albas y Osunas saltando de tejado en tejado, en calidad de alguaciles
mayores del Santo Oficio, para perseguir a los criminales?

---Voy a dar a ustedes un resumen de las fechorías de ese salvador
Monsalud---dijo Lozano de Torres, tirando de la campanilla.---Los
corregidores y las audiencias han suministrado algunos datos, los
cuales, unidos a los informes que tomé en el ministerio de Seguridad
pública, forman un curioso expediente.

Se presentó un oficial de secretaría, el cual, por indicación de Lozano,
trajo poco después un grueso legajo.

---Se cree que tomó parte en la conspiración de Richard para asesinar a
Su Majestad---dijo Lozano fijándose en el primer pliego.

---Se cree\ldots{} eso es; y debe de ser cierto---indicó D.
Buenaventura.---No puede menos de ser cierto.

---Viósele en Granada el año 16---continuó Lozano leyendo,---y al poco
tiempo estuvo en Murcia y Alicante, donde le protegían López Pinto, el
brigadier Torrijos y algunos oficiales del regimiento de Lorena.

---Esta fue la conspiración del regimiento de Lorena, que abortó por
fortuna\ldots{} Ojo, señores. Por empeños de Villela fueron puestos en
libertad los conspiradores.

---El año 17 estuvo en los baños minerales de Caldetas, donde pasaba por
criado del malogrado Lacy, y el 5 de Abril salió de Tarragona con las
dos compañías de Quer. Desapareció en Arenys de Mar.

---Desapareció\ldots---dijo con enfado D. Buenaventura.---Si no
existiera esta sorda y astuta confabulación de todos los pillos, no se
habría evaporado tan fácilmente.

---Volvió a aparecer en Gibraltar, visitando la casa del judío Benoltas,
que dio dinero para la sublevación de Alicante---continuó Lozano,
hojeando los papeles.---Después se le vio en Murcia muy unido a Romero
Alpuente y a Torrijos; pero cuando este fue descubierto y preso, el
otro\ldots{} desapareció.

---¡Desapareció!\ldots{} Lo de siempre.

---Pero al poco tiempo se le vio en Madrid, donde los masones de Murcia
tenían tan buenas aldabas. Sostuvo relaciones epistolares con D. Eusebio
Polo y con Manzanares, oficiales de Estado Mayor, y otros muchos
militares distinguidos que están afiliados en la masonería. Cuando estos
fueron reducidos a prisión, se pudo echar mano al Monsalud; pero al poco
tiempo de encierro\ldots{}

---Desapareció. Ya sabemos lo que son esas desapariciones---afirmó
colérico el familiar de la Inquisición.---Los Hermanos del Grande
Oriente han tenido buen ojo en la elección de sus venerables. Son estos
algunos señores de la grandeza, generales y consejeros como Villela.

---Reapareció en Valencia---prosiguió Lozano,---a principios de este
año. Trabajó con don Diego Calatrava en los preparativos de la
conspiración de Vidal. Frustrada esta, fue herido gravemente y preso con
otros muchos. Llevado a la cárcel en camilla, se le encerró en un
calabozo, donde era imposible la evasión. Cuando fueron a sacarle para
conducirle al patíbulo, encontraron en su lugar\ldots{}

---¿Qué?

---Un muñeco vestido con sus ropas.

---Esto es burla\ldots{} Pero sea lo que quiera, Pipaón ha dicho que el
\emph{desaparecido} está en Madrid.

---Así me lo han asegurado---repuse.---Creo que podemos saberlo con toda
certeza.

---Soy familiar de la Inquisición, y tú, Pipaón, un hombre listísimo. Si
de esta vez no hacemos algo de provecho, tengámonos por dos alcornoques
de tomo y lomo.

---Pero si hacemos algo, mi Sr.~D. Buenaventura---dije,---que sea para
desenmascarar a un magistrado tan corrompido como el señor Villela.

---Vamos---repuso riendo,---a ti lo que te escuece es la vacante de
consejero que Villela se quiere apropiar, caliente aún el cuerpo del
Sr.~Requena. Por mi parte te juro que aborrezco a Villela. Siempre he
visto en él un hombre tan astuto como peligroso, que está sirviendo a la
revolución.

---Ya se lo dirán de misas. Soy\ldots{}

---Cójame a ese Monsalud, Sr.~D. Buenaventura---dijo el
ministro.---Vamos, ¿a que no se atreve?

---¿Que si me atrevo? Pipaón: vete por casa mañana. Hablaremos.

---Pues hasta mañana, señor marqués.

---No hay más que hablar.

\hypertarget{viii}{%
\chapter{VIII}\label{viii}}

Veamos lo que pasaba en mi casa. Detenido en ella el Sr.~D. Miguel de
Baraona por ciertos achaquillos en las piernas que no le permitían
zarandearse en paseos y cafés, mataba el aburrimiento escribiendo cartas
o perorando, si por mi desgracia lograba echarme el guante. Jenara hacía
vida muy distinta. Menos ocupada que antes en sus labores de mano, salía
a la calle con alguna frecuencia, pasando largas horas fuera. Todo
revelaba en la hermosa Jenara que traía entre manos un asunto
importante, asunto de verdadera acción que requería tanta actividad como
cavilaciones. No tuve que hacer grandes esfuerzos para descubrirlo,
porque ella misma me lo reveló todo una noche junto al brasero, después
que Baraona se recogió en su cuarto.

---¿Ha averiguado el Gobierno---me preguntó,---el paradero de Salvador
Monsalud? ¿Sabe que está conspirando?

---El Gobierno, señora---le respondí,---lo sabe todo y no sabe nada;
mejor dicho, sabiendo que se conspira a más y mejor, es completamente
incapaz de descubrir y más aún de castigar las conspiraciones.

---¡Qué Gobierno!---exclamó Jenara.---Bien dice mi abuelo que estos que
hoy mandan son como los muñecos que se ponen en el campo cuando se acaba
de sembrar: espantan a los pájaros, pero no a los hombres. Diga usted
que sabe tanto---añadió con jovialidad,---¿por qué no se habían de
encargar a las mujeres ciertas cosas del Gobierno?

---Porque no. Ahí están Catalina de Rusia, Isabel de Inglaterra y otras,
que gobernaron a sus pueblos\ldots{}

---No, no es eso lo que digo. Gobiernen a los pueblos los hombres; lo
que, según mi entender, podía confiarse a las mujeres, es un trabajo
menudo y que no requiere ciencia de libros; por ejemplo, el
descubrimiento de las conspiraciones.

---En Francia dicen que hay muchas mujeres empleadas en la policía
secreta.

---Las mujeres---dijo Jenara con gravedad y gracia,---son más leales que
los hombres, sirven con más ardor y más honradez a una causa cualquiera,
son menos accesibles a la corrupción, poseen instinto más fino y mayor
agudeza de ingenio, mayor penetración. Ustedes piensan; nosotras
adivinamos.

---Es verdad; ustedes adivinan---dije con mucha sorna.---Vamos a ver:
¿ha adivinado usted el paradero de Salvador Monsalud?

---Sí señor---repuso mirándome con fijeza, y sonriendo vanidosa y
triunfalmente.---Sí señor; lo he adivinado, lo he descubierto, lo sé.

---¿Pero es broma, es sospecha o presunción?\ldots---pregunté lleno de
asombro.

---Es certidumbre, Sr.~D. Juan.

---¡Es usted un tesoro, es usted una diosa, Jenara!---exclamé con
entusiasmo.---Pero dígame usted: esas salidas diarias, esa multitud de
recados, esa ocupación constante durante más de una semana, ¿se han
consagrado al servicio de la patria y del Rey? Me parece inverosímil.

---Si he de hablar con verdad, no he atendido gran cosa al servicio de
la patria y del Rey\ldots{} He tenido fijo el pensamiento en mi esposo,
acuchillado y moribundo.

---Verdad es que la persona a quien queremos castigar ha sido por mucho
tiempo la pesadilla y el espantajo de su familia de usted.

---Yo no sé hacer nada a medias---dijo Jenara con solemne voz.---Me
impulsaba a dar estos pasos un sentimiento que inflama mi corazón, un
sentimiento criminal que ofende a Dios, lo sé; un sentimiento\ldots{}

---¡Jenara!

---Sí, Sr.~de Pipaón, el odio; hablo del odio que se ha fijado en mí
desde hace algunos años como un puñal que me atraviesa el corazón.
Incapaz de tranquilidad, escandalizada de la debilidad de los hombres,
que han dejado sin castigo a tan grave criminal, me he lanzado
resueltamente y con todo el ardor de mi carácter a un trabajo impropio
de mi sexo y condición. He desfallecido muchas veces, he sufrido grandes
sonrojos; pero al fin la fuerza de mi propia pasión me ha dado energía,
y con la energía una luz extraordinaria. ¡Qué no conseguirá la voluntad
de una mujer, su penetrante instinto, su admirable sagacidad!\ldots{}

---Esas prendas, señora, han revuelto el mundo muchas veces, han
provocado guerras y revoluciones---dije contemplándola fijamente, por
ver si descubría cuáles eran las verdaderas ideas y los sentimientos
efectivos de Jenara en aquella ocasión.

No era fácil averiguar esto, y en vano clavaba mis ojos en la marmórea
beldad que ante mí tenía. Por experiencia sabía yo que respecto al
conocimiento del alma de Jenara, era preciso atenerse a lo que decían
sus labios, dejando al tiempo o al acaso la misión de describir el color
y los astros de aquel cielo siempre cubierto de nubes. Al mismo tiempo
no podía hacer grandes observaciones fisiognómicas, porque mis ojos, lo
mismo que mi atención, se distraían con el recreo y embobamiento que tan
grande hermosura les producían. ¡Lástima grande que bajo aquella
serenidad majestuosa, aunque algo artificial como los papeles del
teatro, se escondiese, cual serpiente en nido de rosas, el odio tan
ponderado verbalmente por ella!

---Si es cierto---dije,---que merced a las averiguaciones que ha hecho
usted, como principal agraviada, se logra descubrir y capturar a ese
hombre, el Estado y el Rey están de enhorabuena. Precisamente nuestro
amigo el Sr. Lozano bebe los vientos por ponerle la mano encima. ¿Pues y
D. Buenaventura?\ldots{} Poco contento se va a poner cuando yo le
diga\ldots{} Como que nuestro paisano es el alma y la clave de las
conspiraciones. Parece mentira que una señora haya conseguido lo que
intentaron hasta ahora en vano tantos y tan buenos espías\ldots{}

---¡Espías! Los de la Inquisición, lo mismo que los del Gobierno, están
vendidos a los masones---afirmó Jenara con desprecio.

---Cuénteme usted todo; cuénteme esos prodigios.

Ella sonrió, y por breve rato puso los ojos en el brasero, sin dejar la
sonrisa que parecía esculpida en su rostro.

---Si le contara a usted todo lo que he hecho---dijo al fin,---se
asombraría de algunas cosas y de otras se reiría, formando mala idea de
mí.

---Vamos a ver.

---Es preciso hacerse cargo de la impresión que produjo en mí la vista
de ese hombre en la iglesia del Rosario, para comprender las locuras que
he hecho. Yo estaba aterrada; parecía que me apretaban el corazón con
tenazas de hierro; yo no podía dormir; la terrible imagen iba tras de mí
a todas horas, infundiéndome miedo y una congoja extraña.

---Lo conocí.

---Yo presagiaba toda clase de males; atribuía a ese hombre un poder
maléfico; tenía un desasosiego inexplicable. Era tal mi turbación y lo
preocupada que yo vivía, que una noche creí verle deslizarse por esos
pasillos como un fantasma.

---¡Jenara!

---Sí; la imaginación me lo puso delante\ldots{} ¡y con cuánta verdad!
Vi su cara, sentí el ruido que hacía su capa rozando en las
paredes\ldots{}

Yo me quedé frío.

---Pero no\ldots{} no se asuste usted\ldots{} yo no creo en fantasmas.
¡Cosas de mis ojos, que suelen ver lo que no existe!\ldots{} Ya me ha
pasado lo mismo otras veces\ldots{} Ello es que la propia exaltación mía
me dio fuerzas para sobreponerme al miedo, a la congoja, y furiosa me
revolví contra mi atormentador. El placer de castigarle, de hacerle
sentir el peso de una mano justiciera dirigida por mí, dio mayor fuerza
a mi voluntad. ¡Era preciso buscarle, burlar su astucia, sorprenderle,
cogerle, destrozarle!

---Veamos lo que hizo usted.

---Desde luego, sabiendo que ese hombre estaba en Madrid parecía natural
creer que vivía en alguna parte.

---Eso no tiene la menor duda.

---Yo pensé de otra manera; yo pensé que viviría en muchas partes.

---Ya\ldots{} es decir, que cambiaría todos los días de domicilio para
desorientar a sus perseguidores.

---Justamente. Pero esta idea tenía poco valor, mientras no se
averiguase una por lo menos de las guaridas del miserable. Empecé sin
resultado mis pesquisas, cuando de repente vino en mi ayuda la
casualidad, proporcionándome un nuevo encuentro con él cierta noche que
volvíamos a casa Paquita y yo un poco tarde.

---¿Y le habló a usted?

---¡Qué disparate! No me conoció: yo sí le conocí perfectamente, a pesar
de que iba embozado hasta los ojos.

---¿Y dónde fue ese encuentro?

---En la calle Mayor. Eran las nueve. Él iba en dirección a la plaza de
la Villa. Paquita y yo veníamos de casa del Sr.~Grima, corregidor que
fue de Vitoria.

---Y usted y Paquita, llenas de terror, avivaron el paso para huir de
él.

---Al contrario, volvimos atrás\ldots{} y le seguimos.

---¿Le siguieron?

---Sí, señor. Nos arrebujamos muy bien en nuestros mantones y le
seguimos a cierta distancia. Como él anda tan aprisa, llegamos sin
aliento a la calle de Santiago.

---Donde se escurrió por algún portal, y aquí paz y después gloria.

---Entró, sí, en una casa; pero yo no me desconcerté por eso, y con toda
serenidad examiné el edificio detenidamente. Era un palacio enorme,
pesado y triste, con grandes balcones y un escudo formidable sobre el
del centro. Parecía la vivienda de un Grande de España, y Monsalud, al
entrar en ella, iba a visitar a alguien; de ningún modo a quedarse allí.

---Muy bien pensado; pero las casas de los grandes, sobre todo si los
que las habitan no son muy grandes, suelen tener bohardillas que se
alquilan a gente pobre, y a las cuales se sube por la escalera de
servicio.

---También pensé yo esto---dijo Jenara demostrándome su prodigioso
método de raciocinio;---y para salir de duda me decidí a preguntar al
portero.

---Lo que no dejaba de ser aventurado y sospechoso.

---No me importaba: yo entré resueltamente y dije al portero: «¿Vive en
las bohardillas de esta casa una pobre viuda enferma, llamada Doña
Petra, que ha puesto un anuncio en el \emph{Diario}, pidiendo una
limosna a las almas caritativas?» El portero me informó de lo que yo
quería saber, diciendo: «En esta casa no hay bohardillas alquiladas, ni
aun vivideras, ni aquí vive nadie más que mi amo el Sr.~Conde\ldots» Ya
estaba segura de que Monsalud no vivía allí y de que más tarde o más
temprano saldría. Paquita y yo nos llenamos de paciencia, y aguardamos.

---¡Qué valor, qué constancia sublime!\ldots{} En una noche fría\ldots{}
dos mujeres solas en la calle.

---Nadie se metió con nosotras. Antes de las once Monsalud salió.

---¿Y le siguieron ustedes?

---Le seguimos. Él miraba atrás algunas veces; pero viendo transeúntes
indiferentes o mujeres, seguía tan tranquilo.

---¿Y fue larga la segunda caminata?

---No muy larga. Entró en el café de Levante, pero no por la puerta del
local público, sino por otra lóbrega y estrecha que hay al costado y por
la cual creo se sube a la tertulia.

---Así es en efecto. Supongo que no entrarían ustedes en el café ni
aguardarían tampoco la salida del aventurero, porque tales garitos no se
vacían hasta la madrugada.

---Entrar no; pero aguardar sí---me contestó con una serenidad que me
dejó pasmado.---En aquella acera, que es de gran tránsito a causa de las
puertas de los cafés cercanos, hay muchas mujeres y chicos que piden
limosna, castañeras, ciegos que venden villancicos, y también muchos
rateros y gente sospechosa, con la cual alternan en amor y compaña los
alguaciles. Paquita limpió el lodo junto a la puerta por donde él había
entrado y por donde esperábamos que saliera, y\ldots{}

---¡Jesús, María y José!---exclamé interrumpiéndola:---¿fue usted capaz?

---Sí señor; nos sentamos allí---repuso con la mayor naturalidad del
mundo.---Con los mantos sobre la cabeza, no nos diferenciábamos gran
cosa de la sociedad allí reunida\ldots{} Yo no me acobardaba ante ningún
obstáculo. Resuelta a marchar derecha a mi objeto, llena y encendida
toda el alma con la llama de un aborrecimiento que era mi sostén y mi
martirio, no reparaba en dificultades. Sólo así se vence, Sr.~Pipaón.

---¿Y hasta cuándo duró la guardia?

---Hasta las cuatro de la mañana. Fue aquella noche que estuve fuera de
casa. ¿Se acuerda usted? Entré por la mañana diciendo que había estado
acompañando a una amiga parturienta.

---Me acuerdo, sí.

---Hasta las cuatro, sí. Nos levantamos de allí medio heladas---continuó
riendo.---Él salió con otros tres; marchó hacia la calle Mayor. A la
entrada de la de Boteros, uno de ellos se separó, y Monsalud con los dos
restantes entró en la plaza. Les seguimos a bastante distancia; pasaron
a la calle de Toledo y pasamos también nosotras. Detuviéronse en la
esquina de la calle Imperial, y entonces resolvimos adelantarnos y pasar
junto a ellos para que no sospecharan que les seguíamos. Cuando pasamos
oí claramente la voz de Salvador, que decía a sus compañeros: «Estoy muy
fatigado, y me voy a acostar\ldots» Siguiéndole, pues, hasta el fin, era
seguro que sabríamos dónde vivía.

---¡Qué admirable paciencia! El más astuto y diligente alguacil no haría
otro tanto.

---Esto no puede hacerlo la justicia que es mercenaria y venal; lo hace
una mujer.

---¿Y dónde vivía?

---En la calle de Segovia. Detúvose en una puerta, y después de dar
varios golpes, bajaron a abrirle y entró.

---Dando fin con esto a las investigaciones de usted, pues no
creo\ldots{}

---No entramos\ldots{} ¡qué disparate! Pero examiné cuidadosamente la
casa. En los balcones del piso segundo de ella había los papeles que
suelen ponerse en las casas de pupilos. En la parte exterior del portal
vi una muestra que anunciaba lo siguiente: \emph{Pepita Rojo},
\emph{bordadora en fino}. En el principal, otra tabla decía
\emph{Planchadora}; y en el tercero había un balcón roto y algunos
tiestos.

---¿Significan algo el balcón roto y los tiestos?

---Nada; pero lo digo para que vea usted cómo examiné uno por uno todos
los accidentes de la fachada de aquella casa, como se examinan las
facciones del facineroso que nos ha robado, para poder dar sus señas a
la justicia.

---¿De modo que le tenemos allí?

---No cante usted victoria todavía, señor mío, que aún falta mucho por
contar\ldots{} Nos retiramos a casa. Yo calculaba que un hombre que se
acuesta a las cinco de la mañana no podría levantarse muy temprano.

---¿Pues qué? ¿Proyectaba usted nuevas excursiones?---pregunté con la
mayor sorpresa.

---A las ocho, después de charlar un poco con mi viejo, estábamos en la
calle Paquita y yo. ¿No se acuerda usted?

---Sí, me acuerdo.

---Salimos, sí, en dirección a la calle de Segovia. Llegamos; pregunté
en el portal por \emph{Pepita Rojo}, \emph{bordadora en fino}, y
dijéronme que vivía en el sotabanco; Paquita entró en la casa de
huéspedes del segundo pidiendo pupilaje.

---¡Qué demonio! Fue cuando Paquita estuvo fuera de casa tres días, y
usted dijo que había ido a Daganzo de Abajo a ver a su madre, enferma.

---Eso es. Yo entré en casa de la bordadora a encargarle una obra muy
difícil y costosa. Sin hacer alarde de riqueza, me mostré generosa;
volví al día siguiente, llevando un regalito a sus niños; conocí a su
marido, que es herrero, y no parecía tener trato alguno con
revolucionarios; pero ni mi observación ni mi dinero me dieron luz
alguna.

---¿Y Paquita?

---Vivió allí tres días. Hízose, por encargo mío la desenvuelta, para
comunicarse fácilmente con los demás huéspedes, y principalmente con un
tal Núñez, algo misterioso, que en la misma casa vivía, teniendo consigo
a un primo, que se decía recién llegado de Valencia.

---Ese primo\ldots{}

---Yo iba a visitar a Paquita, porque esta no podía hacer gran cosa
sola. Apenas había visto la fisonomía de Monsalud y no conocía el metal
de su voz. El tercer día de mi visita temblé de pavor y al mismo tiempo
de alborozo; había oído la voz del miserable en una habitación
inmediata. Al punto nos encerramos, y Paquita, practicó sigilosamente un
agujero en el endeble tabique detrás de un cuadro. Oímos algo; pero nada
importante. Núñez y Monsalud habían llamado a la patrona y contaban el
dinero para pagarle, pues se marchaban de la casa. Su conversación era
indiferente, y ni una palabra dijeron que indicase cuál iba a ser su
nuevo domicilio. Llegó entonces un tercero, salieron todos, y metiéndose
en un coche que a la puerta les esperaba, partieron, sin que fuera
posible averiguar nada.

---¡Perdido otra vez! ¿Y no se dio usted por vencida?

---Nada de eso. Paquita y yo entramos después en conversación con la
patrona, tratando de descubrir algo; pero nada sacamos en limpio. La
buena mujer ponderó la puntualidad y largueza con que semanalmente le
pagaba Núñez, calificando a este y a su primo de excelentes sujetos. No
hacía un cuarto de hora que habían salido, cuando llegaron\ldots{}
¿quiénes dirá usted?

---No sé.

---Los alguaciles de la Inquisición de Corte, con un señor familiar a la
cabeza.

---¿A prenderles? ¡Estuvieron buenos!\ldots{} Esa gente es como el humo:
lo ve uno y no puede echarle mano.

---Tranquilizada y en paz la casa, luego que los alguaciles, con el
señor familiar al frente se marcharon, reanudamos nuestra conversación
Paquita, la pupilera y yo. Fingí ser persona de escasos posibles, viuda
de un militar, y dije que me acomodaría en aquella casa al lado de mi
amiga, si me admitían por poco dinero. Era mi deseo penetrar en la
habitación abandonada por los fugitivos, para ver si habían dejado algún
objeto que aclarase un poco las tinieblas en que me encontraba. Enseñome
el cuarto la posadera, y al punto lo examiné todo, paredes, muebles,
piso. En un rincón de este había varios pedazos de papel, una carta
rota. En un momento en que estuvimos solas, los recogí, y guardados
cuidadosamente, me los traje a casa para juntarlos y leerlos.

Diciendo esto, sacó de su costurero un papel en que estaban pegados los
pedazos de la epístola.

---Lo que pude reunir y junté de este modo---dijo mostrándomelo,---no es
más que una tercera parte de la carta, y sólo resultan frases sueltas de
oscuro sentido. Vea usted: «\ldots{} mingo a las nueve de la noche te
espero en la esquina\ldots{} ana vieja no puedes venir a mi casa\ldots{}
que mi ma\ldots{} Caraban\ldots, enojada, furiosa y no mereces\ldots{}
Andrea».

\hypertarget{ix}{%
\chapter{IX}\label{ix}}

---No entiendo una palabra de esta monserga---dije, devolviendo el
papel.

---Pero basta fijarse un poco para comprender que es una cita amorosa.
La firma de la dama es Andrea.

---¡Andrea!\ldots---conozco yo varias Andreas.

---A mí no me importaba conocer a la dama: lo principal era saber el
punto en que se verificaría la cita amorosa, y esto bien se descubría
reflexionando un poco.

---¿En dónde?

---En la esquina de la calle de la Aduana vieja.

---Es verdad\ldots{} el domingo. ¿Y fue usted?

---¿Pues no había de ir? Aquella noche Paquita y yo la pasamos también
en claro. Vi a los dos amantes. Se me figura que él no está muy
entusiasmado; ella debe de valer poco; separáronse pronto.

---¿Y le siguió usted de nuevo?

---Por todo Madrid; hasta que después de diversas paradas y escalas aquí
y allí, paró cerca de la madrugada en la casa donde vivía y donde vive
ahora.

---¡Admirable, sorprendente!

---Desde que descubrí su nuevo albergue comenzó Dios a favorecerme,
porque Paquita reconoció en aquella la casa donde vive una parienta suya
y paisana, con la cual tiene muy buena amistad. Fue a visitarla al día
siguiente, y por ella supe que el marido de Doña Teresona (que así se
llama la de Daganzo) es portero, conserje o guardián de la tal casa,
perteneciente a bienes mostrencos y habitada por un administrador de
estos. El Sr.~Roque pertenece en cuerpo y alma al habitante principal de
la casa. Es difícil corromperle; pero no así la señora Teresona, que
insensible primero a mis ruegos, se ablandó con los regalos que le hice.
Todos mis ahorros y el producto de parte de mis alhajas que vendí, lo he
empleado en tentar la codicia y ganarme la voluntad de aquella mujer. He
penetrado anoche en la casa, y escondida en un miserable cuarto trastero
que da al patio y a la escalera grande, he visto entrar a Monsalud con
otros dos, encender luz y encerrarse en la única pieza habitable del
piso alto, cuyos largos corredores desnudos, abiertos, fríos y
solitarios tiemblan y crujen cuando alguien pasa por ellos. Nada más
necesito decir a usted sino que cuando la justicia quiera apoderarse del
conspirador, puede hacerlo cómodamente y sin peligro ni ruido.

---Mañana mismo---dije frotándome las manos de gozo.---¡Gracias a Dios!
España verá al fin un día de justicia, ya que ha visto tantos de
bajezas, debilidades e infames sobornos.

---¿Y se hará justicia?, pregunto yo ahora---dijo Jenara con
energía.---Este indigno espionaje que he referido, ¿será un vano
capricho de mujer furiosa?

---La Inquisición sabe dónde tiene la mano derecha.

---La Inquisición no sabe nada---repuso ella con desprecio. Sueño con la
justicia, y la justicia debe hacerse, debo hacerla yo misma. ¿Para qué
he de fiar mi justa venganza a la Sala de Alcaldes o a la Inquisición?
¿Necesito acaso de ellos? ¿Por ventura no estoy yo aquí?

Al decir esto, el vivo rayo de sus ojos indicaba una contumacia y una
virilidad (permítase la palabra) que me infundían miedo. Aquella mujer
no necesitaba de nadie para realizar sus ideas.

---Veo---le dije,---que usted será capaz de suplir con su acerada
voluntad a nuestra débil e impotente justicia. A tanto vilipendio han
llegado el siglo y los tiempos, que una mujer sola, sin más auxilio que
su corazón de fuego y su iniciativa poderosa, podrá dar satisfacción a
la moral pública y a la patria ultrajada. ¡Admirable espectáculo! ¡Cuán
grande es la mujer cuando quiere serlo! ¡Qué heroísmo! ¡Qué lección a
los vanos y corrompidos hombres, señora!\ldots{} Dios infunde a una
mujer esta energía potente; Dios envía un destello de su justicia sobre
el ser más débil y más bello de la creación, para que la gran idea no se
extinga en el mundo. Yace la autoridad hecha pedazos en el fango de las
logias y en las alfombras de los palacios. Dios da a una mujer el
encargo de recogerla, y la gran fuerza vuelve a brillar como un acero
terrible sobre la cabeza de los pueblos, atontados y embrutecidos por el
democratismo y la revolución\ldots{}

Jenara, profundamente abstraída, no contestó nada a mis ditirambos.

---Pero yo---continué con el mismo calor,---yo, en cierto modo
representante de esa justicia oficial que tan mal cumple sus deberes,
estoy interesado en que recobre su esplendor; he adquirido cierto
compromiso en este asunto, y por tanto, me atrevo a reclamar el
delincuente.

---¿Para prenderle mañana y soltarle pasado mañana?---dijo con el mayor
desdén.

---No, yo juro a usted por Dios que nos oye, que Salvador no quedará
esta vez sin castigo\ldots{} Pues no faltaba más\ldots{} Respondo de
ello\ldots{}

---Es usted como todos---me dijo gravemente.---Pero este asunto me causa
tanto terror, que no puedo empeñarme en llevar adelante mi primer
pensamiento. Es una locura, un extravío\ldots{} Mi corazón irritado y
furioso me ha impulsado hacia un fin terrible; pero en mi alma hay
también destellos de luz religiosa; tiemblo, retrocedo y me digo:
«Jenara, ¿qué vas a hacer?\ldots» Mientras buscaba a mi insultador y
asesino de mi esposo, no me causaba espanto el considerar la merecida
expiación de sus culpas; pero ahora que le tengo, ahora que le veo en mi
poder, casi puedo decir dentro de una jaula, siento frío en el corazón.
«¿Qué debo a hacer?» me pregunto. Si fuera hombre, la cuestión estaba
resuelta. Si mi esposo estuviera aquí, también. Pero me encuentro sola.
¿Qué puede hacer una mujer? Antes me condenaré a los tormentos del
despecho toda mi vida, que comprar con oro una mano extraña. Si tan
horrible idea cupo un día en mi cerebro, hoy la rechaza mi
corazón\ldots{} Le tengo en mi poder y vacilo\ldots{} Cuando le
perseguía, todas las ferocidades del castigo, hasta el asesinato, me
parecían naturales\ldots{} Mi mano le coge al fin, y todo es congoja e
indecisión\ldots{} Ahora me acuerdo---añadió sonriendo,---de un caso
ocurrido el otro día y que no por trivial, deja de ser muy apropiado a
lo que ahora nos ocupa. Dispénseme usted lo frívolo del cuento y óigalo.
Durante muchas noches me mortificaba en mi cuarto un miserable
ratoncillo, quitándome el sueño y adjudicándose multitud de objetos de
mi propiedad. Cuanto ideamos Paquita y yo para apoderarnos del vándalo
fue inútil. Yo me desesperaba, y desvelaba por las travesuras ruidosas
de nuestro intruso, tramaba mil proyectos de exterminio contra él.
Estrujarle, aplastarle, quemarle vivo, ahogarle, todo me parecía poco.
Oyendo el rumor de sus dientes y sus menudos pasos, mi corazón se
abrasaba (no se ría usted) en furores de venganza. Ningún placer había
comparable al placer de verle en la boca de un gato o en las tenazas de
la cocinera, o en las manos de un pilluelo de las calles\ldots{} Por
último, le cogí en la ratonera que usted nos dio. Cuando le vi preso y
en capilla, toda aquella tempestad de crueldades que rugían en mi
corazón desaparecieron como por encanto: aparté la vista con horror y
repugnancia, y entregando la ratonera a Paquita, le dije: «mátale donde
yo no le vea ni le sienta»\ldots{} ¿Querrá usted creer que me puse
nerviosa\ldots{} que casi estuve a punto de llorar\ldots{} que fui
corriendo de mi cuarto, porque desde él se sentían los chillidos
lastimeros del pobre animal?

---¡Corazón generoso en voluntad firme!---exclamé.---Bien, señora mía;
entrégueme usted esa ratonera donde acaba de caer el vándalo. Yo
juro\ldots{}

---Usted jurará todo lo que quiera; ¿pero de qué valen todas sus buenas
intenciones contra la flojedad del Gobierno? Le prenderán hoy, y
mañana\ldots{}

---Hay una gran irritación contra él; y no es fácil que se le suelte.
Vea usted cómo la señora Fermina Monsalud cayó en poder de la
Inquisición hace años, y aún se pudre en un calabozo, a pesar de los
esfuerzos que hacen los masones para salvarla.

---La prisión y el tormento que han dado a esa buena mujer es una
iniquidad que me horroriza.

---¡También usted se interesa por ella!

---Por la justicia. Toda infamia me irrita, y jamás perdonaré a mi
esposo y a mi abuelo la crueldad con que han tratado a esa pobre señora
inocente. ¿Es ella responsable de los crímenes de su hijo?

---Hasta cierto punto\ldots{}

---Hasta ningún punto---dijo bruscamente y con enojo.---¡Cuántas veces
he reñido con Carlos, echándole en cara su conducta en este particular!
¿No es inicuo, no es contrario a todas las leyes divinas y humanas
atormentar a una infeliz mujer, para qué?\ldots{} para que declare que
es cómplice de los crímenes de su hijo. Si no lo es, ¿cómo ha de
declararlo?

Advertí en el semblante de Jenara una emoción muy visible, fenómeno raro
en ella. Era la primera vez que aparecía conmovida durante nuestro largo
coloquio de aquella noche.

---Veo que el odio de que hablaba usted hace poco---le dije,---tiene
también sus suavidades.

---Sobre mi odio está mi justicia---repuso.---Y qué, ¿puede negarse que
esta iniquidad de mi familia atraerá sobre nosotros la cólera de Dios?
Yo preveo desgracias, yo preveo desastres en mi casa. ¡Ay!, ¿por qué no
somos felices? En este matrimonio, en esta joven familia llena de
tristezas, hay una cosa negra que todo lo envuelve.

Quedose meditabunda. Contemplándola y tratando de penetrar en los antros
de su alma, yo decía entre dientes:

¿Qué misterios hay en ti, mujer? ¿Qué tienes detrás del cielo de esos
ojos?

Luego hablé en voz alta, diciéndole:

---Verdaderamente es una crueldad inútil atormentar a esa desgraciada.
Se conoce que Salvador bebe los vientos por librarla de los señores
inquisidores. Ya vio usted aquella insolente hoja\ldots{}

---Debió usted hacer algo en pro de la infeliz mujer---dijo en tono de
viva reconvención.---¡Qué ocasión tiene usted para hacer una obra de
caridad y contentarme al mismo tiempo!

Dijo esto, y se levantó con la súbita agitación de una persona
impaciente.

---¿Qué más deseo yo sino agradar a usted?

---Dirá usted que es capricho; pero mi conciencia me repite que es ley.

---Y lo será.

---Usted tiene buenos sentimientos.

---Sin duda.

---Pues haga lo que piden la justicia y la piedad: empéñese usted con
Lozano para que mande poner en libertad a la mártir Fermina Monsalud.

Yo me quedé perplejo. La animación de Jenara, su encendido color y el
rayo de sus ojos indicaban sensibilidad muy viva. El cambio repentino de
aquella alma que había pasado de la fría impasibilidad inquisitorial a
un arranque de compasión ardiente, me confundía.

---Es difícil que Lozano de Torres consienta\ldots{}

---Pues me quedo con mi prisionero---exclamó, con un destello de
ira.---Yo haré de él lo que me convenga.

Alcé los hombros, y sin decir nada, acerqué las palmas de mis manos a la
lumbre.

---Me guardo mi prisionero; me guardo mi víctima; me guardo mi reo. Yo
le pondré en capilla cuando me convenga.

---Bueno---dije sencillamente.---En ese caso no hay nada que añadir. Lo
más que puedo hacer es hablar a Lozano de Torres.

---Y hacerle ver la injusticia y atrocidad que están cometiendo---añadió
suavizándose.---¡Ay, Pipaón; desde hace tiempo deseaba yo que alguien de
esta casa se interesase por esa pobre mujer! No me atreví a decirlo por
no enfadar a mi abuelo; pero créalo usted, ¡me causaba tanta
pena!\ldots{} Tenía vergüenza de manifestarlo; ¡parece mentira que cause
bochorno la piedad!\ldots{} Se me figura, además, que esta horrible
injusticia ha de traer grandes calamidades a mi familia; pienso mucho en
esto, estoy viendo venir el castigo de Dios.

---Nada, nada, señora, por mí no quedará.

---Pero qué locuras digo---añadió, tranquilizándose.---¡He dicho que
guardaba a mi prisionero!¿Para qué le quiero yo?\ldots{} No, la obra de
caridad que solicito nada tiene que ver con ese hombre. El perdón de la
madre inocente hará resaltar más la justicia si se castiga al hijo
malvado.

---Usted ha dicho que se reservaba para sí el prisionero.

---Una tontería, Pipaón. ¿Quiere usted saber ahora mismo dónde está
Salvador? En la calle del Divino Pastor, núm. 4, junto a Monteleón.

---Gracias, gracias.

---Justicia, pido justicia; y pues usted se presta a hacerla en mi
nombre, ponga usted en libertad a Fermina Monsalud; líbreme usted de ese
remordimiento que sufro por crueldades ajenas; aparte usted de mi
familia y de mí esa sangre que está cayendo gota a gota sobre nosotros,
y lo agradeceré con toda mi alma.

---Lo intentaré, señora; pero estoy confuso. Los extraños sentimientos
de usted no se explican fácilmente. De pronto una furia inquisitorial
contra el hijo\ldots{} de pronto una sensibilidad plañidera en favor de
la madre. ¿Qué es esto?

---¿Acaso lo sé yo? Amigo D. Juan, la holgazanería del corazón trae
estos extremados apasionamientos.

---¡La holgazanería del corazón!

---La falta de afecciones tranquilas. Mi soledad, el alejamiento de mi
marido, el no ser ni madre, ni hermana de nadie, traen un estado en que
el corazón ocioso trabaja buscando afectos. Es como un desheredado que
ha de ganarse la vida. Trabaja, discurre o coge lo que encuentra.

---Me alegraré de que el Sr.~D. Carlos vuelva pronto. Entre tanto,
señora, abogaré por la mamá; y en cuanto al hijo\ldots{}

---No le nombre usted más---repuso, volviendo el rostro con
repugnancia.---Lo que resta por hacer no me corresponde a mí. Cójale
usted, enciérrele, mátele, descuartícele enhorabuena. No me verá usted
conmovida ni alarmada, con tal que el castigo se haga lejos de mí.

---Le cogeré, le encerraré, le mataré, le descuartizaré.

---Le entrego a usted la ratonera---dijo riendo,---y aparto la cara y me
tapo los oídos. Mi rencor acaba donde empieza el verdugo.

---Muy bien; en el otro asuntillo yo hablaré mañana mismo al ministro.

---No diga usted que es cosa mía. Si Carlos lo supiera\ldots{}

---No, lo haré por mi cuenta. Dudo mucho que consiga nada\ldots{}

---Insista usted. Ponga usted ese favor por condición ineludible para la
entrega del conspirador más atrevido de estos tiempos.

---No es mala idea. ¿Y no se nos escapará de aquí a mañana?

---¿Cree usted que he gastado en balde mi dinero y mi tiempo?---dijo en
tono de seguridad.---Esté usted tranquilo.

---Pues no hay más que hablar.

---Nada más.

Y nos despedimos para retirarnos.

\hypertarget{x}{%
\chapter{X}\label{x}}

Al día siguiente, cuando me disponía a salir, entró un amigo, y me dijo
que corría por Madrid la noticia de que dejaba el Ministerio de Gracia y
Justicia el Sr.~Lozano de Torres. Esto varió de improviso el curso de
mis ideas, obligándome a apresurar mi visita al mencionado señor, y
quitándome al mismo tiempo las pocas esperanzas que tenía de conseguir
de él lo que a solicitar iba, por ser muy difícil tocar la fibra de la
piedad en un ministro sentenciado. Pero no había dado veinte pasos por
la calle Ancha, cuando otro amigo, oficial en el Ministerio de Gracia y
Justicia, me detuvo, diciéndome:

---En la casa se asegura que sucederá a D. Juan Esteban el señor marqués
de M***.

Nuevas confusiones en mi cabeza. Poco después estaba en el despacho de
Su Excelencia. Cuando yo entraba entró también el Sr.~D. Ignacio
Martínez Villela, circunstancia que no carecía de significación para mí.
El Sr.~Lozano estaba meditabundo y como acongojado, sin duda porque veía
encima el palo con que la Majestad de Fernando recompensaría pronto un
amor desmedido. A nuestras preguntas, no obstante, contestó que nada
sabía de destitución, y que el Rey se había mostrado la noche anterior
más cariñoso que nunca, lo cual, en puridad, no quería decir nada. Pero
lo que más me sorprendió desde el principio de mi visita, causándome
mucho gusto, fue que el ministro recibió a Villela con extraordinarias
muestras de aprecio.

---Ya le he dicho a usted---manifestó este,---que ha tiempo que el
marqués le mina a usted el terreno. Usted no quiere hacer caso de mí, no
quiere seguir mis consejos\ldots{}

El Zorro no contestó nada, y seguía muy taciturno.

---Ya nos cayó que hacer---dijo jovialmente Villela, sacando su caja de
tabaco,---porque el Sr.~D. Buenaventura va a entregarse a la persecución
de masones con un celo lamentable, y ahora\ldots{} ya se sabe\ldots{}
vamos a ser masones y jacobinos todos los que no pensamos como él. Seré
masón yo, será masón usted\ldots{}

---¡Yo!\ldots---dijo el ministro.

---Sí, ahora, amigo mío, todo aquel que no tenga la suerte de agradar al
señor marqués\ldots{} ya se sabe.

---Pues que no me busque el señor marqués---exclamó Lozano, súbitamente
arrebatado de ira,---porque me encontrará.

Villela rompió a reír. Su doble barba temblaba al compás de la risa.

---Pero hombre, si se lo estoy diciendo\ldots---gruñó D. Ignacio,---y
usted no quiere creerme; y usted cada vez más condescendiente con el
señor marqués; y usted erre que erre, creyendo que el señor marqués es
el brazo derecho de la nación. Hace tiempo que en esta casa somos
tratados como perros todos los que tenemos esa acendrada admiración y
culto por el ínclito marqués de M***.

---¿Como perros?

---O como masones. Hace tiempo que aquí le niegan a uno hasta los
favores más insignificantes, si no obtienen la venia del Sr.~D.
Buenaventura, de esa lumbrera, sin cuyos resplandores parece que los de
esta casa no se ven la punta de la nariz\ldots{}

---Pues qué, ¿no he accedido a todas las peticiones de usted?---dijo el
ministro con pena.

---A ninguna, Sr.~D. Juan Esteban. En cambio el señor marqués, a quien
se indica para sucesor de usted, y que tanto trabaja para conseguirlo,
no ha tenido más que boquear para ver realizados toda suerte de
antojillos. Ya se cobrará los favores que ha recibido, descuide usted.
Ahora, es corriente, todos somos masones. Preparémonos, Sr.~D. Juan
Esteban, a que caiga sobre nosotros la familiaridad del familiar.

---¿Qué dice a esto, Pipaón?---me preguntó el ministro.

---Sólo sé que en Madrid no se habla de otra cosa que de la entrada del
Sr. D. Buenaventura en este Ministerio---dije con gran aplomo.

---No se habla de otra cosa\ldots---repitió Lozano, sin poder disimular
que tenía traspasado el corazón.

---Y un amigo mío que ahora venía de Palacio me lo dijo
también---añadí.---Si aquí nadie está seguro\ldots{} ¿De qué sirven una
lealtad acrisolada, una disposición extraordinaria y una experiencia no
común?\ldots{} Pero consuélese usted, Sr.~Lozano de Torres, con saber
que quedarán en el país excelentes recuerdos de la paternal
administración de usted\ldots{}

---¿Sí, eh?

---Es evidente. El hombre honrado, el hombre inteligente, el hombre que
cumple con su deber, tiene por premio la admiración y el respeto de los
pueblos, ¿qué más quiere?\ldots{} Goza usted de fama además de hombre
benigno y que aborrece las crueldades\ldots{}

---Lo que es eso\ldots{}

---Hasta cierto punto---dijo Villela sonriendo.

---Hasta donde se ha podido---dije yo.---El Sr.~Lozano no abandonará
esta casa sin dar la última prueba de su caritativo corazón y
sentimientos cristianos. Sí, ¿por qué no he de decirlo de una vez? Hoy
vengo aquí con una pretensión de generosidad que proporcionará a usted,
amigo mío, ocasión de mostrar la bondad de su alma.

---Para pedirme una obra de caridad no se necesita tanto aparato---dijo
el ministro.---Si no es más que eso\ldots{}

---Vengo a solicitar, en nombre y a petición de varios paisanos míos,
que la Inquisición de Logroño ponga en libertad a Fermina Monsalud,
inicuamente atormentada.

Lozano de Torres frunció el ceño.

---Aquí te quiero ver---dijo Villela, echando hacia atrás el inmenso
cuerpo, y riendo como un ídolo asiático.---Si esa es la petición que yo
hice el otro día\ldots{} pero no, no agrada al Sr.~D.
Buenaventura\ldots{} ¡Pues no faltaba más, sino que se fuera a poner en
libertad a una mujer inocente!\ldots{} ¡Duro en ella, señor ministro! La
religión y el Estado exigen que esa mártir perezca.

Sus risas atronaban la sala.

Aquí hay una madre presa y un hijo que conspira---dijo el ministro.

---Eso es---gruñó Villela.---¿No se puede coger al hijo?\ldots{} pues
descoyuntar a la madre. ¿Hay nada más lógico?

---Es una iniquidad---dijo Lozano con movimiento repentino.---Esa pobre
señora debe ser puesta en libertad.

Alargó la mano para tomar pluma y papel.

---Tate, tate---exclamó con toda la fuerza de su mordaz ironía el
Elefante.---¿Qué va usted a hacer? Cuidadito; se enojará D.
Buenaventura\ldots{}

---Es una obra de caridad.

---Masónico, eso es masónico puro---gritó Villela, dejándose caer en el
sillón.

---Mandaremos al Consejo Supremo que disponga inmediatamente la libertad
de esa mujer---dijo Lozano escribiendo.

---Hombre de Dios---manifestó el Consejero variando al fin de tono y
hablando seriamente,---¿no solicité lo mismo hace tres días? Ha
necesitado usted que otro lo recomendara para hacerlo\ldots{}

---Mis paisanos\ldots---indiqué yo.

---Sr.~Pipaón---dijo Villela, volviendo a las burlas.---Usted es masón.

---¿Por qué?

---Porque ha pedido que se pusiera en libertad a una víctima de la
Santa\ldots{} y también yo soy masón, porque lo pedí antes, y también es
masón el Sr. Lozano, porque lo concede. Preparémonos a que los espías
del marqués se metan en nuestras casas.

Lozano escribía.

---¿Usted manda a la Suprema que dé las órdenes?---preguntó el
Consejero, mirando por encima del hombro de Lozano lo que este escribía.

---¡A raja tabla!---respondió Torres, echando una rúbrica que parecía
una puñalada.

Estaba furioso. Parecía un gatillo contrariado, y cuando tiró de la
campanilla para llamar a un oficial, sus ojuelos azules despedían un
fulgor vengativo.

---Ya está hecho---dijo con placer de quien ve el éxito de su primer
rasguño.

---Ha hecho usted una obra admirable---afirmó Villela, alargando sus
brazos hacia el ministro;---permítame que le abrace. Y ahora me toca a
mí. Tenemos que hablar mucho. Si Pipaón tuviera la bondad de dejarnos
solos\ldots{}

---Precisamente tengo que hacer\ldots{}

Di las gracias a Lozano, que me reiteró verbalmente su estimación.
Villela me dijo al despedirme:

---El ministro y yo vamos a hablar de masonería. Si ve usted a D.
Buenaventura, denúnciele esta logia.

---Pues hablemos de masonería---repitió Lozano sentándose junto a la
corpulenta humanidad de su amigo.---Pipaón, adiós.

Yo estaba tan sorprendido como satisfecho. Presentábanse aquel día las
cosas a pedir de boca, pues después de conseguir del ministro amenazado
lo que poco antes me resultara imposible o al menos dificilísimo, me
quedaba ancho y expedito el camino para congraciarme con el ministro
sucesor, proporcionándole uno de los más vivos goces que pudiera
anhelar. La Providencia, que jamás me abandonó, disponía en aquella
ocasión que quedase bien con todos, bien con Lozano de Torres, y mejor
aún con el marqués, principal imán de mis complacencias a la sazón,
porque los servicios que yo le prestara habían de influir mucho en la
provisión de la primer vacante en el Consejo.

Recibiome D. Buenaventura gozoso, aunque con modestas razones aseguró no
tener noticia de su proximidad al sillón de Gracia y Justicia. Cuando le
comuniqué las verídicas noticias que llevaba, púsose más alegre y al
punto se vistió para ir en busca del Gobernador de la Sala de Alcaldes y
el señor Alguacil Mayor de la Inquisición de Corte. El Estado y la
Iglesia estaban de enhorabuena. Tomáronse desde por la mañana con el
mayor sigilo todas las precauciones imaginables, porque el Sr.~D.
Buenaventura era uno de los esbirros más celosos y más diligentes que
por entonces tenía el absolutismo. Para que se vea qué vehemencia
acostumbraba poner aquel piadoso varón en sus gestiones inquisitoriales,
dejaré hablar por un momento a un célebre cronista de aquellos
tiempos\footnote{Van-Halen, \emph{Memorias}.}.

«El marqués de M***, familiar del Santo Oficio, hombre fanático por la
Inquisición, y oficioso por ella con delirio, había por sí y ante sí
organizado una tropa de espías, que él pagaba a sus propias expensas y
en la que figuraba con distinción un antiguo oficial suizo, que
conociendo el flaco de este corifeo, lo embaucaba y hacía creer mil
maravillas. Nadie osó ofrecer al Rey mi nueva captura con la decisión
que este digno caballero».

D. Buenaventura, aunque marqués, vivía en una casa de huéspedes de la
calle de la Abada. Amigo de la casa y obsequiador de las tres hermosas
niñas de la patrona era un tal Núñez, compinche de los conspiradores, el
cual se había dado muy buenas trazas para espiar a los espías del
marqués y al marqués mismo de un modo tan seguro como ingenioso. Y fue
que las niñas habían practicado un agujero en el tabique de la estancia
del familiar, el cual huequecillo, cubierto con un mapa, les permitía
oír desde la pieza inmediata cuanto en aquella se decía. Desde que iba
el suizo a dar parte de sus pesquisas o a recibir órdenes de D.
Buenaventura, ya estaban las niñas con el oído pegado a la pared, y
junto a ellas el travieso Núñez. Véase por esto si daría resultados la
policía del marqués.

Cuando todo quedó concertado, después de mis revelaciones para dar el
golpe seguro contra el astuto agitador, aquella misma noche, mi ilustre
amigo y protector me dijo:

---Querido Pipaón, no puedes figurarte cuánto hemos penado al señor
Alguacil Mayor y yo, noches pasadas. Recorrimos toda una manzana de
casas, saltando de tejado en tejado, más parecidos ambos a gatos que a
grandes de España. El señor duque se destrozó una pierna contra la reja
de una bohardilla, y yo resbalé por las tejas\ldots{} ¡ay!, poco me
faltó para rodar hasta el alero y caer a la calle\ldots{} Y por fin de
fiesta, no cogimos nada\ldots{} por todas partes gente honrada y
piadosa. Madrid, y sobre todo los pisos altos, desvanes, sotabancos y
chiribitiles, están atestados de modelos de virtud\ldots{} Los espías
que pago son perros jóvenes que apenas tienen olfato\ldots{} se
equivocan siempre. Denuncian un conspirador hereje en tal o cual
bohardilla, vamos allá, y resulta un ex---abate hambriento que compone
villancicos y romances para los ciegos\ldots{} Nos hablan de una logia;
corremos a ella, y después de rompernos las piernas contra las
chimeneas, hallamos un altar donde se adora entre flores y velas a la
Santísima Virgen\ldots{} O los espías no sirven para el oficio, o la
sociedad toda es una mentira, pura hipocresía y enredo\ldots{} En fin,
si es verdad lo que me has dicho, esta noche haremos algo de provecho,
mayormente si Su Majestad se digna nombrarme ministro. Como supongo que
estás impaciente por saber el resultado del golpe, en cuanto todo esté
hecho te mandaré un recado con Perico.

Yo dejé a D. Buenaventura entregado a sus dulces proyectos, y después de
despachar varios asuntos, me retiré ya de noche a mi casa, donde
encontré a D. Antonio Ugarte, que pocos días antes había llegado de
Andalucía y me estaba esperando para hablar conmigo, según dijo, de un
negocio interesante.

Desde que le vi, diome un vuelco el corazón, anunciándome con su ignoto
lenguaje que algo grave iba a tratar conmigo el tal sujeto. Era Ugarte
el hombre a quien yo más respetaba en aquella época. Su suprema
inteligencia y tino me subyugaban de tal modo, que no podía dejar de
obedecerle ciegamente. Sus presunciones, sus barruntos, eran leyes para
mí; y a pesar de mi amistad con diversas personas, sólo aquella influía
de un modo poderoso en mis ideas y en mi conducta. Al mismo tiempo él me
tenía por auxiliar tan poderoso de sus planes, que me podía llamar su
brazo derecho. Ugarte no podía ir a mi casa para una tontería. Advertí
que traía un paquete bajo la capa; algo estupendo iba a salir de sus
sibilíticos labios. El coloquio que ambos sostuvimos encerrados en mi
cuarto y sentados frente a frente es tan útil para la perfecta
inteligencia de estas Memorias mías, que no puedo pasarlo en silencio.

\hypertarget{xi}{%
\chapter{XI}\label{xi}}

---Pipaón---me dijo con el tono reprensivo que empleaba siempre para
echarme en cara mi conducta, cuando esta no le convenía,---de algún
tiempo a esta parte estás haciendo tantas y tan grandes tonterías, que
apenas te conozco. No sólo te haces daño a ti mismo, sino que me lo
haces a mí.

---Ya me dijo usted, Sr.~D. Antonio---le respondí con humildad,---que
encontraba censurable mi empeño en ser consejero; pero también he dicho
a usted que no es por el huevo sino por el fuero; que es para mí un caso
de honra, de dignidad.

---Nada de eso hace al caso. Importa poco lo que pretendas por esta o la
otra razón; lo que encuentro perjudicial y aun soberanamente necio es
que lo solicites, cualquiera que sea el motivo. Llevas trazas de no
conseguirlo nunca, y aun de perder lo que has adelantado en tu carrera.

Como no podía penetrar el sentido de aquellas razones, esperé sin decir
nada a que el gran Antonio I me las explicara.

---Mi situación en la Corte no es hoy lo que hace un par de años---dijo
muy preocupado,---ni la tuya tampoco.

---Desde la compra de los malhadados barcos rusos---respondí,---nos
hemos averiado un tanto, y navegamos mal. Demos gracias a Dios por no
habernos estrellado ya.

---¡La compra de los barcos rusos!---exclamó, fija la vista en el suelo
y moviendo la cabeza.---Ahí tienes un servicio eminentemente prestado a
nuestro país, y sin embargo, nadie nos lo ha agradecido.

Hice un esfuerzo supremo para no reírme.

---Verdaderamente---añadió D. Antonio,---los barcos no valían ni para
leña. Hablando aquí en confianza, amigo Pipaón, yo no creí que fueran
tan malos. El Sr.~Bailío me aseguró que podían hacer un viaje.

---No creo que sea posible un negocio peor, Sr.~D. Antonio; dígolo con
referencia al país. Si las quinientas mil libras que nos dieron los
ingleses para indemnizar a los perjudicados por la abolición de la trata
se hubieran repartido equitativamente entre los españoles pobres\ldots{}

---No te hagas eco tú también de las vulgaridades que corren a propósito
de los cinco navíos y la fragata que compramos al Emperador de
Rusia---dijo con cierto enfado.---Si ha resultado que esos buques están
podridos, la culpa no es mía. ¿Entiendo yo de barcos? Además aquí no
quieren sino gangas. ¿Pues qué, con quinientas mil libras, o sean
cincuenta millones de reales, se podían comprar seis buques acabaditos
de salir del astillero?

---Sr.~D. Antonio, si el gran Alejandro sigue con tan buen ojo para los
negocios, pronto no cabrá el dinero en todas las Rusias de Europa y de
Asia.

---¿Y a mí que me cuentas?---dijo amostazándose más.---El tratado
secreto que se celebró para comprarlos, firmelo yo como \emph{secretario
íntimo}; pero fue el Rey quien lo hizo. Era tal su impaciencia por
cerrar el trato de una vez, que estaba el hombre desasosegado y fuera de
sí. Yo quise ir con tiento, yo quise establecer alguna garantía; pero
amigo Pipaón, si vieras cómo estaba, cómo se puso ese hombre\ldots{}
Parecía sediento, ávido; parecíale que si no se compraban pronto los
barcos, se iban a convertir en humo las quinientas mil libras de los
ingleses. ¿Qué dices a esto?

---Parece mentira que tal haga y de tal modo se apure un hombre que
tiene a su disposición más de cien millones del Tesoro público y otras
gangas\ldots{}

---Si es un saco roto. ¡Y el vulgo necio cree que de la compra de los
cachuchos podridos me he aprovechado yo!\ldots---dijo Ugarte con cierta
expresión que indicaba como lástima de sí mismo,---¡yo, Pipaón!\ldots{}
No me ha tocado sino una miseria, un bocado, indigno de mí y de los
muchos afanes que pasé. Pero querido, los revolucionarios se valen de
todos los medios\ldots{} Ni los barcos son tan malos como dicen, ni es
absolutamente imposible que se den a la vela.

---Los marinos han dicho que no se embarcan en ellos.

---¡Los marinos! ¿Ignoras que todos están vendidos a la
masonería?\ldots{} Pero es preciso desplegar gran energía contra esa
gente; sino\ldots{} Al capitán de navío D. Roque Gruzeta se le ha puesto
preso por haber dado un informe desfavorable a los cinco buques.

---Es que no quieren embarcarse, Sr.~D. Antonio; es que nadie quiere ir
a América.

---Exactamente; ese es el mal primero y más grave, y ayer se lo he dicho
claramente a Su Majestad. Ni militares ni marinos quieren correr los
riesgos de una navegación larga, ni exponerse a las epidemias de
América, ni menos entrar en campaña con los rebeldes en un país tan
vasto como aquel. Los que vuelven, escuálidos y moribundos, quitan a los
expedicionarios las pocas ganas que tienen de embarcarse. Con esta
cobardía general, toda guerra ultramarina es imposible, y las Américas
se perderán, amigo Pipaón.

---Claro es que se pierden. Si este último esfuerzo no da algún
resultado\ldots{}

---¿Qué esfuerzo ni qué niño muerto? ¿Pero tú crees que las tropas del
ejército expedicionario que yo dispuse llegarán a embarcarse? ¡Necedad!
Fui a Cádiz hace poco y pude ver por mí mismo cómo está aquella gente.
Hay que oírles, amigo. Con decirte que no hay un solo oficial que no
esté afiliado en alguna sociedad secreta, está dicho todo; hablan con el
mayor desparpajo del mundo de ideas liberales, de constituciones, de
democracia, de soberanía nacional y aun de república. En los círculos de
oficiales y en los cuerpos de guardia no se oye otra cosa que versitos,
pullas y chascarrillos contra el absolutismo, contra el Rey absoluto y
contra todas las personas que le rodean. Hay allí una atmósfera que
marea; al llegar a la Isla se respira revolución, como al acercarse a un
incendio se respira humo.

---No estaba yo muy seguro de las aficiones absolutistas de los
oficiales del ejército, especialmente de los pertenecientes a cuerpos
facultativos---dije participando de las inquietudes de D.
Antonio,---pero no creí que las sociedades secretas estuvieran tan
extendidas.

D. Antonio dio una especie de silbido, que indicaba la plenitud de su
creencia en punto a la enorme extensión de las sociedades secretas.

---Estás en Babia, Pipaón---me dijo sonriendo.---Las sociedades
secretas, llámalas masonería, clubs, orientes, o como quieras, ofrecen
hoy una ramificación inmensa y completa dentro de la sociedad. En ellas
está comprometida toda clase de gente. ¿Crees que sólo los perdidos son
masones? ¡Error, amigo mío; vulgaridad supina! Altos personajes\ldots{}

---Eso lo sé también. Podría citar aquí media docena\ldots{}

---¡Media docena! Yo te citaré centenares. De algunos no tengo seguridad
completa; pero de muchos no puedo dudarlo, porque tengo datos
irrecusables. ¡Y qué hombres, y qué nombres! Precisamente los que mejor
suenan en los oídos del absolutismo son los que más se pronuncian hoy en
las logias. Ministros, tenientes generales y algún capitán general,
vicealmirantes, infinidad de brigadieres, consejeros de Estado, alcaldes
de Casa y Corte, familiares de la Inquisición, hasta inquisidores, hasta
canónigos, hasta frailes hay en la masonería. No me asombraré de ver en
ella a un señor obispo el mejor día\ldots{} Por de contado, el núcleo,
la base, el amasijo fundamental de este gran pastel que se está cociendo
y que pronto fermentará, si Dios no lo remedia, lo forman los oficiales
de todos los cuerpos que guarnecen la Corte y las principales ciudades y
plazas del Reino.

---Vamos, es para volverse loco.

---No; hay que tomarlo con calma, con mucha calma y sangre fría---repuso
D. Antonio mostrando gran dosis de ellas en su voz y semblante.

---Pero entonces, ¿qué va a pasar aquí?

---Qué sé yo\ldots{} allá veremos---dijo alzando los hombros;---pero
cualesquiera que sean los acontecimientos que han de venir, Pipaón, es
preciso estar preparado para ellos.

---¿Y cómo?

---Todo será según y como venga lo que ha de venir---dijo con
aplomo.---Ninguna cosa, ni aun la revolución, es mala de por sí Todo
depende del procedimiento, de la conducta.

---Si mal no recuerdo, Sr.~D. Antonio, he oído decir que frente a las
sociedades masónicas se ha formado también una especie de masonería
absolutista que se llama La Contramina, y cuyo objeto es atajar la
revolución, o ahogarla antes de nacer.

---Ríete de contraminas---repuso.---Conozco a los principales individuos
de ella, y con decirte que esa anti---conjuración la ideó el marqués de
M*** está dicho todo. Nada, nada, Pipaón, es preciso huir siempre de los
necios y no tener nada común con ellos. Todo lo que hoy intenta el
Gobierno contra las sociedades secretas; su tardía diligencia contra
ellas es pura necedad. No se lucha contra todas o casi todas las
capacidades del Reino, en milicia, en dinero, en talento.

---¿Esas tenemos?---exclamé asombrado al ver cómo iba creciendo el
fantasma masónico que Ugarte ponía ante mis ojos ---Esas tenemos, sí; y
todo lo contrario es tontería y ridiculez. Por ejemplo: tú, poniéndote
al servicio de Lozano de Torres, haciéndote lugarteniente del marqués de
M***, llevando mensajes al primero y ayudando al segundo en sus
espionajes grotescos por tejados gatunos y casas de huéspedes, eres tan
soberanamente necio, que al saberlo me he visto en la precisión de venir
a atajarte, a salvarte, a salvar tu porvenir y tu carrera, comprometidos
con la amistad de esos hombres.

Sin acertar a decir nada, miré a D. Antonio lleno de asombro. El punto
grave de nuestra conferencia había llegado.

---¿Piensas tú que vas a sacar algún provecho de tu servilismo? ¿Piensas
atrapar de ese modo la plaza de consejero?---prosiguió.---¡Cuán
equivocado estás! Lozano y el marqués de M***, a pesar de todos sus
humos, y aunque el uno suceda al otro en el Ministerio, son hoy dos
fantasmas de la Corte. Su valimiento es pura farsa y engaño. Agárrate a
sus faldones y te hundirás con ellos.

---Verdaderamente, Sr.~D. Antonio---dije,---después que he dejado de
frecuentar la cámara de Su Majestad, vivo a oscuras de todo.

---Se conoce. Estás con una venda en los ojos; marchas a tientas y te
estrellarás sin remedio. Yo también estoy apartado de Palacio; ignoro lo
que allí pasa; he perdido relaciones muy útiles allí; y ando como tú,
algo desorientado; pero hace tiempo que empiezo a ver claro, y de
resultas de mis recientes observaciones, he sacado en limpio que es un
suicidio tratar de oponerse al creciente poder de las sociedades
secretas.

Abrí los ojos con espanto.

---Durante algún tiempo---continuó D. Antonio,---me he dedicado a
observar esta sociedad, como observa el médico a su enfermo: le he
tomado el pulso y le he mirado la lengua, Pipaón; me he fijado
escrupulosamente en todos los síntomas, y he comprendido que el enfermo
va a dar un estallido.

---¡Un estallido!\ldots{} ¡una revolución!\ldots{}

---Pues qué, ¿lo dudas tú?\ldots{} Por mi parte no moveré la mano para
impulsarla, ni tampoco para contenerla---dijo mirando al techo.---Soy
agente de negocios: yo no soy hombre político. Si los grandes errores
cometidos traen una conmoción popular, casi, casi\ldots{} les está bien
merecido. Lo que ahora me preocupa es que cuando esa revolución venga (y
ten por seguro que vendrá), no me incluya a mí entre los absolutistas
rabiosos\ldots{} ¡Pues no faltaba más! Yo no soy amigo del despotismo
puro; yo he aconsejado la templanza.

---Y yo también.

---Mi plan---continuó,---es el que debe servir de norma a todo español
honrado: ni impulsar ni perseguir la revolución. ¿Que viene?, pues muy
señora mía. ¿Que no viene? Pues lo mismo que antes. Yo no daré un
céntimo para sediciones militares; pero tampoco reñiré ni me enemistaré
con la flor y nata del Reino en talentos, armas y riquezas\ldots{}
porque te lo repito, Pipaón, lo más granado está hoy en las sociedades
secretas.

---Vamos, que a usted, Sr.~D. Antonio, se le están pasando las ganas de
hacer una visita a las logias y codearse con lo más granado.

---No; en eso te equivocas. Jamás iré a las logias. Yo soy agente de
negocios; yo no soy hombre político\ldots{} Pero debo ser franco
contigo. Si personalmente no quiero ir, no me disgustaría tener algún
contacto con esa gente.

Yo empezaba a comprender.

---Esa idea me parece admirable, Sr.~D. Antonio---dije.---Nunca está de
más poner una vela al diablo.

Ugarte se sonrió, y luego en tono resuelto continuó de este modo:

---En una palabra, Pipaón, cuando se me ocurre un asunto delicado, una
dificultad de esas que requieren tacto, cordura y mucha discreción para
ser resueltas, miro a todos lados y no veo más que un hombre, tú.

---Dígamelo usted de una vez, ¿a qué andar con rodeos?

---Pues bien, amigo querido, hazte masón.

No pude menos de soltar la risa, y D. Antonio me acompañó festivamente
en mi desahogo.

---Para ti y para mí, este paso que te aconsejo no puede menos de ser
provechoso. Hazte masón, con reservas, se entiende. No creas que en las
sociedades secretas es todo misterio, lobreguez, sangre, horror, barbas
luengas, palabras enigmáticas: nada de eso. Hoy, los masones son la
gente más cortés y más amable del mundo\ldots{} Vas allá; yo buscaré
quien te lleve; procuras hacerte pasar por muy entusiasta. Di a todo
amén, y cuando los otros den un grito a la Constitución, tú das cuatro.

---Entendido.

---Además, no es preciso dejar de ser sincero. Puedes abrazar la nueva
idea con entera buena fe, porque esto lleva camino, hijo mío\ldots{} ¿Lo
harás?

---No tengo inconveniente.

---¿Romperás con Lozano de Torres, el marqués de M*** y demás hermanos
venerables de la necedad?

---Romperé.

---¿Dejarás el papel de espía y buscador de masones?

---Lo dejaré.

---¿Me darás cuenta de todo lo que veas, oigas y entiendas?

---La daré con mucho gusto, Sr.~D. Antonio; me ha hecho usted ver nuevos
horizontes con unas cuantas palabras. Adelante.

---Adelante. Lo principal es que dejes de mostrar empeño en la
persecución y castigo de los muchos reos políticos que andan por ahí.
Esta oficiosidad, de que ahora haces alarde, puede serte perjudicial en
los momentos presentes, y altamente nociva en los venideros.

---Pues que triunfen y se diviertan los reos políticos.

---Es más, amigo Pipaón. Desde el momento en que vas a ofrecer tu
cooperación a los oscuros trabajadores de las logias, tu deber es
amparar a los que se vean comprometidos\ldots{} No te asustes; podría
citarte una docena de señorones graves, firmísimas columnas del Estado
en el Consejo y en la milicia, los cuales han sido encubridores de la
mayor parte de los comprometidos en las conspiraciones de Porlier, Lacy
y Torrijos. La historia secreta de estas tentativas es muy curiosa. Los
pobrecitos inmolados ofrecieron con su sangre tributo externo al derecho
público; pero tras los cadáveres de Lacy y Porlier, amiguito, se han
escurrido impunes muchas personas cuyos nombres han sonado siempre bien
en Palacio\ldots{} ¿Con que entrarás por la nueva vía?

---Entraré. Usted ha venido a dar a mis ideas giro distinto del que
llevaban. Vivo algo retraído, y cuando usted está fuera de Madrid,
apenas conozco hacia dónde va la marejada.

---¡Ah!---exclamó con cierta tristeza,---la marejada va hacia
adelante\ldots{} y más que de prisa.

---¡Pues que vaya!---exclamé yo con alguna vehemencia.

---Nos veremos. Nos pondremos de acuerdo---dijo poniendo sobre la mesa
el paquete que traía, y que estaba compuesto como de medio centenar de
pequeños cuadernos.---Entre tanto, hazme el favor de repartir estos
folletos a los amigos. Esto se hace con cautela: un día das uno, otro
día das otro\ldots{} Es preciso que vaya cundiendo.

---Pero ¿qué es esto?

---Un admirable folleto que ha escrito en Londres Flórez Estrada. En él
se pintan de mano maestra los males de la nación. Es obra que no tiene
desperdicio; lo digo aunque no soy de los mejor tratados.

---Bien; se repartirá poco a poco.

---Todos los días te echas uno en el bolsillo\ldots{}

---Entendido, entendido\ldots{}

---Con que adiós. Veámonos con frecuencia para que me tengas al tanto de
lo que haces y de lo que ves.

---Todos los días; adiós, mi Sr.~D. Antonio---dije estrechando sus
nobles manos.

---Pues me voy tranquilo. Ya sé que cuento con un auxiliar poderoso.

---Nosotros, ya se sabe\ldots---afirmé abrazándole,---amigos hasta la
muerte.

---Gracias, gracias. Adiós.

Cuando Ugarte se marchaba, un criado llegó a la puerta y me entregó una
carta que decía:

«¡Victoria, amigo Pipaón, victoria completa! El criminal y sus cómplices
están ya en poder de la justicia. Ni uno solo ha podido escapar. Para
celebrar tan fausto suceso, vente a cenar conmigo\ldots{}

\flushright{\textsc{El Marqués de M}***».}
\justify

\hypertarget{xii}{%
\chapter{XII}\label{xii}}

Contesté excusándome, y me quedé en casa. Necesitaba meditar.

Poco después de anochecido entró Jenara a decirme que la cena estaba
preparada, y le di la carta para que la leyese.

---Ya ve usted---le dije,---que la justicia oficial, cuando quiere tener
ojo de lince y brazo de hierro\ldots{}

La señora no hizo ademán alguno de alegría. Tampoco se entusiasmó cuando
le dije que estaba conseguida la libertad de Doña Fermina Monsalud,
aunque me dio las gracias, asegurándome que había librado su alma de un
gran peso. La cena pasó triste y grave, hablando Jenara y yo de asuntos
indiferentes. Como le preguntase los motivos de su melancolía, me dijo:

---Hace muchos días que Carlos no me escribe, y estoy con cuidado.

---Se habrá puesto en camino.

---¿Sin avisármelo?---dijo vivamente y como enojada.

Poco después dimos tertulia al Sr.~de Baraona, que no salía de su
habitación, y para alegrarle un poco el espíritu le notifiqué la prisión
de su enemigo.

---Tengo poca fe---respondió,---en el rigor de estos señores. ¿Quién me
asegura que el criminal recién aprehendido no se paseará mañana por las
calles de Madrid? Ya te he dicho, querido Pipaón, que la justicia está
minada. Es como un doble edificio: en sus magníficas salas se sientan
jueces de cartón que sentencian y discuten y condenan, asistidos de
miserables ministriles. Ve esto el necio vulgo, creyéndolo justicia;
pero no ve el laberinto de entradas y salidas que en lo macizo de sus
paredes y cimientos tiene el tal edificio, por los cuales pasos secretos
se escurren los criminales, a ciencia y paciencia de aquellos señores
jueces de figurón. Desengáñate, hijo, los hombres del Gobierno, los
jueces, los consejeros, los ministros, forman hoy una especie de
retablo, donde mil vistosos personajes accionan y se mueven con las
apariencias de la vida. Acércate, mira bien, y verás que todo es cartón
puro: cartón el cetro del Monarca; cartón la espada de los generales;
cartón la vara del alcalde; cartón la cuchilla del verdugo.

Trajéronle las sopas y calló.

Poco después Jenara y yo, luego que dejamos al viejo dormido, nos
reunimos en el comedor, junto al brasero. Soltaba ella la labor para
tomar un libro, y luego el libro para coger la labor, demostrando en
esto que su espíritu se hallaba atormentado por ideas contrarias y en un
estado de obsesión inquieta que no podía vencer, variando a cada paso el
entretenimiento con que quería darle reposo. Púseme yo a leer el
\emph{Diario}, papel mucho más entretenido entonces que su único
compañero de publicidad la \emph{Gaceta}, y de repente Jenara hizo una
pregunta que me heló la sangre en las venas.

---¿En dónde ahorcan aquí?---dijo.

---En la plazuela de la Cebada---repuse.---Se alquilan balcones, como en
\emph{Corpus}.

Jenara, tomando la labor, empezó a dar terribles pinchazos con la aguja.
Sus dedos parecían el pico de un pájaro hambriento. Torné yo a mi
lectura del \emph{Diario}, y de nuevo me distrajo súbitamente,
diciéndome:

---En verdad, Pipaón, merece usted una corona por la diligencia que ha
mostrado en este negocio.

---¿Servir al Estado y servirle a usted no es estímulo bastante para un
hombre?

Jenara, dejando la labor, tomó otra vez el libro, pero al poco rato
apartolo con hastío.

---No abro el libro una sola vez esta noche---dijo,---sin que mis ojos
encuentren alguna idea triste. Oiga usted:

\small
\newlength\mlend
\settowidth\mlend{Y en olor de sepulcro, en vez de rosas,}
\begin{center}
\parbox{\mlend}{\quad Donde antes rosas y placer, ahora             \\
                Cadáveres y horror huella la planta,                \\
                Y en olor de sepulcro, en vez de rosas,             \\
                El aire tiñe sus funestas alas.}                    \\
\end{center}
\normalsize

---¿Qué poeta es ese?

---Cienfuegos.

---Un majadero. Siga usted mi consejo y mi ejemplo, Jenara. La mejor
lectura es el \emph{Diario}. Oiga usted: «El lunes fue ahorcado en
Valencia\ldots»

---Basta, basta---exclamó interrumpiéndome.---Es particular\ldots{} Me
salen horcas y muertos por todas partes.

---Es usted a veces más valerosa que un águila, y a veces más tímida que
un pajarillo. ¿La idea de la muerte de un hombre, de un malvado, le
causa a usted tanto temor?

---No, señor de Pipaón; ni me asusta ni me aterra la idea de que un gran
criminal expíe sus crímenes; lo que me causa pavor y más que pavor
repugnancia, es la horca, esa herramienta vil\ldots{} Las justicias de
la tierra debieran hacerlas siempre los agraviados en el momento de
recibir la ofensa\ldots{} qué quiere usted\ldots{} yo soy así\ldots{}
tengo esas ideas y no lo puedo remediar.

---Extraña justicia sería esa, Jenara.

---La mejor. Justicia rápida y por la mano del ofendido. Yo no la
concibo de otra manera. Esa que está en manos de hombres pagados,
vestidos de negro, amarillos y casi siempre sucios; esa que da tormento
al reo, y antes de matarlo lo envuelve en una mortaja de papel escrito,
me da tanta tristeza como repugnancia. Detesto al criminal y sería capaz
de matarle yo misma, sí señor, yo misma; pero compadezco al encausado.

No quise seguir tratando aquella cuestión, y los dos permanecimos largo
rato en silencio, que sólo se interrumpió para dar órdenes al nuevo
criado que me servía. Doña Fe se hallaba otra vez en cama, molestada de
sus pertinaces dolores. A pesar de ser ya un poco tarde, ni Jenara ni yo
teníamos ganas de dormir; sin duda una y otro llevábamos tantas ideas en
la cabeza, que el sueño no podía entrar en ella. Aquella respectiva
situación nuestra, nuestro desvelo, el silencio que reinaba en la casa,
las moribundas ascuas del brasero, que servían como de intermediario a
nuestra melancolía meditabunda, trajeron a mi memoria el recuerdo de la
noche en que recibí el singular escrito. No pude reprimir un repentino
acceso de miedo, el cual se apoderó de mi alma y corrió por dentro de mí
y pasó como una influencia eléctrica\ldots{} Pero mi razón se esforzó en
serenarse, diciendo: «ahora no hay cuidado».

De pronto sonaron no sé qué extraños ruidos en lo interior de la casa.
Yo di un grito y Jenara se puso a temblar.

---No es nada---dije.---Una puerta que se ha cerrado a impulsos del
viento\ldots{} ¿Qué es eso, Jenara, tiene usted miedo?

Tengo frío---me contestó arropándose en su mantón.

---¿No se acuesta usted?

---Sí\ldots{} ahora---dijo mirando a todos lados con el recelo propio de
quien busca, y al mismo tiempo teme ver algún objeto desagradable.

Llamé a la doncella, que acudió al punto; acompañelas a las dos hasta su
habitación, y cuando di a la señora las buenas noches, respondiome con
tristeza:

---Muchas gracias\ldots{} pero ya sé que esta noche no he de dormir.

Dirigime pensativo y no completamente libre de susto a mi cuarto. Cuando
abrí la puerta de él, y la luz que yo llevaba iluminó el interior de la
pieza\ldots{} ¡terror incomparable!\ldots{} lancé un grito de espanto y
no quedó gota de sangre en mi cuerpo\ldots{} ¡Jesús mil veces! En mi
cuarto había un hombre.

Un hombre, sí, que tranquilamente sentado en mi propio sillón, clavaba
en mí una mirada fulgurante y burlona a la vez.

¡Cielos divinos!, ¡socorro!\ldots{} ¡Un hombre en mi cuarto!

¿Quién? Salvador Monsalud.

\hypertarget{xiii}{%
\chapter{XIII}\label{xiii}}

Salvador Monsalud en persona.

Largo rato estuve sin habla, sin movimiento, paralizado por el espanto.
Yo no era Pipaón; yo era el miedo mismo. Mi espíritu era incapaz de
reflexión, de comparación, de juicio\ldots{} Las piernas me flaqueaban,
la voz, muerta en la garganta, no podía ni sabía pedir auxilio.

Creí ver un fantasma. Por un instante, perdiendo mi buen sentido, creí
en brujas, en duendes, en almas del otro mundo, en todos los disparates
de los cuentos de viejas.

Pero el fantasma se reía de mi turbación, y alargando un brazo hacia mí,
me dijo:

---No te asustes, Juan. Soy yo, tu amigo Salvador.

---¡Tú, Salvador, Salvadorcillo!\ldots---exclamé con voz ahogada.---¿Por
dónde entraste?\ldots{} Esto es una alevosía.

---Calla, calla---me dijo levantándose, al ver que yo, recobrando el
aliento, iba a alborotar la casa.---Soy tu amigo. No me tengas miedo.
Hablaremos un rato. Vengo a darte las gracias.

---¡Las gracias!\ldots{} ¡a mí!

---Sí, me has hecho un favor, un beneficio inmenso que te agradeceré
toda mi vida. Siéntate.

Imperiosamente me ofreció una silla. Los dos nos sentamos. El miedo y no
sé qué fascinación extraña me subordinaban al intruso visitante.

---Sí---añadió sonriendo y pasando cariñosamente su mano por mi
hombro,---un beneficio inmenso. A ti te debo que se hayan dado hoy las
órdenes para poner en libertad a mi pobre madre.

---¡A mí!\ldots{} es verdad\ldots{} sí, yo\ldots---repuse tratando de
sacar una idea de la confusión espantosa que había en mi cerebro.---Yo
fui quien supliqué al ministro\ldots{}

---Cediste a mi ruego\ldots{}

---Como me lo pedías en aquella hoja\ldots---dije viendo un poco más
claro, y determinando sacar partido de la situación.---Me pareció justo
lo que me pedías\ldots{} Pero dime, ¿con quien mandaste aquel papel?

---Lo traje yo mismo.

---¡Tú!\ldots{} bien puede ser, puesto que ahora estás aquí\ldots{} ¿Y
por dónde has entrado?

Monsalud rompió a reír.

---¿No has caído en ello? Por el agujero de la llave.

---Estas bromas no me gustan. Ya veo que no hay casa segura para la
masonería.

---Ni para el absolutismo. Si yo entro en la tuya, no falta quien entre
en la mía.

---Eso no me lo cuentes a mí. Nunca he sido espía.

---Pero sí amigo del marqués de M***. Escúchame, Juan; esta noche han
querido prenderme. He sospechado que anduvieras tú en este negocio.

Dominome de nuevo el miedo, y haciéndome el sorprendido, repuse:

---¡Prenderte!\ldots{} ¿y qué tengo yo que ver con eso?

---No es más que sospecha\ldots---dijo seriamente.---Te he creído autor
al mismo tiempo de un beneficio y de un agravio. Me ha parecido
inverosímil que me salvaras y me perdieras en un solo día, y he querido
apelar a tu franqueza y lealtad para que me digas la verdad.

---El beneficio, obra mía es; pero el agravio\ldots{}

Salvador me clavaba los ojos con tal fijeza escrutadora, que sus rayos
parecían penetrar en mi alma. Yo también le observé a él. Lejos de
parecerme siniestro y terrible, como decía Jenara, Monsalud tenía
aspecto en extremo agradable y había ganado mucho desde que no nos
veíamos. Su fisonomía era inteligencia y fuerza; la expresión de sus
ojos ejercía inexplicable dominio sobre mí, y toda su persona tenía un
sello de superioridad y nobleza que cautivaba. Vestía bien.

---Esta noche han intentado prenderme con un lujo de precauciones y de
habilidad que me han llamado la atención---dijo.---Gracias a la lealtad
de un hombre, he podido escapar a tiempo, y el señor marqués ha cogido
tan sólo a unos pobres aguadores que dormían en el sótano de la casa. Sé
que una señora desconocida sobornó a la pobre mujer del guarda; sé que
tu amigo el marqués dio las órdenes para sorprenderme; pero desconozco
la trama y los móviles de todo esto. Tú lo sabes y me lo has de decir.

---¡Yo!\ldots{} ¡Yo no sé una palabra! Todo lo que me dices es nuevo
para mí.

---Dime la verdad\ldots{} ¡tú lo sabes todo!---dijo apretándome el
brazo.---Dímelo, Bragas, o te acordarás de mí.

---¡Por mi nombre, por Dios que nos oye; te juro que nada sé!---repliqué
temblando de susto.---A fe que tienes buen modo de agradecerme lo que he
hecho por tu madre.

---Tú eres amigo y confidente íntimo del señor familiar---añadió
Salvador aplacándose.

Fingí gran sorpresa.

---¡Yo!\ldots{} ¡yo amigo de ese majadero!\ldots{} Pero tú no sabes lo
que dices. ¿En qué país vives?

---¿No eres tú de la pandilla de Lozano y del marqués de
M***?---preguntó algo desconcertado por mi aplomo.

---Vaya, vaya\ldots{} veo que no estás enterado de nada\ldots{} ¡Ya esos
tiempos pasaron, Salvador!

---Entonces has variado de ideas y de conducta.

---Sí señor, he cambiado de ideas, de conducta, de todo. Mi ruptura con
toda esa caterva absolutista es completa desde hace tiempo. Les trato y
nada más.

Salvador manifestaba el mayor asombro.

---¡Pues ya!\ldots---continué, cada vez más dueño de mí mismo.---Si así
no fuera, ¿crees que hubiera intercedido por tu madre?\ldots{} ¿crees
que me hubiera expuesto a pasar por cómplice de los conspiradores?

---Juan, por favor, ya seas mi amigo, ya seas mi enemigo, te ruego que
me digas lo que sabes respecto a mi persecución de esta noche.

---Te juro que no sé una palabra, ni tengo parte en ello---respondí con
tanta seguridad, que no se me traslucía en la cara ni la más ligera
turbación.

---Para que seas franco, voy a darte un ejemplo de franqueza. Escúchame
bien: en esta azarosa vida mía, consagrada a un afán que devora a una
pasión que lentamente consume y postra las fuerzas del alma, me he
dejado dominar por vanos caprichos o veleidades amorosas. Mi carácter,
en el cual hay ansiedades que nunca se han satisfecho ni se satisfarán
jamás, me ha impulsado a esto. Me he tolerado yo mismo estas
distracciones, como se tolera el soldado, en medio de la pelea,
descansos cobardes para fortalecer su ánimo. Pues bien, últimamente
amaba a una mujer con más vehemencia de la que suelo poner de algún
tiempo a esta parte en asuntos de amor. Pero no sé qué fatalidad me
persigue: con mi exaltación vino una inexplicable frialdad en la persona
amada: tuve primero celos y luego sospechas de que me vendía. No quiero
entrar en detalles inútiles. Lo principal es esto: al saber hace poco
que una señora había comprado con dinero el secreto de mi morada, se han
aumentado mis sospechas. Herido en lo más delicado de mi alma, he
sentido un furor y deseo de venganza que no puedo expresarte con
palabras; me he vuelto loco a fuerza de discurrir buscando antecedentes
e indicios que confirmaran mi sospecha; he vagado como un insensato por
las calles, jurando muertes y venganza; he prometido no descansar
mientras no aclarase este enigma que me atormenta y me abrasa las
entrañas.

Mi amigo apoyó la cabeza entre las manos. Su hermoso y noble semblante
expresaba viva cólera.

---En esta confusión---prosiguió,---discurrí que tú, como amigo del
familiar, podrías sacarme de dudas.

---No sé una palabra. En un tiempo conocí a todas las familias que
tenían relaciones con D. Buenaventura. ¿Cómo se llama esa señora?

---Andrea.

---No puedo darte ninguna luz, amigo.

---Al mismo tiempo que tal traición infame suponía, otra idea, otra
sospecha aumentaba mi confusión, amigo Juan; idea sobre la cual espero
que puedas darme más luz que sobre la otra.

---A ver.

---Existe otra mujer, a quien también puedo atribuir mi persecución; una
mujer que vive en tu misma casa, y de cuyas acciones, por reservadas que
sean, puedes tener noticias.

---¿Jenara?

---La misma. Esa tiene motivos para aborrecerme. Cuanto haga contra mí
no me sorprenderá. Nada pienso hacer en contra suya. Dejaré que caiga su
mano implacable y pediré a Dios que nos perdone a mí y a ella.

---Pues tampoco puedo sacarte de confusiones. No tengo ni el más leve
indicio de que Jenara\ldots{}

---¿De veras?

---Te lo juro por mi salvación.

---Está de Dios que yo me consuma en el fuego de esta duda
espantosa---exclamó Salvador con imponente afán.

Durante las últimas palabras, así como en diversos momentos de nuestro
diálogo, yo me preocupaba de un rumor que fuera de la alcoba sentía,
rumor como de leves pasos y faldas de mujer, y la idea de que un oído
importuno nos escuchase, empezó a mortificarme. No quise, sin embargo,
llamar sobre esto la atención de mi amigo, y me propuse no decir cosa
alguna que pudiera ser desagradable a la persona que, según mi
presunción, aplicaba su curioso oído a la puerta.

---Creo que puedes tener seguridad completa en ese particular---dije a
mi amigo.---Jenara es incapaz de hacer el indigno papel de inquisidor.

---También lo creo así---me respondió Monsalud.

Diciendo esto, ambos nos quedamos absortos, porque la puerta se abrió
suavemente y apareció ante nuestra vista una magnífica figura blanca,
cuya presencia repentina unida a la belleza y emoción de su rostro,
tenía todo el carácter de las misteriosas apariciones de la poesía y de
la noche.

---Es un error---dijo con voz tan turbada que no parecía la suya.---La
inquisidora he sido yo.

Salvador se levantó; dio indeciso algunos pasos como quien no sabe si
mostrarse cortés o enojado, y habló de este modo:

---¡Que Dios nos perdone a ti y a mí, Jenara!\ldots{} Por esta vez has
errado el golpe.

---En otra ocasión seré más afortunada---dijo la dama dando un paso
atrás y atrayendo la hoja de la puerta hacia sí.

---Aguarda un instante---exclamó Monsalud, corriendo a detenerla.---En
pago de tu crueldad, quiero darte una mala noticia.

Jenara se detuvo.

---Carlos, tu pobre marido, llega mañana\ldots{} Como hace tiempo que
has dejado de quererle, según él dice, por eso llamo a esto mala
noticia.

Salvador acentuaba sus palabras con punzante ironía.

---Pues no ha anunciado su viaje---dije yo, advirtiendo en Jenara una
gran perplejidad y deseando sugerirle una idea para que saliese de ella.

Pero Jenara no dijo nada. En su semblante, que poco antes parecía de
mármol, distinguí una alteración súbita. Leves llamaradas de rubor
tiñeron sus mejillas.

---No ha anunciado su viaje---añadió Monsalud,---porque viene a lo
celoso, callandito\ldots{} Quiere sorprender, acechar, vigilar. ¿Sabes
que está celoso, Jenara?\ldots{} El pobre Carlos no será nunca feliz.

Vi moverse los labios de Jenara y replegarse en torva conjunción sus
cejas. Difícil es conocer lo que pasó entonces en su mente y en su
conciencia (¿nos lo dirá ella misma algún día?), porque en vez de
hablar, cerró con estrépito la puerta, y desapareció como una visión de
teatro. Fui tras ella\ldots{} huía como la corza herida. Creeríase que
tras su fugitiva persona, semejante a la sombra de una diosa ofendida,
había quedado en la atmósfera un suspiro que por breve instante
reprodujo su emoción.

Cuando volví al lado de Monsalud, este reía.

\hypertarget{xiv}{%
\chapter{XIV}\label{xiv}}

---Gran bien me ha hecho tu huéspeda sacándome de dudas. Al fin veo que
no he perdido el tiempo con venir aquí.

---¡Con que era ella!

---¡Esta!---exclamó con júbilo.---¡Oh!, amigo Juan, qué dulce es ver que
sólo nos hacen daño nuestros enemigos\ldots{} Sospechar de un amigo, de
una persona amada, es el mayor de los martirios.

---Quién lo había de decir---indiqué yo, haciendo un esfuerzo para que
no me cogiese en mentira.---Cómo había de figurarme yo que
Jenarita\ldots{}

---¿Y no sospechabas nada?

---Ni una palabra.

---¿Y no te había confiado nada?

---¿A mí? Si no nos podemos ver\ldots{} si somos el perro y el gato.
¡Cuánto me alegro de que venga Carlos, a ver si esta gente se marcha de
una vez de mi casa!

Antes de pronunciar estas palabras me cercioré de que el espionaje había
concluido. Nadie nos oía. Cerradas cuidadosamente todas las puertas, me
senté junto a mi amigo, resuelto a poner en ejecución el hábil plan que
había concebido.

---¿Pero es cierto que no os lleváis bien los Baraonas y tú?---me
preguntó Salvador en tono que indicaba alguna desconfianza.

---No nos podemos ver, te he dicho. Ya conoces las ideas del abuelo. Es
un hombre insolente. Respecto a la implacable soberbia y a los
rencorosos sentimientos de Jenarita, ¿qué puedo decirte que tú no
sepas?\ldots{} Pues digo, si llegan a saber que yo he intercedido por tu
infeliz madre\ldots{} Cuando se les habla de tal asunto, son fieras el
abuelo y la nieta.

---No me hables de esto---dijo Salvador pálido de ira,---porque me
olvidaré de que estoy en casa ajena y en situación poco a propósito para
pedir cuentas a nadie\ldots{} Los Baraonas y los Garrotes son autores de
la prisión y del martirio de mi pobre madre. ¡Venganza miserable! Todo
porque le herí en un duelo leal, provocado por él\ldots{} ¡Si vieras
cuánto he luchado aquí para conseguir la libertad de la pobre
mártir!\ldots{} Diferentes veces se ha logrado lo que hoy te concedió el
ministro; diferentes veces, por empeño de poderosos amigos míos, ha dado
órdenes generosas al Consejo Supremo. Mientras Carlos ha estado en la
Rioja, todo ha sido inútil. Yo no sé cómo se las compone el maldito, que
puede allá más que el Consejo Supremo aquí.

---Tiene amigos y parientes en la Inquisición de Logroño, y es familiar
de ella.

---Mi madre será puesta en libertad pronto gracias a que Carlos ha
salido de allí, a que las órdenes de ahora son muy enérgicas, y sobre
todo a la revolución que se aproxima\ldots{} Pero sálvese o no la
infeliz señora, la infamia de esa gente rencorosa y vengativa como las
furias antiguas no quedará sin pago\ldots{} ¡Me parece mentira que
Carlos Garrote viene a Madrid y que le he de ver delante de mí!

Diciendo esto, eran tan enérgicas la expresión y los ademanes de mi
amigo, que me aparté de su lado, temeroso de alcanzar alguna señal
dolorosa de su indignación.

---Esta gente es atroz---dije.---No veo la hora de que se marchen de mi
casa. Estamos riñendo todo el día. ¡Cuántas veces les he echado en cara
ese furor inútil contra Doña Fermina, por no poder cebarse en ti!

---Por eso te llamará tanto la atención verme en esta casa, albergue de
mis implacables enemigos, y que al mismo tiempo lo es de un rabioso
absolutista.

---¡Absolutista yo!---exclamé comenzando a desarrollar mi plan.---No me
insultes.

---Yo vacilé largo rato antes de presentarme a ti, pero el deseo de que
me sacaras de una cruel duda me decidió. Por un lado, sospechaba que tú,
como familiar del familiar, no dejarías de tener parte en mi
persecución; por otro, el saber que habías implorado la libertad de mi
madre me inspiraba cierta confianza hacia ti, a pesar de tu absolutismo.

---¡Absolutista yo! Vuelvo a decirte que no me insultes. Bien sabes tú
que no soy servil. Si lo creyeras así, no te atreverías a venir a mi
casa.

---¿Por qué no?

---Porque temerías que te detuviese y te entregase a la justicia.
Monsalud se echó a reír, burlándose descaradamente de mí.

---Pues qué, ¿si yo fuera absolutista de los de D. Buenaventura,
estarías tú tan tranquilo delante de mí?

---Dices eso, pobre hombre, porque ignoras que aunque seas absolutista
de los de D. Buenaventura, no puedes nada contra mí dentro de tu propia
casa.

---¡Cómo que no!

---Mírame---añadió desembozándose.---No traigo armas. Esto prueba mi
confianza.

---Y si yo quisiera\ldots---dije lleno de confusión.---Verdad es que
algunos de mis criados está vendido a la masonería.

---Lo están todos.

---¡Todos! De modo que en mi propia casa\ldots{}

---Estoy yo más seguro que lo estuve esta noche en la mía me contestó
riendo.---No te alarmes por eso. Además, el mal es irreparable, porque
si despides a tus criados y tomas otros, sucederá lo mismo\ldots{}
¿Sabes que me encuentro bien aquí? Si me lo permites, descansaré un
poco---añadió, acomodándose holgadamente en el canapé.

Volvió de nuevo el miedo a apoderarse de mí; pero yo había resuelto
seguir la corriente a que me impulsaban mis nuevos propósitos y las
ideas de mi amigo, y le hablé de este modo con amabilidad.

---Por supuesto, Salvador, la traición de mis criados es perfectamente
inútil, porque has de saber que no sólo soy incapaz de perseguirte, sino
que te ocultaré y protegeré en caso de que otros te persigan.

---Vamos---dijo sonriendo amistosamente,---no me confundas más de lo que
estoy. Di que eres mi amigo, di que conservas algo del afecto que hace
años nos teníamos. Lo creeré, no sólo porque mi corazón es crédulo en
materias de amistad, sino porque has dado pruebas de ello hoy mismo
intercediendo por mi madre, lo cual te agradezco en el alma. Dime eso,
querido Juan; dime que eres leal y honrado y generoso conmigo; pero no
me digas que no eres absolutista, porque me echaré a reír.

---Pues te lo repito. Vamos, me enojaré de veras si insistes en tal
absurdo. Ven acá---añadí mostrando el paquete de folletos que me había
dejado D. Antonio Ugarte.---¿Es absolutista el hombre que se ocupa en
repartir estos papeles?

---¡El folleto de Flórez Estrada!

---He repartido ya más de cien. Asómbrate, Salvadorillo: he hecho llegar
este cuaderno a las manos de Su Majestad y de los Infantes.

---Esto es algo---dijo con formalidad;---pero no es una prueba completa
de enemistad con el absolutismo. Quizás tu entendimiento se incline a
otras ideas; pero ya estás muy amoldado, Bragas, estás endurecido en la
forma de los Lozanos de Torres, de los Buenaventura, de los Eguía, de
los Elío\ldots{} Necesitarías que te derritieran y que de nuevo te
fundiesen en otro crisol.

---Tonto---repliqué con brío,---¿y quién te ha dicho que no me he puesto
ya al fuego?

---¡Tú!, el covachuelo, el oficial de Paja y Utensilios, el director de
la Caja de Amortización, el amigo del Sr.~Chamorro, el brazo derecho del
Sr.~Ugarte, el tertulio de Palacio, el mandadero de Su Majestad\ldots{}

---¡Yo, yo, yo, sí!---afirmé con enfado.---¿Quieres que te convenza de
una vez con dos palabras, Salvador?\ldots{} Pues para que comprendas mi
decidida ruptura con todos esos deplorables antecedentes y personas,
óyeme lo que voy a decirte. Quiero ser masón.

Monsalud manifestó el mayor asombro.

---Ser masón es no ser nada, si no se conspira---me dijo.

---¡Quiero conspirar!---exclamé dando fuerte puñetazo sobre la mesa y
metiéndome después las manos en los bolsillos.

---Pero no se conspira para aumentar la autoridad de la Corona, sino
para disminuirla. No se conspira en pro del Rey, sino en pro de la
Nación.

---Pues en pro de la Nación.

---Se conspira para restablecer el Gobierno liberal y la Constitución,
es decir, lo que tú llamabas la mamancia cuando escribías en La Atalaya.

---Para restablecer el Gobierno liberal y la mamancia---repetí
frunciendo el ceño y con los ojos fijos en el suelo.

---Y para dar al traste con la infame polilla de España que mina el
Trono y el País, y al mismo tiempo se los está comiendo.

---¡Para eso, para eso!

---Debo añadirte que hoy se hila un poco delgado debajo de Madrid.

---¡Debajo de Madrid!

---¿No me entiendes? En las logias y reuniones secretas, quiero decir.
Hoy se toman precauciones. Cuando un señorón de categoría elevada, sea
quien fuere, ofrece su ayuda a la revolución, lo cual ocurre todos los
días, queda ligado por compromiso solemne; y las veleidades, querido
Bragas, los arrepentimientos, suelen costar caros a quien los padece.

---Sí, ya sé\ldots---dije, inspeccionando otra vez la puerta, para
cerciorarme de que nadie nos oía.---Hay pruebas rigurosas, palabras
enigmáticas, juramentos que hielan la sangre en las venas\ldots{} y el
que hace traición muere sin remedio.

---No hay nada de eso---me dijo riendo.---Huye de esas reuniones
formularias que establecen el sainete en los sótanos. Ahora no se trata
de eso. Cuando los pueblos padecen y luchan por su emancipación, obran
seriamente y van a su objeto sin necedades de teatro. Ahora, amigo
Bragas, las cosas han llegado a un punto tal, que se trabaja por la
libertad a toda prisa, con la avidez del náufrago que entre las olas
lucha con la muerte y por la vida\ldots{} Fuera misterios y ritos
anticuados y palabras vacías. Todo es acción: las tinieblas y el
misterio han dejado de ser vano velo de las chocarrerías de los
holgazanes.

Yo lo he visto todo desde el principio: he visto las jimias haciendo
muecas entre dos calaveras en la ahumada atmósfera de una cueva; y hoy
veo a los hombres inteligentes y formales labrando en silencio y sin
aparato las palancas poderosas con que pronto ha de moverse lo de
arriba. Sólo en las épocas en que no hay nada que hacer existen esas
vanidades y espantajos ridículos de que habla el vulgo. Ahora la
inmensidad de la tarea une las manos de todos los hombres en una obra
común, y desaparecen las máscaras convencionales y las fórmulas
aparatosas, que más bien eran entretenimiento que utilidad. Eso no quita
que en plena luz, y a la faz del mundo oficial y de la tiranía, se
empleen ciertos signos para reconocerse y obrar de acuerdo; pero allá
dentro, amigo, en nuestro reino escondido, en aquella vida de catacumbas
donde se prepara la nueva vida libre y pública, todo es claridad y
sencillez. Se trabaja, se extiende la acción con arte y fuerza; se
prepara el golpe con la destreza y habilidad necesarias para que no se
malogre como otras veces. Ahora bien, Bragas de Pipaón; tú, servidor
declarado de los poderosos de hoy, ¿quieres servir a la revolución?

---Sí quiero---respondí.---Pero dime antes una cosa: ¿esa revolución
vendrá?

---¡Vendrá! Para ti es condición indispensable que la revolución venga.
Adoras el hecho, no la idea\ldots{} No puedo responderte. Puede venir y
no puede venir. Eso dependerá de este, del otro, de mí, de los demás, de
ti mismo, de todos reunidos. Si hacemos tonterías, ¡cómo ha de venir la
revolución!

---Lo preguntaba porque eso es muy importante. D. Antonio Ugarte, uno de
los hombres más listos y de mejor ojo que hay en España, me ha asegurado
que la revolución vendrá.

Al decir esto, la idea del puesto que me habían negado en el Consejo
estaba fija en mi cerebro como la marca de un hierro encendido. Me
quemaba.

---¡La revolución viene, la revolución viene!---proseguí sintiendo en mí
una especie de voz interior que así me lo decía.---Lo conozco, lo
adivino, lo veo, amigo Monsalud, en la atmósfera que nos rodea, lo veo
en la cara misma de los palaciegos. Es un hecho inevitable, lógico. La
revolución viene, como viene el día después de la noche. Todo lo
anuncia, ilustre amigo. Hasta los pájaros cuando cantan dicen
«revolución».

---Esto te infundirá valor y aliento. La revolución no suprimirá los
destinos\ldots{} por eso tu acción tiene poco mérito. Pero en fin,
quieres ser de los buenos, y el sistema adoptado es recibir a todo el
mundo, venga de donde viniere. Ahora voy a cogerte por la palabra, para
que no te arrepientas de aquí a una hora. ¿Puedes salir conmigo esta
noche?

---¿Por qué no? Vamos a donde quieras.

---Es muy cerca; no andaremos mucho.

---Mi capa, mi sombrero\ldots{} ¡Blas!\ldots{} pero ¿es posible que este
sencillote criado mío esté también vendido a la masonería?

---En cuerpo y alma. Ahora, ciudadano Robespierre---me dijo con
donaire,---convendría que tomásemos algo. Quizás tengamos que estar en
vela toda la noche. Has de saber que no carezco de apetito: es imposible
que en la casa de un hombre que ha servido en tan altos puestos no haya
a estas horas excelentes fiambres.

---Todo lo que quieras. ¡Blas, Blas!\ldots{} Este tunante masón no
viene.

Al fin apareció mi criado, al cual no pude mirar sin rencorosa
prevención, considerándole traidor, y nos sirvió un bocado confortativo.
Mientras comía, meditaba yo sobre aquel nuevo giro que tomaban mis
ideas, sobre aquel nuevo camino que emprendía mi actividad.

---Es preciso---me dije para mí,---que en este mundo desconocido en que
ahora entro procure desde el primer instante disipar los recelos que mi
presencia pudiera despertar. Cuidadito, Pipaón, con mostrar tibieza o
indiferencia, aunque veas toda clase de extravagancias y locuras. Un
celo excesivo y un entusiasmo demasiado ardoroso, no serán tampoco el
mejor sistema. Tomemos por modelo al maestro D. Antonio Ugarte.
Conviene, pues, adoptar una actitud intermedia, poner cara en cuyas
facciones se asocien artística y noblemente el entusiasmo y la dignidad,
la templanza del gobierno y la energía revolucionaria\ldots{} Mi papel
es el de un honrado repúblico que, comprendiendo con dolor la
incapacidad del absolutismo para gobernar a los pueblos, se acerca grave
y triste, pero resuelto a la revolución y le ofrece sus servicios,
porque sería lamentable que la revolución, si algo hace, lo hiciera sin
él\ldots{} Animo y disimulo. Seguro estoy de que al poco tiempo de estar
en la conspiración, me encontraré tan a mis anchas como en la camarilla
de Su Majestad a los dos días de ingreso\ldots; seguro estoy de que mi
sutil travesura volverá lo de arriba abajo y lo de abajo arriba, en esas
escondidas sociedades que voy a visitar\ldots{} seguro estoy de que al
poco tiempo de mi feliz iniciación, armaré más líos y enredos que vio
Creta en su famoso laberinto, y de que no pasarán muchos meses sin que
traduzca en provecho propio las tenebrosas artimañas de estos caballeros
y mi novel liberalismo. ¡Lo haré sin remedio lo haré! ¡Ay!, me conozco
como si me hubiera parido.

\hypertarget{xv}{%
\chapter{XV}\label{xv}}

---¿Duermen todos en la casa?---me dijo Monsalud cuando el reloj de cucú
que exornaba mi sala dio las diez.

---Sí---repuse,---mas para salir nosotros, poco importa que duerman o
no\ldots{} mayormente, señor brujo, cuando ahora vamos a escaparnos por
una grieta misteriosa abierta en la pared o por el cañón de la chimenea
de la cocina. Vamos, haz la invocación y vendrá un señor gentil---hombre
del Tártaro a abrirnos paso.

---Tú puedes hacer la invocación---dijo Salvador poniéndose la capa.

---¿De qué modo?\ldots{} ¿Llamo al Demonio?

---O a Doña Fe, que es lo mismo.

---¡Doña Fe! ¡Señora Doña Fe!

Mis gritos se perdían en las soledades de la casa sin hallar respuesta;
pero al fin un eco de ellos pudo llegar a las orejas de la dueña.

Y en verdad fue como si el mismo Lucifer apareciera justificando la
broma de nuestra demoníaca evocación y brujería, porque había que ver la
fealdad de mi doméstica, soñolienta y amarilla la faz, cerrado un ojo
mientras revolvía el otro en todas direcciones, cual si ambos se
concertaran para turnar en sus funciones, acordando que durmiera el uno
mientras el otro veía. Sin ser vieja, Doña Fe tenía en su desagradable
semblante una especie de decrepitud sin respetabilidad, mientras el
peinado con pretensiones de elegancia y la escofieta picuda la hacían
bastante ridícula. Dando al viento la destemplada y bronca voz, dijo al
llegar a mi presencia:

---De morir tenemos.

---Ya lo sabemos, señora---exclamé con ira;---ya lo sabemos. ¡Maldita
sea usted y toda su casta! Ya he descubierto que está usted engañando a
su amo, que abre usted la puerta de mi casa a hombres
desconocidos\ldots{} porque si ahora ha querido Dios que metiera usted a
un amigo, otra vez podrán ser asesinos y ladrones\ldots{} Señora Doña
Fe, mañana mismo se pone usted en la calle.

---Todo sea por Dios---dijo la dueña con calma imperturbable.---El padre
Beraza me dijo que, haciendo lo que he hecho, servía a Dios.

---Ya, ya ajustaremos cuentas. Respóndame usted. ¿Duerme el señor de
Baraona?

---Sí señor.

---¿Y la señora Doña Jenara?

---También parece que duerme.

---Bueno; retírese usted.

---No, que va a ir delante de nosotros.

---¿A dónde?

---A enseñarnos el camino y abrirnos la puerta.

Doña Fe salió de mi cuarto, y tras ella Monsalud, y tras Monsalud, yo,
sin comprender a dónde íbamos, viajero errante y extraviado dentro de mi
propia casa.

Atravesámosla toda hasta llegar a un sitio próximo a la cocina, donde
estaba la puerta de una escalera que bajaba al patio colindante con el
jardín de la casa inmediata. Como aquella salida no tenía comunicación
directa con la calle, habíala yo condenado al entrar en la casa,
clavándola fuertemente. Sorprendiome mucho verla desclavada y
practicable, y juré en mi interior tomar al siguiente día venganza
pronta y ejemplar de Doña Fe. Por entonces no dije nada, y cuando
Salvador mandó a la dueña que abriese, y esta obedeció, salimos y
bajamos los tres.

---¿Para qué necesitamos ahora a esta infame bruja?---pregunté a
Salvador.

---Ya verás---replicó Monsalud.

Llegamos al patio lóbrego, destartalado y profundo, cuyas humedades e
inmundicias criaban en distintos sitios algunas yerbas raquíticas y
arbustos tristes. Uno de sus cuatro lados era una tapia que limitaba el
jardín inmediato, cuyos elevados árboles secos traspasaban el espacio de
sus dominios para invadir los míos, y alguno de aquellos alargaba sus
dedos flacos, desnudos y ateridos hasta tocar los cristales de mi
comedor. En los otros lados había varias ventanuchas y puertecillas,
tapiadas todas menos una, que se decoraba con media docena de cristales
rotos y una fechadura tomada de viejísimo orín. Doña Fe golpeó con su
mano en uno de los cristales; viose al través de ellos una luz, y al
poco rato se abrió la puerta del modo más natural posible, sin que
precedieran al acto ni fétido olor de azufre ni aullidos de demonios
bufones.

La comunicación abierta dio paso a un anciano robusto, guapo y
sonrosado, cuya alegre fisonomía no me era en verdad desconocida. Al
vernos se sonrió con la franqueza propia de los tunantes hechos a la
farsa y engaños de la vida; rascose una oreja, dejando caer sobre la
sien contraria el sombrero anticuado y mugriento con que cubría su
hermosa cabeza cana, y después nos hizo un saludo tan cortesano y fino
como el de un diplomático.

---Sean bienvenidos sus mercedes.

---Sr.~Mano de Mortero---dijo Doña Fe, mostrando un cazuelo de comida
que en la mano traía.---Ahí tiene usted lo de hoy.

---Venga acá---repuso el gallardo y festivo viejo, dando un paso fuera
de la puerta;---venga esa bendición de Dios. Pero ¿qué hacen estos
caballeros que no pasan adelante?

Franqueamos el estrecho umbral; desapareció Doña Fe, perdiéndose en la
oscuridad del patio; cerrose la puerta y nos hallamos en una ancha
habitación de techo abovedado, cuyo aspecto, sin tener nada de
sobrenatural, ni de infernal, ni aun de extraordinario, me dejó suspenso
y estupefacto. Los cuatro testeros de la tal pieza apenas tenían
superficie para tanto trebejo roto y sucio, para tanto cachivache como
en ellos había acumulado una mano diligente y allegadora. Prescindiendo
de los muebles de uso diario, parecía una prendería del peor género:
había sillas de montar, enteras unas, despedazadas otras; cajas de
violín, frenos y herrajes de caballerías, artesas rotas, copas de cobre
que llevaron lumbre y ora llevaban polvo; armarios que fueron sepulcro
de ejecutorias y eran ya depósito de clavos, hebillas, tenedores, pesas
de reloj, garfios, badilas, espuelas, llaves, tinteros de cuerno,
tacones de palo, asadores, cucharas, lancetas, tabaqueras, tenacillas,
peines, dedales, piedras de chispa y otras mil y mil baratijas de
diferentes edades y sexos, que habían servido para diversos usos de la
vida.

Por aquí y allí, colgadas unas, en pie otras, puestas de costado o boca
abajo, se veían multitud de imágenes, Dolorosas con el pecho traspasado,
Josés con vara, Migueles con demonio, Santiagos a caballo, Roques con
perro, Antones con cerdo, Pedros con llaves y Lorenzos con parrillas;
toda la Corte celestial en suma. Pero entre tanta arrinconada santidad,
sólo una Virgen del Rosario tenía los honores del culto. Puesta en una
especie de altarejo muy singular, adornado con no sé qué estrambóticos
fragmentos (entre ellos las roscas de una trompa y la placa dorada de un
morrión de la guardia), tenía delante algunas flores de trapo y a los
lados algún resto mocoso de velas de cera.

Vi en el ángulo oscuro una cama de no mal aspecto. También había
diversas suertes de armas, tales como espadas, las más sin punta, sables
de guardia, algún coselete que debía de tener memoria de Roldán, y
además pistolas que habían roto el fuego, pero que no tenían más que la
intención, un mosquete, y la más variada colección de trabucos que he
visto en mi vida. Entre los muchos objetos pacíficos que en los rincones
y paredes distinguí, tales como velones, candeleros, platos de metal,
braserillos y loza de china, creí reconocer alguna pieza de mi
pertenencia que había desaparecido de mi casa, sin que nadie pudiese
averiguar quién cargara con ella; pero me callé y seguí observando.

Lo que más llamó mi atención fue una especie de banco de taller, donde
había multitud de figurillas, al parecer juguetes de niños; caballitos,
títeres que movían brazos y piernas con articulaciones de alambre;
panderetas, nacimientos, instrumentos rústicos, dominguillos, peonzas y
otras zarandajas, muchas de las cuales estaban por concluir o a media
pintura, entre tarros de almagre y toscas herramientas.

Ocupaba el centro de la habitación una mesilla de zapatero y junto a
ella un asiento agujereado, del cual parecía acabar de levantarse el
Mano de Mortero, y veíanse a un lado y otro suelas y tacones, con
multitud de gruesos zapatos negros y chinelas juanetudas, pero nada de
obra nueva.

---¿Qué tal? ¿Se trabaja mucho?---preguntó Monsalud al anciano, que, sin
dejar la lámpara de la mano, se disponía a ser nuestro guía.

---Estoy echándole medias suelas al señor Definidor---repuso con
desdén;---poca cosa, señor. Si no fuera por lo que cae\ldots{}

Diciendo esto, dirigió una mirada orgullosa y magistral a los
innumerables chirimbolos que en toda la redondez del cuarto se veían.
Los miró como mira un general su ejército.

---¿El señor es el amo de Doña Fe?---dijo después, mirándome con
impertinencia.---¡Ah! ¡Doña Fe!\ldots{} ¡Excelente señora!\ldots{} ¿No
se le ofrece a usted alguna cosilla? También hago juguetes. Si tiene
usted niños\ldots{}

---Veo que guarda usted una buena colección de\ldots{} preciosidades.

---Yo\ldots{} recojo todo lo que encuentro.

Se había puesto las manos en la cintura, y con el sombrero sobre la ceja
ofrecía la más rufianesca y cómica apariencia que puede imaginarse. Yo
conocía a aquel hombre; pero la perplejidad en que me encontraba era
gran estorbo para mi memoria.

---¿Quieren ustedes pasar allá? Pues vamos---dijo Mortero, tomando su
linterna.

Cuando esto decía, habíamos salido Monsalud y yo, y nos internábamos por
un largo callejón oscuro, que no tenía nada de agradable como paseo. Iba
el viejo despacio, por no permitirle sus piernas mayor actividad, y
Salvador y yo teníamos tiempo para recreamos en las contorsiones y
horribles gestos que hacían nuestras sombras bailando en la pared a
medida que avanzábamos. Según los movimientos de la linterna de Mortero,
corrían aquellas, anticipándose a nosotros, y desde lejos nos miraban,
aguardando a que pasáramos para unírsenos de nuevo: otras veces se
quedaban atrás, y luego en tropel corrían jugando para tomarnos la
delantera.

Llegamos a una puerta, que empujó el anciano, y yo creí que por ella
salíamos al aire libre. Pero mi sorpresa y mi pesadumbre fueron grandes
cuando vi que, en vez del libre espacio, se extendían ante mí negras
bóvedas de ladrillo, cuando en lugar de subir, bajamos una escalerilla
que si no conducía al Infierno, llevaba cuando menos a las antesalas de
este.

---Pero ¿a dónde vamos?---pregunté bastante inquieto.---¿No hemos bajado
bastante todavía? ¿Esto es el Tártaro o qué es?

---Chitón---dijo Monsalud sonriendo y poniéndose el dedo en los labios.

La escalera no era muy larga; pero tan estrecha que sin cesar me iba
aporreando la cabeza contra la bóveda de ella, haciendo de camino gran
acopio de telarañas.

---Estamos en plena novela, amigo Salvador---dije librando mi rostro de
aquellos cendales.---¿Qué demonios es esto? ¿Está tu logia en el centro
de la tierra?

Salvador sonriendo de nuevo, repitió:

---¡Chitón!

Habíamos entrado en un vasto recinto abovedado, que se extendía
considerablemente sin que la vista alcanzase a divisar el fin, dividido
por arcos de ladrillo desnudo. A un lado y otro, la escasa luz de la
linterna permitía distinguir multitud de objetos cuya forma no se
apreciaba claramente. Más que el objeto mismo, veíase la sombra de
ellos; disformes masas que se abrazaban unas a otras, o se repelían,
formando un conjunto semejante al de un gran montón de ruinas en la
penumbra de una noche de luna.

Salvador se detuvo y, poniéndose ante mí, me dijo:

---Bragas, estamos en los calabozos de la Inquisición.

\hypertarget{xvi}{%
\chapter{XVI}\label{xvi}}

Sentí que la sangre se me trocaba en hielo, los cabellos se me pusieron
de punta y por breve rato estuve sin respiración. Mi primer impulso,
cuando pude tener impulso, fue buscar con la vista un hueco por donde
echarme fuera de allí. Mi mayor confusión consistía en no poder asociar
estas dos ideas: la Inquisición y el Sr.~Mano de Mortero.

---No te asustes---dijo Monsalud;---aquí estamos tan seguros como en tu
casa.

Después de todo, esto no es tan feo como parece desde arriba.

Acudió en tropel a mi mente todo lo que había oído, visto y leído
referente al temible tribunal. Aquel solitario y lúgubre sitio en que me
encontraba desmentía un poco con su silencio y abandono las ideas de
espanto que invadieron mi cerebro, porque ni se oían lamentos, ni se
veían los humanos cuerpos arrastrando cadenas sobre el ensangrentado
suelo. Con todo, aquel lugar, bastante pavoroso por sí, lo era mucho más
desde que la fantasía lo asociaba a la tremenda Inquisición. No podía
uno menos de considerarse sepultado allí. No bastaba que la razón
dijera: estoy libre; el corazón se sentía estrechado por una mano de
bronce, y el cuerpo se reconocía cobarde hasta para huir.

Era imposible dejar de ver en los indefinidos objetos que obstruían el
paso horribles aparatos de tormento, que, como manos ávidas, alargaban
sus garfios para agarrarle a uno las carnes; era imposible dejar de ver
en movimiento toda aquella maquinaria infernal, y los apagados hornillos
encenderse, cual miradas del Infierno, ascuas que resplandecían
contemplando y llamando a sus víctimas; y los tornos girar,
zahiriéndolas con su irónico chirrido, semejante a pullas de vieja; y
los potros estirarse, deseosos de descoyuntarse a sí mismos mientras no
les dieran cuerpos humanos que desbaratar; y abrirse las cajas,
murmurando un gruñido sordo, como bostezo de Satanás, para cerrarse
luego, tragándose un cuerpo humano palpitante aún de rabia y dolor. Era
imposible dejar de ver brazos amenazadores, escuetas figuras de
angustia, semblantes doloridos, luengos trajes negros y garabateadas
dalmáticas de ignominia, monteras de papel llenas de gatos y diablillos
pintados, y horribles caperuzas sin rostro, con dos agujeros por donde
asomaba la Suprema sus insaciables ojos, buscando la herejía.

Al cabo de un rato de observaciones, distinguí varias puertas a un lado
y otro.

---¿Son esas las mazmorras donde están los presos?---pregunté a mi
amigo.

---Mazmorras son; pero no hay presos.

---¡Que no hay presos en la Inquisición!

---No: esto es ya una broma, un cachivache histórico que sólo asusta a
los niños de teta. Los dos o tres presos que hay están en el piso
segundo, y se pasean por los corredores tomando el sol.

---¿Y estos instrumentos de suplicio?

---Tú ves visiones: aquí no hay nada que sirva para dar tormento---dijo
Monsalud, dando un puntapié a una caja vacía que retumbó con lastimero
acento.---¿Ves esto? Pues es una caja de botellas de vino.

---Desechos de la comilona que tuvieron el otro día los señores---dijo
Mortero.

---¿Y aquellos maderos que allí se ven?---pregunté señalando unos palos
en cruz, cuyo aspecto me parecía el más siniestro que se podía imaginar.

---Es un catre de tijera colocado patas arriba.

---¿Y aquello que luce y parece metal?

---Un brasero viejo.

---¿Y aquello que tiene cadenas y unas como pesas?

---La garrucha vieja que estaba en el pozo del patio grande---repuso
Mortero.

---¿Y aquel cilindro horrible?

---Un tambor que servía al pregonero de la Bula.

---¿Y aquella argolla enorme?

---El aro de una pandereta con que jugaba en las Pascuas del año pasado
el niño del conserje.

---Por allí veo unas al modo de mandíbulas, que parece se van a comer a
todo el género humano.

---Si es un fuelle viejo sin cuero.

---Y una caperuza.

---Fue la que me puse el Carnaval pasado.

---Algunos cachivaches de tormento deben de quedar aquí---dijo Monsalud.

---Pero están hechos pedazos y cada pieza por su lado---repuso
Mortero.---Yo cojo todos los días madera y hierro para remendar las
guitarras, y hacer obra nueva. Si no fuera esto no tendría materiales
para la juguetería\ldots{} Hago caballitos, nacimientos, peonzas, aros,
ballestas y mil diversiones para los niños\ldots{} Lo que servía para
atormentar se lo llevaron hace poco a la cárcel de la Corona en la calle
de la Cabeza\ldots{} lo pidieron las comisiones de Estado\ldots{} Lo que
ahí queda, entre los ratones y yo lo acabaremos.

Después del temor que yo había experimentado, sufrió mi alma una
transición notoria: un vivo sentimiento de lo cómico se apoderó de mí.
Produjo estos efectos la disparidad que resultaba entre el terrible
tribunal, como la mente lo concebía, y la grotesca realidad de sus
calabozos; pero lo que principalmente había enfriado de súbito mi
terrorífica excitación, era la voz, el gesto, la figura del miserable
viejecillo, cuya persona en aquellas oscuridades inofensivas se asociaba
al siniestro \emph{exurge domine}. Era aquello como el despertar en
sainete después de haber soñado tragedias. Como alta torre que se
desploma, así cayó ante mis ojos el tremendo aparato fantástico de la
Inquisición de Corte, y roto el negro capuchón, aparecía desnudo el vil
mamarracho, cuya grotesca risa más inspiraba desprecio que horror.

---Pero ¿usted quién es?, ¿qué hace usted aquí?---pregunté a Mortero sin
poder refrenar mi curiosidad.

---Yo barro las salas bajas---respondió,---limpio el patio, hago
recadillos a los señores, les arreglo el calzado, subo agua, voy por una
onza de rapé, saco a paseo los niños del conserje, y remiendo y compongo
los sillones, las cajas, las mesas y la estantería del archivo.

Mirándole y recordando al fin su historia, no pude menos de echarme a
reír. Era un antiguo chalán del Rastro, contrabandista y capitán de
matuteros, gran maestro de las tomadoras del dos y hombre de empuje para
todas las empresas difíciles\footnote{Veáse \emph{Napoleón en
  Chamartín.}---1.\textsuperscript{a} serie, tomo 5.\textsuperscript{o}}.
Puestas a un lado las armas, cuando con la edad se acabaron a nuestro
héroe las fuerzas, se dedicó al comercio de las \emph{Américas}, o sea,
el tráfico del \emph{Nuevo Mundo}; que estos nombres tienen hacia el Sud
de Madrid las industrias de compra y venta establecidas en la Ribera de
Curtidores. Mano de Mortero tuvo mala suerte. Parece que la justicia dio
en decir que el almacén de aquel varón insigne se abastecía del hurto,
teniendo por principales acopiadores a todos los ladrones de la Corte.

¡Infame y vil calumnia! Víctima de ella, el pobrecito Mano de Mortero
hubiera sido indignamente perseguido sin la caritativa intervención de
los padres de la Merced que le tenían particular afecto; y no sólo le
libraron estos de las execrables garras de la justicia, sino que
lograron colocarle en un puesto humilde, pero honroso, dependiente de la
conserjería de la Inquisición de Corte. El sueldo era casi una limosna;
pero Mortero era Mortero y se las ingeniaba en aquellas profundidades.
Llevó toda su hacienda al lóbrego departamento que le destinaron y no le
faltaban industrias que ejercer. ¡Extrañas anomalías del siglo! La casa
de la Inquisición ofrecía un refugio al inválido de la matutería, al
insigne Aquiles retirado de las epopeyas del contrabando, al atleta de
las luchas con la autoridad civil. Cuando le hacían notar esta
coincidencia singular y el amparo que recibía en su vejez, decía
sonriendo:

---Buenos barriles de vino les he regalado en mis buenos tiempos. No
volvía nunca a Madrid de mis viajes sin traerles la sarta de chorizos,
la pieza de cotonía inglesa, el jamón de Portugal o las docenas de
pañuelos del Bearn\ldots{}

La Inquisición no era muy escrupulosa en aquellos tiempos para elegir el
bajo personal que le servía. Todo el mundo sabe que cuando la de Murcia
se encargó de los presos políticos después de fracasada la intentona de
Torrijos en 1817, tenía por carcelero a \emph{un gitano}. Fácil fue a
los conspiradores que no habían sido puestos a la sombra, salvar de la
prisión a sus compañeros. La respetable persona que los guardaba hizo lo
que puede suponerse. El historiador que se ocupa del gitano, dice que en
Madrid \emph{no estaba la Inquisición mejor servida que en Murcia}; pero
no nombra al insigne Mano de Mortero, sin duda porque este gitano era
más oscuro y subterráneo que el de Murcia. Lo que sí dice es que ciertos
\emph{conspiradores habían encontrado medio de penetrar en la
Inquisición desde una casa cercana}, a la cual por el mismo camino,
vamos a pasar ahora Monsalud, yo y mis lectores, si quieren por entre
estas tinieblas seguirme.

Pronto dejamos las bóvedas de la Inquisición, subimos otra escalera,
pasamos a un patiecillo, donde despidiéndonos cordialmente nos abandonó
el Sr.~Mano. Salvador llamó a la puerta que allí se veía, y abierta por
un hombre de aspecto común, nos encontramos en una casa, en una
verdadera casa, como todas las que habitamos los hombres. Me parecía
mentira que estaba ya fuera de la región de oscuridad y miedo.

---Aquí se respira, aquí se vive---dije a Salvador.

Después de atravesar varias piezas, llegamos a una en que había varios
estantes con libros, mapas, planos, esferas geográficas y otros objetos
que convidaban al estudio.

---¿Pero estamos en una academia?---pregunté.---Hemos pasado de la
Inquisición a los libros\ldots{} ¡Cuán cerca están el gato y el ratón!

---¿No ha venido nadie?---preguntó mi amigo al hombre que nos guiaba.

---Sí señor---repuso este.---Allá están los señores López Pinto,
Infante, Zorraquín y media docena de paisanos.

---¿Pero en dónde estamos?---pregunté con viva curiosidad cuando nos
dirigíamos al sitio que el portero, criado o lo que fuese designó
simplemente con la palabra \emph{allá}.

---¿No has oído decir que Su Majestad nombró en 1814 una Comisión de
oficiales del ejército, para que escribiese la \emph{Historia de la
guerra de la Independencia}?

---Sí. Dicen que la obra está atrasadilla.

---¿No sabes que se dio a la Comisión un edificio de Mostrencos para que
en él se reuniese, y con todo recogimiento y comodidad pudiera dedicarse
a sus trabajos?

---Sí, en la calle de la Flor Baja.

---Pues en esa calle y en el edificio de la Comisión estamos. Sólo que
los señores oficiales\ldots{}

---En vez de dedicarse a escribir, se dedican a conspirar. También lo
había oído decir. Pero hace poco, ¿no se disolvió la Comisión?

---Sí; pero ellos conservan las llaves del edificio y se reúnen aquí
algunas veces. Has de saber que esto no es logia masónica; es una junta
de patriotas. La iniciación es sencillísima, y basta ser presentado por
cualquiera de nosotros.

---Pero esta reunión\ldots{} ¿cómo la tolera el Gobierno?

Monsalud alzó los hombros.

---Yo creo que el Gobierno tiene noticia de ella; pero el Gobierno está
también minado, como está minada hasta la misma Inquisición.

---Por cierto que no acabo de explicarme\ldots{}

---A poco de frecuentar esta casa, descubrieron algunos que, haciendo
una pequeña obra, se podía pasar fácilmente por los sótanos del edifico
al cercano de la Inquisición. El arquitecto de estas viejísimas casas
previó la confusión que había de venir con los tiempos nuevos y el
trabajo socavador de las ideas que por todas partes se meten y toda
histórica muralla horadan. Logramos seducir primero a dos o tres
empleaduchos del Tribunal, y por último al conserje mismo. Hasta se me
figura que algún inquisidor debe de tener noticia de que solemos pasar
allá y revolverles un poco el archivo, pero no se atreve a decir nada,
porque nos tienen miedo.

---¡Miedo los inquisidores!

---O simpatía\ldots{} también puede ser. La Inquisición es hoy una cosa
que se aburre, un instituto infinitamente fastidiado de sí mismo. Sus
procesos son un bostezo. Si en los Tribunales de provincia se conserva
bastante rigor (testigo de ello, mi madre), el de Corte es una
decrepitud lela, un aburrimiento, como te he dicho, que anuncia la
paralización del sepulcro. Nos burlamos de este perplejo estafermo, que
se duerme con el azote en la mano. El tunante Mortero, convirtiendo en
juguetes para la industria los instrumentos de suplicio, te dirá más que
todos los razonamientos. Por cierto que no se ve tipo más truhanesco que
este antiguo chalán del Rastro, a quien la Inquisición ha dado asilo en
su casa. Una noche estaba yo en la habitación de él admirando sus
industrias y oyéndole contar graciosas historias, cuando vi entrar a
doña Fe. Mientras nosotros ganábamos al buen gitano, este había
explorado la vecindad y héchose amigo de tu sirvienta. Los dos se
entendían admirablemente. En prueba de ello, busca bien en tu casa y
encontrarás no pocos platos de menos.

---Ya lo he notado.

---Comprenderás que sentí curiosidad y deseos de entrar en tu casa, y
que, dado el carácter de Doña Fe, no me fue difícil conseguirlo.

---Tú mismo me dejaste el papel\ldots{} ¡Si supieras qué rato me hiciste
pasar\ldots!

---Esta noche entré como has visto y por los motivos que ya sabes. Vine
aquí después del lance ocurrido en mi casa, y hallándome en esta misma
sala, lleno de confusión, perplejidad y amargas dudas, resolví hacerte
una visita. Ya ves cuán fácil y natural explicación tiene lo que a ti te
ha parecido efecto de masónicos conjuros. No tengas por masones a Doña
Fe y al criado que ella misma te propuso; tenlos por dos grandes
tunantes; échalos a la calle y cuida mejor las puertas de tu casa.

---¡Vive Dios, que has hablado como un libro! Ahora dime qué vamos a
hacer aquí, y con qué clase de gente tenemos que habérnoslas.

---Ya te he dicho que esto es una reunión de patriotas pura y simple, no
una logia masónica. No esperes nada misterioso ni formulario. Eso lo hay
en otras partes; pero la revolución es tan urgente y tiene tanta prisa,
que ha dejado a un lado los floretes para tomar las espadas.

---Pues adelante; entremos.

\hypertarget{xvii}{%
\chapter{XVII}\label{xvii}}

Pasamos a una pieza grande, mejor amueblada que alumbrada, en la cual
había hasta diez personas. Algunas de ellas revelaban claramente su
profesión militar, aunque no tenían uniforme. Hablaban en alta voz con
gran algazara. Cuando Monsalud me presentó a ellos, diciendo mi nombre y
apellido con la añadidura de los cargos que había desempeñado, callaron
todos, y no se oyó más que un murmullo. Creeríase que mi nombre había
caído en la reunión como un jarro de agua en brasero encendido.

Pero el que llamaban Zorraquín, que parecía tener cierta superioridad
sobre los demás, se dignó hablarme con benevolencia.

---Las adhesiones de personas importantes que cada día recibimos---dijo
con petulancia,---prueban que el absolutismo se desmorona.

---Hemos llegado a un punto---repuse,---en que es indispensable tratar
de una revolución en el Gobierno. Yo no valgo nada. Usted me favorece
demasiado\ldots{} Doy a usted las gracias\ldots{}

Y luego para mi capote añadí:

---(¡Cuatro tiros te daría yo de buena gana, tunante!)

---Eso lo reconocen todos los hombres de talento---dijo otro de los
presentes.

---Yo mismo lo vengo sosteniendo---indiqué.---Público es y notorio que
he aconsejado a Su Majestad\ldots{} Pero a ese pobre señor\ldots{} a ese
pobre señor le han puesto una venda en los ojos, y es muy difícil
arrancársela. La corte debiera comprender su interés y transigir con
ustedes.

Y para mis adentros añadí:

(¡Qué bien os vendría un par de carreras de baqueta a cada uno!)

---La cosa ha llegado a tal extremo---dijo el que nombraban López
Pinto,---que ya son contados los personajes importantes que no están
dispuestos a ayudar a la revolución\ldots{} Pero vamos a lo positivo, y
ocupémonos de lo que nos ha reunido aquí. ¿Cómo es la gracia de ese
señor?

Yo di mi nombre, y lo apuntaron.

---¿Quién responde del Sr.~Pipaón?

---Yo respondo---dijo Monsalud.---Pero siguiendo la costumbre, se
extenderá un acta y él la firmará.

Maldita la gracia que me hacía poner mi nombre y rúbrica al pie de un
compromiso revolucionario; pero me acordé de las amonestaciones de D.
Antonio Ugarte, y eché mano a la pluma. En el documento constaba que,
admitido yo a la reunión y hecho partícipe del objeto y plan de ella, me
comprometía a cooperar en la obra revolucionaria. Firmaban cuatro además
del presentado y del presentador, y aquella hoja se unía al cartapacio
que uno de los militares llevaba siempre consigo.

Encabezaba el cuaderno una declaración importantísima, punto capital del
programa revolucionario, y era que aquellos señores y yo, desde tal
momento, prometíamos hacer todos los esfuerzos imaginables para derrocar
el absolutismo y restablecer la Constitución de Cádiz.

(Antes os derrocaría yo la cabeza---dije para mí mientras firmaba,
decorando mi faz con una sonrisilla.)

Con tan breve fórmula quedé armado caballero de la caballería
demagógica, sin más petada ni espaldarazo. Esta sencillez patriarcal no
dejó de llamarme la atención. Zorraquín me dijo:

---No todos los personajes importantes que se abrazan a la revolución,
tienen el valor de venir aquí. Muchos hay que trabajan desde sus casas,
en el mismo Palacio y en los Ministerios. Parece seguro---añadió,
bajando la voz---que el Sr.~Lozano de Torres es nuestro.

---Esta mañana le vi---dije yo,---y no sé por qué me pareció un poco
inflamado de ardor revolucionario.

---Es indudable que esta noche deja de ser ministro.

Empezó a entrar gente, y bien pronto la sala estuvo tan llena, que hacía
allí un calor sofocante. La animada conversación, las preguntas de fuego
sostenían también una elevada temperatura moral. Sorprendíanse algunos
de verme allí, y por mi parte no volvía de mi asombro al ver en tal
sitio a ciertas personas. Aquello tenía todo el aspecto de un club, y no
parecía que nos reuníamos para tratar una cuestión concreta, sino que
nos congregaba el deseo de desahogar por la vía oratoria las pasiones
políticas. Eran oídos los que más gritaban, y en ciertos momentos todos
hablaban a la vez, resultando que ninguno podía ser escuchado. Yo había
resuelto hacerme notar desde el primer momento, y como repetidas veces
me manifestaran deseos de que dijese alguna cosa, me subí sobre un
banco, y con gesto académico y cara sentimental, me expresé de este
modo:

---«Señores: Voy a hablaros con toda la franqueza propia de mi
carácter\ldots{} porque yo llevo siempre el corazón en los labios; yo no
conozco el disimulo; soy un hombre que hasta en sus defectos (pues tengo
muchos, dicho sea sin modestia) lleva el sello de la más pura
lealtad\ldots{} Señores, faltaría a esa misma lealtad de que blasono si
yo viniera aquí ahora haciéndome pasar por liberal de toda mi vida,
cantando himnos a la Constitución y apostrofando al absolutismo. Si eso
se me exigiera por la misma puerta por donde he entrado me marcharía,
con el corazón lleno de amargura, pero con la conciencia tranquila.
(\emph{Bien}, \emph{bien}.)

»No; yo no puedo presentarme aquí alardeando de servicios prestados a la
causa constitucional, ni afectando un entusiasmo tardío. Quédese eso en
buen hora para los que se vuelven siempre al sol que más calienta, para
los que adoran el triunfo, cualquiera que este sea. Yo diré más,
señores: yo levantaré ante vosotros, hombres honrados y leales, mi
cabeza humilde, pero honrada también, y diré: `Señores, he sido
absolutista; he servido al Gobierno absoluto; me he honrado con la
amistad de mi Soberano, a quien desde aquí respetuosamente saludo'. Diré
más aún; diré: `Yo he trabajado contra la revolución; he procurado
atajarla por cuantos medios estaban a mi alcance'. Pues bien, señores,
esta franca declaración mía, ¿no es una garantía de mis intenciones? ¿No
prueba que no soy un aventurero? ¿No indica claramente que traigo aquí
ideas de rectitud, de buen proceder, y sobre todo del más puro
patriotismo y lealtad? (\emph{Sí}, \emph{sí}.)

»Pero los que me escuchan dirán: `¿Cómo este hombre, que ha servido al
absolutismo, viene a servirnos ahora a nosotros?'. Se hablará de
defección, de inconsecuencia, de falta de lógica. No, señores, no, y mil
veces no. Yo he visto el abismo a que es rápidamente conducida la Nación
por hombres perversos; yo veo los graves, los hondos, los inmensos males
de la patria; veo a la corte desbocada, digámoslo así, por un carril de
males; la veo tocando ya al término de la perdición, de la ruina. Hago
esfuerzos para salvarla, y no puedo; quiero detenerla, y me atropella;
le grito, y no oye. ¿Qué hacer, señores, qué hacer? ¿Cruzarme de brazos
y contemplar con fría imperturbabilidad el desdoro y la destrucción de
mi patria? ¿Encerrarme en mi egoísmo, no ver más que mi propia persona y
dejar que la revolución y el absolutismo se despedacen en feroz
encuentro? ¡Oh!, no, señores, y mil veces no. Los que tenemos un corazón
que nace al dulce nombre de la patria; los que hacemos nuestras las
alegrías y las penas de la tierra en que hemos nacido, no podemos
proceder de esa manera. Una voz dolorida suena en nuestro cerebro, y el
corazón palpita al representarse las angustias de la patria agonizante.
Bendita seas una y mil veces ¡oh patria generosa, bella y desdichada!
¡Bendita seas, y malditos los que no estén prontos a derramar por ti la
última gota de su sangre! (\emph{Emoción general}.)

Tuve que detenerme, porque yo también me conmovía y la voz se ahogaba en
mi garganta.

---Perdonadme, señores---continué, reponiéndome y pasando el pañuelo por
mis ojos;---perdonadme si mis palabras desdicen de la gravedad de este
lugar, si me dejo llevar de sentimientos\ldots{} Porque sin
quererlo\ldots{} casi me he puesto en ridículo. (\emph{No, no; que
siga}.) No puedo tratar de ciertos asuntos sin mostrar toda la
sensibilidad de mi corazón\ldots{} Pues decía, señores, que un hombre
honrado no puede permanecer tranquilo en presencia de los males
gravísimos que todos conocemos. Yo, como otros muchos, he fijado los
ojos en la idea que bullía en estos lugares secretos. Por lo mismo que
la combatí, reconozco su poder; ¿a qué negarlo? Nadie se atreverá a
sostener que la idea liberal es mala en sí; nadie, nadie. Yo mismo, que
la he combatido, he dicho, fijaos bien, señores; he dicho que la idea
liberal y aun la Constitución del 12 podían ser de provecho en
determinado día\ldots{} Pues ¿quién duda eso? Estableciose el
absolutismo cuando era natural y lógico que se estableciera, porque la
desorganización nacional, consecuencia lógica de la guerra, exigía una
unidad poderosa que amalgamara los elementos dispersos. Pero el
absolutismo, entiéndase bien esta idea, que yo he sostenido siempre, no
podía considerarse sino como transitorio, como una obra de las
circunstancias. Bien claro lo dice el Manifiesto del 4 de Mayo de 1814.
Pues bien; así como fue natural y lógico establecer el absolutismo,
entiéndase bien, señores, ahora es lógico y naturalísimo que el
absolutismo cese\ldots{} No; España no puede continuar por más tiempo
siendo una excepción en Europa. No sólo Luis XVIII, sino también
Alejandro, el autócrata ruso, ha aconsejado a nuestro Rey la adopción de
una Carta constitucional. Esto es lógico; los tiempos lo reclaman, el
país lo pide a grito herido; porque el país, señores, tiene mejor que
nadie el instinto de su conveniencia; y así como aplaudió hace cinco
años el absolutismo, aplaudirá después el Gobierno liberal, sabiamente
establecido. Y ahora pregunto yo: en estas ideas que he vertido, y que
son norma de mi conducta, ¿hay defección, hay inconsecuencia, hay falta
de formalidad? (\emph{No}, \emph{no}.)

»Repito que yo no vengo aquí a proclamarme revolucionario rabioso. No
soy ni siquiera revolucionario. Mi sistema político se funda en un orden
perfecto, en una concordia preciosa. Gobierno prudente y liberal;
reformas sabias; respeto a Su Majestad; orden, mucho orden. Si se trata
de escándalos, de disturbios sangrientos, me marcharé por donde he
venido, e iré a llorar en la soledad de mi retiro los males de la patria
y los errores y la ceguera de mis conciudadanos. (\emph{Muy bien}.) No
me pidan manifestaciones calurosas. Trabajaré por el cambio de Gobierno.
Trabajaré con ardor y celo, pero sin demostrar esa vana oficiosidad de
los que se unen a las revoluciones para desacreditarlas, mientras sacan
provecho de ellas. Yo no quiero provecho; yo quiero ser el primero en el
trabajo y el último en la recompensa. Quiero ser el último, señores;
quiero permanecer en la oscuridad el día del triunfo. El que no se
acuerde de mí en dicho día, me hará el mejor servicio que puedo
apetecer. Ruego a todos los presentes que no vean en mi más que un
hombre oscuro, que podrá equivocarse, que se ha equivocado tal vez, pero
que jamás ha fingido sentimientos ni ideas que no sintiera. Con la misma
lealtad y franqueza con que expuse antes mis servicios al absolutismo,
declaro ahora que creo en el triunfo de las ideas liberales. Yo no
engaño, yo no finjo, yo no hago papeles diversos; yo no tengo
entusiasmos hoy, frialdades mañana y veleidad y novelería siempre; en
una palabra, yo no sirvo a partidos, ni a pandillas, ni a poderes, ni a
reyes, sino a la madre que reverencio y adoro, a la patria idolatrada,
objeto de todas mis ansias, de todos mis desvelos, de todos mis amores.
Fijos los ojos en la patria, exclamo: \emph{Joven libertad, yo te
saludo}. He dicho».

Concluí mi discurso entre señales de aprobación tan manifiestas y
calurosas, que, a pesar de estar yo en el secreto, como autor de la
pieza oratoria que acaba de leerse, no pude menos de admirarme a mí
mismo. Mi discurso, dicho sea sin modestia, era un modelo en ese género
resbaladizo, flexible y acomodaticio, que sirve, mediante hábiles
perfidias de lógica y de estilo, para defender todas las ideas y pasar
de uno a otro campo. Era un modelo en lo que podemos llamar el género de
la transición. Yo descubría maravillosas facultades para la política.

Los buenos revolucionarios, al aplaudirme y admirarme irreflexivamente
sin reparar mis antecedentes, no hacían mas que cumplir las condiciones
inevitables de su carácter, que eran candor y generosidad. La mayor
parte de ellos tenían una buena fe excesiva, y abrían los brazos a todo
el mundo, viniera de donde viniese. Dejábanse cautivar por los discursos
amañados y retumbantes, sin reparar de qué boca salían, dándose el caso
aquella noche de que a un hombre como yo le festejaran, considerándole
como una esperanza de la joven libertad, a quien ardientemente saludara.

Otros hablaron después que yo; pero no se oyeron más que discursos
violentos, sin aquella mesura y espíritu práctico y justo medio y
prudencia y pulso que resplandecían en el mío. Yo hablé como hombre de
gobierno: ellos como agitadores desalmados. Yo hablé desde un terreno en
que fácilmente se podía volver la vista al absolutismo y al
constitucionalismo, vistiendo al uno con los trajes del otro, según
conviniera; ellos quemaban sus atrevidas naves, declarándose jacobinos.
¡Diferencia notable! El porvenir era mío. Ellos morirían despedazados
por sí propios.

Últimamente la reunión se dividió en grupos, y hablaban todos a un
tiempo. Yo advertí que Monsalud, Zorraquín y otros habían desaparecido
después de mi presentación, sin oír mi discurso, y curioso por saber
dónde se escondían, lo pregunté a un señor ex-colector de Espolios que
conmigo charlaba.

---Están en la sala inmediata---me dijo.---Esas cabezas de la
conspiración deliberan secretamente. Para pasar allí es preciso haber
trabajado mucho y servido bien a la causa. Creo que esta noche hay
noticias importantes: ya nos las dirán. Se dice que va a salir al
momento un comisionado para Andalucía.

Uno que parecía militar de elevada graduación se acercó y nos dijo:

---Se asegura que esta noche misma vendrá aquí por primera vez a
inscribirse y a comprometerse D. Juan Esteban Lozano de Torres.

---¡Hombre!\ldots{} ¡Tan pronto!\ldots---exclamé yo.

---Sr.~de Pipaón, aprendamos a ver claro y a no juzgar a las personas
por lo que aparentan. Yo mismo he visto a Lozano en la logia masónica de
la calle de las Tres Cruces.

---La verdadera masonería dicen que no es revolucionaria.

---Hay de todo; por ahí se empieza.

---No: no es que yo ponga mi mano en el fuego por la pureza
antirrevolucionaria de D. Juan Esteban---dije.---Él, como todos
nosotros, habrá comprendido que es imposible sostener el
absolutismo\ldots{} Quien no se dejará bautizar fácilmente con estas
aguas, amigo, es el señor marqués de M***, a quien se indica para
sucesor de Lozano.

---También lo creo así. El marqués de M*** no será de los nuestros hasta
que no triunfemos. Su anticonstitucionalismo consiste en que no cree en
la posibilidad de la caída. Allá veremos. Me temo que si entra ese señor
en el Ministerio, sea esta la última noche en que nos reunamos aquí.

---Es posible.

---Pero no faltará un agujero. Madrid es muy grande, y la policía, en su
previsión incomparable, no deja de simpatizar con las sociedades
secretas. Felizmente ahora se han reunido fondos\ldots{}

---La cosa---dijo el militar, dando a esta palabra (cosa) el sentido
revolucionario que siempre tiene en vísperas de trastornos---vendrá esta
vez de Andalucía.

---Sí; esta noche misma sale un comisionado para allá. El ejército de la
Isla y las tropas que con motivo de la fiebre están acantonadas en las
Cabezas de San Juan, serán las que nos saquen de penas.

---Conozco a algunos jefes---indiqué.

---Y yo a todos---dijo el militar.

---¿A Rafael del Riego?\ldots{}

---De ese no puede esperarse gran cosa. Es un hombre que por milagro de
Dios sabe leer y escribir.

---Mucho corazón.

---Regular nada más. En lengua sí le ganan poco. Es de los que más
hablan y de los que menos hacen.

De improviso entró en la reunión un hombre a quien yo había visto mucho
en Palacio, y que aun en aquella época privaba mucho con Ramírez de
Arellano y Villar Frontín.

---Señores---gritó con voz estentórea,---el marqués de M*** es ministro
de Gracia y Justicia.

---¡Viva Lozano de Torres!---exclamó uno de los presentes.

---Su Excelencia ha salido desterrado para el castillo de San Antón de
la Coruña.

---No podía faltar el paseíto---dijo el ex-colector.

---Ahora mucho cuidado. El Sr.~D. Buenaventura nos enviará aquí sus
perros. Ya no tendremos un jefe de policía que ampare la reunión.

La conversación se animó. Hubo amenazas, promesas, votos, juramentos y
proyectos. Yo me mantenía siempre en una actitud de dignidad y reserva,
como hombre amante del justo medio y enemigo de escándalos. Se respiraba
allí una atmósfera de pasión que no era la más a propósito para mí y
empecé a sentir hastío. Sin embargo de esto, hice aquella noche algunas
amistades. ¡Cuántos hombres conocidos encontré allí y con cuántos
desconocidos trabé relaciones! Había gran número de personas muy
notorias por su probidad, por su honrada vida en el comercio y en la
industria; había altos empleados que sirvieron o servían aún con buena
nota; liberales exaltados que llevaban en sus manos la señal de las
esposas del presidio, revolucionarios frenéticos y templados, hombres de
ideas nobles y hombres de acción ruda, personas sencillas las unas,
inteligentes y astutas las otras, la violencia y la persuasión, la
sencillez y la anarquía. Para que nada faltase, vi algunos que se habían
distinguido en los seis años por su absolutismo furibundo. El pan que
iba a salir de aquel amasijo, sólo Dios lo sabía.

Al fin aparecieron los que se ocultaron al principio de la sesión, y
Zorraquín dijo:

---Señores, es preciso que nos retiremos. La entrada del marqués de M***
en el ministerio nos quita toda seguridad, y esta casa puede ser
registrada cuando menos se piense. Si el Sr.~Lozano no nos protegía
abiertamente, me consta que hacía la vista gorda; es decir, que no
quería meterse con nosotros, y perseguía tan sólo a nuestros agentes. El
\emph{Tigre} no hará lo que el \emph{Zorro} y dirigirá sus golpes a lo
alto. Quizás a esta hora estén cambiados los agentes de policía.
Precaución, pues, y cada cual a su casa. Se avisará.

Lentamente fueron desfilando todos. Hubo despedidas cariñosas, apretones
de mano, promesas, citas particulares para el día siguiente. Todo era
concordia y entrañable afecto. Monsalud y yo nos quedamos los últimos.
Riéndome, no sé si de mí mismo o de qué, le dije:

---¿Con que soy masón?

---Masón no---me respondió.---La masonería, propiamente dicha, no es
revolucionaria, aunque el vulgo y los absolutistas llaman masones a los
que conspiran. Ya te dije que esto no es una logia, sino una reunión; lo
que en Francia llaman un club.

---¿De modo que no soy todavía masón, propiamente dicho? Pues bien, soy
liberal.

\hypertarget{xviii}{%
\chapter{XVIII}\label{xviii}}

Y rompí a reír con más fuerza. La revolución individual se había
consumado en mí. La segunda casaca, no menos ridícula a mis ojos que la
ropilla encarnada de un bufón, pesaba sobre mis hombros.

---Una cosa no me ha gustado Salvador---le dije cuando salimos a la
calle,---y es que han tratado ustedes secretamente lo más importante de
la reunión. ¿Por qué no había de cooperar yo con mis consejos a lo que
se está tramando?

---¿Acabas de sentar plaza y ya pretendes ser general?

---Qué quieres\ldots{} yo soy así\ldots{} Pero, ¿a dónde vamos ahora?

---Adonde gustes. Yo tengo que salir para Andalucía al rayar el día,
quisiera tomar alguna cosa y descansar un poco.

---¡Ah!, eres tú el comisionado que va a Andalucía---exclamé con
viveza.---Dicen que vendrá de allí eso que llaman \emph{la cosa}. ¿Vas a
llevarles dinero o instrucciones? Se me figura que de todo llevarás.

---Mucho quieres saber en poco tiempo---me dijo.---Te advierto que nunca
he sido indiscreto. Sigue concurriendo a la reunión, muéstrate activo y
servicial, y pondrás tus manos en la masa fina.

---Tienes razón, no debo ser curioso. Pero dime tú que estás en los
secretos, ¿la revolución vendrá pronto?

---Aunque no tengo la fe ciega de otros, creo que esta vez ha de
resultar algo de provecho. Se ha trabajado tanto, se ha llevado el hilo
de la conjuración a tantas partes, que a poco que de él se tire habrá
movimiento en diversos puntos, y cuando el Gobierno quiera cortarlo, se
enredará en él.

---Por lo que veo y por lo que he oído, tú eres de los que más han
trabajado en estos líos---dije procurando ganarme toda la simpatía de mi
amigo.---Desde la conspiración de Porlier andas en danza, Salvadorcillo,
según lo prueba la hoja de servicios que me enseñó Lozano de Torres.
¿Sabes que por mucho que te den el día del triunfo, no habrá bastante
con que recompensarte?

---Yo no trabajo por recompensas, amigo Bragas---replicó;---trabajo por
una pasión irresistible que me ocupa todo desde que me vi maldecido por
mi patria y arrojado al suelo extranjero como una bestia maligna. Esta
pasión es la que me impele, es la que me mueve, haciéndome infatigable;
la que me hace afrontar todos los peligros y despreciar la muerte, a que
mil veces estuve expuesto.

---Yo también tengo una verdadera pasión porque mejore la suerte de mi
querida patria. Salvador, entre tú y yo hemos de hacer algo muy sonado.

---Mi ambición y la tuya son muy distintas. Tú has empezado a creer que
esto va mal desde que has empezado a perder tu valimiento. Yo he creído
siempre lo mismo, y mucho me temo que, aun después del triunfo, sigan
pareciéndome las cosas de mi país tan malas como antes. Esto es un
conjunto tan horrible de ignorancia, de mala fe, de corrupción, de
debilidad, que recelo que esté el mal demasiado hondo, para que lo
puedan remediar los revolucionarios. Entre estos se ve de todo; hay
hombres de mucho mérito, buenas cabezas, corazones de oro; pero así
mismo los hay tan vanos como bullangueros, que buscan el ruido y el
tumulto, no faltando algunos que están llenos de buena fe; pero carecen
de luces y de sentido común. Yo he observado este conjunto en que se
revuelven, sin poderse unir, la grandeza de las ideas con la mezquindad
de las ambiciones; he sentido al principio cierto temor; pero después de
meditarlo, he concluido afirmando que los males que pueda traer la
revolución no serán nunca tan grandes como los del absolutismo. Y si lo
son---continuó desdeñosamente,---bien merecidos los tienen. Si esto ha
de seguir llevando el nombre de Nación, es preciso que en ella se vuelva
lo de abajo arriba y lo de arriba abajo, que el sentido común ultrajado
se vengue, arrastrando y despedazando tanto ídolo ridículo, tanta
necedad y barbarie erigidas en instituciones vivas; es preciso que haya
una renovación tal de la patria, que nada de lo antiguo subsista, y se
hunda todo con estrépito, aplastando a los estúpidos que se obstinan en
sostener sobre sus hombros una fábrica caduca. Y esto se ha de hacer de
repente, con violencia, porque si no se hace así no se hace nunca. Ya
sabemos lo que son las promesas hechas en manifiesto durante los días de
miedo. Aquí se han de romper a hachazos las puertas de la tiranía para
destruirlas, porque si las abrimos con ganzúa o con su propia llave,
quedarán en pie y volverán a cerrarse.

---Salvador, me espantan tus ideas---dije yo, no pudiendo renunciar a mi
papel de sustentador del orden social.

---Pues acabas de comprometerte a defender estas ideas que tanto te
espantan. Si quieres que siga gobernando a una Nación como esta el
capricho de un Rey o la ambición infame de media docena de lacayos; si
quieres que todo el manejo de la fortuna del Reino esté al arbitrio de
una mujerzuela o de un palaciego adulador; si quieres que la parte
principal de la riqueza del país sea chupada por un enjambre de
holgazanes corrompidos, sin ley de Dios ni de los hombres; si quieres
que la ignorancia y la barbarie de los pueblos sean ley del Estado, y
que se proscriban los libros como una plaga; si quieres que un capellán
de monjas más estúpido, aunque menos gracioso que fray Gerundio, ponga
su veto a las obras del entendimiento más sublime; si quieres que siga
este envilecimiento en que tantos seres viven, gobernados como carneros,
y sin saber ni pedir cuenta de su conducta a los que les gobiernan; si
quieres que todos los hombres eminentes se mueran de miseria y dolor en
los calabozos o en los presidios de África, y que los mejores títulos
para escalar las altas posiciones sean aquí la adulación, la bajeza, la
nulidad, la ignorancia, la intriga; si quieres esto, Pipaón, ¿para qué
has salido de Palacio y has entrado en el club?

---Veo, amigo Salvador---le dije con complacencia,---que has aprendido
en la emigración muchas cosas que antes no sabías.

---La desgracia abre los ojos---me contestó,---y la desgracia en países
que son una perpetua lección para el nuestro, es la mejor maestra que se
conoce. Tengo fe inmensa en el éxito definitivo de mis ideas; tengo la
creencia de que al fin y al cabo triunfarán, y serán tan comunes a todos
como son hoy comunes la ignorancia y la ceguera de una gran parte de los
españoles.

---De modo que ahora\ldots{}

---Ahora, si he de hablarte con franqueza, no creo yo que las ideas
liberales sean bien comprendidas, ni menos bien practicadas.

---Es decir, que serán una calamidad.

---Hasta cierto punto, sí.

---Entonces los que las predican hacen mal, y los que tratan de
establecer el sistema liberal, peor.

---No, porque alguna vez se ha de empezar.

---El pueblo necesita ser ilustrado para poder practicar la libertad.

---Y necesita practicar la libertad para ilustrarse. Parece que esto es
un círculo vicioso; pero no lo es realmente. ¿Por dónde se empieza? Esta
es la cuestión. Comprenderás que todas las cosas tienen su principio
doloroso. El hombre antes de andar en dos pies, ha andado a gatas.
Supongo que por evitarte los tropezones que acompañan a los primeros
pasos, no desearás tú que el género humano ande siempre a cuatro pies.

---Ciertamente que no.

---En ese período estamos, amigo.

---¿En el de los cuatro pies?

---Exactamente. Yo le digo a la sociedad española: «levántate», y me
responde: «no sé andar derecha». Los frailes y los palaciegos le
aconsejan que no se meta en la peligrosísima aventura de marchar como la
gente. Al fin le azuzamos tanto, que se levanta.

---¡Y a los pocos pasos, al suelo!

---Pero la estimulamos de nuevo con ruegos, o a latigazos, si es
preciso. Afligida, repite ella: «Si no sé, si me caigo, ¿qué debo hacer
para aprender a andar?» Y le contestamos: «Andar, andar siempre».

---Bien, muy bien, Sr.~Monsalud---dije riendo.---Dios quiera que el
tropezón que vamos a dar ahora no sea tal, que nos rompamos las
narices\ldots{}

---Y andará, al fin tiene que andar---añadió.---Decirte cuánto he
trabajado por que llegue el día del triunfo; pintarte los peligros que
he corrido, y la extraordinaria constancia mía al inaugurar una
tentativa al pie mismo de los cadalsos donde ha expirado la anterior,
sería imposible. Esta fuerza, este afán incesante, sin desmayar nunca,
sin desconfiar del éxito, a pesar de las repetidas contrariedades que
han agobiado y descorazonado a tantos, no se tiene sino cuando el alma
está llena y ocupada por esas ardientes y potentes ideas, por las
pasiones políticas que alientan y queman. Para desafiar la muerte es
preciso no temerla, y este arrojo imperturbable, sólo cabe en corazones
limpios de toda ambición pequeña.

---Comprendo que los trabajos han sido muchos; pero no me hables de los
peligros, porque no creo en ellos. Pues qué, ¿no es sabido que los
conspiradores y masones o lo que sean, burlan la policía y la justicia,
cual si estuviesen de acuerdo con el Gobierno?

---Te diré: es cierto que hoy se ha relajado considerablemente la
justicia; pero es porque al Gobierno le ha entrado ya el mareo de la
perdición, le ha entrado el aturdimiento que indica su próxima ruina. El
absolutismo mismo, esa fiera indócil e incapaz de benignidad, parece
como que quiere congraciarse con la revolución. Esto no es tolerancia,
Pipaón, esto es cobardía\ldots{} Recuerda que Porlier fue ahorcado, Lacy
fusilado y Vidal y sus infelices compañeros inmolados también en un
aparato lúgubre que indica la crueldad más refinada\ldots{} Hoy el
absolutismo no ahorca; más no porque no sepa hacerlo. Ahora le toca a él
tener miedo\ldots{} Sin embargo, la impunidad que hoy disfrutan los
revoltosos, tiene sus límites. Cierto que hacen su voluntad y conspiran
una multitud de personajes que han ocupado altos puestos o los ocupan
hoy. Con estos transigirá siempre el Gobierno, porque no es cosa de
meter en la cárcel a un Consejero de Estado o a un capitán general. Con
los que el absolutismo no transige es con los que, como yo, no son ni
siquiera sargentos, ni siquiera covachuelos, y se atreven, sin embargo,
a atentar contra lo existente. Para los que no somos nada, la impunidad
no existe. Otros, si son cogidos, sufrirán pequeño arresto, o una
detención insignificante, recibiendo algún recadito del Ministro, de tal
dama, o de cual palaciego: en cambio yo y otros como yo, si somos
cogidos, lo pasaremos mal.

---¿No eres amigo del Sr.~Villela?

---Pero el Sr.~Villela, aunque conspira, conspira a lo cortesano, y es
esclavo de las conveniencias. Es mi amigo, pero sólo hasta cierto punto,
y en tanto cuanto no se comprometa por mí. No creas que me fiaría del
Elefante en un caso de apuro. Los protectores y cómplices de la Corte
sirven de poco. ¿Piensas que me hubiera sido fácil escapar de las garras
del marqués de M*** si por desgracia hubiera caído en ellas esta noche?

---Tú me has dicho que has sobornado a muchos polizontes, y por lo que
Zorraquín me indicó, se comprende que la policía no os molestará mucho.

---Pero no estoy libre de la policía de la Inquisición---añadió
Salvador,---lo cual es muy distinto.

---Hace poco, cuando estábamos en aquellos sótanos tan apacibles, me
dijiste que la Inquisición era una burla, un fantasma.

---Una burla y un fantasma porque no es lo que era, es decir, porque no
quema, ni descuartiza, ni descoyunta, pero aún tiene presos y alguna vez
se da el gustazo de atormentar. Si he de hablarte con franqueza, en este
período de perdición y desvanecimiento en que ha entrado el absolutismo,
no temo ni que me ahorquen ni que me fusilen, porque además de la
flojedad del Gobierno, no faltaría quien me salvase; pero temo las
molestias, y sobre todo la falta de libertad. Por eso varío de domicilio
con tanta frecuencia, con objeto de evitar a los infames hurones que
olfatean la revolución, faltos de valor para destruirla. Por eso he
organizado una especie de policía a mi manera, la cual me permite
conocer gran parte de lo que pasa en los ministerios y en Palacio, en la
Corte y fuera de ella.

---¡Admirable habilidad la tuya! Por lo que has hecho en mi casa, juzgo
de lo demás---le dije.---Ya no me sorprende que tuvieras noticia de la
orden secreta dada por el Supremo Consejo para poner en libertad a tu
madre, ni que sepas la venida de Carlos Navarro, cuando su misma mujer
no sabe lo que hace.

---Eso lo sé por un amigo llegado ayer.

---Mientras más hablo contigo, más me alegro de renovar nuestra antigua
amistad---le dije cariñosamente y con franqueza.---Creo que entre los
dos podremos hacer algo de provecho. Sigamos nuestras relaciones\ldots{}
escríbeme\ldots{} Quiero saber día por día cómo va nuestra querida
revolución\ldots{} porque yo, Salvador, soy todo tuyo.

---Entusiasmado estás. Veremos si dentro de algún tiempo dices lo
mismo---me contestó deteniéndose.

Habíamos llegado a la Puerta del Sol y junto al café de Levante.

---¿Es hora ya de que nos separemos?---le pregunté.

---Sí; te ruego que no me acompañes más. Ahora necesito estar solo.

---¿Y no puedo seguir en tu agradabilísima compañía hasta el momento en
que te pongas en camino?

---No, querido Pipaón. Ahora deseo quedarme solo. Unos amigos me esperan
aquí. Tengo que arreglar mi viaje. Con que\ldots{}

---¡Pues adiós, ilustre y heroico joven!---le dije
abrazándole.---¡Cuántas cosas han pasado desde que te apareciste en mi
casa! ¡Qué nuevo mundo de ideas! Entre morir y resucitar no hay tanta
diferencia. ¡Si me parece que he vuelto a nacer!\ldots{} Soy otro,
Salvador.

---Falta que seas consecuente, que comprendas bien la gravedad de tu
misión ahora.

---Tomándote por modelo, mi querido amigo, no me equivocaré\ldots{}
¡Venga otro abrazo\ldots{} otro! Si no me canso de abrazarte. Que
vuelvas pronto y nos traigas la revolución. ¡Oh!, ¡la
revolución!\ldots{}

---Adiós\ldots{}

---Soy todo tuyo\ldots{} todo tuyo y de la libertad. Adiós.

Nos separamos. Yo corrí a mi casa. El frío de la madrugada, azotándome
el rostro, obligábame a marchar velozmente como un ladrón que huye o un
amante que acude a la cita.

Gran asombro me causó hallar a Jenara levantada. Su palidez indicaba
doloroso insomnio. Tenía en los ojos un exceso de atención y de vida,
semejante a los primeros síntomas del delirio mental.

---¿Cómo es eso?\ldots{} ¿En pie a estas horas?---le dije.

---Gusto de madrugar---me respondió, señalando las ventanas, por donde
entraban las primeras luces del día.---Vea usted. Ya amanece.

---¡Ah!, señora---exclamé compungidamente.---Vengo de cumplir el más
penoso de los deberes\ldots{} ¡Terrible trance que ha llenado de
angustia mi corazón!\ldots{} pero en fin, el deber es lo primero.

---¿De qué habla usted?

---¡Y me lo pregunta! ¡Y se hace la ignorante!\ldots{} Pues qué,
¿necesito decir que ese miserable enemigo nuestro se halla en poder de
la justicia, que bien pronto, ¡oh dolorosa y tristísima idea!, le hará
expiar sus nefandos delitos?

---¿El que estaba aquí?\ldots---preguntó, venciendo su perplejidad.

---Pero, Jenara, ¿es posible que no haya comprendido usted mi intención
y el gran celo con que esta noche la he servido?

---¿A mí?

---¡A usted! Francamente, amiga mía, sólo por usted, sólo por el gran
amor que profeso a su familia, he podido yo acometer la penosa empresa
de esta noche\ldots{} Le aseguro que mi corazón está destrozado.

---Nada comprendo. Sólo sé que, después de charlar en confianza,
salieron ustedes juntos.

---¿Y lo demás, es preciso decirlo letra por letra?\ldots{} ¡Qué tonta
es la niña!\ldots{} ¿Pues no se comprende que si salí con él fue para
llevarlo astutamente y con sutil engaño a un punto donde no pudiera
hacer ninguna resistencia?\ldots{}

---¡Para prenderle!---exclamó con asombro.

---Pues es claro\ldots{} ¡Y se asombra!\ldots{} ¿Pues no era este el
gran empeño de usted?\ldots{} El infeliz, al escapar de la emboscada que
le prepararon en su casa, creyó encontrar refugio y amparo en la mía;
pero se la he pegado bien\ldots{} Fingiendo conducirle a paraje seguro,
le puse entre los dientes del dragón. Con que, señora mía, los vivos
deseos de usted están satisfechos. ¿Me he portado bien?

---De modo, que fingiéndose amigo\ldots{}

---Eso es, fingiendo que le protegía, le entregué a los sayones de don
Buenaventura, que darán cuenta de él.

---¡Qué felonía!---exclamó con arranque tan espontáneo que me
desconcerté.

Después, tratando de reponerse, me dijo:

---Pero más vale así, para que no se pierda mi trabajo.

---¡Ah!, lo que es esta vez subirá al cadalso, estoy seguro de
ello\ldots{} Pero noto en el semblante de usted síntomas de lástima,
Jenara.

Y era verdad que los notaba.

---Justicia y generosidad no se excluyen---me respondió.---Ya he dicho
que detesto al delincuente, pero que compadezco al encausado.

---Estoy notando que en el espíritu de usted se encadenan de una manera
misteriosa el odio y la compasión---le dije.---De tal manera las
pasiones humanas, originándose las unas a las otras, llevan el alma a
extremos lamentables.

---¿Dice usted que ahora no escapará?

---Pero, ¿no sabe usted que el marqués de M*** está en el ministerio?
Con esto se ha dicho todo. Lo ahorcarán sin remedio, y pronto, muy
pronto. Ya se acabó la impunidad de los agitadores y jacobinos. Por
cierto, Jenarita, que usted y yo nos hemos lucido. ¡Qué gran servicio
hemos prestado a la patria! Lástima grande que no siguiera usted
descubriendo criminales y yo echándoles el guante.

Dirigiome una mirada rencorosa. Arrojándose en un sillón, apoyaba su
frente en la palma de la mano.

---Cuando se pasa la noche sin dormir---dijo,---la cabeza es de plomo.

---¡Noche de emociones!---indiqué.---Yo sí que las he tenido buenas.
Figúrese usted\ldots{} ¡Tener que vender a un hombre de quien uno ha
sido amigo!\ldots{} ¡Entregarle a la justicia!\ldots{}
¡Engañarle!\ldots{} ¡es horrible!\ldots{} Y todo lo he hecho por usted,
Jenara, por complacerla, por dejar satisfechas esas violentas pasiones
de la mujer más caprichosa de la tierra.

---Mi abuelo dice que ya no ahorcan a nadie---indicó, fijando en mí sus
ojos que pedían no sé qué desconocida misericordia.

---¿Se inclina usted a la generosidad? ¿Venimos ahora con blanduras? Las
mujeres\ldots{} nunca se sabe lo que quieren.

---No\ldots{} dejémonos de generosidades humillantes.

---Eso es\ldots{} palo en él\ldots{} duro. Sea usted como yo,
inexorable.

---Sí---dijo Jenara, levantándose y mostrándome su rostro teñido
súbitamente de apasionados fulgores.---Sí, la palabra de estos tiempos,
el lema de mi familia debe ser: ¡castigo!

---¡Castigo! Sí. ¡Qué bien he interpretado el deseo de usted!

---Mi deseo es\ldots{} ¡que muera!

Descargó la trágica mano en el aire, y su hermoso semblante lleno de
luz, de majestad, de inexplicable imán de amores, se entenebreció con el
ceño propio de una divinidad ofendida y vengadora.

Al mismo tiempo sonaron voces en la puerta de la casa.

---¡Mi marido!---gritó la dama.

Después de breve pausa de confusión y estupor, Jenara corrió al
encuentro de Carlos Navarro, que acababa de llegar en compañía de dos
amigos, dos guerrilleros barbudos, dos salvajes de voz dura y miradas
terribles y cuerpos y voluntades de acero.

Un instante después de su llegada, yo me colgaba al cuello de Carlos
Garrote y estrechándole ardorosamente hasta sofocarle, le decía con voz
conmovida:

---Bien venido sea, bien venido sea el insigne guerrero\ldots{} ¡Gracias
a Dios!\ldots{} No podía usted venir más a tiempo. ¡Parece que le envía
el cielo, ahora que levanta por todas partes su cabeza la hidra
revolucionaria; ahora que bullen las infames sociedades secretas y está
Madrid plagado de miserables conspiradores y masones, los cuales con
horrible alevosía tratan de hacer una revolución\ldots{} ¡oportunidad
admirable!

---¿Revolución? Lo veremos---dijo con acrimonia Carlos, correspondiendo
afectuosamente a mis demostraciones.

\hypertarget{xix}{%
\chapter{XIX}\label{xix}}

Carlos Navarro, al día siguiente de su llegada, me notificó que su
familia abandonaba mi casa. Además de que no parecía de su agrado
aquella residencia, las habitaciones no eran suficientes para cinco
personas, pues Navarro no quería separarse de sus dos amigos. Alquiló,
pues, una hermosa casa amueblada con lujo en la solitaria calle de
\emph{Sal si puedes}, hermosa vivienda, perteneciente a un grande que
viajaba por el extranjero. Carlos era hombre rico y nada tacaño en el
gasto y brillo de su persona: así es que, extinguido el imperio del
avariento Baraona, púsose la familia en un pie de opulencia que eclipsó
mi decorosa medianía. Tenían casa hermosa, aunque pequeña, varios
criados y cuadras y cocheras, anejas al edificio. No sé si he dicho que
Garrote era coronel de ejército, merced al reconocimiento de grados que
se hizo a los guerrilleros; y si él hubiera sido pedigüeño como otros,
habría obtenido la faja.

Como vivíamos tan cerca, casi todos los días me tenían allá. Baraona,
que cada vez se inclinaba más a la tierra, no podía pasar sin mis
noticias ni sin mi atención, cuando soltaba la sin hueso en pro del
régimen absoluto. Carlos se preocupaba mucho también de política.

Jenara me parecía más taciturna después de la llegada de su esposo; y si
he de decir verdad, yo no advertía entre uno y otro aquellas señales de
mutuo afecto, de amable cortesía que indican perfecta paz y concordia en
un matrimonio. Jenara y Carlos se hablaban poco y con frialdad. Nunca
reñían; pero manteníanse a cierta distancia el uno del otro, más bien
como conocidos indiferentes que como esposos. Noté en él no sé qué
desconfianza vigilante, y en ella cierta reserva ocultadora. Por algunas
palabras y acciones de Carlos comprendí que acechaba. Por el silencio y
la conducta de Jenara comprendí que temía\ldots{}

Yo no sabía a qué atribuir tales fenómenos, que habían empezado a
notarse desde que se verificó el matrimonio, aunque no tomaron carácter
alarmante hasta la época a que me refiero. ¿Provenían de una profunda
disconformidad entre sus caracteres? Bien podía ser, porque Carlos,
hombre de corazón recto, era muy rudo y al mismo tiempo sencillo, sin
delicadezas, enemigo acérrimo de novedades dentro y fuera de la casa,
muy reservado, ardiente, profundo, áspero y de una constancia y
perdurabilidad enorme en sus sentimientos y afecciones. Jenara, a quien
yo no conocía bien aún, pareciome que estaba fundida en moldes muy
distintos.

Un día fui, como de costumbre, a charlar con Carlos de política. No
necesito decir que yo disimulaba perfectamente mi complicidad
revolucionaria, pues si aquella gente tan fanática hubiera conocido mis
veleidades, no lo pasara bien este desgraciado. Los Baraonas y los
Garrotes, procedentes de lo más duro de las formidables canteras
vascongadas, eran gentes con las cuales no se podía jugar en materia de
ideas políticas. Después que hablamos un poco los cuatro, salieron a
paseo Jenara y su abuelo, y cuando Carlos y yo nos quedamos solos, aquel
mostró deseo de hablarme de un asunto extraño a las conspiraciones.

---Pipaón---me dijo.---Va usted a tener conmigo tanta franqueza como si
fuéramos hermanos. Se me figura que usted sabe algo que me interesa y
que no me quiere confiar, algo que, según su entender de usted, no debe
decirme.

---No, Sr.~D. Carlos mío; nada sé yo referente a usted que al punto no
pueda decir.

---Usted habrá notado que mi mujer no me hace feliz---dijo, expresándose
con cierta dificultad, como quien no encuentra la palabra
propia,---quiero decir\ldots{} pues\ldots{} quiero decir que no soy
completamente feliz con mi esposa.

---Sr.~D. Carlos, me parecía haber notado eso.

---Sin duda mi carácter es muy opuesto al suyo. Sin duda ella tiene la
cabeza llena de proyectos estupendos y su alma toda entregada a
ilusiones locas. Yo vivo en la tierra, soy rutinario, pacífico, me gusta
la vida ordinaria que se va deslizando tranquila por la suave pendiente
de los fáciles deberes fácilmente cumplidos; ella es un alma de
dificultades\ldots{} no sé si me expreso bien\ldots{} quiero decir que
Jenara no puede vivir sino donde hay tumulto y algún monstruo con quien
luchar.

---Ahora lo entiendo menos,

---Quiero decir que Jenara tiene en su alma un laberinto.

---¿Un laberinto?

---Una batalla constante con sombras, con fantasmas, con cosas grandes y
enormes que atropelladamente se levantan dentro de ella y la llaman y le
arrojan piedras como montañas\ldots{}

---¡Ah! Sr.~D. Carlos, juro a usted que no entiendo una palabra.

---Pues yo sí lo entiendo---repuso con tristeza.---Esto que hablo, ella
misma me lo ha dicho. Me lo dijo a poco que nos casamos. ¡Ah! Sr.~de
Pipaón, yo no debí casarme con Jenara. Ella pudo ser franca también y no
casarse conmigo; debió buscar su igual, y su igual no soy yo.

---Aprensiones, mi Sr.~D. Carlos.

---Realidades, mi Sr.~D. Juan. El resumen de todo es que yo amo
extraordinariamente a mi mujer, porque soy más pequeño que ella, y que
mi mujer no me quiere a mí, porque es más grande que yo. Lo grande
desprecia siempre a lo pequeño; es ley eterna. ¡Oh! Dios mío, ¡cuán
difícil es resolver la cuestión de tamaño en las almas!

---Creo que usted se deja llevar de presunciones falsas, de
cavilaciones\ldots{}

---No, todo es realidad, realidad---dijo Carlos con el aplomo que da una
convicción profunda.---Mi mujer no me ama. Si en esto no hubiese más que
un simple asunto de amores, me callaría; sí, padeciendo, me callaría;
dejaría correr la enorme rueda de molino que da vueltas sobre mi corazón
y lo tritura\ldots{} pero esto es también una cuestión de honor.

---De honor\ldots{}

---¡Sí, porque Jenara no es mi querida, es mi esposa!---exclamó
sombríamente, clavando en mí el rayo de sus negros ojos.---Es mi esposa,
y si mi esposa (entienda usted bien que es mi esposa, unida a mí por
lazo indisoluble), olvidase sus deberes y me fuese infiel\ldots{}

Al decir esto, Carlos me había agarrado el brazo, y con su fuerza
hercúlea me lo estrujaba sin piedad, y se ponía pálido y echaba el globo
de los ojos fuera del casco, y tenía una expresión de ferocidad que me
dejó helado. Acabó la frase, dijo:

---Si me fuera infiel\ldots{} ¿Ha visto usted matar a un pájaro? ¡Pues
lo mismo la mataría!

---Perdone usted, Sr.~D. Carlos---dije con mucha congoja;---pero mi
brazo\ldots{} este brazo que usted quiere convertir en polvo, no ha sido
infiel a nadie, y\ldots{}

Garrote me soltó.

---Lo que quiero, Sr.~de Pipaón---añadió,---es que usted me diga todo lo
que sabe.

---Yo no sé nada.

---Durante mi ausencia, Jenara ha vivido en su casa de usted.

Como las miradas de Carlos despedían saña y rencor, pensé si tendría
celos de mí; absurda idea que a nadie podía ocurrírsele. Yo me
distinguía por mi fealdad, y carecía de cualidades propias para agradar
a mujeres como Jenara. Era imposible que Carlos tuviese tal sospecha.

---Mientras usted ha estado fuera, la conducta de Jenara ha sido
ejemplarísima---le dije.

---¡Mentira!, ¡mentira!---exclamó, sacudiendo la cabeza, que en aquel
instante me parecía una hermosa cabeza de león.---Si usted me oculta la
verdad, sospecharé\ldots{}

---¿De mí?

---Oiga usted---dijo con misterio, frunciendo el torvo ceño.---A fuerza
de dinero, yo he hecho confesar a una Doña Fe que sirvió en la otra
casa. Me ha dicho que mi mujer salía algunas veces a altas horas de la
noche; me ha dicho que se estaba días enteros fuera; que andaba a la
pista de un hombre; que hacía averiguaciones para saber su paradero,
gastando mucho dinero; que algunas veces salía, no volviendo hasta el
día siguiente, siempre en compañía de Paquita, esa criada infame a quien
separé de su lado cuando llegué.

Al oír esto, no pude contener la risa. Carlos, al verme reír, se
enfureció más.

---Calma, mucha calma, amigo mío---le dije.---Si no tiene usted otros
motivos de disgusto\ldots{} Afortunadamente estoy enterado de eso, y
disiparé tales sospechas.

---Ya\ldots{} me dirá usted que mi mujer salía de casa para ocuparse en
cosas de caridad, para repartir limosnas. Aunque torpe, ya conozco el
estribillo.

---Nada de eso. Jenara andaba a la pista de un hombre, de un criminal,
Sr. D. Carlos, de un conspirador. ¿Apostamos a que no lo cree?\ldots{}
¿apostamos a que lo toma usted a risa?\ldots{}

---Sr.~de Pipaón, mi mujer no es alguacil.

---Sr.~D. Carlos, su mujer de usted lo es.

En breves palabras le conté lo ocurrido, empezando por el encuentro de
Jenara con Salvador Monsalud en la Iglesia del Rosario. Después referí
el empeño febril que había mostrado porque le cogiese la policía, y por
último sus afanosas pesquisas, tanto más enérgicas cuanto más impropias
de una mujer. Carlos me oyó atentamente. Parecía muy asombrado de mi
relato; pero no estaba tranquilo.

---¿Le parece a usted inverosímil lo que ha hecho Jenara?---le dije.

---No me parece inverosímil---repuso.---Eso puede caber en su carácter.
Una extravagancia, que en otra sería increíble, es en ella natural.

---Entonces, ya se han disipado las dudas.

---No señor; al contrario.

---¿No cree usted lo que he dicho?

---Lo creo: a quien no creo es a ella; es decir, tengo la convicción de
que mi mujer le engañó a usted haciéndole creer toda esa comedia de
Salvador Monsalud y la conspiración y los alguaciles. El infame jurado
no ha intervenido para nada en este asunto. ¡Farsa, pura farsa!

---Yo tengo pruebas de que Jenara no me engaña.

---¡Farsa, pura farsa!

Traté de convencerle, refiriéndole la frustrada captura de su enemigo y
dándole datos y razones de gran peso; pero no era posible vencer la
tenacidad de aquel pensamiento, al cual se adaptaban las ideas con
invencible cohesión. Era vascongado.

---El ingenio de Jenara---dijo sombríamente,---es inagotable. Dios le ha
dado la filosofía suprema del engaño, la luz divina del disimulo.
Penetrar su pensamiento es obra superior a la perspicacia de los
hombres. Tiene las insondables argucias del Demonio debajo de la sonrisa
de los ángeles. Sólo Dios puede saber lo que hay bajo el azul de sus
ojos. El azul de los cielos, ¿no es una mentira?, pues el mirar de ella
es una inmensidad de embustes.

Una idea acudió veloz a mi mente, y aunque atrevida, no vacilé en
manifestarla, diciendo:

---Oiga usted lo que se me ocurre, amigo mío. Quizás sea esto un
absurdo; pero ya que los dos tratamos de encontrar la verdad\ldots{}

---Venga.

---Si Jenara, según la idea de usted, nos engaña a los dos; si es
evidente que Jenara ama a algún hombre que no es su esposo (lo cual, sea
dicho entre paréntesis, yo no creo); en fin, si tiene usted razón a
atribuir a desvío la conducta de su esposa, es preciso creer que el
hombre por quien olvida sus deberes es el mismo Salvador Monsalud, a
quien aparentaba perseguir. La lógica es lógica, amigo.

Carlos Navarro me miró\ldots{} no sabré decir cómo\ldots{} con mirada
más llena de desprecio que de rencor, con una especie de lástima
iracunda. Alargó su mano hacia mí, como si me quisiera abofetear:
después hizo un gesto de señor que despide a un vil esclavo. Más que
hablarme parecía escupirme, cuando me dijo estas palabras:

---¿Qué está usted hablando?\ldots{} ¡Asquerosa idea! Mi mujer, señor de
Pipaón, podrá ser criminal, pero no degradada. En el corazón de Jenara
cabrá la perversidad, pero no la bajeza. El sujeto a quien usted acaba
de nombrar no puede nunca ser mirado por ella sino como un despreciable
ser, más digno de compasión que de odio. Hay cosas que están fuera del
orden natural. Por Dios, buscando la verdad, no caigamos en ridículos
absurdos. No soltemos lo verosímil que ya tenemos, para agarrar en las
tinieblas lo imposible.

---Pues entonces, Sr.~D. Carlos---dije campechanamente,---fuera
sospechas; fuera dudas ridículas.

---Si algo hay claro en los sentimiento de mi mujer---añadió Navarro en
tono misterioso;---si hay algo que salga a la superficie y aparezca con
luz y forma precisa en medio de las oscuridades espantosas de su
carácter, es el odio y la antipatía profunda que le inspira el hombre
envilecido con quien tuve la desgracia de batirme hace bastantes años.
Dios quiso que su diabólica mano me hiriera\ldots{} Dios lo quiso, sin
duda para abatir mi orgullo\ldots{} Era en tiempo de la guerra; yo era
entonces muy orgulloso. Debí despreciar a Salvador Monsalud\ldots{} Por
no despreciarle me castigó Dios. ¿Usted no le conoce?

Traición, perjurio, cobardía, desvergüenza, jacobinismo; haga usted un
amasijo de todo eso y tendrá a nuestro paisano. Usted no ha logrado
penetrar mis ideas; usted no comprende los grandes temores y recelos que
me atormentan. Jenara, a quien adoro, amará, ama sin duda a un hombre
superior, muy superior a mí, a un hombre que sepa responder con la
grandeza de su entendimiento a la grandeza de las pasiones de ella;
Jenara no se mide con los insectos que andan escarbando la tierra. El
día en que ella quiera perderse, no se arrojara a un charco inmundo,
sino al mar inmenso\ldots{} ¿Cree usted que no lo conozco? Sí, y el
conocerlo y conocer mi pequeñez es lo que me contrista, porque ha de
saber usted que yo soy un bruto.

Dijo \emph{soy un bruto} con tanta sencillez y aflicción como decía
Otelo \emph{soy negro}. Una pena profunda se pintaba en su semblante,
enterneciendo la ruda voz del bravo guerrillero.

---Soy un bruto---añadió,---soy cualquier cosa, un hombre adocenado, un
ignorante, un palurdo, un soldadote, y me he casado con una princesa,
con una maga, con una sibila. Usted no ha visto de cerca a Jenara como
la he visto yo; usted no la conoce. En el fondo de la intimidad es donde
se ven estas cosas y donde se compara bien. Yo vivo en la vida
ordinaria, quiero traer a mi esposa a mi lado, y cuando alzo los ojos la
veo alargando la mano para coger las estrellas. Yo no puedo ofrecerle
sino un puñado de este barro grosero y ramplón con que los vulgares
amasamos la existencia; ella huye de mí sin dignarse mirarme.

---Preocupación.

---¡Realidad, realidad!---continuó, cruzando los brazos y hundiendo la
cabeza.---Estoy convencido, convencidísimo.

---¿De qué?

---De que Jenara tiene para mí un sentimiento peor que el odio, la
indiferencia. El corazón y los pensamientos de mi mujer pertenecen a
otro.

---Pero ¿a quién?

---No lo sé; pero pertenecen a otro. Mi mujer ama a alguien. Lo veo, lo
sé, lo conozco en su silencio, en su frialdad, en su inquietud cuando
está inquieta, en su tranquilidad cuando está tranquila; lo conozco
hasta en su manera de abrir los ojos cuando despierta. Hay otro hombre,
otro hombre---añadió con ferocidad;---le siento, le respiro en el aire.
Los ojos de mi mujer tienen la terrible luz de la infidelidad; están
hablando siempre con alguien. Si miran algún objeto, aquel objeto parece
que me mira a mí y me dice: \emph{¡Carlos, alerta!\ldots{}} ¡Jenara está
enamorada!

---Pero ¿de quién?

---¡De quién!\ldots{} ¡De quién!---exclamó, remedándome con grotesca
ira.---¿Faltan en la

tierra hombres? Descuide usted\ldots{} el que mi mujer ame no será un
cualquiera; será lo que es ella, un portento; pero\ldots{} tan mortal es
el cuerpo de un sabio como el de un imbécil\ldots{} Yo le veo, le
siento\ldots{} por ahí ha de andar---añadió con febril
exaltación.---Tendrá todo lo que yo no tengo; cualidades eminentes,
nobleza de ideas, aparato de sabiduría y de hermosura; pero no, no, ¡no
tendrá un corazón como el mío!

---¡Calma, Sr.~D. Carlos!---dije yo.---Es un capricho, un delirio pensar
en semejante cosa!

---¡Realidad, realidad!---contestó apartando bruscamente mi mano que
alargué para tocar su hombro.---Me confirman esas salidas nocturnas de
mi mujer, esa supuesta persecución de un criminal, de quien ella no
puede en realidad ocuparse más que para despreciarle, porque es indigno
de que ella le persiga\ldots{} ¡Ah!, la conozco bien; Jenara será
criminal, pero nunca tendrá mal gusto. Ella no hace papeles indignos,
ella no es capaz de emplearse en un vil espionaje\ldots{} ¿y por quién?,
¿y contra quién?, contra quien deshonraría la mano del último esbirro.
No, Pipaón, eso no puede ser. Pretexto y nada más que pretexto; un
artificio con el cual ha logrado engañarle a usted; pero no a mí\ldots{}
no a mí, que lo veo todo. Los ojos de los celosos son muy singulares.
Así como los del gato ven en la oscuridad, así los del celoso ven en el
disimulo. En el fondo de la intimidad, amigo mío, es donde todo se
entiende y se descubre. Los breves diálogos que apenas se oyen, las
preguntas no contestadas, los ojos que se cierran para ver mejor lo que
tienen dentro, las respuestas que no vienen al caso, la frialdad de
estudiadas caricias, este es el gran libro, lo demás es error. El
ofendido es quien sabe leer en él; usted, que tiene tanto talento, hará
mil argumentaciones sabias para quitarme esto de la cabeza; pero yo, que
soy un bruto, sé más que usted ahora, y de mi cerebro no se desclavará
jamás este letrero. Al contrario, yo me lo clavo más cada día con mis
propias manos, y si estas letras de fuego dejaran de quemarme un solo
momento, lo tendría por una deshonra\ldots{} y nada más, sino que es lo
mismo que yo digo, ¿entiende usted?\ldots{} y si me contradijeran mucho,
sospecharía que no se me trata con lealtad, ¿entiende usted?\ldots{} y
ya que se me quiere ocultar la verdad, como se oculta la desgracia a las
almas cobardes, no me vengan con sutilezas y palabras bonitas y razones
absurdas, ¿entiende usted?

---Entiendo, sí señor---repuse, sin saber cómo suavizaría la violencia
creciente de mi enojado amigo.---Pero insisto en lo dicho. Mientras no
tengamos un hecho concreto, todo es presunción.

---¡Realidad, realidad!---repitió el guerrillero.

Sus palabras eran tan enérgicas, que cuando movía la mano acentuándolas,
parecía que iba a escupirlas. Yo deseaba variar de conversación. Decía
alguna palabra de política; pero Garrote volvía a su tema. Por último,
libráronme de tal tormento Baraona y Jenara, regresando de su paseo.
Carlos, al ver a su mujer pareció más excitado, más inquieto, más
violento.

---Tengo que hablarte---dijo a Jenara.

Baraona se había retirado a descansar. Despedime yo, y al ver la palidez
y alteración de las facciones de Jenara, no pude menos de decirme al
salir:

---Ahí me las den todas.

\hypertarget{xx}{%
\chapter{XX}\label{xx}}

Resuelto a no apartarme del camino nuevamente emprendido y seguro de que
conducía a buen término, seguí asistiendo a la reunión secreta. A los
que ya me conocen, no necesito decirles que en poco tiempo me congracié
de tal modo con los revolucionarios, que yo parecía un democratista de
toda mi vida. Bien pronto adquirí singular prestigio entre ellos; me
comunicaban acuerdos importantes y se asesoraban de mí para vencer
dificultades. En honor de la verdad debo decir que yo trabajaba con
celo, sin hipocresía ni doblez, al menos aquellos días, que eran los
últimos de 1819: yo no daba cuenta de lo que veía en las reuniones más
que a D. Antonio Ugarte, de quien era poco menos que esclavo. En cambio,
recibía de él noticias e indicios estupendos que con toda diligencia
comunicaba a mis nuevos amigos.

La entrada del Sr.~Marqués de M*** en el ministerio no había cambiado
radicalmente la situación. Verdad es que él, creyéndose un Júpiter de
Gracia y Justicia, descargaba sus rayos a diestro y siniestro. ¡Pobre
hombre! Sus rayos, o mejor dicho sus palos, eran palos de ciego. No dio
un golpe que no cayera sobre inocentes, mientras los verdaderos
criminales bullían en torno suyo, gozándose en la bufante ira del
Ministro. Todos los días decretaba destierros, embargos, prisiones,
registros de casas; el aturrullado Marqués hubiera despoblado a Madrid
sin dar con los verdaderos revolucionarios. ¡Y qué convencido estaba él
de que \emph{iba poco a poco arrancando de cuajo la perniciosa yerba!}
Había que ver al buen señor; había que oírle ponderar el éxito de sus
trabajos, mientras daba pataditas en el suelo, emblemático movimiento
para indicar que \emph{aplastaba la hidra revolucionaria.}

Si apunto estos detalles es porque yo le veía con frecuencia, y si le
veía con frecuencia era porque nuestra antigua amistad no se había
enfriado. Tan lejos estaba el bendito Marqués de tenerme por liberal
como de creer que llovían calabazas. Muy al contrario, me juzgaba
empalagado de amor por el absolutismo, y en ley de tal me hacía
confidente de sus proyectos y de lo bien que le \emph{iba saliendo el
espurgo y limpieza del Reino}. Para que no sospechase, yo me deslenguaba
en denuestos e injurias contra los liberales, y alguna vez iba con el
cuento de una logia descubierta por mí o de una conspiración sospechosa.
De este modo favorecía a mis nuevos amigos, porque si nos reuníamos en
tal calle, llevaba yo el soplo de que la cita era a legua y media de
allí. De este modo, mientras la logia estaba tranquila, descomunal
nublado caía sobre una junta de cofradía o merienda de artesanos
pacíficos.

Entre tanto era evidente que la cosa iba a paso de carga, según opinión
de los más metidos en harina. Al mismo tiempo todo Madrid esperaba algo
estupendo. Había en la población la atmósfera especial del gran suceso
inminente, una ansiedad precursora, sin saberse aún de qué. A pesar de
esto, los adeptos a la comunidad secreta no sabíamos nada fijo; sabíamos
tan sólo que se trabajaba en el ejército. Del de la Isla corrían
versiones muy distintas: unos lo daban por entregado a la revolución;
otros le creían patriota en la idea, pero tímido en la acción. Salían y
entraban comisionados; pero Monsalud no regresó de Andalucía.
Últimamente logré internarme más en el corazón de la conjura, fui dueño
de importantes secretos. El golpe debía darse en la Coruña y en
Zaragoza.

Llegó el 1.º de Enero de 1820; vino el día de Reyes y una noticia
circuló por Madrid con la celeridad del rayo. Fue a despertarme Carlos
Garrote, el cual me dijo que me vistiese con toda presteza para salir
juntos. Estaba tétrico, y sus miradas y sus palabras eran hiel.

---¿Apostamos a que este bruto ha hecho una atrocidad con su
mujer?---dije para mí.

---Levántese usted---me dijo;---ocurren sucesos graves\ldots{}

---¡Pobre Jenara!---exclamé.---Yo tengo la seguridad, Sr.~D.
Carlos\ldots{}

---¿Qué habla usted ahí? No se trata de mi mujer.

---¿Pues de qué, Sr.~D. Carlos?

---Se han sublevado algunas tropas del ejército expedicionario.

---¡Qué picardía! ¿Habrase visto?\ldots---exclamé yo simulando tanto
enojo como espanto.---¿Pero son muchas las tropas sublevadas?

---Unos dicen que son muchas y otros que sólo un par de regimientos.

---¿Y no sabe en qué punto?

---En las Cabezas de San Juan.

---¿Y hacia dónde están esas Cabezas? No conozco más que una, que suele
verse sobre los hombros del Santo Precursor o en la bandeja de Herodías.

---Estas Cabezas, donde se ha consumado tan vil traición, están en
Andalucía, cerca de Jerez. Ya sabe usted que el ejército expedicionario,
por librarse de la fiebre amarilla, se había acampado en las Cabezas de
San Juan, en la Corredera, en Arcos de la Frontera y otros puntos del
interior.

---¿No manda ese ejército el conde de Calderón?---dije haciéndome de
nuevas.

---El mismo: le conozco, es un viejo estúpido.

---¿Y no se sabe qué cuerpos han dado ese aleve grito? ¡Que no los
fusilaran a todos!\ldots{} Sr.~D. Carlos, esto da vergüenza.

---Dicen que el batallón de Asturias ha sido el primero.

---¿Quién lo sublevó?

---Rafael del Riego.

---¡Rafael Riego!---dije yo fingiendo que hacía memoria.---¿Le conoce
usted? ¿No estaba ese muchacho en el regimiento de Valencey?

---Sí; empezó sirviendo en la Guardia de la Real Persona. Durante la
guerra sirvió en el ejército y en las partidas. Sé que estuvo en las
acciones de Balmaseda, San Pedro de Gueñes y Espinosa de los Monteros.
Después le hicieron prisionero, y al cabo de algún tiempo apareció en
Galicia.

---¿Le conoce usted?

---Le vi en Vizcaya al principio de la guerra. Era valiente. Algunos
traidores lo son.

---Si parece increíble, Sr.~D. Carlos---dije vistiéndome
apresuradamente.---¡Que tal canalla haya nacido en España!\ldots{} No sé
qué haría\ldots{} Si todas las cabezas de esos infames rebeldes
estuvieran al alcance de mi mano, las cortaría de un solo golpe.

---Este es el resultado---murmuró Carlos,---de la benignidad del Rey con
los militares que descubiertamente han estado conspirando desde el año
14.

---Dice usted bien. Si Su Majestad no se hubiera andado con
blanduras\ldots{} Vea usted el pago que le dan al mejor y más generoso
de los reyes. ¿Y usted qué piensa hacer?

---Ahora mismo me voy a presentar al Capitán General para que disponga
de mí. Quiero formar parte del primer ejército que salga a combatir a
los insurrectos.

---¡Oh, cuánto siento no ser militar como usted, Sr.~D.
Carlos!---exclamé con calor.---Si yo fuera militar, iría también el
primero y entraría lanza en ristre a esas rebeldes Cabezas de San
Juan\ldots{} ¡La sangre me arde en el cuerpo!\ldots{} Supongo que se
mandará allá un ejército; que este ejército les entrará a saco; que no
dejarán con vida ni a uno solo de esos infames.

---El ejército---dijo Garrote sombríamente,---está corrompido y minado
por el liberalismo.

---¿No se sabe más que la rebeldía del batallón de Asturias?

---Se dicen tantas cosas\ldots{} Todavía no será posible precisar la
extensión del mal. Todo depende de que Cádiz y su guarnición hayan
respondido al movimiento. Se habla también de otro batallón sublevado,
el de España, que manda Antonio Quiroga.

---Ese ha estado preso hace poco por conspirador liberal.

---No sé más de él sino que debió el grado de coronel a la prontitud con
que trajo a Madrid la noticia de la muerte de Porlier.

---¡Linda carrera!\ldots{} pero vamos, vamos a la calle. Le acompañaré a
usted al ministerio de la Guerra, donde sabremos la verdad de todo.

Salimos; la gente iba y venía como de ordinario; pero hacia el centro de
la villa, vimos grupos y gentes curiosas y anhelantes que preguntaban o
respondían, dando curso a imponderables mentiras. Las palabras
\emph{Cabezas, Riego, Quiroga}, sonaban sin cesar en nuestros oídos en
todo el trayecto que recorrimos. Era digno de notarse que los semblantes
alegres eran aquella mañana en mayor número que los tristes. En el
ministerio había tanta gente y charlaban tanto, diciendo tan diversas
cosas, que nada pudimos sacar en limpio. Vimos entrar al señor ministro,
el general Alós, hombre de quien un escritor coetáneo dice que era
\emph{más propio para capellán de un convento de monjas que para
ministro de la Guerra}.

«Que los insurrectos habían entrado ya en Cádiz.

»Que los insurrectos habían sido rechazados en el puente de Suazo.

»Que se les había unido el batallón de Sevilla, a las órdenes de Muñoz.

»Que habían sorprendido y arrestar en Arcos de la Frontera al general en
jefe, conde de Calderón».

»Que el general en jefe les había sorprendido y arrestado a ellos.

»Que el batallón de Canarias, acantonado en Osuna, se les había unido
también.

»Que habían sido atacados y destrozados por el batallón de Canarias.

»Que Riego y Quiroga habían reñido el uno con el otro, dándose de
porrazos por quién de ellos mandaba.

»Que se habían dirigido a Algeciras para embarcarse y refugiarse en
Gibraltar.

»Que venían sobre Córdoba (la ciudad)

»Que Córdova (D. Luis, no la ciudad) iba sobre ellos.

»Que Sevilla se había pronunciado también.

»Que Sevilla no se había pronunciado ni se pronunciaría jamás».

Estas y otras noticias fueron llegando sucesivamente a nuestros oídos.
Era preciso resignarse a no saber nada fijo y cierto hasta que Dios
quisiera; porque entonces había tiempo de hacer todas las revoluciones
imaginables de que la noticia llegase a la Corte. Al medio día separeme
de Carlos, porque deseaba visitar a mis flamantes colegas de
conspiración.

«Que toda Andalucía estaba en armas.

»Que Zaragoza tenía ya formada su Junta revolucionaria.

»Que Murcia y el arsenal de Cartagena habían proclamado ya la
Constitución.

»Que la Coruña y el Ferrol ardían.

»Que \emph{mañana} se daría el golpe en Madrid.

»Que las tropas que se enviaban a combatir la insurrección se negaban a
hacer armas contra sus compañeros.

»Que era gloriosísimo que todo se hubiera hecho sin efusión de sangre.

»Que la Europa nos contemplaba llena de admiración».

Tales fueron las noticias y versiones con que me aturdieron mis
optimistas amigos. Yo, sin embargo, ponía en cuarentena tan lisonjeras
especies.

El marques de M***, a quien vi por la noche, estaba furioso, aunque se
esforzaba en disimularlo, fingiéndose tranquilo y aun gozoso por el giro
que tomaba la rebelión.

---Me alegro de que hayan arrojado la máscara---dijo, dando las
pataditas con que emblemáticamente indicaba la destrucción de la hidra
revolucionaria.---De este modo será mucho más fácil concluir de una vez
con todos ellos.

---La situación, Sr.~D. Buenaventura---dije yo en tono agridulce,---no
es muy lisonjera.

---Ya verás, ya verás---me dijo con cierta acrimonia que me
disgustó,---cómo les sentaremos la mano. Y se me figura que te me estás
volviendo liberalote de algún tiempo a esta parte\ldots{} Pipaón,
tengamos la fiesta en paz.

---¡Yo liberal!---exclamé.---Pero no se trata aquí de ser liberal ni de
dejar de serlo. Trátase de ver si esta oleada que se ha levantado en
Andalucía llegará a la Corte y nos anegará a todos.

---Veo que tienes miedo\ldots{} el miedo es el mayor auxiliar de la
traición.

---Jamás seré traidor; pero hablemos con toda franqueza, Sr.~D.
Buenaventura. Ponga usted la mano sobre el corazón, y dígame si el
gobierno y la administración de nuestro país no exigen pronta y radical
reforma.

---Pero ven acá---repuso, poniéndose rojo como un pimiento.---Dado el
caso de que esa reforma sea necesaria, lo cual es muy dudoso, ¿quién la
realizará? ¿Esos infames perdidos, esos desocupados que charlan en los
cafés, esos desalmados políticos del 12, esos militares revoltosos que
no conocen la disciplina?

---Líbreme Dios de defender a los revolucionarios y
perturbadores---dije;---pero

vengamos a la cuestión.

---Al fondo de la cuestión.

---Eso es, al fondo. El Gobierno absoluto no puede sostenerse. Bien sabe
usted que mi opinión no es sospechosa: ¿no lo he defendido con todas mis
fuerzas? ¿No he puesto a su servicio cuanto yo podía y sabía? Pues bien;
yo, el más humilde soldado de aquel piadoso ejército de patricios que en
1814 derrocó la infame facción, declaro ahora que el absolutismo, tal
como al presente se halla, maleado y corrompido, no puede seguir
rigiendo a la nación.

---¡Ah, gran canalla!---exclamó D. Buenaventura dando fuerte puñada
sobre la mesa.---Te me has pasado, te me has pasado al enemigo\ldots{}
¡Ira de Dios! Ya van hoy doce, doce traiciones. Llega el simple anuncio
de una insurreccioncilla con esperanzas de triunfo, y ved aquí a mi
gente mudando de casaca, como histriones que, concluida la tragedia, se
preparan para el sainete\ldots{} ¡Esto no se puede sufrir! ¡Esto es
ignominioso!\ldots{} ¡Pipaón de todos los demonios, Pipaón maldito,
también tú, o como dijo el gran romano, \emph{tu quoque, fili
mihi!\ldots{}} Serían las seis de la mañana cuando llegó la noticia del
pronunciamiento; fui a Palacio, vine después al ministerio, recibí a
varias personas, y no eran las doce cuando ya me habían manifestado sus
simpatías por la revolución cinco personas, cinco furiosos absolutistas
de aquellos de pelo en pecho que no transigían con nadie y hace poco
amenazaban comerse a quien de liberalismo les hablase\ldots{} En el
resto del día ha aumentado el número de las defecciones repugnantes. Tú
eres el duodécimo\ldots{} Pero estos canallas, ¿dónde tienen la
conciencia? Sin duda creen que la infame facción triunfará. ¡Quieren
congraciarse con los rebeldes por si llega la marimorena de los
destinos\ldots! ¡Ahí os quiero ver, miserables!\ldots{} Que no se os
volvieran veneno los reales despachos\ldots{} Los muy tunantes no se
atreven a vituperar de súbito el paternal Gobierno que nos rige, ni a
ensalzar a los revoltosos; pero van preparando el terreno para la
defección, y con delicada hipocresía dicen: «La verdad es que así no se
puede seguir\ldots{} la arbitrariedad no puede gobernar constantemente a
los pueblos cultos\ldots{} es indispensable que el Rey dé una Carta a la
Nación\ldots{} la Europa no puede consentir\ldots» Y vuelta a la Europa,
y al Rey, y a los pueblos, y a la dichosa Carta, esquela o lo que sea.
Vale más que de una vez salgan por esas calles gritando: \emph{¡Vivan
Robespierre y la guillotina!}, y acabaremos de una vez\ldots{} ¡Ah,
menguado Pipaón!, ¡ah, pérfido discípulo! Eres el cuervo que he criado
para que me saque los ojos\ldots{} ¡Con que te me has pasado a la
masonería y a la revolución!---añadió, tirándome de una oreja con
impertinentísimo movimiento;---¿con que esas tenemos, señor bergante?
¿Con que después de haber explotado el oscurantismo, después de haberle
chupado la sangre al Reino, y al Rey, y a chicos y a los grandes,
reniegas de la generosa cabrita cuyas ubres has puesto, a fuerza de
mamancia, como zurrón vacío?\ldots{} ¡Ah, troglodita! ¿Sabes que desde
hace algunos días sospechaba yo tu defección? Me habían dicho que
mangoneabas en las sociedades secretas; pero no lo quise creer. Te
juzgaba mejor de lo que eres\ldots{} Pero ¿qué puede esperarse de estos
petates, cuando se asegura que hasta hombres como Lozano han caído en la
tentación? Execrable aventurero, ¡qué chasco te vas a llevar! ¡Qué
horrible será el castigo de tu traición indigna! La revolución no
triunfará, porque estamos decididos a aplastarla, sí señor, a
confundirla; y si es preciso, iremos todos allá, desde el ministro hasta
el último empleado; y entre tanto, en este foco de las conspiraciones
buscaremos a los astutos Robespierres, a los violentos Dantonazos, a los
sanguinarios Marates, y les entregaremos a la Inquisición para que dé
buena cuenta de ellos\ldots{} Descuida, que todo se hará, empezando por
ti, monstruo de felonía y doblez\ldots{} ¡Te vigilaré, te pondré preso,
te ahorcaré!!!\ldots{}

Aquel hombre estaba loco o al menos lo parecía, según se inflamaba su
rostro y se hinchaban sus venas y espumarajeaba su boca. Oí la filípica
con aquella calma burlona que me era propia y que tan bien cuadraba
frente a un hombre tan ruidoso como poco temible\ldots{} Pero me
convenía no prolongar más aquella conferencia. Antes que me echase de su
despacho, me marché, para que no se irritase excesivamente, y al salir
llevaba conmigo la seguridad de que hombre tan fiero sería de los más
blandos si los acontecimientos seguían a su resolución con la
precipitada corriente que hasta allí parecían llevar.

Del mismo modo que me trató D. Buenaventura, tratáronme otros personajes
que hasta entonces no sospechaban de mí, y que al fin tuvieron indicios
(de ningún modo certeza) de mi defección. Yo me reía de todos ellos y de
su furor impotente. Hiciéronme desaires y me pusieron avinagrados gestos
en algunas casas que visité; pero en ninguna recibí tan mal trato como
en casa de Carlos Navarro. Verdad es que del fanatismo insensato y
exaltado de aquella gente todo se podía esperar, incluso el repudiar a
un leal amigo por cuestión de ideas. Baraona me dirigió amargas pullas,
Carlos apenas se dignó hablarme, e hizo alusiones tan crueles a mi
conducta, que otro más valiente que yo le habría pedido satisfacción. No
era extraño que me manifestaran tanto desprecio por una simple sospecha,
porque ellos eran atroces, intransigentes, irreconciliables, tenían el
absolutismo en el fondo del alma y en la médula de los huesos, como
tiene el león la fiereza. Además, D. Buenaventura, que iba allí de
tertulia las más de las noches, les había dicho de mí mil picardías.

Únicamente Jenara se mostró amable y cortés conmigo. Por eso sin duda,
al salir yo, noté que su marido la reprendía ásperamente, lo cual me
hizo decir para mi capote como en otra ocasión:

---Ahí me las den todas.

\hypertarget{xxi}{%
\chapter{XXI}\label{xxi}}

Desgraciadamente, los acontecimientos iban con mucha calma. La
revolución, como las carretas de aquellos tiempos, como la
administración española, como toda la vida de antaño, iba despacio.
Parecía una cosa oficial. No había en aquel estadillo aquel progreso
instantáneo, el correr tempestuoso que indican la ira nacional. Yo me
acordaba de cómo se alzaban los pueblos en la guerra de la
Independencia, y al ver aquella pereza, aquella lentitud somnolienta de
1820, se me abrasaba la sangre de impaciencia. «Si viene que venga de
una vez», decía yo. Más que revolución, aquello parecía una fiesta, una
cabalgata suspendida por la lluvia, una procesión atascada en los baches
del camino. No había en ella el incendio popular, sino una especie de
lento deshielo, inseguro, dificultoso.

Durante bastantes días no vino noticia alguna de ventajas obtenidas por
los insurrectos. Se supo con precisión la verdad de lo ocurrido al
principio; pero escaseaba lo nuevo. Eran hechos incontrovertibles la
sublevación del batallón de Asturias al grito de su segundo comandante,
D. Rafael del Riego, de los de España y la Corona, mandados por Quiroga,
y la marcha de ambos jefes insurrectos hacia Cádiz. También era cierta
la sorpresa y prisión del general en jefe con tres generales más. Hasta
aquí no había ocurrido ningún contratiempo; pero cuando los insurrectos,
tomando el puente Suazo, trataron de penetrar en la Isla, tuvieron la
mala suerte de tropezar con un D. Luis Fernández de Córdova, que
acompañado de algunos urbanos les supo detenerles. Igualmente era cierto
que, si los insurrectos no habían podido vencer la obstinación de
Córdova, tampoco fueron desbaratados por D. Manuel Freire, que fue
contra ellos.

Estaban, pues, en situación que no podía llamarse ni próspera ni
adversa. Si cualquiera de ellos hubiera tenido una chispa de genio
militar en su entendimiento, fácilmente habrían adquirido ventaja,
porque las tropas del Gobierno andaban azoradas, como buscando un
pretexto decoroso para insurreccionarse también; pero ni Quiroga, ni
Riego, ni Arco Agüero, ni O'Daly valían todos juntos para componer un
mediano estratégico. Faltos de resolución, de verdadero instinto
revolucionario y de iniciativa, los rebeldes decidieron\ldots{} esperar.
Una sublevación que espera es una sandez. Es como un rayo que tomara
aliento en mitad de su veloz camino.

Dentro de Cádiz, un tal Rotalde, quiso subleva r la guarnición; pero
Córdova ahogó también el pronunciamiento.

En Madrid nos moríamos de angustia. Era tristísimo en verdad, que los
que nos habíamos embarcado en la revolución, aceptando sus hechos y
renegando \emph{in pectore} de sus principios, viésemos frustrados
nuestros honrados planes. ¡Sensible desgracia! Nosotros no éramos
Robespierres ni Marats; nosotros no queríamos cortarle la cabeza a
nadie, ni aun al marqués de M***, ni hacer horrores; queríamos
sencillamente adaptar la revolución a nuestra voluntad, aprovecharnos de
ella, encauzarla en el lecho de nuestras ideas, haciendo de la hidra
espantosa una flexible y condescendiente cortesana que tuviese sonrisas
para todo el mundo y no metiese miedo a nadie. ¡Y por torpeza de
aquellos desdichados militares, el plan admirable iba a fracasar, y nos
veríamos expuestos ¡oh funestos hados!, a quedar en la más crítica
situación del mundo, mal con los liberales, mal con los absolutistas!
¡Esto no se podía sufrir! ¡Esto era el colmo de la injusticia y de la
desgracia! Pensándolo, yo me volvía loco; invocaba el auxilio de mi
ángel de la guarda, sin apartar la mente de Dios y de su Santa Madre,
para que llevasen a seguro puerto el desmantelado bajel de la
revolución.

Pero ¡ay!, Dios y su Santa Madre no me hacían caso. Sin duda protegían
al Rey, como depositario en la tierra de la autoridad divina. ¡Horrible
situación! ¡Contratiempo funestísimo! La revolución, aquella obra tan
cariñosamente preparada por los conspiradores viejos y por los
catecúmenos, que eran (testigo yo) los más diligentes; aquella semilla
tan esmeradamente puesta en la tierra, y a la cual dieron riego
abundante los liberales y abono fecundo los absolutistas convertidos, se
malograba de día en día, se perdía, se secaba\ldots{} ¡Oh desesperación!
¡Y el país consentía tal cosa! Y el país, contemplando las marchas y
contramarchas de aquellos soldados, no profería un grito, ni se
levantaba en masa, ni hacía disparates, ni echaba el Reino por la
ventana, sino que, indiferente, frío y mano sobre mano, esperaba que se
lo dieran todo hecho\ldots{} ¡Qué país, señores, pero qué país!

Pasaban los días todos de Enero, sin que tal situación variase. Cundía
el desaliento entre los revolucionarios, y los absolutistas,
reponiéndose de su susto, sonreían con la vanagloriosa sonrisa del
triunfo y la venganza. Véase, pues, lo que los hombres de orden y de
ideas templadas sacaban de meterse en aventuras con los liberales.
¡Cuando más!\ldots{} Era una ignominia que aquellos holgazanes dejados
de la mano de Dios nos hubiesen comprometido de tal manera,
exponiéndonos a ser ahorcados juntamente con ellos\ldots{} ¡Ya, como si
todos fuéramos unos; como si un Gobierno pudiera medir por el mismo
rasero a jacobinos desharrapados y a hombres rectos y prudentes que sólo
por amor al orden habían auxiliado a la revolución!

Yo renegaba de los masones y del liberalismo y de la Carta y de la
Constitución del 12, y de los derechos del pueblo, y de toda la monserga
con que en las reuniones me volvieron loco, haciéndome cómplice de tales
extravagancias\ldots{} Yo estaba furioso; maldecía los clubs y quien los
inventó; maldecía también a Ugarte que me catequizó y a Monsalud que me
bautizó; y me arrancaba los cabellos pensando en el instante de mi
primera entrada en aquellos oscuros antros de necedad y jacobinismo.

La revolución fracasaba sin remedio; sucumbía al nacer como un engendro
enteco y miserable a quien hace daño el primer aire que respira fuera
del claustro materno\ldots{} Llegó Febrero. En Febrero, como en Enero,
la revolución moría\ldots{} era forzoso tomar precauciones contra el
chubasco, abrir apresuradamente el paraguas de la más exquisita
prudencia. ¿Necesito decirlo palabra por palabra?\ldots{} Pues era
preciso volver al redil, echar tierra a lo pasado y conducirse como si
nada hubiera sucedido; hacer pedazos la nueva casaca, cuidando de
esconder estos donde nadie los viese, y meter el cuerpo en la
antigua\ldots{}

¡Ay!, mi pobrecito corazón afligido necesita desahogarse con alguien;
era un vaso lleno, próximo a desbordarse. Mi alma, agobiada por la
pesadumbre, necesitaba otra alma amiga con quien comunicarse; otra alma
que recogiera parte del enorme fardo que sobre la mía gravitaba. Me
hacía falta un amigo generoso, un hermano, un padre. Tomando una
resolución súbita, alcé la calenturienta cabeza que durante largo rato
había tenido apoyada en las palmas de las manos, y tomando capa y
sombrero, y me fui a ver al marqués de M***, a mi generoso amigo D.
Buenaventura. La turbación del criminal llenaba mi alma; pero un
arrepentimiento sincero me fortalecía.

Contra mi creencia, recibiome con agrado. Estaba contentísimo, y su
semblante era todo felicitación. La alegría daba como una luz singular a
su arrebolado rostro, y aquel sol de Gracia y Justicia parecía puesto en
el zenit de la Administración para repartir calor y vida a todos los
confines de la vida burocrática. Su sonrisa pregonaba el fracaso de la
insurrección. Llevábase el tabaco a la nariz, aspirándolo con la
voluptuosidad a que el alma se entrega cuando no tiene nada que temer y
todo es rosas y paz y claridad en torno suyo.

---¿Ya estás aquí, perillán?---me dijo, señalándome una silla.---¿Qué te
parece el famoso pronunciamiento de las Cabezas? ¿Hemos triunfado o no?
Ya estarás convencido de que España no quiere revoluciones, sino paz.
¡Ay!, este gran pueblo celtíbero, romano, gótico, musulmán, es muy
sensato\ldots{} Ama el sueño y aborrece a todos los que meten
ruido\ldots{} Ya ves cómo la revolución se ha enredado en sus propios
lazos. Ni siquiera ha esperado a que la aplastáramos; se ha muerto ella
sola, dañada por la podredumbre que al nacer trajo en sus entrañas. Aquí
están tan bien dispuestas las cosas y tan bien equiponderadas las
fuerzas sociales, que cuando estalla un pronunciamiento, el Gobierno no
tiene que hacer más que cruzarse de brazos y dejar a los revolucionarios
entregados a su tontería y frivolidad, que es su muerte y nuestra
venganza.

Yo dudaba si hacer mi reconciliación con arte hipócrita o entregarme sin
condiciones, como el hijo pródigo que vuelve al hogar paterno. Después
de pensarlo, me decidí por lo primero, y hablé de este modo:

---A mí no me coge de nuevo el fracaso de la revolución; a todo el mundo
lo dije. Cuando le vi a usted muerto de miedo, bien claramente le
expresé mi creencia de que todo vendría a parar en nada. Pero por eso no
es menos cierto, Sr.~D. Buenaventura, que lo que ha pasado debe
considerarse como una lección, como una advertencia de Dios, para que se
reparen los males causados por la arbitrariedad. No me canso de
repetírselo a usted---añadí con aplomo ciceroniano;---el Gobierno de
estos reinos necesita prudentes reformas.

¿No recuerda usted lo que le dije el otro día? Es preciso que quitemos a
los trastornadores de la paz pública todo pretexto de trastornos\ldots{}
Lo estoy diciendo hace tiempo; lo estoy pregonando en todos los tonos y
nadie quiere hacerme caso\ldots{} ¡Pero qué obcecación, Dios mío! ¡Aquí
están, aquí están los resultados!\ldots{} ¡Es particular que entre tanta
gente, yo solo haya tenido penetración suficiente para ver el peligro!

---¡Oh, tú eres muy listo!---dijo D. Buenaventura, moviendo la cabeza
con una expresión que me pareció algo irónica.

---Eliminado de la Administración, apartado de la política---proseguí
con llorona sensiblería,---he servido siempre al Gobierno absoluto en mi
humilde esfera. ¿Y qué pago se me da? ¡Horroriza el pensarlo! Calumnias,
inicuas sospechas de mi honradez y consecuencia. En verdad que se
necesita tener un corazón muy recto para no dejarse arrastrar por el
despecho y hacer cualquier tontería. Pero, ¡ay! yo quisiera que se
pudiese hacer una investigación irrecusable de la conducta de todos los
hombres notables que usted y yo conocemos. Yo quisiera que existiese un
ojo milagroso para leer en el corazón de cada uno de ellos. Entonces se
vería quiénes son los buenos.

---Vamos, Pipaón, no te enfades---me dijo D. Buenaventura con
bondad,---ya sé que eres hombre honrado. Cierto que me han dicho de ti
algunas cosillas; pero la verdad, no les he dado crédito.

---Gracias, gracias---dije, cobrando nuevos bríos,---yo no esperaba otra
cosa, y cuando el otro día me acusó usted de no sé qué monstruosa
infidencia, mi alma se llenó de angustia\ldots{} Yo lo olvido, Sr.~D.
Buenaventura, yo perdono a los que me han calumniado, y en vista de los
peligros que corre el Gobierno absoluto, elevo como siempre mi voz amiga
para predicar la concordia\ldots{} Unámonos, Sr.~D. Buenaventura;
unámonos hoy, como nos unimos hace seis años para salvar a la Nación del
abismo a que corría. Cesen los chismes ridículos, las hablillas
malévolas con que se han querido manchar reputaciones como la
mía\ldots{} Por mi parte todo lo olvido; no veo más que a nuestro
querido Rey, a nuestra querida patria, a nuestras adoradas prácticas de
gobierno, a las cuales falta poco para ser las más sabias del
mundo\ldots{} Pero ese poco que falta debemos dárselo para aplastar de
una vez al jacobinismo insolente, a las logias inmundas, y a los
liberales soeces que quieren cubrir de ruinas el suelo de España.
Quitémosles todo pretexto para nuevas insurrecciones; reformemos el
Gobierno; ocupemos los hombres de bien todos los puestos que
insolentemente usurpan los pillos, y constituiremos una Nación feliz, y
legaremos a nuestros hijos, si los tenemos, toda clase de prosperidades
y bienaventuranzas.

D. Buenaventura me oía con admiración profunda. Concluido mi discurso,
estrechome la mano, y con benevolencia más ardorosa que lo que el caso
exigía, me dijo:

---No he dudado de ti. Eres un hombre excelente. Verdad es que tuve
sospechas; pero las he disipado. Soy todo tuyo.

---Unámonos, señor marqués\ldots{}

---Unámonos, sí. Reconozco que se te ha postergado con injusticia. Eras
de los primeros y se te puso en las últimas filas. El puesto que tú
debías ocupar en el Consejo se ha dado a hombres nulos que han trabajado
descaradamente por la revolución.

---Yo no guardo rencor a nadie---dije con hipocresía perfecta.---¿Querrá
usted creer que no me había vuelto a acordar de la tal plaza de
consejero, ni de la incalificable ofensa que me hicieron? Yo soy así: el
primero para agradecer, el último para odiar.

---Pero aún es tiempo de repararlo todo---dijo el ministro atracándose
de tabaco.---Hay otra vacante, y anoche me acordé de ti.

---No, no, de ninguna manera. Hágame usted el favor de no dármela; se lo
suplico\ldots{} Vamos, que me pondrá usted en el caso hacer renuncia.

---Bueno; veremos si te atreves a desairarme. Es preciso hacer
reparaciones, reunir toda la gente buena alrededor del Trono. Convengo
contigo en que es preciso hacer alguna cosa para normalizar el Gobierno.

---Por mi parte, señáleseme un puesto de peligro, un puesto en que sólo
haya trabajo y no beneficios, un puesto que permita manifestar la
diferencia que existe entre los aventureros sin conciencia y los hombres
honrados que se desviven por el Rey y por la patria.

Asuntos urgentes reclamaban la atención de Su Excelencia, y
despidiéndome, me dijo con muchísima amabilidad:

Queridito Pipaón, vete a tu casa. No llegará la noche sin que recibas un
recuerdo mío. No salgas en todo el día de tu casa, y espera.

Retireme lleno de gozo\ldots{} ¡Fuera revoluciones!, ¡fuera clubs!,
¡fuera trastornos políticos que alteran la santa armonía de la vida!,
¡fuera jacobinos y logias!\ldots{} Como el que ha vivido algún tiempo en
poder del Demonio y se ve libre de la terrible obsesión, así yo renegaba
de mis veleidades revolucionarias, haciendo voto de no prevaricar más en
mi vida.

Pero me aguardaba un golpe terrible, uno de esos golpes que anonadan,
que hunden, que matan, arrojando a un hombre en los abismos de la
desesperación. Como me había mandado el marqués, aguardé en mi casa todo
el día. Al fin sintiéronse pasos en la puerta: yo creí que me visitaba
un ordenanza de Su Excelencia, portador de pliegos en que se me
notificase algo lisonjero, cuando mi criado me dijo que gran numero de
alguaciles preguntaban por mí.

¡Traición inconcebible! D. Buenaventura había determinado prenderme, y
con su hipócrita zalamería alejaba de mí toda sospecha. Al decirme que
no saliese de mi casa, su intención era que me pudiesen coger fácilmente
sus miserables sayones. En aquel trance supremo, vacilante entre el
miedo y el peligro, pude tomar una determinación salvadora, y corrí a la
puerta interior. Por fortuna, fueme fiel mi criado. Doña Fe ya no estaba
allí. Escurrime por la escalera con tanta presteza, que cuando los
alguaciles registraban mi casa ya estaba yo en el lóbrego aposento del
Sr.~Mano de Mortero, a quien con las más patéticas razones pedí
hospitalidad.

Temí que los tunantes me siguieran, pero el buen gitano me ofreció que
en tal caso me ocultaría en lugar más seguro.

Mi angustia era inmensa. Contemplé con el alma destrozada el sitio en
que me hallaba, mientras Mortero decía:

---Por sí o por no, apaguemos la luz.

Antes de que la soplara, mis ojos se extendieron por la habitación, y vi
que sobre el lecho del Sr.~Mano yacía tendido y como soñoliento un
hombre. La luz se apagó y no pude verle; pero en el mismo instante sentí
pronunciar mi apellido, y por la voz conocí que estaba en compañía de
Salvador Monsalud.

\hypertarget{xxii}{%
\chapter{XXII}\label{xxii}}

La pena y furor que yo sentía no dieron lugar por algún tiempo a la
sorpresa que el encuentro inesperado de mi amigo debía producirme. El
tío Mano, seguro de que no había peligro, encendió de nuevo la luz, y
diciéndome algunas palabras festivas y tranquilizadoras, puso sus manos
en la obra interrumpida. Estaba haciendo un ejército. Yo alcé la vista;
contemplé la bóveda bajo la cual estaba, las macizas paredes, y me creí
sepultado para siempre. Parecía que había caído sobre mi corazón una
losa enorme. La Inquisición, o si se quiere la autoridad, ponía sobre mí
su pie y me aplastaba como a un insecto. Una aflicción inmensa llenó mi
alma, asemejándose a una irrupción de tinieblas que entraban en ella,
ocupándola toda para nunca más salir. Yo no podía formar otra idea que
esta.

---¡Adiós carrera, adiós porvenir, adiós posición mía!

¡Debilidad pueril! Ocultando el rostro entre las manos rompí a llorar
como un chiquillo.

---No hay cuidado ninguno---dijo Mortero.---Aquí no vendrán los
mochuelos. Esto es un sepulcro. Y si vinieran, señor mío, todavía están
ahí los calabozos, y si entraran a registrar los calabozos, todavía nos
quedaba la cisterna.

---Fíate de los amigos querido Pipaón---dijo Monsalud sacudiendo la
pereza.---Pero aquí puedes estar tranquilo.

---También a ti te han querido prender---exclamé con furia .---¿Has
conocido hombre más infame que ese D. Buenaventura? ¡Miserable mastín
del absolutismo! Dios poderoso: ¡permite que se desborden sobre España
las revoluciones más horrendas; permite que se alce una guillotina en
cada calle y que rueden por el suelo las cabezas de todos esos bárbaros
tiranuelos que envilecen el país!! ¡Sí, sí, vengan los disturbios con
sus cuadrillas de asesinos, levántese el pueblo y arrastre a esos
menguados ídolos; ardan España y Madrid!!\ldots{} ¡Pero qué detestable
Gobierno! ¡Qué infames ministros! De modo que a un vecino honrado, a un
hombre de bien, se le pone preso sin más ni más, porque a un ministro se
le antoje\ldots{} De modo que no hay seguridad\ldots{} ¡De modo que la
libertad y la vida de los españoles están a merced de un vil
delator!\ldots{} ¡Esto no se puede sufrir, esto es inicuo! Es preciso
que esto concluya. ¡Salvador, venga la revolución, venga una y mil
veces! Abajo todo esto y venga lo que saliere.

---Vamos: se conoce que te duele. Pues hay que tener paciencia,
amigo---contestó Salvador fríamente.---La revolución no viene.

---¡No viene!

---Se ha constipado en el canal de Santi-Petri.

---Pues debe venir---repuse con furor.---Tú y tus amigos sois unos
menguados cobardes. ¿Por qué no tenéis más energía?, ¿por qué no
atropelláis por todo?, ¿por qué no subleváis en masa al país?, ¿por qué
hacéis las cosas a medias?, ¿por qué andáis con paños calientes?, ¿por
qué no matáis?, ¿por qué no incendiáis?\ldots{} ¡Horrible estado es el
nuestro! ¡Horrible situación la de España, entregada a un espantajo como
D. Buenaventura, y sin encontrar media docena de hombres valerosos que
me salven!

La cólera mía no encontraba otro lenguaje. Mi pecho era un volcán y mis
palabras fuego.

---¡Jacobino estás!---me dijo Monsalud riendo, más sin abandonar su
calma.

---Pero, hombre, ¿no bufas como yo?, ¿no te indignas?, ¿no deseas ver al
infame marquesillo asado en parrillas?\ldots{} Yo quisiera tener cien
bocas para gritar con todas ellas: \emph{¡Viva la libertad!} \emph{¡Viva
la Constitución!\ldots{}} Si no alcanzo cómo hay absolutistas en el
mundo\ldots{} Si no se comprende cómo no son liberales hasta las piedras
de las calles\ldots{} Si no se concibe cómo estas no se levantan solas y
van corriendo por los aires a destrozar a esos miserables
verdugos\ldots{} Si no se concibe que doce millones de españoles
consientan ser tratados como una manada de carneros\ldots{} Si no se
comprende cómo hemos vivido tanto tiempo en compañía de esa vil canalla
sin hacer una revolución cada día y un motín cada hora\ldots{} Salvador,
tú no tienes sangre en las venas, cuando estás ahí tan tranquilo, y no
te irritas al oírme, y no rechinas los dientes y no maldices a nuestros
bárbaros enemigos, y no echas hiel y fuego y veneno por la boca.

---Sigue, sigue---dijo.---Te oigo con gusto.

---¡De modo que estoy perdido para siempre!---exclamé cruzando las manos
con angustia.---¿De modo que esa endiablada revolución no triunfa ya?
¡Qué inicua farsa! Nos comprometéis a tantos hombres honrados y luego lo
perdéis todo por vuestra cobardía\ldots{} Y heme aquí perdido para
siempre, sin carrera, sin más porvenir que el destierro\ldots{} porque
es claro, tendré que emigrar, si no me ahorcan antes\ldots{} Hombre,
horrorízate\ldots{} ten lástima de este desgraciado\ldots{} consuélame,
amigo, dime alguna palabra que alivie mi angustia\ldots{} por Dios,
Salvador, por Dios vivo, ¿no habrá todavía ninguna esperanza?

---Ninguna---contestó secamente mi amigo.

---Pero hombre, ¿es eso verdad?, ¿ninguna, ninguna? ¿Ha fracasado la
revolución?

---Por completo.

---Quizás te engañes. Puede que todavía\ldots{}

---Ya no hay remedio.

---¿Qué sabes tú? Todavía\ldots{}

---Vengo de Andalucía.

---¿Cuándo llegaste?

---Hoy. Nadie sabe mejor que yo lo que allí ha pasado\ldots{}

---Y dices que\ldots{} ¿Pero qué haremos ahora?

---Nada; tener paciencia---repuso con una flema imperturbable que me
exaltaba más.

---¡Tener paciencia! Eso está bueno para ti que nada pierdes, porque
nada tenías; para ti que tan poca cosa eras antes como ahora; mas ¡ay!,
yo estoy arruinado, yo estoy perdido. ¡Adiós carrera, posición,
porvenir!\ldots{} Pero cuéntame. ¿Qué ha pasado en esa fatal Andalucía?
¿Dices que has llegado hoy? ¿Por qué te has metido aquí?

---Porque el señor marqués no se duerme ahora en las pajas. Me han
seguido la pista todo el día; me he visto muy apurado para escapar. Hoy
no se encuentra un amigo por ninguna parte. Los Villelas y comparsa, en
vista del mal éxito, adulan al Gobierno. Después de recorrer varios
albergues, he creído que en ninguna parte estaba tan seguro como aquí.
No he confiado el secreto de este escondrijo ni a mis más íntimos
amigos. ¿Qué habrá sido de ellos?, en el aciago día de hoy, querido
Pipaón, se han hecho más de doscientas prisiones. No hay compasión ni
para los arrepentidos.

---¡Nos hemos lucido! Pero ¿no habrá alguna esperanza? Dime, por Dios,
que sí.

---No, no hay ninguna. Los insurrectos vagan a estas horas por los
llanos de Andalucía, medio muertos de hambre y de cansancio, sin
encontrar apoyo en ninguna parte, viendo disminuir rápidamente su número
en vez de aumentar; y gracias que los últimos consigan llegar vivos a la
raya de Portugal. Ni Riego ni Quiroga valen más que para un momento de
esos en que sólo se necesita arrojo. Cuando el primero arengó a sus
soldados en las Cabezas y les dijo: \emph{Basta de sufrimientos,
valientes camaradas; hemos cumplido con el honor; más larga paciencia
sería vileza y cobardía}, parecía que aquel hombre iba a imprimir a la
insurrección impulso poderoso; pero después le hemos visto perplejo,
vacilante, dejando pasar todas las buenas ocasiones, y corriendo de aquí
para allí como un recluta al cual de golpe y porrazo se le pusiera en la
mano el bastón de general. Tuvieron la mejor coyuntura para batir uno a
uno los batallones que no habían querido insurreccionarse, y la dejaron
perder. Rechazados en la Cortadura, salió Riego de la Isla con mil
quinientos hombres y marchó hacia Algeciras, movimiento cuyo objeto no
se alcanza a nadie. Cuando quiso regresar, supo que Freire bloqueaba la
Isla, donde estaba Quiroga, y corrió a Málaga. Perseguíale D. José
O'Donnell sin conseguir derrotarle ni tampoco ser derrotado por él. La
insurrección hasta entonces no era más que un marchar continuo, sin
aliento, sin entusiasmo, sin espíritu, porque en todos los pueblos del
tránsito no había más que frialdad, indiferencia\ldots{} De Málaga pasó
Riego a Córdoba, donde entró con quinientos hombres.

---¿Y los otros mil?

---Habían desertado, y aprovechándose de la revolución, se iban
tranquilos a sus casas.

---¡Canallas!\ldots{} ¡Pero qué falta de entusiasmo y de patriotismo, sí
señor, de patriotismo!---dije yo, no comprendiendo cómo había quien
desmayase, tratándose de derribar al Gobierno absoluto.

---En Córdoba no fueron hostilizados por la tropa; pero tampoco
vitoreados ni agasajados por el pueblo. No he visto frialdad semejante.
Parece que esto no es Nación, sino un pueblo de sombras.

---¡Qué país!---exclamé con desesperación.---Con que mientras nosotros
trabajamos por variar la forma de gobierno; mientras nos exponemos a
perder las ventajas de una brillante carrera y sufrimos persecuciones,
el bendito país se está mano sobre mano, sin decir esta boca es
mía\ldots{} ¡Pero qué horrible ingratitud, hombre! Lo que tú dices, un
pueblo de sombras.

---Lo que más me ha afligido en este fracaso, no es la mala suerte de
los militares sublevados, sino la apatía del país, su poltronería
política, pues no merece otro nombre. Ve que se levantan unos cuantos
hombres proclamando la libertad para todos, los principios de justicia,
el gobierno ilustrado, y se cruza de brazos, no comprende nada, sonríe
al ver pasar la insurrección, cual si fuera cabalgata de Carnaval. Esto
hiela el corazón\ldots{}

---¿Pero qué es esto, pues? Explícamelo.

---Esto es un triste desengaño; esto significa que España no nos
entiende. Conoce su gran pobreza y envilecimiento; quizás comprende que
otros pueblos viven mejor; pero no se le ocurre que en sí misma tiene
los medios para salir de tal estado. Tres siglos de absolutismo no
podían menos de producir esta modorra intelectual en que el país vive.
Duerme: sueña tal vez. Sufre un encantamiento parecido al de aquellos
caballeros a quienes un mago convertía en estatuas. Es verdad que en
este león encantado hay una cabeza que piensa, la idea que está en la
flor de la sociedad, en algunos centenares de hombres escogidos\ldots{}
pero estos pueden poco. La cabeza viva, puesta en un cuerpo inerte, no
sabe hacer otra cosa que atormentarse con su propio pensamiento. Eso
hacemos nosotros: atormentarnos, discurrir, creer. Tenemos fe, tenemos
ideas; pero ¡ay!, queremos tener acción, y entonces empieza el
desengaño; queremos movernos\ldots{} ¡Cómo se ha de mover una piedra!

---Desconsolador cuadro me pintas, Salvador.

---¡Ojalá no fuese verdadero! En mí notarás una transformación tan
rápida como triste. Mi pensamiento tiñe de negro todo aquello en que se
fija. Ayer estaba lleno de luz, y hoy no hay más que tinieblas dentro de
mí. No tengo ya esperanzas; he perdido todas las ilusiones. Parece
mentira que se pierda todo esto y siga uno viviendo. He visto por mí
mismo la apatía nacional, una congelación lamentable, una incapacidad
absoluta para apropiarse la idea política y los sentimientos que con
ella se relacionan, fuera del sentimiento de la patria y del sentimiento
religioso, concebidos en bruto, a lo salvaje. Aquí el pueblo no entiende
de ideas: sólo los sentimientos enormes del amor al suelo y a Dios le
pueden mover. Hablarles otro lenguaje es hablar a sordos\ldots{}
Nosotros somos muy torpes: confundimos deplorablemente la conspiración
con la revolución; creemos que la connivencia de unos cuantos hombres de
ideas es lo mismo que el levantamiento de un país, y que aquello puede
producir esto. Vemos el instantáneo triunfo de la idea verdadera sobre
la falsa en la esfera del pensamiento, y creemos que con igual rapidez
puede triunfar la acción nueva sobre las costumbres. Las costumbres las
hizo el tiempo con tanta paciencia y lentitud como ha hecho las
montañas, y sólo el tiempo, trabajando un día y otro, las puede
destruir. No se derriban los montes a bayonetazos.

---Siempre creí que España era un pueblo de costumbres
absolutistas---dije yo,---y que la revolución y el liberalismo estaban
sólo en las cabezas exaltadas de cierto número de caballeretes, un tanto
avispados por el alcohol de las lecturas\ldots{} Por eso yo, al
conspirar, no contaba con que se hiciera ninguna revolución verdadera,
sino simplemente una mojiganga de revolución, una cosa teatral y de
mentirijillas, que no alterara nada en el fondo, sino en la superficie,
y que contentándose con fórmulas, verificase un razonable y justo cambio
de personas, que es al fin y al postre lo más conveniente.

---Como tú piensan muchos, muchísimos de los que más han bullido en las
logias, y esta es una de las causas del fracaso. Aquí no hay más que
absolutismo, absolutismo puro arriba y abajo y en todas partes. La
mayoría de los liberales llevan la revolución en la cabeza y en los
labios, pero en su corazón, sin saberlo se desborda el despotismo.

---¿De modo que, según tu frase, España seguirá andando a cuatro pies
por mucho tiempo?

---Por muchísimo tiempo.

---¿Y qué piensas hacer ahora?

---Nada: renunciar a un trabajo que creo no ha de tener resultado
alguno. Yo empecé con mucho ardor; tenía una fe profunda; creía que por
tales medios podía adquirir gloria para mi país y para mí; trabajaba a
ciegas sin ver el material que tenía entre las manos. ¿Me preguntas lo
que pienso hacer? Renunciar a un papel que empieza a ser criminal y
hasta ridículo desde el momento en que sólo puede servir para ayudar a
vulgares ambiciones. Estoy convencido de que la revolución tiene que ser
vana por ahora. Lo he visto por mis propios ojos; lo he tocado con mis
manos\ldots{} Con su nombre pueden elevarse y luchar facciones
miserables, y a facciones no sirvo yo. He sido durante algún tiempo
aventurero, pero en mis aventuras entreveía un hermoso ideal. Mientras
duró el engaño, mi conducta no podía dejar de ser noble. Pero, amigo
mío, ya he visto que los que creía gigantes eran molinos de viento, y
aquí concluye mi caballería andante. Felizmente no he perdido el seso.
Si pude un día aceptar lo que hay de generoso en el papel del gran
caballero de la Mancha, renuncio ya a lo que en él hay de ridículo, y
arrojadas las inútiles armas me vuelvo a mi aldea.

---¿A tu aldea?

---Al extranjero, quiero decir; o a América, qué sé yo\ldots{} En mi
horrible descorazonamiento, amigo Bragas, yo conservo una serenidad
notable, y no tomaré resoluciones atropelladas. No hay que
apurarse\ldots{} Calma. Durmamos ahora tranquilamente y mañana se
pensará lo que se ha de hacer.

---Parece mentira que puedas dormir una noche de desgracias como esta.
¡Qué calma tienes!

---Estamos caídos---dijo con voz que se extinguía poco a poco a causa
del sueño.---Algún día nos levantaremos. Dicen que no hay mal que cien
años dure.

---¿Y serás capaz de dormirte así\ldots{} dejándome solo, sin
consuelo?\ldots{}

---¿Consolarte yo?---repuso dormitándose, sin consideración a mi
soledad.---¡Pobre Pipaón, pobre cortesano!, le han quitado su
destino\ldots{} le han dado un puntapié con sandalia de rosas\ldots{}
Eso no es nada, amigo. Con unas cuantas sonrisas recobrarás tu
favor\ldots{} y si no con un par de lágrimas. El chubasco pasará
y\ldots{} al cabo de cierto tiempo\ldots{} como si tal cosa\ldots{}

Durmiose el infame, dejándome entregado al sombrío martirio de mis
pensamientos\ldots{} ¡Dormir cuando yo estaba perseguido, dormir cuando
el orden natural de las cosas se había alterado! Encontreme enteramente
solo, porque el Sr.~Mano de Mortero había salido poco antes. Estuve
meditando y cavilando con tal laberinto en el cerebro, que al fin
deliraba. Creo que hablé solo largo rato y una visión extraña atraía la
atención de mi espíritu. ¿Qué era aquello que yo contemplaba, Dios mío?
Yo veía un ejército poderoso que avanzaba en gallarda formación. Las
filas de hermosos caballos corrían las unas tras las otras tan
matemáticamente alineadas, que no discrepaban una línea. Los jinetes
todos esgrimían sus sables, y a igual altura se elevaban empenachados
morriones\ldots{} Pasaban, pasaban fila tras fila, escuadrón tras
escuadrón, sin acabarse nunca y sin variar nunca. Era el ejército
infinito, siempre el mismo, siempre marchando y nunca concluido. De las
apretadas y correctas filas salía sin cesar un grito majestuoso, que
penetraba en mi alma como un rayo de luz. El grito era: «¡Viva la
libertad!»

No sé cuánto tiempo duró este fenómeno; pero al fin entró el señor Mano
de Mortero, hizo ruido y me moví. En el rincón frontero y sobre el banco
del taller, continuaba el ejército; más era un escuadrón de groseros
muñecos mal tallados y peor pintados\ldots{} Sin embargo, siempre me
parecía que gritaban con sus bocas de palo: «¡Viva la libertad!»

El Sr.~Mano de Mortero dejó a un lado el farolillo con que se alumbraba,
la capa y el sombrero, y en voz alta nos dijo:

---Buenas y frescas, señores.

Monsalud despertó.

---¿Hay noticias?---pregunté con ansiedad.

---Y buenas. La Coruña ha proclamado la Constitución.

---¿Pero es verdad? ¿Lo dicen por ahí?

---Lo dicen por ahí y es verdad. Y el Ferrol y Vigo también se han
sublevado. Dicen que los ministros están que se les puede ahorcar con un
cabello.

---¡Dios mío, Virgen Santísima!, que sea verdad lo que dice este buen
hombre---exclamé juntando las manos.---¿No has oído, Monsalud, lo que
cuenta el Sr.~de Mano? ¿Qué te parece?, ¿será verdad?

---Puede ser verdad---dijo Salvador con mucha calma.

---Con que la Coruña, el Ferrol, Vigo; es decir, toda Galicia\ldots{}
Principio quieren las cosas. Si saldremos al fin con que triunfa la
marimorena y arde toda España.

---El ejército nada más\ldots---dijo mi amigo fríamente.

---Sr.~de Mano, quién sabe, quién sabe todavía\ldots{} Oye, Salvador, me
ocurre una idea.

---¿Qué?

---Que imploremos de la Divina Misericordia\ldots{}

---¿El perdón de nuestros pecados?

---No, el triunfo de la sedición. Pidamos a Dios con todo fervor y
recogimiento\ldots{} que sea verdad lo que ha dicho este buen hombre;
que sea verdad el levantamiento de la Coruña\ldots{}

Monsalud estaba echado boca arriba en actitud de tranquilidad perfecta.
Había extendido sus dos brazos formando arco alrededor de la cabeza, y
miraba al techo.

---Hombre, no seas impío---añadí,---¿por qué no hemos de impetrar de la
Omnipotencia Divina lo que deseamos? ¿No piden pan los hambrientos y
salud los enfermos? Pues pidamos nosotros revolución. El Evangelio dice:
«pedid y se os dará».

Monsalud reía.

---Sr.~de Mano---añadí yo.---Aquí veo unas hermosísimas imágenes de la
Virgen y del Señor. ¿Por qué no les pone usted una vela?

Salvador no podía tener la risa.

---Hereje, empedernido hereje, calla, calla. Cada uno tiene sus ideas.
Yo soy religioso, yo soy creyente y tú eres un perro judío. Querido
Sr.~de Mortero, encienda usted un par de luces en ese altar que está
junto a la cama.

Mortero encendió las luces.

---Ahora---dije yo,---que la Santísima Madre de Dios, Nuestra Señora del
Rosario, nos dé el inefable beneficio de un pronunciamiento en cada
ciudad de España; que sea un volcán Galicia y otro volcán Aragón; que
caigan por tierra el absolutismo y D. Buenaventura.

---Me parece que se sienten pasos arriba---dijo Salvador en voz muy
baja.

---Es que andan por allá el Sr.~Secretario y un señor
inquisidor---repuso Mortero.---No hagan ustedes ruido. Están sacando
papeles del archivo.

---Es que ven la cosa negra---afirmé yo.---Sin duda temen que el pueblo
penetre en la casa y descubra alguna picardía. Señor Todopoderoso,
Creador del cielo y de la tierra\ldots{}

---¡Es gracioso!---dijo Monsalud mirando la imagen, que era la Virgen
del Rosario con Santo Domingo de Guzmán arrodillado a sus pies.---Si a
esos señores inquisidores que están arriba les dijeran ahora que en un
sótano de la Santa Casa arden velas ante las imágenes cristianas para
implorar de Dios el triunfo de la revolución\ldots{}

---Si se lo dijeran\ldots{} seguramente no lo creerían.

Mi amigo se volvió hacia la pared, y al poco rato dormía.

Yo no cesé de rezar en toda la noche.

\hypertarget{xxiii}{%
\chapter{XXIII}\label{xxiii}}

Al día siguiente muy temprano, Mano de Mortero, que había salido a sus
quehaceres, entró diciendo:

---Gordas y frescas.

---¿Qué, qué hay?

Que lo de Galicia es tremendo. El Rey y la Corte están muy
asustados\ldots{} Toda la noche han estado los ministros en
Palacio\ldots{} Quieren contemporizar\ldots{} les ha entrado el
destemple\ldots{} desconfían de la guarnición\ldots{}

---¡Desconfían de la guarnición! ¿Oyes, Salvador; oyes,
hombre?---exclamé con exaltado júbilo.

---Oigo---repuso mi amigo secamente.

---¡Y de la Guardia de la Real persona!---añadió Mano.

---¡También desconfían de la guardia! ¿Oyes, Salvadorillo de mi alma?

---Oigo.

---Sr.~de Mano, traiga usted cuatro velas; yo las pago.

---Con esa condición, aunque sean ocho---dijo Mortero, abriendo el cajón
de una cómoda.

---No quepo dentro de mí---exclamé saltando del jergón.---Voy a salir a
la calle, aunque me exponga a ser cogido. Me pasearé, comeré en casa de
algún amigo\ldots{} Sr.~de Mano, ¿tiene usted algunas ropas con que
disfrazarme?

---Tengo vestidos de cómicos. ¿Quiere usted ir de rey turco?

---Hombre, no.

---¿Y de senescal de Polonia?

---¡Qué majadero!

---¿Y de majo? Sombrero ancho, capa encarnada, marsellés\ldots{}

---Venga, venga. Me embozaré hasta las cejas.

Mano sacó unos vestidos, que yo me puse, acomodándolos lo mejor posible
a mi cuerpo. Peineme a lo majo, tizneme el rostro, y quedé convertido en
chispero, tan al vivo, que era muy difícil conocerme. Con tal pergenio,
guiado por Mortero, que me llevó por oscuros laberintos, salí a la
calle, embozado hasta las cejas. Monsalud no quiso seguirme. Pasé por
Palacio, y vi que entraban y salían muchos coches; recorrí, luego la
calle Mayor hasta la Puerta del Sol; pero aunque encontré en este sitio
muchos conocidos, no me atreví a hablar a ninguno; tanta era mi cobardía
aun bajo el disfraz de chispero. Estábamos en los primeros días de
Marzo.

Ya conocí en la actitud y semblante de las personas, y en las palabras
que al vuelo cogía, que era ciertísima la alarma anunciada por Mortero.
Sin cesar herían mi oído las voces \emph{Coruña, Ferrol, Junta
Revolucionaria, Don Pedro Agar}, volviéndome loco de alegría. Recorrí la
población sin descubrir mi cara, atendiendo, disimuladamente a todos los
grupos, huroneando, atisbando, olfateando la revolución. ¡Ay!, la
revolución palpitaba; yo la sentía. Quien había puesto tantas veces la
mano sobre el pecho de la sensible villa no podía engañarse.

En estas exploraciones empleé toda la mañana y parte de la tarde. No me
había descubierto a nadie. Llegó por fin una hora en que me picó el
hambre con alarmante viveza; porque el júbilo y esperanza no me
alimentaban; que esto corresponde a las magras y otros condimentos, y de
ningún modo a las sensaciones agradables del alma. ¿Qué hacer? El
Sr.~Mano no podría ofrecerme sino un guisote grosero. ¿Entraría en algún
café o figón? No, porque mi pusilanimidad veía alguaciles en todas
partes, y se me figuraba que ni siquiera me dejarían llevar la cuchara a
la boca. ¿Iría a casa de algún amigo? Ugarte estaba fuera de Madrid, y
quizás perseguido también. De Villela y otros personajes no me fiaba más
que del Demonio. Pensé ir en busca de D. Gil Carrascosa, hombre que me
debía muchos favores, o de D. Bartolomé Canencia; pero luego discurrí
que las casas de donde más rápidamente debía huir eran las de aquellos
que me debían beneficios.

De pronto vi a cuatro personas que me inspiraron una idea felicísima.
Eran Carlos Navarro y D. Miguel de Baraona, que iban por la calle de la
Montera hacia la Puerta del Sol, acompañados de los dos zafios amigos
que con el primero vinieron del Norte. Antes me metiera yo mismo en la
cárcel que presentarme ante aquellos hombres fanáticos, capaces de hacer
conmigo una felonía; pero teniendo la certeza de que estaban ambos fuera
de casa, bien podía pedir amparo a la señora doña Jenara, que de fijo no
me lo negaría ni me vendería.

---Si Jenarita está en su casa---me dije corriendo en dirección de la
calle Ancha,---comeré, y comeré bien.

Poco después entraba yo en la calle de \emph{Enhoramala vayas}, para
pasar a la de \emph{Sal si puedes}. Esta tenía poco que andar.
Componíanla dos casas humildes, otra suntuosa, y una tapia de corrales o
jardines. La suntuosa, como muchas personas, tenía mejor alma que
cuerpo; es decir, que su aspecto vetusto y feo no correspondía a su
comodidad interior. De poca fachada, extendíase mucho en el fondo de la
manzana, y lo mejor de ella era la crujía de Poniente, que daba a un
patio donde estaban las cocheras. Este patio tenía la salida a la calle
de \emph{Aunque os pese}. Aquel pequeño barrio de nombres tan extraños,
era entonces más solitario aún que ahora.

Entré resueltamente. Por fortuna Jenara estaba, y estaba sola. Tan sólo
su doncella tuvo noticia de mi visita.

Expuse a la generosa dama la aflictiva situación de mi estómago,
rogándole encarecidamente que si me daba de comer lo hiciera pronto para
evitar el peligro de un encuentro con los feroces Navarro y Baraona.
Ella se rió mucho de mi extraña facha, y me dijo:

---Hace usted bien en temer a mi marido y a mi abuelo. Ellos no
disculpan ni perdonan. Están furiosos contra usted y si le encontraran
aquí, serían capaces de entregármele atado de pies y manos a D.
Buenaventura.

---¡Miserable sayón!

---Anoche estuvo aquí, y dijo de usted mil picardías. ¡Pero qué
atrocidades ha hecho usted, Pipaón!\ldots{} Conspirar así; escribir
cartas; juntar dinero\ldots{} qué sé yo\ldots{} Es usted un Robespierre.
Dice el marqués que no se consolará en toda la vida de que se le
escapara usted, y que daría un ojo de la cara por atraparle.

---¡Bandido!\ldots{} Pero si usted tuviera la bondad de darme de
comer\ldots{} Ahora o nunca: me muero de hambre.

---Al momento---repuso riendo.---Pero van a decir que soy encubridora de
revolucionarios y el marqués querrá prenderme también.

Inmediatamente dio órdenes a su doncella para que me trajese lo que tan
imperiosamente pedía mi pobre cuerpo. Ella misma tendió un pequeño
mantel en el velador de aquella estancia que era la suya, y me iba
poniendo delante los platos, amenizando el festín con discretas
observaciones y celestiales sonrisas. Yo caí sobre los manjares como el
tigre sobre su presa.

---Perdone usted, si como groseramente---le dije.---Un condenado a
muerte tiene derecho a prescindir de ciertas reglas.

---¡Parece mentira!---exclamó.---¡Usted revolucionario, usted
liberal!\ldots{}

---Señora, no haga usted caso de infames calumnias. Mis enemigos
discurren infernales embustes para perderme. Ya disiparé yo las nubes
que empañan el limpio sol de mi reputación. Deje usted que pase este
chubasco\ldots{}

---Triunfen o no los revolucionarios---dijo ella sentándose frente a mí
y apoyando el codo en la misma mesa donde yo comía,---lo cierto es que
los conspiradores lo pasarán mal. Casi todos están presos, ¿no es
verdad?

---Creo que sí.

---Sin embargo, no se oye decir que ajusticien a ninguna persona
conocida.

---Incomparable está esta gallina---repuse, más atento a la reparación
de mis fuerzas que a la suerte de los conspiradores.

Cuando empecé a reponerme y a sentirme dueño de mí mismo, fijáronse mis
ojos con singular deleite en la hermosísima figura que tenía delante de
mí. Nunca me había parecido Jenara tan bella. En la nueva mansión su
hermosura soberana se realzaba con el lujo que el generoso marido había
acumulado allí, labrando de este modo el único estuche digno de alhaja
tan preciosa. Fuera por una irradiación admirable de la privilegiada
naturaleza de Jenara, fuera porque la casa era en realidad muy linda,
todo lo que veían mis ojos tenía el más puro sello artístico. Cuadros,
tapices, muebles, cornucopias, ofrecían mil formas encantadoras que
extasiaban la vista. El oro y los pastosos tonos, las tintas brillantes
admirablemente armonizadas, llevaban los ojos de sorpresa en sorpresa.
Los excesos del lujo, que generalmente traen el mal gusto, eran allí, o
al menos a mí me lo parecía, un esfuerzo sublime de la imaginación,
comedida siempre en su delirio.

En su propia persona, los encantos de Jenara eran, como siempre,
superiores; pero allí su grave y patética sencillez brillaba más que
cuando vivía en mi casa. Siempre tuvo el raro instinto de ataviarse
elegantemente, y la no aprendida ciencia, en virtud de la cual una mujer
privilegiada sabe estar preciosa con el adorno más insignificante.
Aquella tarde en que me dio de comer, estaba vestida con la negligencia
cuidadosa que parece han de emplear las que siempre quieren estar bien,
aun sabiendo que nadie las ha de ver. Sobre su cuerpo no había más que
dos colores, el blanco y el negro; este en una copiosa sarta de cuentas
que pendían de su cuello, adorno muy usado entonces. Su traje blanco,
conjunto delicado de graciosos caprichos de aguja, de pliegues y rizos,
era un plumaje maravilloso, que a causa de la estrechez de los talles de
entonces cubría delicadamente sus incomparables formas sin
desfigurarlas, respetando cuanto el divino cincel modeló en aquel
hermoso barro humano, es decir, no aplastando ningún bulto, ni llenando
ningún hueco, ni alterando con importuno arte la más acabada estatua en
cuyo tibio mármol han vibrado nervios y corrido, por las azules venas,
menudas venas de impetuosa sangre.

Cuando se movía de aquí para allá trayéndome lo que yo había de comer,
parecía una hechicera de leyenda que cuidaba de mí, niño extraviado en
la caverna de magia, entre maravillosas transformaciones; primero
maltratado por ogros horribles, después mimado y agasajado por las
blancas manos de las hadas. Caía la tarde, y la dulce luz crepuscular
que entraba en la estancia por las ventanas abiertas al patio y a la
calle de \emph{Aunque os pese}, derramaba en torno mío, entre ella y yo,
una dulce onda de tristeza. Cuando yo concluía de comer, sentose como he
dicho, frente a mí, apoyando el codo en la mesa y la mejilla en el puño.
En primer término yo veía un brazo que a ningún otro puede compararse,
blanco, torneado, de una pureza y corrección admirables. Distinguíanse
en la suave penumbra de lo interior de la manga las morbideces del
ante-brazo que se perdía al fin entre la batista, seguido hasta lo
último por mi ansiosa vista. Tenía los ojos medio cerrados. No sé por
qué todo allí era tristeza. Yo exhalé un suspiro tan hondo, que Jenara
se conmovió cual si oyese un grito.

---¿Qué tiene usted?---me dijo.

Estaba pensando, señora mía, que el Sr.~D. Carlos, mi antiguo amigo y
esposo de usted, es el hombre más feliz de la tierra.

---¿Por qué?

---Porque es dueño de tanta hermosura.

Jenara hizo un gesto de desdén.

---Pero no sabe apreciar su felicidad, señora mía---añadí,---y con sus
ridiculeces y manías mortifica a este ángel de gracia y de bondad.

---Galán está usted---me dijo sonriendo.---No extraño que usted hable
así de Carlos. Todo el mundo conoce lo mal que me trata. Ni siquiera
tiene el tacto de guardar para mí sola sus impertinencias, sino que
delante de los amigos me suele ofender\ldots{}

---Él mismo confiesa que es un bruto; pero su alma y su corazón son
excelentes. Procure usted domesticarle, y\ldots{}

---No sirvo para domadora---me contestó, moviendo con insistencia su
linda cabeza.---Él se cansará o se corregirá. ¿Qué puedo hacer para
convencer a un hombre que se encariña con sus errores y con sus
sospechas? Cuando alguien intenta quitárselas, Carlos se enoja como si
le quisieran robar un tesoro.

---Sí, muy bien dicho. Es avaro de sus tenacidades y equivocaciones.
¡Cabeza de granito! Se estrellará, pero no dirá jamás: «me equivoqué».

---Esto tiene que concluir de un modo o de otro---afirmó.---Es imposible
vivir así. Cada día una cuestión, cada hora una disputa. ¿Y por qué? Por
nada, por fantasmas. Sepa usted que el cerrar los ojos y el abrirlos es
en mí un indicio de infidelidad, según mi marido. Aprenda usted a tener
perspicacia.

---¡Detestable sistema es ese! Conozco algunos maridos que por buscar
tres pies al gato, han hallado los cuatro. Mucho cuidado, Sr.~Garrote,
vais por mal camino\ldots{} No crea usted; yo le reprendí y le dije
media docena de verdades\ldots{} pero no hace caso. Tiene a gloria el
equivocarse. En disparatar consiste su orgullo.

---Ahora, con estas cosas de la revolución que viene, está
insoportable---dijo la dama con ademán ponderativo.---No se le puede
resistir\ldots{} Ahora paso los días entre el temor y la tristeza,
asustada cuando le espero y creo que va a llegar, triste cuando estoy
sola. Con él tiemblo; sola me aburro. ¿Puede haber situación más
horrible? ¡Ha de saber usted que Carlos, con sus impertinencias ha
llegado a lo que nunca creí, a malquistarme con mi abuelo, que también
sospecha, también! Figúrese usted si será deliciosa mi existencia. Ellos
dos, es decir, toda mi familia, están contra mí. A mi lado no hay nadie
más, ni hermanos, ni hijos, ni siquiera amigos\ldots{} Las amistades,
cualesquiera que sean, me están prohibidas\ldots{} ¿No es verdad que soy
digna de envidia? La cabeza hecha un volcán y el corazón vacío,
enteramente vacío.

---¡El corazón vacío!, es decir, holgazán\ldots{} ¿Qué de cosas no
discurrirá el muy tunante para poder entretenerse?\ldots{} ¿eh?

En el mismo instante sentimos ruido de voces y pasos en el interior de
la casa.

---¡Carlos!---exclamó Jenara con el mayor sobresalto.

---¡Jesús, María y José!---dije yo sintiendo que flaqueaban mis
piernas.---¿Dónde me escondo, dónde?

---Váyase usted. Está usted perdido si él le ve.

Jenara y yo, llenos de confusión, no sabíamos qué partido tomar.

Escóndase usted aquí---me dijo la dama, mostrándome un armario, que
abrió precipitadamente.---Después saldrá usted.

Escurrime dentro. Yo no era hombre, yo era un papel. Creo que me hubiera
metido entre dos platos. De tal modo me hacía flexible el miedo.

Poco después de esconderme, entró Carlos. Yo no le veía; pero le sentía.
El resoplido de la fiera, llegando a mis oídos, me ponía los cabellos de
punta. Acompañábale uno de sus amigos, el llamado Zugarramurdi, que era
el más bruto. Estuvieron los tres en silencio durante breve rato. Sin
duda Carlos estudiaba el semblante de su mujer.

---Jenara---dijo al fin,---el portero me ha dicho que entró hace poco un
hombre y que no ha salido.

---¡Un hombre!\ldots---repuso Jenara.---No sé\ldots{}

Su voz temblaba.

---¡Es singular cosa!---dijo Carlos con marcado acento de ironía,---pero
como en estos tiempos hay tantos ladrones\ldots{}

---Se registrará la casa---indicó con bronca voz el amigo.

Yo me quedé yerto; yo era un cadáver.

---Como no sea\ldots---dijo Jenara.---Sí\ldots{} hace poco estuvo aquí
un señor, preguntando\ldots{}

---¿Preguntando qué?---vociferó Garrote.---Sosiégate, mujer\ldots{} te
doy tiempo para que medites lo que quieras decirme\ldots{} no se ocurren
siempre buenas ideas para ocultar la verdad. Los más listos se
turban\ldots{} Con que entró uno preguntando\ldots{}

Sentí el chasquido de los maderos de la silla en que la bestia se sentó.

---Un hombre, no sé quién\ldots---continuó Jenara en tono más tranquilo
y algo altanero.---Si no lo quieres creer, no lo creas. Me parece que
era el que anoche fue contigo en busca de Pipaón.

Hubo una pausa. ¿Le convencería?

---¡Pipaón!---dijo el amigo.---Juraría que le encontramos hoy en la
calle.

---¿Y por qué no me lo dijiste?---repuso Carlos con violencia.---Crees
que me importa pescar en medio de la calle a un sapo, liarle una cuerda
a los brazos y llevarle a la superintendencia de policía.

Yo daba diente con diente.

---Pues sí---dijo Jenara con voz serena,---ese creo que era\ldots{}

Y deseando variar de conversación, repuso:

---¿En dónde has dejado al abuelo?

---Fue solo al Príncipe, a comprarte billetes para esta noche.

---¿Qué función es?

---Una ópera nueva, una sandez, qué sé yo---dijo Zugarramurdi.

---Se llama \emph{La inútil presunción} o \emph{El barbero de Sevilla},
por un tal Rufini o Rossini---gruñó Carlos con malísimo humor.

---Anoche se estrenó: es un sainete ridículo, según me han
dicho---añadió el amigo.---Un tutor estúpido, un barbero sin vergüenza,
una pupila descocada, un amante que se finge soldado borracho para
meterse en la casa, después se hace maestro de música, y luego entra por
el balcón.

---Por el balcón---repetí yo, apropiándome con calenturiento afán
aquella idea.

De repente Carlos, que sin duda no estaba para pensar en óperas, dijo
levantándose:

---¿Cerré yo la puerta interior al marcharme?

---Creo que sí---dijo el amigo.---Lo mejor sería registrar la casa. Hay
ahora tantos ladrones\ldots{}

Carlos y su camarada salieron.

Jenara, al verse sola, abrió precipitadamente el armario, y me dijo:

---Esta farsa no puede seguir\ldots{} ¡qué compromiso!\ldots{} Es
preciso que yo diga la verdad a mi marido\ldots{} Ya no es fácil que
usted pueda marcharse\ldots{}

---¡Señora!\ldots{} ¡por compasión!

---La verdad, más vale decir la verdad\ldots{} ¿a qué vienen estos
enredos?\ldots{} Bastantes tengo con los que él inventa\ldots{}

---¡Señora!\ldots{} ¡por piedad!---exclamé de rodillas.

Y me dirigí al balcón que daba al patio.

---Por aquí---dije, asomándome para medir la distancia.

---Se va usted a estrellar.

Felizmente el descenso era muy fácil. Había bajo el balcón una alta
ventana con reja de hierro, que casi era una escalera. No lo pensé más.

---Se puede, sí, se puede---dijo Jenara.---¡Pronto abajo! Por fortuna no
hay nadie en el patio ni en las cuadras\ldots{} La puerta que da a la
calle de \emph{Aunque os pese} está siempre abierta.

Lieme la capa en la cintura, y con presteza sin igual me deslicé, sin
más contratiempo que algunas rozaduras en las manos. Embozándome hasta
los ojos, salí sin obstáculo a la calle; pero no había dado dos pasos,
cuando vi al Sr.~de Baraona que atentamente me observaba. No quise
detenerme y apreté a correr, diciendo para mí lo de marras:

---Ahí me las den todas.

\hypertarget{xxiv}{%
\chapter{XXIV}\label{xxiv}}

---Salvadorillo, albricias---dije a mi amigo, entrando en la cueva del
Sr. Mano,---todo va bien, la revolución marcha. Madrid ofrece un aspecto
imponente\ldots{} ¡Si vieras qué cosas me han pasado!\ldots{} ¡qué
aventuras!\ldots{} ¡qué peligros!\ldots{} soy un héroe. Pero en fin, he
comido como un príncipe. ¿A que no sabes dónde? Pues en casa de tus
amigos los Baraonas. Jenara, con sus propias manos divinas, me sirvió de
comer.

---¿En dónde viven ahora?---me preguntó Salvador con indiferencia.

---En la calle de \emph{Sal si puedes}\ldots{} bonito nombre\ldots{}
aquí cerca.

---Te lo pregunto porque quizás me dé una vuelta por allá.

---Me alegraré de que busques camorra a esa canalla. Pero aguarda a que
triunfe la revolución. Entonces les meteremos en un puño. Cuando la
policía sea nuestra, es preciso tomar venganza. Enviaremos a Garrote a
presidio y a Baraona a una casa de locos.

Monsalud se estaba arreglando y vistiendo. Habíale proporcionado Mortero
un vestido de majo, como el mío, pero mucho más elegante: marsellés
nuevo, calzas y pantalones negros, capa de grana y sombrero redondo. Su
figura no podía ser más hermosa.

---¿Vas a salir esta noche? Te acompañaré. Me aburre este agujero. En
Madrid se respira, amigo mío, el aliento sulfúreo de la revolución. La
conmoción viene, el trueno retumba ya muy cerca.

Salimos juntos. Habíase disipado en gran parte mi miedo, y la compañía
de Monsalud infundíame valor. Desde los primeros encuentros con varias
personas conocidas, comprendimos que no corría ya gran peligro nuestra
libertad. Las noticias eran tremendas para el absolutismo, y según
dijeron, se preparaba para el día siguiente un decreto haciendo
concesiones y prometiendo reunir Cortes. Tanta cobardía inflamaba más a
los revolucionarios.

Visitamos aquella noche con el mayor descaro algunas tertulias, que no
eran otra cosa que las mismas reuniones perseguidas por D. Buenaventura;
pero con la súbita esperanza de triunfo, la revolución había arrojado la
máscara y se burlaba del Gobierno. En este no había un solo ministro
propio para la gravedad del caso. Hombres todos de miserable espíritu,
no servían más que para la adulación. Todo Madrid se reía de ellos. Los
conspiradores que no estaban presos afectaban en las calles y en sitios
públicos un desprecio a la autoridad que rayaba en desvergüenza.

Al día siguiente, tranquilos ya con el aspecto que tomaban las cosas,
abandonamos Salvador y yo el escondrijo del Sr.~Mano de Mortero, y
tuvimos hospitalidad en casa de un amigo.

Era el 6 de Marzo, cuando llegó la noticia de la sublevación de las
tropas que estaban en Ocaña. El júbilo y osadía de los revolucionarios
eran tan grandes, que por momentos se temía en Madrid un alzamiento
popular. La atención de todos se fijaba en la guarnición de Madrid,
formada de algunos regimientos de la Guardia y de otros de línea. En
Palacio, según me dijo el Sr. Villela, a quien encontré en un estado de
indecisión extraordinaria, todo era tumulto y azoramiento. La Reina
Amalia lloraba, el Rey bufaba de ira y los palaciegos iban y venían
consternados, sin saber si pondrían la vela al santo o al demonio, o a
entrambos a la vez, que era lo más seguro. Escondíanse el duque de
Alagón y los demás favoritos, y diversos personajes, oscurecidos u
olvidados por la corte, se presentaban llamados por el Rey o espoleados
por su propia ambición.

Desde que amaneció el día 7, Madrid ofrecía el aspecto propio de los
días en que va a pasar algo extraordinario. Inútil es decir que desde
muy temprano recorrí yo las principales calles, en unión de algunos
individuos que iban sembrando la semilla del tumulto de barrio en
barrio. Recordaba yo las escenas famosas del 1.º de Mayo de 1814, y me
parecía que nada había cambiado. Las caras eran las mismas, los gritos
parecidos. Ciertamente que la idea era distinta; pero como la idea no se
ve, de aquí la ilusión.

No hay cosa más parecida a un motín absolutista que un motín
revolucionario. Se asemejan como una calabaza a otra. No trabajar,
cerrar las tiendas, salir chillando, derribar lápidas y letreros,
injuriar a los caídos, proclamar nombres nuevos, levantar ídolos,
mezclar tal o cual arranque generoso a salvajes actos, esto fue lo que
vi en 1814, y lo que se repitió ante mis ojos en 1820. En una y otra
época, por rara coincidencia, fui agente eficaz en el movimiento, y las
dos veces mi astuto aguijón pinchó a la bestia feroz para que gruñese.
Antes había gruñido en las Cortes; ahora debía gruñir en Palacio.

Comprendiendo la gravedad del asunto y la conveniencia de que el trabajo
de seis años no se malograse, desplegué aquella mañana facultades
verdaderamente maravillosas que llenaron de asombro a los
revolucionarios viejos. Ya se comprenderá que los nuevos éramos atroces.
No perdonábamos.

Debo advertir que en Marzo de 1820 yo notaba en la población un
movimiento mucho más espontáneo y general que en Mayo de 1814. Todos los
tenderos, todo el comercio alto y bajo de los barrios del Sur y del
Centro se asociaba al impulso con una franca y natural alegría que me
llenó de admiración. En los empleados, en todo el personal de la clase
media, había un sentimiento de simpatía que más tarde llegó a
manifestarse en hechos. Había, pues, en aquel día dos corrientes, la
corriente natural de la gente de buena fe que se alegraba del cambio
previsto, y la corriente del tumulto, que tenía encargo de vociferar y
hacer demostraciones locas. Ambas se mezclaban y juntas invadían las
calles, llenando los aires con sordo mugido, sin que se pudiese
determinar dónde acababa el oro y empezaba el plomo. En la generalidad
de la población resplandecía la más franca hombría de bien, una especie
de candor revolucionario, si así puede decirse, un júbilo patriarcal que
era del mejor augurio.

Por la tarde la muchedumbre formaba una apretada masa en los alrededores
de Palacio. Escenas bulliciosas de animación, de risas, de plácemes, de
gritos, de palabrillas un poco jacobinas alegraban las calles del Arenal
y Mayor.

«Que el Rey juraba.

»Que el Rey no deseaba otra cosa que jurar.

»Que los ministros y palaciegos eran unos tunantes, pero que Fernando el
hombre mejor del mundo.

»Que, a Dios gracias, nos íbamos a ver libres de pillos.

»Que en aquellos momentos se estaba formando un nuevo Gobierno.

»Que por la noche la guarnición de Madrid, incluso la guardia real,
debía apoderarse del Retiro, para desde allí enviar una diputación al
Rey pidiéndole el juramento consabido.

»Que la Reina decía entre lágrimas y suspiros que la habían engañado, y
que se quería volver a Sajonia.

»Que Ballesteros, recién llegado por mandato del Rey, había dicho que
nada se podía hacer ya.

»Que los hombres de la corte opinaban que no era cosa de trastornar al
Reino y de pasar sustos por un juramento de más o de menos».

Esto y otras cosas que omitimos decía la gente. Yo no quise hacer
demostraciones en público; pero me daba a conocer a todos mis amigos, no
recatándome de nadie, porque ya no había para qué. Con los liberales me
hacía el exaltado y con los templados el indiferente.

Cerca de Palacio, la multitud prorrumpía en desaforados gritos: allí
estaba nuestra gente pidiendo a voces la Constitución y el juramento con
tanto ardor, que parecía no poderse pasar ni un momento más sin ello.
Pero los balcones de Palacio permanecían cerrados; no se veía ni aun la
nariz del Infante D. Carlos, generalísimo de los ejércitos.

Iba cayendo la tarde, y no había novedad. Algunos jinetes de la guardia
decían al pueblo que se retirase. Su actitud no era hostil, sino tan
conciliadora, que despertaba general simpatía. La guardia, que tanto dio
que hacer después, estaba aquel día como un guante. Verdad es que aquel
día era un fenómeno por la generalización súbita de los sentimientos
liberales. Había contagio sin duda. Los exaltados contagiaban a los
tibios; los tibios a los indiferentes; los hombres contagiaban a las
mujeres, las mujeres a los niños, y los niños a los pájaros, que de rama
en rama piaban \emph{Constitución}.

La noche enfrió el entusiasmo de muchos; pero exacerbó más el furor de
otros. Aquellos que a toda costa deseaban una escena y la pedían y la
estaban buscando, no querían irse a sus casas sin saber la determinación
de Su Majestad. Diversas comisiones entraron en Palacio, pero el pueblo
ignoraba todo. Por eso cuando corrieron voces de que era inútil esperar
nada positivo hasta la mañana siguiente, un bramido de despecho circuló
de un cabo a otro. Gracias a que nuestro pueblo es dócil, poco exigente,
humilde, y conserva sentimientos de profundo respeto al Trono en medio
de sus más soeces expansiones, que si no fuera así, algo grave habría
ocurrido aquella noche.

Mientras los vecinos se iban a sus casas o a las tertulias o a los
cafés, los que mangoneábamos en la maquinaria oculta del alboroto
popular, azuzábamos a los beneméritos patriotas para que manifestasen
sus altas dotes, ora rompiendo algunos vidrios absolutistas, ora
entonando canciones que a toda prisa improvisaron ramplonas musas. Todo
lo hicieron a pedir de boca; pero aquello donde más lució su destreza
fue la algazara que armaron en la Plaza Mayor al poner una lapidilla
provisional, que más tarde fue sustituida por otra de mármol. Diversas
turbas, roncas a fuerza de gritos y aguardiente, daban vivas a la
Constitución, y había grupos carnavalescos, semejantes a los que forman
los gallegos la víspera de los Santos Reyes.

Aquella vez, entre lucientes antorchas no llevaban escaleras, sino el
libro de la Constitución, abierto e izado en un palo. La gracia de esta
apoteosis consistía en hacer que todo transeúnte besase el libro, previa
inclinación del palo hacia el suelo. Se obligaba a los transeúntes a
ponerse de rodillas, siendo de notar que la mayor parte lo hacían de muy
buen grado. Fuera de este inocente desahoguillo, no hubo ningún exceso
aquella noche, ni se vertió sangre, ni nadie fue arrastrado, ni se
realizó ninguno de aquellos siniestros augurios que en tiempo de la
conspiración se hacían. Todo era una especie de juego de chiquillos.

Así pasó la noche. Ya no tuve recelo de entrar en mi casa, en la cual
encontré aún dos o tres polizontes, que me recibieron sombrero en mano,
con exagerados cumplidos y servilismo. Yo les mire de un modo altanero,
y entonces cada uno de ellos me rogó que le proporcionase un ascenso,
puesto que ya de vencido me trocaba en vencedor e iba a estar pronto en
candelero. Prometiles a tan guapos chicos mi favor, y se despidieron
diciendo que si el nuevo Gobierno les mandaba prender a D. Buenaventura,
lo harían de mil amores. Por último, les recomendé que al día siguiente
muy de mañana saliesen por las calles dando vivas a la Constitución y a
la libertad, que vigilasen la casa de Baraona por ver si entraban en
ella gentes sospechosas, y que se pusiesen en todos los sucesos del día
al lado de los buenos y ardientes patriotas.

El 8 fue día de júbilo, de triunfo, de algazara, de expansión
incomparable. El pueblo, más niño en las buenas que en las malas,
parecía haber recibido un juguete por mucho tiempo deseado. Viendo tanto
entusiasmo, ¡quién creería que bien pronto el muñeco había de ser hecho
pedazos por las mismas manos que entonces le recibían! Todo estaba
consumado; la revolución estaba hecha; lo de arriba había pasado abajo y
lo de abajo arriba; la cabeza era pie y el pie cabeza; la soberanía del
pueblo, representada en un papel escrito, había subido al majestuoso
zenit del Estado, echando de allí a la soberanía real para ponerla
debajo. La gran jugarreta que hacen los siglos a los siglos estaba
consumada, y el hoy había triunfado sobre el ayer. El Monarca de derecho
divino, el escogido de Dios, se había prosternado moralmente ante los
gallegos, que, cual comparsa de noche de Reyes, recorrían las calles con
escobas encendidas, y había besado de rodillas el libro puesto en un
palo. Ya era público el famoso decreto del 7 de Marzo, y desde muy
temprano no había ciudadano de la improvisada nación constitucional que
no repitiese el \emph{me he decidido a jurar la Constitución promulgada
por las Cortes generales y extraordinarias de} 1812. \emph{Tendreislo
entendido\ldots{} etc\ldots{}}

\hypertarget{xxv}{%
\chapter{XXV}\label{xxv}}

¡Cobardía y debilidad!\ldots{} Pero a mí no me importaba averiguar los
sentimientos que dictaron aquella resolución, y salí gritando como todo
el pueblo, como los discretos y los ignorantes, como los ancianos y las
mujeres, como las viejas y los chiquillos de escuela: \emph{¡Viva la
Constitución!\ldots{}} Era una fiesta nacional, un desbordamiento
impetuoso de alegría: ¡la mayor parte no sabían por qué! Se alegraban
por el gozo extraño.

En todos los balcones pendían cortinas, las famosas y eternas y
apolilladas guirindolas que habían festejado la primera entrada de
Fernando en Abril del año 8, la entrada de Wellington después de
Arapiles, la proclamación de la Constitución en Agosto del 12 y su caída
en Mayo del 13, la segunda arrebatadora entrada del ídolo al volver de
Valencey, la entrada de Isabel, que había pasado por el Trono como una
sombra simpática y bienhechora, y la de Amalia, que, rosario en mano,
sustituyó a Isabel. Las cortinas se iban ya poniendo algo viejas. ¿Qué
dirían ellas de tantas y tan repetidas ventilaciones como recibían por
distintos motivos? El viejo y miserable caserío de entonces, no renovado
completamente todavía, cubierto de harapos rojos y blancos, tenía
perfecta similitud con una risueña cara de vieja emperifollada. La gente
invadía las calles. En estos días el vecindario, con irresistible
impulso de bullanguería, siente un aguijón que lo expulsa de las casas.
Hay necesidad absoluta de salir, de preguntar lo que ya se sabe, de
comunicar las impresiones, los sustos y las alegrías. Al mismo tiempo y
mientras se empavesaban los balcones, mil candilejas, puestas en los
antepechos y goteando su aleve aceite sobre los transeúntes, amenazaban
con una iluminación general en la próxima noche. Lozano de Torres
hubiera creído que la Reina estaba de parto.

Imposible es para mí describir las manifestaciones cariñosas de que fui
objeto. La gratitud, llenando mi corazón, ahogaba mi voz. Todos me
felicitaban, me estrechaban la mano, dándome parabienes por mi libertad
y por el fin de la horrible persecución que había sufrido. Rogábanme
otros que les tuviese presentes; los liberales me ponían en las nubes, y
los absolutistas, buscando el modo más decoroso de elogiar la
revolución, decían: «Es preciso confesar que se ha hecho muy bien; ni
una gota de sangre, ni un atropello. En verdad que no me asusta la
revolución. Yo pensé que era otra cosa».

Todo era abrazarse y congratularse. ¡Qué hombres tan negros blanquearon
su semblante con la sonrisilla del regodeo liberal! ¡Qué trasmutación de
rostros, qué quitar y poner de caretas, conforme el caso exigía! Muchos
derramaban lágrimas.

En la calle Mayor encontré a Salvador Monsalud, a quien no había visto
desde la noche del 6, y al punto corrí a abrazarle. Estaba regocijado
sin exaltación.

---¿Qué te parece---le dije,---el hermoso, el ejemplar espectáculo que
están dando Madrid y la Nación? Esto es un modelo de pueblos sensatos.
Di ahora que no sabemos practicar la libertad.

---El primer día---repuso,---todo es concordia y festejos. No quiero
decir que no sea muy satisfactorio. Estoy contento, y este espectáculo
llena mi alma de alegría.

---Y disipará tus dudas ridículas.

---Eso no; las conservo---repuso.---Aquí, todo lo que pasa tiene un
sello oficial que destruye la espontaneidad. Yo he visto los pueblos del
campo y las pequeñas ciudades, que es ver la Nación desnuda y entregada
a sí misma obrando por su propio impulso; y lo que he visto me ha
infundido ideas que tus banderolas no pueden disipar.

---¿Asegurarás que no hay aquí un verdadero amor a la Constitución?

---Aquí sí, aunque ese amor no será tampoco muy firme\ldots{} Sin
embargo, fuerza es aprovechar lo que existe, poco o mucho, y trabajar
sobre ello.

---Pues a trabajar. Has de saber, amigo, que aún falta mucho que hacer.
Todavía puede volverse la tortilla. No nos fiemos de promesas. Es
indispensable que el Rey nos dé una garantía sólida. ¿Vienes conmigo? Es
preciso alborotar mucho esta tarde.

---Pues entonces no voy. Alborota tú.

---¡Vaya un revolucionario!

---Cada uno lo es a su modo. Si la mudanza deseada está ya hecha, ¿a qué
más ruido?

---Amiguito, es que todavía falta lo mejor---contesté con mucho
apuro.---Estamos en el momento crítico. Se ha de nombrar una junta,
ayuntamiento, autoridades, cualesquiera que ellas sean. Si no acudimos
en el primer momento de la marejada, y metemos ruido y nos ponemos en
primer lugar, es fácil que nos quedemos fuera. ¿Vienes?

---No quiero ser autoridad.

---¿Pero qué hay en ti? ¿Qué calma es esa? ¿A dónde vas?\ldots{}
Ya\ldots{} perplejidades de hombre enamorado, que no piensa más que en
su dama. Salvador, ten juicio, sé al fin un verdadero y grave hombre
político, un hombre de orden, un padre de la patria, un sostén del
Estado\ldots{}

---Adiós---me dijo riendo.

---Pero ¿a dónde vas?

---A prepararme. Saldré mañana de Madrid.

---¡Ahora!---exclamé en la mayor confusión.---¡Salir de Madrid, es decir
de Jauja!\ldots{}

---Voy a Logroño a reunirme con mi madre, que debe de estar libre.
Después iremos a la Puebla. Volveré a Madrid.

---Volverás. No creas que me olvidaré de ti. Al contrario\ldots{} Yo te
aconsejo que optes por \emph{Paja y Utensilios}. Ahí empecé yo\ldots{}
Puedes ir descuidado. Yo velaré por ti, Salvador. Dale expresiones a
Doña Fermina\ldots{} ¡apreciable señora!\ldots{} ¿Sabes---añadí
riendo,---que los Baraonas y Garrotes habrán tragado a estas horas mucha
hiel? Infames servilones\ldots{} ¡Qué bien merecido les está!\ldots{}
Dime, ¿piensas sentarle la mano a Carlos, como dijiste?

---Tal vez no---repuso Monsalud con tristeza.---Están caídos y les
perdono.

---¡Generosidad ridícula!\ldots{} ¿Sabes lo que me han dicho esos guapos
chicos de la policía? Que ayer y anoche han entrado misteriosamente en
casa de Garrote algunos pájaros gordos, Eguía, el marqués de M***,
Alagón. Me parece que traman algo. ¡Qué buena ocasión para darles un
susto! Yo estoy muy ocupado; encárgate tú. Me alegraría de que les
pusieras las peras a cuarto. Yo te proporcionaré media docena de
ciudadanos que te acompañen con buenos garrotes\ldots{} Anda, hombre,
anímate.

---En caso de ir, iría solo\ldots{} Pero hemos vencido; basta ya de
violencia. El derrotado bastante amargura tiene en su derrota. Seamos
generosos.

---Pues adiós. Voy a ver lo que se hace esta tarde. Que escribas\ldots{}
Pídeme lo que quieras. Aunque nunca me has dicho nada\ldots{} en fin,
por algo se empieza. Haré por ti lo que pueda\ldots{} habrá tantas
solicitudes, tantas pretensiones, serán tantos los que abran la
boca\ldots{} pero no te olvidaré, no.

---Adiós---me dijo estrechándome la mano cordialmente y sin hacer caso
de mis últimas palabras.

\hypertarget{xxvi}{%
\chapter{XXVI}\label{xxvi}}

El Rey había prometido jurar; pero no juraba, ni se nombraba nuevo
Gobierno, ni siquiera nuevo Ayuntamiento. Estábamos a merced de un golpe
de mano, y si el ejército había dado al país la libertad, el ejército
podía quitársela de la noche a la mañana. Las reuniones secretas, que ya
eran públicas, trabajaron toda la tarde y parte de la noche, mientras
seguían las demostraciones populares, juego inocente que nos daba risa.

Amaneció el día 9, el gran día. El pueblo, aguijoneado por quien sabía
hacerlo, se reunió en los alrededores de Palacio, puso su planta en la
puerta y dijo que quería entrar. La guardia callaba y dejaba hacer. El
pueblo entró en el patio grande y se paseó de un extremo a otro, dando
gritos y entonando las canciones de aquellos días. Por los vidrios de la
galería alta asomaban las caras pálidas de medrosas damas y tímidos
palaciegos que preveían un desastre. Cansado de esperar en el patio, el
importuno visitante bramaba de impaciencia. Era aquella una visita que
no se hace todos los días, y como cosa nueva carecía de reglas de
etiqueta. El pueblo, pues, anhelaba subir antes de que se lo mandasen, o
antes que lo echaran a la calle. El amo de la casa, sintiendo desde su
gabinete el resoplido del animal que tan descortésmente quería penetrar
hasta él, se sentaba y se levantaba, reía y bufaba, y a ratos pálido, a
ratos rojo, dirigía preguntas a todos. Hubiera deseado que su mirada
fuese un rayo que desde arriba, traspasando las paredes, cayese sobre la
bestia y la aniquilara.

Al mismo tiempo el amo de la casa forjaba proyectos de venganza y
estudiaba un papel, papel difícil que rara vez se desempeña bien ante el
peligro. No es lo mismo recibir al cuerpo diplomático entre sonrisas de
oficio y estudiadas fórmulas, que recibir al pueblo entre rugidos.

Fernando no se atrevía a formular el terrible \emph{que pase adelante}.
Pero el pueblo parecía dispuesto a colarse sin que se lo mandaran.
Inquietos pero decididos los de abajo, inquieto y vacilante el de
arriba, no era fácil prever en qué iba a parar aquello. ¡Si hubiera
habido un batallón de la guardia dispuesto a desafiar las
navajas!\ldots{} pero los emperejilados guardias se mantenían tiesos y
hermosos, empuñando sus armas como empuñaban sus palitos blancos las
figuras del tío Mano de Mortero.

Por último, todos tomaron una resolución, los de abajo y el de arriba.
La visita quería posesionarse del estrado; el señor había dispuesto
enviar un mensaje a los del patio, rogándoles y prometiendo. Estos
habían nombrado una comisión. La comisión y los mensajeros del Rey se
encontraron en la escalera. Allí hubo expresiones benévolas, un cambio
feliz de sentimientos conciliadores, y el asunto empezó a tomar aspecto
risueño. Subieron al fin los comisionados que eran seis, y al poco rato
bajaron con la noticia de que Su Majestad había mandado al marqués de
Miraflores que estableciese el Ayuntamiento del año 14.

El Palacio quedó poco a poco libre y el movimiento del pueblo era en
dirección a la Casa de la Villa. Los que deseaban mangonear en los
primeros momentos y coger para sí los primeros peces del revuelto río,
no tenían tiempo que perder. Yo fui de los más veloces en invadir las
Casas Consistoriales, en ocupar las oficinas, en apoderarme de una resma
de papel de oficio, en expedir órdenes menudas a los subalternos. Así es
que cuando Miraflores llegó, ya estaba yo allí dictando leyes, como un
déspota, expidiendo órdenes y preparándolo todo para el gran acto que se
iba a realizar.

De buena gana me hubiera nombrado alcalde a mí mismo; pero yo no era del
14. Con aquella presteza febril y verdaderamente maravillosa que yo
tenía para las improvisaciones oficinescas, me impuse desde el primer
momento, y a los diez minutos de intrusión, ya no podía hacerse nada sin
mí. Yo solo sabía dónde estaban los pliegos, yo solo sabía en qué
términos debían hacerse los oficios, cómo se había de ordenar lo que
entonces se llamaba la \emph{Tabla del Excelentísimo Ayuntamiento}.

También salí al balcón con otros, teniendo la suerte de enjaretar unos
parrafillos tan bien dichos, tan conmovedores y del caso, que me
aplaudieron frenéticamente. Yo fui quien inauguró los abrazos que tanto
entusiasmaron a la generosa muchedumbre. Sin más ni más abracé al que
tenía a mi lado, un liberalote furioso de toda su vida; este abrazó al
vecino, y entre lágrimas y patrióticos pucheros nos abrazamos todos
repetidas veces. Yo gritaba: «¡Se acabaron las discordias, se acabaron
los odios! ¡Ya no hay más que españoles leales y amantes de la
Constitución! Todos son hermanos. ¡Viva España, que es la Nación más
sabia y más gloriosa del mundo! ¡Viva la Constitución! ¡Viva el Rey!»

¿Quién puede olvidar aquellos sublimes instantes? ¡Inefable día!

El marqués de Miraflores iba pronunciando los nombres de los individuos
del Ayuntamiento. El pueblo aplaudía o denegaba, gritando: \emph{bien,
bien, o ése no, ése no que es servil}. Concluido esto, dirigiose a
Palacio el Ayuntamiento recién establecido, para recibir el juramento de
Su Majestad, y por el tránsito todo fue bullicio, loca alegría, vivas
roncos, embriaguez indescriptible. Poco después, Madrid entero sabía que
Fernando VII había jurado la Constitución.

¡Viva el Rey! Ya todo estaba hecho. Ya podían venir las iluminaciones,
los festejos, las alegrías, las ceremonias llenas de exaltación política
mezclada de religioso entusiasmo. Una nueva era se presentaba, una nueva
era, sí, vasto campo a la actividad de los hombres listos. Yo no salí
aquel día del Ayuntamiento y trabajé con ardor en diversos asuntos.

El 10 apareció el Manifiesto en que están las célebres palabras:
\emph{Marchemos francamente, y yo el primero, por la senda
constitucional}. El 14 dio D. Carlos su programa al ejército,
congratulándose del juramento de la Constitución. El mismo día 9 nombró
Su Majestad la Junta provisional consultiva que debía suplir al
Ministerio mientras este se formaba, y tuve tan buena mano y tacto, que
me congracié soberanamente con todos y cada uno de los esclarecidos
individuos de ella, en tales términos, que no sabían cómo recompensar
mis servicios. Estos eran importantísimos. Yo estaba siempre en primer
término; yo salía siempre al encuentro de todo; yo era la previsión, el
cálculo, la prudencia. Híceme de tal modo necesario, que mi nombre
sonaba aquí y allí donde quiera que ocurrían dificultades. Debía esto a
mi tino para todo, a mi destreza y experiencia suma de los hombres y las
cosas. Por eso supe encaramarme dentro de la revolución a puestos tan
altos como los que ocupé dentro del absolutismo, y en uno de los
primeros consejos de ministros que se celebraron se acordó darme la
plaza de consejero, \emph{en premio de los servicios que había prestado
al liberalismo, y como compensación de las horribles persecuciones de
que había sido objeto}.

¡Ventura incomparable! ¡Qué bien sentaba a mi gallardo cuerpo la nueva
casaca! ¡Cómo me reía yo de D. Buenaventura y de todos aquellos
vanidosos prohombres que me habían postergado en 1819! Ellos purgaban
sus culpas con la ignominia que les resultaba de humillarse ante la
revolución, después de haberle combatido hasta el último momento. Verdad
es que pronto le declararon nueva guerra; pero fue porque la revolución,
despreciándoles, no quiso nada de ellos ni con ellos.

Largo tiempo estuve en gracia con la revolución, la cual no era tan
fiera como nos la pintábamos los absolutistas cuando la combatíamos.
¡Matrona más condescendiente no la vieron mis ojos! ¡Qué excelente
señora! En muchas, en muchísimas cosas del Gobierno apenas se conocía su
existencia. Verdad es que sus noveles servidores hacíamos lo posible por
ponerle una venda en los ojos para que nada viese y renunciase a la
fatal manía de innovar, que era su flaco. Con mi nuevo y flamante
destino renació la dicha en mi alma y la holgura en mi casa, que ya se
iba desmejorando con el largo vagar; me vi de nuevo favorecido y adulado
por grandes y chicos, y Su Majestad me mandó asistir a sus tertulias. El
pobrecito no podía pasarse sin mí.

\noindent{\dotfill}

No puedo seguir, no puedo hablar más, porque la alegría embarga mi
espíritu y ahoga mi voz. Aunque algo sé digno de contarse, lo entrego a
otro narrador para que con más aliento que yo lo continúe; y postrado y
sin fuerzas doy fin aquí a mis curiosas Memorias, encargando al copista
de ellas que me sustituya en las últimas páginas de este libro.

\hypertarget{xxvii}{%
\chapter{XXVII}\label{xxvii}}

Concluidas las \emph{Memorias} que por dichosa casualidad vinieron a
nuestras manos, seguimos contando por cuenta propia.

El 8 de Marzo, uno de los tres días de bulliciosa huelga que sirvieron
de introito a la revolución, un anciano avanzaba al caer de la tarde por
la plazuela de Santo Domingo, en dirección a la calle Ancha de San
Bernardo. Su paso era vacilante; su actitud la de un descaecimiento
lamentable. Fijaba la vista en el suelo y movía la cabeza, cual si no
tuviera en su cuello fuerza suficiente para mantenerla derecha. A ratos
hacía con los brazos y las manos súbito movimiento, como el de quien se
ocupa en cazar moscas. Hablaba consigo mismo y daba bastonazos en el
suelo tan fuertemente como los ciegos que reconocen el terreno. Su
cuerpo encorvado tropezaba a menudo con los transeúntes, sin que el
choque le distrajera de su penosa marcha meditabunda.

Al llegar a la entrada de la calle Ancha, un obstáculo que no podía
vencer le detuvo. Tropezó con una muralla. Había allí tanta gente
reunida que no se podía seguir.

---¡Otra pared de carne!\ldots---gruñó el viejo con impaciencia.---¡Y no
hay quien la derribe a cañonazos!

Trató de abrirse paso, pero no pudo. Se abría ante él un boquete; pero
al punto se volvía a cerrar, dejándole tapiado dentro de una ardiente
mampostería de brazos, muslos y espaldas. El viejo movía sus codos y
avanzaba la mano y el palo como una cuña. En una de estas, dos piedras
enormes se juntaron, cogiéndole en medio y exprimiéndole sin piedad.

---¡Mil demonios!---chilló el viejo con voz angustiosa.---Que me
aplastan ustedes\ldots{} Atrás, animales\ldots{} Dejen pasar a un hombre
de bien, que no se mete en estas danzas y aborrece la
bullanguería\ldots{} ¡Eh!, so bruto que me destroza usted con su anca.

---¡Maldito viejo!---gritó uno de los más cercanos.---¿Para qué se
meterán entre el gentío estos escarabajos? ¡Hermano, váyase al hospital!

---Si todo el mundo estuviera en su casa---dijo el anciano,---si el
Gobierno no permitiera estas atrocidades ridículas, no se obstruirían
las calles.

---¿Quién es ese cernícalo que grazna?

---Señor abate, señor capellán, señor sepulturero o lo que sea---dijo un
individuo en tono compasivo,---sálgase usted de este laberinto, porque
le van a hacer tortilla.

---¡Paso, paso!---gritaba el viejo con un arranque de cólera y de
energía que contrastaba extraordinariamente con su miserable
cuerpo.---¿No hay quien meta en cintura a esta canalla?

En torno al anciano se elevó un murmullo siniestro, entre burlón y
hostil, que hubiera asustado a otro, pero que a él no le alteró; tan
grande era su ánimo.

---Sí, lo repito---añadió echando fuego por los ojos,---estas borricadas
existen, porque no hay un Rey que tenga calzones.

Diciendo esto, el sombrero del anciano voló por los aires, y unas manos
vigorosas, cogiéndole ambas orejas, le hicieron dar grotescas cabezadas.
Risas generales celebraron el hecho. El pobre viejo rugía como un noble
animal prisionero e insultado. Todo cuanto la lengua contiene de
festivo, de grosero, de ignominioso y de mordaz resonó en las insolentes
bocas. El anciano fue empujado, estrujado, arrastrado y su endeble
cuerpo, escurriéndose dolorosamente por una grieta, erizada de agudos
codos y de crueles manos, fue a chocar contra una pared de la calle de
la Inquisición. Pegado a ella, con las manos cruzadas, la boca
espumante; llenos de luz y de ponzoña los ojos vengativos, parecía una
pantera vieja, que en su agonía estaba resuelta a hacer estragos.

---¡Miserables!, ¿pensáis que os temo?---exclamó más bien rugiendo que
hablando.---Yo no temo a nadie, yo no temo a indignas sabandijas que
huyen del peligro y se ensañan picando a los débiles; yo temo a hombres
valientes; no a una vil chusma gritona.

---Es un demente---repitieron varias voces.

---Es un hombre de bien---gritó él,---es un buen patricio, es un
cristiano, es un español. Cáfila de rateros y farsantes, respetad a los
que nunca han robado, ni conspirado, ni maldecido a Dios, ni hecho
revoluciones; respetadle o no faltará quien os enseñe a hacerlo.

Una mano cogió el cuello del frenético viejo, y otra mano le golpeó.

---Está bien---dijo con voz ahogada cuando quedó libre.---De este modo
abofetearon a Cristo. Escúpeme también, sayón.

Le golpearon de nuevo, y el anciano añadió:

---Está bien. Burro, acepto tus coces.

---Dejarle; es un pobre viejo inofensivo---indicó una voz.---¿No veis
que está demente?

---Desprecio tu misericordia---gritó el inexorable hombre caído.---Si no
insultarais, si no escupierais, si no deshonrarais, si no rebuznarais,
no seríais lo que sois: masones, revolucionarios, ateos, jacobinos.

---Vamos, padrito; levántese usted y se le dará un vaso de agua.

---Aparta tus manos de mí---repuso con desprecio,---y ve a coger las
tijeras, sastre. No abras tu boca para hablarme, y ve a mascar la suela,
zapatero. No me toques y ve a espumar los pucheros, pinche. Soy un
caballero. Señores sastres, zapateros, pinches y albéitares, que hacéis
revoluciones y quitáis al Rey sus derechos y enmendáis la obra de Dios,
buscad para vuestra miserable obra un Reino que no sea este Reino de
España, esta tierra de caballeros, de santos, de soldados\ldots{}

¡Cómo se reían al oírle!

---Haced revoluciones---prosiguió,---degradad más el suelo que pisamos;
manchadlo todo, imbéciles. Haced un estercolero con las banderas
gloriosas, con los laureles, con las coronas de santos y reyes, y el
Demonio estará contento\ldots{} Poned la historia toda bajo vuestras
patas y bailad encima, acompañados del Cabrón. El Infierno triunfa.

Dicho esto lanzó una carcajada siniestra.

---Es un servil---dijeron algunos.

---No hacerle daño---añadió un compasivo.

---Colgarle de una reja de la Inquisición---añadió un cruel.

En aquel instante todas las miradas se fijaron en un edificio, a cuya
puerta el gentío se apretaba, cual si todos quisieran entrar a un
tiempo. Era la Inquisición de Corte, cuyo frontispicio, marcado hoy con
el número 4 de la calle de Isabel la Católica, nada tenía de particular.
Componíase de algunas ventanas y una puerta grotesca en el piso bajo, de
una serie de balcones en el piso principal y de varios huequecillos
enrejados en el sótano. Los balcones estaban llenos de paisanos. En la
calle y arriba el general bramido de triunfo e impaciencia formaba una
algarabía infernal. Un hombre echó el cuerpo fuera en el balcón
principal, y sacudiendo las manos arrojó una gran masa de papeles que
cayeron a la calle. Multitud de hojas quedaban suspendidas y flotando de
aquí para allí, llevadas por el viento. Iban y venían como pájaros que
han recobrado la libertad. Eran las causas de la Inquisición. El pueblo
soberano estaba inventariando a su modo el archivo.

Casi todos querían entrar para ver los terribles calabozos. Penetraron
muchos; pero salían descorazonados, diciendo que todo había sido
ocultado a tiempo y que no restaba nada. Quién sacó una tarima de
brasero, quién un fuelle roto; este una sartén vieja, aquel un cazo. No
se encontraron otros instrumentos de tortura. De repente un individuo
apareció en la puerta principal. Venía cargado de extrañas cosas.
Arrojolo todo en el suelo, diciendo así:

---Ahí están las picardías.

Una lluvia de soldaditos a pie y a caballo, de muñecos articulados, de
peones, de animalillos de cartón, de reyes magos, de pastores de Belén,
de panderetas y rabeles, cayó sobre las cabezas y los hombros del
gentío. Carcajadas generales acogieron el regalo.

Después de esto despejose un tanto el terreno, y una turba de chiquillos
cayó, cual manada de lobos, sobre tan rica presa.

Poco después oyose un rumor de júbilo. Por el portal grande apareció un
grupo de gente gritona, que sobre sus hombros, a manera de trofeo
glorioso, sacaba tres personajes, nada flacos ni extenuados. Eran los
únicos presos que se encontraron en el piso alto del edificio; uno de
ellos, D. Luis Ducós, rector de Hospitalarios.

Tras la procesión siguió toda la muchedumbre, dando vivas a la libertad,
y la calle de la Inquisición empezó a despejarse, mientras se llenaba la
de Torija, junto al edificio de la Suprema.

Era ya completamente de noche, y el infeliz viejo a quien dejamos
rugiendo de cólera entre un grupo de ciudadanos, continuaba en el mismo
sitio, arrojado en el suelo, con la espalda y la cabeza apoyadas en la
pared. No hablaba ya ni se movía. Un hilo de sangre corría por su
rostro, desapareciendo por el cuello entre la ropa. En derredor suyo
había nuevo corro de ciudadanos, pero de ciudadanos prudentes y
compasivos, que en silencio le miraban, guardando religiosa compostura
en torno suyo, sin atreverse a tocarle, llenos de curiosidad y aun de
respeto. Eran Currito el de la carbonera, de ocho años; Joselito
González, el del covachuelista, de siete; Paco el de D. Robustiano, de
diez; Isidorillo, el de la tía Rampiosa, de seis y medio, y otros que la
historia y la tradición no recuerdan bien. Entre todos eran una docena.
Cada cual llevaba en su mano un objeto de los que estaban desparramados
en la calle ante la puerta de la Inquisición.

Acercábase uno a mirar de cerca el rostro del anciano, y con ademán
pavoroso decía: «Está muerto». Reían todos, mirándose unos a otros, y ya
se disponían a retirarse juntos, cuando Isidorillo el de la tía
Rampiosa, que por ser el más chico era el más travieso de todos, tuvo
una feliz idea, que al instante puso en ejecución. Llevaba en la mano
una varita delgada y larga, y con la punta de ella exploró por dentro la
nariz del desgraciado anciano. Este hizo una mueca, se movió, y un coro
de risas infantiles acompañó a su movimiento.

Abrió el anciano los ojos, miró a todos lados, pasose la mano por la
frente, dio un suspiro\ldots{}

---¡Qué buena turca ha cogido usted, hermano!---dijo Currito el de la
carbonera.

El anciano revolvió sus ojos a todos lados, amedrentando con la fiereza
de ellos al regocijado concurso, y en voz ronca, habló así:

---¡A esto llamáis una revolución! Menguados, si queréis hacer una
verdadera revolución, hacedla; alzad la guillotina; cortadnos la cabeza
a todos los que tenemos en ella la idea de Dios, la idea del deber, la
idea de la justicia, la idea del honor y de la hidalguía\ldots{}
¿Queréis acabar con los buenos?, pues a ello. Combatidnos y se os
vencerá. Matadnos y resucitaremos en otra forma. Pero no, no llaméis
revolución a este conjunto de graznidos y patadas\ldots{} Sois
miserables y grotescos bufones que deshonráis el suelo de la patria.
Apartaos de mí, despreciables bailarines. ¿Creéis que una Nación es el
tabladillo de un teatro?\ldots{} Inmundos tiples, no chilléis más en mi
oído\ldots{} Mi voz atruena.

Una algazara de risas siguió a estas palabras. Los pajarillos piando con
alegría en torno al buitre moribundo, no se hubieran expresado de otro
modo. El anciano hizo esfuerzos por levantarse; sus huesos crujían; pero
al fin lo consiguió y se puso derecho, apoyándose en la pared. Los
ciudadanitos, agrupándose en torno de él, no le dejaban dar los primeros
pasos.

---Fuera de aquí, hombres pequeños---dijo el viejo empujándoles a un
lado y otro.---Queréis hacer revoluciones y ninguno de vosotros alza una
vara del suelo.

Cuando los muchachos se oyeron llamar hombres pequeños, redoblaron las
risas. Siempre con las manos en la pared, siguió andando el viejo. Los
chicos le seguían, tirándole de la ropa e impidiéndole el paso. Él
observaba las fachadas de las casas, como para orientarse; doblaba todas
las esquinas que encontraba al paso. De este modo recorrió lentamente
varias calles, y después de muchas idas y venidas, entró en la de
Amaniel. Los chicos habían ido desertando poco a poco. Al fin Joselito
González, que era el más pesado, le dejó solo. El anciano se detuvo,
reconoció la calle, y con voz débil murmuró: «no es por aquí». Volvió
atrás, dobló varias esquinas, siguió a lo largo de la pared apoyándose
en ella\ldots{} pero sus pies vacilaban, temblaban sus piernas; su
cuerpo abatiose rozando el muro y cayó al suelo sin sentido.

\hypertarget{xxviii}{%
\chapter{XXVIII}\label{xxviii}}

Estaba en la calle de Eguiluz. No pasaba nadie por allí. Poco después,
al extremo de la calle abriose una puerta y aparecieron en un oscuro
hueco dos personas, hombre y mujer; el uno despidiéndose de la otra, a
juzgar por las breves palabras cariñosas que en el silencio de la calle
resonaron sin que ningún extraño las oyera. Después de confundirse los
dos bultos en uno, efecto sin duda de la oscuridad de la noche, se
separaron; la mujer desapareció, y el hombre echó a andar por la calle
adelante, hasta que el obstáculo de un cuerpo atravesado en la acera le
detuvo. En el mismo instante una vieja, llegando por el otro lado, se
detenía también. Inclináronse ambos, examináronle el rostro, le
palparon, le movieron, y el joven dijo:

---Es el Sr.~D. Miguel de Baraona.

Trataron de reanimarle. Respiraba, pero no se movía. El joven, después
de un rato de vacilación, se terció la capa, enlazó con sus brazos
vigorosos el desmadejado cuerpo del anciano, y se lo echó a cuestas como
un saco.

Felizmente el peso del \emph{Patriarca del Zadorra} no era excesivo, ni
el humanitario joven tenía que andar mucho para llegar a la calle de
\emph{Sal si puedes}. Los curiosos que en el camino se le unieron
quedáronse a la puerta de la casa, y él subió solo. Ni porteros ni
criados salieron a su encuentro en la escalera. Abrió la puerta una
criada, y bien pronto sonaron en la casa gritos y lamentos de mujer,
angustiosos diálogos, preguntas, órdenes rápidas.

Baraona fue puesto en el suelo. El que le había llevado permanecía en
pie. Jenara miraba al uno y al otro con muda sorpresa; pero el dolor no
dejaba lugar en su corazón a otro sentimiento. Las dos mujeres,
azoradas, llamaron; acudió un criado; entre todos trasportaron al
enfermo a su cuarto, tendiéndole de largo a largo en la cama. Abrió, al
sentirse en ella los ojos, y lanzando un hondo suspiro, dijo:

---¡Me muero!

---¿Pero está herido?---exclamó Jenara.---Esa sangre\ldots{} ¿Qué le han
hecho? ¡Dios mío!\ldots{} ¡Abuelo!

Interrogaba con los ojos al portador de tan gran desgracia; pero este,
alzando los hombros, decía:

---No sé una palabra. Así le encontré en la calle.

Salió del cuarto, y en el laberinto de los pasillos medio oscuros
preguntó que por dónde se salía.

---Por allí---le indicó Jenara, que a su lado pasó rápidamente,
corriendo en busca de remedios caseros.

Dirigiose el joven a la puerta en el momento en que, abierta por fuera,
daba paso a tres hombres. Carlos avanzó el primero, y tras él sus
inseparables amigos. Vieron a aquel hombre, y la sorpresa les detuvo y
les inmovilizó un instante, como cuando se ve lo imposible.

---¿Qué buscas aquí?---gritó Navarro, mirando colérico a Salvador.

---¡Has entrado aquí!---rugió destempladamente el que llamaban
Zugarramurdi, asiendo al joven por el brazo.

El que llamaban Oricaín corrió a asegurar la puerta.

---¿Qué haces en esta casa?---repitió Navarro con mirada furibunda y
amenazadora.

---Nada---respondió Monsalud, dando un paso hacia la puerta,---y por
eso, me marcho.

La voz de Jenara, que llegó volando más bien que corriendo, puso término
a aquella escena.

---¡Carlos, Carlos!---gritó.---El abuelo enfermo\ldots{} herido\ldots{}
¡Se muere!\ldots{} Este\ldots{} este buen hombre le ha traído de la
calle\ldots{} Un accidente desgraciado, un atropello\ldots{} qué sé yo.
Ven al instante\ldots{}

Navarro miró a Monsalud, como pidiendo más explicaciones.

---Estaba en la calle de Eguiluz, arrojado sin movimiento ni sentido,
sobre la acera---dijo Salvador.---No sé más.

Navarro tomó una determinación súbita.

---Yo averiguaré lo que hay en esto---afirmó.---Oricaín cierra esa
puerta. Zugarramurdi, detén a este hombre.

Y corrió hacia dentro.

Carlos y Jenara se acercaron al lecho del enfermo, e hiciéronle mil
preguntas; vendáronle su herida, le abrigaron, tratando de reanimarle
por todos los medios. Baraona sufría un temblor convulsivo.

---La canalla me ha insultado---murmuró.---Pero les dije cuatro
verdades\ldots{} No pudo conmigo\ldots{} ¡Conmigo no puede nadie!,
¡nadie!

---¿Pero quién, pero quién?\ldots{} Dígame usted quién ha
sido---vociferó ciego de ira Carlos, cerrando los puños.---¡Dígame usted
quién ha sido!

---Muchos, muchísimos. Los revolucionarios---murmuró el enfermo.---Sus
manos inmundas me golpeaban\ldots{} Está bien: ¿no abofetearon los
judíos al Señor?\ldots{}

Carlos rugía como un león y sus dedos se clavaban como garras en los
colchones de la cama.

---Maldito sea yo si no me vengo---gritó.---¿Y usted no recuerda quién
le trajo aquí?

---¿Quién me ha traído?---dijo el anciano con la mayor sorpresa,
abriendo mucho los ojos.---Nadie: he venido yo solo; he venido por mi
pie.

---No sabe lo que se dice---indicó en voz baja Jenara.

---Pero ¿por qué gritáis tanto?---murmuró Baraona cerrando los
ojos.---¿Qué ruido, qué algazara infernal es esa?\ldots{} Callad por
Dios\ldots{} necesito descanso, necesito dormir\ldots{} ¿No habrá nunca
silencio en esta casa?

Cuando esto decía, el silencio era profundo en la habitación. Jenara y
su marido observaban fijamente la fisonomía del enfermo.

Mientras esto ocurría en la alcoba, el señor Zugarramurdi, que era un
hombrazo corpulento, de espesa barba rubia, frente estrecha y miembros
poderosos, se acercaba a Salvador Monsalud en la antesala, y dejando
caer sobre el hombro de este una de sus gruesas manoplas, le decía con
voz áspera y cavernosa:

---¿Sabes quién soy?

---Sí---repuso Salvador mirándole con desprecio.---Ya sé que eres un
bruto.

Oricaín, pequeño, regordete, de ojos negros, cubiertos por una sola ceja
pobladísima y corrida de sien a sien, guardaba la puerta.

---Soy Zugarramurdi---dijo el de este nombre.---Estuve en la batalla de
Vitoria. ¿Te acuerdas de la retirada, juradillo?

---Sí; me acuerdo. Tú estabas entre los mulos.

---¿Te acuerdas del que hirió a nuestro amigo y jefe Carlos
Garrote?---prosiguió el vizcaíno.---¿Recuerdas que yo te guardaba y que
te me escapaste, porque una señora compró a los centinelas?

---¡Déjame!---gritó con violencia Salvador apartando bruscamente el
brazo del guerrillero.---Oricaín, abre esa puerta.

---Ven a abrirla---repuso imperturbablemente el navarro.---¿Sabes quién
soy?

---Sí; ya lo sé: ladrabas en la jauría de Garrote. Abre esa puerta, o
pasaré por encima de ti.

---Ya te espero\ldots---dijo Oricaín;---como no me coges de espaldas, no
hay que temerte.

---Abusáis de mí, porque veis que no llevo armas---dijo Salvador
conteniendo su ira.---Estoy indefenso, porque yo no muerdo como
vosotros.

Carlos se presentó en el mismo instante, fruncido el ceño, pálido el
rostro, con un visible sello de dolor y de desesperación en su grave
persona.

---Carlos---dijo Monsalud.---¿He entrado en una guarida de lobos?

---Es espía de los ateos---dijo Oricaín clavado siempre en la
puerta,---y viene a saber lo que hacemos para contárselo a esa canalla.

---Ha venido a provocarte y a desafiarte---dijo Zugarramurdi.---Nosotros
le enseñaremos a ser comedido.

---¡Carlos!---gritó Monsalud perdiendo toda prudencia.---¡Mira que no
tengo armas!\ldots{} ¡Esto es una infamia!\ldots{}

---¿A qué has venido aquí? Lo mismo te desprecio amigo que enemigo; lo
mismo te desprecio espía que servidor. Vete y di a los revolucionarios
que mañana salimos para Navarra a levantar partidas.

---Yo no soy espía\ldots{} ¿Pagas con tan vil sospecha el servicio que
acabo de hacerte?\ldots{}

---No sé si te debo un servicio o una nueva ofensa.

---Yo no me ocupo de ofenderte---dijo Monsalud con desprecio.---Has sido
conmigo cruel, implacable y sañudo como una fiera. Tu corazón de piedra
no se ha movido al ver los tormentos de una pobre mujer inocente; te has
opuesto a que la pusieran en libertad; has redoblado el furor de los
inquisidores, verdugo. Y sin embargo de esto, cuando ha concluido el
martirio de mi madre; cuando ha venido la revolución, y triunfábamos, y
tenía yo todos los medios para tomar venganza de ti; cuando me era fácil
prenderte, molestarte, denunciarte a los vencedores, nada he hecho
contra ti, Carlos, y no queriendo abusar de la gran ventaja adquirida,
te he perdonado.

---¡Dice que me ha perdonado!\ldots{} ¡que me ha perdonado!---exclamó
Garrote, con el rostro encendido.

---Sí, te he perdonado; he tenido lo que tú no conoces: generosidad.

Navarro permaneció un momento en extraña perplejidad.

---Vamos---dijo al fin con desdeñoso acento de ironía,---es un modo raro
de pedir misericordia. Salvador, tu odio y tu generosidad, tu venganza y
tu perdón, son igualmente despreciables para mí\ldots{} No quiero
hacerte el honor de mirarte. Zugarramurdi, Oricaín, registradle bien, y
si veis que no tiene armas, dejadle salir.

---Sí, eso, eso---dijo Oricaín con pena,---para que nos denuncie a los
ateos, y vengan acá y nos prendan.

---Y nos impidan salir mañana para Navarra---añadió Zugarramurdi.

---Que vaya\ldots{} que lo diga\ldots{} que vengan esos cobardes
bullangueros a detenernos---dijo Navarro.---Ya sabía yo que algunos
polizontes atisbaban estas noches mi casa.

---No hay duda de que es espía---gritó Oricaín.---Me consta.

---No se burlará de nosotros, ¡con cien mil demonios!

Zugarramurdi asió con violencia los dos brazos del joven, que se
estremeció al sacudimiento de aquellas tenazas, sin poder desasirse de
ellas. Oricaín acudió en auxilio del otro sayón; vino también un criado,
le sujetaron, le contuvieron, le amordazaron, le liaron una larga cuerda
en brazos y piernas, y llevándole a una habitación cercana donde había
un pie derecho a manera de poste, resto de un tabique antiguo recién
derribado, le sujetaron a él tan fuertemente, que el desgraciado joven
no podía mover ni un dedo. Palpitante, sofocado, rugiente, como un
volcán obstruido; amenazado de violenta congestión, Salvador veía a sus
enemigos delante de sí, y no se podía defender sino mirándoles\ldots{}
La rabia de sus ojos era su única arma. Se contraían sus músculos: la
prisionera sangre hinchaba sus venas.

---¿Qué pensáis hacer?---preguntó Carlos a sus amigos, cuando concluyó
la operación, sin que él se dignara tomar parte en ella.

---Cuando nos marchemos---repuso Oricaín,---le ahorcaremos.

En aquel instante Jenara pasaba.

---Es demasiado---dijo Navarro.---Le dejaremos así. Basta que no pueda
hacernos daño de aquí a mañana\ldots{} ¿Sabes que esa postura es buena
para conspirar contra el Trono?---añadió, contemplando con hosca
serenidad a la víctima.---¿Por qué no vas ahora de Herodes a Pilatos,
comprometiendo oficiales, repartiendo proclamas, engañando al país,
difundiendo la rebeldía contra Dios y contra el Trono? ¡Miserables
conspiradores! Ve y di a tus revolucionarios que vengan a sacarte de
aquí. Llámales, invoca, la libertad, los derechos del hombre. ¡Que
vengan Riego y Quiroga a desatarte!\ldots{} ¡Oh!, si desde un principio
hubieran puesto a la masonería y al ateísmo como estás ahora, ¿habría
revoluciones? Que me den el mando un solo día, y verás qué gran soga lío
alrededor del gran cuerpo. ¿Por qué no conspiras ahora? ¿Por qué no
sublevas regimientos? Abre la boca y predica libertad y
jacobinismo\ldots{} ¡Ah!, tú creerás que eres un mártir digno de
lástima. ¿No lo has de creer, si en ti y en esta canalla que acaba de
triunfar no hay idea de justicia?\ldots{} ¡Justicia! ¡Castigo del
crimen! ¡Qué sublimes ideas! En medio de la impunidad espantosa que
invade el reino todo como una plaga, aquellas grandes ideas se ven
realizadas en un rincón de Madrid\ldots{} en un rincón de mi
casa\ldots{}

Cuando esto decía, Jenara volvió a pasar.

---¡Bonita imagen de la revolución tenemos delante!---prosiguió Carlos
con amarga ironía.---¡Qué emblema tan hermoso del sistema curativo de
una Nación revolucionaria! En esa postura se olvida el modo de andar y
se pierden los deseos de agitarse mucho; se puede meditar tranquilamente
en Dios y reconocer las ofensas que se le han hecho\ldots{} La voz se
olvida de que ha dicho herejías e infamias. Se aprende a obedecer y a
callar, y el que manda, manda\ldots{} Yo querría que toda España fuera
pasando por esa puerta y viera a su revolucionario\ldots{} el pobrecito
no mueve brazo ni pierna; no habla ni gruñe. Está convertido, y ya no
hace daño ni con su lengua ni con su brazo\ldots{} ¡Qué lección,
Sr.~Monsalud!\ldots{} ¡Si esos locos o imbéciles que chillan por las
calles vieran esto!\ldots{} Si estoy por abrir entrada pública y
exponerte como una cosa rara, anunciando «el gran fenómeno de la
justicia», o sea «la revolución en la soga\ldots» Esto abriría los ojos
a muchos\ldots{} Tal idea debe cundir y propagarse; es admirable. Todos
los que han atentado contra su Rey deberían atravesar ese pasillo y
mirar adentro\ldots{} Se te pondrán luces\ldots{}

Jenara pasó de nuevo.

---Mi opinión---añadió Garrote,---es que no se te quite la vida, a no
ser que resulte que has maltratado a mi abuelo, como sospecho. Si eres
inocente, no te haremos daño. La enemistad privada que tenemos tú y yo,
me obliga a ser generoso. Ni aun consentiría la violencia que sufres si
yo y mis amigos no estuviéramos en peligro de ser denunciados por ti;
pero es preciso asegurarse, señor masón\ldots{} ¡Cuánto me alegraría de
tenerte así el día del triunfo de mis ideas para soltarte y decirte:
«Ahora, los dos a solas, arreglaremos una cuenta antigua!\ldots» Pero yo
estoy caído, y tus amigos son poderosos\ldots{} es preciso tener algún
rigor con los vencedores, mientras se puede; que tiempo tienen ellos
después para abusar de su victoria. Cuando esto pase, cuando yo y mis
amigos no corramos riesgo de ser denunciados a un partido vengativo, nos
veremos, ¿eh?\ldots{} No haya miedo que se te aten entonces las manos.
Al contrario, te las multiplicaría si en mi poder estuviese\ldots{} ¿Me
buscarás tú? ¿Será preciso que yo te busque? ¿Entrarás entonces
furtivamente en mi casa para espiarme? ¿Golpearás en la calle a mi
infeliz abuelo, con el fin de encontrar después, socolor de ampararle,
un pretexto para meterte en el domicilio de un hombre de bien? Esto se
averiguará\ldots{} Me parece que penetro tu intención\ldots{} eres
astuto\ldots{} Sabías que aquí se conspiraba\ldots{} sabías que aquí nos
reunimos en estos días algunos hombres del partido del Rey. Sin duda les
viste entrar. Bien, Sr.~Salvador; todas esas cuentas se arreglarán
después\ldots{} Hasta la vista.

Cuando Carlos salió, Jenara pasaba otra vez.

Cerraron la puerta y Monsalud se quedó solo. Los rumores de la casa
sonaban a lo lejos. En su desesperación sentía transcurrir el tiempo sin
darse cuenta de él, y pasaron minutos que le parecieron horas.
Cualquiera que fuese el delirio de su mente y la exagerada proporción
que daba a todo, ello es que pasó mucho tiempo, y un reloj cercano le
iba marcando los plazos solemnes de su agonía. Imposibilitado de
moverse, luchaba con extraordinaria fuerza del espíritu y del cuerpo;
pero no le era posible vencer. Su sangre era una corriente de fuego:
sentíala en el palpitar de las sienes, semejante al golpe de un hacha.
Al fin perdió el sentido claro de las cosas.

A hora bastante avanzada creyó sentir mayor intensidad en los ruidos de
la casa, el ir y venir y el precipitarse, que indican la gravedad de un
enfermo y la consternación de una familia. Constantemente subía y bajaba
gente por la escalera principal, que cercana de su prisión estaba.
Sintió al fin gran rumor de pasos, como si subiera mucha gente a la vez,
y acompañaba a este rumor el triste son de una campanilla y rezos en
latín. El Viático entraba en la casa. Monsalud distinguió lejano
resplandor de faroles; después de un gran silencio, sólo interrumpido
por algunas voces que en lo más hondo de la casa sonaban, semejantes a
los tristes ecos del coro de un convento. Luego se oyó el estrépito de
los pasos, la misma campanilla, los mismos rezos. Dios salía.

No supo apreciar bien el tiempo que trascurrió después. Su pensamiento
estaba fijo en la idea terrible de que después de entrar Dios en la
casa, continuase la iniquidad que en su persona se cometía\ldots{} La
fiebre empezó a trazar sus vertiginosos y atormentadores círculos dentro
del cerebro del infeliz; pero al fin, trascurrido un plazo de difícil
apreciación, distinguió una claridad que parecía la de la aurora; vio
claramente que la puerta se abría, que alguien entraba sin hacer ruido,
más semejante a una sombra que a una persona, y por último, que unas
manos blandas y frías tocaban su cuerpo.

\hypertarget{xxix}{%
\chapter{XXIX}\label{xxix}}

El Sr.~de Baraona pasó muy mal la noche. El médico dijo que no saldría
de la madrugada. A esta hora la claridad de sus facultades mentales le
permitió hacer sus disposiciones y recibir a Dios, lo cual verificó con
piedad suma y unción evangélica, que fue causa de gran emoción entre los
circunstantes. Su aplanamiento fue después muy grande, y todo hacía
presumir rápido desenlace. Sin embargo, hablaba el enérgico anciano
todavía, y dando explicaciones del triste accidente, aseguró no conocer
a ninguno de los que le maltrataron. No hacía memoria de que un extraño
le había traído a su casa, y con toda firmeza aseguraba haber venido por
su pie. Carlos y Jenara no se apartaban de su lado. Zugarramurdi y
Oricaín, que salieron en compañía del Viático, tardaron bastante en
volver.

Principiaba a lucir el día, cuando Baraona dijo:

---Tengo que hablarte, amado Carlos; tengo que decirte dos palabras.
Sentiría llevármelas conmigo y no poder soltarlas\ldots{} ¡pesan tanto!

Carlos y Jenara se inclinaron hacia él, a un lado y otro del lecho.

---Lo que tengo que decir---indicó el patriarca mirando a Jenara,---tú
no debes oírlo. Querida nieta, sal de aquí por un momento. Carlos y yo
debemos estar solos.

Jenara salió: el moribundo y Carlos quedaron solos.

---Hijo mío---dijo Baraona expresándose con dificultad,---en esta hora
suprema me veo obligado a hacerte una revelación penosa. Mucho me
cuesta, pero la verdad es lo primero\ldots{} Hace tiempo que me has
manifestado dudas y sospechas acerca de la fidelidad de tu esposa, mi
querida nieta.

---Sí---repuso sombríamente Navarro.

Reinó por breve rato un silencio tal, que los dos parecían muertos.

---Sabes que yo la he defendido---añadió Baraona,---aunque al fin la
fuerza de tus argumentos y la evidencia de ciertos síntomas, me han
hecho dudar también, hasta que al fin\ldots{}

Carlos miró al moribundo con terrible ansiedad.

---Hasta que al fin\ldots---repitió el anciano haciendo un
esfuerzo.---No puedo acusar terminantemente a mi adorada nieta; pero sí
te diré que al anochecer del sábado vi a un hombre que se descolgaba al
patio por el balcón del cuarto de tu mujer.

---¡Un hombre!

---Sólo los ladrones y los amantes salen de este modo de las casas. He
estado dudando si te lo revelaría o no\ldots{} creo ya que en conciencia
debo decírtelo\ldots{} ¡Averigua\ldots{} indaga! Quién sabe\ldots{}
quizás sea inocente\ldots{}

---¡Un hombre!---repitió Carlos ahogando un bramido.

---Un hombre vestido con el traje de la gente del pueblo\ldots{} capa de
grana, sombrero redondo\ldots{} calzón negro\ldots{} De su cara nada te
puedo decir. Ya sabes que la puerta del patiecillo estaba siempre
abierta; desde entonces la cerré y guardé la llave. Bajó del balcón,
apoyándose en la reja. Mi primera intención fue gritar y echarle mano;
pero no quise dar escándalo ni comprometer la honra de Jenara hasta no
hacer averiguaciones. Bien podía ser algún enredo de la criada\ldots{}
Carlos, con un pie en el sepulcro, te pido que no condenes a mi pobre
nieta sin oírla. Ten prudencia, calma y tino, y no seas arrebatado ni
ligero. Si Jenara es inocente, pídele en nombre mío perdón de esta
sospecha. Si es culpable\ldots{} ¡que Dios tenga misericordia de
ella!\ldots{} Ahora puedes llamarla. Me parece que ya me apago\ldots{}
¡Dios sea conmigo! Quiero despedirme de todos. ¿Dónde están tus buenos
amigos? Jenara, Carlos, venid todos.

~

Carlos salió de la habitación. Bajo el fruncido ceño, sus negros ojos,
despidiendo rayos, exploraban en la penumbra de la casa con feroz
curiosidad. Pasó por el cuarto oscuro y miró hacia adentro. Monsalud no
estaba allí. En el suelo se veían los pedazos de la cuerda y el cuchillo
con que acababan de ser cortados.

Garrote dio un rugido y saltó afuera.

Deslizose por el corredor hacia el cuarto de su mujer. Entró. El balcón
estaba abierto, y Jenara, asomada en él, se inclinaba hacia fuera,
diciendo: «¡pronto, pronto, que puede venir!»

El rencor de Carlos era mudo porque era inmenso. Abalanzose hacia el
balcón y hacia Jenara, que sintió el bronco resuello de su marido,
semejante a una llamarada de volcán que le quemaba el rostro. Volviose y
su grito de espanto aumentó el furor de Carlos. Este pudo ver claramente
a un hombre en el momento en que se desasía de la reja del piso bajo, y
envolviéndose rápidamente en su capa de grana, echaba a correr hacia la
puerta.

¡Instante más breve que la palabra, acción más breve que el
pensamiento!\ldots{} Jenara y Carlos se miraron. En el semblante de ella
brilló de súbito una serenidad profunda. El hombre que huía se detuvo un
instante en la puerta del patiecillo, porque al entrar en la cerradura
la llave, esta y aquella no obedecían.

---¡Dos vueltas a la llave y tirar hacia adentro!---gritó Jenara con
verdadero acento de inspiración.

La ira del esposo estalló como un trueno.

---¡Traidora!---gritó agarrando a Jenara por un brazo y apartándola del
balcón.

Su mano de hierro, tirando fuertemente del brazo y del cuerpo de la
mujer, hízola dar rápida vuelta en torno suyo. Las flotantes faldas
describieron, arremolinadas, un disco blanco, en cuyo centro el busto
admirable de Jenara, al caer de rodillas, se alzaba con el semblante
vuelto hacia el esposo, los cabellos en desorden, la mirada ardiente. De
su pecho contraído y sofocado por la veloz caída, salió una voz que
dijo:

---¡Salvaje, haz de mí lo que quieras!\ldots{} ¡Ya sabes que te
aborrezco!

Carlos alzó con movimiento brusco a la infeliz mujer, y de nuevo la dejó
caer o la impulsó contra el suelo. Una imprecación horrible sonó en la
sala, y en el mismo instante sonaron también las palabras angustiosas de
una criada, que súbitamente entró diciendo:

---El señor se muere.

Navarro llevó, mejor dicho, arrastró a su esposa hasta la habitación del
enfermo.

Baraona respiraba con dificultad. Sus ojos, medio apagados ya, se
fijaban en un Santo Cristo que frontero de la cama había. Jenara, puesta
de rodillas junto al lecho y apoyada el rostro en él, ocultaba sus
lágrimas. Los dos amigos de Carlos entraron en aquel instante, y con la
cabeza descubierta se acercaron al moribundo. Carlos, lívido y terrible,
estaba en pie, la vista fija en el suelo.

Baraona recobró de repente la energía. Una llamarada, último esfuerzo
del vivir que se despedía, inflamó con fugaz esplendor su naturaleza. De
los hundidos ojos brotó un rayo, y la lengua articuló palabras claras.

---Hijos míos, amigos míos---dijo dirigiéndose a todos.---Adiós; ahí os
queda el mundo. Tal como hoy está, no es gran regalo\ldots{} Muero en
Dios, muero proclamando la justicia y la ley. Sed buenos. Hija mía
querida, ama y obedece a tu esposo\ldots{} Amado hijo mío, respeta y
dirige a tu mujer.

Los sollozos de Jenara le hicieron callar un momento.

---A todos perdono---continuó poniendo la flaca mano sobre la cabeza de
Jenara.---Si alguno hay con mancha de pecado, que mi perdón sea la señal
de su arrepentimiento\ldots{} Y vosotros, valientes amigos, y tú, noble
hijo mío y de aquella tierra de Álava que no ven mis ojos en este triste
momento, recibid mi bendición, recibidla todos. Valientes jóvenes, muero
aborreciendo la revolución, muero abofeteado, escupido, azotado,
inmolado por ella, como Jesús por los judíos. ¿Qué mayor gloria?\ldots{}
¡Gracias, gracias, Dios mío!

Entusiasmo y gozo vibraban en su voz.

---Valientes jóvenes, mirad la imagen del Dios---Hombre, que está frente
a mí; mirad ese cuerpo bendito puesto en la cruz. Juradme ante él que
derramaréis hasta la última gota de vuestra sangre en defensa de los
buenos principios, de la justicia, de la ley de Dios. Jurádmelo, si
queréis que muera contento, y que mi alma angustiada se arroje libre de
toda zozobra y desconsuelo en los inmensos, en los infinitos brazos de
Dios.

Los tres jóvenes miraron la sagrada imagen. Estaban juntos en imponente
grupo. Los tres extendieron el brazo derecho hacia la efigie, alzaron
orgullosamente la cabeza, y con voz entera y solemne dijeron a un
tiempo:

---¡Lo juramos!

Los tres brazos continuaron alzados breve rato, y en el trágico grupo
reinó el silencio de las grandes emociones.

Carlos dijo:

---¡Que mi alma arda en el Infierno eternamente si no lo cumplo!

---¡Muerte a los infames!---bramó Zugarramurdi.

---¡Muerte!---repitió Oricaín.

Los sollozos de Jenara se confundían con los terribles juramentos.

La energía de Baraona se extinguió de improviso. Empezó a apagarse, a
pestañear, a oscilar tenuemente, como brillo del ascua que va a ser
tragada por las lóbregas fauces de la oscuridad.

---Júramelo otra vez---murmuró en voz queda y con los ojos cerrados,
hablando desde el fondo de su agonía.

Los tres repitieron, alzando el brazo:

---¡Lo juramos!

Al bronco sonido del juramento, los enormes cuerpos crecían. Todo tomaba
proporciones enormes. Las manos del Crucifijo parecían tocar a Oriente y
Occidente.

En aquel momento se oyó un rumor lejano, el resuello profundo del
pueblo, que volvía a invadir el recinto de la Inquisición, gritando:
«¡Viva la Libertad!»

Baraona abrió los ojos, y señalando con el dedo al punto por donde
parecía venir el discorde ruido, murmuró:

---La ola de estupidez se acerca.

Después se estremeció, y cruzando las manos, exhaló un hondo suspiro. En
su pecho cavernoso retumbaron estas huecas palabras como un ronquido:

---¡Hasta la última gota de vuestra sangre!

---¡Hasta la última!---repitió Navarro sordamente.

El mugido de Baraona se repitió más lento, más apagado, más lejano.

Parecía una voz que se alejaba de caverna en caverna, y decía:

---¡Acabar con todos ellos!

---¡Con todos ellos!---dijo Oricaín.

---¡Hasta el último!---dijo Navarro.

Baraona, después de ligera convulsión, había abierto desmesuradamente
los párpados, y sus pupilas, semejantes a insensibles globos de vidrio,
continuaban fijas en el Santo Crucifijo con aterradora insistencia. Su
alma navegaba ya por la inmensidad de las olas eternas.

El rumor de la calle se acercaba, y el solemne reposo de la estancia era
turbado por este grito:

---¡Viva el pueblo! ¡Viva la libertad!

Carlos dirigió a la calle una mirada terrible. Mientras Jenara cerraba
los ojos de su abuelo, los tres jóvenes juntaron espontánea e
instintivamente sus manos, y alzando con insolente soberbia la cabeza,
gritaron:

---¡Viva el Rey! ¡Viva la religión!

\flushright{Madrid, Enero de 1876.}

~

\bigskip
\bigskip
\begin{center}
\textsc{Fin de la segunda casaca}
\end{center}

\end{document}
