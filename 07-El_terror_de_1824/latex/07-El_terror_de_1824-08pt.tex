\PassOptionsToPackage{unicode=true}{hyperref} % options for packages loaded elsewhere
\PassOptionsToPackage{hyphens}{url}
%
\documentclass[oneside,8pt,spanish,]{extbook} % cjns1989 - 27112019 - added the oneside option: so that the text jumps left & right when reading on a tablet/ereader
\usepackage{lmodern}
\usepackage{amssymb,amsmath}
\usepackage{ifxetex,ifluatex}
\usepackage{fixltx2e} % provides \textsubscript
\ifnum 0\ifxetex 1\fi\ifluatex 1\fi=0 % if pdftex
  \usepackage[T1]{fontenc}
  \usepackage[utf8]{inputenc}
  \usepackage{textcomp} % provides euro and other symbols
\else % if luatex or xelatex
  \usepackage{unicode-math}
  \defaultfontfeatures{Ligatures=TeX,Scale=MatchLowercase}
%   \setmainfont[]{EBGaramond-Regular}
    \setmainfont[Numbers={OldStyle,Proportional}]{EBGaramond-Regular}      % cjns1989 - 20191129 - old style numbers 
\fi
% use upquote if available, for straight quotes in verbatim environments
\IfFileExists{upquote.sty}{\usepackage{upquote}}{}
% use microtype if available
\IfFileExists{microtype.sty}{%
\usepackage[]{microtype}
\UseMicrotypeSet[protrusion]{basicmath} % disable protrusion for tt fonts
}{}
\usepackage{hyperref}
\hypersetup{
            pdftitle={EL TERROR DE 1824},
            pdfauthor={Benito Pérez Galdós},
            pdfborder={0 0 0},
            breaklinks=true}
\urlstyle{same}  % don't use monospace font for urls
\usepackage[papersize={4.80 in, 6.40  in},left=.5 in,right=.5 in]{geometry}
\setlength{\emergencystretch}{3em}  % prevent overfull lines
\providecommand{\tightlist}{%
  \setlength{\itemsep}{0pt}\setlength{\parskip}{0pt}}
\setcounter{secnumdepth}{0}

% set default figure placement to htbp
\makeatletter
\def\fps@figure{htbp}
\makeatother

\usepackage{ragged2e}
\usepackage{epigraph}
\renewcommand{\textflush}{flushepinormal}

\usepackage{indentfirst}

\usepackage{fancyhdr}
\pagestyle{fancy}
\fancyhf{}
\fancyhead[R]{\thepage}
\renewcommand{\headrulewidth}{0pt}
\usepackage{quoting}
\usepackage{ragged2e}

\newlength\mylen
\settowidth\mylen{...................}

\usepackage{stackengine}
\usepackage{graphicx}
\def\asterism{\par\vspace{1em}{\centering\scalebox{.9}{%
  \stackon[-0.6pt]{\bfseries*~*}{\bfseries*}}\par}\vspace{.8em}\par}

 \usepackage{titlesec}
 \titleformat{\chapter}[display]
  {\normalfont\bfseries\filcenter}{}{0pt}{\Large}
 \titleformat{\section}[display]
  {\normalfont\bfseries\filcenter}{}{0pt}{\Large}
 \titleformat{\subsection}[display]
  {\normalfont\bfseries\filcenter}{}{0pt}{\Large}

\setcounter{secnumdepth}{1}
\ifnum 0\ifxetex 1\fi\ifluatex 1\fi=0 % if pdftex
  \usepackage[shorthands=off,main=spanish]{babel}
\else
  % load polyglossia as late as possible as it *could* call bidi if RTL lang (e.g. Hebrew or Arabic)
%   \usepackage{polyglossia}
%   \setmainlanguage[]{spanish}
%   \usepackage[french]{babel} % cjns1989 - 1.43 version of polyglossia on this system does not allow disabling the autospacing feature
\fi

\title{EL TERROR DE 1824}
\author{Benito Pérez Galdós}
\date{}

\begin{document}
\maketitle

\hypertarget{i}{%
\chapter{I}\label{i}}

En la tarde del 2 de octubre de 1823 un anciano bajaba con paso tan
precipitado como inseguro por las afueras de la puerta de Toledo en
dirección al puente del mismo nombre. Llovía menudamente, pero sin
cesar, según la usanza del hermoso cielo de Madrid cuando se enturbia, y
la ronda podía competir en lodos con su vecino Manzanares, el cual
hinchándose como la madera cuando se moja, extendía su saliva fangosa
por gran parte del cauce que le permiten los inviernos. El anciano
transeúnte marchaba con pie resuelto, sin que le causara estorbo la
lluvia, con el pantalón recogido hacia la pantorrilla y chapoteando sin
embarazo en el lodo con las desfiguradas botas. Iba estrechamente
forrado, como tizona en vaina, en añoso gabán oscuro, cuyo borde y
solapa se sujetaban con alfileres allí donde no había botones, y con los
agarrotados dedos en la parte del pecho, como la más necesitada de
defensa contra la humedad y el frío. Hundía la barba y media cara en el
alzacuello, tieso como una pared, cubriéndose con él las orejas y el ala
posterior del sombrero, que destilaba agua como cabeza de tritón en
fuente de Reales Sitios. No llevaba paraguas ni bastón. Mirando sin
cesar al suelo, daba unos suspiros que competían con las ráfagas de aire
revuelto. ¡Infelicísimo varón! ¡Cuán claramente pregonaban su desdichada
suerte el roto vestido, las horadadas botas, el casquete húmedo, la
aterida cabeza y aquel continuo suspirar casi al compás de los pasos!
Parecía un desesperado que iba derecho a descargar sobre el río el fardo
de una vida harto pesada para llevarla más tiempo. Y sin embargo, pasó
por el puente sin mirar al agua y no se detuvo hasta el parador situado
en la divisoria de los caminos de Toledo y Andalucía.

Bajo el cobertizo destinado a los alcabaleros y gente del fisco, había
hasta dos docenas de hombres de tropa, entre ellos algunos oficiales de
línea y voluntarios realistas de nuevo cuño en tales días. Los paradores
cercanos albergaban una fuerza considerable cuya misión era guardar
aquella principalísima entrada de la Corte, ignorante aún de los sucesos
que en el último confín de la Península habían cambiado el Gobierno de
constituciona dudoso en absoluto verídico y puro, poniendo fin entre
bombas certeras y falaces manifiestos, a los \emph{tres llamados años}.
En aquel cuerpo de guardia se examinaban los pasaportes, vigilando con
exquisito esmero las entradas y salidas, mayormente estas últimas, a fin
de que no escurriesen el bulto los sospechosos ni se pusieran en cobro
los revolucionarios, cuya última cuenta se ajustaría en el tremendo
Josafat del despotismo.

El vejete se acercó al grupo de oficiales y reconociendo prontamente al
que sin duda buscaba, que era joven, adusto y morenote, bastante
adelantado en su marcial carrera como proclamaban las insignias, díjole
con mucho respeto:

---Aquí estoy otra vez, señor coronel Garrote. ¿Tiene Vuecencia alguna
buena noticia para mí?

---Ni buena ni mala, señor\ldots{} ¿cómo se llama usted?---repuso el
militar.

---Patricio Sarmiento, para servir a Vuecencia y a la compañía; Patricio
Sarmiento, el mismo que viste y calza, si esto se puede decir de mi
traje y de mis botas. Patricio Sarmiento, el\ldots{}

---Pase usted adentro---díjole bruscamente el militar, tomándole por un
brazo y llevándole bajo el cobertizo.---Está usted como una sopa.

Un rumor, del cual podía dudarse si era de burla o de lástima, y quizás
provenía de las dos cosas juntamente, acogió la entrada del infeliz
preceptor en la compañía de los militares.

---Sí, señor Garrote---añadió Sarmiento;---soy, como decía, el hombre
más desgraciado de todo el globo terráqueo. Ese cielo que nos moja no
llora más que lloro en estos días, desde que me han anunciado como
probable, como casi cierta la muerte de mi querido hijo Lucas, de mi
niño adorado, de aquel que era manso cordero en el hogar paterno y león
indómito en los combates\ldots{} ¡ah! señores. ¡Ustedes no saben lo que
es tener un hijo único y perderlo en una escaramuza de Andalucía, por
descuidos de un general, o por intrepidez imprudente de un
oficialete!\ldots{} ¿Pero hay esperanzas todavía de que tan horrible
noticia no sea cierta? ¿Se ha sabido algo? Por Dios, señor Garrote, ¿ha
sabido Vuecencia si mi idolatrado unigénito vive aún o si feneció en
esas tremendas batallas?\ldots{} ¿Hay algún parte que lo
mencione?\ldots{} porque Lucas no podía morir como cualquiera, no: había
de morir ruidosa y gloriosísimamente, de una manera tal, que dé gusto y
juego a los historiadores\ldots{} ¿Ha sabido algo Vuecencia de ayer acá?

---Nada---repuso Garrote fríamente.

---Ha seis días que vengo todas las tardes y siempre me dice Vuecencia
lo mismo---murmuró Sarmiento con angustia.---¡Nada!

---Desde el primer día manifesté a usted que nada podía saber.

---Pero a todas horas entran heridos, soldados dispersos, paisanos,
correos que vienen de las Andalucías. ¿Se ha olvidado usted de
preguntar?

---No me he olvidado---indicó el coronel con semblante y tono más
compasivos,---pero nadie, absolutamente nadie tiene noticia del
miliciano Lucas Sarmiento.

---¡Todo sea por Dios!---exclamó el preceptor mirando al cielo.---¡Qué
agonía! Unos me dicen que sucumbió, otros que está herido
gravemente\ldots{} ¿Han entrado hoy muchos milicianos prisioneros?

---Algunos.

---¿No venía Pujitos?

---¿Y quién es Pujitos?

---¡Oh! Vuecencia no conoce a nuestra gente.

---Soy forastero en Madrid.

---¡Oh! Pasaron aquellos tiempos de gloria---exclamó D. Patricio con
lágrimas en los ojos y declamando con cierto énfasis que no cuadraba mal
a su hueca voz y alta figura.---¡Todo ha caído, todo es desolación,
muerte y ruinas! Aquellos adalides de la libertad, que arrancaron a la
madre España de las garras del despotismo, aquellos fieros leones
matritenses, que con sólo un resoplido de su augusta cólera desbarataron
a la Guardia Real ¿qué se hicieron? ¿Qué se hizo de la elocuencia que
relampagueaba tronando en los cafés, con luz y estruendo sorprendentes?
¿Qué se hizo de aquellas ideas de emancipación que inundaban de gozo
nuestros corazones? Todo cayó, todo se desvaneció en tinieblas, como
lumbre extinguida por la inundación. La oleada de fango frailesco ha
venido arrasándolo todo. ¿Quién la detendrá volviéndola a su inmundo
cauce? ¡Estamos perdidos! La patria muere ahogada en lodazal repugnante
y fétido. Los que vimos sus días gloriosos, cuando al son de patrióticos
himnos eran consagradas públicamente las ideas de libertad y nos
hacíamos todos libres, todos igualmente soberanos, lo recordamos como un
sueño placentero que no volverá. Despertamos en la abnegación, y el peso
y el rechinar de nuestras cadenas nos indican que vivimos aún. Las
iracundas patas del déspota nos pisotean, y los frailes nos\ldots{}

---Basta---gritó una formidable voz interrumpiendo bruscamente al
infeliz dómine.---Para sainete basta ya, señor Sarmiento. Si abusa usted
de la benignidad con que se le toleran sus peroratas en atención al
estado de su cabeza, nos veremos obligados a retirarle las licencias.
Esto no se pued resistir. Si los desocupados de Madrid le consienten a
usted que vaya de esquina en esquina y de grupo en grupo, divirtiéndoles
con sus necedades y reuniendo tras de sí a los chicos, yo no permito que
con pretexto de locura o idiotismo se insulte al orden político que
felizmente nos rige\ldots{}

---¡Ah! señor Garrote, señor Garrote---dijo Sarmiento moviendo
tristemente la cabeza y sacudiendo menudas gotas de agua sobre los
circunstantes.---Vuecencia me tapa la boca que es el único desahogo de
mi alma abrasada\ldots{} Callaré: pero deme Vuecencia nuevas de mi hijo,
aunque sean nuevas de su muerte.

Garrote encogió los hombros y ofreció una silla al pobre hombre, que
despreciando el asiento, juzgó más eficaz contra la humedad y el fresco
pasearse de un rincón a otro del cobertizo, dando fuertes patadas y
girando rápidamente, como veleta, al dar las vueltas. Los demás
militares y paisanos armados no ocultaban su regocijo ante la grotesca
figura y ditirámbico estilo del anciano, y cada cual imaginaba un tema
de burla con que zaherirle, mortificándole también en su persona. Este
le decía que Su Majestad pensaba nombrarle ministro de Estado y llavero
del Reino, aquel que un ejército de carbonarios venía por la frontera
derecho a restablecer la Constitución, uno le ponía una banqueta delante
para que al pasar tropezase y cayese, otro le disparaba con cerbatana un
garbanzo haciendo blanco en el cogote o la nariz. Pero Sarmiento, atento
a cosas más graves que aquel juego importuno, hijo de un sentimiento
grosero y vil, no hacía caso de nada, y sólo contestaba con monosílabos
o llevándose la mano a la parte dolorida.

Había pasado más de un cuarto de hora en este indigno ejercicio, cuando
de la venta salió un hombre pequeño, doblado, de maciza arquitectura,
semejante a la de esos edificios bajos y sólidos que no tienen por
objeto la gallarda expresión de un ideal, sino simplemente servir para
cualquier objeto terrestre y positivo. Siendo posible la comparación de
las personas con las obras de arquitectura, y habiendo quien se asemeja
a una torre gótica, a un palacio señorial, a un minarete árabe, puede
decirse de aquel hombre que parecía una cárcel. Con su musculatura de
cal y canto se avenía maravillosamente una como falta de luces, rasgo
misterioso e inexplicable de su semblante, que a pesar de tener cuanto
corresponde al humano frontispicio, parecía una fachada sin ventanas. Y
no eran pequeños sus ojos ciertamente, ni dejaban de ver con claridad
cuanto enfrente tenían; pero ello es que mirándole no se podía menos de
decir: «¡qué casa tan oscura!»

Su fisonomía no expresaba cosa alguna, como no fuera una calma torva,
una especie de acecho pacienzudo. Y a pesar de esto no era feo, ni su
correctas facciones habrían formado mal conjunto si estuvieran de otra
manera combinadas. Tales o cuales cejas, boca o narices más o menos
distantes de la perfección, pueden ser de agradable visualidad o de
horrible aspecto, según cual sea la misteriosa conexión que forma con
ellas una cara. La de aquel hombre que allí se apareció era ferozmente
antipática. Siempre que vemos por primera vez a una persona, tratamos,
sin darnos cuenta de nuestra investigación, de escudriñar su espíritu y
conocer por el mirar, por la actitud, por la palabra lo que piensa y
desea. Rara vez dejamos de enriquecer nuestro archivo psicológico con
una averiguación preciosa. Pero enfrente de aquel sótano humano el
observador se aturdía diciendo: «Está tan lóbrego que no veo nada».

Vestía de paisano con cierto esmero, y todas cuantas armas portátiles se
conocen llevábalas él sobre sí, lo cual indicaba que era voluntario
realista. Fusil sostenido a la espalda con tirante, sable, machete,
bayoneta, pistolas en el cinto hacían de él una armería en toda regla.
Calzaba botas marciales con espuelas a pesar de no ser de a caballo; mas
este accesorio solían adoptarlo cariñosamente todos los militares
improvisados de uno y otro bando. Chupaba un cigarrillo y a ratos se
pasaba la mano por la cara, afeitada como la de un fraile; pero su
habitual resabio nervioso (estos resabios son muy comunes en el
organismo humano) consistía en estar casi siempre moviendo las
mandíbulas como si rumiara o mascullase alguna cosa. Su nombre de pila
era Francisco Romo.

D. Patricio, luego que le vio, llegose a él y le dijo:

---¡Ah! Sr.~Romo, ¡cuánto me alegro de verle! Aquí estoy por sexta vez
buscando noticias de mi hijo.

---¿Qué sabemos nosotros de tu hijo, ni del hijo del Zancarrón? Papá
Sarmiento, tú estás en Babia\ldots{} No tardarás mucho en ir al Nuncio
de Toledo\ldots{} Ven acá, estafermo---al decir esto le tomaba por un
brazo y le llevaba al interior de la venta que servía de cuerpo de
guardia ,---ven acá y sirve de algo.

---¿En qué puedo servir al Sr.~Romo? Diga lo que quiera con tal que no
me pida nada de que resulte un bien al absolutismo.

---Es cosa mía---dijo Romo hablando en voz baja y retirándose con
Sarmiento a un rincón donde no pudieran ser oídos.---Tú, aunque loco,
eres hombre capaz de llevar un recado y ser discreto.

---Un recado\ldots{} ¿a quién?

---A Elenita, la hija de D. Benigno Cordero, que vive en tu misma casa,
¿eh? Me parece que no te vendrán mal tres o cuatro reales\ldots{} Este
saco de hueso está pidiendo carne. ¿Cuántas horas hace que no has
comido?

---Ya he perdido la cuenta---repuso el preceptor con afligidísimo
semblante, mientras un lagrimón como garbanzo corría por su mejilla.

---Pues bien, carcamal: aquí tienes una peseta. Es para ti si llevas a
la señorita doña Elena\ldots{}

---¿Qué?

---Esta carta---dijo Romo mostrando una esquela doblada en pico.

---¡Una carta amorosa!---exclamó Sarmiento ruborizándose.---Sr.~Romo de
mis pecados. ¿por quién me toma usted?

El tono de dignidad ofendida con que hablara Sarmiento, irritó de tal
modo al voluntario realista, que empujando brutalmente al anciano le
vituperó de este modo:

---¡Dromedario! ¿qué tienes que decir?\ldots{} Sí, una carta amorosa. ¿Y
qué?

---Que es usted un simple si me toma por alcahuete---dijo D. Patricio
con severo acento.---Guarde usted su peseta y yo me guardaré mi gana de
comer. ¡Por vida de la chilindraina! No faltan almas caritativas que
hagan limosna sin humillarnos\ldots{}

Inflamado en vivísima cólera el voluntario y sin hallar otras razones
para expresarla que un furibundo terno, descargó sobre el pobre maestro
aburrido uno de esos pescozones de catapulta que abaten de un golpe las
más poderosas naturalezas, y dejándole tendido en tierra, magullados y
acardenalados el hocico y la frente, salió del cuerpo de guardia.

A D. Patricio le levantaron casi exánime, y su destartalado cuerpo se
fue estirando poco a poco en la postura vertical, restallándole las
coyunturas como clavijas mohosas. Se pasó la mano por la cara, y dando
un gran suspiro y elevando al cielo los ojos llorosos, exclamó así con
dolorido acento:

---¡Indigno abuso de la fuerza bruta, y de la impunidad que protege a
estos capigorrones!\ldots{} Si otros fueran los tiempos, otras serían
las nueces\ldots{} Pero los yunques se han vuelto martillos y los
martillos de ayer son yunques ahora. ¡Rechilindrona! ¡Malditos sean los
instantes que he vivido después que murió aquella preciosa
libertad!\ldots{}

Y sucediendo la rabia al dolor, se aporreó la cabeza y se mordió los
puños. Habíanle abandonado los que antes le prestaran socorro, porque
fuera se sentía gran ruido y salieron todos corriendo al camino. D.
Patricio, coronándose dignamente con su sombrero, al cual se empeñó en
devolver su primitiva forma, salió también arrastrado por la curiosidad.

\hypertarget{ii}{%
\chapter{II}\label{ii}}

Era que venían por el camino de Andalucía varias carretas precedidas y
seguidas de gente de armas a pie y a caballo, y aunque no se veían sino
confusos bultos a lo lejos, oíase un son a manera de quejido, el cual si
al principio pareció lamentaciones de seres humanos, luego se comprendió
provenía del eje de un carro, que chillaba por falta de unto. Aquel
áspero lamento unido a la algazara que hizo de súbito la mucha gente
salida de los paradores y ventas, formaba lúgubre concierto, más lúgubre
a causa de la tristeza de la noche. Cuando los carros estuvieron cerca,
una voz acatarrada y becerril gritó: \emph{¡Vivan las caenas!} ¡viva el
Rey absoluto y muera la Nación! Respondiole un bramido infernal como si
a una rompieran a gritar todas las cóleras del averno, y al mismo tiempo
la luz de las hachas prontamente encendidas permitió ver las terribles
figuras que formaban procesión tan espantosa. D. Patricio, quizás el
único espectador enemigo de semejante espectáculo, sintió los
escalofríos del terror y una angustia mortal que le retuvo sin
movimiento y casi sin respiración por algún tiempo.

Los que custodiaban el convoy y los paisanos que le seguían por
entusiasmo absolutista estaban manchados de fango hasta los ojos.
Algunos traían pañizuelo en la cabeza, otros sombrero ancho, y muchos,
con el desgreñado cabello al aire, roncos, mojados de pies a cabeza,
frenéticos, tocados de una borrachera singular que no se sabe si era de
vino o de venganza, brincaban sobre los baches, agitando un jirón con
letras, una bota escuálida o un guitarrillo sin cuerdas. Era una
horrenda mezcla de bacanal, entierro y marcha de triunfo. Oíanse
bandurrias desacordes, carcajada, panderetazos, votos, ternos,
kirieleisones, vivas y mueras, todo mezclado con el lenguaje carreteril,
con patadas de animales (no todos cuadrúpedos) y con el cascabeleo de
las colleras. Cuando la caravana se detuvo ante el cuerpo de guardia, y
entonces aumentó el ruido. La tropa formó al punto, y una nueva
aclamación al Rey neto alborotó los caseríos. Salieron mujeres a las
ventanas, candil en mano, y la multitud se precipitó sobre los carros.

Eran estos galeras comunes con cobertizo de cañas y cama hecha de
pellejos y sacos vacíos. En el delantero venían tres hombres, dos de
ellos armados, sanos y alegres, el tercero enfermo y herido, reclinado
doloridamente sobre el camastrón, con grillos en los pies y una larga
cadena que, prendida en la cintura y en una de las muñecas, se enroscaba
junto al cuerpo como una culebra. Tenía vendada la cabeza con un lienzo
teñido d sangre, y era su rostro amarillo como vela de entierro. Le
temblaban las carnes, a pesar de disfrutar del abrigo de una manta, y
sus ojos extraviados así como su anhelante respiración anunciaban un
estado febril y congojoso. Cuando el coronel Garrote se acercó al carro
y alzando la linterna que en la mano traía, miró con vivísima curiosidad
al preso, este dijo a media voz:

---¿Estamos ya en Madrid?

Sin hacer caso de la pregunta, Garrote, cuyo semblante expresaba el goce
de una gran curiosidad satisfecha, dijo:

---¿Con que es usted\ldots?

Uno de los hombres armados que custodiaban al preso en el carro, añadió:

---El héroe de las Cabezas.

Y junto al carro sonó este grito de horrible mofa:

---¡Viva Riego!

Garrote se empeñó en apartar a la gente que rodeaba el carro, apiñándose
para ver mejor al preso e insultarle más de cerca.

Un hombre alargó el brazo negro y tocando con su puño cerrado el cuello
del enfermo, gritó:

---¡Ladrón, ahora la pagarás!

El desgraciado general se recostó en su lecho de sacos, y callaba,
aunque harto claramente imploraban compasión sus ojos.

---Fuera de aquí, señores, a un lado---dijo Garrote, aclarando con
suavidad el grupo de curiosos.---Ya tendrán tiempo de verle a sus
anchas\ldots{}

---Dicen que la horca será la más alta que se ha visto en
Madrid---indicó uno.

---Y que se venderán los asientos en la plaza, como en la de
toros---dijo otro.

---Pero déjennoslo ver\ldots{} por amor de Dios. Si no nos lo comemos,
señor coronel---gruñó una dama del parador cercano.

---Si no puede con su alma\ldots{} ¿Y ese hombre ha revuelto medio
mundo? Que me lo vengan a decir\ldots{}

---¡Qué facha! ¿Y dicen que este es Riego?\ldots{} ¡qué bobería!\ldots{}
Si parece un sacristán que se ha caído de la torre cuando estaba tocando
a muerto\ldots{}

---Este es tan Riego como yo.

---Os digo que es el mismo. Le vi yo en el teatro, cantando el himno.

---El mismo es. Tiene el mismo parecido del retrato que paseaban po
Platerías.

Hasta aquí las mortificaciones fueron de palabra. Pero un grupo de
hombres que habían salido al encuentro de los carros, una gavilla mitad
armada, mitad desnuda, desarrapada, borracha, tan llena de rabia y cieno
que parecía creación espantosa del lodo de los caminos, de la hez de las
tinajas y de la nauseabunda atmósfera de los presidios, un pedazo de
populacho, de esos que desgarrándose se separan del cuerpo de la Nación
soberana para correr solo manchando y envileciendo cuanto toca, empezó a
gritar con el gruñido de la cobardía que se finge valiente fiando en la
impunidad:

---¡Que nos lo den; que nos entreguen a ese pillo, y nosotros le
ajustaremos la cuenta!

---Señores---dijo Garrote con energía,---atrás; atrás todo el mundo. El
preso va a entrar en Madrid.

---Nosotros le llevaremos.

---Atrás todo el mundo.

Y los pocos soldados que allí había, auxiliados con tibieza por los
voluntarios realistas, empezaron a separar la gente.

Unos corrieron a curiosear en los carros que venían detrás y otros se
metieron en la venta, donde sonaban seguidillas, castañuelas y
desaforados gritos y chillidos. Un cuero de vino, roto por los golpes y
patadas que recibiera, dejaba salir el rojo líquido, y el suelo de la
venta parecía inundado de sangre. Algunos carreteros sedientos se habían
arrojado al suelo y bebían en el arroyo tinto; los que llegaron más
tarde apuraban lo que había en los huecos del empedrado, y los chicos
lamían las piedras fuera de la venta, a riesgo de ser atropellados por
las mulas desenganchadas que iban de la calle a la cuadra, o del tiro al
abrevadero. Poco después veíanse hombres que parecían degollados con
vida, carniceros o verdugos que se hubieran bañado en la sangre de sus
víctimas. El vino mezclado al barro y tiñendo las ropas que ya no tenían
color, acababa de dar al cuadro en cada una de sus figuras un tono crudo
de matadero, horriblemente repulsivo a la vista.

Y a la luz de las hachas de viento y de las linternas, las caras
aumentaban en ferocidad, dibujándose más claramente en ellas la risa
entre carnavalesca y fúnebre que formaba el sentido, digámoslo así, de
tan extraño cuadro. Como no había cesado de llover, el piso inundado era
como un turbio espejo de lodo y basura, en cuyo cristal se reflejaban
los hombres rojos, las rojas teas, los rostros ensangrentados, las
bayonetas bruñidas, las ruedas cubiertas de tierra, los carros, las
flacas mulas, las haraposas mujeres, el movimiento, el ir y venir, la
oscilación de las linternas y hasta el barullo, los relinchos de brutos
y hombres, la embriaguez inmunda, y por último, aquella atmósfera
encendida, espesa, suciamente brumosa, formada por los alientos de la
venganza, de la rusticidad y de la miseria.

En el segundo carro estaban presos también y heridos los compañeros de
Riego, a saber: el capitán D. Mariano Bayo, el teniente coronel
piamontés Virginio Vicenti y el inglés Jorge Matías. D. Patricio
Sarmiento, que no se atrevió a acercarse al primer carro, se detuvo
breve rato junto al segundo, pasó indiferente por el tercero, donde sólo
venían sacos y un guerrillero con su mujer, y se dirigió al cuarto,
llamado por una voz débil que claramente dijo:

---Sr.~D. Patricio de mi alma\ldots{} ¡Bendito sea Dios que me permite
verle!

---¡Pujitos!\ldots{} ¡Pujitos mío!\ldots---exclamó Sarmiento extendiendo
sus brazos dentro del carro.---¿Eres tú?\ldots{} Sí, tú mismo\ldots{}
Dime, ¿estás también herido? Por lo visto, también vienes preso.

---Sí señor---repuso el maestro de obra prima,---herido y preso
estoy\ldots{} Diga usted ¿nos ahorcarán?

---¿Pues eso quién lo duda?

---¡Infeliz de mí!\ldots{} Vea usted los lodos en que han venido a parar
aquellos polvos. Bien me lo decía mi mujer\ldots{} Sr.~D. Patricio, al
que está como yo medio muerto de un bayonetazo en la barriga, le
deberían dejarle en manos de Dios para que se lo llevase cuando a su
Divina Majestad le diese la gana ¿no es verdad?

---Sí, Pujitos mío---repuso Sarmiento estrechándole la mano.---¿Sabes
que tiemblo y tengo frío? más frío y más miedo que tú, porque voy a
preguntarte por mi hijo en cuya compañía has vivido por esas tierras, y
según lo que me contestes, así moriré o viviré\ldots{} Hace seis días
que estoy en la incertidumbre más horrible; hace seis días que bajo a
este camino para interrogar a todos los que llegan\ldots{} ¡Ah! por fin
encuentro quien me diga la verdad. Pujitos de mi alma, tú me la dirás,
aunque sea terrible.

---Sí señor, sí señor, yo se la diré---repuso Pujitos, cubriéndose con
ambas manos el rostro y rompiendo a llorar como un chicuelo.

---¡Conque es cierto, amigo, conque es verdad que mi pobre
Lucas!\ldots---gimió el preceptor con la voz entrecortada por el
llanto.---¡Pobre hijo de mi alma!

---¡Pobre amigo mío!---añadió Pujitos, secando sus lágrimas.---¡Y era
tan cariñoso, tan bueno, tan leal!\ldots{} Sin cesar estaba nombrándole
a usted y cavilando sobre lo que haría usted en Madrid o lo que no
haría\ldots{} «Si tendrá discípulos, decía; si pasará trabajos. Ahora
estará barriendo la escuela»\ldots{} No nos separábamos nunca, partíamos
nuestra ración y éramos en todo como hermanos. En las batallas siempre
nos escondíamos juntos.

---¡Os escondíais!---exclamó D. Patricio levantando el rostro con
dignidad, pues esta era tan grande en él, que ni el dolor podía
vencerla.

---¡Ah! señor\ldots{} el pobre Lucas era el mejor chico del
mundo\ldots{} ¡Pobrecito!\ldots{}

---Ha tiempo que el dardo estaba clavado en mi corazón\ldots{} Yo le
tenía por muerto; pero la falta de noticias ciertas me daba alguna
esperanza. Me agarraba con desesperación a las conjeturas. Pero tú has
disipado mis dudas. Más vale la desgracia verdadera y declarada que una
vacilación desgarradora.

---Aquí está todo lo que resta del pobre Lucas---dijo el herido
mostrando un pequeño lío de ropa.

D. Patricio se abalanzó a aquel objeto mudo, testimonio tristísimo de su
última esperanza muerta y lo besó con ardiente cariño. Breve rato le vio
Pujitos con la cabeza apoyada en el borde del carro, oprimiendo con ella
el lío de ropa y regándolo con sus lágrimas. Respetuoso con el dolor del
padre, el maestro de obra prima no decía nada.

---Esto es hecho---exclamó al fin D. Patricio irguiendo la frente
caduca, mas bastante fuerte para soportar, mediante la energía de su
espíritu, el peso de una gran pena.---El Autor de todas las cosas lo
quiere así. Ya no tengo hijo\ldots{} Toda esperanza acabó y con ella la
vida mía\ldots{} Ahora leal amigo, ahora excelente joven que has sido el
Pílades de aquel noble Orestes, cuéntame sin omitir nada los pormenores
de la muerte de mi hijo; dime cómo se extinguió aquella vida preciosa,
porque siendo Lucas de ánimo tan esforzado e intrépido, no podía morir
como los demás milicianos, sino de una manera grande\ldots{} ¿me
entiendes? de una manera gloriosa, y en un momento de sublime heroísmo.

---Precisamente heroísmo no, Sr.~D. Patricio---dijo Pujitos con
embarazo.---Yo le contaré a usted\ldots{} Lucas\ldots{}

---Heroísmo ha habido: no me lo niegues, porque yo conozco muy bien la
raza de leones de que viene mi hijo, yo sé qué casta de bromas gastamos
los Sarmientos con el enemigo en un campo de batalla. Si por modestia
callas las acciones homéricas en que tú has tomado parte, haces mal, que
al fin y al cabo todo se ha de saber, y si no ahí están los
historiadores que en un abrir cerrar de ojos desentrañarán lo más
escondido.

---Si no ha habido acciones heroicas ni cosa que lo valga, hombre de
Dios---objetó Pujitos con pena.---Nosotros estábamos en Málaga con el
general Zayas, cuando este representó a las Cortes al tenor de lo que
dijo Ballesteros a capitular; ¿usted me entiende? Vino entonces Riego
mandado por las Cortes, tomó el mando y nos llevó contra Ballesteros;
¿usted me entiende?

---Y entonces se trabaron esas crueles batallas que yo imagino.

---No hubo más sino que el general llevaba el encargo de
inflamarnos\ldots{} Sí señor, de inflamarnos, porque todos estábamos muy
abatidos y sin ganas de guerra, porque la veíamos muy negra.

---¿Y os inflamó?

¿Cómo se puede inflamar la nieve? Fuimos en busca de Ballesteros y le
hallamos en Priego. Allí se armó una\ldots{}

---¡Corrieron mares de sangre!

---No señor. Todo era \emph{¡Viva Ballesteros!} por un lado, y por otro
\emph{¡Viva Riego!} Nos abrazamos y los generales conferenciaron. Como
no se pudieron avenir, Riego arrestó a Ballesteros.

---Bien hecho, muy bien\ldots{} ¿Y Lucas?

---Lucas tan bueno y tan sano\ldots{} Era aquella la mejor vida del
mundo, porque como no había balas sino conferencias\ldots{} Pero un día
se presentó delante de nosotros Balanzat y tiros van tiros
vienen\ldots{} Desde entonces perdió la salud el pobre Lucas, porque le
entró como un súpito y se quedó frío y yerto, temblando y quejándose de
que le dolía esto y lo otro.

---¡Desgraciado hijo mío! Su principal pena consistiría en no poder
batirse en primera fila.

---Puede que así fuera. Lo cierto es que empezó a decaer, a decaer, y la
calentura seguía en aumento, y deliraba con los tiros. Riego abandonó el
campo; nos fuimos con él y el pobre Lucas parecía que recobraba la vida
según nos íbamos alejando de las tropas de Balanzat. El general fue
perdiendo su gente porque oficiales y soldados desertaban a cada hora.
¡Qué tristeza, Sr.~D. Patricio! Pero el pobre Lucas se alegraba y decía:
«Amigo Pujos, esto parece que acabará pronto». Había mejorado bastante,
y estaba limpio de calentura\ldots{} Pero de repente cuando íbamos cerca
de Jaén, aparecen los franceses\ldots{}

---¡Oh! ¡Me tiemblan las carnes al oírte! ¡Cómo correría la sangre en
ese glorioso cuanto infausto día!

---Más corrieron los pies, Sr.~Sarmiento. Yo, la verdad sea dicha, no
fuí de los que más corrieron, porque no podía abandonar al pobre Lucas,
que se descompuso todo, y se quedó en un hilo. Arrojamos los fusiles que
nos pesaban mucho y nos refugiamos en una casa de labor. ¡Ay, pobre
amigo mío! Le entró tal calenturón que su cuerpo parecía un volcán,
perdió el conocimiento, y a las treinta horas\ldots{}

---No sigas que se me parte el corazón---dijo D. Patricio con voz
entrecortada por los sollozos.---¡Cuánto padecería al ver que su mísero
estado corporal no le permitía batirse! ¡Qué lucha tan horrenda la de
aquella alma de león, al sentirse sin cuerpo que la ayudara!

---El pobrecito en su delirio nombraba a los franceses y se metía debajo
del jergón. Serían las doce y media de la noche cuando entregó su alma
al Señor\ldots{}

---¡Ay, parece que me arrancan las entrañas! Calla ya.

---Yo caí prisionero, fuí herido de un bayonetazo, y después de tenerme
algunos días en un calabozo de la Carolina me metieron en este carro.
Por el camino se nos unió el general preso y herido también, y juntos
hemos llegado aquí. Dicen que nos van a ahorcar a todos.

---Eso es indudable---contestó Sarmiento en tono que más era de
satisfacción y orgullo que de lástima.---¡Fin lamentable, pero glorioso!
¿Qué mayor honra que morir por la libertad y ser mártires de tan sublime
idea?

Pujitos, que sin duda no había dado hospedaje en su pecho a tan elevados
sentimientos, suspiró acongojadamente.

---Bendice tu muerte, hijo mío---añadió Sarmiento, extendiendo hacia él
sus venerables manos, en la actitud de un sacerdote antiguo,---bendice
tus nobles heridas, pregoneras de tu indomable valor en los combates.
Has sido atravesado de un bayonetazo, y además tienes heridos la cabeza
y el brazo.

---Esto que tengo en el arca del estómago es fechoría de un francés a
quien vea yo comido de perros. Lo de la cabeza es una pedrada, y lo del
brazo un mordisco. En los pueblos por donde hemos pasado nos han
recibido lindamente, señor. Como los curas salían diciendo que estábamos
todos condenados y que ya nos tenían hecha la cama de rescoldo en el
infierno, no había para nosotros más que palos, amenazas y pedradas. En
Santa Cruz de Mudela nos dieron una rociada buena. El general y yo
salimos descalabrados, y gracias a que los carros echaron a andar; que
si no, allí nos quedamos como San Esteban. En Tembleque nos quisieron
matar, y si la tropa no nos defiende a culatazos, allí perecemos todos.
Hombres y mujeres salían al camino aullando como lobos. Uno que debía de
ser pariente de caníbales, después de molerme a coces y puñadas me clavó
los dientes en este brazo y me partió las carnes\ldots{} ¿Qué ganará el
Rey absoluto con esto? Mala peste le dé Dios\ldots{} Pero dicen que todo
esto es por obra y gracia de los condenados frailes\ldots{} ¿Es verdad,
Sr.~D. Patricio?

---Hijo mío, mucho me temo que esos bribones se venguen ahora de lo que
les hicimos con razón. Y no serán como nosotros, generosos y templados
en el condenar, sino fieros, vengativos y sanguinarios cual líbicas
hienas\ldots{} Hemos de ver lo que nadie ha visto, ¡por vida de la
ch\ldots!

No pudo seguir su frase el buen preceptor, porque un voluntario realista
se acercó al carro y brutalmente gritó:

---Atrás, D. Camello, o le parto\ldots{} ¡fuera de aquí, estantigua!

Sarmiento corrió dando zancajos hacia el parador. Con su gran levitón,
cuyos faldones se agitaban en la carrera, parecía una colosal ave flaca
que volaba rastreando el suelo. Después de recoger del fango su sombrero
que había perdido en la huida, confundiose entre la multitud para estar
más seguro. Entonces oyó al coronel Garrote dar esta orden al capitán
Romo.

---Siga adelante el convoy. Custódielo usted con su media compañía.
Tengo orden de que no entre en las calles de Madrid. Pase el río; tome
la ronda a la izquierda hacia la Virgen del Puerto; adelante siempre, y
subiendo por la cuesta de Areneros, diríjase al Seminario de Nobles,
donde esperan a los presos. En marcha, pues. Guárdense los curiosos de
seguir al convoy porque haré fuego sobre ellos. Marche cada cual a su
casa y buenas noches.

El convoy se puso en movimiento, carro tras carro, oyéndose de nuevo el
rechinar áspero y melancólico de los ejes, que aun desde muy lejos se
percibía clarísimo en el tétrico silencio de la noche. Los farolillos
recogíanse poco a poco en el cuerpo de guardia como luciérnagas que
corren a sus agujeros; se apagaron las hachas y se extinguieron los
graznidos, cayendo todo en una especie de letargo, precursor del
profundo sueño en que termina la embriaguez.

Sarmiento se alejó de allí, y antes de tomar el camino de los Ocho Hilos
para subir a la puerta de Toledo, parose para ver los carros que ya a
mediana distancia iban por el paseo Imperial. Bien pronto dejó de
verlos, a causa de la oscuridad, mas conocía su situación por el
farolillo que el vehículo delantero llevaba. Con voz sorda habló así el
viejo patriota:

---¡Oh! tú, el héroe más grande que han producido las edades todas,
insigne campeón de la libertad española, soldado ilustre, Riego, amigo
mío, si ahora vas conducido entre sayones en ignominioso carro, mañana
tendrás un trono en el corazón de todos los españoles. Si te arrastran a
suplicio afrentoso los infames verdugos a quienes perdonamos cuando
éramos fuertes, tu nombre, que tanto repugna a despóticos oídos, será un
símbolo de libertad y una palabra bendita cuando humillada la tiranía se
restablezca tu santa obra. Subirás a la morada de los justos entre coros
de patrióticos ángeles qu entonen tu himno sonoro, mientras tu patria se
revuelve en el lodo de la reacción domeñada por tus verdugos. ¡Oh, feliz
tú, feliz cuanto grande y sublime! ¡Varón excelso, el más precioso que
Dios ha concedido a la tierra, si fuera dable a este humilde mortal
participar de tu gloria!\ldots{} ¡Si al menos pudiera yo compartir tu
martirio y entrar contigo en la cárcel, y oír juntos la misma sentencia,
y subir juntos a la misma horca!\ldots{} Este honor, yo lo ambiciono y
lo deseo con todas las fuerzas de mi alma. Vacío y desierto está el
mundo para mí, después que he perdido al lucero de mi existencia, a
aquel preciosísimo mancebo inmolado como tú al numen sanguinario de la
reacción\ldots{} Quiero morir, sí, y moriré.

Inflamado en furor que no tenía nada de risible, añadió corriendo con
agitación:

---Quiero morir gloriosamente; quiero ser víctima sublime; quiero ser
mártir de la libertad; quiero subir al patíbulo\ldots{} ¡Sicarios, venid
por mí!

Tropezando en un árbol, estuvo a punto de caer en tierra. Entonces
añadió hablando consigo mismo:

---¡Ah, Patricio, tu noble arranque me causa la más viva
admiración!\ldots{} Mañana has de hacer algo digno de pasar a las más
remotas edades. Sí, mañana. Vámonos a casa.

Echó a andar, y al poco rato dijo:

---¿Pero en dónde está mi casa? Pues no se me ha olvidado dónde está mi
casa\ldots{}

Miraba a la tierra como quien ha perdido el sombrero.

---¡Ah! Ya me acuerdo---exclamó sonriendo.---Tu casa está en la calle de
la \emph{Emancipación Social}, ¿no es verdad Patricio?

Meditaba con el índice puesto en la punta de la nariz.

---No\ldots---dijo después de una pausa, en el tono gozoso del que hace
un descubrimiento útil.---Es que yo solicité del Ayuntamiento que
llamase calle de la \emph{Emancipación Social} a la de Coloreros; pero
no accedió y sigue llamándose \emph{calle de Coloreros}. Allí vivo,
pues.

Entró en Madrid resueltamente. Subiendo por la calle de Toledo, dijo:

---Tengo hambre.

Pero después de registrar todos los bolsillos de su ropa que no bajaban
de ocho, adquirió una certidumbre aterradora, que expresó en angustiosos
suspiros.

---Parece que se me doblan las piernas y que voy a caer
desfallecido\ldots{}

¡Comer! ¡que esto sea indispensable!\ldots{} Miserable carne, ¿por qué
eres así?\ldots{} ¿A dónde iré?\ldots{} Mi casa está vacía: no hay en
ella ni una miga de pan\ldots{} ¿Pediré limosna? Jamás. Los hombres de
mi temple sucumben, pero no se humillan. A casa, Sr.~D. Patricio; si es
preciso se comerá usted el palo de una silla; ¡a casa!

Al entrar en la calle de Coloreros encontrola oscura y desierta por ser
muy avanzada la noche. Como su extenuación era grande, se habían
debilitado sus sentidos, particularmente el de la vista, y necesitó
palpar las paredes para encontrar la puerta. Sin saber por qué vino
entonces a su mente un recuerdo muy triste, que ya otras veces había
turbado profundamente su espíritu. Parecíale estar viendo delante de sí,
en una noche oscura como aquella, al sin ventura Gil de la Cuadra
arrojado en el suelo, arrastrando ignominiosa cadena, insultado por los
polizontes. De todos los incidentes de aquella lúgubre escena, el más
presente en la memoria de D. Patricio y el que le causaba más dolor era
el ocurrido cuando su infeliz vecino preso pidió agua y Sarmiento,
inspirándose en el más cruel fanatismo, se la negó.

---Ya, ya lo sé---dijo D. Patricio cerrando los ojos para dominar mejor
su terror,---ya sé que aquello fue una gran bellaquería.

Y abriendo, no sin trabajo, la puerta, entró, apresurándose a cerrar
tras sí porque le parecía que feos espectros y sombras iban en su
seguimiento y que oía el lamentable son de la cadena de Gil de la
Cuadra, arrastrando por las baldosas. Buscó en sus bolsillos eslabón y
yesca para encender luz, mas nada halló de que pudiera sacarse lumbre.
Sin desanimarse por esto, acometió la escalera con mucho cuidado y
empezó a subir, deteniéndose en cada escalón para tomar fuerzas. Pero no
había subido ocho cuando le fue preciso andar a gatas porque las piernas
no podían con el peso del desmayado cuerpo.

---Si me iré a morir aquí---dijo con angustia bañado en sudor
frío.---¡Oh! Dios mío. ¿Me estará reservada una muerte oscura, en mísera
escalera, aquí, olvidado de todo el mundo\ldots? Piedad, Señor\ldots{}

Sus fuerzas, a causa de la inacción, se extinguían rápidamente. Llegó a
no poder mover brazo ni pierna. Entonces dio un ronquido y entregose a
su malhadado destino.

---¡Oh! no, Señor---pensó allá en lo más hondo de su pensar;---no era
así como yo quería morir.

Sus sentidos se aletargaron; pero antes de perder el conocimiento, vio
un espectro que hacia él avanzaba.

Era un hermoso y brillante espectro que tenía una luz en la mano.

\hypertarget{iii}{%
\chapter{III}\label{iii}}

Cuando volvió en su acuerdo, el buen anciano se encontró en un lugar que
era indudablemente su casa y que sin embargo bien podía no serlo. Llena
de confusión su mente, miraba en derredor y decía:

---Indudablemente es mi casa; pero mi casa no es así.

Se incorporó en el canapé donde yacía, tocó la pared cercana, midió con
la vista las distancias, y a medida que se aclaraba su entendimiento,
más grande era su confusión. La semejanza entre su casa y aquella en que
estaba era muy grande, pero también había diferencias, siendo las
principales el aseo, los muebles y el orden perfecto de todo. Pero lo
que más sorprendió al maestro de escuela fue ver en mitad de la
encantada pieza una mesa puesta como para cenar, alumbrada por lámpara
de pantalla, y que en la blancura de sus manteles y en el brillo de los
platos revelaba las hacendosas manos que habían andado por allí. Como la
mesa puesta, y puesta de aquel modo era el más grande fenómeno que podía
presentarse ante los ojos de Sarmiento en su propia casa, creyose
juguete de duendes o artes demoníacas. Probó a levantarse y pudo
sostenerse en pie aunque apoyándose en la silla. Junto a la mesa había
un sillón, y como Sarmiento lo creyese destinado a su persona, no vaciló
en ocuparlo. En el mismo instante llegaron a su nariz olores de comida
muy picantes y aperitivos. El anciano exclamó con mayor confusión:

---No, esta no es mi casa.

Decíalo por aquellos olores que hacía mucho tiempo habían dejado de
acompañarle en su domicilio. A pesar de no ser supersticioso afirmose en
la idea de hallarse bajo la acción de una magia o bromazo de Satanás. Y
sin embargo, era la cosa más sencilla del mundo. Pronto se convenció de
ello nuestro amigo viendo entrar a una joven vestida de negro, la cual
se llegó a él sonriendo y le dijo:

---Buenas noches, Sr.~D. Patricio. ¿Ya se le pasó a usted el desmayo?
Bien decía yo que no era nada. Sin embargo, mandamos llamar un médico.

---¡Por vida de cien mil chilindrones!---repuso Sarmiento, saliendo poco
a poco del estupor en que había caído.---Pues no me queda duda de que
estoy hablando con Solita en persona.

---La misma---dijo la joven acercándose a la mesa y apoyando ambas manos
en ella para contemplar más de cerca al viejo.

¿Y cómo es que estoy en mi casa y no estoy en ella?

---Está usted en la mía.

---¡Ah! bien lo decía yo, bien lo decía. Estos platos, estos ricos
olores, este arreglo no pueden existir en la casa de un pobre maestro de
escuela sin discípulos. Como todos los cuartos de la casa son iguales,
de aquí que\ldots{} Pues con permiso de usted\ldots{} me retiro a mi
vivienda\ldots{}

---Antes cenará usted---dijo la muchacha sonriendo con bondad.---Me han
dicho que no hay gran abundancia por allá arriba.

---¿Cómo ha de haber abundancia donde reina con imperio absoluto la
desgracia? He caído, señorita D.ª Sola, a los más profundos abismos de
la miseria. Vea usted en mí una imagen del santo patriarca Job. ¡Dios me
ha quitado todo, me ha quitado a mi hijo!

---Cómo ha de ser\ldots{} Es preciso aceptar con resignación esos golpes
y todos los que vengan detrás. Ahora cene usted, que Dios manda a los
desgraciados no abandonarse al dolor y dar al cuerpo todo lo que el
cuerpo necesita.

---Usted me invita a cenar\ldots{}

---No invito, sino que obligo---afirmó Sola poniendo en la mesa pan y
vino.---Aguarde usted un momento, que no le haré esperar.

Al poco rato volvió con una cazuela de sopas, cuyo gratísimo olor
despertó en Sarmiento las más dulces sensaciones y una generosa
reconciliación con la vida.

---Debe usted recordar, Srta. D.ª Sola---dijo el preceptor, cuando la
joven le ataba las dos puntas de la servilleta detrás del cogote,---que
yo fuí encarnizado enemigo de su padre de usted, porque jamás he
transigido ni podré transigir con las perras ideas absolutistas.

---Lo recuerdo, sí; pero eso no hace al caso.

---Es que mi delicadeza---añadió Sarmiento tomando la cuchara,---no me
permite aceptar un banquete\ldots{} Con usted personalmente no hay
resentimiento\ldots{} pero ¿a qué negarlo? Usted y yo no podemos ser
amigos hoy ni nunca\ldots{} dígolo para que no se crea que adulo, que me
dejo seducir y sobornar por este fino obsequio, que agradezco.

---Cene usted, cene usted\ldots---dijo Solita llenándole el vaso.---La
mucha conversación podrá ser perjudicial a su cabeza, que según me han
dicho, no está del todo buena.

---Cenaré, señora, puesto que usted lo toma tan a pechos\ldots{} Conste
que yo no he mendigado esta cena; conste que me han traído aquí por
fuerza; que no he solicitado esta amistad, conste, en fin, que no
podemos ser amigos.

---Aunque no quiera serlo mío, yo me empeño en serlo de usted y lo he de
conseguir---dijo Soledad sonriendo, y hablando al viejo en el tono que
se emplea con los chiquillos.

---Dale, dale---repuso Sarmiento engullendo aprisa.---Conque amiguitos,
¿eh? ¡Chilindrón!\ldots{} Como si no hubiera pasado nada\ldots Usted no
tiene memoria, sin duda.

---Verdaderamente no tengo mucha para el daño recibido.

---Su dichosito papaíto de usted y yo éramos como el agua y el
fuego\ldots{} Mi deber era perseguirle, denunciarle, no dejarle
respirar\ldots{} Yo siempre cumplo mi deber, yo soy esclavo de mi deber.
Pertenezco a mi patria, una idea, ¿me entiende usted?

---Entiendo.

---Con nada transijo. El enemigo de la patria es mi enemigo, y la hija
del enemigo de mi patria es mi enemiga. ¿Qué dice usted a eso?

---Que no ha tratado a las sopas como enemigas de la patria.

---No ciertamente, porque hace mucho tiempo que no las había comido tan
buenas.

---Ahora voy por la perdiz.

---¿Perdiz?\ldots{} Vamos, esto parece un cuento de brujas\ldots{} Si se
empeña usted\ldots{} pero conste que yo no he pedido la perdiz; que yo
no he mendigado nada, que\ldots{}

Un momento después Sola partía la perdiz, ofreciéndola pedazo tras
pedazo al hambriento anciano.

---Está sabrosísima\ldots{} Pero con la sorpresa de esta cena había
olvidado\ldots{} ¿Cuándo ha llegado usted, Sra. D.ª Solita? ¿Qué tal le
ha ido en su viaje?

---He llegado esta mañana. Los de Cordero me hablaron de usted\ldots{}
Dijéronme que estaba usted loco\ldots{}

---¡Loco yo!

---O poco menos. Que andaba usted mal de fondos.

---Eso sí que es como el Evangelio.

---Que había perdido usted a su hijo Lucas.

---También ¡ay! es verdad.

---Esperé verle a usted y ofrecerle algo de lo poco que yo tengo.

---Gracias\ldots{}

---Pero usted había salido antes que yo llegara. Había ido, según me
dijeron, a correr por las calles divirtiendo a los chicos, y sirviendo
de entretenimiento, con sus discursos, a los desocupados de los cafés y
de la Puerta del Sol.

---¡Yo!

---Descansé un poco. Todo el día lo he empleado en arreglar mi casa. He
buscado una sirviente, he hecho parte de lo mucho que hay que hacer
cuando se ha tenido todo abandonado a causa de una ausencia de cinco
meses. Ya muy entrada la noche sentí pasos en la escalera y después
lamentos y quejidos como de una persona enferma. Salimos y hallamos al
gran D. Patricio tendido boca abajo. Los vecinos salieron, y unos
decían: «¡Buena turca ha cogido!» otros: «¡Ya las pagó todas juntas!»
¡Cómo reían algunos!\ldots{} «El maldito viejo ya echó su último
discurso\ldots» «¡Qué feísimo está!» Don Juan de Pipaón dijo: «No tiene
sino hambre. Denle a oler sopas y verán cómo resucita\ldots» Me pareció
que esta opinión era la más razonable. Entre el mancebo de los Corderos,
mi criada y yo entramos el cuerpo desmayado en mi casa, que estaba seis
escalones más arriba, le tendimos en ese sofá\ldots{}

---Conste que yo no entré por mi pie, que no pedí\ldots---dijo Sarmiento
con viveza arqueando las cejas.

---Le abrigamos bien, vino el veterinario del sotabanco y dijo que usted
padecía estos desvanecimientos desde que había dado en el hito de hablar
mucho y no comer\ldots{} Yo había cenado ya: al momento dispuse otra
cena para el nuevo huésped.

---Traído por fuerza; es decir, acogido, secuestrado, usurpado durante
su desmayo.

---Mandé venir un médico, mientras hacía la cena---añadió Sola
observando con la mayor complacencia el buen apetito de Sarmiento.---Yo
creí que al pobre hombre no le vendrían mal estos cuidados. Yo dije para
mí: «Cuando se ponga bueno y se le despeje la cabeza, abrirá de nuevo la
escuela, se llenarán sus bolsillos, y podrá vivir otra vez solo y
holgado en su casa. Entretanto le conservaré en la mía, si quiere, y
partiré con él lo poco que tengo».

---¡Cuidarme, conservarme aquí, darme asilo!\ldots---murmuró D. Patricio
con cierto aturdimiento.

---Me han dicho que el casero le va a plantar a usted en la calle esta
semana.

---Ese troglodita será capaz de hacerlo como lo dice.

---En aquel cuarto le he preparado a usted una cama---manifestó Soledad,
señalando una alcoba cercana.

D. Patricio miró y vio un lecho, cuyas cortinas blancas le deslumbraron
más que si fueran rayos de sol.

---¡Una cama!\ldots{} ¡para mí!\ldots{} ¡para mí que hace cinco meses
duermo en el suelo!\ldots{}

---Aquí podrá usted vivir. Yo estoy sola, quizás lo esté por mucho
tiempo---añadió la joven poniendo delante del anciano un plato de
uvas.---La casa es demasiado grande para mí\ldots{} No tendrá usted que
ocuparse de nada\ldots{} le cuidaré, le alimentaré.

---¡Me cuidará, me alimentará!\ldots{} Repito que esto es magia.

---Es caridad\ldots{} ¿Por ventura no entienden de caridad los
patriotas?

---Sí entendemos, sí---replicó Sarmiento tan aturdido ya que no sabía
qué decir.---¡La caridad! sublime sentimiento. Pero no ha de
sobreponerse al tesón ni a la fijeza de ideas. La caridad puede llegar a
ser un mal muy grande si se emplea en los enemigos de la patria, en los
ministros del error\ldots{} ¿Qué le parece a usted?

---Que las uvas no deben de ser ministros del error, según las ha cogido
usted.

---Están riquísimas\ldots{} Yo ¿cómo negarlo? agradezco a usted sus
obsequios\ldots{} Quizás pueda algún día corresponder a tantas finezas
con otras igualmente delicadas\ldots{} Conque dice que me dará una
cama\ldots{}

---Aquella\ldots{}

---Y desayuno\ldots{}

---También.

---Y comida\ldots{}

---Y cena. Soy pobre; pero tengo para vivir algún tiempo. Después Dios
nos dará más. Ya ve usted que si a veces quita, también da cuando menos
se espera.

---Es cierto, sí, es cierto---dijo Sarmiento con viva emoción que se
apresuró a disimular.---Pero me asombra una cosa.

---¿Qué?

---La poca memoria de usted.

---¿Poca memoria? En verdad no es mucha---dijo Sola ofreciéndole un vaso
de agua.---A veces no sirve la memoria sino de estorbo.

---Pues sí---añadió Sarmiento mascullando las palabras y algo
cortado.---Usted no se acuerda\ldots{} de que yo\ldots{} no era santo de
la devoción de su papá de usted\ldots{} Porque que digan arriba, que
digan abajo, su papá de usted conspiraba. Así es que yo\ldots{} Mire
usted, siempre que me acuerdo de esto, tengo una congoja\ldots{} Cierta
noche, cuando llevaron preso al Sr.~Gil de la Cuadra, yo\ldots{} Repito
que él conspiraba y que hacían bien en prenderle\ldots{} ¿Usted
recuerda\ldots?

Soledad, pálida y abatida, miraba fijamente el mantel.

---Usted recuerda que su papá\ldots{} cuando le pusieron las cadenas,
¿eh?\ldots{} pues sí, parece que tenía sed. Me pidió agua, y yo no se la
quise dar. Hice mal, mal, mal; aquello fue una bellaquería, una
brutalidad\ldots{} una infamia: seamos claros. Más adelante, cuando
vivían ustedes en casa de Naranjo\ldots{} que, entre paréntesis, era un
gran bribón, yo\ldots{} en fin, recordará usted que la noche en que
murió el señor Gil de la Cuadra, me metí en la casa con otros milicianos
para registrarla\ldots{} Confiese usted que teníamos razón, porque su
papá de usted conspiraba, es decir, nones, ya no conspiraba por causa de
estar muerto; pero\ldots{}

La confesión de sus brutales actos de fanatismo costaba al preceptor
sudores y congojas; pero sentía la necesidad imperiosa de echar de sí
aquel tremendo peso, y como con tenazas iba sacándose las palabras.

---Ello es que yo me porté mal aquella noche\ldots{} Verdad que éramos
enemigos; que él conspiraba contra la libertad; que yo tenía una misión
que cumplir\ldots{} el Gobierno descansaba en mi vigilancia\ldots{} Pero
de todos modos, Sra. D.ª Solita, usted no obra cuerdamente al tratarme
como me trata.

---¿Por qué?---dijo la joven alzando sus ojos llenos de lágrimas.

---Porque somos enemigos políticos.

Bañado el rostro en lágrimas, Sola se echó a reír, lo que producía
singular contraste.

---Porque somos enemigos encarnizados\ldots{} porque me porté mal, y si
ahora salimos con que usted me da cama y mesa\ldots{} Además mi dignidad
no me permite aceptarlo, no señora. Parecerá que he cedido en mis
opiniones\ldots{} que transijo con ciertas ideas.

Sola reía más.

---Usted se burla de mí. Bien: no hablemos más del asunto. Se me figura
que usted me perdona aquellos desmanes. Bien, muy bien. Reconozco que es
un proceder admirable; pero yo\ldots{} póngase usted en mi lugar\ldots{}

---Me parece---dijo Sola,---que ya es hora de que se acueste usted.

---¿En esa cama?---dijo Sarmiento con incredulidad y abriendo mucho lo
ojos.

---En esa.

---¡Y tiene colchones!

---Y manta\ldots{} Ya que tiene usted repugnancia de aceptar lo que le
ofrezco, no insistiré---dijo la muchacha con malicia;---pero valga mi
hospitalidad por esta noche. Mañana se volverá usted a su casa.

---Bien, bien---exclamó Sarmiento.---Por vida de la chilindraina, que es
una excelente idea. Mañana lo decidiremos, y esta noche como estoy tan
cansado\ldots{} En verdad, ¿para qué necesito yo colchones ni platos
exquisitos si están contados mis días?\ldots{} ¡Ay! La pérdida de mi
hijo me ha secado el corazón. Para mí ha concluido el mundo. Conozco que
estoy de más y me apresuro a emprender el viaje. Pero ha de saber usted
que mi idea es morir gloriosamente, mi plan tener un fin que corresponda
a la grandeza de las doctrinas que he sustentado en vida. Yo no puedo
morir como otro cualquiera, Sra. D.ª Solita, y aquí me tiene usted en
camino de llenar una página de la historia.

Sola parecía inquieta oyendo los disparates de su huésped.

---Sí señora---añadió Sarmiento exaltándose y echando lumbre por los
ojos.---Voy a morir por la patria, voy a morir por la libertad, por esa
luz que ilumina al mundo; voy a ser mártir; voy a elevar mi frente como
los héroes, conquistando con un fin heroico la inmortalidad.

---Lo que yo veo es que era cierto lo que me habían dicho.

D. Patricio se levantó y tomando una actitud de estatua, prosiguió de
este modo:

---¿A qué arrastrar una vejez oscura y miserable, cuando las
circunstancias me brindan con la inmortalidad? El ejemplo de ese héroe a
quien he visto conducido como los criminales y que subirá al Calvario
dentro de poco, me sirve de guía. ¡Oh luz de mi inteligencia, bendita
seas por haberme inspirado esta idea!

Tomando luego bruscamente el tono familiar, dijo a Solita:

---Pocos días me restan de vida. Quizás tres, quizás dos, quizás uno
solo. Como he de molestar por tan poco tiempo, apreciable señora, me
quedaré aquí.

---Está muy bien pensado. Ahora a dormir.

Vino el médico que habían llamado, y Sarmiento le despidió de mal
talante, diciendo que no necesitaba medicinas, porque para él, el cuerpo
no era nada y el alma todo. El médico que ya le conocía, encargole mucho
cuidado con la cabeza, advirtiendo reservadamente a Sola que le
encerrara si tenía empeño en que tal enfermo viviese. Después de la
partida del Galeno, D. Patricio mostró deseos de acostarse.

---Buenas noches, señora---dijo el preceptor entrando en la
alcoba.---¿Mañana tomaré chocolate?

---¿Eso había de faltar? Si no fuera por esa dichosa muerte heroica que
le espera, le tomaría usted muchos días. ¡Qué necedad privarse de ese
gusto por la gloria que no es más que humo!

---Usted habla en broma---dijo D. Patricio, cuya voz se oía débilmente
desde la sala, porque había cerrado la puerta para acostarse.---No puedo
comprender que su claro entendimiento compare unas cuantas onzas de
soconusco con la inmortalidad y la gloria\ldots{} ¡Ah! señora mía, lo
único que me consuela de la pérdida que acabo de experimentar, es el
saber que mi adorado hijo está gozando de esa inextinguible luz de la
gloria, premio justo de los que han muerto defendiendo la libertad.
¡Mártir sublime, que Dios te bendiga como te bendigo yo! ¡Yo que me
apresuro a imitarte!\ldots{} ¿Solita, se ha marchado usted?

---No señor, aquí estoy oyéndole con mucho gusto. ¡Cuánto siento la
muerte del pobre Lucas!\ldots{} ¡Era tan buen muchacho!\ldots{}

---¡Válgame Dios lo que he perdido! Era un dechado de virtudes---dijo
Sarmiento dando un gran suspiro,---y de amor filial. Su inteligencia
superior se remontaba a las más altas concepciones. Su valor indomable
no tenía igual, y creeríase al verle que en él había resucitado un héroe
antiguo. Vamos, que en aquel famoso 7 de Julio, dejó bien puesto el
pabellón\ldots{} ¡Pobre hijo mío! Sus nobles facciones eran idénticas a
las de su madre. ¡Si supiera usted cuán hermosa era mi Refugio!\ldots{}
¿Está usted ahí, Solita?

---Aquí estoy. Sí, debía de ser muy hermosa D.ª Refugio.

¡Ah! ¡Si usted la hubiera visto!\ldots{} ¡Qué boca!\ldots{} ¡qué
ojos!\ldots{} ¡qué pie!\ldots{} Me parece que la estoy mirando. La
llamaban la diosa de Calabazar del Buey por ser este el lugar de su
nacimiento\ldots{} ¡Oh dulces memorias! ¿por qué venís a atormentarme en
estas aflictivas horas?\ldots{} Yo me enamoré de Refugio como un
insensato, porque siempre he sido así, un fuego vivo. ¡Cuánto me costó
sacarla de la casa paterna!\ldots{} en fin, nos unimos en dulce lazo el
día de la Encarnación\ldots{} Por Noche-Buena nació nuestro pobre Lucas,
que parecía una bola de oro y manteca\ldots{} ¡Oh tiempos!\ldots{}
señora doña Solita.

---¿Qué?

---¿Se ha marchado usted?

---No señor, aquí estoy.

---Parece que se ríe usted.

---De ningún modo.

---Hágame usted el favor de abrir la puerta, porque deseo verla a usted
antes de dormir. Es una necesidad de mi pobre espíritu.

Soledad abrió. Completamente arrebujado en las sábanas, D. Patricio no
mostraba más que la cabeza.

---Está usted mucho más guapa que cuando vivía el Sr.~Gil de la
Cuadra---insinuó el viejo.

---Podrá ser.

---¿Se acuesta usted ya?

---Antes tengo que hacer.

---Pues buenas noches, porque a causa del mucho cansancio\ldots{}
Perdone usted mi descortesía; pero no lo puedo remediar; me duermo como
un animal. ¡Oh gloria, oh lauros inmortales, oh libertad!\ldots{} Esta
cama\ldots{} es tan\ldots{} buena\ldots{}

\hypertarget{iv}{%
\chapter{IV}\label{iv}}

Pasando sobre treinta y cinco días, nos trasladamos con el lector al 6
de Noviembre.

La plazuela de la Cebada, prescindiendo del mercado que hoy la ocupa
desfigurándola y escondiendo su fealdad, no ha variado cosa alguna desde
1823. Entonces, como hoy, tenía aquel aire villanesco y zafio que la
hace tan antipática, el mismo ambiente malsano, la misma arquitectura
irregular y ramplona. Aunque parezca extraño, entonces las casas eran
tan vetustas como ahora, pues indudablemente aquel amasijo de tapias
agujereadas no ha sido nuevo nunca. La iglesia de Nuestra Señora de
Gracia, viuda de San Millán desde 1868, tenía el mismo aspecto de
almacén abandonado, mientras su consorte, arrinconado entre las
callejuelas de las Maldonadas y San Millán, parecía pedir con suplicante
modo que le quitaran de en medio. La fundación de D.ª Beatriz Galindo no
daba a la plaza sino podridos aleros, tuertos y llorosos ventanuchos,
medianerías cojas y covachas miserables. La elegante cúpula de la
capilla de San Isidro, elevándose en segundo término, era el único
placer de los ojos en tan feo y triste sitio.

Esta plazuela había recibido de la Plaza Mayor, por donación graciosa,
el privilegio de despachar a los reos de muerte, por cuya razón era más
lúgubre y repugnante. Aquella boca monstruosa y fétida se había tragado
ya muchas víctimas, y ¡cuántas le quedaban aún por tragar desde aquella
célebre fecha de Noviembre de 1823, que ennobleció la plaza-cadalso,
dándole nombre más decoroso que el que siempre ha llevado!

En la mañana del 6 estaba llena de curiosos que por las calles
afluyentes entraban para ver los dos palos largos plantados en medio de
tal plaza, y asistir con curiosidad afanosa a la tarea de seis hombres
que se ocupaban en unir los topes de dichos árboles con un tercer madero
horizontal. Los corrillos eran muchos y la gente iba y venía paseando
como en los preliminares de una fiesta. Veíanse hombres uniformados,
otros con armas y sin uniforme, mucha gente del populacho que por
aquellos barrios abajo tiene sus albergues, y no pocas personas de la
clase acomodada. Un hombre alto, seco, moreno, de ojos muy saltones, de
rostro fiero y ademán amenazador, mirar insolente, boca bravía, como de
quien no muerde por no menoscabar la dignidad humana; un hombre que
francamente mostraba en todo su condición perversa, y en cuyo enjuto
esqueleto el uniforme de brigadier parecía una librea de verdugo, avanzó
resueltamente por entre el gentío, abriéndose calle bastón en mano; y
dirigiéndose después con airada voz y gesto a los que trabajaban en el
cadalso, les dijo:

---¡Malditos!\ldots{} Mal haya el pan que se os da\ldots{} ¿No he
mandado que se pusieran los palos más grandes que hay en los almacenes
de la Villa?

Uno que parecía jefe de los aparejadores balbució algunas excusas que no
debieron de satisfacer al vestiglo, porque al punto soltó por su
abominable boca nueva andanada de denuestos:

---¡Ahora mismo, ahora mismo, canallas!\ldots{} quitarme de ahí ese
juguete, si no quieren que los cuelgue en él\ldots{} Traigan los palos
grandes, los más grandes, aquellos que estaban la semana pasada en el
Canal\ldots{} ¿Entienden lo que digo?\ldots{} ¿Hablo yo en
castellano?\ldots{} Los palos grandes.

Otra vez se disculparon los aparejadores, pero el del bastón repitió sus
órdenes.

---Si hace falta más gente, venga más gente\ldots{} Estos holgazanes no
comprenden la gravedad de las circunstancias, ni están a la altura de un
suceso como este\ldots{} Por vida del Santísimo Sacramento que yo les
haré andar a todos derechos\ldots{} Sr. Cuadrado, lleve usted al Canal a
todos los operarios de la Villa para transportar esos leños, y si no iré
yo mismo, que lo mismo sirvo para un fregado que para un barrido.

Tres horas más tarde, el deseo de aquel hombre tan atroz se empezaba a
cumplir, y la gente allí reunida (porque había más gente) vio que se
elevaban con majestad dos maderos como mástiles de barco, gruesos,
lisos, hermosos, gallardos.

---¡Ah, muy bien!---dijo el endriago, observando desde lejos el golpe de
vista.---Esto es otra cosa. Así es como el Gobierno quiere que se haga.
¡Magnífico efecto!

Sus miradas de satisfacción recorrieron toda la plaza, por encima del
mar de cabezas, y parecía decir: «¡Feliz el pueblo que tiene al frente
de su policía un hombre como yo!»

Clavados los altos maderos, los aparejadores se ocuparon en atar la
traviesa horizontal. El efecto era soberbio.

Daba nuevas órdenes para perfeccionar tan bella obra el formidable
polizonte, cuando se llegó a él un hombre cuadrado y de semblante oscuro
e indescifrable, que le saludó cortésmente.

---¿Qué te parece Romo lo que hemos hecho?---dijo el del bastón,
cruzando atrás las manos con el emborlado instrumento de su autoridad.

---¡Oh! es la mayor que se ha elevado en Madrid---repuso contemplando la
horca.---Y si hubiera maderos de más talla, a mayor altura la
pondríamos. Esto debiera verse de toda España.

---Desde todo el mundo; que fuera de aquí también hay pillos a quienes
escarmentar\ldots{} Yo traería mañana a esta plaza a todos los españoles
para que aprendieran cómo acaban las porquerías revolucionarias\ldots{}
No hay enseñanza más eficaz que esta\ldots{} Como el nuevo Gobierno no
se empeñe en ir por el camino de la tibieza, habrá buenos ejemplos,
amigo Romo.

---Es que si se empeña en ir por el camino de la tibieza---dijo Romo
dando un golpe en el puño de su sable,---nosotros no le dejaremos
ir\ldots{}

---Bien, bien, me gustan esos bríos---afirmó un tercer personaje, casi
tan parecido a un gato como a un hombre, y que de improviso se unió a
los dos anteriores.---No ha salido el Rey de manos de los liberales para
caer en las de los tibios.

---Sr.~Regato---dijo el del bastón,---ha hablado usted como los cuatro
Evangelios juntos.

---Sr.~Chaperón---añadió Regato,---bien conocidas son mis ideas\ldots{}
¿Ve usted esa horca? Pues todavía me parece pequeña.

---Se puede hacer mayor---dijo el que respondía al nombre de
Chaperón.---Por vida del Santísimo Sacramento, que no se quejará el
Cabezudo\ldots{} y su bailoteo será bien visto.

---¿Conoce usted la sentencia?---preguntó Regato.

---Será conducido a la horca arrastrado por las calles---dijo Romo.---Si
hubieran omitido esto los jueces habría sido una gran falta.

---Es claro: hay que distinguir\ldots{} Según pedía el fiscal, la cabeza
se colocará en el pueblo donde dio el grito nefando el año 20, y el
cuerpo se dividirá en cuatro cuartos.

---Para poner uno en Madrid, otro en Sevilla, otro en Málaga y otro en
la isla de León---añadió Chaperón dando gran importancia a tan horribles
detalles.

---Pues ayer se dijo\ldots{} yo mismo lo oí\ldots---afirmó Regato,---que
los dos cuartos delanteros quedarían en Madrid. Yo no lo aseguro: pero
así se dijo.

---En puridad---dijo Chaperón,---esto no es lo más importante. En vez de
perder el tiempo descuartizando buscaremos nueva fruta de cuelga, que no
faltará en Madrid\ldots{} ¿Pero qué alboroto es ese?\ldots{} ¿Por qué
corre mi gente?

Volvió los saltones ojos hacia Nuestra Señora de Gracia, donde los
grupos se arremolinaban y se oía murmullo de vivas. El fiero jefe de la
Comisión Militar frunció el ceño al ver que el buen pueblo confiado a su
vigilancia relinchaba sin permiso de la policía.

---No es nada, Sr.~Chaperón---dijo Regato.---Es que tenemos ahí a
nuestro famoso Trapense.

---Hace un rato---añadió Romo,---venía por Puerta de Moros con su
escolta. Entró a rezar en Nuestra Señora de Gracia y ya sale otra vez.
Viene hacia acá.

En efecto, avanzaba hacia el centro de la plaza la más estrambótica
figura que puede ofrecerse a humanos ojos en esos días de revueltas
políticas, en que todo se transfigura, y sale a la superficie confundido
con la clara linfa el légamo social. Era un hombre a caballo, mejor
dicho, a mulo. Vestía hábitos de fraile y traía un Crucifijo en la mano,
y pendientes del cinto sable, pistolas y un látigo. Seguíanle cuatro
lanceros a caballo y rodeábale escolta de gritonas mujeres, pilluelos y
otra ralea de gente de esa que forma el vil espumarajo de las
revoluciones.

Era el Trapense joven, de color cetrina, ojos grandes y negros, barba
espesa, con un airecillo más que de feroz guerrero, de truhán redomado.
Había sido lego en un convento, en el cual dio mucho que hacer a los
frailes con su mala conducta, hasta que se metió a guerrillero, teniendo
la suerte de acaudillar con buen éxito las partidas de Cataluña.
Conocedor de la patria en cuyo seno había tenido la dicha de nacer,
creyó que sus frailunas vestiduras eran el uniforme más seductor para
acaudillar aventureros, y al igual de las cortantes armas puso la imagen
de Crucificado. En los campos de batalla, fuera de alguna ocasión
solemne, llevaba el látigo en la mano y la cruz en el cinto; pero al
entrar en las poblaciones colgaba el látigo y blandía la cruz, incitando
a todos a que la besaran. Esto hacía en el momento en que le vemos por
la plazuela adelante. Su mulo no podía romper sino a fuerza de cabezadas
y tropezones la muralla de devotos patriotas, y él afectando una
seriedad más propia de mascarón que de fraile, echaba bendiciones. El
demonio metido a evangelista no hubiera hecho su papel con más donaire.
Viéndole fluctuaba el ánimo entre la risa y un horror más grande que
todos los horrores. Los tiempos presentes no pueden tener idea de ello,
aunque hayan visto pasar fúnebre y sanguinosa una sombra de aquellas
espantables figuras. Sus reproducciones posteriores han sido
descoloridas, y ninguna ha tenido popularidad, sino antes bien, el odio
y las burlas del país.

Cuando el bestial fraile, retrato fiel de Satanás a caballo, llegó junto
al grupo de que hemos hablado, recibió las felicitaciones de las tres
personas que lo formaban y él les hizo saludo marcial alzando el
Crucifijo hasta tocar la sien.

---Bienvenido sea el padre Marañón---dijo el jefe de la Comisión Militar
acariciando las crines del mulo, que aprovechó tal coyuntura para
detenerse.---¿A dónde va tanto bueno?

---Hombre\ldots{} también uno ha de querer ver las cosas
buenas---replicó el fraile.---¿A qué hora será eso mañana?

---A las diez en punto---contestó Regato.---Es la hora mejor.

---¡Cuánta gente curiosa!\ldots{} No me han dejado rezar,
Sr.~Chaperón---añadió el fraile inclinándose como para decir una cosa
que no debía oír el vulgo.---Usted, que lo sabe todo, dígame ¿conque es
cierto que se nos marcha el Príncipe?

---¿Angulema? Ya va muy lejos camino de Francia. ¿Verdad, padre Marañón,
que no nos hace falta maldita?

---¿Pues no nos ha de hacer falta, hombre de Dios?---dijo el fraile
andante soltando una carcajada que asemejó su rostro al de una gárgola
de catedral despidiendo el agua por la boca.---¿Qué va a ser de nosotros
sin figurines? Averigüe usted ahora cómo se han de hacer los chalecos y
cómo se han de poner las corbatas.

---Los tres y otros intrusos que oían rompieron a reír, celebrando el
donair del Trapense.

---Queda de general en jefe el general Bourmont.

---Por falta de hombres buenos a mi padre hicieron alcalde---dijo
Chaperón.---Si Bourmont se ocupara en otra cosa que en coger moscas, y
se metiera en lo que no le importa, ya sabríamos tenerle a raya.

---Me parece que no nos mamamos el dedo---repuso el fraile.---Y me
consta que Su Majestad viene dispuesto a que las cosas se hagan al
derecho, arrancando de cuajo la raíz de las revoluciones. Dígame usted,
¿es cierto que se ha retractado en la capilla?

---¿Quién, Su Majestad?

---No, hombre, Rieguillo.

---De eso se trata. El hombre está más maduro que una breva. ¿No va
usted por allá?

---¿Por la capilla?\ldots{} No me quedaré sin meter mi cucharada\ldots{}
Ahora no puedo detenerme: tengo que ver al obispo para un negocio de
bulas y al ministro de la Guerra para hablarle del mal estado en que
están las armas de mi gente\ldots{} Con Dios, señores\ldots{} ¡arre!

Y echó a andar hacia la calle de Toledo, seguido del entusiasta cortejo
que le vitoreaba. Chaperón, después de dar las últimas órdenes a los
aparejadores y de volver a observar el efecto de la bella obra que se
estaba ejecutando, marchó con sus amigos hacia la calle Imperial, por
donde se dirigieron todos a la cárcel de Corte. En la plazuela había
también gente, de esa que la curiosidad, no la compasión, reúne frente a
un muro detrás del cual hay un reo en capilla. No veían nada, y sin
embargo, miraban la negra pared, como si en ella pudiera descubrirse la
sombra, o si no la sombra, misterioso reflejo del espíritu del condenado
a muerte.

Los tres amigos tropezaron con un individuo que apresuradamente salía de
la Sala de Alcaldes.

---¡Eh! no corra usted tanto, Sr.~Pipaón---gritole el de la Comisión
militar.---¿A dónde tan a prisa?

---Hola, señores; salud y pesetas---dijo el digno varón
deteniéndose.---¿Van ustedes a la capilla?\ldots{}

---No hemos de ser los últimos, hombre de Dios. ¿Qué tal está mi hombre?

---Va a comer\ldots{} Una mesa espléndida, como se acostumbra en estos
casos. Conque Sr.~Chaperón, Sr.~Regato\ldots{}

---¡A dónde va usted que más valga!---dijo Chaperón deteniéndole por un
brazo.---¿Hay trabajillo en la oficina?

---Yo no trabajo en la oficina, porque estoy encargado de los festejos
para recibir al Rey---repuso Bragas con orgullo.

---¡Ah! no hay que apurarse todavía.

---Pero no es cosa de dejarlo para el último día. No preparamos una
chabacanería como las del tiempo constitucional, sino una verdadera
solemnidad regia como lo merecen el caso y la persona de Su Majestad. El
carro en que ha de verificar su entrada se está construyendo. Es digno
de un Emperador romano. Aún no se sabe si tirarán de él caballos o
mancebos vistosamente engalanados. Es indudable que llevarán las cintas
los voluntarios realistas.

---Pues se ha dicho que nosotros tiraríamos del carro---dijo Romo con
énfasis, como si reclamara un derecho.

---Ahí tiene usted un asunto sobre el cual no disputaría yo---insinuó
Regato blandamente.---Yo dejaría que tiraran los caballos.

---Ya se decidirá, señores, ya se decidirá a gusto de todos---dijo
Bragas con aires de transacción.---Lo que me trae muy preocupado es
que\ldots{} verán ustedes\ldots{} me he propuesto presentar ese día
doscientas o trescientas majas lujosamente vestidas. ¡Oh! ¡qué bonito
espectáculo! Costará mucho dinero ciertamente; pero ¡qué precioso
efecto! Ya estoy escogiendo mi cuadrilla. Doscientas muchachas bonitas
no son un grano de anís. Pero yo las tomo donde las encuentro\ldots{}
¿eh? De los trajes se encarga el Ayuntamiento\ldots{} Me han dado
fondos. ¡Caracoles! es una cuestión peliaguda\ldots{} espero lucirme.

---Este Pipaón es de la piel de Satanás\ldots{} ¿De dónde van a sacar
ese mujerío?

---Yo daría la preferencia a los arcos de triunfo---dijo Romo.---Es
mucho más serio.

---¿Arcos?\ldots{} Si ha de haber cuatro. Por cierto que el Sr.~Chaperón
nos ha hecho un flaco servicio llevándose para la horca los grandes
mástiles que sirven para armar arcos de triunfo.

---Hombre, por vida del Santísimo Sacramento---dijo Chaperón mostrando
un sentimiento que en otro pudiera haber sido bondad,---ya servirán para
todo. Pues qué, ¿vamos a ahorcar a media España?

---Entre paréntesis, no sería malo\ldots{} Conque ahora sí que me voy de
veras.

Estrechó Pipaón sucesivamente la mano de cada uno de sus tres amigos.

---Ya nos veremos luego en las oficinas de la Comisión.

---Pues qué, ¿hay algo nuevo?

---Hombre no se puede desamparar a los amigos.

---¡Recomendaciones!---vociferó el brigadier mostrando su fiereza.---Por
vida del Santísimo, que eso de las recomendaciones y las amistades me
incomoda más que la evasión de un prisionero. Así no hay justicia
posible, señor Pipaón, así la justicia, los castigos y las
purificaciones no son más que una farsa.

El terrible funcionario se cruzó de brazos, conservando fuertemente
empuñado el símbolo de su autoridad.

---Es claro---añadió Romo por espíritu de adulación ,---así no hay
justicia posible.

---No hay justicia posible---repitió Regato como un eco del cadalso.

---Amigo Chaperón---dijo el astuto Bragas con afabilidad y desviando un
poco del grupo al Comisario para hablarle en secreto,---cuando hablo de
amigos me refiero a personas que no han hecho nada contra el régimen
absoluto.

---Sí, buenos pillos son sus amigos de usted.

---No es más sino que al pobre D. Benigno Cordero le está molestando la
policía de Zaragoza y es posible que lo pase mal. Ya recordará usted que
D. Benigno dio cien onzas bien contadas porque se le comprendiera en el
Decreto del 2 de Octubre fechado en Jerez. Acogiéndose a la proscripción
se libraba de la cárcel y quizás de la horca\ldots{} Pues en Zaragoza me
le han puesto en un calabozo. Eso no está bien\ldots{}

---Bueno, bueno---dijo Chaperón disgustado de aquel asunto.---También
Romo me ha recomendado a ese Cordero.

Romo no dijo una palabra, ni abandonó aquella seriedad que era en él
como su mismo rostro.

---Por última vez, señores, adiós---chilló Bragas,---ahora sí que me voy
de veras.

---Abur.

Dirigiéronse a la puerta de la cárcel por la calle del Salvador; pero
les fue preciso detenerse porque en aquel momento entraba una cuerda de
presos. Iban atados como criminales que recogiera en los caminos la
antigua Hermandad de Cuadrilleros, y por su traje, ademanes, y más aún
por el modo de expresar su pena, debían de pertenecer a distintas clases
sociales. Los unos iban serenos y con la frente erguida, los otros
abatidos y llorosos. Eran veinte y dos entre varones y hembras, a saber:
tres patriotas de los antiguos clubs dos ancianos que habían desempeñado
durante el régimen caído el cargo de vocales del Supremo Tribunal de
Justicia, un eclesiástico, dos toreros, cuatro cómicos, un chico de
siete años, descalzo y roto, tres militares, un indefinido, como no se
clasificara entre los pordioseros, una señora anciana que apenas podía
andar, dos de buena edad y noble continente, que pertenecían a clase
acomodada, y dos mujeres públicas.

Chaperón echó sobre aquella infeliz gente una mirada que bien podía
llamarse amorosa pues era semejante a las del artista contemplando su
obra, y cuando el último preso (que era una de las damas de equívoca
conducta) se perdió en el oscuro zaguán de la prisión, rompió por entre
la multitud curiosa y entró también con sus amigos.

\hypertarget{v}{%
\chapter{V}\label{v}}

Lo más cruel y repugnante que existe después de la pena de muerte es el
ceremonial que la precede y la lúgubre antesala del cadalso con sus
cuarenta y ocho mortales horas de capilla. Casi es más horrendo que la
horca misma aquella larga espera y agonía entre la vida y la muerte,
durante la cual la víctima es expuesta a la compasión pública como son
expuestos a la pública curiosidad los animales raros. La ley, que hasta
entonces se ha mostrado severa, muéstrase ahora ferozmente burlona,
permitiendole la compañía de parientes y amigos y dándole de comer a qué
quieres boca. Algún condenado de clase humilde prueba en esos dos días
platos y delicadas confituras, cuyo sabor no conocía. Señores,
sacerdotes y altos personajes le dan la mano, le dirigen vulgares
palabrillas de consuelo, y todos se empeñan en hacerle creer que es el
hombre más feliz de la creación, que no debe envidiar a los que incurren
en la tontería de seguir viviendo, y que estar en capilla con el
implacable verdugo a la puerta es una delicia. Sin embargo, a nadie se
le ha ocurrido solicitar expresamente tanta felicidad, ni contar a
Nerón, Luis XI, D. Pedro de Castilla, Felipe II, Robespierre y Fernando
VII entre los bienhechores de la humanidad.

Desde el 5 de Noviembre a las diez de la mañana gustaba D. Rafael del
Riego las dulzuras de la capilla. Aquel hombre famoso, el más pequeño de
los que aparecen injeridos sin saber cómo en las filas de los grandes,
mediano militar y pésimo político, prueba viva de las locuras de la fama
y usurpador de una celebridad que habría cuadrado mejor a otros
caracteres y nombres condenados hoy al olvido, acabó su breve carrera
sin decoro ni grandeza. Un noble martirio habría dado a su figura el
realce heroico que no pudo alcanzar en tres años de impaciente agitación
y bullanga; pero tan desgraciada era la libertad en nuestro país, que ni
al morir bajo las soeces uñas del absolutismo, pudo alcanzar aquel
hombre la dignidad y el prestigio de la idea que se avalora sucumbiendo.
Pereció como la pobre alimaña que expira chillando entre los dientes del
gato.

La causa del revolucionario más célebre de su tiempo fue un tejido de
iniquidades y de absurdos jurídicos. Lo que importaba era condenarle
emborronando poco papel, y así fue. Desde que le leyeron la sentencia el
preso cayó en un abatimiento lúgubre, hijo según algunos, de sus
dolencias físicas. Creeríase que confiaba hasta entonces en la clemencia
de los llamados jueces o del Rey, que es todo el caudal de inocencia que
puede caber en espíritu de hombre nacido. A diferencia de otros que en
horas tan tremendas se atracan de los ricos manjares con que engorda el
verdugo a sus víctimas, no quiso comer o comió muy poco. Ningún amigo
pudo visitarle porque la visita hubiera sido quizás el primer paso para
compañía perpetua hasta la eternidad; pero le vieron muchos individuos
particulares de categoría, deseosos de hartar sus ojos con la vista de
aquel hombre que conmovió con su nombre a toda España; sacerdotes que
solícitamente se prestaban a encaminarle al cielo; hermanos de diversas
hermandades; personas varias, en fin, compungidas las unas, indiferentes
otras, curiosas las más: pero en tal número que no dejaban al preso un
momento de descanso.

Estaba frío, caduco, con los ojos fijos en el suelo, amarillo como las
velas que ardían junto al Crucifijo del altar. A ratos suspiraba,
parecía vagar en sus labios la palabra perdón, acometíanle desmayos y
hacía preguntas triviales. Ni mostró apego a las ideas políticas que le
habían dado tanto nombre, ni dio alas a su espíritu con la unción
religiosa, sino que se abatía más y más a cada instante, apareciendo
quieto sin estoicismo, humilde su resignación. Chaperón y otros de igual
talla gozaban viendo llorar como un alumno castigado al general de la
Libertad, al pastor que con la magia de su nombre arrastraba tras sí
rebaño de los pueblos. En el delirio de su triunfo no habían ellos
soñado con una caída semejante que les desembarazara no sólo de su
enemigo mayor, sino del prestigio de todos los demás.

La retractación del héroe de las Cabezas fue una de las más ruidosas
victorias del bando absolutista. ¡Qué mayor triunfo que mostrar a los
pueblos un papel en que de su puño y letra había escrito el hombre
diminuto estas palabras: «Asimismo publico el sentimiento que me asiste
por la parte que he tenido en el Sistema llamado constitucional, en la
revolución y en sus fatales consecuencias, por todo lo cual pido perdón
a Dios de mis crímenes\ldots» Han quedado en el misterio las
circunstancias que acompañaron a este arrepentimiento escrito, y aunque
el carácter de Riego y su pusilanimidad en las tremendas horas
justifican hasta cierto punto aquella genuflexión de su espíritu, puede
asegurarse que no hubo completa espontaneidad en ella. El fraile que le
asistía, Chaperón y el escribano Huerta sabrían acerca de este suceso
cosas dignas de pasar a la posteridad, porque a ellos debieron los
absolutistas el envilecimiento del personaje más culminante, si no el
más valioso de la segunda época constitucional. Ahora, cuando ha pasado
tanto tiempo y la losa del sepulcro los guarda a todos, ahorcadores y
ahorcados, no podemos menos de deplorar que los que acompañaron en la
capilla a D. Rafael del Riego en la noche del 6 al 7 de Noviembre no
hubieran hecho públicos después los argumentos empleados para arrancar
una abdicación tan humillante.

El 7 a las diez de la mañana le condujeron al suplicio. De seguro no ha
brillado en toda nuestra historia un día más ignominioso. Es tal que ni
aun parece digno de ser conocido, y el narrador se siente inclinado a
volver, sin leerla, esa página sombría, y a correr tras de una ficción
verosímil que embellezca la descarnada verdad histórica. Una víctima sin
nobleza, arrastrada al suplicio por verdugos sin entrañas es el
espectáculo más triste que pueden ofrecer las miserias humanas; es el
mal puro sin porción ninguna de bien, de ese bien moral que aparece más
o menos claro aun en los más horrendos excesos del furor político y en
los suplicios a que es sometida la inocencia. Una víctima cobarde parece
que enaltece al verdugo, y al hablar de cobardía no es que echemos de
menos la arrogancia fanfarrona con que algunos desgraciados han querido
dar realce teatral a su postrer instante, sino la dignidad personal que
unida a la resignación religiosa rodean al mártir jurídico de una
brillante aureola de simpatías y compasión. Ninguna de aquellas especies
de valor tuvo en su desastroso fin el general Riego, y creeríase al
verle que víctima y jueces se habían confabulado para cubrir de
vilipendio el último día de la libertad y hacer más negro y triste su
crepúsculo. La grosería patibularia y el refinamiento en las fórmulas de
degradación empleadas por los unos, parece que guardaban repugnante
armonía con la abjuración del otro.

Sacáronle de la cárcel por el callejón del Verdugo, y condujéronle por
la calle de la Concepción Jerónima, que era la carrera oficial. Como si
montarle en borrico hubiera sido signo de nobleza, llevábanle en un
serón que arrastraba el mismo animal. Los hermanos de la Paz y Caridad
le sostuvieron durante todo el tránsito para que con la sacudida no
padeciese; pero él, cubierta la cabeza con su gorrete negro, lloraba
como un niño, sin dejar de besar a cada instante la estampa que sostenía
entre sus atadas manos.

Un gentío alborotador cubría la carrera. La plaza era un amasijo de
carne humana. ¿Participaremos de esta vil curiosidad, atendiendo
prolijamente a los accidentes todos de tan repugnante cuadro? De ninguna
manera. ¡Un hombre que sube a gatas la escalera del patíbulo, besando
uno a uno todos los escalones, un verdugo que le suspende y se arroja
con él, dándole un bofetón después que ha expirado, una ruin canalla que
al verle en el aire grita: «Viva el Rey absoluto»\ldots! ¿acaso esto
merece ser mencionado? ¿Qué interés ni qué enseñanza ni qué ejemplo
ofrecen estas muestras de la perversidad humana? Si toda la historia
fuese así, si no sirviera más que de afrenta, ¡cuán horrible sería!
Felizmente aun en aquellos días tan desfavorecidos, contiene páginas
honrosas aunque algo oscuras, y entre los miles de víctimas del
absolutismo húbolas nobilísimas y altamente merecedoras de cordial
compasión. Si el historiador acaso no las nombrase, peor para él; el
novelador las nombrará, y conceptuándose dichoso al llenar con ellas su
lienzo, se atreve a asegurar que la ficción verosímil ajustada a la
realidad documentada, puede ser en ciertos casos más histórica y
seguramente es más patriótica que la historia misma.

\hypertarget{vi}{%
\chapter{VI}\label{vi}}

El triste día de la ejecución todo Madrid asistió a ella, lo mismo los
absolutistas rabiosos que los antiguos patriotas, a excepción de los que
no podían salir a la calle sin peligro de ser afeitados o arrojados en
los pilones de las fuentes, cuando no hechos trizas por el vulgo. Pero
entre tanto gentío faltó un hombre que durante el verano había vivido
casi constantemente en la calle, entreteniendo a los desocupados y dando
que reír a los pícaros. Echábanle de menos en las esquinas de la Puerta
del Sol y en los diversos mentideros, por lo cual le creían muerto. No
era cierto. Sarmiento vivía, gozando además de una regular salud.

La primera noche que se quedó en casa de Solita durmió de un tirón once
horas, y habiendo despertado al medio día, llamó con fuertes voces para
que le llevaran chocolate. Dióselo la misma dueña de la casa con mucha
amabilidad, y entre sorbo y sorbo, el preceptor decía:

---Puedo aceptar estos obsequios porque hoy mismo entraré por la senda a
que me lleva mi destino\ldots{} Si fuera por mucho tiempo de ningún modo
aceptaría\ldots{} Mi carácter, mi dignidad, los recuerdos de nuestro
antagonismo no me lo permiten.

---¿Qué tal está el chocolate?---le preguntó Sola con malignidad.

---Así, así\ldots{} mejor dicho, no está mal\ldots{} quiero decir, muy
bueno, excelente, o hablando con completa franqueza, riquísimo.

---¿Hoy se marcha usted?

---Ahora mismo\ldots{} Me presentaré a las autoridades---repuso
Sarmiento dejando el cangilón y arropándose de nuevo entre las
sábanas,---y les diré: «Aquí tenéis, infames sicarios, al que os ha
hecho tanto daño; quitadme esta miserable vida; bebed mi sangre,
caníbales. Quiero compartir la inmortalidad del insigne Riego\ldots»

---¿Todo eso va a decir usted?\ldots{} Pues un poco perezosillo está mi
buen viejo para hacer y decir tantas cosas.

---¡Yo perezoso!---exclamó incorporando el anguloso busto y extendiendo
los brazos.---¡Venga al punto mi ropa!

Soledad le mostró ropa blanca limpia y planchada.

---He estado arriba---dijo.

---¿En mi casa?

---Sí; saqué la llave del bolsillo de usted, subí, revolví todo buscando
ropa mejor que la que usted tiene puesta\ldots{} pero no encontré nada.

---¡Cómo había de encontrar, alma de Dios, lo que no tengo! No se burle
usted de mi miseria\ldots{} Pero entendámonos, ¿qué ropa es esta que me
ofrece?

---Estaba en la casa\ldots{} son piezas desechadas, pero en buen uso.

---¡Ah! ya\ldots{} es ropa desechada del señor D. Salvador
Monsalud\ldots{} Pues mire usted, si fuera obsequio de otra persona lo
rehusaría; pero siendo de aquel noble patriota lo acepto. Conste que no
he pedido nada.

---De ropa exterior podríamos arreglarle algunas piezas decentes---dijo
Sola sonriendo,---siempre que usted tarde algunos días en marchar a la
inmortalidad.

---¡Tardar! Basta de bromas\ldots{} ¿Para qué quiero yo ropas bonitas?
¿Voy acaso a entrar en algún salón de baile o en los Elíseos Campos,
donde los justos se pasean envueltos en mantos de nubes?\ldots{}
Fígurese usted la falta que me hará a mí la buena ropa\ldots{}

---Puede que tarden en matarle a usted un mes o dos. Y si siguen estos
fríos no le vendrá mal una buena capa.

---Tanto como venir mal precisamente no\ldots{} ¿La tiene usted?

---La buscaremos.

---No, no es preciso\ldots{} Voy a levantarme.

Soledad se retiró y al poco rato apareció en la sala D. Patricio
completamente vestido. Sentose en el sofá, y contemplando a la joven con
bondadosa mirada, dijo así:

---Desde el tiempo de mi Refugio, no había dormido en una cama tan
buena\ldots{} ¡Ay! ¡ella era tan hacendosa, tan casera! Nuestro
domicilio estaba como un oro, y nuestro lecho nupcial podía haber
servido para que en él se revolcara un Rey\ldots{} ¡Pobre Refugio! Si me
vieras en mi actual miseria\ldots{} ¡Pobre Lucas, pobre hijo mío! Hoy tu
muerte es digna de envidia porque estás en la morada de los héroes y de
los elegidos; pero tu padre no tiene consuelo, ni puede vivir sin
verte\ldots{}

Derramó algunas lágrimas y por largo rato estuvo silencioso y cabizbajo,
dando muestras de verdadero dolor. Soledad, ocupada en sus quehaceres,
no se presentó a él sino a la hora de la comida.

---Supongo que no saldrá usted hasta después de comer---le dijo poniendo
la mesa.

---Saldré antes, ahora mismo, señora---dijo Sarmiento irguiéndose
súbitamente como un asta de bandera.---El peso de la vida me es
insoportable. Una voz secreta me grita: «Anda, corre\ldots» Todo mi ser
avanza en pos de la gloria que me está destinada.

---¡Cuánto mejor irá usted después de comer!\ldots{} ¿Es que desprecia
usted mi mesa?

---¡Oh! no señora, de ningún modo---replicó Sarmiento con
cortesía;---pero conste que sólo por acompañar a usted\ldots{}

Comieron tranquilamente, siendo de notar que el espiritual D. Patricio,
creyendo sin duda poco conveniente el aventurarse por los ideales
senderos con el estómago vacío, diose prisa a llenarlo de cuanto la mesa
sustentaba.

---¡Qué buena comida!---dijo permitiendo a su paladar aquel desliz de
sensualismo.---¡Qué bien hecho todo, y con cuánto primor presentado!
Solita, si usted se casa su marido de usted será el más feliz de los
hombres.

Al final de la comida, los ojos de D. Patricio brillaron con
resplandores de gozo, viendo una taza llena de negro licor.

---¡También café!\ldots{} ¡Oh! ¡cuánto tiempo hace que no pruebo este
delicioso líquido!\ldots{} el néctar de los dioses, el néctar de los
héroes\ldots{} Gracias, mil gracias por tan delicada fineza.

---Yo sabía que a usted le gusta mucho este brebaje.

---¡Gracias!\ldots{} ¡y qué bueno es!\ldots{} ¡qué aroma!

---Será el último que beba usted, porque en la cárcel no dan estas
golosinas.

---¿Y qué importa?---repuso el anciano con solemne acento.---¿Acaso
somos de alfeñique? Cuando un hombre se decide a escalar con gigantesco
pie el último círculo del cielo, ¿de qué vale el liviano placer de los
sentidos?

Dijo, y poniéndose el farolillo de fieltro que desempeñaba en su cabeza
las funciones propias de un sombrero, se dispuso a salir.

---Adiós, señora---murmuró,---gracias por sus atenciones, que no
esperaba en persona de quien soy encarnizado enemigo\ldots{} político.
Su papá de usted y yo nos aborrecimos y nos aborreceremos en la otra
vida\ldots{} Abur.

Salió precipitadamente hacia la puerta, mas no pudiendo abrirla, volvió
diciendo:

---La llave, la llave\ldots{}

Soledad rompió a reír.

---¡Y creía el muy tonto que iba a dejarle salir!---exclamó.---No
faltaba más. Eso querrían los chicos para divertirse. ¿Quiere usted
quitarse ese sombrero, hombre de Dios, y sentarse ahí y estarse
tranquilo?

---Señora, señora---dijo Sarmiento moviendo la cabeza y pateando
ligeramente en muestra de su decoroso enfado,---ábrame usted la puerta y
déjeme en paz, que cada uno va a su destino y el mío es\ldots{} el que
yo me sé.

---No abro.

---Señora, señorita, que yo soy hombre de poca paciencia. Ábrame la
puerta o reñimos de veras.

---Que no abro la puerta---repuso Sola, remedando el tonillo de
cantinela de su digno huésped.

---Basta de bromas, basta, repito---vociferó Sarmiento tomando el aire y
tono tragi-cómicos que empleaba al reprender a los alumnos.---Yo soy un
hombre formal\ldots{} De mí no se ríe nadie y menos una chiquilla
loca\ldots{} Ea, niña sin juicio, abra usted si no quiere saber quién es
Patricio Sarmiento.

---Un loco, un majadero, un vagabundo de las calles, a quien es preciso
recoger por caridad y encerrar por fuerza, para que no se degrade en las
calles como un pordiosero, haciendo el saltimbanquis y muriéndose de
miseria, ya que por el estado de su cabeza no puede morirse de
vergüenza.

Esto le dijo la muchacha con tanta seriedad y entereza, que por breve
rato estuvo el patriota aturdido y confuso.

---Aquí hay algo, aquí hay algún designio oculto que no puedo
comprender---afirmó el anciano,---pero que tiene por objeto, sí, tiene
por objeto impedir una resolución demasiado ruidosa y que quizás
perjudicaría al absolutismo.

Otra vez se echó a reír Sola de tan buena gana, que Sarmiento se
enfureció más.

---Por vida de la Chilindraina---gritó agitando sus brazos,---que si
usted no me da la llave, la tomaré yo donde quiera que se encuentre.

---Atrévase usted---dijo Soledad con festiva afectación de valor,
incorporándose en su asiento.---Mujer y sin fuerzas no temo a un
fantasmón como usted\ldots{} Quieto ahí, y cuidado con apurarme la
paciencia.

---Señora, no puedo creer sino que usted se ha vuelto loca---gruñó
Sarmiento con sarcasmo.---¡Querer detener a un hombre como yo! No sabe
usted las bromas que gasto. Repito que aquí hay una conjuración
infame\ldots{} ¡Oh! si es usted hija del conspirador más grande que han
abortado los despóticos infiernos\ldots{} ¡Ah, taimada muchachuela!
ahora me explico a qué venían los chocolatitos, la ropita blanca, el
buen cocido y mejor sopa\ldots{} ¡Quite usted allá! ¿Cree usted que con
eso se ablanda este bronce? ¿Cree usted que así se abate esta montaña?
¿Soy yo de mantequillas? Aunque fuera preciso derribar a puñetazos estas
paredes y arrancar con los dientes esos cerrojos del despotismo, yo lo
haría, yo\ldots{} porque he de ir a donde me llama mi hado feliz, y mi
hado, fatum que decían los antiguos, se ha de cumplir, y la víctima
preciosa inscrita en el eterno libro no puede faltar, ni la sangre
redentora puede dejar de derramarse, ni la libertad ha de quedarse sin
la víctima que necesita. De modo que saldré, pese a quien pese, aunque
tenga que emplear la fuerza contra miserables mujeres, lo que es
impropio de la nobleza de mi carácter.

---¿Se atreverá usted?

---Sí; deme usted la llave de esa puerta nefanda---contestó Sarmiento
con énfasis petulante que no tenía nada de temible,---o se arrepentirá
de su crimen\ldots{} porque esto es un crimen, sí señora\ldots{} ¡La
llave, la llave!

---Ahora lo veremos.

Corriendo afuera, prontamente volvió Sola con un palo de escoba, y
enarbolándole frente a D. Patricio, le hizo retroceder algunos pasos.

---Aquí están mis llaves, pícaro, vagabundo. O renuncia usted a salir, o
le rompo la cabeza.

---Señora---exclamó D. Patricio acorralado en un ángulo de la sala,---no
abuse usted de mi delicadeza\ldots{} de mi dignidad, que me impide poner
la férrea mano sobre una hembra\ldots{} ¡Esto es un ardid, pero qué
ardid!\ldots{} una trama verdaderamente absolutista.

---Siéntese usted---gritó Soledad conteniendo la risa y sin dejar el
argumento de caña.---Fuera el sombrero.

---Vaya, me siento y me descubro---repuso Sarmiento con la sumisión del
esclavo.---¿Qué más?

---¿Se compromete usted a no salir en quince días?

---Jamás, jamás, jamás. Antes la muerte---murmuró cerrando los
ojos.---Pegue usted.

---Esto es una broma---dijo Soledad arrojando el palo, sentándose junto
al anciano y poniéndole la mano amorosamente sobre el hombro.---¿Cómo
había yo de castigar al pobre viejecito demente y miserable que se pasa
la vida por las calles divirtiendo a los muchachos? Si no hay en el
mundo ser alguno más digno de lástima\ldots{} ¡Pobre viejecillo! Me he
propuesto hacer una buena obra de caridad y lo he de conseguir. Yo he de
traer a este infeliz a la razón. ¿Y cómo? Asistiéndole, cuidándole,
dándole de comer cositas buenas y sabrosas, arreglándole su ropa para
que esté decente y no tenga frío, proporcionándole todo lo necesario
para que no carezca de nada y tenga una vejez alegre y pacífica.

Estas palabras debieron de hacer ligera impresión en el espíritu del
viejo, porque moviendo la cabeza, se dejó acariciar y no dijo nada.

---Jesucristo nos manda hacer el bien a los pobres, cuidar a los
enfermos y aliviar a los menesterosos---añadió Sola acercando su
gracioso rostro a la rugosa efigie del vagabundo.---Y cuando esto se
hace con enemigos, el mérito es mayor, mucho mayor, y el placer de
hacerlo también aumenta. Recordando que este pobre iluso y fanático negó
un vaso de agua a mi padre en un trance terrible, más me alegro de
hacerle beneficios, sí, porque además yo sé que este desgraciado vejete
loco no es malo en realidad, ni carece de buen corazón, sino que por
causa del condenado fanatismo hizo aquella y otras maldades\ldots{} Por
consiguiente, papá Sarmiento, aquí estarás encerradito, comiendo bien y
cenando mejor, libre de chicos, de insultos, de atropellos, de hambre y
desnudez; aquí vivirás tranquilo, haciéndome compañía, porque yo soy
sola como mi nombre, y estaré sola por mucho tiempo, quizás toda la
vida\ldots{} ¿Quedamos en eso? Ya ves que te tuteo en señal de
parentesco y familiaridad.

---¡Ah! mujer melosa y liviana---dijo Sarmiento haciendo un esfuerzo de
energía, semejante al de los anacoretas cuando se veían en grande y
peligrosa tentación.---¡Quita allá! mi alma es demasiado fuerte para
sucumbir a tus pérfidos halagos.

---Esta noche cenaremos---dijo Soledad hablando como cuando se les
anuncia a los niños lo que han de comer.---Oye tú lo que cenaremos:
pollo chuletas, uvas\ldots{}

Iba contando por los dedos cada cosa, y haciendo gran pausa en cada
parada.

---Mañana---añadió,---voy a ocupar a mi ancianito en cosas útiles. Me ha
de trabajar para que pueda tratarle bien. Yo necesito reformar mi letra,
porque escribo patas de mosca y no tengo ortografía. El viejecillo me
dará lección todas las noches. Por el día le emplearé en algo que le
entretenga. Darele buenos libros\ldots{} nada de política\ldots{} y
cuando esté domesticado, le sacaré a paseo por las tardes.

A D. Patricio se le humedecieron los ojos. Difícil es saber lo que
pasaba en su alma.

---¿Y mi gloria, pero esa gloria que me está llamando?---dijo dando
fuerte porrazo en el brazo de la silla.---¡Vaya un modo de hacer
caridades, señora, quitándole a uno la inmortalidad, el lauro de oro que
se le tiene destinado!

D. Patricio dijo esto con una seriedad que hacía llorar y reír al mismo
tiempo.

---¿Qué gloria?---repuso Soledad.---No conozco sino la que Dios da a los
que se portan bien y cumplen sus mandamientos.

---¿Pero y esa víctima de quien necesita la libertad?

---La libertad no necesita víctimas, sino hombres que la sepan
entender\ldots{} Conque Sarmientillo, seremos amigos. De aquí no se
sale, mientras esa cabeza no esté buena.

Diole dos cariñosas palmadas en ella la encantadora joven, mientras el
insigne patriota exhalaba de su noble pecho un suspiro de a libra,
permítase la frase. ¿Era que hacía el sacrificio de su ideal sublime?
¿Era que pedía a su espíritu fuerzas para sobreponerse a seducción tan
terrible? No es fácil saberlo. Los próximos sucesos lo dirán.

---¡Ah! señora---exclamó tomando la mano de Sola,---no sabe usted bien
lo que hace. La historia, quizás, pedirá a usted cuentas de su acción
abominable, aunque declaro que es inspirada por un noble impulso de
caridad\ldots{} Engañosa Circe; no sabe usted bien qué clase de ímpetus
sojuzga y contiene al encerrarme; no sabe usted bien qué especie de
monstruo encarcela ni qué heroicas acciones se pierden con este hecho,
ni qué días gloriosos serán borrados de la serie del tiempo.

Dijo, y un rato después dormía la siesta.

\hypertarget{vii}{%
\chapter{VII}\label{vii}}

En los días sucesivos tuvo D. Patricio los mismos deseos de salir, si
bien, a excepción de una vez, no fueron tan ardientes; pero hubo gritos,
amenazas, volvió a funcionar el inocente palo y la carcelera a desplegar
las armas de su convincente piedad y de la graciosa prudencia que tan
buenos efectos produjera el primer día. Horas enteras pasaba el
vagabundo patriota, corriendo de un ángulo a otro de la sala, como
enjaulada bestia, deteniéndose a veces para oír los ruidos de la calle,
que a él le sonaban siempre como discursos, proclamas o himnos, y
poniéndose a cada rato el sombrero para salir. Este acto de cubrirse
primero y descubrirse después al caer en la cuenta de su encierro era
gracioso, y excitaba la risa de su amable guardiana. En la comida y cena
mostrábase más manso, y se ponía con cierto orgullo las prendas de
vestir que Sola le había arreglado. Desde la cabeza a los pies cubríase
con lo perteneciente al antiguo dueño de la casa, de cuya adaptación no
resultaba gran elegancia, a causa de la diferencia de talle y estatura.

Por las noches daba a Soledad lección de escritura, poniendo en ella
tanto cuidado la discípula como el maestro. Él particularmente mostraba
una prolijidad desusada, esmerándose en transmitir a su alumna sus altos
principios caligráficos y la primorosa maestría de ejecución que poseía
y de que estaba tan orgulloso.

---Desde que el mundo es mundo---decía observando los trazos hechos por
Soledad sobre el papel pautado,---no se han dado lecciones con tanto
esmero. Hanse reunido, para producir colosales efectos, la disposición
innata de la discípula y la destreza del maestro. Ahora bien, señora y
carcelera mía, la justicia y el agradecimiento piden que en pago de este
beneficio me conceda usted la libertad que es mi elemento, mi vida, mi
atmósfera.

---Bueno---respondió Sola,---cuando sepa escribir te abriré la puerta,
viejecillo bobo.

En los primeros días de Noviembre estuvo muy tranquilo, apenas dio
señales de persistir en su diabólica manía, y se le vio reír y aun
modular entre dientes alegres cancioncillas; pero el 7 del mismo mes
llegaron a su encierro, no se sabe cómo (sin duda por el aguador o la
indiscreta criada) nuevas del suplicio de Riego, y entonces la
imaginación mal contenida de D. Patricio perdió los estribos. Furioso y
desatinado, corrió por toda la casa gritando:

---¡Esperad, verdugos; que allá voy yo también! No será él solo\ldots{}
Esperad, hacedme un puesto en esa horca gloriosa\ldots{} ¡Maldito sea el
que quiera arrancarme mis legítimos laureles!

Soledad tuvo miedo; mas sobreponiéndose a todo, logró contenerle con no
poco trabajo y riesgo, porque Sarmiento no cedía como antes a la virtud
del palo, ni oía razones, ni respetaba a la que había logrado merced a
su paciencia y dulzura tan gran dominio sobre él. Pero al fin triunfaron
las buenas artes de la celestial joven, y Sarmiento, acorralado en la
sala, sin esperanzas de lograr su intento, tuvo que contentarse con
desahogar su espíritu poniéndose de rodillas y diciendo con voz sonora:

---¡Oh! tú, el héroe más grande que han visto los siglos, patriarca de
la libertad, contempla desde el cielo donde moras esta alma atribulada
que no puede romper las ligaduras que le impiden seguirte. Preso contra
todo fuero y razón; víctima de una intriga, me veo imposibilitado de
compartir tu martirio y con tu martirio tu galardón eterno. Y vosotros,
asesinos, venid aquí por mí si queréis. Gritaré hasta que mis voces
lleguen hasta vuestros perversos oídos. Soy Sarmiento, el digno
compañero de Riego, el único digno de morir con él; soy aquel Sarmiento,
cuya tonante elocuencia os ha confundido tantas veces, el que no os ha
ametrallado con balas sino con razones, el que ha destruido todos
vuestros sofismas con la artillería resonante de su palabra. Aquí estoy,
matad la lengua de la libertad, así como habéis matado el brazo. Vuestra
obra no está completa mientras yo viva, porque mientras yo viva se oirá
mi voz por todas partes diciendo lo que sois\ldots{} Venid por mí. La
horca está manca: falta en ella un cuerpo. No será efectivo el
sacrificio sin mí. ¿No me conocéis, ciegos? Soy Sarmiento, el famoso
Sarmiento, el dueño de esa lengua de acero que tanto os ha hecho
rabiar\ldots{} ¿No daríais algo por taparle la boca? Pues aquí le
tenéis\ldots{} Venid pronto\ldots{} El hombre terrible, la voz
destructora de tiranías callará para siempre.

Todo aquel día estuvo insufrible en tal manera que otra persona de menos
paciencia y sufrimiento que Solita le habría puesto en la calle,
dejándole que siguiera su glorioso destino; pero se fue calmando y un
sueño profundo durante la noche le puso en regular estado de despejo.
Habíale traído Soledad tabaco picado y librillos de papel para que se
entretuviese haciendo cigarrillos, y con esto y con limpiar la jaula de
un jilguero pasaba parte de la mañana. Sentándose después junto a la
huérfana mientras esta cosía, hablaban largo rato y agradablemente de
cosas diversas. Uno y otro contaban cosas pasadas: Sarmiento sus bodas,
la muerte de Refugio y la niñez de Lucas; Sola su desgraciado viaje al
reino de Valencia.

Continuaban las lecciones de escritura por las noches, y después leía el
anciano un libro de comedias antiguas que Sola trajo de la casa de
Cordero. Cuidaba muy bien de que en la vivienda no entrase papel ninguno
de política, y siempre que el anciano pedía noticias de los sucesos
públicos se le contestaba con una amonestación acompañada a veces de tal
cual suave pasagonzalo. Poco a poco iba acomodándose el buen viejo a tal
género de vida, y sus accesos de tristeza o de rabia eran menos
frecuentes cada día. Su carácter se suavizaba por grados, desapareciendo
de él lentamente las asperezas ocasionadas por un fanatismo brutal y la
irritación y acritud que en él produjera la gran enfermedad de la vida,
que es la miseria. A las ocupaciones no muy trabajosas de hacer
cigarrillos y cuidar el pájaro, añadió Soledad otras que entretenían más
al anciano. Como no carecía de habilidad de manos y había herramientas
en la casa, todos los muebles que tenían desperfectos y todas las sillas
que claudicaban recibieron compostura. En la cocina se pusieron vasares
nuevos de tablas, y después nunca faltaba una percha que asegurar, una
cortina que suspender, una lámpara que colgar, una lámina que mudar de
sitio o una madeja de algodón que devanar.

Llegó el invierno, y la sala se abrigaba todas las noches con hermoso
brasero de cisco bien pasado, en cuya tarima ponían los pies el
vagabundo, inclinándose sobre el rescoldo sin soltar de la mano la
badila. Era notable Don Patricio en el arte de arreglar el brasero, y se
preciaba de ello. Su conocimiento de la temperatura teníale muy
orgulloso, y cuando el brasero empezaba a desempeñar sus funciones, el
patriota extendía la mano como para palpar el aire y decía: «Ya
principia a tomar calor la habitación\ldots{} Va aumentando\ldots{} Un
poquito más y tendremos bastante. Yo no necesito más termómetro que la
yema del dedo meñique».

Más de una vez dijo, repitiendo una idea antigua:

---Desde el tiempo de mi Refugio no había visto yo un brasero tan bueno.

Por la mañana levantábase muy temprano y barría toda la casa,
cantorriando entre dientes. No habían pasado tres meses desde el primer
día de su encierro, cuando parecía haber adquirido conformidad casi
perfecta con su pacífica existencia. Sus ratos de mal humor eran muy
escasos, y por lo general las turbonadas cerebrales estallaban mientras
Solita estaba fuera, disipándose desde que volvía. Para el espíritu del
pobre anciano la huérfana era como un sol que lo vivificaba. Verla y
sentir efectos semejantes a los de la aparición de una luz en sitio
antes oscuro, era para él una misma cosa.

---Parece que no---decía para sí,---y le estoy tomando cariño a esa
muchachuela\ldots{} Quién lo había de decir, siendo como éramos enemigos
irreconciliables\ldots{} ¡Ah! Patricio, Patricio, si ahora te abrieran
la puerta de la casa y te echaran fuera, ¿abandonarías sin pena a esta
pobre huérfana que te mira como miraría la hija más cariñosa al padre
más desgraciado?

Un día, allá por Febrero o Marzo del 24, Sarmiento observó que Sol
estaba más triste que de ordinario. Atribuyolo a no haber recibido las
cartas que una vez al mes causábanla tanta alegría. El siguiente día lo
pasó la huérfana llorando de la mañana a la noche, lo que afligió
extremadamente al patriota. Por más que agotó Sarmiento todo el
repertorio, no muy grande por cierto, de sus trasnochados chistes, no
pudo sacarla de aquel estado, ni menos obligarla a revelar la causa de
su tristeza. Durante la cena, que casi fue de pura fórmula, Sarmiento
dijo:

---Pues si usted no se pone contenta, yo me volveré patriota como antes,
ea\ldots{} Así estaremos los dos iguales\ldots{} Me marcharé, sí señora,
estoy decidido a marcharme\ldots{} y lo siento, porque le he tomado a
usted mucho cariño, tanto cariño que\ldots{}

Se echó a llorar y tuvo que correr a ocultar sus lágrimas en la alcoba
inmediata.

Tres días después Sola salió muy de mañana, y volvió asaz contenta,
disipada la aflicción y con frescos colores en la cara, que eran como la
irradiación de su alegría, demasiado grande para contenerse en los
límites del alma. Tampoco entonces pudo el preceptor saber la causa de
tan rápido cambio; pero contentose con ver los efectos, y se puso a
bailar en medio de la sala, diciendo:

---¡Viva mi señora D.ª Solita, que ya está contenta, y yo también! No
más lágrimas, no más suspiros. Señora, si usted me lo permite me voy a
tomar la libertad de darle un abrazo.

Soledad aceptó con júbilo la idea, y el anciano la estrechó en sus
brazos con fuerza.

---¿Sabe usted---dijo limpiándose una lágrima,---que hoy se quedó la
llave en casa, y que habría podido escaparme si hubiera querido?

---¿Y por qué no saliste, viejecillo bobo?

---Porque no me ha dado la gana, vamos a ver\ldots{} porque estoy aquí
muy re-que-te-bien.

---¡Cosa más rara!---observó Soledad jovialmente.---Ya no quieres
salir\ldots{}

---No señora, no. Vea usted lo que son los gustos. Ya no quiero salir, y
no saldré sino cuando usted me arroje. Así de bóbilis bóbilis me he ido
acostumbrando a esta vida tonta, y\ldots{} No es que yo renuncie al
cumplimiento de mi destino; pero ya vendrá la ocasión, ¿no es verdad,
niña mía? Hay más días que longanizas, y tiempo hay, tiempo hay.

D. Patricio hacía con su mano derecha movimientos semejantes al fluctuar
de las olas, queriendo expresar de este modo el lento rodar del tiempo.

---Ahora, hija mía\ldots{} no se me enfade usted si le doy este nombre,
que me sale del corazón\ldots{} sí señor, porque usted se ha portado
conmigo como una hija, y es justo que yo sea un buen padre para
usted\ldots{} Pues decía, hija querida, que si usted no lo tiene a
mal\ldots{} me estorba en la boca el tratamiento de usted\ldots{} si no
te llamo de \emph{tú}, reviento\ldots{} Pues decía, hija de mi alma, que
ya es hora de que me des de comer.

Un momento después comían los dos alegremente, departiendo sobre cosas
placenteras, que no hay cosa que tan bien acompañe a un buen apetito
como la conversación amistosa y grata. Por la tarde, Soledad preparaba a
su viejo una bonita sorpresa.

---Como te vas portando bien---dijo,---y vas curándote de esas ideas
ridículas, voy a darte una golosina.

---¿Qué, hija de mi alma?---preguntó D. Patricio con la curiosidad de
los niños, cuando se les anuncia algún regalo.

---Una golosina\ldots{} ya la verás.

---¿Pero qué es? Estoy rabiando. ¿Café? Si lo tomo todos los
días\ldots{} ¿Un periódico?

---Ahora no hay periódicos.

---¡No hay periódicos!\ldots{} ¡Oh! vil absolutismo. ¿Conque no hay
prensa periódica?

Con un simple gesto apagó Soledad aquel chispazo de la hoguera que
parecía sofocada.

---¿Pues cuál es la golosina? Dímelo, angelito de mi corazón.

---La golosina es un paseo\ldots{} Esta tarde te llevaré a dar un
paseíto. Está hermosa la tarde.

---Bien, bravísimo, archi-bravísimo---exclamó el vagabundo arrojando su
sombrero al aire.---Estrenaré esa magnífica capa que me has arreglado.
Vamos pronto\ldots{} Mira, hija, que puede llover\ldots{}

---Si no hay nubes\ldots{}

---Puede ocurrir cualquier cosa.

---Nada puede ocurrir. Aguardaremos.

¡Qué hermoso día! Haces bien en sacarme a pasear. Mira que tengo ganitas
de saber lo que es el aire libre.

Salieron a las calles y de las calles al campo con vivo contento del
patriota que experimentó grandísimo gozo por tal expansión, y luego se
volvieron casa haciendo planes para nuevos paseos en los días sucesivos.
Así corría mansamente la vejez del buen maestro, que se asombraba de
encontrarse feliz sin saberlo, es decir, que miraba aquel maravilloso
cambio de sus sentimientos y de sus gustos sin acertar a darse cuenta de
él, como observa el vulgo los grandes fenómenos de la Naturaleza sin
explicárselos. Él pensaba a ratos en estas cosas, tratando de examinar
de cerca la metamorfosis de su alma, y decía:

---Es que yo soy todo corazón\ldots{} Esta joven me ha recogido, me ha
dado de comer y de vestir, me trata como a un padre. ¿Cómo no adorarla?
Patricio no es, no puede ser ingrato, y su corazón está dispuesto a
encenderse, a arder, a derretirse con los sentimientos más vivos, así
como los más delicados\ldots{} No es que en mí se hayan enfriado los
sublimes afectos de la patria, no, de ningún modo\ldots{} (Ponía mucho
empeño en convencerse a sí mismo de esta verdad). Soy lo mismo que era,
el mismo gran patriota, y persisto en mi noble idea de sacrificarme por
la libertad, ofreciendo mi sangre preciosísima\ldots{} Esto no puede
faltar, porque está escrito en el sacrosanto libro del destino\ldots{}
Es que Dios no quiere que sea tan pronto como yo esperaba. Vendrá el
sacrificio, el cruento martirio, los lauros, la inmortalidad; pero
vendrán en oportuna sazón y cuando suene la hora. A cada sublime momento
de la historia le suena su hora, y entonces \emph{consummatum
est}\ldots{} He aquí que Dios me depara un medio de corresponder a las
bondades de ese mi ángel tutelar. (Al decir esto se frotaba las manos en
señal de gozo). Es evidente que yo no tengo ningún bien mundano que
dejarle, pues carezco de fincas y de dinero, como no sea el que ella
misma me da. ¿Quiere decir esto que no pueda legarle algo? No\ldots{} le
dejaré un tesoro que vale más que todas las fincas y caudales, un tesoro
que es para beneficio del espíritu, no del cuerpo; le dejo, pues, mi
gloria, y así cuando la vean, dirán: «Esa es la compañera del gran
Sarmiento, esa es su hija adoptiva, la que le socorrió en sus últimos
días. ¡Loor eterno a la muchacha!»

Como se ve, el patriota no estaba curado, pero su enfermedad ofrecía
menos peligro, por haber entrado en un período que podremos llamar
médicamente de revulsión. El cariño que Sarmiento había tomado a su
favorecedora era síntoma muy favorable, y bien podía verse en aquello
más que la extirpación del fanatismo, una nueva dirección de él. No
mentía el infeliz al decir que era todo corazón. Capaz era este de los
sentimientos más delicados, así como de los más ardientes; bastaba que
las misteriosas corrientes de la vida consumasen su obra, llevando, como
las del cielo, la tempestad a otra región y zona distinta; pero el
pensamiento no podía obedecer a este cambio, porque había en la máquina
del cerebro Sarmentil una clavija rota que no podía y quizás no debía
componerse nunca.

También Sola había tomado mucho cariño al desvalido anciano. Le recogió
por caridad; propúsose realizar sin ayuda de nadie uno de esos
admirables actos de la voluntad, tanto más meritorios cuanto son más
oscuros, y sofocando resentimientos antiguos, indignos de la grandeza de
su alma, consumó valerosamente su obra bendita, digna de figurar en el
\emph{Flos Sanctorum}. Con el tiempo encendiose en su pecho un vivo
afecto hacia el mendigo abandonado, y esto, unido a los dulces placeres
que trae consigo el amar, fue el más digno premio de su noble acción.
Llegó a acostumbrarse de tal modo a la compañía del patriota vagabundo,
que la habría echado muy de menos si en cualquiera ocasión le faltara.

Un día Sarmiento le dijo:

---Querida Sola, hoy voy a pedirte un favor que creo no me has de
negar\ldots{} Es un caprichillo de anciano mimoso, un antojillo de
abuelo\ldots{} Si me lo niegas por cualquier pretexto, no me enfadaré,
pero me pondré muy triste.

---¿Qué es?

---Que me permitas darte un beso, hija mía. Hace muchos días que estoy
bregando con esta idea en la imaginación. Ya no puedo esperar más.

Soledad corrió hacia él, y D. Patricio la tuvo largo rato sobre las
rodillas prodigándole tiernas caricias.

---Por vida de la grandísima Chilindraina, niña de mi corazón---exclamó
hecho un mar de lágrimas,---si ahora me separan de ti, juro que me
moriría de pena. ¡Bendita seas tú mil veces!\ldots{} Bendita seas,
angelito mío, angelito mío, consuelo de mi vejez y heredera de mi
gloria\ldots{} ¡Toda, toda ella será para ti!

\hypertarget{viii}{%
\chapter{VIII}\label{viii}}

Parece que es urgente decir algo de la singular vida de esta solitaria
joven, e inquirir su conducta para deducir de su conducta sus proyectos.
Sin duda aquel espíritu valeroso, contrariado por lo que hemos convenido
en llamar suerte, no llevaba una existencia pasiva, entregándose a la
arbitraria fluctuación de los acontecimientos, sino que vivía en
actividad grande, aunque escondida, trabajando en obra misteriosa o
luchando con obstáculos tan oscuros como sus esfuerzos. Para afirmar
esto nos fundamos en conjeturas y en el conocimiento que de su carácter
tenemos; mas nada positivo afirmamos aún.

Nos consta, sí, que recibía cartas de cuyo contenido no enteraba a nadie
que a veces pasaba largas horas fuera de su casa; que escribía a altas
horas de la noche algún pliego y lo rompía después para volverlo a
escribir, repitiendo este trabajo cuatro o cinco veces, hasta quedar
medianamente satisfecha; que su semblante expresaba con fidelidad
pasmosa cambios muy bruscos en su espíritu, presentándola ya
sombríamente melancólica, ya festiva y dichosa; que no cesaba un punto
en su actividad, y cuando los asuntos de la casa le daban reposo,
discurría sobre mil temas concernientes a la faena del día venidero.

No le conocemos otras relaciones de amistad que las que tenía con la
familia de Cordero, la cual, a consecuencia de las calamidades de la
época, había ido a vivir en la misma casa, descendiendo algunos grados
en la escala social.

Ya es conocido de nuestros lectores el gran D. Benigno
Cordero\footnote{Véase \emph{Siete de Julio}.} comerciante de la subida
a Santa Cruz, hombre que se preciaba de ocupar dignamente su lugar en
todas las ocasiones, y que sabía ser bondadoso padre de familia, honrado
tendero, puntual amigo y también héroe glorioso, según lo que exigían
las circunstancias. Siendo tímido por naturaleza, mandole un día su
deber que fuese héroe y lo fue. Desgraciadamente no hay ninguna calle,
ni monumento, ni lápida, ni escultura, que recuerden a la posteridad su
nombre, símbolo de la inocencia; pero los veteranos del 7 de Julio saben
que hubo en Boteros un Leónidas de nariz picuda y roja como guindilla,
de gafas de oro y cuerpo más propio para sobresalir de la tabla de un
mostrador que para erguirse sobre el pedestal de gloria a quien llaman
campo de batalla.

La espantosa reacción absolutista, como furibunda riada que todo lo
arrastra, arrastró también al digno patricio, que en su tienda de
encajes había adquirido la idea de que los pueblos no se han hecho para
los Reyes. Esta idea se pagaba entonces con la cabeza, con la ruina o
con el destierro. Muchos perdieron la primera; infinito número buscó
refugio en el suelo extranjero. No era en verdad de los más delincuentes
el buen D. Benigno, porque no había ejercido cargo público del Estado
durante los \emph{tres llamados años}. Su crimen había sido pertenecer a
la Milicia y vestir su honroso uniforme sin tacha, con la circunstancia
agravante de haber cargado charreteras como representante de las más
altas jerarquías. Su sobrino D. Primitivo Cordero, que se había
significado altamente como correveidile político (el grado
inmediatamente inferior al de personaje), fue condenado a muerte, y tuvo
que huir al extranjero disfrazado de pastor, abandonando su comercio de
hierro a la autoridad que lo embargara; mas con D. Benigno fueron más
humanos, condenándole tan sólo a hacer una visita a Melilla o otra de
las cortes del África, en lo que recibió más disgusto que si le
destinaran a la horca.

Él, no obstante, diose su maña, y con ella, un poco de paciencia y un
puñado de onzas de oro (que entonces corrían de lo lindo para estos
arreglos), logró de la generosidad absolutista que se le comprendiera en
el Decreto de proscripción de Jerez, el cual mandaba que todos los que
se habían significado durante el malhadado imperio del Régimen famoso,
sin llegar al grado de culpabilidad necesario para incurrir en otras
penas mayores, no pudiesen hallarse a cinco leguas en contorno de los
puntos que recorría el Rey en su viaje, cerrándoseles además la Corte y
Sitios Reales dentro del radio de quince leguas. Cien mil individuos
fueron por este ridículo Decreto privados de la contemplación de la
Corte y Sitios Reales.

Abandonando su tienda y su familia partió Cordero a Zaragoza, donde fue
molestado y reducido a prisión por la feroz policía de aquella ciudad,
viéndose precisado a buscar en su bolsa nuevos argumentos contra la
famélica justicia de aquel bendito tiempo. Entretanto la familia vivía
en Madrid en la mayor aflicción, esperando todos los días nuevas tristes
de Zaragoza, atendiendo al comercio de encajes con el mayor celo y
economizando todo lo posible para ver de reparar los estragos hechos por
la política en el erario Corderil. Esta última razón fue la que les
impulsó a mudar de domicilio, pues una habitación arreglada cuadraba
admirablemente a su presupuesto más estirado ya que cuerda de ballesta.
Desde Noviembre se instalaron en el principal de la casa que ya
conocemos en la calle de la Emancipación Social según D. Patricio, y de
Coloreros según el Municipio. La tienda continuaba en el mismo sitio, a
mano derecha, como vamos a la plazuela de Santa Cruz y a la cárcel de
Villa.

Componían tan hidalga familia la señora de Cordero y tres hijos, hembra
la mayor y ya mujer, varones y pequeñuelos los otros dos. Acontecía en
aquel matrimonio un contraste que no deja de ser frecuente en este
extravagantísimo mundo, a saber, que si el esposo era diminuto y ligero,
la esposa era corpulenta y pesada. D.ª Robustiana podía coger a su
marido debajo del brazo como un falderillo y aun jugar con él a la
pelota si hubiera tenido tal antojo. Era avilesa y natural de Arenas de
San Pedro, de una familia nombrada Toros de Guisando, sin duda porque en
la antigüedad adquirió fama de dar hombres y mujeres de gran
corpulencia. Alta estatura, blancas y apretadas carnes, admirables
contornos y blanduras que estirando la tela pugnaban por mostrarse,
arrogante cabeza con ojos negros y cejas de terciopelo, manos gruesas,
semblante más correcto que agraciado, con cierto ceño no muy simpático y
algo de mohín avinagrado, boca demasiado pequeña con blancos dientes,
carrillos con demasiada carne, nariz castellana, escasísima agilidad en
los movimientos y mucha fuerza en los puños componían la persona de D.ª
Robustiana Toros de Guisando de Cordero.

De la incongrua pareja que formaba esta mujer con el benemérito
hombrecillo del arco de Boteros (pareja admirablemente acordada en el
orden moral) había nacido el día mismo de la batalla de Trafalgar (21 de
Octubre de 1805) Elena Cordero, en cuya persona se verificó una preciosa
amalgama del ser físico del padre y del de la madre. No salió a ella ni
a él, sino a los dos, realizando en sí uno de esos maravillosos términos
medios que sólo resultan bien en los divinos talleres de la Naturaleza.
No era Elena grande ni chica, ni gorda ni flaca, sino admirablemente
proporcionada en talle, color y estatura. Su cabeza era de las más
hermosas que pueden imaginarse, de tal modo que viéndola se comprendía
que el valor sereno de don Benigno no era el único parentesco de aquella
familia con la raza helénica. Su cara era la más bella que se ha visto
durante muchos años en toda la zona del comercio matritense desde
Majaderitos a la calle de Milaneses.

Quizás faltaba a su rostro aquella movilidad de la fisonomía española,
que es como el temblor de la luz jugando sobre la superficie del agua
agitada; quizás le faltaba esa facultad de hablar en silencio, lenguaje
admirable del cual son signos las pestañas, el iris negro que alumbra
como una luz, la sombra de la cara, el modo de mover el cuello, la
olvidada guedeja sobre la sien, el rumorcillo del pendiente que se mueve
ensartado en la oreja. Quizás Elena era demasiado selecta y tenía
demasiada corrección en su persona; mas no por esto dejaba de ser
acabado tipo de hermosura. Verdad es que miraba y reía, se peinaba y se
adornaba de una manera harto metódica; mas es posible que su corta edad
y su educación circunspecta la tuvieran en tal estado. Sus apasionados
alegaban para defenderla que era más bella su timidez inocente y aquella
perfección muñequil tan esmerada en sus limpios perfiles que la
desenvoltura y graciosa viveza de otras. Algunos la ponían resueltamente
en el orden de los juguetes finos; otros, en el de las imágenes de
iglesia. Pero, no obstante tal diversidad de opiniones, era generalmente
admirada, contribuyendo además la fama de su virtud a aumentar la
aureola de respeto y consideración que circundaba como nimbo luminoso a
toda la familia de Cordero.

De los dos varones poco puede decirse; eran pequeñuelos, traviesos y muy
devotos hermanos de la hermandad del Novillo. En aquel tiempo las
familias discurrían el modo de congraciarse con el bando dominante, y
uno de los sistemas más eficaces durante el trienio había sido vestir a
los niños de milicianos nacionales. Cambiadas radicalmente las cosas,
D.ª Robustiana, que quería estar en paz con la situación, siguió la
general moda vistiendo a los borregos de frailes. Los domingos Primitivo
y Segundito salían a la calle hechos unos padres priores que daban gozo.

La familia, que antes de la catástrofe de la Constitución era feliz y
vivía tranquila en su paz laboriosa, había caído en gran desaliento y
tristeza desde la proscripción del padre. Temían nuevas desgracias, y
como no veían en torno de sí más que cuadros de luto, ignominia,
venganzas horribles, asesinatos jurídicos, delaciones infames, horcas y
traición, no respiraban. Resuelta D.ª Robustiana a no ser en manera
alguna sospechosa a los ojos de la reacción, se esmeraba en variar los
vestidos domingueros de los niños, dándoles la forma y color de todas
las órdenes religiosas imaginables.

Compartían el tiempo hija y madre entre la tienda y la casa. En la
primera tenían un mancebo jovenzuelo que era muy despierto y les
prestaba no poca ayuda. En la casa vivían recogidamente, sin cultivar
amistades que podrían resultar peligrosas; huyendo de tratar mucha y
diversa gente; consagrando bastantes horas a rezar por la vuelta del
padre, y a imaginar medios pacíficos y legales para hacer su situación
menos aflictiva. La amistad más íntima y cariñosa que cultivaban era la
de Sola, que bajaba todos los días un par de horas lo menos, cuando no
subía Elena a hacerle compañía y ayudarla en sus quehaceres. La amistad
de la huérfana databa de 1822 en vida de su padre, que era paisano de
Cordero; pero se había aumentado y encendido más el afecto con la común
desgracia. Elena había sentido desde luego por ella una de esas vivas
inclinaciones de la primera juventud, que establecen lazos duraderos
para toda la vida, y a la cual daban aliciente la belleza moral de Sola
y aquel peculiar atractivo indefinible que sometía los corazones. La de
Cordero reconocía en ella una gran superioridad espiritual, que le
infundía respeto no inferior a su cariño, y subyugada por el misterioso
e invencible despotismo que ejerce a la callada la aristocracia moral,
se sometía a los pensamientos y al sentir de Sola, con la docilidad de
la niñez ante la edad madura. Siendo Sola poco menos joven que ella, se
le representaba, por la seriedad de sus consejos y su precoz
experiencia, como de edad mucho más alta. Hermana mayor antes que amiga,
la huérfana fue erigida en confesor, en consejero, y en depositaria de
los secretos del corazón de Elenita, porque el corazón de la muñeca tan
perfilada, metódica y acabadita tenía secretos.

Otra principal amistad de los Corderos era con la familia de los Romos,
y particularmente con Francisco Romo, jefe a la sazón del comercio
conocido con este nombre en la plazuela de Herradores. Las excelentes
relaciones mercantiles entre ambos tenderos fueron parte a anudar las de
la amistad, y durante la emigración de D. Benigno, Romo colmó de
atenciones y finezas a la familia, sirviéndoles al mismo tiempo de
amparo contra la reacción, por ser voluntario realista de los más
significados. D.ª Robustiana fiaba mucho en la amistad de aquel joven de
tanto poder entre las turbas realistas, y por nada del mundo la diera en
cambio de la de un príncipe. Creía tener en él fortísimo escudo contra
las brutalidades de la época y fiaba en que por mediación suya sería
restituido prontamente Cordero a la dulzura de su hogar.

---Hay que tener un poquito de paciencia---les decía Romo .---Se hace
todo lo que se puede para que el Sr.~D. Benigno vuelva a su casa; pero
no se podrá mucho, hasta que los liberales no estén sometidos. Figúrese
usted, señora D.ª Robustiana, que el Gobierno abre un poco la mano y
empieza a perdonar, a perdonar\ldots{} pues ya tiene usted la revolución
encima. No lo digo por el Sr. D. Benigno, que es un hombre de bien, sino
por esos pillos que están acechando nuestra debilidad para soltar las
riendas de su desvergüenza\ldots{} No se aflijan ustedes; que vamos a
dar una amnistía, una amnistía amplia, general, con excepción de todos
los pillos se entiende, y entonces o no soy quien soy, o D. Benigno será
comprendido en ella.

Con estas promesas se consolaba la familia; pero pasaban los meses y la
deseada amnistía no era más que una esperanza. En su lugar veíanse
nuevas proscripciones, encarcelamientos, la horca siempre en pie, la
venganza más cruel gobernando a la Nación, y la vida de los españoles
pendiente del capricho de un salvaje frailón o de fieros polizontes. Las
delaciones, como puñaladas recibidas en la oscuridad, traían en gran
consternación a la Corte. Desaparecían los ciudadanos sin que fuera
posible saber en qué calabozo habían caído. Las cárceles tragaban gente
como las tumbas en una epidemia. Nadie, libre hoy, podía estar seguro de
conservar la libertad mañana, porque la virtud más pura no podía estar
segura del golpe secreto, como no puede estarlo del miasma invisible.

Al fin, allá en Mayo del 24, vino la amnistía. Por ella se concedía
\emph{indulto y perdón general}; mas eran tantas las excepciones, que
antes que amnistía parecía el Decreto de una sangrienta burla. Se
perdonaba a todo el mundo y se exceptuaba después a todo el mundo. La
familia de Cordero, viendo que pasaban meses sin que el proscrito
volviese, examinaba detenidamente los 15 artículos de las excepciones,
por ver si D. Benigno podía ser comprendido en alguno de ellos; pero
Romo tranquilizaba a las dos señoras, diciéndoles:

---Eso corre de mi cuenta. D. Benigno vendrá; en caso que la
Superintendencia de policía tenga algún escrúpulo, le purificaremos
y\ldots{}

Santas Pascuas.

En efecto, una mañana del mes de Agosto hallábase D.ª Robustiana en el
mostrador midiendo algunas varas de puntilla, cuando vio que oscurecía
la luz de la puerta un objeto, un bulto, un cuerpo, un hombre, ¡D.
Benigno!\ldots{} Cayósele de las manos la vara de medir, y dando un
grito, extendió los macizos brazos por encima del mostrador. Cordero, a
quien la emoción tenía mudo y aturdido, no acertaba a abrazar a su
esposa convenientemente, hallándose por medio, como guión entre dos
letras, la dura tabla del mostrador, y le dio una cabezada en el pecho.
Entonces D.ª Robustiana cogiole con sus robustas manazas, tiró de él
suspendiéndole, y D. Benigno quedó de rodillas sobre el mostrador. Su
amante esposa le oprimía contra su delantera y así estuvieron largo rato
entre babas y sollozos, hasta que vencida por su sensibilidad que era
más fuerte que ella, cayó redonda al suelo la esposa, como un colchón
que recobra su posición natural. El mancebo corrió en busca de un
sangrador.

---Esto no es nada---dijo D. Benigno corriendo a desabrochar el corsé de
su esposa, que no era tarea de un momento.---Robustiana\ldots{}
Robustiana\ldots{} ¿Y qué tal? ¿Están buenos los niños? ¿Y
Elena?\ldots{} ¿En dónde están mis hijos?

El héroe de Boteros se bebía las lágrimas. No tardó la señora en volver
de su soponcio, y abrazándose nuevamente ambos, derramaron más lágrimas.
D. Benigno dijo entre pucheros:

---No más política, no más tonterías. La lección ha sido buena. Viva mi
familia, que es lo único que me interesa en el mundo.

Los amigos de las tiendas cercanas acudieron a felicitarle; el mancebo
corrió a traer a los chicos que ya habían ido a la escuela, y él, no
pudiendo refrenar su impaciente anhelo de ver a Elena, corrió a la calle
de Coloreros. Por el camino topaba a cada instante con amigos que le
daban la bienvenida, y como casi todos se empeñaban en manifestarle su
gozo con apretones de manos, abrazos y otras muestras de sensibilidad,
al feliz padre le consumía el desasosiego, y procurando desasirse de las
amistosas manos, exclamaba:

---Yo bueno\ldots{} estoy bien\ldots{} Hasta luego, señores\ldots{} Voy
a ver a mi hija querida.

Y penetrando en el portal, decía:

---Estará sola la pobrecita\ldots{} ¡qué alegría tendrá cuando me
vea!\ldots{} ¡Pobre ángel de mi vida!

Subió temblando y al acercarse a la puerta, y cuando alargaba la mano
para tomar el verde cordón de la campanilla, sintió una voz de hombre
que sonaba dentro de la casa. Era una voz agria, bronca, y pronunciaba
atropelladamente palabras que no podían entenderse bien desde la
escalera. Luego oyó D. Benigno la voz de su hija, expresándose con
agitación. Al buen ciudadano matritense se le heló la sangre en las
venas, a pesar de no haber formado aún idea concreta de lo que oía, y
llamó fuertemente con la campanilla y con los puños, gritando:

---Elena, hija mía, soy yo\ldots{} ¡tu padre!

\hypertarget{ix}{%
\chapter{IX}\label{ix}}

Aquella mañana, cuando D. Benigno estaba aún a dos leguas de la Corte,
Sola entraba en su casa después de una breve excursión por las tiendas.

---Querida niña---le dijo D. Patricio suspendiendo el barrido y
apoyándose en el palo de la escoba,---Elenita Cordero ha venido a
buscarte para que la acompañes un poco. Hoy está sola todo el día.

---¿Y no ha venido nadie más?

---Sí, ha venido también el caballero que estuvo ayer---repuso Sarmiento
poniendo ceño de disgusto.---Puede que él crea que yo no le conozco, a
pesar de las barbas de capuchino que gasta\ldots{} Si me parece que le
estoy viendo en la sala de armas del castillo\ldots{} Pero más vale
callar\ldots{} ¡Ah! se me olvidaba decirte que ha dejado un paquete para
ti.

---Sí\ldots{} hoy debía traerle---dijo Sola mirando a todos lados con
ansiedad.---¿En dónde lo ha dejado?

D. Patricio señaló una puerta, por la cual entró Sola corriendo. Fue
derecha a tomar un paquete que estaba sobre su cama. Pálida y con los
labios secos, le dio vueltas en sus manos temblorosas, buscando la
lazada del cordón que lo ataba. La veía, la tocaba sin acertar a
deshacerla, de tal modo se había vuelto torpe a causa de su gran
emoción.

En el paquete había cartas, muchas cartas; pero Sola buscó entre todas
una que debía de ser la principal, y hallada se puso a leerla. Por temor
a ser interrumpida, encerrose en la alcoba, y sentándose en un rincón,
arrojó todo su espíritu sobre un papel escrito. Allí estuvo largo rato
aleteando sobre él, como la mariposa sobre la flor, y tan pronto lloraba
como reía según los sentimientos expresados por aquella sombra de un ser
vivo a la cual se llama carta. Después miró uno por uno los sobrescritos
de las otras, y al hacer esto no mostraba mucho contento, antes bien
miedo. Además el paquete contenía una cajita pequeña con dinero en
monedas de oro. Contolas una por una y después lo guardó todo
cuidadosamente, a excepción de las cartas que no eran para ella. De
estas hizo un nuevo paquete que ocultó en su seno.

Púsose la mantilla para salir. D. Patricio vio pintado en el semblante
de la joven el gran gozo que la dominaba, y dando el último escobazo, se
dirigió a ella sonriendo. Sola se detuvo en la puerta, y mirando a su
protegido con expresión de lástima y de bondad, le dijo:

---Abuelo Sarmiento, si yo tuviera que marcharme para Inglaterra, ¿qué
harías tú, viejecillo bobo?

Y diciendo esto y sin dejar de mirarle bajó la escalera.

Inmóvil y perplejo D. Patricio, empuñando con su derecha mano el palo de
la escoba, y alzando la siniestra hasta la altura de su frente, parecía
la estatua erigida para conmemorar la petrificación del hombre.

Solita entró en casa de Cordero. Elena, que corrió a abrirle la puerta,
le dijo:

---Hace una hora que te espero\ldots{} quítate la mantilla\ldots{} estoy
sola con Reyes\ldots{} Tengo muchas cosas que contarte.

Entraron en la sala. En el centro de ella había una gran mesa llena de
puntillas que Elenita cosía unas con otras\ldots{}

---¿Pero no te quitas la mantilla?---repitió la de Cordero, emprendiendo
la obra interrumpida.---Hoy no sales de aquí en todo el día.

---Ahora mismo me voy---replicó Solita dejando escapar el contento por
los ojos.

---¡Vaya unas amigas!---dijo Elena manifestando en el tono su
tristeza.---¿A dónde vas ahora? Hay mucho calor.

---Tengo que hacer---repuso la huérfana tocándose el pecho para ver si
se le habían perdido las cartas.---Hay cosas que no se pueden dejar para
mañana.

---Es verdad---dijo la muñeca poniendo un hilo entre los dientes.---Si
yo pudiera dejar esto para la semana que entra lo dejaría\ldots{} Parece
que estás contenta\ldots{}

---Siempre no hemos de estar tristes.

---¿A dónde fuiste esta mañana?

---A comprar un vestido.

---¿Y ahora a dónde vas?

Sola vaciló un instante, porque era preciso mentir, y su inventiva no
era grande.

---A comprar otro---repuso al fin.

---¡Qué lujo!\ldots---exclamó Elena en son de amistosa burla.

---Qué quieres tú\ldots{} Es posible que tenga que salir de Madrid para
ir a\ldots{}

---¿A dónde?---preguntó la de Cordero con viveza.

---A\ldots{} otra parte---repuso la huérfana cayendo en la cuenta de que
había sido indiscreta.---Todavía no hay nada de cierto.

---De modo que me quedaré sola\ldots{} Pero muy satisfecha, muy oronda
estás hoy.

Sola se echó a reír. Este era el desahogo de un espíritu, a quien la
prudencia imponía silencio absoluto. Cuando una alegría tiene en la boca
de su cráter una gran piedra de discreción que la tapa y la ahoga, sólo
puede calmar su hervor riendo como los chicos y los tontos.

---Tú ríes y yo estoy desesperada---dijo la primorosa muñeca dando una
patadita en el suelo y rompiendo de un tirón el hilo que tenía entre los
dientes.---Solilla, anoche\ldots{} si supieras lo que pasó
anoche\ldots{}

---¿Qué?

Este monosílabo lo pronunció Sola distraída y maquinalmente, porque
tenía fija toda su atención en sí misma.

---¡Anoche!

---¡Anoche!\ldots---repitió la amiga volviéndose a tocar el pecho para
ver si había perdido las cartas.

---Todavía no se me ha quitado el miedo---dijo Elena suspendiendo su
obra para que ningún acto perjudicase a la expresión de lo que iba a
decir.---Antes ese hombre me era muy antipático; pero ahora\ldots{} te
juro que le aborrezco con toda mi alma.

---¡Pobrecito!\ldots{} no, no, quiero decir que le está bien
merecido\ldots{} El Sr.~Romo no cautivará a ninguna mujer. Sin ser feo,
es tal que parece más feo que los que lo son adrede.

---Justamente, has dicho la verdad\ldots{} El amigo de la casa se empeña
en quererme y en que yo le he de querer\ldots{} ¡Ay! amiga, tú tienes
razón en decir que ese hombre es malo\ldots{} Tiene en la cara una
cosa\ldots{} ¿qué es? Parece que va pasando por delante de él una
máscara horrible que le hace sombra en la cara. ¿No es así?

---Así mismo es, así---dijo Sola mirándose en un espejo que frente a
ella había y haciendo la observación de que no se encontraba tan poco
bonita como antes creyera.

---Pues ve a decirle a mamá que Francisco Romo no es la flor y nata de
los caballeros\ldots{} Todo lo bueno lo hace el Sr.~Romo\ldots{} «Ay,
cuándo vendrá el Sr.~de Romo para contarle lo que nos pasa!\ldots» «De
este apuro nadie más que el Sr.~de Romo puede sacarnos\ldots» «Si el
Sr.~de Romo no nos devuelve a tu padre, tenlo por perdido\ldots» Y dale
con el señor de Romo.

---¿Por qué no le cuentas a tu madre lo que te pasa?

---No puedo\ldots{} de ningún modo---dijo Elenita mostrando en su
hermoso rostro perfilado la imagen de la mayor confusión---¡Ay! ¡pobre
de mí qué desgraciada soy! ¡sí, la más desgraciada de todas las mujeres!

Diciendo esto, la figurita de porcelana cayó en una silla y llevó a los
ojos, acompañadas de un largo pañuelo, sus dos lindas manos. Alarmada
Solita acudió hacia ella y abrazola tiernamente, rogándole que explicase
aquellas desgracias tan enormes que abrumaban a la gentil doncella.

---Yo no puedo querer a Romo---afirmó esta sollozando,---porque es muy
feo, muy bastote y porque no me gusta\ldots{} ¿Qué culpa tengo yo de que
otro me haya parecido mejor? Dime tú si cualquier mujer a quien le
pongan delante a Francisco Romo y a Angelito Seudoquis puede dudar.

---¡Oh! no, de ningún modo. Angelito Seudoquis se ha de llevar la palma.

---Pues está claro---dijo Elena recibiendo gran consuelo con la
declaración de su amiga.---El pobre muchacho es muy bueno, de muy noble
familia, superior a nosotros, que somos tenderos; es muy honrado, muy
caballero, muy fino, muy valiente, según él mismo me ha dicho, y quiere
casarse conmigo.

---¿Y por qué no se ha de casar?

---Porque yo soy muy desgraciada\ldots{} no te rías\ldots{} la más
desgraciada de las mujeres---exclamó la doncella llorando como una
Magdalena,---y además porque he sido mala, muy mala y Dios me está
castigando.

---¿Qué has hecho?

---Escribí una carta a Angelito---dijo Elena observando atentamente su
pañuelo.

---Eso sí que no me lo habías dicho.

---Pensaba decírtelo hoy\ldots{} Le he escrito dos cartas.

---¿Dos?

---No\ldots{} me parece que han sido tres\ldots{} o quizás sean cuatro.

---¿Cuatro?

---La verdad, amiga de mi alma; le ha escrito ya cinco cartas.

---No digas más, porque si sigue la cuenta, va a resultar que le has
escrito cincuenta.

---Él pasaba todos los días por aquí\ldots{} yo sentía sus taconazos con
el rechinchín de las espuelas, y me daba mucha lástima\ldots{} No podía
menos de asomarme\ldots{} un día me mando con Reyes un papelito\ldots{}
En fin, en la última carta que le escribí\ldots{}

---Eso es, vamos a la última.

---En la última carta le decía muchas boberías\ldots{} Como él es tan
tierno y en las cartas pinta muchos corazones atravesados chorreando
sangre\ldots{}

---¿Tú también le pintaste corazones?

---No\ldots{} pero le decía que Romo es un animal\ldots{} porque está
celoso de Romo\ldots{} También le decía que con él (es decir, con
Angelito) o con nadie\ldots{} que me metería monja\ldots{} que el
sepulcro me era más dulce que casarme con otro\ldots{} En fin, esas
cosillas que se dicen\ldots{}

---¿Y nada más?

---Pero el caso es que la policía ha puesto preso a Angelito ayer por la
mañana.

---¡Jesús, mujer!

---Sí---añadió Elena más acongojada.---Le han puesto preso, porque
parece que un hermano suyo que estaba emigrado en Inglaterra ha venido
para conspirar. Le buscan, y como no pueden encontrarle, han cogido al
hermanito\ldots{} y\ldots{} y\ldots{}

Elena soltó un torrente de lágrimas y se deshizo en sollozos.

---¡Y\ldots{} y le van a ahorcar!---prosiguió con lastimeros ayes.

---No seas tonta, mujer---le dijo Sola, que se había puesto muy
pálida.---Y dices que por haber llegado su hermano\ldots{}

---Sí, un condenado masón que ha venido a armar revoluciones; y como no
le han podido coger\ldots{}

Soledad pasó de la sorpresa a la estupefacción más profunda.

---¡Esos infames polizontes son tan malos!\ldots---añadió la de
Cordero.---¿Qué culpa tiene el pobre Angelito?\ldots{} Él es liberal,
muy liberal; pero se hall decidido, así me lo ha dicho, a no desenvainar
su espada contra el Rey\ldots{} Ya sabes que es cadete. No, no, jamás
Angelito atentará a los derechos del Trono\ldots{} Pues volviendo a ese
vil Romo\ldots{} Ya sabes que él es amigo de los de la policía y de
Chaperón.

Sola no oía nada. Estaba absorta y no apartaba su mano del seno. Creía
sentir sobre él un peso colosal que la abrumaba.

---Como es amigo de la policía\ldots---añadió Elena.---Ya sabes que
registran a todos los presos\ldots{} Romo encontró en el bolsillo de
Angelito la última carta que le escribí\ldots{} ¿Conoces tú desgracia
semejante?

---¿Y qué?

---Que la tiene él\ldots{} Romo\ldots{} y me la enseñó anoche\ldots{} y
dice que se la va a enseñar a mamá y a papá cuando venga\ldots{} y dice
que cuando ahorquen a Angelito él le tirará de los pies\ldots{}

Un nuevo temporal deshecho de lágrimas, ayes y acongojados sollozos
interrumpió la narración de la inocente doncella.

---Yo me voy---dijo Sola levantándose bruscamente.

---No digas eso---repuso Elena tirando de la falda de su amiga.---Voy a
estar llorando todo el día: acompáñame.

---Después.

---Ahora.

---Tengo que salir---repitió Sola sin mirar a su amiga y oprimiéndose el
seno.

---¿Qué llevas ahí?---preguntó Elena tocando también y sintiendo rumor
de papeles.

---Nada, nada---repuso la huérfana con turbación.

¡Ah! pícara\ldots{} las cartas de tu novio\ldots{} y no me has querido
decir quién es\ldots{} y dices que no tienes ninguno; ¡y te escribe
tantos pliegos!\ldots{} Ahí llevas una resma\ldots{} No te vayas, por
amor de Dios.

Sola se despidió de su amiga con gran desasosiego.

---Parece que se te ha desvanecido la alegría---le dijo la muñeca.

---Adiós.

---Espera un rato.

---Ni un minuto\ldots{} Voy a ver a una persona\ldots{}

---¿No me has dicho que a comprar otro vestido?

---Es verdad\ldots{} volveré pronto. Adiós.

\hypertarget{x}{%
\chapter{X}\label{x}}

Elenita se quedó sola en la calma y silencio de la casa, apenas
interrumpidos por los cantorrios de la criada que chillaba en la cocina
acompañándose con el almirez.

La desgraciada joven, más infeliz que todas las mujeres nacidas, según
su propio parecer, reanudó su trabajo de coser puntillas, el cual, si no
ponía la artífice gran atención, había de salir muy imperfecto. No iba a
las mil maravillas la obra, por cuya razón Elena deshacía con frecuencia
lo hecho, tornando a empezar. A ratos aparecían entre la delicada tela
de araña algunas lágrimas que se quedaban temblando en los menudos hilos
negros, como insectos de diamantes cogidos en una red de pelo. A ratos
los suspiros de la obrera hacían moverse y volar los pedazos más
pequeños, que se remontaban en busca de otros climas. Frecuentemente se
picaba Elenita con la aguja, y muy a menudo se le enredaba el hilo entre
los dedos obligándola a detenerse y a perder los minutos. También solía
pasar la aguja con tanta presteza como si fuera puñal y con él tratara
de atravesar un corazón aborrecido.

Absorta en sus reflexiones, la niña no advirtió que habían llamado a la
puerta, que la criada acababa de abrir y que un hombre avanzaba con pie
muy quedo, al modo de ladrón, hacia la salita donde estaba el taller de
encajes. Así es que al sentir las palabras: «¿Se puede pasar?» la joven
dio un grito y saltó despavorida, cual si se viera en presencia de un
toro del Jarama.

---Váyase usted Sr.~de Romo, váyase usted---exclamó con terror,
refugiándose en un rincón de la estancia.---Mamá no está aquí\ldots{}
estoy sola\ldots{}

---Mejor---repuso Romo sonriendo y tratando de dar a su rostro y a su
ademán el aire no aprendido de la cortesía.---¿Me como yo a la gente?
¿Soy ladrón o facineroso?\ldots{} No: yo vengo aquí con móviles de
honradez\ldots{} ¿Podrán todos decir lo mismo?

---No, aquí no ha entrado nadie, nadie más que usted.

---Puesto que usted lo dice, Elenita, lo creo---dijo el hombre oscuro
tomando una silla.---Con la venia de usted me sentaré. Estoy muy
fatigado.

---¡Y se sienta!

---Sí, porque tenemos que hablar. Atención, Elenita, yo tengo la
desgracia de estar prendado de usted.

---Pues mire usted, yo tengo muchas desgracias, menos esa.

Romo contrajo su semblante, expresando sus afectos como los animales, de
una manera muy opaca, digámoslo así, por ser incapaz de hacerlo de otro
modo. No podía decirse si era el ruin despecho o la meritoria
resignación lo que determinaba aquel signo ilegible, que en él
reemplazaba a la clara sonrisa, señal genérica de la raza humana.

---Pues mire usted---dijo afectando candidez,---a otros les ha pasado lo
mismo, y al fin, a fuerza de paciencia, de buenas acciones y de finezas
se han hecho adorar de las que les menospreciaban.

---No conseguirá usted tal cosa de la hija de mi madre.

---Pues qué\ldots{} ¿tan feo soy?---preguntó Romo indicando que no tenía
la peor idea respecto a sus desgracias personales.

---No, no; es usted monísimo---dijo Elena con malicia,---pero yo estoy
por los feos\ldots{} ¿Quiere usted hacer una cosa que me agradará mucho?

---No tiene usted más que hablar, y obedeceré.

---Pues déjeme sola.

---Eso no\ldots---repuso frunciendo el ceño.---No pasa un hombre los
días y las noches oyendo leer sentencias de muerte, y acompañando negros
a la horca; no pasa un hombre, no, su vida entre lágrimas, suspiros,
sangre y cuerpos horribles que se zarandean en la soga, para venir un
rato en busca de goces puros junto a la que ama y verse despedido como
un perro.

---Pero yo, pobre de mí, ¿qué puedo remediar?---dijo Elena cruzando las
manos.

---Es terrible cosa---continuó el hombre-cárcel con hueco acento,---que
ni siquiera gratitud haya para mí.

---¿Gratitud?\ldots{} eso sí\ldots{} nosotros estamos muy agradecidos.

---Se compromete uno, se hace sospechoso a sus amigos, intercediendo
siempre por un don Benigno que mató a muchos guardias del Rey en el Arco
de Boteros; trabaja uno, se desvive, se desacredita, echa los
bofes\ldots{} y en pago\ldots{} vea usted\ldots{} ¡Rayo! hay una niña
que en nada estima los beneficios hechos a su familia\ldots{} ¿Qué le
importan a ella la buena opinión del favorecedor de su padre, su
honradez, su limpia fama en el comercio?\ldots{} Todo lo pospone al
morrioncillo, a las espuelas doradas y al bigotejo rubio de un mozalbete
que no tiene sobre qué caerse muerto, hijo y hermano de
conspiradores\ldots{}

Encendida como la grana, Elena se sentía cobarde. Pero si su valor
igualara a su indignación y sus tijeras pudieran cortar a un hombre como
cortaban un hilo, allí mismo dividiera en dos pedazos a Romo.

---Calle usted, cállese usted---exclamó sofocada.

---Y sin embargo---añadió el hombre opaco poniéndose más amarillo de lo
que comúnmente era,---soy bueno, tengo paciencia, me conformo, callo y
padezco\ldots{} Es verdad que tengo en mi poder un instrumento de
venganza\ldots{} pero no lo emplearé por razón de amor, no, lo emplearé
tan sólo por el decoro de esta familia a quien estimo tanto.

Elena tuvo un arranque de esos que se han visto alguna vez, muy pocas,
pero se han visto, en las palomas, en los corderos, en las liebres, en
las mariposas, en los seres más pacíficos y bondadosos, y pálida de ira,
con los labios secos y los puños cerrados, apostrofó al amigo de su
familia, gritando así:

---Usted es un malvado, y si yo supiera que algún día había de caer en
el pecado de quererle, ahora mismo me quitaría la vida para que no
pudiera llegar ese día. Usted es un tunante, hipócrita y falsario, y si
mi padre dice que no, yo diré que sí, y si mi padre y mi madre me mandan
que le quiera, yo les desobedeceré. Hágame usted todo el daño que guste,
pues todo lo que venga de usted lo desprecio, sí señor, lo desprecio,
como desprecio su persona toda, sí señor; su alma y su cuerpo, sí
señor\ldots{} Ahora, ¿quiere usted quitárseme de delante, o tendré que
llamar a la vecindad para que me ayude a echarle por la escalera abajo?

Al concluir su apóstrofe, la doncella se quedó sin fuerzas y cayó en una
silla; cayó blanda, fría, muerta como la ceniza del papel cuando ha
concluido la rápida llama. No tenía fuerzas para nada, ni aun para mirar
a su enemigo, a quien suponía levantado ya para matarla. Pero el
tenebroso Romo más que colérico parecía meditabundo, y miraba el suelo,
juzgando sin duda indigno de su perversidad grandiosa el conmoverse por
la flagelación de una mano blanca. Su resabio de mascullar se había
hecho más notable. Parecía estar rumiando un orujo amargo, del cual
había sacado ya el jugo de que nutría perpetuamente su bilis. Veíase el
movimiento de los músculos maxilares sobre el carrillo verdoso donde la
fuerte barba afeitada extendía su zona negruzca. Después miró a Elena de
un modo que si indicaba algo era una especie de paciencia feroz o el
aplazamiento de su ira. La córnea de sus ojos era amarilla como suele
verse en los hombres de la raza etiópica y su iris negro con azulados
cambiantes. Fijaba poco la vista, y raras veces miraba directamente como
no fuera al suelo. Creeríase que el suelo era un espejo, donde aquellos
ojos se recreaban viendo su polvorosa imagen.

Levantose pesadamente, y dando vueltas entre las manos al sombrero,
habló así:

---Y sin embargo, Elena, yo la adoro a usted\ldots{} Usted me insulta, y
yo repito que la adoro a usted\ldots{} Cada uno según su natural; el mío
es requemarme de amor\ldots{} ¡Rayo! si usted me quisiera, aunque no
fuese sino poquitín, me dejaría gobernar como un perro faldero\ldots{}
Sería usted la más feliz de las mujeres y yo el más feliz de los
hombres, porque la quiero a usted más que a mi vida.

Sus palabras veladas y huecas parecían salir de una mazmorra. Sin
embargo, hubo en el tono del hombre oscuro una inflexión que casi casi
podría creerse sentimental; pero esto pasó; fue cosa de brevísimo
instante, como la rápida y apenas perceptible desafinación de un buen
instrumento músico en buenas manos. Elena se echó a llorar.

---Ya ve usted que no puede ser---balbució.

---Ya veo que no puede ser---añadió Romo mirando a su espejo, es decir,
a los ladrillos.---Puede que sea un bien para usted. Mi corazón es
demasiado grande y negro\ldots{} Ama de una manera particular\ldots{}
tiene esquinas y picos\ldots{} de modo que no podrá querer sin hacer
daño\ldots{} A mí me llaman el hombre de bronce\ldots{} Adiós,
Elenita\ldots{} quedamos en que me resigno\ldots{} es decir, en que me
muero\ldots{} Usted me aborrece\ldots{} ¡Rayo! ¡con cuánta
razón!\ldots{} Es que soy malo, perverso y amenacé a usted con hacer
ahorcar a ese pobre pajarito de Seudoquis\ldots{} No lo haré\ldots{} si
le ahorcara, al fin le olvidaría usted, olvidándose también de
mí\ldots{} Eso sí que no me gusta. Es preciso que usted se acuerde de
este desgraciado alguna vez.

Elena no comprendiendo nada de tan incoherentes razones, vacilaba entre
la compasión y la repugnancia.

---Además yo había amenazado a usted con otra cosa---dijo Romo
retrocediendo después de dar dos pasos hacia la puerta.---Yo tengo una
carta, sí, aquí está\ldots{} en mi cartera la llevo siempre. Es una
esquela que usted escribió a esa lagartija. En ella dice que yo soy un
animal\ldots{} Bien: puede que sea verdad. Yo dije que iba a mostrar la
carta a su mamá de usted\ldots{} No, ¿a qué viene eso? Me repugnan las
intriguillas de comedia. ¡Yo enseñando cartas ajenas, en que me llaman
animal!\ldots{} Tome usted el papelejo y no hablemos más de eso.

Romo largó la mano con un papel arrugado, del cual se apoderó Elena,
guardándolo prontamente.

---Gracias---murmuró.

En aquel instante oyose la campanilla de la puerta, y la voz de D.
Benigno que gritaba:

---Hija mía, soy yo, tu padre.

Elena corrió a abrir, y el amoroso D. Benigno abrazó con frenesí a su
adorada hija, comiéndose a besos la linda cara, sonrosada de llorar.
También él lloraba como una mujer.

---¿Quién está aquí?\ldots{} ¿Con quién hablabas?---preguntó con viveza
el padre, luego que pasaron las primeras expansiones de su amor.

Al entrar en la sala, D. Benigno vio a Romo que iba a su encuentro
abriendo también los brazos.

---¡Ah! ¿estaba usted aquí\ldots{} era usted\ldots? ¡amigo mío!

---No esperábamos todavía al Sr.~Cordero---dijo Romo.---Desconfiaba de
que le soltaran a usted.

---¿Por qué llorabas, hija mía, antes de yo entrar?---dijo el patriota,
fijando en esto toda su atención.

---El Sr.~Romo---repuso Elena muy turbada, pero en situación de poder
disimularlo bien---acababa de entrar\ldots{}

---Yo creí que estaría aquí D.ª Robustiana---añadió el realista.

---Y me decía---prosiguió Elena,---me estaba diciendo que usted\ldots{}
pues, que no había esperanzas de que le soltaran a usted, padre.

---Eso me dijeron esta mañana en la Superintendencia; pero por lo visto
las órdenes que se dieron la semana pasada han hecho efecto.

---Venga acá el mejor de los amigos, venga acá---exclamó D. Benigno con
entusiasmo, abriendo los brazos para estrechar en ellos a su
salvador.---Otro abrazo\ldots{} y otro\ldots{} A usted debo mi libertad.
No sé cómo pagarle este beneficio\ldots{} Es como deber la vida\ldots{}
Venga otro abrazo\ldots{} ¡Haber dado tantos pasos para que no me
maltrataran en Zaragoza, haberme servido tan lealmente, tan
desinteresadamente! No, no se ve esto todos los días. Y es más admirable
en tiempos en que no hay amigo para amigo\ldots{} Yo liberal, usted
absolutista, y sin embargo, me ha librado de la horca. Gracias, mil
gracias, Sr.~D. Francisco Romo---añadió con emoción que brotaba como un
torrente de su alma honrada.---¡Bendita sea la memoria de su padre de
usted! Por ella juro que mi gratitud será tan duradera como mi vida.

Era la hora de comer; y cerrada la tienda, llegaron la señora, los niños
y el mancebo. Quiso D. Benigno que les acompañase Romo a la frugal mesa;
pero excusose el voluntario y partió, dejando a la hidalga familia
entregada a su felicidad. Elena no respiró fácilmente hasta que no vio
la casa libre de la desapacible lobreguez de aquel hombre.

\hypertarget{xi}{%
\chapter{XI}\label{xi}}

Dejamos a D. Patricio como aquellas \emph{estatuas vivas de hielo}, a
cuya mísera quietud y frialdad quedaban reducidas, según confesión
propia, las heroínas de las comedias tan duramente flageladas por
Moratín. El alma del insigne patriota había caído de improviso en
turbación muy honda, saliendo de aquel dulce estado de serenidad en que
ha tiempo vivía. Dudas, temores, desconsuelo y congoja le sobresaltaron
en invasión aterradora, sin que la presencia de Sola le aliviara, porque
la huérfana habló muy poco durante todo aquel día y no dijo nada de lo
que a nuestro anciano había quitado hasta la última sombra de sosiego.

Mas por la noche, cuando la joven se retiraba, volvió a decir la
terrible frase:

---Si yo me fuera a Inglaterra, ¿qué harías tú, viejecillo bobo?

D. Patricio no pudo hablar, porque su garganta era como de bronce y todo
el cuerpo se le quedó frío. No pudo dormir nada en toda la noche,
revolviendo en su mente sin cesar la terrible pregunta.

---¡Consagrar yo mi vida a una criatura como esta!\ldots---exclamaba en
su calenturiento insomnio:---¡amarla con todas las fuerzas del alma, ser
padre para ella, ser amigo, ser esclavo, y a lo mejor oír hablar de un
viaje a Inglaterra!\ldots{} ¡Ingrata, mil veces ingrata! Te ofrezco mi
gloria, trasmito a ti, bendiciéndote, los laureles que han de ornar mi
frente, y me abandonas!\ldots{} ¡Ah! Señor, Señor de todas las
cosas\ldots{} ¡La ocasión ha llegado! El momento de mi sacrificio
sublime está presente. No espero más. ¡Adiós, hija de mi corazón; adiós,
esperanza mía, a quien diputé por compañera de mi fama!\ldots{} Tú a
Inglaterra, yo a la inmortalidad\ldots{} ¿Pero a qué vas tú a
Inglaterra, grandísima loca? ¿a qué?\ldots{} Sepámoslo. ¡Ay! te llama el
amor de un hombre, no me lo niegues, de un hombre a quien amas más que a
mí, más que a tu padre, más que al abuelo Sarmiento\ldots{} ¡Por vida de
la Ch\ldots! Esto no lo puedo consentir, no mil veces\ldots{} yo tengo
mucho corazón\ldots{} Sola, Sola de mi vida\ldots{} ¿por qué me
abandonas? ¿por qué te vas, y dejas solo, pobre, miserable, a tu buen
viejecito que te adora como a los ángeles? ¿Qué he hecho yo? ¿Te he
faltado en algo? ¿No soy siempre tu perrillo obediente y callado que no
respiraría si su respiración te molestara?

Diciendo esto sus lágrimas regaban la almohada y las sábanas revueltas.

Al día siguiente notó que Sola estaba también muy triste y que había
llorado; pero no se atrevió a preguntarle nada.

Por la noche luego que cenaron, Sola, después de larga pausa de
meditación, durante la cual su amigo la miraba como se mira a un oráculo
que va a romper a hablar, dijo simplemente:

---Abuelito Sarmiento; tengo que decirte una cosa.

D. Patricio sintió que su corazón bailaba como una peonza.

---Pues abuelito Sarmiento---añadió Sola, mostrando que le era muy
difícil decir lo que decía,---yo, la verdad\ldots{} ¡tengo una pena, una
pena tan grande!\ldots{} Si pudiera llevarte conmigo te llevaría, pero
me es imposible, me es absolutamente imposible. Me han mandado ir sola,
enteramente sola.

D. Patricio dejó caer su cabeza sobre el pecho, y le pareció que todo él
caía, como un viejo roble abatido por el huracán. Lanzó un gemido como
los que exhala la vida al arrancar del mundo su raíz y huir.

---Es preciso tener resignación---dijo Sola poniéndole la mano en el
hombro.---Tú, en realidad, no eres hombre de mucha fe, porque con esas
doctrinas de la libertad los hombres de hoy pierden el temor de Dios, y
principiando por aborrecer a los curas acaban por olvidarse de Dios y de
la Virgen.

---Yo creo en Dios---murmuró Sarmiento.---Ya ves que he ido a misa desde
que tú me lo has mandado.

---Sí, no dudo que creerás, pero no tan vivamente como se debe creer,
sobre todo cuando una desgracia nos cae encima---dijo la huérfana con
enérgica expresión.---Ahora que vamos a separarnos, es preciso que mi
viejecito tenga la entereza cristiana que es propia de su edad y de su
buen juicio\ldots{} porque su juicio es bueno, y felizmente ya no se
acuerda de aquellas glorias, laureles, sacrificios, inmortalidades, que
le hacían tan divertido para los granujas de las calles.

---Yo no he renunciado ni debo renunciar a mi destino---repuso el
anciano humildemente.

---Ni aun por mí\ldots{}

---Por ti tal vez; pero si te vas\ldots{}

---Si me voy, será para volver---replicó Sola con ternura\ldots---yo
confío en que el abuelito Sarmiento será razonable, será juicioso. Si el
abuelito en vez de hacer lo que le mando, se entrega otra vez a la vida
vagabunda, y vuelve a ser el hazme reír de los holgazanes, tendré
grandísima pena. Pues qué, ¿no hay en el mundo y en Madrid otras
personas caritativas que puede cuidar de ti como he cuidado yo? Hay, sí,
personas llenas de abnegación y de amor de Dios, las cuales hacen esto
mismo por oficio, abuelito, y consagran su vida a cuidar de los pobres
ancianos desvalidos, de los pobres enfermos y de los niños huérfanos. A
estas personas confiaré a mi pobre viejecillo bobo, para que me le
cuiden hasta que yo vuelva.

D. Patricio que había empezado a hacer pucheros, rompió a llorar con
amargura.

---Soledad, hija de mi alma\ldots---exclamó.---Ya comprendo lo que
quieres decirme. Tu intención es ponerme en un asilo\ldots{} ¡Lo dices y
no tiemblas!

Después, variando de tono súbitamente, porque variaba de idea, ahuecó la
voz, alzó la mano y dijo:

---¡Y crees tú que a un hombre como este se le mete en un hospicio!
Sola, Sola, piénsalo bien. Tú has olvidado qué clase de mortal es este
que tienes en tu casa. ¡Y me crees capaz de aceptar esa vida oscura, sin
gloria y sin ti, sin ti y sin gloria! ¡ay! los dos polos de mi
existencia\ldots{} Mira, niña de mi alma, para que comprendas cuánto te
quiero y cómo has conquistado mi gran corazón, te diré que yo no soy el
que era, que si mis ideas no han variado han variado mis acciones y mi
conducta.

Y luego con una seriedad que hizo sonreír a Sola en medio de su pena, se
expresó así:

---Es evidente\ldots{} porque esto es evidente como la luz del
día\ldots{} que yo estoy destinado a coronarme de gloria, a adornar mi
frente de rayos esplendorosos sacrificándome por la libertad,
ofreciéndome como víctima expiatoria en el altar de la patria, como el
insigne general, mi compañero de martirio, que me espera en la mansión
de los justos, allá donde las virtudes y el heroísmo tienen eterno y
solemne premio\ldots{} Pues bien, es tanto lo que te quiero, que por tu
cariño he ido dejando pasar días y días y días y hasta meses sin cumplir
esto que ya no es para mí una predestinación tan sólo, sino un deber
sagrado. ¿Me entiendes?

Soledad le pasó la mano por la cabeza, incitándole a que no siguiese
tocando aquel tema.

---Por ti, sólo por ti\ldots---prosiguió el viejo.---¡Me da tanta pena
dejarte!\ldots{} Así es que me digo: «Tiempo habrá, Señor»\ldots{}
¿Creerás que aquí en tu compañía se me han pasado semanas enteras sin
acordarme de semejante cosa?\ldots{} Hay más todavía: yo estaba
dispuesto a hacer un sacrificio mayor\ldots{} ¿te espantas? que es el de
sacrificarte mi sacrificio, ¿no lo entiendes?\ldots{} Sí, poner a tus
pies mi propia gloria, mi corona de estrellas\ldots{} Sí, chiquilla, yo
estaba dispuesto a no separarme jamás de ti y a no pensar más en la
política\ldots{} ni en Riego, ni en la libertad\ldots{} ¡Oh! hija mía,
tú no puedes comprender la inmensidad de tal sacrificio. Por él juzgarás
de la inmensidad del amor que te tengo. ¡Y cuando yo renuncio por ti a
lo que es mi propia vida, a mi idea santa, gloriosa, augusta, tú me
abandonas, me echas a un lado como mueble inútil, me mandas a un
hospicio y te vas!\ldots{}

Soledad veía crecer y tomar proporciones aquel problema de la separación
que le causaba tanta pena. Su alma no era capaz de arrepentirse del bien
que había hecho al desvalido anciano; pero deploraba que por los
misteriosos designios de Dios, la caridad que hiciera algunos meses
antes, le trajese ahora aquel conflicto que empezaba a surgir en su
cristiano corazón.

---El Señor nos iluminará---dijo, remitiendo su cuita al que ya la había
salvado de grandes peligros.---Confío en que Dios nos indicará el mejor
camino. Si tú le pidieras con fervor, como yo lo hago, luz, fuerzas,
paciencia y fe, sobre todo fe\ldots{}

---Yo le pediré todo lo que tú quieras, hija de mi alma; yo tendré
fe\ldots{} Dices que tengo poca; pues tendremos mucha. Me has contagiado
de tantas cosas, que no dudo he de adquirir la fe que tú, sólo con
mirarme, me estás infundiendo.

---Para adquirir ese tesoro---dijo Sola con cierto entusiasmo,---no
basta mirarme a mí ni que yo te mire a ti, abuelo; es preciso pedirlo a
Dios y pedírselo con ardiente deseo de poseer su gracia, abriendo en par
en par las puertas del corazón para que entre; es preciso que nuestra
sensibilidad y nuestro pensamiento se junten para alimentar ese fuego
que pedimos y que al fin se nos ha de dar. Teniendo ese tesoro, todo se
consigue, fuerzas para soportar la desgracia, valor para acometer los
peligros, bondad para hacer bien a nuestros enemigos, conformidad y
esperanza, que son las muletas de la vida para todos los que cojeamos en
ella.

---Pues yo haré que mi sensibilidad y mi pensamiento se encaminen a
Dios, niña mía---replicó el vagabundo participando del entusiasmo de su
favorecedora.---Haré todo lo que mandas.

---Y tendrás fe.

---Tendremos fe\ldots{} sí; venga fe.

---Con ella resolveremos todas las cuestiones---dijo Sola acariciando el
flaco cuello de su amigo.---Ahora, abuelito, es preciso que nos
recojamos. Es tarde.

---Como tú quieras. Para los que no duermen, como yo, nunca es tarde ni
temprano.

---Es preciso dormir.

---¿Duermes tú?

---Toda la noche.

---Me parece que me engañas\ldots{} En fin, buenas noches. ¿Sabes lo que
voy a hacer si me desvelo? Pues voy a rezar, a rezar fervorosamente como
en mis tiempos juveniles, como rezábamos Refugio y yo cuando teníamos
contrariedades, alguna deudilla que no podíamos pagar, alguna enfermedad
de nuestro adorado Lucas\ldots{} Ello es que siempre salíamos bien de
todo.

---A rezar, sí; pero con el corazón, sin dejar de hacerlo con los
labios.

---Adiós, ángel de mi guarda---dijo Sarmiento besándola en la
frente.---Hasta mañana, que seguiremos tratando estas cosas.

Retirose Soledad, y el anciano se fue a su cuarto y se acostó,
durmiéndose prontamente; mas tuvo la poca suerte de despertar al poco
tiempo sobresaltado, nervioso, con el cerebro ardiendo.

---Ea, ya estamos desvelados---dijo dando vueltas en su cama, que había
sido para él durante diez meses un lecho de rosas.---Voy a poner por
obra lo que me mandó la niña; voy a rezar.

Disponiendo devotamente su espíritu para el piadoso ejercicio, rezó todo
lo rezable, desde las oraciones elementales del dogma católico hasta la
que en distintas épocas ha inventado la piedad para dar pasto al
insaciable fervor de los siglos. Sarmiento rezó a Dios, a la Virgen, a
los Santos que antaño habían sido sus abogados, sin olvidar a los que
fueron procuradores de Refugio, mientras esta, desterrada en el mundo,
les necesitara.

Mas a pesar de esto, el anciano no advirtió que entrara gran porción de
calma en su espíritu, antes al contrario, sentíase más irritado, más
inquieto con propensiones a la furia y a protestar contra su malhadada
suerte. Como llegara un instante en que no pudo permanecer en el
abrasado lecho, levantose en la oscuridad y se vistió a toda prisa sin
estar seguro de ponerse la ropa al derecho. Sentía impulsos de salir
gritando por toda la casa y de llamar a Sola y echarle en cara la
crueldad de su conducta y decirle: «Ven acá, loca, ¿quién es el infame
que te llama desde Inglaterra?\ldots{} ¿Qué vas tú a hacer a
Inglaterra?\ldots{} ¡Ah! Es un noviazgo lo que te llama. Y si es
noviazgo, ¡vive Dios! ¿quién es ese monstruo? Dímelo, dime su nombre, y
correré allá y le arrancaré las entrañas».

En la sala distinguió débil claridad, por lo que supuso que había luz en
el cuarto de su amiga. Paso a paso, avanzando como los ladrones,
dirigiose allá; empujó suavemente la puerta, pasó a un gabinete,
deslizose como una sombra extendiendo las manos para tocar los objetos
que pudieran estorbarle el paso.

La puerta de la alcoba estaba entreabierta; había luz dentro, pero no se
oía el más leve rumor. Alargando el cuello Sarmiento vio a Sola dormida
junto a una mesa en la cual había papeles y tintero.

---Estaba escribiendo---pensó,---y se ha dormido. Veremos a quién.

Entró en la alcoba, andando despacio, quedamente y con mucho cuidado
para no hacer ruido. Su rostro anhelante, su cuerpo tembloroso, sus ojos
ávidos y saltones dábanle aspecto de fantasma, y si la joven despertase
en aquel momento se llenaría de terror al verle. Estaba profundamente
dormida, con la cabeza apoyada en el respaldo del sillón y ligeramente
inclinada. Delante tenía una carta a medio escribir, y otra muy larga y
de letra extraña que parecía ser la que estaba contestando.

---Yo conozco esa letra---pensó Sarmiento, devorando con los ojos el
escrito, que estaba apoyado en un libro puesto de canto a manera de
atril.

Conteniendo su respiración, el vagabundo examinó el pliego, que, abierto
por el centro, no presentaba ni el principio ni el fin. Después fijó los
ojos en la carta a medias escrita por Sola. D. Patricio miraba y fruncía
el ceño apretando las mandíbulas. Tenía un aspecto tal de ferocidad
aviesa, que si él mismo pudiera verse tuviera miedo de sí mismo. No
tardó mucho en satisfacer su curiosidad; pero esta era tan intensa, que
después de leer una vez leyó la segunda. Después de la tercera no estaba
tampoco satisfecho; mas temiendo que la joven despertara, se retiró como
había venido. Al llegar a su cuarto se dejó caer en la cama, y dando un
gran suspiro exclamó para sí:

---¡Bien lo decía yo: los emigrados!\ldots{}

\hypertarget{xii}{%
\chapter{XII}\label{xii}}

Muy gozoso y satisfecho estaba D. Benigno Cordero con el suceso de su
vuelta a la patria y al hogar querido, y resuelto a que el durase mucho
el contento, hacía propósito firmísimo de no tornar a mezclarse en
política, ni vestir uniforme, ni menos hacer heroicidades en Boteros ni
en otro arco alguno. Verdad es que guardaba en su pecho cual tesoro
riquísimo o como los restos queridos de una persona amada que se
depositan en secreta urna, las mismas aficiones políticas a que debió su
destierro. Eso sí: antes creyera que el sol salía de noche que dejar de
ver en la libertad, en el progreso y en la soberanía del pueblo, la
felicidad de las Naciones. Mas era preciso poner una losa sobre estas
cosas y D. Benigno la puso.

---Desde hoy---dijo,---Benigno Cordero no es más que un comerciante de
encajes. No adulará al absolutismo, no dirá una sola palabra en favor de
suyo; pero no, ya no tocará más el pito constitucional ni la flauta de
la milicia. A Segura llevan preso. Yo tengo ideas, sí, ideas firmes,
pero tengo hijos. Es posible, es casi seguro que otros, que también
tienen mis ideas, las hagan triunfar; pero mis hijos por nadie serán
cuidados si se quedan sin padre. Atrás las doctrinas por ahora, y
adelante los muchachos. Ahora silencio, paz, retraimiento
absoluto\ldots{} cabeza baja y pico cerrado\ldots{} pero ¡ay! alma mía,
allá recogida en ti misma y sin que te oigan los oídos de la propia
carne en que estás encerrada, no ceses de gritar: «¡Viva, viva y mil
veces viva la señora libertad!»

Los muchos amigos del ex-jefe de milicianos le felicitaban cordialmente,
y sus parroquianos así como sus compañeros de comercio recibieron gran
contento al verle. Como era tan generoso, y tenía un natural por demás
expansivo, antejósele, ocho días después del de su vuelta, obsequiar a
los amigos con un modesto banquete dedicado a grabar en la memoria de
todos el fausto evento de su liberación; pero D.ª Robustiana, cuyo
sentido práctico igualaba al peso de su cuerpo, le quitó de la cabeza la
idea de aquella manifestación dispendiosa, arguyéndole así:

---Desgraciadamente no estamos para fiestas. Acuérdate del dinero que
has gastado en congraciarte con esos pillos; que tiempo hay de dar
banquetes. Mañana domingo, 28 de Agosto, haremos para la cena un
extraordinario de poca monta, y convidaremos a Romo, al Sr.~de Pipaón
que también nos ha servido, y a Sola. Total: tres convidados. Basta,
hombre, basta. Tiempo hay de echar la casa por la ventana, y no faltará
un motivo para ello ni tampoco elementos, ¿me entiendes?\ldots{} porque
si siguen los frailes reponiendo la ropa del altar, no faltará venta de
encaje blanco para todo el año que corre.

D. Benigno, como siempre, armonizó su opinión con la de su cara esposa,
y a consecuencia de tan dulce avenencia, al día siguiente la cocina de
los Corderos despedía inusitado aroma de ricas especias, el cual
anunciaba a toda la vecindad la presencia de un extraordinario. A la
hora de la cena resplandecía el comedor con la luz de dos quinqués,
colocados en contrapuestos sitios, y alrededor de la mesa se sentaron el
Sr.~de Pipaón, Sola y los de Cordero, sin excluir los niños, que
ocupaban un extremo junto a su hermana. El puesto más preeminente entre
los de convite estaba vacío, lo cual causaba gran disgusto a D. Benigno.

---¿Por qué no habrá venido Romo?---decía.---Es particular: no le hemos
visto desde el día de mi llegada. ¿Estará enojado con nosotros?

Se esperó un rato; pero viendo que no parecía, dio principio el
banquete.

El digno anfitrión estaba intranquilo por aquella ausencia de su amigo,
y a cada instante miraba a su esposa como para preguntarle qué opinaba
ella de tan extraño caso. Ya D.ª Robustiana había dicho:

---Estará muy ocupado en la Comandancia de Voluntarios. Se le han
mandado tres avisos al anochecer. Ustedes no saben bien la calma que
gasta el Sr.~de Romo. Otra noche le convidamos a cenar y se descolgó
aquí a las diez de la noche.

La señora presidía majestuosamente la mesa y gobernaba con mucha
destreza aquella maniobra de los banquetes antiguos, consistente en
estar pasando platos de aquí para allí y de derecha a izquierda, como si
los convidados en vez de reunirse para comer lo hicieran para jugar al
juego de \emph{sopla} y \emph{vivo de lo doy}. Descollaba su hermoso
busto por encima de la blanca mesa, a manera de un trono forrado en tela
oscura sobre el cual colocaran su cabeza como provisionalmente y
mientras parecía el cuello perdido. Con la estrechez del ajuste, los
abundantes dones que en ella acumuló sin tasa Natura formaban un
circuito de tanta extensión que una mosca (esto puede asegurarse y lo
certificaron testigos oculares), una mosca, decimos, que salió de uno de
los brazos para ir al otro pasando por delante, tardó no se sabe cuánto
tiempo en dar la vuelta y llegar a su destino.

En el otro extremo de la mesa Primitivo y Segundo, que por ser día de
fiesta vestían de padres provinciales de la orden dominica, estaban bajo
la vigilancia de Soledad y Elena respectivamente, las cuales no podían
probar bocado, entretenidas en enseñar a los frailescos ángeles el modo
de comer; y mientras el uno se rociaba con sopa los hábitos, llevábase
el otro la cuchara a los ojos, sin cesar de pedir, chillar y hacer
comentos varios sobre cuanto desde la fuente a sus platos pasaba.

Pipaón, cuyo apetito parecía crecer a medida que había menos motivos
aparentes para ello, amenizaba con sus chistes la comida. Estaba
elegantísimo, como de costumbre, el ingenioso cortesano, ataviado con su
calzón de punto blanco, su levita polonesa de mangas jamonadas, su
corbata metálica destinada a anticipar la idea de la muerte en garrote,
por si acaso algún día era el individuo condenado a ella. Revueltos los
cabellos con artístico desorden, parecía su cabeza una escoba, en lo
cual cumplía a maravilla con los preceptos de la moda corriente. ¡Oh!
era aquel un señor muy bondadoso y sencillo, que lo mismo se sentaba a
la mesa del rico que a la del pobre, con tal que en ellas hubiera buenos
manjares que comer; y sin dar privadamente excesiva importancia a las
ideas políticas, lo mismo fraternizaba con el negro que con el blanco,
siempre que ni el uno ni el otro le estorbasen en su prodigioso medro.
Menos alegre que su comensal a caus de la ausencia de Romo, D. Benigno
conversaba con chispa y donaire, volviendo con graciosa movilidad el
rostro hacia Pipaón, hacia su esposa y hacia la silla vacía donde se
echaba de menos la torva figura del voluntario realista; y ¡cosa
singular! aquella silla donde no se sentaba el hombre oscuro, tenía
cierto aspecto lúgubre. Romo no estaba allí, y sin embargo parecía que
estaba.

Esquivando entrar en el tema político a que la verbosidad importuna y
mareante de Pipaón quería llevarle, D. Benigno dijo:

---Ya he manifestado cuál es mi propósito. Y qué, Sr.~D. Juan, ¿cree
usted que me será difícil cumplirlo? De ningún modo. Los que necesitan
de la política para vivir, porque si no hay bullanga no comen,
difícilmente aceptarán esta oscura vida privada que es mi delicia. Quite
usted a los intrigantes la política y será como si les cortaran las
manos a los rateros o los pies a las bailarinas. ¿Digo mal? Hoy con este
partido, mañana con el otro, ello es que siempre se les ve a
flote\ldots{}

A D. Benigno se le cayó del tenedor un pedazo de calabacín que en él
tenía, aguardando a que la boca callase para entrar. La causa de tan
inesperado siniestro fue que D.ª Robustiana le estaba tocando el codo,
primero suavemente y después con fuerza, para que su marido cayese en la
cuenta de que estaba haciendo la sátira de Pipaón.

---Verdad es que no todos los que se ocupan de política son así---dijo
el honrado comerciante pinchando de nuevo la hortaliza,---ya se
comprende; pero ni a unos ni a otros quiero parecerme. La vida privada
es hoy mi sueño de oro\ldots{} No quiere decir que en lo íntimo de mi
alma no exista siempre\ldots{} pero dejemos esto. Puede uno llevar en su
fuero interno el fardo que más le acomode, sin necesidad de ponerse una
etiqueta en la frente\ldots{} esto es claro como el agua. No hay
necesidad de meter ruido. En la vida privada puede tener el buen
ciudadano mil ocasiones de realizar fines patrióticos y de servir a la
patria. ¿Cómo? Cumpliendo lealmente esa multitud de pequeños esfuerzos
que en conjunto reclaman tanta energía como cualquier acto de heroísmo;
así lo ha dicho Juan Jacobo Rous\ldots{} tente lengüita. Dejemos a ese
caballero en su casa, pues hay palabras que ahorcan\ldots{} Yo me
concreto a lo siguiente: vea usted mi plan, Sr. de Pipaón.

Antes que el plan de D. Benigno, merecía la atención de Bragas una lonja
de ternera, cuyo especioso condimento bastaba a acreditar la ciencia
culinaria de la señora de Cordero.

---Muy bien, Sr.~D. Benigno---gruñó Pipaón engullendo.---Su plan de
usted me parece muy bien asado\ldots{} No, no, quiero decir que la
ternera está muy bien asada y que su plan de usted es excelente,
sabrosísimo, es decir, atinadísimo.

---Mi plan es el siguiente: Yo trabajo todo el día con excepción de los
domingos; yo cumplo con los preceptos de Nuestra Santa Madre la Iglesia
oyendo misa, confesando y comulgando como se me manda; yo cumplo
asimismo mis obligaciones comerciales; yo no debo un cuarto a nadie; yo
educo a mis hijos; yo pago mis contribuciones puntualmente; yo obedezco
todas las leyes, decretos, bandos y órdenes de la autoridad; yo hago a
los pobres la limosna que mi fortuna me permite; yo no hablo mal de
nadie, ni siquiera del Gobierno; yo sirvo a los amigos en lo que puedo;
yo no conspiro; yo celebro mucho que todos vivan bien y estén contentos;
en suma, yo quiero ser la más ordenada, puntual y exacta clavija de esta
gran máquina que se llama la patria, para que no dé por mi causa el más
ligero tropezón\ldots{} ¿Qué tal? ¿Me he explicado bien?

Conversación tan interesante hubo de interrumpirse porque uno de los
chicos tuvo la ocurrencia de derramar sobre su hábito toda la salsa que
había en el plato, mientras el otro barraqueaba como un ternero porque
no le permitían comer con las manos. Calmada la agitación al otro
extremo de la mesa, D. Benigno continuó:

---Siempre ha sido mi norma de conducta\ldots{} Segundito,
cuidado\ldots{} ocupar el puesto que me señalaban las circunstancias. He
sido y soy esclavo de mi deber\ldots{} Primitivo, que te estoy mirando;
¿cómo se coge el tenedor?\ldots{} Un día las circunstancias me dijeron:
«es preciso que seas valiente» y fuí valiente. Heridas tengo que darán
razón de ello. Hoy me dicen las circunstancias: «es preciso que seas
pacífico» y pacífico soy\ldots{} Niños ¿me enfado?\ldots{} Mi conciencia
está tranquila con tan juicioso plan de conducta; a mi conciencia
obedezco y nada más.

En esto sonaron fuertes campanillazos en la puerta de la casa. D.ª
Robustiana se sobresaltó.

---A buena hora viene ese señor\ldots{} cuando ya estamos en los
postres---dijo D. Benigno.---De seguro es Romo.

---No, no llama él de ese modo---observó la señora, poniendo atención
para oír en el momento que la criada abría.

---Puede que sea Romo---indicó Pipaón dirigiendo sus dedos en
persecución de una pera que rodaba por el mantel.

---Son dos señores, dos hombres---dijo la criada entrando en el
comedor.---Preguntan por el amo.

---Allá voy---dijo Cordero levantándose.

---Que esperen---manifestó D.ª Robustiana con mal humor.---¡Que siempre
te has de levantar de la mesa\ldots!

D. Benigno salió con la servilleta sujeta al cuello. En la sala encontró
dos hombres desconocidos.

---Una luz, Reyes---gritó a la criada.

La claridad de la vela que trajo la moza permitió al honrado patriota
distinguir bien las fisonomías. Creía reconocer aquellas caras. Ninguna
de las dos despertaba grandes simpatías, y en cuanto a los cuerpos eran
de lo más sospechoso que puede imaginarse.

---¿Es usted D. Benigno Cordero?---le preguntó uno de ellos secamente.

---Para lo que ustedes gusten mandar. ¿Qué quieren ustedes?

---Que venga usted con nosotros.

---¿A dónde?

---¡Toma\ldots{} a la cárcel!---exclamó el individuo esgrimiendo su
bastoncillo y admirado de que no se hubiera comprendido el objeto de tan
grata visita.

D. Benigno se quedó aturdido\ldots{} Creía soñar\ldots{} estaba lelo.

---¡A la cárcel!---murmuró.

---Y pronto. Tenemos que hacer\ldots{}

---A la cárcel\ldots---dijo otra vez Cordero, como el delirante que
repite un tema.---Yo\ldots{} ¿por qué?\ldots{} ¿yo\ldots? ¿han dicho que
a la cárcel\ldots?

---Sí señor, a la cárcel\ldots{} nosotros no tenemos que
explicar\ldots{} No somos jueces---graznó el polizonte con desenfado y
altanería, consecuente con el tono general de los pillastres que se
dedican a perseguir a la gente honrada.

---Aguarden ustedes un momento---dijo Cordero sin saber lo que
decía.---Voy\ldots{} Les diré a ustedes\ldots{}

Dio varias vueltas, tropezó con una puerta. Parecía un hombre que ha
perdido la cabeza y la está buscando. Sin propósito deliberado, fue al
comedor, entró. Su esposa y su hija perdieron el color al ver su cara,
que era la cara de un muerto.

---Son dos caballeros---murmuró Cordero con voz trémula.---Dos
amigos\ldots{} No hay que asustarse\ldots{} Tengo que salir con
ellos\ldots{} Pipaón amigo, salga usted a ver qué es eso\ldots{} mi
sombrero, ¿en dónde está mi sombrero?

Dio una vuelta alrededor de la mesa y salió otra vez. Sin duda había
perdido el juicio.

---Conque dicen ustedes que\ldots{} ¡a la cárcel!\ldots{} ¿y se podrá
saber\ldots?

---Si usted no viene pronto---dijo el polizonte con ira,---llamaremos a
los voluntarios que están abajo.

El otro bribón había encendido un cigarro y fumaba mirando los cuadros
de la sala.

---Pues vamos. Esto es una equivocación---dijo el comerciante recobrando
un poco su entereza.

---¿Pero su hija de usted no se presenta?---preguntó el primer esbirro.

---¡Mi hija!

---¡Sí señor, su hija!---exclamó el mismo abriendo las manos y mostrando
en dos abanicos de carne sus diez dedos sucios, negros, nudosos y con
las yemas amarillas por el uso del cigarro de papel.

---¿Y para qué tiene que presentarse mi hija?

---¿Pues qué?\ldots{} ¿No le dije que su hija tiene que venir también a
la cárcel?

---Usted no me ha dicho nada, y si me lo hubiera dicho, no lo habría
creído---afirmó Cordero sintiendo que su corazón se oprimía.

---Vea usted este papel---dijo el funcionario mostrando un
volante.---Benigno Cordero y su hija Elena Cordero.

---¡Mi hija!---exclamó D. Benigno, lanzando un gemido de dolor.---¿Pues
qué ha hecho mi hija?

---¡Eh! que suban los voluntarios. Así despacharemos pronto.

D. Benigno se había vuelto idiota. No se movía. Pipaón que había oído
algo desde la puerta, se acercó diciendo:

---Esto ha de ser alguna equivocación de la Superintendencia.

Al verle los de la policía le hicieron una reverencia, como suele
usarlas la infame adulación cuando quiere parecerse a la cortesía.

---¿No es usted el que llaman Mala Mosca? ¿No me debe usted su
destino?---preguntó Pipaón.

---Sí señor---repuso el infame mostrando tras los replegados labios una
dentadura que parecía un muladar.---Soy el mismo, para servir al señor
de Pipaón.

---A ver la orden.

Pipaón leyó a punto que entraban en la sala, sobrecogidas de terror, las
tres mujeres y los dos frailecitos y la criada.

---Nada, nada, esto debe de ser un \emph{quid pro quo}---dijo Bragas con
disgusto evidente;---pero es preciso obedecer la orden. Desde este
momento empezaré a dar los pasos convenientes\ldots{}

Los de Cordero se miraron unos a otros. Se oía la respiración. En aquel
instante de congoja y pavura, Elena fue la que tuvo más valor, y
haciendo frente a la situación exclamó:

---¿Yo también he de ir presa? Pues vamos. No tengo miedo.

---¡Hija de mi alma!---gritó D.ª Robustiana abrazándola con furor.---No
te separarás de mí. Si a los dos os llevan presos, yo voy también a la
cárcel y me llevo a los niños.

---Con usted no va nada, señora---dijo el polizonte.---El señor mayor y
la niña son los que han de ir\ldots{} Conque andando.

Arrojose como una hiena la señora sobre aquel hombre, y de seguro lo
habría pasado mal el funcionario de la Superintendencia si D.ª
Robustiana, en el momento de clavar las manos en la verrugosa cara de su
presa no hubiera quedado sin sentido, presa de un breve síncope.
Acudieron todos a ella, y el policía gritó, poniéndose rojo y horrible:

---¡Al demonio con la vieja!\ldots{} Vamos al momento, o que suban los
voluntarios. No podemos perder el tiempo con estos remilgos.

D. Benigno, cuyo espíritu estaba templado para hacer frente a las
situaciones más terribles, elevose sobre aquella tribulación, como el
sol sobre la bruma, e iluminando la lúgubre escena con un rayo de
heroísmo que a todos les dejó absortos, gritó:

---Vamos, vamos a la cárcel. Ni mi hija ni yo temblamos. La inocencia no
tiene miedo, cobardes sayones\ldots{} Vamos a la cárcel, al patíbulo, a
donde queráis, canallas, mil veces canallas\ldots{} Yo había vuelto la
espalda a la libertad, y la libertad me llama\ldots{} ¡Allá voy, ideal
divino; aquí estoy; adelante!\ldots{} Vamos, miserables, abandono a mi
esposa, a mis hijos. Todo se queda aquí\ldots{} Tan miserables sois
vosotros como Calomarde que os manda. Vamos a la cárcel, y ¡Viva la
Constitución!

Salió bizarra y noblemente, lleno de entusiasmo y valor, rodeando con su
brazo el cuello de Elena, que al heroico arrojo de su padre respondió
diciendo también:---«¡Viva la Constitución!»

Al salir encargó a Soledad que cuidase de su madre y de sus hermanos.
Algo más pensaba decir; pero los sayones no la dejaron. El compañero de
Mala Mosca se quedó para registrar la vivienda.

\hypertarget{xiii}{%
\chapter{XIII}\label{xiii}}

Al día siguiente, después de las doce, entró Pipaón en la casa, muy
agitado y sudoroso, como hombre que ha subido en pocas horas todas las
escaleras de las oficinas de Madrid. Halló a D.ª Robustiana en
lamentable estado. Yacía la atribulada señora en cama, y desde la noche
anterior, lejos de calmarse sus ataques nerviosos, se habían exacerbado
a causa de la inquebrantable resistencia a tomar alimento. Cuando Pipaón
entró, no podía dar un paso en la estancia, porque estaba casi a oscuras
con objeto de que la luz no molestase a la señora; mas por los suspiros
que oía se fue guiando hasta que dio con el lecho y pudo distinguir a
Solita, sentada junto a este sin apartar la atención ni un punto de su
infeliz amiga.

El ilustre cortesano de 1815 se sentó, cuidando de exhalar también un
gran suspiro para que no se dudase de la autenticidad de su pena, y
después de enterarse con mucha solicitud del estado de la paciente, dijo
así:

---Señora, he visto a Chaperón.

D.ª Robustiana contestó con un quejido lastimero.

---Señora---añadió Bragas,---he visto a Aymerich, jefe de los
voluntarios realistas.

Respondiole otro quejido seguido de sollozos.

---Señora, he visto a Ugarte, a Cea Bermúdez, a varios individuos de la
Junta Secreta de Estado, a dos individuos de la Comisión Militar.

No obtuvo respuesta.

---Señora, he visto a Calomarde, he hablado con él: estaba almorzando,
me hizo pasar, le dije lo que ocurría, contestome que viese a D. José
Manuel de Arjona. También es amigo mío: hemos hablado largamente. Voy a
enterar a usted con toda claridad de la verdadera situación en que
estamos, situación grave, señora, ¿a qué ocultarlo? pero no desesperada.
Yo creo que se deben pintar los sucesos tales cuales son, porque de nada
valdría desfigurarlos, ¿estamos en eso? Pues bien: juzgue usted por sí
misma.

D.ª Robustiana parecía hallarse en estado de no poder juzgar nada por sí
misma; pero el impávido Pipaón habló así:

---Ya sabrá usted que ha habido audaces tentativas revolucionarias en
Tarifa, Almería y otros pueblos de la costa del Mediodía. Esos tunantes
salieron de Gibraltar. El desembarco les salió mal. Gracias a la
vigilancia d las autoridades, tan grande iniquidad quedó frustrada. De
hoy a mañana, señora, serán fusilados en Tarifa trescientos de esos
pillos.

Pipaón notó que el lecho se estremecía.

---Ya sabrá usted---añadió,---que por el Decreto del 20 se condena a
muerte a todos los que por cualquier medio pretendan restablecer el
sistema representativo. Aquí será fusilado Gregorio Iglesias, un
chicuelo de 18 años que intentó unirse a los revolucionarios del
Mediodía. También parece que hoy ha sido condenado a muerte otro
jovenzuelo, Tomás Franco, por haber proferido expresiones contra la vida
de Su Majestad\ldots{} En La Coruña ha sido preciso sentar la mano.
Muchos de los sentenciados a la última pena han sido ejecutados ya;
otros se han suicidado con opio o abriéndose las venas\ldots{} En fin,
señora, esto es muy triste, pero usted comprenderá que el Gobierno,
viéndose acosado por esos infames demagogos negros sedientos de
desorden, necesita mostrarse riguroso, pero muy riguroso\ldots{} Yo
pregunto a todas las personas imparciales y juiciosas: «¿En vista de lo
que pasa, puede el Gobierno ser benigno?»

El discreto amigo no recibió contestación ni de la enferma, ni de
Soledad, pero lo mismo que si la recibiera, prosiguió diciendo:

---Exactamente: no puede ser benigno. Los frailes, los obispos, todos
los absolutistas de temple incitan al Gobierno a extirpar la negrería;
los voluntarios realistas que son más levantiscos e indomables que la
malhadada Milicia Nacional de marras, amenazan con sublevarse si no se
les da todos los días sangre de liberales, horcas y más horcas. ¿Y qué
se ha de hacer? Sobre ellos, sobre esa base poderosa se asienta el
edificio del absolutismo y ¡ay de todo esto el día en que los
voluntarios de la fe pasen del descontento a la sedición y de las
palabras a los hechos! Por lo dicho, comprenderá usted que en la
situación actual, cuando alguno, aunque sea inocente, tiene la desgracia
de caer en la cárcel, no es fácil sacarle de ella a dos tirones\ldots{}

D.ª Robustiana exhaló la mitad de su alma en un gemido.

---No quiere esto decir que D. Benigno y su niña no puedan
salir---añadió Bragas;---saldrán, sí señora, saldrán con la ayuda de
Dios. Pero es difícil, sumamente difícil, ¿por qué he de decir otra
cosa? ¿Por qué he de engañar a usted con ilusiones que luego serían
amargos desengaños? Ahora examinemos el delito de nuestros queridos
presos.

Al oír esto, estremeciose otra vez el lecho, y oyéronse sílabas
torpemente articuladas.

---El Sr.~D. Benigno y su hija han sido delatados, no se sabe por quién
ni es fácil saberlo. Por más que yo he tratado de averiguarlo, no me ha
sido posible. Acúsanles de\ldots{} pero vamos por partes, para mayor
claridad. Parece que Elenita tiene un novio llamado Ángel
Seudoquis\ldots{}

---¡Es mentira, es una infame impostura!---exclamó D.ª Robustiana,
sobreponiéndose a su estado nervioso.---Mi hija no tiene novio.

---Ángel Seudoquis---prosiguió Pipaón, dando poca importancia a la
negativa de la enferma,---hermano de D. Rafael Seudoquis, militar sin
purificar, degradado y aun creo que condenado a muerte por varios
horrorosos crímenes de Estado. Según consta en la delación, Rafael
Seudoquis, que ha venido de Inglaterra con órdenes de los
revolucionarios para hacer una tentativa, se valió de su hermano Ángel,
novio de la niña, para ponerse en comunicación con D. Benigno, el cual
parecía tener encargo de ayudarle\ldots{}

---¡Qué horrible maquinación! ¡Qué tejido de infames mentiras!---murmuró
D.ª Robustiana ahogando los sollozos.---Sola, tú que nos conoces y sabes
quién entra y sale en nuestra casa, ¿no te horrorizas de oír tales
calumnias?

Soledad no contestó nada. Tenía un nudo en la garganta.

En la delación consta también---prosiguió el amigo de la casa,---que
Rafael Seudoquis entró dos veces seguidas disfrazado\ldots{} grandes
barbas, aspecto fiero\ldots{} yo no le conozco. Ello es que le vieron
entrar. Guardábale el bulto su hermano, paseando en la calle. Consta que
Elena recibía de él papeles que luego entregaba a D. Benigno, y constan
otras estupendas cosas que no recuerdo en este momento.

---Consta que los jueces y delatores son un enjambre de miserables
bandidos---afirmó doña Robustiana con ira, incorporándose.---Sola, ¡por
Dios santo! tú que nos conoces, di a ese hombre que se engaña, porque
también él, con ser nuestro amigo, parece dar crédito a tales patrañas.

---Yo ni afirmo ni niego\ldots{} poco a poco---manifestó Pipaón,
conservándose en aquel saludable justo medio que le había llevado a
considerables alturas burocráticas.---El Sr.~D. Benigno y su hija pueden
ser inocentes y pueden no serlo: de un modo o de otro es el Sr.~Cordero
un excelente amigo, a quien debo servir y serviré con todas mis fuerzas.

Levantose. La enferma, acometida por una convulsión, desplomose sobre
las almohadas.

---Ánimo, señora---dijo con la frialdad del médico que pone recetas en
el momento de la muerte.---Usted me conoce y sabe que haré cuanto de mí
dependa. El caso es grave, gravísimo: ignoro hasta dónde puede llegar mi
influencia; pero hay que confiar en Dios, que hace milagros, que los ha
hecho algún día, que los volverá a hacer, señora, si es preciso. Dios
ampara a los buenos.

Emitida esta máxima, se llevó el pañuelo a los ojos, como si quisiera
limpiar la humedad de una lágrima auténtica, y después de echar un
suspirillo mal sacado, salió de la alcoba, dejando a las dos mujeres más
atribuladas de lo que estaban antes de su aparición.

~

Muy avanzada la noche, cuando la enferma, vencida por la fatiga, pudo
hallar en un ligero sueño alivio a las penas de su alma, Sola subió a su
casa. Ordinariamente subía la escalera en veloces saltos, cual pájaro
que vuela a su nido; aquella noche la subió lentamente, con tanto
trabajo como si cada escalón fuese una montaña. No apartaba los ojos del
suelo, y su rostro estaba lívido. Sin duda veía dentro de sí misma
espectros que la horrorizaban.

¿Qué tienes, niña mía?---le preguntó Sarmiento, que había salido a
abrirle.---¡Cuánto tiempo sin verte!\ldots{} Esa pobre gente estará muy
afligida. Y gracias que tienen un ángel como tú para que les acompañe.

La huérfana no contestó nada. La voz de D. Patricio parecía no ser para
ella más interesante ni más expresiva que el áspero chirrido de los
goznes de la puerta.

---¿Qué tienes? ¿en qué piensas?---dijo el anciano sentándose junto a
ella.---Tú tienes algo.

Después de una pausa en que silenciosamente la contempló, dijo con la
más viva amargura:

---¡Ya comprendo, pobre de mí! Ha llegado el momento de separarte de tu
viejo, de meterme en un hospicio y de marcharte para Inglaterra. Como me
has tomado algún cariño, esta separación no puede menos de afligirte.

---Ya no me voy para Inglaterra---murmuró Sola con una seriedad
sepulcral que desconcertó más a Sarmiento.

---Pues entonces\ldots{} eso que me has dicho me causa muchísima
alegría, hija de mi corazón. ¿Conque no te vas? ¡Qué sabrosas nuevas has
traído esta noche a tu viejecito! Dame un abrazo.

Al caer en los brazos del vagabundo, y cuando este la estrechaba con
amante ardor en ellos, Sola gimió dolorosamente y se echó a llorar,
diciendo:

---¡Ay, abuelo!\ldots{} ¡qué desgraciada es tu niña!\ldots{} Más le
valdría no haber nacido.

\hypertarget{xiv}{%
\chapter{XIV}\label{xiv}}

En la planta baja del edificio que se llamó primero Cárcel de Corte,
después Sala de Alcaldes, más tarde Audiencia y que ahora va camino de
llamarse, según parece, Ministerio de Ultramar, estaba situada la
Superintendencia General de Policía. La cárcel ocupaba el inmundo
edificio, que ya no existe, en la manzana inmediata, hacia la Concepción
Jerónima, y que fue casa y hospedería de los padres del Salvador. Desde
uno a otro caserón la distancia era insignificante, como la que existe
entre la agonía y la muerte, y a falta de un Puente de los Suspiros,
existía el callejón del Verdugo, de fácil tránsito para los que del
tribunal pasaban a los calabozos o de los calabozos a la horca.

Las respetables oficinas de aquella institución (firme columna del orden
político dominante entonces), tenían alojamiento tan digno de los jueces
como de las leyes, en las indecorosas crujías que ha visto no hace mucho
todo el que tuvo la desgracia de frecuentar los Juzgados de primera
instancia. La Comisión Militar, que era la que juzgaba a toda clase de
delincuentes, tenía su albergue en un antiguo edificio de la plazuela de
San Nicolás; pero el Presidente de ella frecuentaba tanto la
Superintendencia que se había mandado arreglar un despacho en el ángulo
que da al callejón del Verdugo. El Superintendente recibía en la sala
contigua a la callejuela del Salvador. El contraste horriblemente
burlesco entre los nombres de las fétidas callejuelas por donde
respiraban los dos instrumentos más activos del poder judicial y
político, no establecían diferencia esencial entre ellos, porque ambos
eran igualmente patibularios. Las odiosas antesalas de la horca eran
negras, tristes, frías, con repulsivo aspecto de vejez y humedad,
repugnante olor a polilla, tabaco, suciedad, y una atmósfera que parecía
formada de lágrimas y suspiros.

En todas las grandes poblaciones y en todas las épocas ha existido
siempre un infierno de papel sellado compuesto de legajos en vez de
llamas y de oficinas en vez de cavernas, donde tiene su residencia una
falange no pequeña de demonios bajo la forma de alguaciles, escribanos,
procuradores, abogados, los cuales usan plumas por tizones, y cuyo
oficio es freír a la humanidad en grandes calderas de hirviente
palabrería que llaman autos. El infierno de aquella época era el más
infernal que puede imaginar la human fantasía espoleada por el terror.

En una serie de habitaciones sucias y tenebrosas tenían sus mesas los
demonios inferiores, muy semejantes a hombres a causa de su hambrienta
fisonomía y de su amarillo color, resultado al parecer de una inyección
de esencia de pleito, que se forma de la bilis, la sangre y las lágrimas
del género humano. Con los brazos enfundados en el manguito negro,
desempeñaban entre desperezos, cuchicheos y bocanadas de tabaco, sus
nefandas funciones que consistían en escribir mil cosas ineptas. Con su
pluma estos diablillos pinchaban, martirizando lentamente; pero más
allá, en otras salas más negras, más indecorosas y más ahumadas con el
hálito brumoso de la curia, los demonios mayores descuartizaban como
carniceros. Sus nefandas rúbricas, compuestas por trozos nigrománticos,
abrían en canal a las pobres víctimas, y cada vez que llenaban un pliego
de aquella simpática letra cuadrada y angulosa que ha sido el orgullo de
nuestros calígrafos, daban un resoplido de satisfacción, señal de que el
precito estaba bien cocho por un lado y era preciso ponerlo a cocer por
el otro.

Las mesas negras, desvencijadas, cubiertas de un hule roto por donde
corría libremente la arenilla secante esperando a que se acercara una
mano sudorosa para pegarse a ella, sostenían los haces de llamaradas,
los paquetes de ascua, en forma de barbudos legajos amarillentos, todos
garabateados con la pez hirviente de los tinteros de plomo o de cuerno,
en cuyo horrendo abismo se cebaban las ávidas plumas.

Mientras algunos de estos demonios escribían, otros no se daban reposo,
entrando y saliendo de caverna en caverna y llevando recados a la
Superintendencia y a la cárcel. Los alguaciles y ordenanzas, que eran
unos pajecillos infernales muy saltones, transportaban grandes
cargamentos de materia ígnea de un rincón a otro: sonaban las
campanillas, como una señal demoníaca para activar los tizonazos y la
quemazón; se oían llamamientos, peticiones, apuradas preguntas;
buscábase entre mil legajos el legajo \emph{A} o \emph{B}, se
recriminaban unos a otros los de manguito en brazo y pluma en oreja, se
arrojaban fétidas colillas, volaba el papel con el pesado aire que
entraba al abrir y cerrar las puertas, oíase chirrido de plumas trazando
homicidas rúbricas, y movíanse, gimiendo sobre sus goznes mohosos, las
mamparas en cuyo lienzo roto se leía: \emph{Departamento de
purificaciones\ldots{} Padrón general\ldots{} Sentencias\ldots{}
Pruebas\ldots{} Negociado de sospechosos.}

La Superintendencia de policía y la Comisaría Militar se diferenciaban
poco en el fondo y en la forma, y no se juzgue a la segunda por su
calificativo, creyendo que imperaba en ella el criterio comúnmente
pundonoroso y honrado, aunque severo, de nuestro ejército. La presidía
un terrible individuo que vestía de brigadier, para baldón del uniforme
español; militares eran también sus vocales y el fiscal; pero todo su
mecanismo interno, su personal secundario así como sus procedimientos
habían sido tomados de la curia más abyecta. Entonces no había
propiamente ejército, porque casi todo él estaba sujeto al juicio de
purificación. Los voluntarios realistas, cuyo jefe era el ministro de la
Guerra, sostenían el orden social, auxiliando a los sanguinarios
tribunales y también imponiéndose a ellos. La Comisión Militar, que
contaba en el número de sus diversas misiones, la de purificar a aquel
nefando ejército, casi totalmente afecto a la Constitución, estaba en
absoluto sometida a la voluntad de aquella odiosa palanca del Gobierno
llamada D. Francisco Chaperón. Los demás altos individuos del aborrecido
tribunal eran figuras decorativas que sólo servían para hacer resaltar
con su penumbra la roja aureola infernal del presidente.

El público aguardaba en la portería de la Comisión (plazuela de San
Nicolás), impaciente, mugidor, grosero, blasfemante. Componíase en gran
parte de los oscuros ministros de la delación y de los testigos de
cargo, porque los de descargo no eran en ningún caso admitidos. Había
personas de todas clases, abundando las de la clase popular. De la clase
media eran pocos, de la más elevada poquísimos. Reuniéndolo todo, lo de
dentro y lo de fuera, el gentío que escribía y el que esperaba, los
diablos todos, grandes y pequeños y sus cómplices delatores podría
haberse formado un magnífico presidio. La inocencia no habría reclamado
para sí sino a poquísimas personas.

Grande era el alboroto entre los que esperaban por querer cada uno
entrar antes que los demás, y los voluntarios tenían que forcejear a
brazo partido para mantener el orden y establecer un turno riguroso.

---Yo estaba primero, señora\ldots{} Échese usted atrás.

---¿Usted primero? Si estoy aquí desde la madrugada\ldots{}

---Guardia, aquí se ha colado esta mujer. Ha venido después que yo y
está delante.

---Le digo a usted que estoy aquí desde la madrugada.

---¿A qué viene usted, hermosa? Si viene usted como testigo ha de
esperar a que la llamen\ldots{} aunque no se admiten aquí testigos con
faldas.

---No vengo como testigo.

---¿Viene a reclamar?\ldots{} Tiempo perdido.

---No vengo a reclamar.

---¿A delatar?

La mujer calló. Era joven, vestía modestamente de negro, con mantilla.
Su cara estaba pálida; sus ojos grandes y oscuros se abatían con
tristeza.

---¿Pero usted a qué viene?---le preguntó el voluntario encargado de
mantener el orden.

---A ver al Sr.~Chaperón. Ya se lo he dicho a usted seis veces.

---Acabáramos\ldots{} ¿Y no podría usted ver en su lugar al segundo
jefe?

---No señor. Tengo que hablar con el señor Chaperón, con el mismo
Sr.~Chaperón.

---Pues aún aguardará usted un ratito.

Una hora después, el mismo se acercó a ella y en tono de benevolencia le
dijo:

---Ahora en cuanto salga ese señor sacerdote que acaba de entrar, pasará
usted.

---Ya es tiempo.

---¿Ha esperado usted mucho, niña?

---Seis horas: son las diez. Apenas puedo ya tenerme en pie. Ayer
también estuve a las ocho de la mañana. Me dijeron que esto era cosa de
la Superintendencia. Fui a la Superintendencia. Allí esperé seis horas;
fui de oficina en oficina y al fin un señor muy gordo me dijo que yo era
tonta y que la Superintendencia no tenía nada que ver con lo que yo iba
a decir; que marchase a ver al Sr. Chaperón. Por la noche le busqué en
su casa; dijéronme que viniese aquí\ldots{}

---Usted viene a dar \emph{informes} a la Comisión Militar---dijo el
voluntario realista encubriendo con estas palabras la infante idea de la
delación.

La joven no contestó nada.

---Ya puede usted pasar---oyó decir al fin; y otro voluntario, especie
de Caronte de aquellos infernales pasadizos, la guió adentro.

Al atravesar el lóbrego pasillo, oprimiósele el corazón y tembló toda,
creyendo que una infernal boca se la tragaba y que jamás vería la clara
luz del día. Rechinó una mampara. La mujer vio una estancia regularmente
iluminada por los huecos de dos ventanas, y entró. Allí había dos
hombres.

\hypertarget{xv}{%
\chapter{XV}\label{xv}}

Uno estaba en pie, colocado frente al marco de la puerta, de modo que
recibiendo la luz por detrás, todo él parecía negro, negro el uniforme,
negras las manos, negra la cara. Pero en la sombra podía reconocerse
fácilmente al celoso funcionario que dispuso la elevación de la horca en
la plaza de la Cebada el 6 de Noviembre de 1823.

El otro estaba sentado y escribía con la soltura y garbo de quien ha
consagrado una existencia entera al oficio curialesco. Era un viejecillo
encorvado y pergaminoso, con espejuelos verdes, las facciones amomiadas,
el cuerpo enjuto. Mientras escribía, su espinazo afectaba una perfecta
curva, cuyo extremo, o sea la región capital, casi tocaba el papel. Al
dejar la pluma, recobraba lentamente su posición vertical, que siempre
era bastante incorrecta, por tener su cabeza cierta tendencia a colgar
balanceándose, como fruta madura que va a caer de la rama. Tenía la
costumbre de subirse a la frente las antiparras verdes mientras
escribía, y entonces parecía estar dotado de cuatro ojos, dos de los
cuales se encargaban de vigilar la estancia mientras sus compañeros
cubrían el papel de una hermosa letra de Torío que en claridad podía
competir con la de imprenta. Su nariz y la desaforada boca combinaban
armoniosamente sus formas para producir una muequecilla entre satírica y
benévola que producía distintos efectos en los que tenían la dicha de
ser mirados por el licenciado Lobo, pues tal era el nombre de este
personaje, no desconocido para nuestros lectores\footnote{Véase \emph{La
  Corte de Carlos IV}, \emph{Napoleón en Chamartín} y otros volúmenes de
  la \emph{Primera serie}.}.

La joven balbució un saludo dirigiéndose al de la mesa, que le parecía
más principal. Después extendió sus miradas por toda la pieza, que se le
figuró no menos triste y lóbrega que un panteón. Cubría los polvorientos
ladrillos del suelo una estera de empleita que a carcajadas se reía por
varios puntos. Los muebles no superaban en aseo ni elegancia al resto de
las oficinas, y las mesas, las sillas, los estantes se decoraban con el
mismo tradicional mugre que era peculiar a todo cuanto en la casa
existía, no librándose de él ni aun el retrato de nuestro Rey y señor D.
Fernando VII, que en el testero principal, y dentro de un marco
prolijamente decorado por las moscas, mostraba la augusta majestad neta.
Los grandes ojos negros del Rey, fulgurando bajo la espesa ceja corrida,
parecían llenar toda la sala con su mirada aterradora.

---¿Qué quiere usted?---gritó bruscamente Chaperón, mirando a la joven.

La turbación suele causar algo de sordera; así es que la interpelada
dejose caer en una silla con muestras de gran cansancio.

---Gracias, señor, me sentaré. Estoy muy fatigada; no me puedo tener.

Su entrecortado aliento, su palidez, la sequedad de sus labios indicaban
una fatiga capaz de producir la muerte si se prolongara mucho.

---No he dicho a usted que se siente, sino que qué quiere---manifestó
con desabrimiento el brigadier.

La joven se levantó vacilante como un ebrio.

---Puede usted sentarse, sí, siéntese usted---dijo Chaperón con menos
dureza.

Lobo le hizo una seña amistosa, obsequiándola al mismo tiempo con un
ejemplar de su sonrisa.

---Yo---dijo la joven dirigiéndose a Lobo que le parecía más
amable,---quería hablar con el Sr.~de Chaperón.

---Pues pronto, amiguita---gruñó este,---despachemos, que no estamos
aquí para perder el tiempo.

---¿Es Vuecencia el Sr.~D. Francisco Chaperón?

---Sí, yo soy\ldots{} ¿qué se te ofrece?---repuso el funcionario
practicando su sistema de tutear a los que no le parecían personas de
alta calidad.

---Quería hablar a Vuecencia---dijo la muchacha temblando,---acerca de
D. Benigno Cordero y su hija.

---Cordero\ldots---dijo Chaperón recordando.---¡Ah! ya\ldots{} el
encajero. Está bien. ¿Tú has servido en su casa?

---No señor.

---Su causa está muy adelantada. No creo que haya nada por esclarecer.
Sin embargo\ldots{} Señor licenciado Lobo, recoja usted las
declaraciones de esta joven.

---¿Cómo se llama usted?---preguntó Lobo tomando la pluma.

---Soledad Gil de la Cuadra.

---¡Gil de la Cuadra!---exclamó Chaperón con sorpresa dando algunos
pasos hacia la joven.---Yo conozco ese nombre.

---Mi padre---dijo Sola reanimándose,---era muy afecto a la causa del
Rey. Quizás Vuecencia le conocería.

---D. Urbano Gil de la Cuadra\ldots{} Ya lo creo. ¿Se acuerda usted,
Lobo?\ldots{} Últimamente se oscureció y no supimos más de él\ldots{}
Era un benemérito español que jamás se dejó embaucar por la canalla.

---Murió pobre y olvidado de todo el mundo---manifestó Sola, triste por
la memoria y gozosa al mismo tiempo por una circunstancia que
despertaría tal vez interés hacia ella en el ánimo de aquellos señores
tan serios.---Sabiendo quién soy y recordando la veracidad y honradez de
mi padre, tengo mucho adelantado en la opinión de Vuecencias.

---Seguramente.

---Y darán crédito a lo que diga.

---El pertenecer a una familia que se distinguió siempre por su
aborrecimiento a las novedades constitucionales, es aquí la mejor de las
recomendaciones.

---Pues bien, señores---dijo Soledad animándose más,---yo diré a
Vuecencias muchas cosas que ignoran en el asunto de D. Benigno Cordero.

---Anote usted, licenciado\ldots{} En efecto, siempre me han parecido
algo oscuros los hechos de ese endiablado asunto de Carnero\ldots¿no es
Carnero?\ldots{} No, Cordero. Tengo la convicción de su culpabilidad;
pero\ldots{}

---¡Oh! señor---dijo Soledad con viveza,---precisamente yo vengo a decir
que el Sr. D. Benigno y su hija son inocentes.

Chaperón, que iba en camino de la ventana, dio una rápida vuelta sobre
su tacón, como los muñecos que giran en las veletas al impulso del
viento.

---¡Inocente!---exclamó arrugando todas las partes arrugables de su
semblante, que era su modo especial de manifestar sorpresa.

Lobo dejó la pluma y bajó sus anteojos.

---Sí señor, inocente---repitió Sola.

---Oye, tú---añadió Chaperón.---¿Habrás venido aquí a burlarte de
nosotros?\ldots{}

---No señor, de ningún modo---repuso la huérfana temblando.---He venido
a decir que el Sr.~Cordero es inocente.

---Cordero\ldots{} inocente\ldots{} Inocente\ldots{} Cordero\ldots{}
¡Qué bien pegan las dos palabrillas, eh!---dijo el Comisario militar con
la bufonería horripilante que le aseguraba el primer puesto en la
jerarquía de los demonios judiciales.

Se había acercado a la joven, casi hasta tocar con sus botas marciales
las rodillas de ella, y cruzando los brazos y arrugando el ceño, la
miraba de arriba abajo desdeñosamente, como pudiera mirar el can a la
hormiga. Soledad elevaba los ojos para poder ver la tenebrosa cara
suspendida sobre ella como una amenaza del cielo. Su convicción y su
abnegación dábanle algún valor, por lo cual, desafiando la siniestra
figura, se expresó de este modo:

---Yo afirmo que los Corderos son inocentes, que están presos por
equivocación. Ya se supone que no habré venido sin pruebas.

Ella ignoraba que en aquel odioso tribunal las pruebas no hacían falta
para condenar ni para absolver. No hacían falta para lo primero porque
se condenaba sin ellas, ni para lo segundo, porque se condenaba también,
a pesar de ellas.

---Conque pruebas\ldots---dijo el vestiglo marcando más el tono de su
bufonería.---¿Y cuáles son esas pruebecitas?

---Yo no vengo a negar el delito---afirmó Soledad con voz entrecortada,
porque apenas podía hablar mientras sintiera encima el formidable peso
de la mirada chaperoniana.---Yo no vengo a negar el delito, no señor;
vengo a afirmarlo. Pero he dicho\ldots{} que el Sr.~Cordero es inocente
de ese delito, que el delito ¿me entienden ustedes? se ha achacado al
Sr.~Cordero por equivocación\ldots{} y esto lo probaré revelando quién
es el verdadero\ldots{} culpable, sí señor; el culpable del
delito\ldots{} del delito.

---Eso varía---dijo Chaperón apartándose.---Para probarme que no vienes
a burlarte de nosotros, dime cuál es el delito.

---Un oficial del ejército llamado D. Rafael Seudoquis, vino de Londres
con unas cartas.

---¡Ah!\ldots{} estás en lo cierto---dijo Chaperón con gozo,
interrumpiéndola.---Por ahí, por ahí\ldots{}

---Como Seudoquis no podía estar en Madrid sino día y medio, las cartas
venían en un paquete a cierta persona que las debía distribuir y recoger
las contestaciones.

---Admirable---dijo Chaperón como un maestro que recibe del examinando
la contestación que esperaba.---Y Seudoquis no celebró entrevistas con
Cordero, sino con otra persona. ¿No es eso lo que quieres decir?

---Sí señor; Cordero ni siquiera le conoce. Lo del noviazgo de Elena con
Angelito es verdad; pero D. Rafael no ha visto a su hermano ni a ninguna
otra persona de su familia en las treinta horas que estuvo en Madrid.

---Vamos, veo que conoces el paño\ldots{} Bien, paloma. Ahora, revélanos
todo lo que sabes. Lobo, anote usted.

Lobo tomó la pluma y subió otra vez a la frente sus verdes ojos sin
pestañas.

---Yo no diré nada---afirmó Soledad con la firmeza de un mártir,---no
diré una palabra, aunque me den tormento, si antes Vuecencia no me da
palabra de poner en libertad al Sr.~Cordero y a su hija.

---Según y conforme\ldots{} Aquí no somos bobos. Si yo veo clara la
equivocación\ldots{}

---¡Pues no ha de verla!\ldots{} Deme Vuecencia su palabra de ponerles
en libertad desde que conozca al verdadero culpable.

---Bueno; te la doy, te doy mi palabra; mas con una condición. No
soltaré a los Corderos si no resulta que el verdadero delincuente es un
ser vivo y efectivo, ¿me entiendes? Aquí no queremos fantasmas. Si es
persona a quien podemos traer aquí para que confiese y dé noticias y
vomite todo lo que sabe y expíe sus crímenes\ldots{} corriente.
Tendremos mucho gusto en reparar la equivocación. ¿Para qué estamos aquí
si no es para hacer justicia?

---El delincuente---dijo Sola con firmeza,---es un ser vivo y efectivo,
podrá confesar, podrá expiar su culpa\ldots{} Acabemos, señores, soy yo.

Chaperón y el experto licenciado habían visto muchas veces en aquella
misma siniestra sala y en otras dependencias del tribunal, personas que
negaban su culpabilidad, otras que delataban al prójimo, algunas que
intentaban con lágrimas y quejidos ablandar el corazón de los jueces;
habían visto muchas lástimas, infamias sin cuento, algo de abnegación en
pocos casos, afectos diversos y diversísimas especies de delincuentes;
pero hasta entonces no habían visto a ninguno que a sí mismo se acusara.
Hecho tan inaudito les desconcertó a entrambos y se miraron
consultándose aquella jurisprudencia superior a sus alcances morales.

---¿De modo que tú dices que tú misma eres quien cometió esos delitos
que Su Majestad nos ha mandado castigar? ¿Tú?\ldots{}

---Sí señor, yo misma.

---¿Y tú misma lo aseguras?\ldots{} de modo que te delatas a ti
misma\ldots---insistió Chaperón no dando entero crédito a lo que
oía.---Anote usted, Lobo. Esto es singularísimo, lo más singular que
hemos visto aquí. Lobo, anote usted.

Si en vez de decir «anote usted», hubiera dicho: «Lobo, muerda usted»,
el leguleyo no se habría arrojado con más ferocidad sobre la pluma y el
papel. La extrañeza del caso hacía estremecer todas las fibras de su
corazón, digámoslo así, de curial.

---Soledad Gil de la Cuadra---dijo el magistrado militar
dictando,---compareció\ldots{} etc\ldots{}

Después, volviéndose a la víctima que observaba el mover de la pluma de
Lobo, como si desde su sitio pudiera leer lo que este escribía, le dijo:

---¿Conque tú has sostenido relaciones con los emigrados? ¿Cuántas
veces? ¿Con varios o con uno solo?

---Con uno solo.

---Relaciones políticas, se entiende---indicó Chaperón más bien
afirmando, que preguntando.

---No señor, relaciones de amistad---dijo Soledad vacilando a cada
palabra.

---¿De amistad?\ldots{} ¿Quién es él?

Solita, después de dudar breve instante, pronunció un nombre. Pudo
observar que Lobo, al anotar aquel nombre, frunció primero el ceño,
exagerando después hasta llegar a la caricatura la contracción burlesca
de su boca.

---¿Tienes tú parentesco con ese bergante?---pregunto Chaperón.

---No señor.

---Entonces, ¿qué relaciones son esas?

---Es mi hermano\ldots{} quiero decir, mi amigo, mi protector.

---Ya, ya sabemos lo que quieren decir esas palabrillas---gruñó el
hombre-horca dando a luz una especie de sonrisa.---Háblanos con
franqueza; que juez y confesor vienen a ser lo mismo. ¿Eres tú su
querida?

Soledad se puso como la grana. Dominándose, hablo así:

---Condéneme usted; pero no me avergüence. Yo no soy querida de nadie.

---¿Venimos aquí con vergüencilla?---vociferó el ogro riendo con brutal
jovialidad.---¡Ay! ¡qué mimos tan monos!\ldots{} Paloma, recoge ese
colorete. ¿Ruborcillo tenemos? Aquí se conoce el mundo. Sr.~Lobo, anote
usted que ha revelado tener relaciones ilícitas con el susodicho\ldots{}

---No es cierto, no es cierto---exclamó Soledad levantándose y corriendo
hacia la mesa.

---¡Orden!---gritó Chaperón señalando a la víctima su asiento.

La huérfana, que había acopiado gran caudal de resignación, volvió a su
sitio y tan sólo dijo:

---Si tengo valor para sacrificarme por un inocente, también lo tendré
para calumniarme.

---¡Calumniarse!\ldots{} ¿Seguimos con las palabrejas retumbantes?
Pasemos a otra cosa. ¿Ese descuellacabras te ha escrito muchas veces?

---Seis veces desde que está en Inglaterra.

---¿Te ha hablado de sucesos políticos?

---Muy poco y por referencia.

---¿Conservas las cartas?

---No señor, las he roto.

---Ya lo averiguaremos. ¿Se ha anotado el domicilio de la reo?

---Sí señor.

---Adelante. Llegamos a D. Rafael Seudoquis. Ese señor trajo de Londres
un paquete de cartas para que tú las repartieras\ldots{}

---Sí señor\ldots---repuso la joven con firmeza.---Puedo asegurar que
Seudoquis no conoce a D. Benigno Cordero; que este no podía encargarse
de repartir las cartas, ni menos su hija, porque ni uno ni otra tenían
noticia de semejante cosa. Vivimos en la misma casa, yo en el segundo,
ellos en el principal, y como alguien de la policía vio al Sr.~Seudoquis
entrar en la casa, supuso que iba a la habitación de Cordero, cuando en
realidad iba a la mía.

---Muy bien, anote usted eso. Puede muy bien resultar que el tal Cordero
sea inocente, ¿por qué no?\ldots{} la justicia y la verdad por delante.
Sepamos ahora a quién iban dirigidas esas cartas. Este es el punto
principal\ldots{} Cordero no supo darnos noticia alguna. Si tú lo haces,
tendremos la mejor prueba de que no has venido a burlarte de nosotros.

Soledad vaciló un instante. Helado sudor corría por su frente, y sintió
como un torbellino en su cerebro. Era aquel un caso que la infeliz no
había previsto, porque su alma llena toda de generosidad y ofuscada por
la idea del bien que a realizar iba, no supo calcular la ignominia que
podía salirle al paso y detenerla en su gallardo vuelo. Aquel acto de
abnegación era de esos que no pueden realizarse con éxito feliz sin
tropezar con la infamia, poniendo a la voluntad en la alternativa de
retroceder o incurrir en actos vergonzosos. Espantada Sola de los
peligros que aparecían en su camino, no se atrevió a acometerlos, ni
supo tampoco esquivarlos, porque carecía de la destreza y travesuras
propias de tan gran empeño. Su única fuerza consistía en un valor
heroico, pasivo, formidable, y robusteciendo su alma con él, dijo al
severo magistrado:

---Yo me acuso a mí misma; pero no delataré a los demás.

---Me gusta\ldots{} sí, me gusta la salida---afirmó Chaperón cruzándose
de brazos delante de ella y moviendo el cuerpo como si fuera a dar un
salto.---¿Sabes que tienes frescura?\ldots{} Esto es dejarnos con un
palmo de narices\ldots{} Dime, mocosa, si no aclaras eso de las cartas,
¿qué ventaja sacamos de que seas tú el delincuente en vez de serlo
Cordero y su hija? ¿Qué diferencia hay?

---La diferencia que hay de la verdad a la mentira---replicó Soleda
imperturbable .---Si ellos son inocentes, ¿por qué han de estar en la
cárcel ocupando un puesto que me corresponde a mí?

---Música, música---dijo el funcionario haciendo sonar como castañuelas
los dedos de su mano derecha.---Aquí no estamos para perder el tiempo en
distingos. Hay mucho que hacer para resguardar Trono y Sociedad de los
ataques de esa gentualla negra. A ver: ¿qué hemos sacado en limpio de tu
acusación contra ti misma? Nada entre dos platos. ¡Por vida del
Santísimo Sacramento! Yo creí que en punto a noticias frescas y bonitas
nos ibas a traer aquí oro molido\ldots{} ¡Que es inocente D. Benigno! ¿Y
qué? ¡Que las cartas las recibiste tú y no él ni tampoco su hija! ¿Y
qué? ¡Por vida del Sant\ldots! esto es burlarse de la Comisión Militar.
Aquí se viene a servir al Estado, no a hacer comedias. ¿Eres tú
partidaria del Altar y del Trono, o por el contrario, eres amiga de la
canalla? ¿Te has prestado inocentemente a esa maquinación sin saber lo
que hacías?\ldots{} Hablemos claro.

Diciendo esto, Chaperón demostraba en la voz y en el gesto hallarse muy
satisfecho de su elocuencia y del incontrastable poder de sus razones.
Después de una pausa se acercó a Sola, y mirándola desde la altura de su
corpachón negro, capaz de intimidar al más bravo; accionando
enérgicamente con la mano derecha, cuyo dedo índice se erguía, tieso e
inflexible como un emblema de la autoridad, habló de este modo:

---El Gobierno de Su Majestad, que nos ha puesto aquí para que
vigilemos, tiene recompensas para los que le sirven, ayudándole a
esclarecer las maquinaciones de los pillos, ¿te vas enterando? y tiene
también castigos muy severos, muy severos, pero merecidos, para los que
encubren a los malvados con su punible silencio, ¿te vas enterando?

---¿Eso lo dice Vuecencia para que delate a los que recibieron las
cartas?---preguntó Soledad cerrando los ojos cual si estuviera
suspendida sobre su cuello el hacha del verdugo.---Siento mucho desairar
a Vuecencia; pero no puedo decir nada.

Chaperón se detuvo en su paseo por el cuarto. Viósele apretar las
mandíbulas, contraer los músculos de la nariz, como si fuera a lanzar un
estornudo, revolver los ojos\ldots{} Sin duda su cólera augusta iba a
estallar. Pero afortunadamente detuvo la formidable explosión un hombre
entre soldado y alguacil, de indefinible jerarquía, mas de indudable
fealdad, el cual abriendo la mampara, dijo:

---Vuecencia me dispense; pero la señora que vino esta mañana está ahí,
y quiere pasar.

---Que espere\ldots{} ¡Por vida del\ldots!

---Está furiosa---observó con timidez el que parecía soldado, alguacil,
polizonte, sin ser claramente ninguna de estas tres cosas.

Chaperón dudaba. Iba a decir algo, cuando una señora empujó
resueltamente la mampara y entró.

\hypertarget{xvi}{%
\chapter{XVI}\label{xvi}}

Era una mujer hermosísima, arrogante y tan airosa y guapetona en su
rostro y figura, como elegante en su vestir y tocado, de modo que
Naturaleza y Arte se juntaban para formar un acabado tipo de mujer a la
moda. La mirada que echó a Chaperón y a su legista, semejante a una
limosna dada más bien por compromiso que por voluntad, indicaba que la
modestia no era virtud principal en la señora. Pero su gallarda
altanería ¡cuán grato es decirlo! venía como de molde enfrente de
aquellos despreciables hombres tan duros con los desgraciados.

---Ni para ver al Rey se necesitan más requisitos---dijo la dama
sentándose en la silla que Chaperón le ofreció sonriendo.---Vi a
Calomarde esta mañana y me mandó venir aquí\ldots{} Yo creí que era cosa
de un momento\ldots{} pero si hay más de doscientas personas en la
puerta\ldots{} ¡Y qué gente! Diga usted ¿a qué viene toda esa gente, a
delatar? Si yo fuera la Comisión, empezaría por ahorcar a todo el que
delatara sin pruebas\ldots{} ¿No tienen ustedes otro sitio para que
hagan antesala las personas decentes?

---Señora---repuso Chaperón en tono adulador, que no galante,---siempre
que usted venga, pasará desde luego a mi despacho. Tengo mucho gusto en
complacerla, no sólo por estimación particular, sino por lo mucho que
respeto y admiro al Sr. Calomarde, mi amigo.

---Gracias---dijo la señora con indiferencia.---Vamos a mi asunto. D.
Tadeo me prometió que esto quedaría resuelto en tres días.

---D. Tadeo desde su poltrona halla muy fáciles los negocios de policía.
Yo quisiera verle aquí enredado con tanta gente y tanto papel\ldots{}
¡En tres días amigo Lobo, en tres días!

El licenciado apoyó la idea de su jefe, moviendo la cabeza con expresión
de lástima de sí mismo, por el mucho trabajo que entre manos traía.

---Esto es vergonzoso---exclamó la señora sin disimular su
enfado.---¿Conque para despachar un pasaporte se ha de gastar más tiempo
que para juzgar y condenar a muerte a un hombre?\ldots{} ¡Qué
tribunales, Santo Dios! ¡Qu Superintendencia y qué Comisión Militar!
Pongan todo eso en manos de una mujer y despachará en dos horas lo que
ustedes no saben hacer en una semana.

---Pero usted, señora---dijo Chaperón con el tono que en él pasaba por
benévolo,---no tiene en cuenta las circunstancias\ldots{}

---Veo que aquí las circunstancias lo hacen todo. Invocándolas a cada
paso se cometen mil torpezas, infamias y atropellos. Si volviera a
nacer, Dios mío, querría que fuese en un país donde no hubiera
circunstancias.

---Si se tratara aquí del pasaporte de una señora---indicó el presidente
de la Comisión con énfasis como el que va a desarrollar una tesis
jurídica,---ande con Barrabás\ldots{} Pero usted lleva dos criados, los
cuales es preciso que antes se definan y se purifiquen, porque uno de
ellos perteneció en tiempo de la Constitución a la clase de tropa, y el
otro sirvió largos años al ministro Calatrava\ldots{} Pero nos
ocuparemos del asunto sin levantar mano\ldots{}

---Yo deseo partir mañana---dijo la señora con displicencia .---Voy muy
lejos, señor Chaperón, voy a Inglaterra.

---Empezaremos, empezaremos ahora mismo. A ver, Lobo\ldots{}

Al dirigirse a la mesa, Chaperón fijó la vista en la víctima cuyo
proceso verbal había sido suspendido por la entrada de la soberbia dama.

---¡Ah!\ldots{} ya no me acordaba de ti---dijo entre dientes.---Voy a
despacharte.

Soledad miraba a la señora con espanto. Después de observarla bien,
cerciorándose de quién era, bajó los ojos y se quedó como una muerta.
Creeríase que batallaba angustiosamente con su desmayado espíritu,
tratando de infundirle fuerza, y que entre sollozos imperceptibles le
decía: «Levántate, alma mía, que aún falta lo más espantoso».

---Con el permiso de usted, señora---dijo Chaperón mirando a la
dama,---voy a despachar antes a esta joven. Lobo, extienda usted la
orden de prisión\ldots{} Llame usted para que la lleven\ldots{} Orden al
alcaide para que la incomunique\ldots{}

La víctima dejó caer su cabeza sobre el pecho.

Después miró de nuevo a la dama; pero esta vez encendiose su rostro y
parecía que sus ojos relampagueaban con viva expresión de amenaza. Esto
duró poco. Fue la sombra del espíritu maligno al pasar en veloz corrida
por delante del ángel oscureciendo su luz.

La señora estaba también pálida y desasosegada. Indudablemente no
gustaba de ver a quien veía, y en presencia de aquella humilde
personilla condenada parecía tener miedo.

---Aquí tienes, mala cabeza---dijo Chaperón dirigiéndose a la
huérfana,---el resultado de tu terquedad. Demasiado bueno he sido para
ti\ldots{} ¿Qué hemos sacado de tu declaración? Que Cordero es inocente.
¿Y qué ganamos con eso, qué gana con eso la justicia? Tú y nosotros
adelantamos muy poco\ldots{} Si hablaras sería distinto\ldots{} Tú
habrás oído decir aquello de\ldots{} quien te dio el pico, te hizo rico.
¿Te vas enterando? pero ahora, picarona, lo meditarás mejor en la
cárcel\ldots{} Allí se aclaran mucho los sentidos\ldots{} verás. Esta
linda pieza---añadió señalando a la víctima y mirando a la señora,---es
la estafeta de los emigrados, ¿qué tal? Ella misma lo confiesa, lo cual
no deja de tener mérito; pero nos ha dejado a media miel, porque no
quiere decir a quién entregó las cartas que ha recibido hace unos días.

Soledad se levantó bruscamente.

---Una de las cartas de los emigrados---dijo con tono grave extendiendo
el brazo,---la entregué a esa señora.

Después de señalarla con fuerza, cayó en su asiento con la cabeza hacia
atrás. Breve rato estuvieron mudos y estupefactos los tres testigos de
aquella escena.

---Es verdad---balbució la dama.---He recibido una carta de un emigrado
que está en Inglaterra; no sé quién la llevó a mi casa\ldots{} ¿qué mal
hay en esto?

Chaperón, que estaba como aturdido, iba a contestar algo muy importante,
cuando la señora corrió hacia la huérfana, gritando:

---Se ha desmayado esa infeliz.

En efecto, rendida Sola a la fuerza superior de las emociones y del
cansancio, había perdido el conocimiento.

La señora sostuvo la cabeza de la víctima, mientras Lobo, cuya
oficiosidad filantrópica no se desmentía un solo momento, acudió
trasportando un vaso de agua para rociarle el rostro.

---Eso no es nada---afirmó Chaperón.---Vamos, mujer, ¡qué mimos
gastamos! Todo porque la mandan a la cárcel\ldots{}

La puerta se abrió dando paso a cuatro hombres de fúnebre aspecto, que
parecían pertenecer al respetable gremio de enterradores.

---Ea, llevadla de una vez\ldots---dijo don Francisco
resueltamente.---El alcaide le dará algún cordial\ldots{} No quiero
desmayos en mi despacho.

Los cuatro hombres se acercaron a la condenada.

---Un poco de vinagre en las sienes\ldots---añadió el jefe de la
Comisión Militar.---Ea, pronto\ldots{} quitadme eso de mi despacho.

---¡A la cárcel!---exclamó con lástima la señora, acercándose más a la
víctima como para defenderla.

---Señora, dispense usted---dijo Chaperón apartándola con enfática
severidad.---Deje usted a la justicia cumplir con su deber\ldots{}
Vamos, cargar pronto. No le hagáis daño.

Los cuatro hombres levantaron en sus brazos a la joven y se la llevaron,
siendo entonces perfecta la similitud de todos ellos con la venerable
clase de sepultureros.

La mampara, cerrándose sola con estrépito, produjo un sordo estampido,
como golpe de colosal bombo, que hizo retumbar la sala.

\hypertarget{xvii}{%
\chapter{XVII}\label{xvii}}

Aquel mismo día ¡por vida de la Chilindraina! ¡cuán amargas horas pasó
el pobre don Patricio! Habrían bastado a encanecer su cabeza si ya no
estuviera blanca, y a encorvar su cuerpo, si ya no lo estuviera también.
Sus suspiros eran capaces de conmover las paredes de la casa: sus
lágrimas corrían amargas y sin tregua por las apergaminadas mejillas. No
podía permanecer en reposo un solo instante, ni distraerse con nada, ni
comer, ni aposentar en su cerebro pensamiento alguno, como no fuera el
fúnebre pensamiento de su desamparo y de la gran pena que le desgarraba
el corazón. Este lastimoso estado provenía de que Solita había salido
temprano, diciéndole:

---No sé cuándo volveré. Quizás vuelva pronto, quizás mañana, quizás
nunca\ldots{} Escribiré al abuelo diciéndole lo que debe hacer.
Adiós\ldots{}

Y dirigiéndole una mirada cariñosa, se limpió las lágrimas, y había
bajado rápidamente la escalera y había desaparecido ¡Santo Dios! como un
ángel que se dirige al cielo por el camino del mundo.

---¿Será posible que haya salido hoy para Inglaterra?---se preguntaba D.
Patricio apretándose el cráneo con las manos para que no se le escapara
también.---¡Pero cómo, si aquí está toda su ropa, si no ha hecho
equipaje, si en la cómoda ha dejado todo su dinero!\ldots{} ¿Pues adónde
ha ido entonces?\ldots{} «Quizás vuelva pronto, quizás mañana, quizás
nunca\ldots» Nunca, nunca.

Y repetía esta desconsoladora palabra, como un eco que de su cerebro
salía a sus labios. Otro motivo de gran confusión para él era que
Soledad había despedido a la criada el día anterior. Estaba, pues, el
viejo solo enteramente solo, encerrado en la espantosa jaula de sus
tristes pensamientos, que era como una jaula de fieras. Pasaba del
sentimentalismo más patético a la desesperación más rabiosa, y si a
veces secaba sus lágrimas despaciosamente, otras se mordía los puños y
se golpeaba el cráneo contra la pared. En los momentos de exaltación
recorría la casa toda desde la sala a la cocina, entraba en todas las
piezas, salía para volver a entrar, daba vueltas, y tropezaba y caía y
se levantaba. Como entrara en la alcoba de Sola, vio su ropa y
abalanzándose sobre ella hizo con febril precipitación un lío y
oprimiéndolo contra su pecho, cual si fuera el cuerpo mismo de la
persona amada y fugitiva, exclamó así con lastimero acento:

---Ven acá, paloma\ldots{} ven acá, niña de mi corazón\ldots{} ¿Por qué
huyes de mí? ¿por qué huyes del pobre viejo que te adora? Ángel divino,
ángel precioso de mi guarda cuya hermosura no puedo comparar sino a la
de la diosa de la Libertad, circundada de luz y sonriendo a los pueblos;
adorada hija mía, ¿en dónde estás? ¿no oyes mi voz? ¿no oyes que te
llamo? ¿no ves que me muero sin ti? ¿no te sacrifiqué mi gloria?\ldots{}
¡Ay!\ldots{} Mi destino, mi glorioso destino me reclama ahora, y no
puedo ir, porque sin ti soy un miserable y no tengo fuerzas para nada.
Contigo al suplicio, a la gloria, a la inmortalidad, a los Elíseos
Campos; sin ti a la muerte oscura, a la ignominia. Sola, Sola de mi
vida, ¿en dónde estás? Dímelo, o revolveré toda la tierra por
encontrarte.

Esto decía cuando llamaron fuertemente a la puerta. Corrió a abrir más
ligero que una liebre\ldots{} No era Sola quien llamaba, eran seis
hombres, que sin fórmula alguna de cortesía se metieron dentro. Uno de
ellos soltó de la boca estas palabras:

---¿No es éste el viejo Sarmiento que predicaba en las esquinas?\ldots{}
Echadle mano, mientras yo registro.

---¡Ah!\ldots---exclamó D. Patricio algo confuso.---¿Son ustedes de la
policía?\ldots{} Sí, yo recuerdo\ldots{} conozco estas caras.

---Procedamos al registro---dijo solemnemente el que parecía jefe de los
corchetes.---Toda persona que se encuentre en la casa, debe ser presa.
Cuidado no se escape el abuelo.

---Quiere decir---balbució Sarmiento,---que estoy preso.

---Ya se lo dirán allá---replicó el polizonte
desabridamente.---Andando\ldots{} Llévenme para allá al vejete, que aquí
nos quedamos dos para despachar esto.

Según la orden terminante del funcionario, (que era un funcionario
vaciado en la común turquesa de los cazadores de blancos en aquella
tenebrosa e infame época), Sarmiento fue inmediatamente conducido a la
cárcel, y sólo por un exceso de benevolencia incomprensible y hasta
peligros para la reputación de aquella celosa policía, le dieron tiempo
para ponerse el sombrero, recoger el pañuelo y media docena de
cigarrillos.

No se daba cuenta de lo que le pasaba el infeliz maestro, y durante el
trayecto de su casa a la cárcel de Corte, que no era largo, fue con los
ojos bajos, el cuerpo encorvado, las manos a la espalda y en un estado
tal de confusión y aturdimiento, que no veía por dónde pasaba, ni oía
las observaciones picarescas de los transeúntes. Cuando entraron en la
cárcel, el anciano se estremeció, revolviendo los ojos en derredor. Su
entrada había sido como el choque del ciego contra un muro, símil tanto
más exacto cuanto que D. Patricio no veía nada dentro de las paredes del
tenebroso zaguán por donde se comunicaba con el mundo aquella mansión de
tristeza y dolor.

Lleváronle al registro y del registro a un patio, donde había algunas
personas que imploraban la misericordia de los carceleros para poder ver
a los detenidos. Hiciéronle subir luego más que de prisa por hedionda
escalera que se abría en uno de los ángulos del patio, y hallose en un
largo corredor o galería, que parecía haber sido claustro, pero que
tenía entonces tapiadas todas sus ventanas, sin dejar más entrada a la
luz que unos ventanillos bizcos en la parte más alta.

Al entrar en la galería, Sarmiento oyó gritos, lamentos, imprecaciones.
Era al caer de la tarde, y como la luz entraba allí avergonzada al
parecer y temerosa, deteniéndose en los ventanillos por miedo a que la
encerraran también, no se podía distinguir de lejos las personas.
Veíanse sombrajos movibles, los cuales, al acercarse a ellos, resultaban
ser la simpática humanidad de algún calabocero que entraba en las celdas
o salía de ellas.

Había centinelas de trecho en trecho, cuya vigilancia no podía ser muy
grande, porque a cada instante les era forzoso apartar de las puertas de
las celdas a personas importunas que iban a turbar la tranquilidad de
los reos. Las llorosas mujeres, abusando de los miramientos a que tiene
derecho su sexo, molestaban a los señores cabos pidiéndoles noticia de
tal o cual preso, dándoles cualquier recadillo verbal o encargo enojoso,
como llevar pan a alguno de los muchos hambrientos que se comían los
dedos dentro de las celdas. En una de estas debía de estar encerrado un
loco furioso, cuya manía era dar golpes en la puerta, con lo cual
estaban muy disgustados los carceleros, hombres celosísimos de la paz de
la casa. El dolor y la desesperación, callado el uno, ruidosa la otra,
hacían estremecer las frágiles paredes, porque el mezquino edificio era
indigno de la rabia que contenía, y a ser tal como a ella cuadraba,
hubiera tenido más piedras que el Escorial y más hondos cimientos que el
alcázar de Madrid.

Sarmiento fue introducido en una pieza relativamente grande, cuya
suciedad parecía ser resumen y muestrario de todas las suertes de
inmundicia que los años y la incuria de los hombres habían acumulado en
la indecorosa cárcel de Corte. En la zona más baja, una especie de faja
mugrienta marcaba el roce de muchas generaciones de presos, de muchas
generaciones de alguaciles, de muchas generaciones de jueces y curiales.
Alumbrábala el afligido resplandor de un quinqué colgado del techo, que
parecía acababa de oír leer su sentencia de muerte, y se disponía con
semblante contrito a hacer confesión de sus pecados. Como el techo era
muy bajo, y los allí presentes se movían de un lado para otro en torno
al ajusticiado quinqué, las sombras bailaban en las paredes haciendo
caprichosos juegos y cabriolas. En el fondo había la indispensable
estampa de Su Majestad, y sobre ella un Crucifijo cuya presencia no se
comprendía bien, como no tuviera por objeto el recordar que los hombres
casi son tan malos después como antes de la Redención.

Delante de Su Majestad en efigie y de la imagen de Cristo crucificado,
estaba en pie, apoyándose en una mesa, no fingido, sino de carne y
hueso, horriblemente tieso y horriblemente satisfecho de su papel, el
representante de la justicia, el apóstol del absolutismo, don Francisco
Chaperón, siempre negro, siempre de uniforme, siempre atento al crimen
para confundirle donde quiera que estuviese en honra y gloria del Trono,
del orden y de la Fe católica. Pocas veces se le había visto tan
fieramente investigador como aquella noche. Indudablemente parecía que
el tal personaje acababa de llegar del Gólgota y que aún le dolían las
manos de clavar el último clavo en las manos del otro, del que estaba
detrás y en la cruz, sirviendo de sarcástico coronamiento al retrato del
señor D. Fernando VII.

A la derecha había una mesa donde estaban media docena de diablejos
vestidos con el uniforme de voluntario realista y acompañados por el
licenciado Lobo, prestos todos a lanzar las plumas dentro de los
tinteros. La izquierda era ocupada por un banquillo pintado de color de
sangre de vaca: en él se sentaba alguien a quien D. Patricio no vio en
el primer momento. El anciano no había salido aún de aquel estupor que
le acometiera al ser conducido fuera de su casa; miró con cierta
estupidez al tremendo fantasma, miró después a toda la chusma curialesca
que le rodeaba, al licenciado Lobo; miró al Santo Cristo, al Rey
pintado, y por fin, clavando los ojos en el banco de color de sangre,
vio a su adorada hija y compañera.

---¡Sola!\ldots{} ¡hija de mi alma!\ldots---gritó lanzando ronca
exclamación de alegría.---Tú aquí\ldots{} yo también\ldots{} ¡parece que
esto es la cárcel!\ldots{} ¡el suplicio!\ldots{} ¡la gloria!\ldots{} ¡mi
destino!\ldots{}

\hypertarget{xviii}{%
\chapter{XVIII}\label{xviii}}

Clarísima luz entró de improviso en la mente del afligido viejo;
desaparecieron las percepciones vagas, las ideas confusas para dar paso
a aquella siempre fija, inmutable y luminosa que había dirigido su
voluntad durante tanto tiempo, llenando toda su vida moral.

---Ya estoy en mí---dijo en tono de seguridad y
convicción.---Soledad\ldots{} ¡tú y yo en este sitio! Al fin, al fin
Dios ha señalado mi día. ¿No lo decía yo?\ldots{} ¿no decía yo que al
fin vendría la hora sublime? ¡Destino honroso el nuestro, hija mía! He
aquí que no sólo heredas mi gloria, sino que la compartes, y los dos
juntamente, unidos aquí como lo estuvimos allá, somos llamados\ldots{}

---Silencio---gritó Chaperón bruscamente.---Responda usted a lo que le
pregunto. ¿Cómo se llama usted?

---Excusada pregunta es esa---repuso con aplomo y dignidad D.
Patricio,---pues todo el mundo sabe en Madrid y fuera de él que soy
Patricio Sarmiento, adalid incansable de la idea liberal, compañero de
Riego, amigo de todos los patriotas, defensor de todas las
Constituciones, amparo de la democracia, terror del despotismo. Soy el
que jamás tembló delante de los tiranos, el que no tiene en su corazón
una sola fibra que no grite \emph{libertad}, y el que aun después de
muerto sacará la cabeza del sepulcro para gritar\ldots{}

---Basta---dijo Chaperón, notando que las palabras del reo provocaban
murmullos.---Charlatán es el viejo\ldots{} Responda usted. ¿Conoce a
esta joven?

---¿Que si la conozco? Que si conozco a Sola\ldots{} Si no temiera
faltar al respeto que debo a todo juez quienquiera que sea, diría que es
necia pregunta la que Vuecencia acaba de hacerme. Esta es mi hija
adoptiva, mi ángel de la guarda, mi amparo, mi compañera de vida, de
muerte, de cielo y de inmortalidad. Dios, que dispone todas las
grandezas, así como el hombre es autor de todas las pequeñeces, ha
dispuesto que este ángel divino me acompañe también ahora. ¡Admirable
solución de la Providencia! Yo creí haberla perdido y la encuentro junto
a mí en la hora culminante de mi vida, cuando se cumple mi destino;
aparece a mi lado, no para darme esos triviales consuelos que no
necesita mi corazón magnánimo, sino para compartir mi sacrificio y con
mi sacrificio mi gloria. Adelante, señores jueces, adelante. Acaben
ustedes. Soledad y yo nos declaramos reos de amor a la libertad, nos
declaramos dignos de caer bajo vuestras manos, y confesamos haber
trabajado por el triunfo del santo principio, ahora y antes y siempre,
porque para ello nacimos y por ello morimos.

Causaba diversión a los diablillos menores y aun al diablazo grande el
desenfado del buen viejo, por lo cual no habían puesto tasa a la charla
de este. Mas Chaperón, que deseaba concluir pronto, dijo al reo:

---¿Es cierto que esta joven recibió un paquete de cartas de los
emigrados para repartirlas a varias personas de Madrid?

---¿Y eso se pregunta?---replicó Sarmiento como si admirara la candidez
del vestiglo.---¿Pues qué había de hacer sino trabajar noche y día por
el triunfo de la sagrada causa?\ldots{} ¿No he dicho que para eso
nacimos y por eso morimos?

Soledad miraba con ojos muy compasivos a su amigo y al juez
alternativamente. Mas pronto dejó de mirarlos y se reconcentró en sí
misma, mostrando estoica indiferencia hacia aquel lúgubre diálogo entre
un insensato y un verdugo. Había hecho ya con Dios pacto de resignación
absoluta y se entregaba a la voluntad divina, prometiendo no oponer
ninguna resistencia a los accidentes humanos, ni aceptar otro papel que
el de víctima callada y tranquila. Entre el instante en que la sacaron
desmayada de la caverna del gran esbirro hasta aquel en que le pusieron
delante al compañero de su infortunio, habían pasado para ella horas muy
angustiosas. Pero su espíritu se había rendido al fin, aceptando la
fórmula esencial del cristiano, que es rendirse para vencer y perderse
absolutamente para absolutamente salvarse. Si algún pequeño combate
sostenía aún su alma, era porque el propósito de pensar solamente en
Dios no podía cumplirse aún con rigurosa exactitud. Pensaba en algo que
no era Dios, pero aun así, iba conquistando la tranquilidad y un pasmoso
equilibrio moral, porque había arrojado fuera de sí valerosamente toda
esperanza.

---Usted sabrá sin duda a quién venían dirigidas esas cartas---preguntó
Chaperón a Sarmiento.

---¿Pues qué?\ldots{} ¿ella no lo ha dicho?---repuso el anciano
nuevamente admirado de la ignorancia del tribunal.---Esto no se puede
considerar como delación, porque esas personas son leales patricios que
también anhelan llegue la coyuntura de sacrificarse por la libertad.
Nosotros no tenemos secretos, nosotros, como los héroes de la
antigüedad, lo hacemos todo a la luz del día. Fue preciso prestar un
servicio a la santa causa, facilitando las comunicaciones entre todos
los que conspiran dentro y fuera para hacerla triunfar, y lo prestamos,
sí señor, lo prestamos a la clarísima luz del sol, \emph{coram populo}.
Las cartas eran cuatro.

---Atención---dijo D. Francisco acercándose a la mesa de los escribanos.

---Una era para D. Antonio Campos, ese gran patriota que acaba de llegar
de Tarifa y Almería, otra para un oficial de la antigua guardia que se
llama Ramalejo, la tercera venía dirigida a D. Roque Sáez y Onís, y la
cuarta a D.ª Jenara de Baraona.

---Muy bien---gruñó Chaperón, asemejándose mucho en su gruñido al perro
que acaba de encontrar un hueso perdido.---Veo que el viejo y la niña
son la peor casta de conspiraciones que se conoce en Madrid.

---Sí---dijo Sarmiento con exaltación,---insúltenos usted\ldots{} Eso
nos agrada. Los insultos son coronas inmarcesibles en la frente del
justo. Mire usted las espinas que lleva en su cabeza aquel que está en
la cruz.

---Silencio---gritó Chaperón.---Veo que él es tan parlachín como ella
hipocritona. Ya sabemos lo de las cartas, linda pieza\ldots{} Ahora el
buen viejo nos informará de todas las particularidades que hayan
ocurrido en la casa. ¿Tiene noticia de que entrara en estos líos don
Benigno Cordero?

---¡Cordero!---exclamó Sarmiento con asombro.---Cordero es un hombre
vulgar, un tendero, un cualquiera\ldots{} ¿Cómo puede ser capaz
semejante hombre de intervenir en un complot de esos que sólo acometen
las almas grandes y valerosas?

---¿Seudoquis fue muchas veces a la casa?

---Dos veces, dos. Para nada hay que mentar a Cordero. Nuestra gloria es
nuestra, señor mío, y de nadie más. ¡Ay de aquel que intente quitarnos
una partícula de ella, siquiera sea del tamaño de un grano de alpiste!
Nosotros, nosotros solos somos los héroes, nosotros las víctimas
sublimes. Fuera intrusos y gentezuela que se presenta en el festín de la
gloria con sus manos lavadas reclamando lo que no les pertenece ni han
sabido ganar con su abnegación. Nosotros solos, ella y yo, nadie más que
ella y yo.

---El que enviaba las cartas---añadió don Francisco dando un paso hacia
Sarmiento,---¿no hablaba de lo de Almería y Tarifa ni de la revolución
que estaban preparando?

---Nosotros---repuso Sarmiento con desdén,---no nos ocupamos de frívolos
detalles. ¡Almería, Tarifa! ¿qué vale eso ni qué significa? Hechos
aislados que ni precipitan ni detienen el hecho principal, que es la
victoria de la libertad. Si al fin tiene que ser, si ha de venir tan de
seguro como saldrá el sol mañana\ldots{} Que se frustre una intentona,
que salga mal un desembarco, que fusiléis a trescientos o a mil o a un
millón de patriotas\ldots{} nada importa, señores. Lo que ha de venir,
vendrá. Si pretendéis atajarlo con patíbulos, vendrá más pronto. Los
patíbulos son árboles fecundos, que con el riego de la sangre dan frutos
preciosísimos. Echad sangre, más sangre; eso es lo que hace falta. Las
venas de los patriotas son el filón de donde mana la nueva vida.

«No me habléis de conspiraciones parciales; yo no entiendo de eso. El
que escribió las cartas, lo mismo que mi hija, lo mismo que yo,
cooperamos con nuestra voluntad y nuestros deseos más íntimos y más
ardientes en ese gran complot moral cuyas ramificaciones se extienden
por todo el mundo. ¡Ah! señores, no conocéis la gran conspiración del
tiempo. A ella pertenezco, a ella pertenecen todas vuestras
víctimas\ldots{} Ea, despachemos pronto. Basta de fórmulas y de
procedimientos necios. El patíbulo, el patíbulo, señores, esa es nuestra
jurisprudencia. De él hemos de salir triunfantes, trocados de humanos
miserables en inmaculados espíritus. Lo mismo nos da que nos ahorquéis
de esta o de la otra manera, más o menos noblemente. ¿A los mártires del
circo romano les importaba que el tigre que se los comía tuviera la
oreja negra o amarilla? No, porque no atendían más que a la sublime
idea; lo mismo nosotros no atendemos más que a esta idea que nos lleva
en pos del suplicio, la cual es como un fuego sacrosanto que nos
embelesa y nos purifica. No tenemos ya sentidos, no sabemos lo que es
dolor\ldots{} ¡La carne!\ldots{} ¡ah! no nos merece más interés que el
despreciable polvo de nuestros zapatos. Adelante, pues. Cumpla cada uno
con su deber: el vuestro es matar, el nuestro sucumbir carnalmente, para
vivir después la excelsa, la inacabable y deliciosa vida del
espíritu\ldots{} Vamos allá; ¿en dónde, en dónde está esa bendita
horca?»

Había tanta naturalidad en las entusiastas expresiones del exaltado
viejo patriota y al mismo tiempo un tono de dignidad tan majestuoso, que
los empleados de la Comisión, así militares como civiles, no podían
resistir al deseo de oírle. Aunque el sentimiento que a la mayoría
dominaba era de burla con cierta tendencia a la compasión, no faltaba
quien oyese al estrafalario viejo con un interés distinto del que
comúnmente inspiran las palabras de los tontos. El mismo Chaperón se
mostraba complacido, sin duda porque le divertía su víctima, haciéndolo
mucho más barato que el célebre gracioso Guzmán que empezaba su carrera
en el teatro del Príncipe. Pero como la dignidad del tribunal no
permitía tales comedias, Don Francisco mandó al reo que diese por
terminada la representación.

Los empleados de policía que se quedaron registrando la casa de Sola,
aparecieron. Según parecía, habían encontrado alguna cosa de gran valor
jurídico; habían hecho provisión de pedacitos de papel, fragmentos de
cartas, sin olvidar un polvoriento retrato de Riego, hallado entre los
bártulos de D. Patricio, dos o tres documentos masónicos o comuneros y
una carta dirigida al maestro de escuela. Examinolo todo ávidamente
Chaperón y lo entregó después a Lobo para que constase en el proceso. En
tanto D. Patricio se había acercado a su compañera de infortunio y en
voz baja le decía:

---Animo, ángel de mi vida, cordera mía. Que en esta ocasión solemne no
deje de estar tu espíritu a la altura del mío. Inspírate en mí.
Reflexiona en la gloria que nos espera y en el eco que tendrán nuestros
sonorosos nombres en los siglos futuros perpetuándose de generación en
generación. ¿Por qué estás triste en vez de estar alegre como unas
castañuelas? ¿Por qué bajas los ojos en vez de alzarlos como yo, para
tratar de ver en el cielo el esplendoroso asiento que nos está
destinado? Tu destino es mi destino. Ambos están escritos en el mismo
renglón. Hay gemelos del morir como los hay del nacer: tú y yo somos
mellizos y juntos saldremos del vientre de este miserable mundo a la
inmensa vida del otro\ldots{} Posible es que no lo comprendieras antes,
niña de mis ojos; yo tampoco lo creía, y era engañado por hechos
mentirosos. Tu proyecto de abandonarme era una ficción del destino para
sorprenderme después con esta unión celestial. Mi entrada en tu casa, el
amparo que me diste, ¿qué significan sino la preparación para estas
nuestras bodas mortuorias, de las cuales saldremos unidos por siempre
ante el altar de la glorificación eterna? Tú necesitas de mí para este
santo objeto, así como yo necesito de ti\ldots{} Bien sabía yo que
conspirabas\ldots{} ¡Y conspirabas por la santa libertad! Bendita
seas\ldots{} Serás condenada y yo también. ¡Seremos condenados!\ldots{}
¿Ves cómo no es posible la separación? ¿Ves cómo lo ha dispuesto Dios
así? Viviremos juntos eternamente. ¡Qué inefable dicha!\ldots{} Solilla
de mi vida, ten ánimo; que la flaca naturaleza corporal no soborne con
sus halagos tu alma de patriota. Vive como yo la excelsa vida del
espíritu. Desprécialo todo, mira al cielo, nada más que al cielo y a mí,
que soy tu compañero de gloria, tu gemelo, tu segundo tú, a quien has de
estar unida por los siglos de los siglos.

Soledad miró a su amigo. La serenidad que en él producía un loco
entusiasmo producíala en ella la resignación, ese heroísmo más sublime
que todas las exaltaciones de valor, y al cual damos un nombre oscuro:
lo llamamos paciencia, y germina como flor invisible y modesta en el
alma de los que parecen débiles.

---Veo que no lloras---dijo D. Patricio observando aquel semblante
plácidamente tranquilo, a quien la virtud mencionada daba angelical
hermosura.---No lloras, no estás demudada\ldots{}

---¿Yo llorar? ¿por qué?

---Así me gusta---exclamó Sarmiento con entusiasmo.---¡Oh! almas
sublimes, ¡oh! almas escogidas. ¡Y pensar que os han de intimidar horcas
y suplicios!\ldots{} Señores jueces, aquí aguardamos la hora del
holocausto. Llevadnos ya: subidnos a esos gallardos maderos que llamáis
infamantes. Mientras más altos mejor. Así alumbraremos más. Somos los
fanales del género humano.

Chaperón mandó que los dos reos fuesen conducidos cada cual a su
calabozo; mas como el alcaide manifestase la imposibilidad de ocupar dos
departamentos, se dispuso que ambos gemelos de la muerte fuesen
encerrados en un solo cuarto.

---Vamos---dijo D. Patricio enlazando con su brazo la cintura de Sola.

Esta se dejó llevar. Cuando iban por la oscura galería, la joven
huérfana oyó claramente en su oído estas palabras dichas en voz muy
baja, como un silbido:

---Señora, no se sofoque usted mucho\ldots{} se hará un esfuercito por
salvarla\ldots{} Una persona que se interesa por usted\ldots{} que se
interesa, sí\ldots{} me encarga de advertírselo.

Soledad volviose prontamente y vio unos ojos verdes y grandes del tamaño
de huevos. Estos ojos brillaban, reflejando la claridad del farol de los
carceleros, en un semblante amojamado y partido en dos por la hendidura
sonriente de la prolongada boca, casi vacía. En vez de tranquilizarse,
Soledad tuvo miedo.

\hypertarget{xix}{%
\chapter{XIX}\label{xix}}

El licenciado Lobo, asesor privado del señor Chaperón, tenía su oficina
en el ángulo más oscuro y apartado de la planta baja de la Comisión
Militar. Cubría el piso la estera más vieja, servíale de escritorio la
mesa más rota que contaba entre sus propiedades el Estado, y el pupitre,
el tintero, la estantería denotaban con honrosa vejez haber acompañado
en toda su larga vida a las antiguas covachuelas. Hasta el retrato de
Fernando VII, que decoraba la pared, era el más feo de toda la casa, y
comido de polilla, no presentaba a la admiración del espectador más que
los ojos y parte del cuerpo. Lo demás era una mancha irregular con
grandes brazos al modo de tentáculos. Parecía un gran cefalópodo que
estaba contemplando a su víctima antes de chupársela.

En el centro de este mueblaje y encorvado sobre una mesa llena de
descoloridos papeles, aparecía el leguleyo, cuya figura encajaba en tal
marco como el cernícalo en su nido. La diestra pluma rasgueaba sin cesar
cual si fuera absolutamente imprescindible su actividad para la
existencia de todo aquello, o como si fuera la clave cabalística de que
dependían las imágenes del despacho y del retrato y de los muebles y del
licenciado mismo. Cuando la pluma paraba parecía que todo iba a
desvanecerse. Si no fuera porque en los ratos de descanso el asesor se
ponía a tararear alguna tonadilla trasnochada de las del tiempo de la
Briones y de Manolo García, se le hubiera tenido por momia automática o
por alma en pena a quien se había impuesto la tarea de escribir mil
millones de causas para poderse redimir.

Al día siguiente de la prisión de Sarmiento y cuando aún no había
despachado regular porción de su faena de la mañana, una señora se
presentó sin anunciarse en el escondrijo del asesor.

---¡Oh! señora\ldots---exclamó Lobo suspendiendo la escritura.---No
esperaba a usted tan tempranito. Hágame usted el obsequio de tomar
asiento.

Ya la señora lo había hecho en la única silla que servía para el caso.
Era la misma dama a quien vimos en el despacho de Chaperón, guapa si las
hay, seductora mujer de cara y cuerpo y apostura, \emph{tota totalitate}
hermosa. Envolvíase en un rico chal blanco que a Lobo le pareció, sobre
los lindos hombros y entre los brazos de verde vestidos, como el más
gracioso capricho de la nieve entre las plantas de un jardín. Como a los
viejos feos se les permite ser galantes, Lobo dijo que la cara de la
señora era una rosa con la cual no se había atrevido la nieve, temiendo
que una mirada la derritiera.

---Déjese usted de sandeces---dijo ella.---Yo he venido a salir de
dudas.

---¿Respecto a esa jovenzuela que se delató a sí misma?\ldots{} Confieso
que es el primer caso que he visto desde que tengo esta nobilísima pluma
en la mano. Usted se interesa por ella\ldots{}

---Mucho, muchísimo---repuso la dama con pena.---Anoche he tenido una
pesadilla\ldots{} no es la primera vez que sueño con ella\ldots{} ¿Pues
no he dado en soñar que soy verdugo y que la estoy ahorcando?

---Graciosísimo, señora mía, graciosísimo. ¿La conoce usted hace tiempo?
¿De qué procede ese interés tan vivo? Ella no demuestra tenerla a usted
grabada en las telas de su corazón. Recordemos cómo declaró haberle
entregado una de las cartas. Sin duda quería perderla a usted. ¡Infame
víbora! ¡Y usted quiere favorecerla! ¡Oh generosidad inaudita!

---¡Ella me aborrece!

---Se conoce: sí, porque lo de la carta es una calumnia.

---No es calumnia, no. Recibí la carta---dijo la señora
suspirando.---Pero Chaperón me ha dicho que no seré molestada por esa
declaración. Mostraré la carta si es preciso. No contiene nada que
trascienda a conspirar.

---Todo sea por Dios---dijo Lobo con ademán distraído.---Pues todo se
arreglará. Basta que usted se interese por ella, para que Don Francisco
sea benigno. Para él no hay más Dios que Calomarde, y como mi señora
tiene felizmente todo el favor de nuestro querido Ministro y también el
de Quesada\ldots{}

---No me fío yo mucho del Ministro---dijo la dama nublando su hermoso
semblante con las sombras de la duda.---Muy amigo mío era don Víctor
Sáez y me prendió en Cádiz, como usted sabe. Aquello duró poco; pero fuí
maltratada del modo más grosero. No hay que fiar de las amistades en
estos tiempos.

---No, no hay que fiar, señora mía---repuso Lobo riendo y bajando la voz
como el que va a decir un secreto peligroso.---¡Estamos en los tiempos
más perros que se han visto desde que hay tiempos, y bregamos con la
gente más mala que se ha visto desde que el hombre, esa infame bestia
inteligente, apareció sobre la tierra! Empero, usted conseguirá lo que
desea. ¿Es cuestión de gratitud? ¿Ha recibido usted favores de esa
infeliz o de su familia?

---No, no es eso---repuso la dama, mostrando que la importunaba la
curiosidad del hombre de leyes.---Es cuestión de conciencia.

---¿Debe usted favores a esa desgraciada?

---No, ella me debe a mí un disfavor muy grande. Yo he sido mala,
Sr.~Lobo\ldots{} pero no, no soy tan mala como yo misma creo. No faltan
voces en mi conciencia\ldots{} Verdad es que tengo un genio arrebatado,
que soy capaz en ciertos momentos\ldots{} Vamos, lo diré, soy capaz
hasta de coger un puñal\ldots{}

La hermosa dama, moviendo su brazo como para matar, convirtiose por
breve momento en una figura trágica de extraordinaria belleza.

---Pero estos furores me pasan---añadió pasándose la mano por los
ojos.---Pasan, sí, y como Dios castiga y advierte\ldots{} Yo he sido
mala; pero no he cerrado mis ojos a las advertencias de Dios. No es
posible siempre reparar el mal que se ha causado\ldots{} pero se me
presenta ahora la ocasión de hacer un bien y lo he de hacer: quiero
sacar de la prisión a esa joven.

---El Sr.~D. Francisco\ldots{}

---No me fío yo del Sr.~D. Francisco. Es demasiado amigo de mi esposo
para que yo haga caso de sus palabrejas corteses. Usted, usted puede
arreglarlo fácilmente.

---¿Cómo?

---Componiendo la causa de modo que aparezca la reo tan inocente de
conspiración como los ángeles del cielo, aunque no sé yo si Chaperón y
Calomarde podrán convencerse de que los ángeles no conspiran.

---¡La causa, señora!---exclamó Lobo sonriendo con malicia.

---Sí, componer la causa, hombre de Dios, poner lo blanco negro y lo
negro blanco.

---Pero Sra. D.ª Jenara de mis pecados, si aquí no hay causas, ni
jurisprudencia, ni ley, ni sentencia, ni testimonio, ni pruebas, ni nada
más que el capricho de la Comisión Militar y de la Superintendencia,
sometidas, como usted sabe, al capricho más bárbaro aún de los
voluntarios realistas. Si todo este fárrago de papeles que usted ve aquí
es tan inútil para la suerte de los presos como las piedras de que está
empedrada la calle\ldots{} Si todo esto es vana fórmula; si yo escribo
porque me pagan para que escriba; si esto es puramente lo que yo llamo
\emph{pan de archivo}, porque no sirve más que para llenar esa gran boca
que está siempre abierta y nunca se sacia\ldots{} ¡Oh inocencia, oh
candor pastoril! No hable usted de causas ni de procedimientos, porque
si todo esto (señaló los legajos que en grandes pilas le rodeaban) se
escribiera en griego, serviría para lo mismo que en castellano sirve,
para nada\ldots{} ¡Pobres ratones! ¡y es tan inhumana la sala, que manda
poner ratoneras para impedirles que se coman esto!

El licenciado después que concluyó de hablar siguió riendo un buen rato.

---Entonces es preciso emprender la conquista de Chaperón.

---Cosa muy fácil, pero facilísima\ldots{} tenga usted de su parte a
Calomarde y a Quesada y échese usted a dormir, señora.

---Es que ahora---repuso la dama muy preocupada,---dicen que apretarán
mucho la cuerda y que no perdonarán a nadie.

---Sí, el Gobierno necesita ahora más que nunca demostrar gran celo para
perseguir a los liberales. Los voluntarios realistas le acusan de que
ahorca poco.

---¡Qué horror!---exclamó la señora con espanto.

---De que ahorca poco. Pues bien, el Gobierno se verá en el caso de
ahorcar mucho.

---¡Y a esa pobre joven\ldots!

---Esa pobre joven\ldots{} La verdad es que la causa, como causa de
conspiración, es de las que más alto piden un desenlace trágico. Ahora
me acuerdo de una circunstancia que favorece mucho su deseo de usted.

---¿Qué?

---Anoche nos han traído al que figura como cómplice de la tunantuela.

---¿Sarmiento?\ldots{} le conozco---dijo la señora desanimándose.---Es
un pobre tonto, a quien la Comisión no puede considerar como reo.

---Poquito a poco. La ley está de tal modo redactada, que yo no me
atrevería a absolverle. Puesto que la señora quiere que yo dé unos
cuantos toques a la causa, se hará. Nada se pierde en ello. Verá usted
cómo resulta que el culpable de todo es Sarmiento, y que la joven jamás
ha roto un plato.

---Buena idea, si ese infeliz estuviese en su claro juicio; si tuviera
responsabilidad\ldots{}

---Ahí está el \emph{quid}. Anoche dijo Chaperón que iba a mandarle al
Nuncio de Toledo. Puede que persista en esta humanitaria idea. Allá
veremos\ldots{} Ya sabe usted que la cabeza de mi jefe es una
berroqueña.

---Lo que sé---dijo la dama en tono humorístico,---es que su jefe de
usted es uno de los hombres más brutos que han comido pan en el mundo.

---Señora---repuso Lobo como quien da expansión a un sentimiento
contenido por el deber,---yo le aseguro a usted que no come cebada por
no dar qué decir. Así anda el Reino en manos de esta gente.
Malaventurados los que se ven en la dura necesidad de servirle, como yo,
por ejemplo, que pudiendo estar pavoneándome en una sala del Consejo,
cual lo piden mis merecimientos y servicios, me hallo reducido a la
triste condición en que usted me ve. ¡Ay! señora de mi vida---añadió
haciendo pucheros.---Esto me pasa por haber sido una mala cabeza, por
haber fluctuado entre los dos partidos sin decidirme por ninguno. Desde
la guerra vengo haciendo quiebros como un bailarín sin saber a qué
faldón agarrarme. Mis vacilaciones, mi timidez natural, y ¿por qué no
decirlo? mi honradez me han traído al estado en que me veo, simple
secretario de un Chaperón, yo que llegué a posarme en la sala de Mil y
quinientas\ldots{} ¡Y que no he pasado yo congojas en gracia de
Dios!\ldots---al decir esto movía la cabeza como los muñecos que la
tienen pegada al cuerpo por una espiral de alambre.---¡Sin destino y
teniendo que mantener esposa, dos suegras y once becerros mamones! Es
verdad que Dios se llevó de mi casa a la gente mayor; pero vinieron
nietecillos\ldots{} ¡y qué casorios los de mis hijas!\ldots{} En fin,
señora, me callo, porque si sigo hablando de mis lástimas ha de llorar
hasta el tintero. ¡Qué hubiera sido de mí sin la pensión que me dio
durante tres años el Sr.~de Araceli, y sin el favor de personas
generosas como usted y otras a quienes viviré eternamente
agradecido!\ldots{} Pero me callo, positivamente me callo, porque si
siguiera hablando\ldots{}

---Una persona de tantas tretas como usted---manifestó Jenara poco
atenta a las lamentaciones del curial,---puede ingeniarse para que yo
vea satisfecho mi deseo. Estoy segura de que no he de quedar
descontenta.

---En estos tiempos, señora, ¿quién es el guapo que puede dar una
seguridad? ¿No ve usted que todo está sujeto al capricho?

Jenara, vagamente distraída, contemplaba el cefalópodo formado por la
humedad sobre el retrato del Monarca. De repente sonaron golpes en la
puerta y una voz gritó:

---El señor Presidente.

---Con perdón de usted, señora---dijo levantándose.---Ya está ahí ese
Judas Iscariote. Tengo que ir al despacho.

El licenciado salió un momento como para curiosear, y al poco rato
volvió corriendo con su pasito menudo y vacilante.

---Señora---dijo a su amiga en tono de alarma.---Con Chaperón ha entrado
el Sr. Garrote, su digno esposo de usted.

---¡Jesús, María y José!---exclamó la dama llena de turbación.---Me voy,
me voy\ldots{} ¿Por dónde salgo, Sr.~Lobo, de modo que no
encuentre\ldots?

---Por aquí, por aquí\ldots---manifestó el curial guiándola fuera de la
pieza por oscuros pasillos, donde había alcarrazas de agua, muebles
viejos y esteras sin uso.---No es muy bueno el tránsito, pero saldrá
usted a la calle de los Autores sin tropezar con bestias cornúpetas
mayores ni menores.

---Ya, ya veo la salida\ldots{} Adiós, gracias, Sr.~Lobo. Vaya usted
luego por mi casa---dijo la señora recogiéndose la falda para andar más
ligera.

Al poner el pie en el callejón, pasaba por delante de ella, tocándola,
una figura imponente y majestuosa. Cruzáronse dos exclamaciones de
sorpresa.

---¡Señora!

---¡Padre Alelí!\ldots{}

Era un fraile de la Merced, alto, huesudo, muy viejo, de vacilante paso,
cuerpo no muy derecho, y una carilla regocijada y con visos de haber
sido muy graciosa, la cual resaltaba más sobre el hábito blanco de
elegantes pliegues. Apoyábase el caduco varón en un palo, y al andar
movía la cabeza, mejor dicho, se le movía la cabeza, cual si su cuello
fuera más que cuello una bisagra.

---¿A dónde va el viejecito?---le dijo la señora con bondad.

---¿Y usted de dónde viene? Sin duda de interceder por algún
desgraciado. ¡Qué excelente corazón!

---Precisamente de eso vengo.

---Pues yo voy a la cárcel, a visitar a los pobres presos. Dicen que han
entrado muchos ayer. Voy a verlos. Ya sabe usted que auxilio a lo
condenados a muerte.

---Pues a mí me ha entrado el antojo de visitar también a los presos.

---¡Oh! magnánimo espíritu\ldots{} Vamos, señora\ldots{} Pero, tate,
tate, no mueva usted los piececillos con tanta presteza, que no puedo
seguirla. Estoy tan gotoso, señora mía, que cada vez que auxilio a uno
de estos infelices, me parece que veo en él a un compañero de viaje.

Después de recorrer medio Madrid con la pausa que la andadura de Su
Paternidad exigía, entraron en la cárcel. Al subir por la inmunda
escalera, la dama ofreció su brazo al anciano que lo aceptó
bondadosamente, diciendo:

---Gracias\ldots{} Si estos escalones fueran los del cielo, no me
costaría más trabajo subirlos\ldots{} Gracias: se reirán de esta pareja;
¿pero qué nos importa? Yo bendigo este hermoso brazo que se presta a
servir de apoyo a la ancianidad.

\hypertarget{xx}{%
\chapter{XX}\label{xx}}

Chaperón entró en su despacho con las manos a la espalda, los ojos fijos
en el suelo, el ceño fruncido, el labio inferior montado sobre su
compañero, la tez pálida y muy apretadas las mandíbulas, cuyos tendones
se movían bajo la piel como las teclas de un piano. Detrás de él
entraron el coronel Garrote (de ejército) y el capitán de voluntarios
realistas Francisco Romo, ambos de uniforme. En el despacho aguardaba
holgazanamente recostado en un sofá de paja el diestro cortesano de
1815, Bragas de Pipaón.

A tiro de fusil se conocía que el insigne cuadrillero del absolutismo
estaba sofocadísimo por causa de reciente disgusto o altercado. ¡Ay de
los desgraciados presos! ¡Si los diablillos menores temblaban al ver a
su Lucifer, cómo temblarían los reos si le vieran!

Garrote y Romo no se sentaron. También hallábanse agitados.

---No volverá a pasar, yo juro que no volverá a pasar---dijo Chaperón
dando una gran patada.---Por vida del Santísimo Sacramento\ldots{} vaya
un pago, vaya un pago que se da a los que lealmente sirven al Trono.

Hubiérase creído que la estera era el Trono, a juzgar por la furia con
que la pisoteaba el gran esbirro.

---Todavía---añadió mirando con atónitos ojos a sus amigos,---le parece
que no hago bastante; que dejo vivir y respirar demasiado a los
liberales. ¿Hase visto injusticia semejante? «Señor Chaperón, usted no
hace nada, Sr.~Chaperón, las conspiraciones crecen y usted no acierta a
sofocarlas. Los conspiradores le tiran de la nariz y usted no los
ve\ldots» «Pero Sr.~Calomarde, ¿me quiere usted decir cómo se persigue a
los liberales, a los comuneros, a los milicianos, a los compradores de
bienes nacionales, a los clérigos secularizados, a toda la canalla, en
fin? ¿Puede hacerse más de lo que yo hago? ¿Cree usted que esa polilla
se extirpa en cuatro días?\ldots» Pues que no, y que no, que para arriba
y que para abajo, que yo soy tibio, que soy benigno, que dejo hacer, que
no tengo ojos de lince, que se me escapan los más gordos, que me trago
los camellos y pongo a colar a los mosquitos. Y vaya usted a sacarlos de
ahí. Convénzales usted de que no es posible hacer otra cosa, a menos que
no salgamos a la calle con una compañía y fusilemos a todo el que
pase\ldots{} Esta misma noche he de procurar ver a Su Majestad y decirle
que si encuentra otro que le sirva mejor que yo en este puesto, le
coloque en lugar mío. Francisco Chaperón no consentirá otra vez que D.
Tadeo Calomarde le llame zanguango.

---No hay que tomarlo tan por la tremenda---dijo Garrote con su natural
franqueza, apoyándose en el sable.---Si el Ministro y el Rey se quejan
de usted, me parece injusto\ldots{} ahora si se quejan de la
organización que se ha dado a la Comisión Militar, me parece que están
acertados.

---Eso, eso es---afirmó Romo sin variar su impasible semblante.

---No lo entiendo---dijo D. Francisco.

---Es muy sencillo. Las Comisiones están organizadas de tal modo que
aquí se eternizan las causas. Papeles y más papeles\ldots{} Los presos
se pudren en los calabozos\ldots{} ¡Demonio de rutina! Para que esto
marchara bien, sería preciso que los procedimientos fueran más
ejecutivos, enteramente militares, como en un campo de batalla\ldots{}
¿Me entiende usted?\ldots{} ¿Se quiere arrancar de cuajo la revolución?
Pues no hay más que un medio.---(Al decir esto se puso en el centro de
la sala accionando como un jefe que da órdenes perentorias).---A ver,
tú, ¿has conspirado contra el Gobierno de Su Majestad? Pues ven
acá\ldots{} Ea, fusilarme a esta buena pieza. A ver, tú: ¿has gritado
«viva la Constitución»?\ldots{} Ven acá, te vamos a apretar el gaznate
para que no vuelvas a gritar\ldots{} Y tú, ¿qué has hecho? ¿compraste
bienes del clero? Diez años de presidio\ldots{} Y nada más. Entonces sí
que se acababan pronto las conspiraciones. Juro a usted que no se había
de encontrar un revolucionario aunque lo buscaran a siete estados bajo
tierra.

Chaperón hundía la barba en el pecho acariciándosela con su derecha
mano.

---Lo que dice el amigo Navarro---afirmó Romo,---no tiene vuelta de
hoja. Nosotros los voluntarios realistas hemos salvado al Rey. Los
franceses no habrían hecho nada sin nosotros. Somos el sostén del Trono,
las columnas de la Fe católica. Pues bien, dígase con franqueza, si
tenemos las preeminencias que nos corresponden. Los liberales nos
insultan y no se les castiga.

Chaperón hizo un brusco movimiento. Iba a responder.

---Quiero decir, que no se les castiga como merecen---añadió el
voluntario realista.---En vez de tener absoluta confianza en nosotros,
se nos quiere sujetar a reglamentos como los de la Milicia Nacional. Nos
miran con desconfianza\ldots{} ¿y por qué? porque no permitimos que se
falte al respeto a Su Majestad y a la Fe católica, porque estamos
siempre en primera línea cuando se trata de sofocar una rebelión o de
precaverla. Nuestro criterio debiera ser el criterio del Gobierno. ¿Y
cuál es nuestro criterio? Pues es ni más ni menos que exterminio
absoluto, no perdonar a nadie, cortar toda cabeza que se levante un
poco, aplacar todo chillido que sobresalga. ¡Ah! señores, si así se
hiciera otro gallo nos cantara. Pero no se hace. Aunque el Sr.~Chaperón
se enfade, yo repito que hay lenidad, mucha lenidad, que no se castiga a
nadie, que las causas se eternizan, que dentro de poco los negros han de
reírse en nuestras barbas, que así no se puede estar, que peligra el
Trono, la Fe católica\ldots{} Y no lo digo yo solo, lo dice todo el
instituto de voluntarios realistas, a que me glorio de
pertenecer\ldots{} Y estamos trinando, sí, señor Chaperón, trinando
porque usted no castiga como debiera castigar.

El hombre oscuro emitió su opinión sin inmutarse, y las palabras salían
de su boca como salen de una cárcel los alaridos de dolor sin que el
edificio ría ni llore. Tan sólo al fin, cuando más vehemente estaba,
viose que amarilleaba más el globo de sus ojos y que sus violados labios
se secaban un poco. Después pareció que seguía mascullando como en él
era costumbre, el orujo amargo de que alimentaba su bilis.

---Todo sea por Dios---dijo Chaperón, alzando del suelo los ojos y dando
un suspiro.---¡Y de tantos males tengo yo la culpa!\ldots{} Ya verán
quién es Calleja.

Diciendo esto se encaminó a la mesa. Ya el licenciado Lobo ocupaba en
ella su puesto.

---A ver, despachemos esas causas---dijo al leguleyo.

---Aquí tenemos algunas---repuso Lobo poniendo su mano sobre un montón
de infamia,---a las que no falta sino que Vuecencia falle.

---A ver, a ver. Con bonito humor me cogen. Vamos a prepararle su
trabajo al fiscal.

Lobo tomó el primer legajo y dijo:

---Número 241. Esta es la causa de aquel comunero que propuso establece
la república.

---Horca---dijo Chaperón prontamente y con voz de mando, como un oficial
que a las tropas dice «fuego».---Sea condenado a la pena ordinaria de
horca.

---Número 242---añadió Lobo tomando otro legajo.---Causa de Simón
Lozano, por irreverencias a una imagen de la Virgen.

---Horca---gruñó Chaperón, cual si se le pudriera la palabra en el
cuerpo.---Adelante.

---Número 243. Causa de la mujer y de la hija de Simón Lozano, acusadas
de no haber delatado a su marido.

---Diez años de galera.

---Número 244. Causa de Pedro Errazu por expresiones subversivas en
estado de embriaguez.

---El estado de embriaguez no vale. ¡Horca! Añada usted que sea
descuartizado.

---Número 245. Causa de Gregorio Fernández Retamosa, por haber besado el
sitio donde estuvo la lápida de la Constitución.

---Diez años de presidio\ldots{} no, doce, doce.

---Número 246. Causa de Andrés Rosado por haber exclamado: «¡Muera el
Rey!»

---Horca.

---Número 247. Causa del sargento José Rodríguez por haber elogiado la
Constitución.

---Horca.

---248. Causa de su compañero Vicente Ponce de León, por haber
permanecido en silencio cuando Rodríguez elogió la Constitución.

---Diez años de presidio y que asista a la ejecución de Rodríguez,
llevando al cuello el libro de la Constitución que quemará el verdugo.

---249. Causa de D. Benigno Cordero y de su hija Elena Cordero por
conspiración\ldots{}

---¡Alto!---gritó una voz desde el otro extremo de la sala.

Era la de Pipaón que se adelantó extendiendo su mano como una divinidad
protectora.

---Si es criminal perdonar al culpable, criminal es, criminalísimo,
condenar al inocente---dijo con énfasis.---Yo me opongo, y mientras
tenga un hálito de vida alzaré mi voz en defensa de la inocencia.

---Vaya, recomendaciones habemos---observó Garrote riendo.---Eso no
puede faltar en España. Favorcillo, amistades, empeños\ldots{} Mientras
tengamos eso, no habrá justicia en nuestro país\ldots{} ¡Recomendación!
Yo empezaría por ahorcar esa palabra. Me repugna.

---No se trata aquí de recomendar a un amigo a la generosidad de D.
Francisco---dijo el cortesano poniéndose rojo de tanto énfasis.---Es que
la inocencia de D. Benigno está ya tan clara como la diáfana luz del
día. ¿Le consta a usted que no?

---A mí no me consta nada---repuso Navarro alzando los hombros.---Si no
le conozco\ldots{} Pero me ha llamado la atención una cosa, y es que se
han sentenciado en este mismo momento varias causas por desacato, por
exclamaciones, por besos, por sacrilegio, sin que hayamos oído una voz
que se interese por los reos; pero aparece una causa de conspiración (al
decir esto dio una gran palmada) y en seguida vemos venir la
recomendación. Si no hay gente más feliz que los conspiradores\ldots{}
Yo no sé cómo se las componen, que siempre encuentran amigos.

---Hablemos claro---dijo el cortesano tragando saliva.---Yo no
recomiendo a un conspirador: solamente afirmo que el Sr.~Cordero no ha
conspirado jamás. ¿No está el Sr.~Chaperón convencido de ello? ¿No se ha
demostrado que los verdaderos culpables son otros?

---Este es un caso extraño---afirmó D. Francisco.---Cierto es que los
Corderos son inocentes.

---Bueno, si hay realmente inocencia, no digo nada---objetó sonriendo
Navarro.---Pero es particular que sólo los que conspiran resulten
inocentes.

---Sólo los que conspiran---añadió Romo en tono del más perfecto
asentimiento.

---¿Pues qué?---dijo Pipaón con mayor dosis de énfasis y encarándose con
el voluntario realista.---¿No será usted capaz de sostener que nuestro
amigo D. Benigno y su hija son inocentes del crimen que les imputó un
delator desconocido?

Romo miró a todos uno tras otro impasiblemente. Jamás había su rostro
aparecido más frío, más oscuro, de más difícil definición que en aquel
instante. Era como un papel blanco, en cuya superficie busca en vano la
observación una frase, una línea, un rasgo, un punto.

---Bien conocen todos---dijo con tranquilo tono,---mi carácter leal, mi
amor a la veracidad. Para mí la verdad está por encima de todos los
afectos, hasta de los más sagrados. Soy así y no lo puedo remediar. ¿Por
qué me llaman los compañeros, \emph{el voluntario de bronce}? Porque soy
como de bronce, señores; a mí no hay quien me tuerza, ni me doble, ni me
funda. ¿Se trata de una cosa que es verdad? Pues verdad y nada más que
verdad. (Romo hizo tal gesto con el dedo índice que parecía querer
agujerear el suelo). Si mi padre falta y me lo preguntan digo que sí. No
significa esto que sea insensible, no. Yo también tengo mis blanduras.
Soy de bronce y tengo mi cardenillo\ldots{} (el hombre duro y lóbrego se
conmovía). Yo también sé sentir. Bien saben todos que quiero mucho a D.
Benigno Cordero. Bien saben todos que trabajé porque volviera a Madrid.
Pues bien, supongamos que me preguntan ahora si creo que D. Benigno
Cordero conspiraba: yo responderé\ldots{} que no lo sé.

Díjolo de tal modo, que dudando afirmaba. Lo que el hombre de bronce
llamaba su cardenillo, si para él era un afecto, para los demás podía
ser un veneno.

---¡Que no lo sabe!---exclamó Pipaón con ira.---Por fuerza usted ha
perdido el juicio.

---No lo sé---repitió el voluntario mirando al suelo.---Si no lo sé,
¿por qué he de decir que lo sé, faltando a mi conciencia? ¿Qué importan
mis afectos ante la verdad? Yo cojo el corazón y lo cierro como se
cierra un libro prohibido, y no lo vuelvo a abrir aunque me
muera\ldots{} porque no tengo que fijar los ojos más que en la
verdad\ldots{} y la verdad es antes que nada, y maldito sea el corazón
si sirve para apartarnos de la verdad.

---El amigo Romo---dijo Navarro,---nos da un ejemplo de honradez que es
muy raro y tendrá muy pocos imitadores.

---Pues yo---afirmó Pipaón subiendo todavía algunos puntos en la escala
de su énfasis,---digo que si la verdad está sobre el corazón, la caridad
está sobre la verdad\ldots{} Pero no necesitan los Corderos implorar la
caridad sino alegar su derecho, porque son inocentes. Señor D. Francisco
Chaperón, ¿no cree usted que son inocentes?

---Yo creo que sí---replicó el Presidente con acento de convicción.---El
delito que a ellos se imputaba ha sido cometido por otras personas. Así
consta por declaración de los mismos reos. La delación ha sido
equivocada.

---¿Lo ven ustedes?---dijo Bragas rompiéndose las manos una con otra.

---Por lo que veo, el delito no desaparece---indicó Garrote.---Lo que
hay es un cambio de delincuente.

---Eso es, una sustitución de delincuente.

---¿Y se castigará?---preguntó con incredulidad el coronel del ejército
de la Fe.

---¡Bueno fuera que no!\ldots{} ¿Estamos en Babia?\ldots{} A fe que
tengo hoy humor de blanduras. Siga usted, Lobo.

---Causa de D. Benigno Cordero.

Chaperón meditó un rato. Después, tomando un tonillo de jurisconsulto
que emite parecer muy docto, habló así:

---Absolución. Solamente les condeno a dos meses de cárcel, por no haber
denunciado las visitas de Seudoquis al piso segundo de su misma casa.

---¡Qué bobería!---murmuró por lo bajo Pipaón, arqueando las cejas.

---Número 251. Causa de D. Ángel Seudoquis---cantó el licenciado.

---Diez años de prisión y pena de degradación militar, por no haber dado
parte a la autoridad de la llegada de su hermano a Madrid\ldots{} Las
cartas que se le han encontrado son amorosas\ldots{} No hay la menor
alusión a las cosas políticas. Adelante.

---Número 252. Causa de Soledad Gil de la Cuadra y de Patricio
Sarmiento.

---Es la más rara que se ha conocido en esta Comisión.

---Sí, la más rara---añadió Romo,---porque presenta un caso nunca visto,
señores, el caso más admirable de abnegación de que es capaz el espíritu
humano. Figúrense ustedes una joven inocente que por salvar a dos
personas que le han hecho favores se declara culpable\ldots{} mentira
pura\ldots{} una mentira sublime, pero mentira al fin.

---Abnegación---indicó Chaperón con cierto aturdimiento.---¿Qué
entendemos nosotros de eso? Cosas del fuero interno, ¿no es verdad,
Lobo? Al grano, digo yo, es decir, a los hechos y a la ley. El delito es
indudable. La prueba es indudable. Tenemos un reo convicto y confeso.
Caiga sobre él la espada inexorable de la justicia, ¿no es verdad, Lobo?

El licenciado no decía nada.

---Pero aparecen ahí dos personas---dijo Navarro.

---Una joven y un viejo tonto. Ella parece la más culpable. Del
mentecato de Sarmiento no debemos ocuparnos. Sería gran mengua para este
tribunal.

---Si tras de lo desacreditado que está---dijo Navarro con sorna,---da
en la flor de soltar a los cuerdos y ajusticiar a los imbéciles\ldots{}

---Nada, nada. Adelante---manifestó Chaperón con
impaciencia.---Despachemos eso.

---Soledad Gil---cantó Lobo.

---Pena ordinaria de horca. Y sea conducido D. Patricio a la casa de
locos de Toledo. Esto propondré a la Sala pasado mañana.

Miró a sus amigos con expresión de orgullo semejante a la que debió de
tener Salomón después de dictar su célebre fallo.

---Me parece bien---afirmó Garrote.

---Admirablemente---dijo Pipaón, tranquilizado ya respecto a la suerte
de sus amigos y fiando en que le sería fácil después librarles de los
dos meses de cárcel.

---Y yo digo que habrá no poca ligereza en el tribunal si aprueba
eso---insinuó con hosca timidez Romo.

---¡Ligereza!

---Sí; averígüese bien si la de Gil de la Cuadra es culpable o no.

---Ella misma lo asegura.

---Pues yo la desmentiré, sí señor, la desmentiré.

---Este es un hombre que no duerme si no ve ahorcados a sus amigos.

---Aquí no se trata de amigos---exclamó Romo con cierto calor que se
podía tomar por rabia.---Yo no tengo amigos en estas cuestiones; yo no
soy amigo de nadie, más que del Rey y de la sacratísima Fe católica.
Romo, \emph{el voluntario de bronce}, no tiene amistades más que con la
justicia y con la verdad. Y ya que hablamos del Sr.~Cordero, diré que
dejé de frecuentar su casa desde que vi en ella ciertas cosas.

---¿Qué ha visto usted?---preguntó vivamente el cortesano, tan sofocado
por su enojo como por su collarín metálico que le condenaba
elegantemente a garrote.

---No tengo para qué decirlo ahora---repuso el voluntario volviendo la
espalda.---Está sentenciada la causa ¿para qué añadir una palabra más?

---Me parece---dijo Bragas en tono de sarcasmo,---que el amigo Romo está
durmiendo y ve visiones, como las veía el que delató a nuestros amigos.

---¿Se sabe quién los ha delatado?---preguntó Navarro al presidente de
la Comisión.---¿Es persona que merece crédito?

---Dos individuos de nuestra policía. Generalmente obran por
indicaciones de personas afectas a Su Majestad.

---Esas personas son entonces los verdaderos denunciadores.

---En efecto, esas son---dijo Romo,---a esas personas hay que agradecer
el expurgo que se está haciendo y al cual deberá su tranquilidad el
Reino. ¿Quién se atrevería a vituperar a los médicos porque dijeran:
«Córtese usted ese dedo que está gangrenado»?

---Pues si aquí no ha habido una mala inteligencia, ha habido una infame
intención---replicó Bragas firme en su puesto.---Mi amigo Cordero ha
sido víctima de una venganza.

---Usted no sabe lo que dice---afirmó Romo con desprecio.---En las
oficinas del Consejo y en los gabinetes de las damas se entenderá de
intrigar, de entorpecer la marcha de la justicia; pero de purificar el
Reino, de hacer polvo a la revolución\ldots{}

---¿Y cómo se purifica el Reino? ¿Atropellando a la inocencia,
condenando a un hombre de bien por la delación de cualquier desconocido?

---Repito que usted no sabe lo que habla---dijo Romo presentando en su
rostro creciente alteración que le hacía desconocido.---Los que pasan la
vida enredando para poner en salvo a los mayores delincuentes; los que
se entretienen en escribir billetes de recomendación para favorecer a
todos los pillos, no entienden ni entenderán nunca la rectitud del
súbdito leal que en silencio trabaja por su Rey y por la Fe católica.
Mírenme a la cara (el Sr.~Romo estaba horrible), para que se vea que sé
afrontar con orgullo toda clase de responsabilidades. Y para que no
duden de la verdad de una delación por suponerla oscura, se aclarará, sí
señores, se aclarará\ldots{} Mírenme a la cara (cada vez era más
horrible); yo no oculto nada. Para que se vea si la delación de Cordero
es una farsa, declaro que la he hecho yo.

Al decir \emph{yo} diose un gran golpe en el pecho que retumbó como una
caja vacía. Brillaban sus ojos con extraño fulgor desconocido; se había
transfigurado, y la cólera iluminaba sus facciones antes oscuras. El
lóbrego edificio donde jamás se veía claridad, echaba por todos sus
huecos la lumbre amarillenta y sulfúrea de una cámara infernal. Haciendo
un gesto de amenaza, se expresó así:

---El que sea guapo que me desmienta.

Y salió sin añadir una palabra. Pipaón, que era hombre de muy pocos
hígados como se habrá podido observar en otras partes de esta historia,
se quedó perplejo, pero afectaba la indecisión de un valiente que medita
las atrocidades que ha de hacer, Chaperón dijo:

---No se decida nada sobre esas dos causas. Quédense para otro día.

Un diablillo menor entró muy gozoso, diciendo a su jefe:

---Acabamos de recibir una gran noticia de la Superintendencia. Rafael
Seudoquis ha sido preso en Valdemoro. Esta noche llegará a Madrid.

---¡Suceso providencial!---exclamó D. Francisco con júbilo.---Cayó el
principal pez. Vea usted, Sr.~Pipaón, de qué manera vamos a salir pronto
de dudas. Sobre ese sí que no habrá dimes y diretes. Apunte usted,
Lobo\ldots{} horca ¡tres veces horca!

---Saldremos de dudas---indicó Pipaón decidiéndose a aflojar la hebilla
de su collarín metálico, cuya presión se le hacía insoportable.---Ese
hombre es la providencia de mis amigos.

\hypertarget{xxi}{%
\chapter{XXI}\label{xxi}}

Decir cuánto padeció el magnánimo espíritu del Presidente de la Comisión
Militar en aquellos días fuera imposible. Había en el fondo, muy en el
fondo de su alma, perdido entre el légamo de los más perversos
sentimientos, un poco de equidad o rectitud. Verdad es que esta virtud
era un diminuto corpúsculo, un ser rudimentario, como las \emph{moneras}
de que nos habla la ciencia; pero su pequeñez extraordinaria no
amenguaba la poderosa fuerza expansiva de aquel organismo, y a veces se
la veía extenderse tratando de luchar en las tinieblas con el cieno que
la oprimía, y de abrirse paso por entre la masa de yerbas inmundas y
groseras existencias que llenaban todo el vaso de la conciencia
chaperoniana.

Convencido de la inocencia de Cordero y de su hija, D. Francisco sentía
que la mónera de su alma le gritaba con vocecita casi imperceptible que
les pusiera en libertad. Sus compañeros de Comisión, aunque generalmente
deliberaban y votaban por fórmula, dejándole a él toda la gloria de la
iniciativa (y reservándose sólo los sueldos), opinaban también que
Cordero debía ser absuelto. Los últimos escrúpulos de D. Francisco se
disiparon con las declaraciones de Rafael Seudoquis, el cual, si al
principio se mostró reservado, después por la virtud de un hábil
interrogatorio capcioso, echó gran luz sobre el suceso de las cartas,
dejando ver la inculpabilidad absoluta del tendero de encajes y de su
hija.

La declaración de Soledad, la de Seudoquis, la opinión de todos los
individuos de la Comisión Militar, las gestiones del habilidoso Bragas y
su propia conciencia (guiada esta vez por el mísero corpúsculo que
crecía en el fondo de ella) decidieron a D. Francisco a firmar la orden
de excarcelación, novedad inaudita en aquellas diabólicas regiones, cuya
semejanza con el infierno se completaba por la imposibilidad de que
salieran los que entraban.

Pero aquí comenzaron las tribulaciones del funcionario absolutista, (y
no es forzoso ponernos de su parte) porque el mismo día en que dictara
la excarcelación, recibió tales vejaciones y desaires de sus amigos los
voluntarios realistas, que estuvo a riesgo de reventar de cólera, aunque
la desahogaba con votos y ternos, asociando la vida del Santísimo
Sacramento a todas las picardías habidas y por haber. Al ir por la
mañana al tribunal para oír misa vio un pasquín infamante en la esquina
de la parroquia de San Nicolás, en el cual documento se hablaba de las
onzas de oro que percibía el brigadier traga-muertos por cada preso que
soltaba. Recibió diversos anónimos amenazándole con descubrir sus
artimañas, y supo que en el cuerpo de guardia habían pintado los
voluntarios su simpática imagen pendiente de la horca con amenos
versículos al pie.

---Esos bergantes, a quienes se permite la honra de parecerse a los
soldados---decía para sí midiendo con las piernas al modo del compás, el
suelo de su despacho,---se van a figurar que reinan con Fernando
VII\ldots{} Sí\ldots{} como no les corten las alas, ya verán qué bonito
se va a poner esto\ldots{} ¿Tenemos aquí otra vez la Milicia Nacional?
porque es lo mismo; llámese blanco, llámese negro, es exactamente lo
mismo. Miserables saltimbanquis, ¿de qué me acusáis? ¿de que no castigo
a los conspiradores? ¿Pues qué he de hacer, marmolejos con fusil, sino
castigarlos? ¿Entendéis vosotros de ley, borrachos? Que no castigo las
conspiraciones\ldots{} que desde que sucedió lo de Almería y Tarifa, no
ha sido condenado ningún conspirador. ¿Pues no está ahí Seudoquis? ¿No
están también sus cómplices, sus infames cómplices?\ldots{} ¡porque
estos sí que son malos! Ahí les tenéis, presos por conspiración.
¿Queréis más, ladrones de caminos? Ahí tenéis a Seudoquis, a quien
veréis en la horca, ahí tenéis a la muchachuela a quien veréis en la
horca\ldots{} ¿Queréis más carne muerta, cuervos? ¡Por vida del
Santísimo! ¿queréis también al imbécil?\ldots{} Sr.~Lobo, a ver esa
causa.

Lobo, que silenciosamente cortaba su pluma, diole las últimas
raspaduras, y hojeó después varios legajos.

---Al punto voy, excelentísimo señor---dijo melifluamente.

Aquel día se notaba en el licenciado un extraordinario recrudecimiento
de amabilidad y oficiosa condescendencia.

---Esa endiablada causa, excelentísimo señor\ldots{} aquí la tenemos.
Abulta, abulta que es un primor. Ya se ve: como que está llena de
picardías\ldots{} No vaya a creer Vuecencia que consta de dos o tres
pliegos como algunas. Esto es un archivo. Y que he trabajado poco en
gracia de Dios\ldots{} No, no es tan fácil hinchar un perro.

---De Seudoquis no se hable---dijo Chaperón tomando asiento frente a su
asesor, e implantando los dos codos sobre la mesa para unir las manos
arriba, de modo que resultaba la perfecta imagen de una horca.---Ese
está juzgado. En cuanto a la joven, su culpabilidad es indudable, y yo
creo que la debemos ahorcarla también. ¿Qué le parece a usted,
licenciado de todos los demonios?

---¿Quiere Vuecencia que le hable como jurisconsulto o como
amigo?---preguntó Lobo con cierto misterio.

---Como usted quiera, con tal que hable claro.

---¿Como jurisconsulto?

---Dale.

---Como asesor opino\ldots{} Sr.~D. Francisco, haga usted lo que más le
acomode.

Ahora, si me consulta Vuecencia como amigo\ldots{} ¿Quiere que le hable
con completa claridad y confianza?

---Sí.

---Pues en confianza, si la Comisión ahorca a esa madamita, me parece
que hace una gran barbaridad.

---¿Eh?

---Una barbaridad de \emph{á folio}.

---¿Por qué?

---Porque es inocente.

---¿Esas tenemos?\ldots{} ¡Por vida del Santísimo!---exclamó con
ira,---como usted no tiene la responsabilidad de este delicado cargo;
como a usted no le acusan de tibieza, ni de benignidad, ni de
venalidad\ldots{} Ya les echaré yo un lazo a mis detractores\ldots{}
pero vamos al caso. ¿Dice usted que es inocente?

---Sí, y lo pruebo---repuso Lobo tomando la más solemne expresión de
gravedad judicial.

---Lo prueba, lo prueba\ldots---dijo Chaperón con sarcástica
bufonería.---Lo que usted probará será el aguardiente si se lo dan.
Grandísimo borracho, escriba usted, escriba usted mi fallo.

---Escribiremos, excelentísimo señor---dijo Lobo resignadamente, como el
que habiendo recibido una coz no se cree en el caso de devolver otra.

Chaperón encendió un cigarro. Después de la primera chupada, dijo:

---La condeno a pena ordinaria de horca.

Luego se quedó un rato contemplando la primera bocanada de humo, que
salía del horrendo cráter de sus labios.

\hypertarget{xxii}{%
\chapter{XXII}\label{xxii}}

La primera noche de su encierro D. Patricio y su compañera de cárcel no
durmieron.

La prisión no pecaba ciertamente de estrecha; pero en luces competía con
la noche absoluta, siendo difícil asegurar quién llevaba la ventaja, si
bien al filo del medio día parecía vencer la cárcel a su rival a causa
de ciertas claridades que se entraban por el enrejado ventanillo,
temerosas y sobrecogidas de miedo, y embozadas misteriosamente en
espesas capas de telarañas. Dichas claridades recorrían con pasos de
ladrón el techo y las paredes, miraban con cautela a los negros rincones
y al piso, y a eso de las dos o las tres volvían la espalda para
retirarse dejando la fúnebre pieza a oscuras. Dos sillas, una tarima
pegada a la pared y una mesa constituían el mísero ajuar. Los ladrillos
del suelo respondían siempre a cada pisada de los presos con un
movimiento de balanza y un sonido seco, señales ciertas de su disgusto
por verse molestados en su posición horizontal. Seguramente ellos, como
toda la casa, habrían vuelto con gozo a poder de los Padres del
Salvador, sus antiguos dueños, hombres pacíficos que jamás lloraban, ni
hacían escándalos, ni pateaban desesperadamente, ni pedían a gritos que
los sacaran de allí.

La primera noche, como hemos dicho, Sarmiento y su amiga, no muy bien
avenidos con su residencia en tan ameno sitio, no durmieron nada y
hablaron poco. El viejo, como si su entusiasta locuacidad delante del
tribunal le hubiera agotado las fuerzas y secado el rico manantial de
sus ideas, estaba taciturno. Los excesos de espontaneidad producían en
él una reacción sobre sí mismo. Después de divagar por el exterior,
libre, sin freno, cual andante aventurero que todo lo atropella, se
metía en sí como cartujo. Soledad también sufría la reacción
correspondiente a una espontaneidad que sin duda le estaba pareciendo
excesiva. Pero su espíritu estaba tranquilo; su pensamiento, después de
pasar revista con cierto desdén a los sucesos próximos, se remontaba
orgullosamente a las alturas desde donde pudiera descubrir horizontes
más gratos y personas más dignas de ocuparlo. Había llegado a adquirir
la certidumbre de un trágico fin; pero lejos de sentir el terror propio
de tales casos y muy natural en una débil muchacha inocente, se
sobrepuso con ánimo grandioso a la situación; supo mirar desde tan alto
su propia persona, su prisión, su proceso, sus verdugos, las causas e
incidentes de aquella lamentable aventura, que fue creciendo, creciendo,
y bien pronto cuanto la rodeaba, incluso Madrid, la Nación y el mundo
entero, se quedó enano. ¡Admirable resultado del espíritu religioso y de
la elasticidad del corazón, cuya magnitud, cuando él se decide a crecer,
se pierde en las indefinidas dimensiones de lo infinito!

Al día siguiente, D. Patricio, que había llegado ya al límite de su
tétrico silencio y no podía permanecer más tiempo mudo, se expresó así:

---Hija mía, me parece que esto es hecho.

---¿Por qué no te echas a ver si duermes un ratito?---le dijo Sola con
bondad.---La tarima no es como las camas de casa; pero a falta de otra
cosa\ldots{}

---¡Dormir\ldots{} dormir yo!---exclamó Sarmiento con voz
lastimera.---Ya el dormir profundo está cercano. Te digo que esto es
hecho.

---Sí, esto no puede ser más hecho\ldots{} Ya que no quieres levantarte
del suelo, al menos tiéndete de largo y recuesta esa pobre cabecita
sobre mis rodillas.

Sola, que estaba sentada en la silla, se puso en el suelo, dando después
una palmada sobre su falda, para indicar que podía servir de blanda
almohada. D. Patricio, sentado contra la pared, con las rodillas en
alto, los brazos cruzados sobre aquellas y la barba sobre los brazos,
formando con su cuerpo dos ángulos opuestos y muy agudos, no quiso dejar
tan encantadora postura de \emph{zig-zag}.

---No, niña mía; aquí estoy bien. Lo que te digo es que esto es hecho.

---Se me figura que estás cobarde, viejecillo tonto.

---¡Cobarde yo!---exclamó Sarmiento con un rugido.---No me lo digas otra
vez, porque creeré que me insultas.

---Como te he visto tan parlanchín delante de los jueces y ahora tan
callado\ldots---dijo la reo extendiendo su mano en la oscuridad para
palpar la cabeza del anciano.

---Es que el alma humana tiene grandes misterios, niña querida. Desde
que entramos aquí estoy pensando una cosa.

---Con tal que no sea algún disparate, deseo saberla.

---Pues verás\ldots{} Me ocurre\ldots{} que esto es hecho, quiero decir,
que se cumple al fin mi altísimo destino, que las misteriosas veredas
trazadas por el Autor de todas las cosas y de todos los caminos, me
traen al fin a la excelsa meta a donde yo quiero ir. Pero\ldots{}

---Veamos ese pero, abuelito Sarmiento. Hasta ahora no había peros en
ese negocio del destino.

---Pero\ldots{} hay una cosa en la cual yo no había pensado bien hasta
que salimos de aquel endiablado tribunal. Respecto de mi suerte no hay
duda\ldots{} ¿pero y tú?

---No tengo yo dudas respecto a la mía---dijo Sola con seriedad.---Los
dos moriremos.

---¡Tú\ldots{} tú también!

Oyose un bramido de horror y después largo silencio.

---Eso no puede ser, eso es monstruoso, inicuo---gritó el preceptor
agitando en la oscuridad sus brazos.

---Ahora te espanta, viejecillo, y cuando estábamos en el tribunal te
parecía natural. ¿No decías, «moriremos los dos, somos mellizos de la
muerte\ldots?» ¿No dijiste también: «vamos a la horca, mientras más alta
será mejor. Así alumbraremos más. Somos los fanales del género humano»?

---Es verdad que tales cosas dije, pero has de tener en cuenta que yo me
hallaba entonces en uno de esos momentos de inspiración, en los cuales
pronuncio las sorprendentes piezas oratorias que me han dado tanta fama.
Yo no esperaba encontrarte allí. ¡Ay cuando te vi presa y condenada por
conspiradora\ldots{} porque tú has conspirado, niña de mis ojos\ldots{}
sentí una alegría tan grande!\ldots{} Me pareció que Dios te destinaba
también al martirio; pero ahora veo que esto no debe ser. Calmada
aquella estupenda exaltación, la voz de la Naturaleza ha resonado en mí,
diciéndome que no debo asociar a mi muerte a ningún otro ser. Tú eres
una muchacha oscura, y tu sacrificio no puede ser de gran beneficio a la
causa santa.

---¡Ah!---dijo Soledad sonriendo, pero sin que nadie pudiera ver su
sonrisa, como no fueran las mismas tinieblas,---ya comprendo: tienes
envidia de que vaya a quitarte un poquito de esa gloria.

---Tonta, pero tonta---replicó el anciano muy expresivamente,---si toda
has de heredarla tú, toda, toda. Si no es preciso que tú mueras como yo,
ni eso viene al caso.

---Los jueces no creerán lo mismo.

---¡Pues son unos bribones, unos!\ldots---exclamó Sarmiento ronco de ira
moviendo sus piernas para levantarse.---Yo les diré que eso no puede
ser\ldots{} Les convenceré, sí; pues no he de convencerles\ldots{}

Soledad se echó a reír.

---Te ríes\ldots{} pues esto es muy serio. Yo no creo que te condenen;
pero si te condenaran\ldots{}

Oyose un chasquido que bien podía ser causado por una gran manotada que
el preceptor se dio en la cabeza.

---Sí, me condenarán, porque mi delito de recoger y repartir las cartas
está más que probado, y si no, con la declaración tuya\ldots{}

---Yo declaré\ldots{} ¿qué declaré yo?\ldots{}

Soledad repitió a Sarmiento lo que él mismo había dicho respecto a las
cartas y a las personas que las recibieron.

---¡Yo declaré todo eso, yo!---dijo el patriota muy perplejo, como un
beodo que va poco a poco recobrando el sentido.---¿Y por eso dices que
te condenarán?\ldots{} Me parece que no estás en lo cierto. De ahí se
desprende que el delincuente, según ellos, soy yo, yo el conspirador, yo
el apóstol y el agent secreto de la libertad, y como yo tengo además la
nota de Demóstenes constitucional y de haber revuelto a media España con
mis conmovedoras arengas, de aquí que yo sea el condenado y tú no.

---Me parece---dijo la huérfana tocando el hombro de Sarmiento,---que mi
viejecito ve las cosas al revés. Yo seré condenada y él irá a un sitio
donde se vive muy bien y tratan caritativamente a los pobres.

---¡Por vida de ochenta millones de Chilindrainas!---gritó Sarmiento
poniéndose de un salto en pie,---no me digas que tú serás condenada a
muerte sin mí, porque me vuelvo loco, porque soy capaz de derribar de un
puñetazo esas férreas puertas, y hacer añicos a Chaperón y los demás
jueces, y demoler a puntapiés la cárcel y pegar fuego a Madrid
entero\ldots{} ¡Tú condenada a muerte!

---Somos los fanales del género humano.

---No, no, esa es una figura de retórica, tonta---dijo el fanático
pasando del tono trágico al familiar.---Aquí no hay más fanal que yo. Tú
me acompañas en mi última hora, me acompañas, ¿entiendes?\ldots{} pero
no mueres. ¡Morir tú!\ldots{} ¿por qué, ángel delicado e
inocente?\ldots{} ¿Habrá un juez que falle tal infamia?\ldots{} Si tu
muerte no es provechosa a la santa causa\ldots{} ¿A qué ni para qué? Yo
solo, yo solo, ¿lo entiendes bien? ¡yo solo! Este es el destino, esta la
voluntad, esto lo que está trazado en los libros inmortales, cuyos
renglones dicen a cada siglo sus grandezas, a cada generación su papel,
a cada hombre su puesto\ldots{} Pobre y desvalida niña de mis entrañas,
no me digas que vas a morir también, porque me siento cobarde, me
convierto de águila majestuosa en tímido jilguerillo, se me van las
ideas sublimes, se me achica el corazón, me trastorno todo, me siento
caer desplomándome como una torre secular que es sacudida por temblores
de tierra, me evaporo, niña mía, desfallezco, dejo de ser un Cayo Graco
para no ser más que un Juan Lanas.

Arrastrándose por el suelo, Sarmiento tanteaba con las manos en la
oscuridad hasta que dio con el cuerpo de Sola. Echándose entonces como
un perro, hundió la cabeza en su regazo. Soledad no dijo nada.

\hypertarget{xxiii}{%
\chapter{XXIII}\label{xxiii}}

Prolongábase el silencio de ambos cuando se abrió la puerta del calabozo
y entraron dos personas: el carcelero y el padre Alelí. Acostumbraba el
buen sacerdote visitar a los presos para consolarles u oírles en
confesión, y frecuentemente pasaba largos ratos con alguno de ellos
hablando de cosa festivas, con lo cual se amenguaban las tristezas de la
cárcel. Era el padre Alelí un varón realmente santo y caritativo: su
bondad se mostraba en dos especies de manías: dar almendras a los
muchachos de las calles y palique a los presos. Parecía que unos y otros
eran su familia y que no podía vivir sin ellos.

Con su fórmula de costumbre saludó a nuestros dos infortunados amigos,
que apenas distinguían en la lobreguez del cuarto la escueta figura
blanca del fraile, vaga, semi-fantástica, cual un capricho de la
oscuridad para engañar a los ojos. El padre Alelí tocó en tierra y en
las paredes con un palo, como los ciegos, y al mismo tiempo decía:

---¿Pero dónde están ustedes?\ldots{} ¡Ah! ya toco aquí un cuerpo.

Soledad, tomándole del brazo, le ofreció una silla.

---No, tengo que marcharme. Hoy he de hacer muchas visitas\ldots{}
Gracias, señora\ldots{} ¿Es usted la que llaman Soledad? Debo advertirle
una cosa que le consolará mucho: hay una dama que se interesa por
usted\ldots{} Ahí fuera está\ldots{} No la han dejado entrar; pero me
encarga diga a usted que hará todo lo posible para evitar una
desgracia\ldots{} ¡Qué señora tan angelical, qué corazón de oro!\ldots{}
¿Y el ancianito dónde está?\ldots{} Anímese usted, buen hombre. Ya, ya
me han dicho que está demente.

Oyose entonces una voz sorda e inarticulada, que parecía expresar amargo
desprecio.

---¿Está en el suelo el pobre hombre?---añadió Alelí, tanteando
suavemente con su palo.---Me parece que le siento roncar\ldots{} Si
todos tuvieran el buen abogado que este tiene\ldots{} ¡Su demencia le
salvará!\ldots{} Adiós, hijos míos, no puedo detenerme\ldots{} mañana
será más larga la visita.

Retirose y los dos presos quedaron solos todo el día. Al anochecer les
interrogaron. Después volvieron a quedar solos, ella muda y recogida,
Patricio taciturno a ratos y a ratos poseído de furor que con ninguna
especie de consuelos podía calmar su compañera. Tampoco aquella noche
durmieron gran cosa, y al día siguiente que era el 1.º de Setiembre
volvió el padre Alelí, a quien el carcelero dejó encerrado dentro.

---Hoy puedo dedicar a mis amigos un ratito---dijo dejándose conducir
por Soledad a la silla.---Ya estoy\ldots{} Gracias, señora\ldots{} Me
han dicho que es usted muy simpática\ldots{} En estos cavernosos cuartos
no se ve nada\ldots{} ¿Y el pobre tonto cómo se encuentra?

---¡Quieres dejarnos en paz, endiablado frailón!---gritó una voz ronca,
irritada, temblorosa, que parecía ser la voz misma de la oscuridad que
había tomado la palabra.

El padre Alelí sintió cierto terror.

---¡Jesús, María y José!---exclamó santiguándose.---Verdaderamente esta
no es casa de orates. Todo sea por Dios.

---Abuelito Sarmiento---dijo Soledad acariciando al anciano que arrojado
a sus pies estaba.---No es propio de persona cortés y bien educada como
tú, el tratar así a un sacerdote.

---¡Que se vaya de aquí!\ldots{} ¡Que nos deje solos!---gruñó el
fanático, arrastrándose como un tigre enfermo.---¿Qué busca aquí el
frailucho? ¿qué quiere?

---¡Ave María purísima!\ldots{}

---Si al menos nos trajera buenas noticias\ldots{}

---Buenas las traigo para usted\ldots{}

---A ver, a ver\ldots---dijo D. Patricio incorporándose de improviso.

---Usted será absuelto libremente.

Sarmiento se desplomó en el suelo, haciendo temblar los ladrillos.

---¡Maldita sea la boca que lo dice!\ldots---murmuró con hondo bramido.

---Siento no poder dar nuevas igualmente lisonjeras a esta
señora---añadió el fraile tomando la mano de la joven y estrechándosela
entre las suyas.---No puedo decir lo mismo, ni quiero dar esperanzas que
no han de verse realizadas. Las circunstancias obligan al tribunal a ser
muy severo\ldots{} ¡Cómo ha de ser! Más padeció Jesucristo por nosotros.
Si tiene usted resignación, paciencia cristiana; si purificando su alma
sabe desprenderla de las miserias del mundo y elevarla al cielo en este
trance de apariencia aflictiva, será más digna de envidia que de
lástima.

---¡Maldita sea la boca que lo dice!

Sarmiento al hablar así, arrastrábase hasta el ángulo opuesto.

---¿Qué es la vida?---añadió Alelí tomando un tono melifluo.---Nada, un
soplo, aire, una ilusión. ¿Qué es el tiempo que contamos en el mundo?
Nada, un momento. La vida está allá. ¿Qué importan un sufrimiento
pasajero, un dolor instantáneo? Nada, nada, porque después viene el
eterno gozar y la plácida eternidad en que se deleitan los justos. Nadie
es mejor recibido allá que los que aquí han padecido mucho. Los
perseguidos por la justicia son los primeros entre los bienaventurados.
Los pecadores que se depuran por el arrepentimiento y el castigo
corporal forman en la línea de los inocentes, todos juntos penetran
triunfantes en la morada celestial.

A esta homilía, dicha con arte y sentimiento, siguió largo silencio. El
padre Alelí suspiraba. Su mucha práctica en consolar a los reos de
muerte no había gastado en él los tesoros de sensibilidad que poseía,
antes bien, los había enriquecido más. Estaba sujeto a grandes
aflicciones por razón de su oficio y se identificaba tanto con sus
penitentes, que decía: «Me han ahorcado ya doscientas veces y tengo
sobre mí un par de siglos de presidio».

Después que cobró ánimos, habló así:

---Hoy he visto a esa señora; ¡qué angelical bondad la suya! Está
desesperada por no haber podido conseguir cosa alguna en pro de usted.
Sin embargo, no cede en su empeño\ldots{} aún tiene esperanza\ldots{}
Yo, si he de decir la verdad, ya no la tengo.

---Yo tampoco la tengo ni la quiero---dijo Soledad con un arranque de
unción religiosa.---Me resigno a mi desgraciada suerte y sólo espero
morir en Dios.

Por grandes que sean los bríos de un alma valerosa, la idea del morir y
de un morir violento, antinatural y vergonzoso la turba y la acomete con
fiera sacudida, prueba clara de que sólo a Dios corresponde matar. Sola
derramó algunas lágrimas y el fraile notó que sus heladas manos
temblaban. Ya a aquella hora, que era la del medio día, habían
aparecido, puntuales en su cuotidiana visita, las claridades advenedizas
que se paseaban por el cuarto. A favor de ellas se distinguían bien los
tres personajes: el fraile sentado en la silla, todo blanco y puro como
un ángel secular que hubiera envejecido, Soledad de rodillas ante él,
vestida de negro, mostrando su cara y sus manos de una palidez
transparente, D. Patricio echado en el rincón opuesto, con la cara
escondida entre los brazos y estos sobre los ladrillos, cada vez más
semejante a un tigre enfermo, cuya respiración era calenturiento
ronquido.

---Llore usted, llore---dijo el padre Alelí a su penitente,---que así se
calma la congoja. Yo también lloro, querida mía, también me lleno de
agua la cara, a pesar de estar tan acostumbrado a ver lástimas y
dolores. ¿El mundo qué es? barro amasado con lágrimas, ni más ni menos.
Lloramos al nacer, lloramos también al morir que es el verdadero
nacimiento.

---Padre---dijo la huérfana,---si ve Su Reverencia hoy a esa señora,
hágame el favor de manifestarle que le doy gracias de todo corazón por
lo que ha hecho por mí, aunque sus buenos deseos hayan sido inútiles. Al
mismo tiempo quiero que Su Reverencia le ruegue que me perdone\ldots{}
Su Reverencia no está en antecedentes. Yo cometí el día de mi prisión
una grave falta; me dejé arrastrar por la ira, y por la primera vez en
mi vida sentí en mi corazón el ardor de una pasión infame, la venganza.
No sé cómo fue aquello\ldots{} Me hizo tanto daño mi propio furor, que
me desmayé. Nunca había sentido cosa semejante. Parece que pasó por
dentro de mí como un rayo. Verdad es que yo tenía motivos, sí, padre,
motivos\ldots{} Pero no hablemos de eso\ldots{} Yo ruego a esa señora
que me perdone.

---Y yo me comprometo a asegurar a usted que ya está perdonada---replicó
el fraile con bondad.---Conozco a la señora y sé que sabe perdonar.

---Su Reverencia podrá decirme si le ocasionarán algún perjuicio a esa
señora las palabras que yo dije delante del juez.

---Presumo que no le ocasionarán daño alguno. Esté usted tranquila por
ese lado. Creo haber entendido (quizás me equivoque, porque estoy ya un
poco lelo), que entre usted y ella hay un resentimiento antiguo. Parece
que la señora, en un momento de delirio, porque los tiene, sí, tiene
esos momentos de delirio\ldots{}

---No quisiera que se nombrase eso más---replicó Sola con presteza,
extendiendo la mano como para taparle la boca al fraile.---Soy la
agraviada, y desde que estoy aquí me he propuesto olvidar ese y otros
agravios perdonándolos con todo mi corazón.

---Bien, muy bien. Esa cristiana conducta me gusta más que cien mil
rosarios bien rezados.

---¿Su Reverencia conoce bien lo que pasa en la Comisión Militar? Estoy
muy ansiosa por saber si el Sr.~Cordero y su hija han sido puestos en
libertad.

---Desde ayer, hija, desde ayer están en su casa tan contentos.

---¡Oh, qué dicha!---exclamó Sola cruzando las manos.---Eso es lo que yo
quería\ldots{} porque son inocentes y estaban presos por un delito que
yo cometí. Yo le contaré todo a Su Reverencia. Quiero hacer confesión
general.

---A punto estamos---repuso el fraile, acomodando el codo en la mesa y
sosteniendo la frente en la mano.

Sola se acercó más, dando principio al solemne acto.

Duró próximamente media hora. El padre Alelí dio su absolución en voz
alta y con los ojos cerrados, trazando lentamente la cruz en el aire con
su brazo blanco y su mano flaca y delicada. Concluido el latín, dijo en
castellano a la penitente:

---Adquisición admirable hará el reino de Dios muy pronto con la entrada
de un alma tan hermosa.

Sola, que sentía mucho dolor en las rodillas, se echó hacia atrás
sentándose sobre sus propios pies.

En el mismo momento oyose un feroz ronquido y el roce de un cuerpo
contra el suelo. La voz cavernosa y terrible de Sarmiento se expresó
así:

---¿Quiere usted marcharse con cien mil docenas de demonios?\ldots{}
¿Qué cuchichean ahí?

El fraile se levantó y dando dos pasos hacia el punto en donde sonaban
las tremendas voces, dijo:

---Su compañera de usted ha confesado. ¿Quiere usted hacer lo mismo?

---¡Yo!\ldots{} Por vida de la re-condenada Chilindraina, Sr.~D.
Majadero, que si no se me quita pronto de delante\ldots{}

El padre Alelí se tocó la sien con su dedo índice, moviendo la cabeza en
señal de lástima.

---¡Confesar yo!\ldots{} ¡yo, que soy un volcán de rabia!---añadió el
desgraciado tratando de levantarse con fatigosos movimientos que hacían
bailar a los ladrillos.---¡Repito que no hay Dios!\ldots{} ¡no, no hay
Dios! Todo es una mentira. El mundo, la gloria, el destino, fábula y
palabrería. Denme un cuchillo, porque me quiero matar, me avergüenzo de
vivir\ldots{} Al primero que se me ponga por delante, le muerdo.

Las claridades que un momento se habían alejado, volvieron juguetonas,
sin abandonar sus capisayos de telarañas, y con ellas pudo ver el padre
Alelí que la pobre bestia enferma alzaba la cabeza y mostraba una
horrible cara amoratada y polvorienta, toda llena de viscosa baba. Sus
ojos daban miedo.

---¡Desgraciado!---murmuró con dolor el padre Alelí.---Tú que vivirás
eres más digno de lástima que ella, destinada a morir.

---No me lo digas, no me lo digas---gritó Sarmiento incorporando su
busto por un movimiento rapidísimo de sus remos delanteros.---No me lo
digas porque te mato, infame fraile, porque te devoro.

---Eres un pobre demente.

---Soy un hombre que ha perdido su ideal risueño, un hombre que soñó la
gloria y no la posee, un hombre que se creyó león y se encuentra cerdo.
Mi destino no es destino, es una farsa inmunda, y al caer y al
envilecerme y al pudrirme como me pudro, tengo la desgracia de conservar
intacto el corazón para que en él clave su vil puñal la justicia humana,
matando a mi hija\ldots{} Infame frailucho, ¿has venido a gozarte en mi
miseria? Vete pronto de aquí, vete. Mira que no soy hombre, soy una
bestia.

Clavaba sus uñas en los ladrillos y estiraba el amenazante rostro
descompuesto.

---Que Dios se apiade de ti---dijo grave y solemnemente el fraile
bendiciéndole.---Adiós.

Y después de encargar a Sola que tuviera resignación, mucha resignación
por las diversas causas que lo exigían (señalaba al infortunado viejo),
se retiró considerando la magnitud de los males que afligen a la raza
humana.

\hypertarget{xxiv}{%
\chapter{XXIV}\label{xxiv}}

¡Válganos Dios y qué endiablado humor tenía D. Francisco Chaperón, a
pesar de haber procedido conforme a lo que en él hacía las veces de
conciencia! Pues no llegaba el cinismo de los voluntarios realistas al
incalificable extremo de vituperarle aún, después que tan clara prueba
de severidad y rectitud acababa de dar\ldots{} ¡Cuán mal se juzga a los
grandes hombres en su propia patria! Varones eminentes, desvelaos;
consagrad vuestra existencia al servicio de una idea, para que luego la
ingratitud amargue vuestra noble alma\ldots{} ¡Todo sea por
Dios!\ldots{} ¡Por vida del Santísimo Sacramento, esto es una gran
bribonada!

Todavía vacilaba el D. Francisco en perdonar a Cordero, después de
haberlo propuesto en junta general a la Comisión; pero el cortesano de
1815 añadió a las muchas razones anteriormente expuestas otras de mucho
peso, logrando atraer a su partido y asociar hábilmente a su trabajo a
un hombre cuya opinión era siempre palabra de oro para el digno
Presidente de la Comisión. Este hombre era el coronel don Carlos
Garrote. Para seducirle, Bragas no necesitó emplear sutiles argucias.
Bastole decir que Jenara bebía los vientos por sacar de la cárcel a Sola
aunque en sustitución de ella fuese preciso ahorcar a todos los Corderos
y a todos los Toros de Guisando nacidos y por nacer. No necesitó de
otras razones Navarro para sugerir a Chaperón la luminosa idea
siguiente:

---Vea usted cómo voy comprendiendo que la hija de Gil de la Cuadra es
una intrigante. De esta especie de polilla es de la que se debe limpiar
el Reino. Apuesto a que es la querida de Seudoquis.

No se habló más del asunto. Aunque decidido a castigar severamente,
Chaperón no había de reconquistar las simpatías perdidas en el cuerpo de
voluntarios. Hubiéralo llevado con paciencia el hombre-horca, y casi
casi estaba dispuesto a consolarse, cuando un suceso desgraciadísimo
para la causa del Trono y de la Fe católica vino a complicar la
situación, exacerbando hasta el delirio el inhumano celo del señor
brigadier. En la noche del 2 al 3 de Setiembre, un preso, el más
importante sin duda de cuantos guardaba en su inmundo vientre la cárcel
de Corte, halló medios de evadirse, y se evadió. No se sabe si anduvo en
ello la virtud del metal que es llave de corazones y ganzúa de puertas,
o simplemente la destreza, energía y agudeza del preso. No discutiremos
esto: basta consignar el hecho tristísimo (atendiendo al Trono y a la Fe
católica) de que Seudoquis se escapó. ¿Fue por el tejado, fue por las
alcantarillas, fue por medio de un disfraz? Nadie lo supo, ni lo sabrá
probablemente. En vano D. Francisco, corriendo a la cárcel muy de mañana
(pues ni siquiera tuvo tiempo de tomar chocolate) mandó hacer
averiguaciones y registrar las bohardillas y sótanos, y prender a casi
todos los calaboceros e interrogar a la guardia, y amenazar con la horca
hasta al mismo santo emblema de la Divinidad humanada, que tan
asendereado estaba siempre en su irreverente y fiera boca.

A la hora del despacho se encerró con Lobo. Estaba tan fosco, tan
violento, que al verle, se sentían vivos deseos de no volverle a ver más
en la vida. Para hablarle de indulgencia se habría necesitado tanto
valor como para acercar la mano a un hierro candente. Chaperón sólo se
hubiera ablandado a martillazos.

---¿Está corriente la causa de esa?\ldots{} Es preciso presentarla sin
pérdida de tiempo al tribunal---dijo a su asesor.

---Ahora mismo la remataré Excelentísimo Señor.

---Me gusta la calma\ldots{} Yo he de ocuparme de todo\ldots{} No sirven
ustedes para nada\ldots{} Voy a llamar al primer asno que pase por la
calle para encomendarle todo el trabajo de esta secretaría.

En aquel mismo instante entró Jenara. No podía presentarse en peor
ocasión, porque venía a pedir indulgencia. Nunca había sido tampoco tan
interesante ni tan guapa, porque sus atractivos naturales se sublimaban
con su generosidad y con el valor propio de quien intrépidamente penetra
en una caverna de lobos para arrancarles la oveja que ya han empezado a
devorar.

La fiera estaba tan mal dispuesta en aquella nefanda hora, que sin
aguardar a que Jenara se sentase, díjole con voz ahogada:

---Por centésima vez, señora\ldots{}

Se detuvo moviendo la cabeza sobre el metálico cuello, cual si este le
estrangulara impidiendo el fácil curso de las palabras.

---Por centésima vez\ldots---gruñó de nuevo poniéndose rojo.

---Acabemos, hombre de Dios.

---Por centésima vez digo a usted que no puede ser\ldots{} En bonita
ocasión me coge\ldots{} Ciertamente que están las cosas a propósito para
perdonar\ldots{} Seudoquis escapado\ldots{} los Corderos en
libertad\ldots{} La Comisión desacreditada, acosada, vilipendiada,
escarnecida\ldots{} No somos jueces, somos vinagrillo de mil
flores\ldots{} No sé cómo no entran los chicos de las calles y nos tiran
de la nariz\ldots{} Me han pintado colgado de la horca\ldots{} y con
razón, con mucha razón\ldots{} Más vale que digan de una vez: «se acabó
el Gobierno absoluto; vuelvan los liberales\ldots» Malditas sean las
recomendaciones\ldots{} Ellos conspiran y nosotros perdonamos\ldots{}
Con tales farsas pronto tendremos al Cojo de Málaga en el Trono\ldots{}
Seudoquis escapado\ldots{} ¡la impunidad! aquí no hay más que
impunidad\ldots{} Se ahorca por besar el sitio donde estuvo la lápida de
la Constitución, y damos chocolate a los conspiradores\ldots{} Señora,
usted me toma por un Dominguillo\ldots{} Señora\ldots{} ¡Seudoquis
escapado!\ldots{} ¡la impunidad!\ldots{} esa malhadada impunidad\ldots{}
lepra horrible, horrible\ldots{}

Echaba las palabras a borbotones, interrumpidos a intervalos por
sofocadas toses y gruñidos. Los temblorosos labios parecían el obstruido
caño de una fuente, por donde salía el agua en violentas bocanadas con
intermitencias de resoplidos de aire. A cada segundo se metía los dedos
en el duro cuello negro de cartón para ensanchárselo y respirar mejor.

---Tanto enfado me mueve a risa---dijo la dama con burlona sonrisa y
demostrando mucha tranquilidad.---Cualquiera que a usted le viese
creería que estoy en presencia del mismo Soberano absoluto de estos
Reinos. Sr.~Chaperón, ¿por quién se ha tomado?

---Señora---dijo el brigadier enfrenando su cólera,---usted puede
tomarme por quien quiera; pero esta vez no cedo, no cedo\ldots{} Ya
comprendo la intriga, me trae usted una cartita de Calomarde\ldots{} Es
inútil, inútil, no hago caso de recomendaciones. Si Calomarde me manda
atender al ruego de usted, presentaré al punto mi dimisión. De mí no se
ríe nadie: soy responsable de la paz del Reino, y si vienen
revoluciones, tráigalas quien quiera, no yo.

---Calomarde no ha querido darme carta de recomendación---manifestó
Jenara sin abandonar su calma.

---Ya lo presumía. Hemos hablado anoche\ldots{} hemos convenido en la
necesidad de apretar los tornillos, de apretar mucho los tornillos.

---Calomarde y usted apretarán la hebilla de sus propios corbatines
hasta ahogarse si gustan---dijo ella con malicioso desdén,---pero en las
cosas públicas no harán sino lo que se les mande.

---Señora, permítame usted que no haga caso de sus bromitas. La ocasión
no es a propósito para ello. Tenemos que hacer\ldots{} ¿Pero qué es eso?
Veo que me trae usted una carta.

---Sí señor---replicó Jenara alargando un papel,---lea usted.

---Del Sr.~Conde de Balazote, gentil-hombre de Su Majestad---dijo el
vestiglo abriendo y leyendo la firma.---¿Y qué tengo yo que ver con ese
señor?

---Lea usted.

---¡Ah!\ldots{} ya\ldots---murmuró Chaperón quedándose estupefacto
después de leer la carta,---el señor gentil-hombre me besa la
mano\ldots{}

---¡Ya ve usted qué fino!

---Y me hace saber que Su Majestad me ordena presentarme inmediatamente
en Palacio.

---Para hablar con Su Majestad.

---Quiere decir que Su Majestad desea hablarme\ldots{}

Chaperón volvió a leer. Después dio dos o tres vueltas sobre su eje.

---Mi sombrero\ldots---dijo demostrando grandísima inquietud,---¿en
dónde está mi sombrero\ldots? Señora, usted dispense\ldots{} Lobo,
aguárdeme usted\ldots{}

---Yo aguardo aquí---indicó Jenara.

---Veremos lo que quiere de mí Su Majestad---añadió D. Francisco en
estado de extraordinario aturdimiento.---¿Y mi bastón, en dónde he
puesto yo ese condenado bastón?\ldots{} ¿Habré traído los
guantes?\ldots{} Señora, dispense usted que\ldots{} A los pies de
usted\ldots{} ¿Su Majestad me espera?\ldots{} Sí, me esperará, no saldrá
hasta que yo no vaya\ldots{} Y yo no recordaba que la Corte había venido
ayer de la Granja para trasladarse a Aranjuez\ldots{} Adiós; vuelvo.

Una hora después Chaperón entraba de nuevo en su despacho. Venía, si así
puede decirse, más negro, más tieso, más encendido, más agarrotado
dentro del collarín de cuero. Cruzando sus brazos se encaró con Jenara,
y le dijo:

---Vea usted aquí a un hombre perplejo. Su Majestad me ha hablado, me ha
tratado con tanta bondad como franqueza, me ha llamado su mejor amigo, y
por fin me ha mandado dos cosas de difícil conciliación, a saber: que
sea inexorable y que acceda al ruego de usted.

---Eso es muy sencillo---replicó Jenara con gracia suma.---Eso quiere
decir que sea usted generoso con mi protegida y severo con los demás.

---¡Inexorable, señora, inexorable!---exclamó D. Francisco apretando los
dientes y mirando foscamente al suelo.

---Inexorable con todos menos con ella. ¿Hay nada más claro?

---Dije a Su Majestad que se había escapado Seudoquis, y me
contestó\ldots{} ¿qué creerá usted que me contestó?

---Alguna de sus bromas habituales.

---Que había hecho perfectamente en escaparse, si se lo habían
consentido.

---Eso es hablar como Salomón.

---Veremos cómo salgo yo de este aprieto. Tengo que contentar al Rey, a
usted, a los voluntarios realistas, a Calomarde; tengo que contentar a
todo el mundo, siendo al mismo tiempo generoso e inexorable, benigno y
severo.

Chaperón se llevó las manos a la cabeza expresando el gran conflicto en
que se veía su inteligencia.

---¡Qué lástima que soltáramos a ese Cordero!\ldots---dijo después de
meditar.---Pero agua pasada no mueve molino, veamos lo que se puede
hacer. Formemos nuestro plan\ldots{} Atención, Lobo. Lo primero y
principal es complacer a la Sra. D.ª Jenara\ldots{} ¿Qué filtros ha dado
usted a nuestro Soberano para tenerle tan propicio?\ldots{} Atención,
Lobo. Lo primero es poner en libertad a esa joven\ldots{} escriba
usted\ldots{} \emph{por no resultar nada contra ella}.

Jenara aprobó con un agraciado signo de cabeza.

---Ahora pasemos a la segunda parte. Esta prueba de benevolencia no
quiere decir que erijamos la impunidad en sistema. Al contrario, si la
inocencia es respetada\ldots{} porque esa joven será inocente\ldots{} si
la inocencia es respetada, el delito no puede quedar sin castigo\ldots{}
Atienda usted, Lobo\ldots{} Esta conspiración no quedará impune de
ningún modo. Soledad Gil de la Cuadra es inocente, inocentísima ¿no
hemos convenido en eso? Sí; ahora bien, sus cómplices, o mejor dicho,
los que aparecen en este negocio de las cartas que se
repartieron\ldots{} No, no hay que tomarlo por ese lado de las cartas.
Lobo, quite usted de la causa todo lo relativo a cartas. Veamos el
cómplice.

---Patricio Sarmiento.

---¿Ese hombre está en su sano juicio?

---Permítame Vuecencia---dijo Lobo,---que le manifieste\ldots{} El
hablar de la imbecilidad de ese hombre me parece\ldots{} Si Vuecencia,
excelentísimo señor, me permite expresarme con franqueza\ldots{}

---Hable usted pronto.

---Pues diré que eso de la imbecilidad de Sarmiento me parece una
inocentada.

---Eso es: una inocentada---repitió Jenara.

---Pues qué, ¿no constan en la causa mil cosas que acreditan su buen
juicio?

Se le encontró entre sus papeles un paquete de cartas sobre la
organización de la Comunería, y consta que fue uno de los que más parte
tuvieron en el asesinato de Vinuesa.

---¿Hay pruebas, hay testigos?

---Diez pliegos están llenos de las declaraciones de innumerables
personas honradas que han asegurado haberle visto entrar, martillo en
mano, en la cárcel de la Corona.

---Admirable. Adelante.

---Después ha fingido hallarse demente para poder insultar a Su
Majestad, burlarse de la religión y apostrofar a los defensores del
Trono.

---¡Se ha fingido demente!

---Está probado, probadísimo, excelentísimo señor.

Chaperón dudaba, hay que hacerle ese honor. La \emph{monera} de que
antes hablamos se agitaba inquieta y alborotada entre el cieno, haciendo
esfuerzos por mostrarse.

---Pero esas pruebas de que se fingía demente\ldots---murmuró.---¿Hay
dictamen facultativo?

Jenara no veía con gusto aquella discusión y guardaba silencio.

---¿Qué dice el artículo 7.º del Decreto del 20 de este mes?---preguntó
Lobo con extraordinario calor.

---\emph{Que la fuerza de las pruebas en favor o en contra del acusado
se dejan a la prudencia e imparcialidad de los jueces}. Bien, admitamos
que la ficción de demencia es cosa corriente. No hay más que hablar.

---¿Qué dice el artículo 11 del mismo Decreto?

---\emph{Que se castigue con el último suplicio a los que griten «Viva
la Constitución, mueran los serviles, mueran los tiranos, viva la
libertad\ldots»} ¡Ah! aquí no puede haber quebraderos de cabeza. Según
este artículo, Sarmiento debía haber sido ahorcado cien veces\ldots{}
Pero la imbecilidad, la locura o como quiera llamarse a esa su semejanza
con los graciosos de teatro\ldots{}

---¿Qué dice el artículo 6.º del mismo Decreto?---preguntó de nuevo Lobo
con tanto entusiasmo que sin duda se creía la imagen misma de la
jurisprudencia.

---Dice que \emph{la embriaguez no es obstáculo para incurrir en la
pena}.

---¿Y qué es la embriaguez más que una locura pasajera?\ldots{} ¿Qué es
la locura más que una embriaguez permanente? Consulte Vuecencia
excelentísimo señor, todos los autores y verá cómo concuerdan con mi
parecer. Vuecencia podrá fallar lo que quiera; pero de la causa resulta,
claro como la luz del día, que la muchacha y los ángeles del cielo
rivalizan en inocencia, y que el Sarmiento es reo convicto del asesinato
de Vinuesa, de propagación de ideas subversivas, del establecimiento de
la Comunería, de predicación en sitios públicos contra la única
soberanía que es la real, de connivencia con los emigrados, etc., etc.

---¡Oh! Sr.~D. Francisco---dijo la dama con generoso arranque.---Si
quiere usted merecer un laurel eterno y la bendición de Dios, perdone
usted también a ese pobre viejo.

---Señora, poquito a poco---repuso el funcionario poniéndose muy
serio.---Antes que erigir en sistema la impunidad, cuidado con la
impunidad, ¡por vida del\ldots! presentaré mi dimisión. Bastante ha
conseguido usted.

La dama inclinó la cabeza, fijando los ojos en el suelo. Otra vez
suplicó, porque no podía resistir impasible a la infame tarea de
aquellos inicuos polizontes; pero Chaperón se mostró tan celoso de su
reputación, de su papel y de atender a las circunstancias (¡siempre las
circunstancias!) que al fin la intercesora, creyéndose satisfecha con el
triunfo alcanzado, no quiso comprometerlo, aspirando a más. Se retiró
contenta y triste al mismo tiempo. Necesitaba ver aquel mismo día a los
demás individuos de la Comisión, pues aunque el Presidente lo era todo y
ellos casi nada, convenía prevenirlos para asegurar mejor la victoria.

Cuando se quedaron solos, Chaperón dijo a su asesor privado:

---Arrégleme usted eso inmediatamente. Extienda usted la sentencia y
llévela al comandante fiscal para que la firme. Hoy mismo se presentará
al tribunal. Mañana nos reuniremos para sentenciar a la mujer que robó
el almirez de cobre y el vestido de percal viejo\ldots{} Pasado mañana
tocará sentenciar eso\ldots{} ¡Oh! veremos si los compañeros quieren
hacerlo mañana mismo\ldots{} Quesada me ha recomendado hoy la mayor
celeridad en el despacho y en la ejecución de las sentencias\ldots{}

Y cabizbajo, añadió:

---Veremos cómo lo toma la Comisión. Yo tengo mis dudas\ldots{} mi
conciencia no está completamente tranquila\ldots{} pero, ¿qué se ha de
hacer? todo antes que la impunidad.

Y aquel hombre terrible, que era Presidente de derecho del pavoroso
tribunal, y de hecho fiscal, y el tribunal entero; aquel hombre, de cuya
vanidad sanguinaria y brutal ignorancia dependía la vida y la muerte de
miles de infelices, se levantó y se fue a comer.

La Comisión, reunida al día siguiente para fallar la causa de la mujer
que había robado un almirez de cobre y un vestido de percal viejo, falló
también la de Sarmiento. No pecaban de escrupulosos ni de vacilantes
aquellos señores, y siempre sentenciaban de plano conformándose con el
parecer del que era vida y alma del tribunal. Todas las mañanas, antes
de reunirse, oían una misa llamada \emph{de Espíritu Santo}, sin duda
porque era celebrada con la irreverente pretensión de que bajara a
iluminarles la tercera persona de la Santísima Trinidad. Por eso
deliberaban tranquila, rápidamente y sin quebraderos de cabeza. Todos
los días, al dar la orden de la plaza y distribuir las guardias y
servicios de tropa, el Capitán General designaba el sacerdote castrense
que había de decir la misa \emph{de Espíritu Santo}. Esto era como la
señal de ahorcar\footnote{Véase cualquier número del \emph{Diario de
  Avisos}, año de 1824.}.

Al anochecer del día en que fue sentenciada la causa de Sarmiento,
previa la misa correspondiente, el escribano entró en la prisión y a la
luz de un farolillo que el alguacil sostenía, leyó un papel.

Oyéronle ambos reos con atención profunda. Sarmiento no respiraba. No
había concluido de leer el escribano, cuando D. Patricio enterado de lo
más sustancial, lanzó un grito y poniéndose de rodillas elevó los
brazos, y con entusiasmo que no puede describirse, con delirio sublime,
exclamó:

---¡Gracias, Dios de los justos, Dios de los buenos! ¡Gracias, Dios mío,
por haber oído mis ruegos!\ldots{} ¡Ella libre, yo mártir, yo dichoso,
yo inmortal, yo santificado por los siglos de los siglos!\ldots{}
Gracias, Señor\ldots{} Mi destino se cumple\ldots{} No podía ser de otra
manera. Jueces, yo os bendigo. Pueblo, mírame en mi trono\ldots{} Estoy
rodeado de luz.

\hypertarget{xxv}{%
\chapter{XXV}\label{xxv}}

La capilla de los reos de muerte que estaba en el piso bajo y en el
ángulo formado por la calle de la Concepción Jerónima y el callejón del
Verdugo, era el local más decente de la cárcel de Corte. No parecía en
verdad decoroso, ni propio de una nación tan empingorotada que los reos
se prepararan a la muerte mundana y salvación eterna en una pocilga como
los departamentos donde moraban durante la causa. Además en la capilla
entraban movidos de curiosidad o compasión muchos personajes de viso,
señores obispos, consejeros, generales, gentiles-hombres, y no se les
había de recibir como a cualquier pelagatos. Tomaba sus luces esta
interesante pieza del cercano patio, por la mediación graciosa de una
pequeña sala próxima al cuerpo de guardia; mas como aquellas llegaban
tan debilitadas que apenas permitía distinguir las personas, de aquí que
en los días de capilla se alumbrara esta con la fúnebre claridad de las
velas amarillas encendidas en el altar. Lúgubre cosa era ver al reo,
aquel moribundo sano, aquel vivo de cuerpo presente, en la antesala de
la horca, y oírle hablar con los visitantes y verle comer junto al
altar, todo a la luz de las hachas mortuorias. Generalmente los
condenados, por valientes que sean, toman un tinte cadavérico que
anticipa en ellos la imagen de la descomposición física, asemejándoles a
difuntos que comen, hablan, oyen, miran y lloran para burlarse de la
vida que abandonaron.

No fue así D. Patricio Sarmiento, pues desde que le entraron en la
capilla en la para él felicísima mañana del 4 de Setiembre, pareció que
se rejuvenecía, tales eran el contento y la animación que en sus ojos
brillaban. Rosicler mustio le tiñó las ajadas mejillas, y su espina
dorsal hubo de adquirir por maravilloso don una rectitud y esbelteza que
recordaban sus buenos tiempos de Roma y Cartago. Soledad, a quien
permitieron acompañarle todo el tiempo que quisiera, se hallaba en
estado de viva consternación, de tal modo que ella parecía la condenada
y él el absuelto.

---Querida hija mía---le dijo D. Patricio cuando juntos entraron en la
capilla,---no desmayes, no muestres dolor, porque soy digno de envidia,
no de lástima. Si yo tengo este fin mío por el más feliz y glorioso que
podría imaginar, ¿a qué te afliges tú? Verdad es que la Naturaleza
(cuyos Códigos han dispuesto sabiamente los modos de morir) nos ha
infundido instintivamente cierto horror a todas las muertes que no sean
dictadas por ella, o hablando mejor, por Dios; pero eso no va con
nosotros, que tenemos un espíritu valeroso, superior a toda
niñería\ldots{} Ánimo, hija de mi corazón. Contémplame y verás que el
júbilo no me cabe en el pecho\ldots{} Figúrate la alegría del prisionero
de guerra que logra escaparse y anda y camina, y al fin oye sonar las
trompetas de su ejército\ldots{} Figúrate el regocijo del desterrado que
anda y camina y ve al fin la torre de su aldea. Yo estoy viendo ya la
torre de mi aldea, que es el Cielo, allí donde moran mi padre, que es
Dios, y mi hijo Lucas, que goza del premio dado a su valor y a su
patriotismo. Bendito sea el primer paso que he dado en esta sala,
bendito sea también el último; bendito el resplandor de esas velas,
benditas esas sagradas imágenes; bendita tú que me acompañas, y esos
venerables sacerdotes que me acompañan también.

Soledad rompió a llorar, aunque hacía esfuerzos para dominarse, y D.
Patricio fijando los ojos en el altar y viendo el hermoso Crucifijo de
talla que en él había y la imagen de Nuestra Señora de los Dolores,
experimentó una sensación singular, una especie de recogimiento que por
breve rato le turbó. Acercándose más al altar, dijo con grave acento:

---Señor mío, tu presencia y esos tus ojos que me ven sin mirarme
recuérdanme que durante algún tiempo he vivido sin pensar en ti todo lo
que debiera. El gran favor que acabas de hacerme me confunde más en tu
presencia. Y tú, Señora y Madre mía, que fuiste mi patrona y abogada en
cien calamidades de mi juventud, no creas que te he olvidado. Por tu
intercesión sin duda, he conseguido del Eterno Padre este galardón que
ambicionaba. Gracias, Señora, yo demostraré ahora que si mi muerte ha de
ser patriótica y valerosa para que sea fecunda, también ha de ser
cristiana.

Admirados se quedaron de este discurso el padre Alelí y el padre Salmón
que juntamente con él entraron para prestarle los auxilios espirituales.
Ambos frailes oraban de rodillas. Levantáronse y tomando asiento en el
banco de iglesia que en uno de los costados había, invitaron a Sarmiento
a ocupar el sillón.

---Yo no daré a Vuestras Reverencias mucho trabajo---dijo el patriota
sentándose ceremoniosamente en el sillón,---porque mi espíritu no
necesita de cierta clase de consuelillos mimosos que otras vulgares
almas apetecen en esta ocasión; y en cuanto al auxilio puramente
religioso, yo gusto de la sencillez suma. En ella estriba la grandeza
del dogma.

El padre Alelí y el padre Salmón se miraron sin decir nada.

---Veo a Sus Reverencias como cortados y confusos delante de mí---añadió
Sarmiento sonriendo con orgullo.---Es natural, yo no soy de lo que se ve
todos los días. Los siglos pasan y pasan sin traer un pájaro como este.
Pero de tiempo en tiempo Dios favorece a los pueblos dándole uno de
estos faros que alumbran el género humano y le marcan su camino\ldots{}
Si una vida ejemplar alumbra muy mucho al género humano, más le alumbra
una muerte gloriosa\ldots{} Me explico perfectamente la admiración de
Sus Paternidades; yo no nací para que hubiera un hombre más en el mundo;
yo soy de los de encargo, señores. Una vida consagrada a combatir la
tiranía y enaltecer la libertad; una muerte que viene a aumentar la
ejemplaridad de aquella vida, ofreciendo el espectáculo de una víctima
que expira por su fe y que con su sangre viene a consagrar aquellos
mismos principios santos; esta entereza mía; esta serenidad ante el
suplicio, serenidad y entereza que no son más que la convicción profunda
que tengo de mi papel en el mundo, y por último la acendrada fe que
tengo en mis ideas, no pertenecen, repito, al orden de cosas que se ven
todos los días\ldots{}

El padre Alelí abrió la boca para hablar; mas Sarmiento, deteniéndole
con un gesto que revelaba tanta gravedad como cortesía, prosiguió así:

---Permítame Vuestra Paternidad Reverendísima que ante todo haga un
declaración importante, sí, sumamente importante. Yo soy enemigo del
instituto que representan esos frailunos trajes. Faltaría a mi
conciencia si dijese otra cosa; yo aborrezco ahora la institución como
la aborrecí toda mi vida, por creerla altamente perniciosa al bien
público. Ahí están mis discursos para el que quiera conocer mis
argumentos. Pero esto no quita que yo haga distinciones entre cosas y
las personas, y así me apresuro a decirles que si a los frailes en
general les detesto, a Vuestras Paternidades les respeto en su calidad
de sacerdotes y les agradezco los auxilios que han venido a prestarme.
Además, debo recordar que ayer, hallándome en mi calabozo, traté
groseramente de palabra a uno de los que me escuchan, no sé cuál era.
Estaba mi alma horriblemente enardecida por creerse víctima de
maquinaciones que tendían a desdorarla, y no supe lo que me dije. Los
hombres de mi temple son muy imponentes en su grandiosa ira. Entiéndase
que no quise ofender personalmente al que me oía, sino apostrofar al
género humano en general y a cierto instituto en particular. Si hubo
falta la confieso y pido perdón de ella.

El padre Alelí, aprovechando el descanso de Sarmiento, tomó la palabra
para decirle que tuviese presente el sitio donde se encontraba, y
rompiese en absoluto con toda idea del mundo para no pensar sino en
Dios; que recordase cuál trance le aguardaba y cuáles eran los mejores
medios para prepararse a él; y finalmente, que ocupándose tanto de
vanidades, corría peligro de no salvarse tan pronto y derechamente como
de la limpieza de su corazón debía esperarse. A lo cual D. Patricio,
volviéndose en el sillón con mucho aplomo y seriedad, dijo al fraile que
él (D. Patricio) sabía muy bien cómo se había de preparar para el fin no
lamentable sino esplendoroso, que le aguardaba, y que por lo mismo que
moría proclamando su ideal divino, pensaba morir cristianamente, con lo
cual aquél había de aparecer más puro, más brillante y más ejemplar.

Esto decía cuando llegaron los hermanos de la Paz y Caridad, caballeros
muy cumplidos y religiosos que se dedican a servir y acompañar a los
reos de muerte. Eran tres y venían de frac, muy pulcros y atildados,
como si asistieran a una boda. Después que abrazaron uno tras otro
cordialmente a D. Patricio, preguntáronle que cuándo quería comer,
porque ellos eran los encargados de servirle, añadiendo que si el reo
tenía preferencias por algún plato, lo designara para servírselo al
momento, aunque fuese de los más costosos.

Sarmiento dijo que pues él no era glotón, trajeran lo que quisieran, sin
tardar mucho, porque empezaba a sentir apetito. Desde los primeros
instantes los tres cofrades pusieron cara muy compungida, y aun hubo
entre ellos un que empezó a hacer pucheros, mientras los otros dos
rezaban entre dientes; visto lo cual por Sarmiento, dijo muy
campanudamente que si habían ido allí a gimotear, se volviesen a sus
casas, porque aquella no era mansión de dolor, sino de alegría y
triunfo. No creyendo por esto los hermanos que debían abandonar su papel
oficial, comenzaron a soltar una tras otra las palabrillas emolientes
que eran del caso y que tantas veces habían pronunciado, \emph{verbi
gratia}\ldots{} «Querido hermano en Cristo, la celestial Jerusalém abre
sus puertas para ti»\ldots{} «Vas a entrar en la morada de los
justos»\ldots{} «Ánimo. Más padeció el Redentor del mundo por nosotros».

---Queridos hermanos en Cristo---dijo el reo con cierta jovialidad
delicada.---Agradezco mucho sus consuelos; pero he de advertirles que no
los necesito. Yo me basto y me sobro. Así es que no verán en mí
suspirillos, ni congojas, ni babas, ni pucheros\ldots{} Me gusta que
hayan venido, y así podrán decir a la posteridad cómo estaba Patricio
Sarmiento en la capilla, y qué bien revelaba en su noble actitud y
reposado continente (al decir esto erguía la cabeza, echando el cuerpo
hacia atrás) la grandeza de la idea por la cual dio su sangre.

Pasmados se quedaron los hermanos así como los frailes, de ver su
serenidad, y le exhortaron de nuevo a que cerrase el entendimiento a las
vanidades del mundo. Sola, de rodillas junto al altar, rezaba en
silencio.

\hypertarget{xxvi}{%
\chapter{XXVI}\label{xxvi}}

Empezaron los hermanos a servir la comida. Sentose D. Patricio a la
mesa, invitando a todos a que le acompañaran. No había comenzado aún,
cuando entró el Sr.~de Chaperón, que jamás dejaba de visitar a sus
víctimas en la antesala del matadero. Como de costumbre en tales casos,
el señor brigadier trataba de enmascarar su rostro con ciertas muecas y
contorsiones y gestos encargados de expresar la compasión, y helo aquí
arqueando las cejas y plegando santurronamente los ángulos de la boca,
sin conseguir más que un aumento prodigioso en su fealdad.

Saludó a Sarmiento con esa cortesía especial que se emplea con los reos
de muerte, y que es una cortesía indefinible e incomprensible para el
que no ha visto muestras de ella en la capilla de la cárcel; urbanidad
en la cual no hay ni asomos de estimación, porque se trata de un
delincuente atroz, ni tampoco desprecio o encono a causa de la
proximidad del morir. Es una callada fórmula de repulsión compasiva,
sentimiento extraño que no tiene semejante como no sea en el alma de
algún carnicero no muy novicio ni tampoco mu empedernido.

---Hermano en Cristo---dijo D. Francisco poniendo su mano, tan semejante
al hacha del verdugo, sobre el cuello del preceptor,---supongo que su
alma sabrá buscar en la religión los consuelos\ldots{}

Esta formulilla era de cajón. Aquel funcionario de tan pocas ideas la
llevaba prevenida siempre que a los reos visitaba.

---Sr.~D. Francisco---replicó Sarmiento levantándose,---si Vuecencia
quiere acompañarme a la mesa\ldots{}

---No, gracias, gracias, siéntese usted\ldots{} ¿Qué tal estamos de
salud?\ldots{} ¿Y el apetito?

Lo preguntaba, como lo preguntaría un médico.

---Vamos viviendo---repuso el patriota.---O si se quiere, vamos
muriendo. Todavía no ha llegado el instante precioso en que sea
innecesario este grosero sustento de la bestia\ldots{} Hemos de
arrastrar el peso del cuerpo, hasta que llegue el instante de dejarlo en
la orilla y lanzarnos al océano sin fin, en brazos de aquellas olas de
luz que nos mecerán blandamente en presencia del Autor de todas las
cosas.

Chaperón miró a los frailes e hizo un gesto que indicaba opinión
favorable del juicio de Sarmiento.

---Y ya que Vuecencia ha tenido la bondad de visitarme---añadió el reo,
después de saborear el primer bocado,---tengo el gusto de declarar que
no siento odio contra nadie, absolutamente contra nadie. A todos les
perdono de corazón, y si de algo valen las preces de un escogido como yo
(al decir esto su tono indicaba el mayor orgullo) he de alcanzar del
Altísimo que ilumine a los extraviados para que muden de conducta,
trocando sus ideas absolutistas por el culto puro de la libertad\ldots{}
Sí señor; se intercederá por los que están ciegos, para que reciban luz;
se recomendará a los crueles para que hallen misericordia en su día.
Patricio Sarmiento es leal, pío, generoso, como apóstol de la misma
generosidad, que es el liberalismo\ldots{} En mi corazón ya no caben
resentimientos; todos los he echado fuera, para presentarme puro y sin
mancha. El mártir de una idea, el que con su sangre ha puesto el sello a
esa idea ¿me entienden ustedes? para que quede consagrada en el mundo,
no enturbiará su conciencia con odios mezquinos. Reconozco que con
arreglo a las leyes mi condenación ha sido razonable. Vuecencia que me
oye no ha hecho más que cumplir con la ley que se le ha puesto en la
mano. Así me gusta a mí la gente. Venga esa mano, Sr.~D. Francisco.

Diole tan fuerte apretón de manos, que Chaperón hubo de retirar la suy
prontamente para que no se la estrujara.

---Además---prosiguió Sarmiento,---yo sé que los que hoy me condenan, me
admirarán mañana, si viven, y los que me vituperan hoy, luego me pondrán
en el mismo cuerno de la luna\ldots{} Porque esto durará poco, Sr.~D.
Francisco; el absolutismo, a fuerza de estrangular, se sostendrá un año,
dos, tres, pongamos cuatro\ldots{} En este guisado de vaca---añadió
dirigiéndose a uno de los hermanos de la Caridad---se le fue la mano a
la cocinera: lo ha cargado de sal\ldots{} Pongamos cuatro años; pero al
fin tiene que caer y hundirse para siempre, porque los siglos muertos no
resucitan, señor D. Francisco, porque los pueblos, una vez que han
abierto los ojos, no se resignan a cerrarlos, y así como cada estación
tiene sus frutos, cada época tiene su sazón propia, y los españoles, que
hasta aquí hemos amargado de puro verdes, vamos madurando ya, ¿me
entiende Vuecencia? y se nos ha puesto en la cabeza que no servimos para
ensalada. Vuecencias ahorquen todo lo que quieran. Mientras más ahorquen
peor. El absolutismo acabará ahorcándose a sí mismo. ¿No lo quieren
creer? Pues lo pruebo. Empezó creando para su defensa y sostenimiento la
fuerza de voluntarios realistas. Son estos unos animalillos voraces y
tragaldabas que no se prestan a servir a su amo, si este no les alimenta
con cuerpos muertos. Una vez cebados y enviciados con el fruto de la
horca, mientras más se les da más piden, y llegará un momento en que no
se les pueda dar todo lo que piden, ¿me entiende Vuecencia?

D. Francisco, sin contestarle, y dirigiendo maliciosas ojeadas a los
frailes, hacía señas de asentimiento.

El padre Salmón, que atendía con sorna a las razones del preso, bajó la
cabeza para ocultar la risa. Pero el padre Alelí, que devotamente rezaba
en su breviario, alzó los ojos y mirando con expresión de alarma al reo,
le dijo:

---Hermano mío, veo que lejos de apartar usted su pensamiento de las
ideas mundanas, se engolfa más y más en ellas, con gran perjuicio de su
alma. Los momentos son preciosos; la ocasión impropia para hacer
discursos.

---Y yo digo que es menos propia para sermones---replicó Sarmiento dando
un golpecillo en la mesa con el mango del tenedor.---Yo sé bien lo que
corresponde a cada momento, y repito que consagraré a la religión y a mi
conciencia todo el tiempo que fuere necesario.

---Bastante ha perdido usted en vanidades.

---Poquito a poco, señor sacerdote---dijo Sarmiento frunciendo las
cejas:---yo nada le quito a Dios. No se quite nada tampoco a las ideas,
que son mi propia vida, mi razón de ser en el mundo, porque, entiéndase
bien, son la misión que Dios mismo me ha encargado. Cada uno tiene su
destino: el de unos es decir misa, el de otros es enseñar e iluminar a
los pueblos. El mismo que a Su Paternidad Reverendísima le dio las
credenciales me las ha dado a mí.

---Reflexione, hombre de Dios---indicó el padre Salmón, rompiendo el
silencio,---en qué sitio se encuentra, qué trance le espera, y vea si no
le cuadra más preparar su alma con devociones, que aturdirla con
profanidades.

---Vuestras Paternidades me perdonen---dijo Sarmiento grave y
campanudamente después de beber el último trago de vino,---si he hablado
de cosas profanas, que no les agrada. Yo soy quien soy y sé lo que me
digo. Sé mejor que nadie por qué estoy aquí, por qué muero y por qué he
vivido. Allá nos entenderemos Dios y yo, Dios que llena mi conciencia y
me ha dictado este acto sublime, que será ejemplo de las generaciones.
Pero pues las religiosidades no están nunca demás, vamos a ellas y así
quedarán todos contentos.

---Esas divagaciones, hombre de Dios---dijo Salmón con puntos de
malicia,---confirman uno de los delitos que le han traído a este sitio.

---¿Qué delito?

---El de fingirse enajenado para poder tratar impunemente de cosas
vedadas.

---Hablillas---dijo Sarmiento sonriendo con desdén.---Señores hermanos
de la Paz, si tuvieran ustedes la bondad de darme cigarros, se lo
agradecería\ldots{} Hablillas del vulgo. Si fuéramos a hacer caso de
ellas, ¿cómo quedaría el padre Salmón en la opinión del mundo? ¿No dicen
de él que sólo piensa en llenar la panza y en darse buena vida? ¿No goza
fama de ser mejor cocinero que predicador?\ldots{} ¿de frecuentar más
los estrados de las damas para hablar de modas y comidas, que el coro
para rezar y la cátedra para enseñar? Esto dice el vulgo. ¿Hemos de
creer lo que diga? Pues del padre Alelí que me está oyendo y que es
persona apreciabilísima, ¿no se dijo en otro tiempo que era volteriano?
¿No le tuvo entre ojos la Inquisición? ¿No decían que antaño era amigo
de Olavide y que después se había congraciado con los realistas para no
ser molestado? Esto se dijo: ¿hemos de hacer caso de las necedades del
vulgo?

El padre Alelí palideció, demostrando enojo y turbación. Chaperón se
mordía los labios para dominar sus impulsos de risa. Ofrecía en verdad
la fúnebre capilla espectáculo extraño, único, el más singular que puede
presentarse. Frente al altar veíase una mujer de rodillas, rezando sin
dejar de llorar, como si ella sola debiera interceder por todos los
pecadores habidos y por haber; en el centro una mesa llena de viandas y
un reo que después de hablar con desenfado y entereza recibía cigarros
de los hermanos de la Paz y Caridad y los encendía en la llama de un
cirio; más allá dos frailes, de los cuales el uno parecía vergonzoso y
el otro enfadado; enfrente la tremebunda figura de D. Francisco
Chaperón, el abastecedor de la horca y el terror de los reos y de los
ajusticiados, sonriendo con malicia y dudando si poner cara afligida o
regocijada; todo esto presidido por el Crucifijo y la Dolorosa, e
iluminado por la claridad de las velas de funeral que daban cadavérico
aspecto a hombres y cosas, y allá más lejos en la sala inmediata una
sombra odiosa, una figura horripilante que esperaba: el verdugo.

D. Francisco Chaperón se despidió de su víctima. En la sala contigua y
en el patio encontró a varios individuos de la Comisión Militar y a
otros particulares que venían a ver al reo.

---¡Que me digan a mí que ese hombre es tonto!---exclamó con evidente
satisfacción.---Tan tonto es él como yo. No es sino un grandísimo
bribón, que aún persiste en su plan de fingirse demente, por ver si
consigue el indulto\ldots{} Ya, ya. Lo que tiene ese bergante es mucho,
muchísimo talento. Ya quisieran más de cuatro\ldots{} Por cierto que
entre bromas y veras ha hablado con un donaire\ldots{} Al pobre Salmón
le ha puesto de hoja de perejil, y Alelí no ha salido tampoco muy
librado de manos de este licenciado Vidriera\ldots{} Es graciosísimo:
véanle ustedes\ldots{} Por supuesto bien se comprende que es un
solemnísimo pillo.

Y D. Francisco se retiró, repitiéndose a sí mismo con tanta firmeza como
podría hacerlo un reo ante el juez, que D. Patricio no era imbécil, sino
un gran tunante. Tal afirmación tenía por objeto sofocar la rebeldía de
aquel insubordinado corpúsculo, a quien llamamos antes la \emph{monera}
de la conciencia chaperoniana, y que desde que Sarmiento entró en
capilla, se agitaba entre el légamo, queriendo mostrarse y alborotar y
hacer cosquillas en el ánimo del digno funcionario. Con aquella
afirmación, D. Francisco aplacó la vocecilla y todo quedó en profundo
silencio allá en los cenagosos fondajes de su alma.

\hypertarget{xxvii}{%
\chapter{XXVII}\label{xxvii}}

Durante la noche arreció el nublado de visitantes, sin que su curiosidad
importuna y amanerada compasión causaran molestia al reo; antes bien
recibíalos este como un soberano a su corte. Situado en pie frente al
altar, íbalos saludando uno por uno, con ligeros arqueos de la espina
dorsal y una sonrisa protectora, cuya intensidad de expresión amenguaba
o disminuía según la importancia del personaje. Todos salían haciéndose
lenguas de la serenidad del reo, y en la sala-vestíbulo, inmediata al
cuerpo de guardia oíase cuchicheo semejante al que se oye en el atrio de
una iglesia en noches de novena o tinieblas. Los entrantes chocaban con
los que salían, y la sensibilidad de los unos anticipaba a la curiosidad
de los otros noticias y comentarios.

Pipaón, que se había presentado de veinte y cinco alfileres, y parecía
un ascua de oro según iba de limpio y elegante, estuvo largo rato en
compañía del reo, y le dio varias palmadas en el hombro, diciéndole:

---Ánimo, Sr.~Sarmiento, y encomiéndese a Su Divina Majestad y a la
Reina de los cielos, Nuestra Madre amorosísima, para que le den una
buena muerte y franca entrada en la morada celestial\ldots{} Adiós,
hermano mío. Como mayordomo que soy de la hermandad de las Ánimas, le
tendré presente, sí, le tendré presente para que no le falten
sufragios\ldots{} Adiós\ldots{} Procure usted serenarse\ldots{} Medite
mucho en las cosas religiosas\ldots{} este es el gran remedio y el más
seguro lenitivo\ldots{} ¡La religión, la dulce religión! ¡Oh! ¿qué sería
de nosotros sin la religión?\ldots{} es nuestro consuelo, el rocío que
nos regenera, el maná que nos alimenta\ldots{} Adiós, hermano en Cristo,
venga un abrazo (al dar el abrazo Pipaón tuvo buen cuidado de que no
fuera muy expresivo, para que no se chafaran los encajes de su
pechera)\ldots{} Estoy conmovidísimo\ldots{} Adiós, repítole que medite
mucho en los sagrados misterios y en la pasión y muerte de Nuestro Señor
Jesucristo\ldots{} No le faltarán sufragios, muchos sufragios. Quizás
nos veamos en el Cielo, ¡ay de mí! si Dios es misericordioso conmigo.

Este fastidioso discurso, modelo exacto de la retórica convencional y
amanerada del cortesano, agradó mucho a cuantos le oyeron; mas D.
Patricio lo acogió con seriedad cortés y cierto desdén que apenas se
traducía en ligero fruncimiento de cejas. Pipaón salió y aunque iba muy
aprisa derecho a la calle, detuviéronle en el patio algunos amigos.

---Estoy afectadísimo\ldots{} no puedo ver estas escenas---les dijo
respondiendo a sus preguntas.---Fáltame poco para desmayarme.

---Dicen que es el reo más sereno que se ha visto desde que hay reos en
el mundo.

---Es un prodigio. Pero aquella vanidad e hinchazón son cosa
fingida\ldots{} ¡Cuánto debe padecer interiormente! Se necesitan los
bríos de un héroe para sostener ese papel sin faltar un punto.

---¡Farsante!

---Es el perillán más acabado no he visto en mi vida. Seguramente espera
que le indulten; pero se lleva chasco. El Gobierno no está por indultos.

---Entremos\ldots{} todo Madrid desea verle. Vuelva usted, Pipaón.

---¿Yo? por ningún caso---repuso el cortesano estrechando manos diversas
una tras otra.---Voy a una reunión donde cantan la Fábrica y
Montresor\ldots{} ¡Qué aria de la \emph{Gazza Ladra} nos cantó anoche
esa mujer! Montresor nos dio el aria de \emph{Tancredo}. ¡Aquello no es
hombre, es un ruiseñor!\ldots{} ¡Qué portamentos, qué picados, qué
trinos, qué vocalización, qué falsete tan delicioso! Parece que se
transporta uno al sétimo cielo. Con que adiós, señores\ldots{} tengo que
ensayar antes un paso de gavota. Señores, divertirse con el viejo
Sarmiento.

Aún no se había separado de sus amigos, cuando salió al patio un señor
obispo que venía también de visitar al reo. Todos se descubrieron al
verle, haciéndole calle. Pipaón, después de besarle el anillo, le habló
del condenado a muerte.

---Mi opinión---dijo su ilustrísima (que era una de las lumbreras del
Episcopado)---es que si no constara en los autos, como aseguran consta
de una manera indubitable, que se ha fingido y se finge loco para hablar
impunemente de temas vedados, la ejecución de este hombre sería un
asesinato. Desempeña este desgraciado su papel con inaudita perfección,
y apreciándole por lo que dice, no hay en aquella mollera ni el más
pequeño grano de juicio\ldots{} A propósito de juicio, Sr.~de Pipaón, no
lo ha tenido usted muy grande fijando para el lunes la gran fiesta de
desagravios a Su Divina Majestad que celebra la Hermandad de
\emph{Indignos esclavos del Santísimo Sacramento}, porque siendo el
lunes día de la Natividad de Nuestra Señora, la \emph{Real Congregación
de la Guardia y Custodia} dispone por antiguo privilegio de la iglesia
de San Isidro.

Pipaón respondió, \emph{mutatis mutandis}, que no correría sangre a
causa de un conflicto entre ambas hermandades, y que él respondía de
arreglarlo todo a gusto de clérigos y seglares, y sin que se quejaran el
Santísimo Sacramento ni Nuestra Señora, con lo cual y con aceptar la
carroza de Su Ilustrísima para trasladarse a la calle de la Puebla donde
había de hacer el ensayo de la gavota antes de la tertulia, tuvo fin
aquel diálogo.

Ya avanzada la noche se cerró la capilla a los curiosos, y también la
puerta de la cárcel, después que entraron seis presos recién sacados de
sus casas por delaciones infames. Una nueva conspiración descubierta dio
mucho que hacer aquella noche y en la siguiente mañana al Sr.~Chaperón.

D. Patricio se acostó a dormir en la alcoba inmediata a la capilla; pero
su sueño no fue tranquilo. Velábanle solícitos y siempre prontos a
servir en todo los hermanos de la Paz y Caridad. Sola no se apartó de la
capilla ni un solo instante ni de día ni de noche.

---Abuelito querido---le dijo al amanecer,---estoy muerta de pena,
porque ve que tu conducta no es propia de un buen cristiano.

---Adorada hija---repuso Sarmiento besándola con ardiente cariño,---si
es propia de un filósofo, lo será de un cristiano, porque el filósofo y
el cristiano se juntan, se compendian y amalgaman en mí
maravillosamente. Hazme el favor de ver si esos señores hermanos me han
preparado el chocolate\ldots{} No extraño tus observaciones, hija mía.
Eres mujer y hablas con tu preciosa sensibilidad, no con la razón que a
mí me alumbra y guía. ¡Bendito sea Dios que me permite tenerte a mi lado
en estas horas postreras! Si no te estuviera viendo, quizás me faltaría
el valor que ahora tengo. Una sola cosa me afecta y entristece, nublando
el esplendoroso júbilo de mi alma, y es que mañana a la hora de las
diez\ldots{} porque supongo que\ldots{} eso será a las diez\ldots{}
dejaré de recrear mis ojos con la contemplación de tu angelical
persona\ldots{} Pero ¡ay! tú debes seguir viviendo; no ha llegado aún la
hora de tu entrada en la mansión divina; llegará, sí, y entrarás, y el
primero a quien verás en la puerta abriendo los brazos para recibirte en
ellos amoroso y delirante será tu abuelito Sarmiento, tu viejecillo
bobo.

La voz temblorosa indicaba una viva emoción en el reo.

---Y te llevaré a presencia del Padre de todo lo existente y le diré:
«¡Señor, aquí la tienes; esta es, mírala!\ldots» Pero no quiero
afligirte más. Ahora oye varios consejos que debo darte y algunos
encarguillos que quiero hacerte\ldots{} ¿Está ese chocolate?\ldots{}
Dame la mano para levantarme, hija mía. ¿Sabes que están pesados y duros
mis pobres huesos?\ldots{} ¡Ah! pronto tendrás este bocado, ¡oh
carnívora tierra! pronto, pronto se te arrojará esta piltrafa, que por
lo acecinada demuestra que te pertenece ya. El noble espíritu abandona
este inmundo saco, y vuela en busca de su patria y de sus congéneres los
ángeles.

Levantose delante de Sola porque estaba vestido. Un hermano le trajo el
chocolate, y quedándose solo con su amiga, le dijo estas palabras que
ella oyó con profundísima atención:

---Idolatrada hija, mañana a las diez nos separaremos para siempre. Dios
me dio la inefable dicha de conocerte, para que mi espíritu se
confortase antes de dejar el mundo. Te condujiste conmigo tan noble y
caritativamente que no vacilo en declararte merecedora de inmortal
premio. Yo te lo aseguro, yo te lo profetizo---dijo esto cerrando los
ojos y extendiendo solemnemente los brazos en actitud de profeta,---yo
te lo fío bendiciéndote. Creo tener poderes para ello. Gozarás de la
eterna dicha por tu cristiana acción. Ahora bien; hablando de cosas más
terrestres, te diré que es mi deseo partas en seguida para Inglaterra a
ponerte bajo el amparo de ese hombre generoso que ha sido tu protector y
hermano. Le conozco y sé que su corazón está lleno d bondades. Como me
intereso también por él, declaro ante ti que ese joven debe tomarte por
esposa, de lo cual resultará ventaja para entrambos; para ti porque
vivirás al arrimo de un hombre de mérito, capaz de comprender lo que
vales; para él porque tendrá la compañera más fiel, más amante, más
útil, más hacendosa, más cristiana y más honesta con que puede soñar el
amor de un hombre. Tengo la seguridad de que él lo comprenderá así---al
decir esto mostraba la convicción de un apóstol.---Si no lo
comprendiese, dile que yo se lo mando, que es mi sacra voluntad, que yo
no hablo por hablar, sino transmitiendo por el órgano de mi lengua la
inspiración celeste que obra dentro de mí.

Sola oyó este discurso con recogimiento y admiración, pasmada de
advertir una profundísima concordancia entre la demencia de su amigo y
ciertas ideas de antiguo arraigadas en ella. No acertó a decir una
palabra sobre aquel tema, y su viejecillo bobo se le representó entonces
grande y luminoso, cual nunca lo había visto, más respetable que todo lo
que como respetable se presenta en el mundo.

Después de una pausa, durante la cual apuró el pocillo, Sarmiento
prosiguió así:

---Querida hija de mi corazón, voy a hacerte un encargo, atañedero a
cosas terrestres. Las cosas terrestres también me ocupan, porque de la
tierra salí, y en ella he de dejar las preciosas enseñanzas que se
desprenden de mi martirio. El género humano merece mi mayor interés. La
dicha del Cielo no sería completa, si desde él no contempláramos la
constante labor de este pobre género humano, sin cesar trabajando en
mejorarse. Los que de él salimos no podemos dejar de enviarle desde allá
arriba un reflejo de nuestra gloria, sin lo cual se envilecería,
acercándose más a las bestias que a los ángeles. Hay que pensar en el
género humano de hoy, que es el coro celestial e inmenso de mañana, y
todo hombre es la crisálida de un ángel, ¿me entiendes? Si las criaturas
superiores, al remontarse sobre los mundanos despojos, miraran con
desprecio esta pobre turba inquieta y enferma a que pertenecieron; si no
atendiendo más que al Eterno Sol, hicieran del deseo de la
bienaventuranza un egoísmo, adiós universo, adiós pasmoso orden de cielo
y tierra, adiós concierto sublime. No, yo miro a la tierra y la miraré
siempre. Le dejo un don precioso, mi vida, mi historia, mi ejemplo, hija
mía, ¿sabes tú lo que vale un buen ejemplo para esta mísera chusma
rutinaria? Sí, mi historia será pronto una de las más enérgicas
lecciones que tendrá el rebaño humano para implantar la libertad que ha
de conducirle a su mejoramiento moral. Pero digo yo, ¿es fácil escribir
esa historia? No.~Bien conocidos son mis discursos, y aunque yo no los
he escrito, como todo el mundo lo tiene grabados en la memoria, no
faltará quien los dé a la estampa. Sócrates no dejó escrito nada\ldots{}
Pero si serán perpetuados mis discursos, habrá gran escasez de datos
biográficos respecto a mí. Oye, pues, lo que voy a decirte.

Tomando a Sola por un brazo, la acercó a sí:

---Viviendo en tu casa---añadió,---apunté no hace dos meses, los
principales datos de mi vida, tales como el día de mi nacimiento, el de
mi bautizo, el de mi confirmación, el de mi boda con Refugio, el del
feliz natalicio de Lucas, el de mi entrada en la enseñanza y otros: son
datos preciosísimos. Como los historiadores han de empezar desde mañana
mismo a revolver archivos y libros parroquiales, yo te encargo que les
saques de apuros. Mira tú; el apunte en que constan esos datos está
escrito con lápiz\ldots{} Me parece que lo puse debajo del hule de la
cómoda. Búscalo bien por toda la casa, y entrégalo a esos señores. Al
punto sabrás quiénes son, porque no se hablará de otra cosa en todo el
mundo. No te descuides, y evitarás mil quebraderos de cabeza, y quizás
inexactitudes y errores que darán ocasión a desagradables polémicas.

Sola sintió al oír esto que la admiración despertada por anteriores
palabras del viejecillo bobo, se disipaba como humo. ¡Cuán difícil era
señalar la misteriosa línea donde los desvaríos de Sarmiento se trocaban
en ingeniosas observaciones, o por el contrario, sus admirables vuelos
en lastimoso rastrear por el polvo de la necedad! La joven prometió
cumplir fielmente todo lo que le mandaba.

Al poco rato apareció el padre Alelí preparado para decir la misa, y
empezada esta, Sarmiento la ayudó con extraordinaria devoción y acierto,
tan seguro en las ceremonias como si hubiera sido monaguillo toda su
vida. Soledad la oyó con gran edificación acompañada de los hermanos y
de algunos empleados de la cárcel. Después, por orden del Sr.~Chaperón,
se cerró la capilla al público.

\hypertarget{xxviii}{%
\chapter{XXVIII}\label{xxviii}}

Poniendo sobre todas las cosas su anhelante deseo de llegar pronto al
fin de la jornada vital, que era el comienzo de su triunfo, Sarmiento
deploraba que la justicia de aquellos tiempos hubiese fijado en cuarenta
y ocho horas el plazo de la preparación religiosa. Con diez o doce horas
había bastante, según él. Los dos frailes que le asistían aprovecharon
la ocasión de su soledad para hablarle recio en el negocio de la
salvación, logrando que D. Patricio atendiese a él, y consintiera en oír
el trasnochado sermoncillo que preparado traía el padre Salmón. Después
de comer, cuando Sola vencida por el cansancio había cedido al sueño y
dormitaba sentada, el padre Alelí logró hacerse oír de Sarmiento con
mayor interés. Por la noche pareció que el espíritu del buen viejo se
recogía y como que se amilanaba algún tanto, mostrándose además en su
rostro y cuerpo cierto desmayo o fatiga. El patriota no permanecía ya en
pie, sino recostado con abandono en el sillón, fijando la vista en el
suelo cual si cayera en meditación taciturna. Silencio profundísimo
reinaba en la cárcel; las velas se habían consumido mucho y ardían en el
último cabo de ellas, elevando entre la vacilante luz el negro pábilo
caduco, y derramando cera amarilla en grandes chorros sobre los
candeleros y sobre el altar. El Crucifijo y la Dolorosa parecían
entregados a un sopor misterioso. Nunca, como en aquella tristísima
hora, había parecido la capilla lúgubre y conmovedora. Su ambiente de
panteón daba frío, su luz tenue convidaba a morirse y enterrarse. Era la
madrugada del último día.

No fue insensible el espíritu de Sarmiento a esta influencia externa, y
conociéndolo Alelí, le dijo que ya le quedaban pocas horas; que viese lo
que hacía si no deseaba arder perpetuamente en los infiernos. Al oír
esto, mirole Sarmiento con desdén y levantándose del sillón, se puso de
rodillas.

---Puesto que Su Paternidad quiere que confiese, confesaré---dijo
lacónicamente.

---No es preciso que se arrodille usted, hermano mío---indicó el buen
fraile levantándole.---En estos casos permitimos al penitente que haga
la confesión sentado para evitarle cansancio.

---Yo prefiero estar de rodillas, porque no soy de alfeñique---dijo el
reo volviéndose a hincar.---Ahora, si Vuestra Paternidad tiene oídos,
oiga\ldots{} Yo amo a Dios sobre todas las cosas. ¿Cómo no amarle, si es
fuente de todo bien, manantial de toda idea, origen de toda vida? Él dio
la idea moral al mundo, y el mundo, después de mil luchas, disputas y
sangre, aceptó la ley moral que felizmente lo rige. Después le dio la
idea política, es decir, la libertad, para que se gobernase, y todavía
el mundo no la ha aceptado en su totalidad. Estamos en la época de la
predicación, del martirio\ldots{}

---Basta---dijo Alelí con enfado.---Está usted profanando el nombre de
Dios con absurdas afirmaciones. Poco adelantamos por ese camino, hermano
querido. Confiese usted su amor a Dios, sin mezcla de extravagancia
alguna. Me basta con eso por ahora, y adelante.

---Confieso---añadió el penitente,---que con frecuencia he jurado su
santo nombre en vano, y además que he usado votos y ternos raros, pues
adquirí tiempo ha la pícara costumbre de sacar a todo el Chilindrón y la
Chilindraina; pero, con perdón de Vuestra Reverencia, creo que pecados
como este no llevan a casa de Pedro Botero. Tampoco he santificado las
fiestas como está mandado\ldots{} desidia, pura desidia y abandono. En
el cuarto, ¿qué he de decir sino que jamás he faltado a él ni en
pensamiento? Pues en lo de matar, si alguien perdió por mí la vida fue
en leal acción de guerra y cuando el honor de mi bandera me lo mandaba
así. No obstante, un pecado grave tengo en lo tocante a este
mandamiento, y ese lo voy a confesar aquí con la boca y con el corazón,
porque ha tiempo pesa sobre mi conciencia, y aunque estoy muy
arrepentido, paréceme que jamás logro echar de mí la mancha y peso que
me dejó. Hallándose preso y encadenado un vecino mío, padre de esta
joven que me acompaña, pidió un vaso de agua y se lo negué. ¡Qué infame
bellaquería! Pero válgame mi contrición sincera y el cariño ardiente que
después he puesto en la bendita hija de aquel desgraciado.

---Adelante---murmuró Alelí satisfecho de que hubiese algún pecado
evidente que justificase su ministerio.

---Del sexto no diré más sino que después de la muerte de mi Refugio,
que acaeció hace veintidós años, he observado castidad absoluta, a pesar
de ser solicitado para faltar a aquella preciosa virtud por más de una
hembra que no debió de mirarme cual saco de paja. Tampoco he robado
jamás a nadie ni el valor de un alfiler, y en el ramo de mentir si
alguna vez falté a la verdad fue en negocios baladís y de poca monta.

---Alto, alto---dijo Alelí con interés sumo, viendo llegado el tema que
abordar quería.---Usted ha mentido, y ha mentido gravemente por sistema
sosteniendo un papel engañoso con la terquedad del hombre más perverso.
Es opinión general que usted se finge demente, poseyendo en realidad un
claro juicio; es público y notorio, y así consta en la causa, que todos
esos disparates con que ha divertido a Madrid son obra del talento más
astuto, para poder vivir en una sociedad que proscribe a los
revolucionarios. Vamos a ver, hermano mío, repare usted delante de quién
está, mire esa imagen sacratísima, considere que le restan pocas horas
de vida, considere que ya no es posible la mentira, y ábrame su corazón
y arroje la máscara y dígame si en efecto este hombre exaltado que vemos
es un hábil histrión. ¡Ah! hermano mío, aseguran que usted sostiene su
papel, esperando que le indulten por tonto\ldots{} ¡error, error, porque
no es ese el camino del indulto! Más fácil le sería conseguirlo con una
confesión franca de su pecado\ldots{} Al menos haciéndolo así, tendrá el
perdón de Dios y la gloria eterna.

---¡Yo farsante, yo histrión, yo\ldots! ¡yo!---exclamó Sarmiento
clavando amba manos, como garras, en su pecho.

Miraba al padre Alelí con los ojos encendidos y con expresión de
sorpresa, que bien pronto se tornó en amargo desdén.

---Usted no me comprende\ldots---dijo levantándose.---Vaya usted a
confesar colegiales, señor padre Alelí. Me confesaré solo.

Y arrodillándose delante del altar, alzó las manos y sin quitar los ojos
del Crucifijo, habló así:

---Señor, Tú que me conoces no necesitas oír de mi boca lo que siente mi
corazón, que pronto dará su último latido dejándome libre. Sabes que te
adoro, que te reverencio, y que ejecuto puntualmente la misión que me
señalaste en el mundo. Sabes que la idea de la libertad enviada por Ti
para que la difundiéramos, fue mi norte y mi guía. Sabes que por ella
vivo y por ella muero. Sabes que si cometí faltas, me he arrepentido de
ellas con grandísima congoja. Sabes que perdono de todo corazón a mis
enemigos, y que me dispongo a rogar por ellos, cuando mi espíritu pueda
hablar sin boca y ver sin necesidad de ojos. Mi confesión está hecha
públicamente. Óigala todo el que tiene oídos.

Y después volviéndose al fraile que enfrente y absorto le miraba,
díjole:

---Ahora, padre Alelí, espero que no tendrá Vuestra Paternidad
reverendísima inconveniente alguno en darme el pan Eucarístico. Bien se
ve que puedo recibir a Dios dentro de mí. Estoy puro de toda mancha: soy
como los ángeles.

Entonces viose una cosa extraña, que por lo extraña parecía horrible en
aquel sitio y ocasión. El padre Alelí no pudo evitar una sonrisa.
Diríase que esta brilló en la fúnebre capilla como un reflejo mundano
dentro de la región de los difuntos. Pero contuvo al punto su hilaridad,
y gravemente dijeron a dúo ambos frailes:

---No podemos dar a usted la Eucaristía, desgraciado hermano.

Mientras Sola acudió a consolar a Sarmiento que parecía muy contrariado
por aquella negativa, Alelí llevó aparte a Salmón y le dijo:

---Es más tonto que hecho de encargo. Yo repito que ajusticiar a este
hombre es un asesinato, y Chaperón, los jueces que le sentenciaron y
nosotros que le asistimos, estamos más locos que él. Yo no puedo ver
este horrible espectáculo. ¿Pero no es evidente que ese hombre es necio
de capirote? Estamos coadyuvando a una obra inicua. ¡Y esperábamos que
confesase su comedia!

---Como siempre le tuve por mentecato redomado, no me he llevado chasco.

No sé para qué nos traen aquí.

---Ni yo. Voy a hablar con Chaperón.

---Yo no me tomaría el trabajo de hablar con nadie.

---Pues yo sí.

---Pues yo no.

Poco después de esto el reo vio los objetos y las personas con una
claridad que le conturbó sobremanera sin saber por qué. Era que había
avanzado el día y la capilla recibía un poco de luz, ante la cual
palidecía ligeramente la de las soñolientas velas, casi consumidas.
Aquel débil resplandor del astro rey hizo daño a la retina y al espíritu
del viejo, sin que su entendimiento pudiera explicarse la razón de ello.

---Es de día---dijo con cierto asombro, y al punto se quedó taciturno.

Los hermanos de la Caridad aparecían más compungidos que en el día
anterior, y rezaban devotamente arrodillados ante el altar. Salmón rogó
al condenado que se sentase, y poniéndose a su lado hízole exhortaciones
encaminadas a apartar su alma del tremendo abismo a cuyo borde se
encontraba.

---Pocas horas me restan---murmuró el patriota, dando un gran
suspiro.---Mi alma será más fuerte cuanto más cerca esté el instante
lisonjero de su liberación. ¿Cuántas horas faltan?

---No cuente usted las horas\ldots{} ¿Qué valen dos ni tres horas
comparadas con la eternidad?

Sarmiento no respondió nada. Observaba los ladrillos del piso y fijaba
su vista con minuciosidad aritmética en todos aquellos que tenían el
ángulo gastado. Diríase que los contaba.

---¿En dónde está mi hija?---dijo de súbito moviendo la cabeza con
ansiedad.---Sola, niña de mi corazón, no te separes de mí.

Sola se arrojó llorando en sus brazos. Notó que tenía las manos frías y
temblorosas.

---Dentro de poco dejaré de verte---exclamó el viejo haciendo esfuerzos
verdaderamente heroicos para dominar su emoción.---Que sea tan flaca y
miserable esta humana Naturaleza, que ni aun teniendo por segura la
entrada en la morada celestial, pueda mirar con absoluto desprecio los
afectos del mundo\ldots{} Aquí me tienes más valiente que un león (sus
labios temblaban al decirlo y su voz era como el ronco trinar de una ave
moribunda), y sin embargo, esto de separarme de ti, esto de dejarte
sola\ldots{}

Se pasó la mano por la frente, y durante un rato tapose los ojos.

---No sé por qué está triste el día---murmuró con disgusto.---¡Qué ruido
hay en la cárcel!\ldots{} ¿qué voces son esas? Parece un canto desacorde
o un graznido de pájaros llorones. ¿Qué es eso?

Soledad no contestó nada, y apoyó su frente sobre el pecho del anciano.
A la capilla llegaba una repugnante música llorona de gritos humanos que
parecía formada de todos los rencores, de todos los sarcasmos, de todas
las lágrimas y de todos los suspiros encerrados en la cárcel.

El padre Alelí, que había salido al amanecer, volvió muy cabizbajo, y
sin hablar una sola palabra al reo ni a los demás preparose para decir
la misa. En tanto, uno de los hermanos departía con Sarmiento de cosas
religiosas, sabedor de que estas habían de llevar gran alivio y fuerzas
al espíritu del reo.

---Hoy---le dijo,---celebramos en Santa Cruz los Mayordomos de esta Real
Archicofradía misa solemne de rogativa para implorar los divinos
auxilios en la última hora del pobre condenado a muerte. Ya sabe usted
que Nuestro Santísimo Padre Pío VII ha concedido indulgencia plenaria a
todos nosotros y a los fieles que asistan a esa misa y hagan oración por
la concordia de los Príncipes cristianos, extirpación de las herejías y
exaltación de la Fe católica.

---De modo---dijo Sarmiento con amarga ironía,---que en esa misa se hace
oración por todo menos por mí.

---No, hermano mío, no---dijo el cofrade con la melosidad del
beato,---que también habrá lo que llamamos ejercicio de agonía, donde se
hace la recomendación del alma del reo; luego siguen las jaculatorias de
agonía y se cantará el \emph{ne recorderis}. Los más bellos himnos de la
Iglesia y las piadosas oraciones de los fieles acompañan a usted en su
tránsito doloroso\ldots{} ¿qué digo doloroso? gloriosísimo. Piense usted
en la pasión de Nuestro Señor Jesucristo, y se sentirá lleno de valor.
¡Oh, feliz mil veces el que abandona esta vida miserable libre de todo
pecado!

El hermano inclinó la cabeza a un lado, bajando los ojos y cruzando las
manos en mística actitud. Después rezó en silencio.

El padre Alelí dijo la misa, que oyó Sarmiento como el día anterior, de
rodillas y con profunda atención. Al concluir sentose con muestras de
gran cansancio; mas ponía mucho empeño en disimularlo.

---¿No quiere usted tomar nada?---le dijo uno de los hermanos.---Hemos
preparado un almuerzo ligero. ¿Se siente usted mal, hermano querido?
Vamos, un huevo frito y un poco de jamón\ldots{} Si para eso no se
necesita gana---añadió viendo que el patriota hacía signos negativos con
la cabeza y con la mano.---Sí, lo traeremos, y también un vaso de vino.

---No quiero nada.

---¿Ni café?

---Tomaré el café por complacer a ustedes---repuso Sarmiento sonriendo
con tristeza.

Alelí se sentó junto a él y tomándole la mano se la apretó cariñosamente
diciéndole:

---Hermano mío, en nombre de Dios y de María Santísima, a cuya presencia
llegará usted pronto, si sabe morir como cristiano en estado de
contrición perfecta, le ruego que no me oculte sus pensamientos, si por
ventura son distintos de lo que ha manifestado aquí y fuera de aquí.

---Si yo ocultara mis pensamientos, si yo no fuera la misma
verdad---replicó D. Patricio con la entereza más noble,---no sería digno
de este nobilísimo fin que me espera\ldots{} ¡Ah! señores, la taimada
naturaleza nos tiende mil lazos por medio de la sensibilidad y del
instinto de conservación; pero no, no será mi grande espíritu quien
caiga en ellos. Vamos, vamos de una vez.

Y se levantó.

---Calma, calma, hermano mío; aún no es tiempo---le dijo Alelí tirándole
del brazo.---Siéntese usted. Por cierto que no es nada conveniente para
su alma esa afectación de valor y ese empeño de sostener el papel de
héroe. Una resignación humilde y sin aparato, una conformidad decorosa
sin disimular el dolor y un poco de entereza que demuestre la convicción
de ganar el cielo, son más propias de esta hora que la fanfarronería
teatral. Usted está nervioso, desazonado, inquieto, sin sosiego,
tiémblanle las carnes y se cubre su piel de frío sudor.

---El que era Hijo de Dios sudó sangre---afirmó Sarmiento con brío;---yo
que soy hombre, ¿no he de sudar siquiera agua?\ldots{} Vamos pronto.
Repito que tengo vivos deseos de concluir.

Entonces sintiose más fuerte el coro de lamentos, y al mismo tiempo
ronco son de tambores destemplados.

---He aquí las tropas de Pilatos---observó Sarmiento.

---Hermano, hermano querido---le dijo Alelí abrazándole.---Una palabra,
una palabra sola de verdadera piedad, de verdadera religiosidad, de amor
y temor de Dios. Una palabra y basta; pero que sea sincera, salida del
fondo del corazón. Si la dice usted, todos esos pensamientos livianos de
que está llena su cabeza, como desván lleno de alimañas, huirán al ver
entrar la luz.

---Cristiano católico soy---afirmó Sarmiento.---Creo todo lo que manda
creer la Iglesia, creo todos los misterios, todos los sagrados dogmas,
sin exceptuar ninguno. He oído misa, he confesado sin omitir nada de lo
que hay en mi conciencia, he deseado ardientemente recibir la
Eucaristía, y si no la he recibido ha sido porque no han querido
dármela. ¿Qué más se quiere de mí? ¡Oh! Señor de cielos y tierra, ¡oh!
tú, María, Madre amantísima del género humano, a vosotros vuelvo mis
miradas, vosotros lo sabéis, porque veis mi rostro, no este de la carne
sino el del espíritu. Los que no ven el de mi espíritu, ¿cómo pueden
comprenderme? Hacia Vosotros volaré, invocándoos, llevando en mi diestra
la bandera que habéis dado al mundo, la bandera de la libertad, por la
cual he vivido y por la cual muero.

Salmón y Alelí movieron la cabeza. Su pena y desasosiego eran muy
profundos. Soledad, sin fuerzas ya para luchar con su dolor, estaba a
punto de perder el conocimiento. Don Patricio, dicho su último discurso,
examinaba una grieta que en el techo había y después la costura del paño
del altar. Creeríase al verle que aquellos dos objetos insignificantes
merecían la mayor atención.

Varias personas entraron en la capilla, todas decorando sus caras con la
aflicción más edificante. El reo se levantó y sin dejar de observar la
costura del altar, habló así solemnemente:

---Cayo Graco, Harmodio y Aristogitón, Bruto\ldots{} héroes inmortales,
pronto seré con vosotros\ldots{} y tú, Lucas, hijo mío, que estás en las
filas de la celestial infantería, avanza al encuentro de tu dichoso
padre.

Los frailes, puestos de rodillas, recitaban oraciones y jaculatorias,
empeñándose en que el reo las repitiera; pero Sarmiento se apartó de
ellos afirmando:

---Todo lo que puede decirse lo he dicho en mi corazón durante la misa y
después de ella.

Oyose el tañido de la campana de Santa Cruz.

---Tocan a muerto---dijo Sarmiento.---Yo mandaría repicar y alzar arcos
de triunfo, como en el día más grande de todos los días. ¡Ya veo tus
torres, oh patria inmortal, Jerusalén amada! ¡Bendito el que llega a ti!

El alcaide le saludó, enmascarándose también con la carátula de piedad
lastimosa que pasaba de rostro en rostro, conforme iban entrando uno y
otro personaje. Después separáronse todos para dar paso a un hombre
obeso, algo viejo, vestido de negro, cuyo aire de timidez contrastaba
singularmente con su horrible oficio: era el verdugo, que avanzando
hacia el reo, humilló la frente como un lacayo que recibe órdenes.

D. Patricio sintió en aquel momento que un rayo frío corría por todo su
cuerpo desde el cabello hasta los pies, y por primera vez desde su
entrada en la fúnebre capilla sintió que su magnánimo corazón se
arrugaba y comprimía.

---Sí, sí, perdono, perdono a todo el mundo---balbució el reo, fijando
otra vez toda su atención en los ladrillos del piso.---Vamos ya\ldots{}
¿No es hora de ir?

Pero su ánimo, rápidamente abatido, forcejeó iracundo en las tinieblas y
se levantó. Fue como si se hubiera dado un latigazo. La dosis de energía
que desplegara en aquel momento era tal, que sólo estando muerta hubiera
dejado la mísera carne de responder a ella. Tenía Sarmiento entre las
manos su pañuelo y apretando los dedos fuertemente sobre él, y separando
las manos lo partió en dos pedazos sin rasgarlo. Cerrando los ojos
murmuraba:

---¡Cayo Graco!\ldots{} ¡Lucas!\ldots{} ¡Dios que diste la libertad al
mundo\ldots!

El verdugo mostró un saco negro. Era la hopa que se pone a los
condenados para hacer más irrisorio y horriblemente burlesco el crimen
de la pena de muerte. Cuando el delito era de alta traición la hopa era
amarilla y encarnada. La de Sarmiento era negra. Completaba el ajuar un
gorro también negro.

---Venga la túnica---dijo preparándose a ponérsela.---\emph{Reputo el
saco como una vestidura de gala y el gorro como una corona de
laurel}\footnote{Estas palabras las dijo el valeroso patriota ahorcado
  el 24 de Agosto 1825. Su noble y heróico comportamiento en las últimas
  horas, da en cierto modo carácter historico al personaje ideal que es
  protagonista de esta obra.}.

Después le ataron las manos y le pusieron un cordel a la cintura, a
cuyas operaciones no hizo resistencia, antes bien, se prestó a ellas con
cierta gallardía. Incapacitados los movimientos de sus brazos, llamó a
Sola y le dijo:

---Hija mía, ven a abrazar por última vez a tu viejecillo bobo.

La huérfana lo estrechó en sus brazos, y regó con sus lágrimas el cuello
del anciano.

---¿A qué vienen esos lloros?---dijo este sofocando su emoción.---Hija
de mi alma, nos veremos en la gloria, a donde yo he tenido la suerte de
ir antes que tú. De mi imperecedera fama en el mundo, tú sola, tú serás
única heredera, porque me asististe y amparaste en mis últimos días. Tu
nombre, como el mío, pasará de generación en generación\ldots{} No
llores; llena tu alma de alegría, como lo está la mía. Hoy es día de
triunfo; esto no es muerte, es vida. El torpe lenguaje de los hombres ha
alterado el sentido de todas las cosas. Yo siento que penetra en mí la
respiración de los ángeles invisibles que están a mi lado, prontos a
llevarme a la morada celestial\ldots{} es como un fresco
delicioso\ldots{} como un aroma delicado\ldots{} Adiós\ldots{} hasta
luego, hija mía\ldots{} no olvides mis dos recomendaciones, ¿oyes? Vete
con ese hombre\ldots{} ¿oyes?\ldots{} lo apuntes\ldots{} Adiós, mi
glorioso destino se cumple\ldots{} ¡Viva yo! ¡Viva Patricio Sarmiento!

Desprendieron a Sola de sus brazos; tomola en los suyos el alcaide para
prestarle algún socorro, y D. Patricio salió de la capilla con paso
seguro.

El padre Alelí le ató un Crucifijo en las manos y Salmón quiso ponerle
también una estampa de la Virgen; pero opúsose a ello el reo diciendo:

---Con mucho gusto llevaré conmigo la imagen de mi Redentor, cuyo
ejemplo sigo; pero no esperen Vuestras Paternidades que yo vaya por la
carrera besando una estampita. Adelante.

Al llegar a la calle presentáronle el asno en que había de montar, y
subió a él con arrogantes movimientos, diciendo:

---He aquí la más noble cabalgadura cuyos lomos han oprimido héroes
antiguos y modernos. Ya estoy en marcha.

Al llegar a la calle de la Concepción Jerónima y ver el inmenso gentío
que se agolpaba en las aceras y en los balcones, en vez de amilanarse,
como otros, se creció, se engrandeció, tomando extraordinaria altitud.
Revolviendo los ojos en todas direcciones, arriba y abajo, decía para
sí:

---Pueblo, pueblo generoso, mírame bien, para que ningún rasgo de mi
persona deje de grabarse en tu memoria. ¡Oh! ¡si pudiera hablarte en
este momento!\ldots{} Soy Patricio Sarmiento, soy yo, soy tu grande
hombre. Mírame y llénate de gozo, porque la libertad por quien muero
renacerá de mi sangre, y el despotismo que a mí me inmola perecerá
ahogado por esta misma sangre, y el principio que yo consagro muriendo,
lo disfrutarás tú viviendo, lo disfrutarás por los siglos de los siglos.

El murmullo del pueblo crecía entre los roncos tambores, y a él le
pareció que toda aquella música se juntaba para exclamar:

---¡Viva Patricio Sarmiento!

El padre Alelí le mostraba el Crucifijo que en su mano llevaba (el mismo
padre Alelí) y le decía que consagrase a Dios su último pensamiento.
Después el venerable fraile rezaba en silencio, no se sabe si por el
reo, o por sus jueces. Probablemente sería por estos últimos.

Al llegar a la plazuela, Sarmiento extendió la vista por aquel mar de
cabezas, y viendo la horca, dijo:

---¡Ahí está!\ldots{} ahí está mi trono.

Y al ver aquello, que a otros les lleva al postrer grado de abatimiento,
él se engrandeció más y más, sintiendo su alma llena de una exaltación
sublime de entusiasmo expansivo.

---Estoy en el último escalón, en el más alto---dijo.---Desde aquí veo
al mísero género humano, allá abajo, perdido en la bruma de sus rencores
y de su ignorancia. Un paso más y penetraré en la eternidad, donde está
vacío mi puesto en el luminoso estrado de los héroes y los mártires.

Al pie de la horca, rogáronle los frailes que adorase al Crucifijo, lo
que hizo muy gustoso, besándolo y orando en voz alta con entonación
vigorosa.

---Muero por la libertad como cristiano católico---exclamó ¡Oh! Dios, a
quien he servido, acógeme en tu seno.

Quisieron ayudarle a subir la escalera fatal; pero él desprendiéndose de
ajenos brazos, subió solo. El patíbulo tenía tres escaleras; por la del
centro subía el reo, por una de las laterales el verdugo y por la otra
el sacerdote auxiliante. Cada cual ocupó su puesto. Al ver que el cordel
rodeaba su cuello, Sarmiento dijo con enfado:

---¿Y qué? ¿no me dejan hablar?

Los sacerdotes habían empezado el Credo. Callaron. Juzgando que el
silencio era permiso para hablar, el patriota se dirigió al pueblo en
estos términos:

---Pueblo, pueblo mío, contémplame y une tu voz a la mía para gritar:
¡Viva la\ldots!

Empujole el verdugo y se lanzó con él.

Cayeron de rodillas los sacerdotes que habían permanecido abajo, y
elevando el Crucifijo exclamaron consternados:

---¡Misericordia, Señor!

La muchedumbre lanzó el trágico murmullo que indicaba su curiosidad
satisfecha y su fúnebre espanto consumado.

El padre Alelí dijo tristemente:

---Desgraciado, sube al Limbo.

\hypertarget{xxix}{%
\chapter{XXIX}\label{xxix}}

¿Qué sabía él?\ldots{} A pesar de ser fraile discreto y gran sabedor de
teología, ¿qué sabía él si su penitente había ido al Limbo o a otra
parte? ¿Quién puede afirmar a dónde van las almas inflamadas en
entusiasmo y fe? ¿Habrá quien marque de un modo preciso la esfera donde
el humano sentido merecedor de asombro y respeto, se trueca en la
enajenación digna de lástima? Siendo evidente que en aquella alma se
juntaban con extraña aleación la excelsitud y la trivialidad, ¿quién
podrá decir cuál de estas cualidades vencía a la otra? Glorifiquémosle
todos. Murió pensando en la página histórica que no había de llenar, y
en la fama póstuma que no había de tener. ¡Oh, Dios poderoso! ¡Cuántos
tienen esta con menos motivo, y cuántos ocupan aquella habiendo sido tan
locos como él, y menos, mucho menos sublimes!

\flushright{Madrid, Octubre de 1877.}

~

\bigskip
\bigskip
\begin{center}
\textsc{Fin de el terror de 1824.}
\end{center}
\normalsize

\end{document}
