\PassOptionsToPackage{unicode=true}{hyperref} % options for packages loaded elsewhere
\PassOptionsToPackage{hyphens}{url}
%
\documentclass[oneside,11pt,spanish,]{extbook} % cjns1989 - 27112019 - added the oneside option: so that the text jumps left & right when reading on a tablet/ereader
\usepackage{lmodern}
\usepackage{amssymb,amsmath}
\usepackage{ifxetex,ifluatex}
\usepackage{fixltx2e} % provides \textsubscript
\ifnum 0\ifxetex 1\fi\ifluatex 1\fi=0 % if pdftex
  \usepackage[T1]{fontenc}
  \usepackage[utf8]{inputenc}
  \usepackage{textcomp} % provides euro and other symbols
\else % if luatex or xelatex
  \usepackage{unicode-math}
  \defaultfontfeatures{Ligatures=TeX,Scale=MatchLowercase}
%   \setmainfont[]{EBGaramond-Regular}
    \setmainfont[Numbers={OldStyle,Proportional}]{EBGaramond-Regular}      % cjns1989 - 20191129 - old style numbers 
\fi
% use upquote if available, for straight quotes in verbatim environments
\IfFileExists{upquote.sty}{\usepackage{upquote}}{}
% use microtype if available
\IfFileExists{microtype.sty}{%
\usepackage[]{microtype}
\UseMicrotypeSet[protrusion]{basicmath} % disable protrusion for tt fonts
}{}
\usepackage{hyperref}
\hypersetup{
            pdftitle={Memoria de un cortesano de 1815},
            pdfauthor={Benito Pérez Galdós},
            pdfborder={0 0 0},
            breaklinks=true}
\urlstyle{same}  % don't use monospace font for urls
\usepackage[papersize={4.80 in, 6.40  in},left=.5 in,right=.5 in]{geometry}
\setlength{\emergencystretch}{3em}  % prevent overfull lines
\providecommand{\tightlist}{%
  \setlength{\itemsep}{0pt}\setlength{\parskip}{0pt}}
\setcounter{secnumdepth}{0}

% set default figure placement to htbp
\makeatletter
\def\fps@figure{htbp}
\makeatother

\usepackage{ragged2e}
\usepackage{epigraph}
\renewcommand{\textflush}{flushepinormal}

\usepackage{indentfirst}

\usepackage{fancyhdr}
\pagestyle{fancy}
\fancyhf{}
\fancyhead[R]{\thepage}
\renewcommand{\headrulewidth}{0pt}
\usepackage{quoting}
\usepackage{ragged2e}

\newlength\mylen
\settowidth\mylen{...................}

\usepackage{stackengine}
\usepackage{graphicx}
\def\asterism{\par\vspace{1em}{\centering\scalebox{.9}{%
  \stackon[-0.6pt]{\bfseries*~*}{\bfseries*}}\par}\vspace{.8em}\par}

 \usepackage{titlesec}
 \titleformat{\chapter}[display]
  {\normalfont\bfseries\filcenter}{}{0pt}{\Large}
 \titleformat{\section}[display]
  {\normalfont\bfseries\filcenter}{}{0pt}{\Large}
 \titleformat{\subsection}[display]
  {\normalfont\bfseries\filcenter}{}{0pt}{\Large}

\setcounter{secnumdepth}{1}
\ifnum 0\ifxetex 1\fi\ifluatex 1\fi=0 % if pdftex
  \usepackage[shorthands=off,main=spanish]{babel}
\else
  % load polyglossia as late as possible as it *could* call bidi if RTL lang (e.g. Hebrew or Arabic)
%   \usepackage{polyglossia}
%   \setmainlanguage[]{spanish}
%   \usepackage[french]{babel} % cjns1989 - 1.43 version of polyglossia on this system does not allow disabling the autospacing feature
\fi

\title{Memoria de un cortesano de 1815}
\author{Benito Pérez Galdós}
\date{}

\begin{document}
\maketitle

\hypertarget{i}{%
\chapter{I}\label{i}}

En el nombre del Padre, del Hijo y del Espíritu Santo, doy principio a
la historia de una parte muy principal de mi vida; quiero decir que
empiezo a narrar la serie de trabajos, servicios, proezas y afanes, por
los cuales pasé en poco tiempo, desde el más oscuro antro de las regias
covachuelas, a calentar un sillón en el Real Consejo y Cámara de
Castilla.

Abran los oídos y escuchen y entiendan cómo un varón listo y honrado
podía medrar y sublimarse por la sola virtud de sus merecimientos, sin
sentar el pie en los tortuosos caminos de la intriga, ni halagar
lisonjero las orejas de los grandes con la música de la adulación, ni
poner tarifa a su conciencia o vil tasa a su honor, cual suelen hacer
los menguados ambiciosillos del día, después que las sanas costumbres,
la modestia, la sobriedad y la cristiana mansedumbre han huido
avergonzadas del mundo, y son tan míseros de virtud los tiempos, que no
se encuentra un hombre de bien aunque den por él medio millón de pícaros
vividores.

¡Bendito sea Dios, padre de los menesterosos, sustento de los débiles,
proveedor de los hambrientos, aposentador de los desamparados, amparo de
los desnudos, alivio de todos los pobrecitos que quieren ganarse la
vida, y despensero de las hormigas, de los pájaros y de los
pretendientes!\ldots{} ¡Bendito sea Dios, digo, que me ha conservado mis
sueldos, gajes, pensiones, viáticos, emolumentos y obvenciones, para que
desahogadamente y sin importunos cuidados pueda contar todos los pasos
de mi fabulosa carrera! ¡Oh! ¿Por qué he de ocultarlo? Carrera como la
mía no la hicieron más de cuatro, desde que brotó en la fecunda tierra
el tallo de los empleos públicos y abrieron sus polvorientas corolas de
papel los expedientes de Arbitrios, Propios, Tercias reales, Noveno,
Pósitos, Paja y Utensilios, Frutos civiles, Mandas, Renta de la Abuela,
Chapín de la Reina y demás yerbas que componían el placentero jardín de
la Administración.

Verdad es que si a grandes altitudes llegué, buenos porrazos recibí en
aquella bendita escala, luchando y desgreñándome a machaca-liendres con
los que querían subir antes que yo; si mucho y rápidamente subí,
agarreme también a buenos faldones. Y no se diga que manchan mi vida,
como la de otros muy lucidos en sus carreras, acciones feas y
vergonzosas. Eso no; que antes que nada es la inmaculada blancura de mi
alma cristiana. Dios es testigo de que jamás metí la mano en bolsillo
ajeno\ldots{} ¡Jesús, qué horror! Antes me habría dejado tostar en
parrillas que tomar de las arcas del Tesoro un ochavo de los que allí
estaban, conforme a los libros de cuenta y razón\ldots{} ¡Huye, Luzbel
maldito! \emph{Vade retro!}\ldots{} Detesto las violentas acciones,
mayormente cuando al varón allegador y celoso de su propio bien, no
faltan mil ingeniosos arbitrios, sutilezas prudentes y habilísimas
industrias para remediar sus escaseces. No fui yo el inventor de tales
alivios; que los aprendí de maestros muy doctos, cargados de
emolumentos, veneras, excelencias, y que pasaban por las más firmes
columnas del Estado y de la Iglesia, de lo cual colijo que las
sobredichas ingeniosidades no debían de ser pecaminosas. Y no digo más
por ahora, que a su tiempo y sazón se verán palmariamente las agudezas
de mi ingenio, y el filósofo así como el moralista, no podrán menos de
aprobarlas.

«¿Y quién es Vd.?\ldots»---preguntarán seguramente los que me leen.---Yo
soy aquel ---respondo,---que en los primeros años de su vida
administrativa se llamaba Juan Bragas, nombre que a decir verdad no se
distingue por su música, ni tiene saborcillo de elegancia, ni sonsonete
o cancamurria de nobleza; así es, que no bien comencé a sacar el pie del
lodo, añadí al apellido de mis padres el lugar de mi nacimiento, por lo
cual, siendo este Pipaón en Rioja de Álava, vine a llamarme D. Juan
Bragas de Pipaón. Sonaba esto pomposamente en mis orejas, y yo repetía
en voz alta mi propio nombre para señorearme con su grandiosidad, la
cual anunciaba por el solo efecto del silabeo la persona de un
embajador, consejero de Indias, fiscal de la Rota o Asistente de
Sevilla. Más adelante, como el Bragas no me pareciese del mejor gusto,
lo suprimí completamente, quedándome para el mundo presente y para la
posteridad en D. Juan de Pipaón, nombre breve y rotundo, que va dejando
ecos armoniosos doquiera que se pronuncia, y al cual no le vendrá mal la
conterilla del marquesado o condado que tengo entre ceja y ceja.

Bendito sea Dios, vuelvo a decir, que no abandona jamás a los
menesterosos; bendita sea la pródiga mano que a cada cual le da su
remedio, ora un pedazo de pan, si padece hambre, ora un buen amigo que
le ayude, si tiene ambicioncillas de medro. ¿Qué habría sido de mí si no
hubiera tropezado de manos a boca con aquel nobilísimo, con aquel sin
par sujeto, que echó de ver mis disposiciones y me llevó desde el
Purgatorio de la oscuridad y miseria, al Paraíso del favor, de la fama y
de la hartura? Hombre mejor no nació de vientre de mujer, ni se ha visto
un talentazo igual para todo aquello que fuera de la jurisdicción de la
suprema intriga, por cuyas prendas era la gran cabeza de aquellos
tiempos y un maravilloso regalo hecho por Dios a la afortunada nación
española, para que la sacara del mal traer en que se encontraba.

No estamparé aquí su nombre, porque los de personajes insignes no deben
ser expuestos a la vergüenza de las letras de molde, donde corren riesgo
de que la Historia y la Posteridad (ambas señoras muy amigas de meterse
en vidas ajenas) los tomen por su cuenta, atribuyéndoles esta o la otra
picardía y desfigurando con pérfido criterio sus honrados manejos. Pero
sin nombrar al santo, puedo referir los milagros. Era mi protector
diputado en las Cortes del año 14, donde brilló por su buen ojo y mejor
mano para meter en un laberinto de enredos y compromisos al bando
reformador. Acaudilló con singular tino a los que poco después se
llamaron \emph{Persas,} y fue uno de los que prepararon el paso dado por
Fernando (a quien todos llamaban entonces el \emph{suspirado}), contra
la Constitución. Gozaba mi protector fama de hombre ignorantísimo,
opinión que hubo de ser efecto de la ruin envidia, pues de su excelso
ingenio fueron muestras la zancadilla que echó a todos los reformistas,
y aquel celo y consumada destreza suya para ponerse en primer lugar,
luego que el \emph{Rey recobró sus legítimos derechos,} así como la
prontitud con que se proporcionó tres o cuatro sueldos por Obra Pía,
Pósitos, Penas de Cámara, etc\ldots, de los cuales el menor habría
contentado a un triste pedigüeño de otros tiempos.

Dios Todopoderoso, a quien no cesa de invocar mi gratitud, hizo que el
cuitado narrador de estos sucesos, topara con Su Excelencia en Enero de
1814, y que le cautivase principalmente por su buena letra y
singularísima habilidad para remedar la ajena, especialmente en toda
suerte de firmas y rúbricas. ¡Oh, y qué elogios hacía aquel buen hombre
de mis talentos caligráficos! ¡Y cómo ponderaba mi pulso, mi excelente
ojo y aquella soltura con que despachaba en cuatro rasgos las más
difíciles y para él inverosímiles imitaciones! Así es, que me traía en
palmitas, regalábame copiosamente, y aunque a veces solía decirme las
cosas entre una sofocante llovizna de bofetones, mi humildad y la
mansedumbre cristiana que Dios me dio, le volvían a su pacífico ser y a
sus bondades y deferencias conmigo.

El primer asunto importante en que su merced me ocupara, fue aquel que
la historia llama \emph{el asunto Oudinot,} y que fue saladísimo, como
obra de tales ingenios, aunque de escaso efecto por torpeza de algunos.
Con su poderosa inventiva fantaseó mi protector una conspiración que se
suponía fraguada por los liberales, de acuerdo con Napoleón, para
establecer en España la república Iberiana. ¡Diantre con la república, y
cuánto nos dio que reír, y cuántas cuchufletas y bufonadas entretuvieron
las nocturnas horas en que a solas nos dedicábamos a inventar cartas, a
remedar tipos de letra, a confeccionar programas y comunicaciones en
cifra! Lo cierto es que la conspiración salió que ni pintada, y daba
gusto ver aquella sutil trama, en la cual D. Agustín Argüelles aparecía
carteándose con un pinche francés, a quien nosotros por ensalmo hicimos
\emph{general Oudinot,} con otras muchas imaginarias picardías puestas
tan al vivo, que aún los autores de todo llegamos a creerlo, y nos
indignábamos contra los \emph{republicanos iberianos napoleónicos}.

Todo se lo llevó la trampa, a pesar de estar hecho con tanto esmero en
largas vigilias\ldots{} ¡Lástima de trabajo! La torpeza del necio
Berteau, criado de la duquesa de Osuna, y de cierto cura de Granada (a
quien después hicieron arzobispo), echó por tierra el más grandioso
edificio que levantaran humanos entendimientos. Descubriose que todo era
invención; formose causa, y aunque nadie se metió con nosotros, tuvimos
el pesar de que los mismos jueces se escandalizaran de tan
\emph{atrevida y necia calumnia.}

Pero desde entonces se redobló la buena amistad y estimación de mi
generoso protector, quien me puso en el secreto de graves planes,
convidándome a cooperar en su realización con todas las fuerzas de mi
talento y travesura. Véase, pues, qué pronto me había destinado la
divina Providencia a tomar parte en sucesos culminantes, de esos que
mudan y trastornan las naciones. Sí, señores, delante de mí, en una sala
del convento de Atocha, mi buen amigo, asistido de algunos padres graves
de dicha casa, redactó el famoso manifiesto de los \emph{Persas,} que
quedó perfilado y puesto en limpio por mí en 12 de Abril. Firmáronlo
sesenta y nueve individuos de lo más aprovechado que había en el reino y
en las Cortes, hombres estimadísimos del soberano, que entre ellos
repartió mitras y togas, para que no quedara sin premio su lealtad.

En cuanto a la mía acrisolada, continuó sin más premio por entonces que
el antiguo destinillo en la covachuela, y hasta después del 10 de Mayo y
de la caída de la \emph{Mamancia} y de la entrada en Madrid del
\emph{encantador} Fernando, no di señales de adelanto en mi carrera.
¡Oh, qué días aquellos! ¡Cuánta ansiedad sentíamos los buenos patricios,
esclavos de la libertad, suspensos entre la vida y la muerte, sin saber
cuándo veríamos el fin de la horrible tiranía de los \emph{mamones,
caparrotas, cuácaros, lameplatos y ceposquedos}, pues estos y otros
graciosos nombres daba a los liberales en su \emph{Atalaya de la Mancha}
el reverendo Padre Castro! ¡Y qué trasudores y congojas experimentamos
en todo Abril, ora creyendo segura la llegada del Rey con el
desquiciamiento de todo el catafalco constitucional, ora sospechando que
los infames francmasones nos secuestrarían al \emph{suspirado} Rey,
haciéndolo perdidizo en cualquier desfiladero, para encajarnos la
república Iberiana, que tanto daba que hablar en los barrios bajos y en
los claustros de mendicantes!

Pero la aproximación de las tropas de Wittingham nos dio aliento, y la
llegada del general Eguía, completa tranquilidad acerca del buen
resultado de lo que entre manos traían los \emph{Persas}. ¡Qué hombre
aquel! Era de los pocos, y es lástima que nuestra nación, agradecida a
su destreza y heroísmo, no le elevase una estatua ecuestre,
representándolo con su peluca de coleta, su gran joroba y aquel aire
chusco, cascarrón y altanero, que le hacía tan temible. General más
valiente no le han conocido los siglos. Los historiadores, que todo lo
enredan, han dado en decir que D. Francisco Eguía no hizo más que
desaciertos y majaderías, cuando mandó el ejército del Centro en la
Mancha, antes de la batalla de Ocaña; pero aún falta probar, que nuestro
general no fue un Gran Federico en aquella campaña. Han dicho que no
quería combatir; que apremiado por la Regencia para que atacase a los
franceses, contestó que \emph{él sólo anhelaba sucesos grandes que
salvaran a la nación}, dando a entender el noble deseo de no gastar su
ingenio estratégico en batallejas de tres por un cuarto.

Pero sea de esto lo que quiera, y aun considerando que la Regencia tuvo
razón al separarle del mando en 1809, no se le puede negar su heroísmo
militar y ciencia en 1814. Como que él solo, ayudado de una división del
ejército del Centro, dio al traste con la inmensa balumba de las Cortes,
poniendo en vergonzosa fuga a más de cien diputados liberales, que se
escondieron en sus casas sin atreverse a asomar las narices\ldots{} ¿Qué
tal? Hombres como aquel bravísimo Eguía, son el mayor galardón que Dios
Omnipotente puede hacer a las atribuladas y huérfanas naciones.
Admirablemente lo hizo, y allí era de ver cómo se presentó con su tropa
en casa del Presidente de las Cortes, notificándole, con serenidad
sublime, la ruina de la Constitución, y cómo ocupó después resueltamente
y sin asomos de miedo, casi sin pestañear, el palacio de las Sesiones,
declarando con voz entera y firme que todo estaba por los suelos.

¡Qué noche la del 10 de Mayo de 1814! ¡Oh sin igual ventura! ¡Oh
inolvidable regocijo del alma después de tan larga opresión! Yo había
pasado todo el día escribiendo un articulito que remití a \emph{La
Atalaya}, por encargo de mi excelente patrono. Estoy tan orgulloso de
aquella pieza, fruto precioso del frenético entusiasmo mío y de los
ardores fernandistas de mi exaltado corazón, que no quiero que estas
fieles memorias vayan a los confines de la posteridad, sin llevar
siquiera un par de párrafos para que, reconociendo mi patriotismo, se
juzgue de mi caliente estilo y de las gallardías de mi pluma. Decía así:

«¡A dónde estáis, potencias de mi alma! ¡Os busco, y por ninguna parte
os encuentro! ¿Habéis volado en busca de aquel imán de nuestros
corazones? ¿A dónde está {\textsc{Fernando}}? Hechizo de mi corazón, ¿a
dónde te encontraré? ¡Mi alma no {\textsc{}} acierta en la efusión de su
placer a expresar de ningún modo los sentimientos de que se halla
inundada! ¡Mi memoria\ldots{} mi voluntad\ldots{} mi entendimiento,
sí!\ldots{} Todo es vuestro, ¡Dios Eterno! Pero si {\textsc{Fernando}}
está en vos y vos en {\textsc{Fernando}}, en vos mismo gozaré de su
amorosa presencia; sí, Dios Omnipotente, permitid que me regocije en
vos, pues que vos le elegisteis desde vuestros eternos alcázares para
nuestro digno {\textsc{Rey}}; vos le perseverasteis con vuestra
providencia en el principio; vos le guardasteis bajo la sombra de
vuestras divinas alas\ldots; vos le quitasteis de un suelo manchado con
tantos crímenes, para que no presenciase el espantoso castigo con que
ibais, aunque tan lleno de misericordia, a castigar a tus hijos\ldots{}
sí, amado {\textsc{Fernando}}\ldots{} sí, apetecido consuelo de todas
nuestras aflicciones\ldots{} sí, hermoso y deseado iris en todas
nuestras horribles borrascas\ldots{} tus fieles y huérfanos hijos te
lloraron como miserables pupilos, y no hubo un placer verdadero en sus
amantes corazones, considerándote cautivo\ldots»

\hypertarget{ii}{%
\chapter{II}\label{ii}}

Y así seguía, soltando la abundosa vena de mi inspiración, para que sin
tasa corriese, con lo cual se embobaba el vulgo, llegando mi fama como
escritor hasta el punto de que un padre de la Merced, el venerable
Salmón, dijese de mí que allá me iba con Cervantes en el manejo de la
pluma. Pero la verdad es que mi genio me llamaba por caminos distintos
de los de la literatura. ¿Se creerá que en aquella felicísima noche del
10 de Mayo, no pudiendo contener mi exaltación en pro de Fernando, ni
menos mi enojo contra los llamados \emph{mamones}, me uní a los esbirros
y jueces que iban de calle en calle prendiendo en sus casas a los
famosos corifeos de las Cortes?

Uno de los jueces de policía era amigo mío, y también un oficial de los
que mandaban la tropa encargada de proteger a los jueces. Fui, pues, de
casa en casa, y no puedo dar idea de la indignación que ardía en mi alma
contra aquellos bribones, a quienes era preciso buscar dentro de sus
propias guaridas para prenderlos. Era en realidad vergonzoso que varones
tan eminentes como aquellos intachables jueces de policía, anduviesen
cual cuadrilleros de la Santa Hermandad, corriendo a caza de un
Argüelles, de un Martínez de la Rosa, de un Calatrava\ldots{} ¡Tunantes!
¡Cuándo recibieron ellos mayor honra que la de ser huroneados por
individuos de toga, los cuales en su desmedido ardor por la causa del
Rey, iban sudando gotas como puños; que tales angustias trae el oficio
de polizonte!

La pesquería no fue mala, y si bien se nos escaparon Toreno, Antillón,
Gallego y otros, cogimos a Argüelles (a quien no le valió su
\emph{divinidad}) en la calle de la Reina; a Gallardo, en la del
Príncipe; a Canga Argüelles, en la misma calle y casa de San Ignacio; a
Page, en la de Hita; a Cepero y a Martínez de la Rosa, en la calle de
San José; a Larrazábal, en la de Jacometrezo; a García Herreros, en la
plazuela de Celenque, y en diversos sitios que no recuerdo, a Quintana
el Seminarista, a Feliú, Villanueva, Muñoz Torrero, Cano Manuel, Álvarez
Guerra, O-Donojú, Capaz, Cuartero, a los cómicos Máiquez y Bernardo Gil,
sin omitir al célebre \emph{cojo de Málaga}.

¡Oh, vil caterva de charlatanes! ¡Y qué bien os llegó vuestro San
Martín! ¡Y con qué oportunidad y destreza fueron burladas vuestras malas
artes y destruidos vuestros execrables planes! Mala peste os consuma, y
demos gracias a Dios que nos deparó el remedio contra vuestra perfidia
en la férrea mano de Eguía. Ni qué falta hacían en el mundo vuestros
heréticos discursos, ni a cuenta de qué venía esa endiablada
Constitución\ldots{} ¡Ay! Aquella noche las almas se desbordaban de
gozo, viendo destruida la infame facción, muerta la herejía, enaltecido
el sacrosanto culto, restaurado el trono, confundidos volterianos y
masones. Yo no cesaba de dar gracias a Dios por lo bien que conducía
desde su celeste altura la empresa, y siempre que salíamos de una
madriguera para entrar en otra, asegurado ya uno de los abominables
delincuentes, me santiguaba devotísimamente, poniendo los ojos en el
cielo, para que ni por un instante nos desamparase la bondad divina en
tal trance, y llegáramos al fin de la jornada sin tropiezo alguno.

A medida que iban cayendo los llevábamos a la cárcel de la Corona y al
cuartel de Guardias de Corps o a San Martín, donde quedaban encerrados.
No se les dejó papel que no se guardase para dar luz sobre los procesos
que se les iban a formar, porque habría sido en verdad lastimoso que las
picardías de tanto malsín no tuviesen comprobación cumplida en los
autos, para que a nadie quedase duda de sus maldades. Pues digo\ldots{}
si no se hubiera tenido mucho cuidado de cogerles los papeles, la
justicia habría tenido que romperse los cascos para inventarlos después,
lo cual es tarea larga y que da larga fatiga y quita mucho tiempo a los
señores de la Comisión de Estado.

Siempre me acordaré de la insolencia de los diputadillos, que en vez de
echarse a llorar y pedirnos perdón cuando les prendíamos, nos miraban
con altaneros ojos, afectando una serenidad tranquila, propia de justos
o inocentes, y expresándose en tales términos, que al oírles, ¡mal
pecado!, parecía que no habían roto plato ni escudilla. Quien les viera,
creyéralos a ellos jueces y a nosotros ladrones en cuadrilla, trocados
los papeles, y convertidos los ajusticiadores en ajusticiados. Viendo
tan descarada desvergüenza, no me pude contener, y a varios de ellos les
dije cuatros frescas bien dichas y dos docenas de verdades como puños,
siendo tal su cobardía, que no se atrevieron a contestarme, ni aun
siquiera a soportar el mortífero rayo de mis ojos.

Yo les veía pasar de sus casas a las cárceles, y siempre me parecían
pocos. Hubiera deseado que aquellos bergantes se multiplicaran para que
fuese más grande el esplendor de la hazaña que estábamos consumando.
¡Oh!, ver a Madrid limpio de liberales, de gaceteros, de discursistas,
de preopinantes, de soberanistas, de republicanos, de volterianos, de
masones\ldots{} ¡Esto era para enloquecer al menos entusiasta!

Llegaste al fin, ¡oh día 11 de Mayo, y tus primeras luces vieron al
devoto pueblo de Madrid corriendo por las calles como impetuoso río, sin
que ningún dique bastase a contener las desbordadas olas de su gozo! ¡
Oh, qué pueblo! ¡Y cómo gritaba celebrando el acabamiento de la tiranía!
¡Y con cuánto amor invocaba al Dios Todopoderoso y a su Santísima Madre,
llevando en triunfo a los benditos frailes y arrastrando por las
enlodadas calles las sacrílegas imágenes de la libertad, que exornaban
el palacio del charlatanismo; arrancando la lápida de la Constitución y
cuantos letreros y signos y figuras, recordasen la conjurada
borrasca!\ldots{} De seguro lo pasaran mal los señores encarcelados, si
por acaso les echara la zarpa el discreto y sapientísimo vulgo. Hubo
quien a grito herido pidió que se permitiera al pueblo hacer justicia
por sí mismo en la ruin persona de los orgullosos caídos, pero la cosa
no pasó de aquí.

Por mi parte trabajé en aquel día más que en otro alguno de mi vida.
¡Virgen de las Angustias! ¡Qué idas y venidas, qué mareo, qué
ansiedad!\ldots{} Sólo por causa tan santa y por el inextinguible amor
del inocente Fernando, puede un hombre molerse y descoyuntarse como yo
lo hice aquel día, con los hígados en la boca durante diez horas, sin
dar paz a los pies ni a la lengua, ora arengando a estos, ora
recomendando a los otros lo que habían de hacer, disponiendo y
ordenando, conforme a la voluntad de mi patrono y de otros personajes de
viso que andaban en el negocio.

¡Jesús, María y José! Flojita era la tarea en gracia de Dios\ldots{} Al
más pintado se la doy yo, seguro de que a la mitad de la jornada
desfallecería, como no recibiera del cielo broncíneas piernas y garganta
de acero. Ahí es nada\ldots{} era preciso ir repartiendo dinero por los
barrios bajos y convocar a determinados individuos de la majería,
cuidando de andar con mucho pulso en lo del distribuir, porque a mucho
que se abriera la mano, no quedaba nada para el repuesto del
comisionado. Asimismo era indispensable ir de taberna en taberna y de
garito en garito, contratando gente; avistarse con el tío Mano de
Mortero, con Majoma y otros próceres del Rastro, para encomendarles
delicadas comisiones, de esas que sólo a delicadísimos entendimientos
pueden fiarse. También había que avisar a los padres franciscanos y
agustinos, que estaban ocultos, para que saliesen a arengar a la
muchedumbre; hacer correr noticias falsas de conspiraciones fraguadas
por los revolucionarios; con otros muchos menesteres y ocupaciones que
habrían rendido el organismo más fuerte y desquiciado el más sólido
entendimiento y la más firme voluntad. Pero ¿de qué sirve la fe, si no
es para hacer prodigios? Por la fe los hice yo en aquel memorable día;
por la fe tuve cuerpo y alma y sentidos e ideas para tantas cosas; por
la fe hice más yo solo que veinte compañeros encargados de iguales
trapisondas.

Recordando aquel día y mi cansancio, el alma se me inunda de frenético
gozo. Habíamos vencido a la infame pandilla, a un centenar de
deslenguados charlatanes; les habíamos vencido sin más auxilio que un
ejército y la autoridad del Rey, acompañado de la grandeza, del clero,
de las clases poderosas; habíamos triunfado en sin igual victoria, y la
monarquía absoluta, tal como la gozaron con pletórica felicidad nuestros
bienaventurados padres, estaba restablecida; habíamos pisoteado la hidra
asquerosa del democratismo extranjero, de la inmunda filosofía,
devolviendo al trono su esplendor primero y a la autoridad real el
emblema de su origen divino; habíamos derrotado a la impiedad, sacando a
la religión sacrosanta de la sombra y abatimiento en que yacía; habíamos
realizado una maravilla; habíamos sido los soldados de Cristo; sentíamos
en nuestro pecho el aliento divino, y el regocijo de la bienaventuranza
enardecía nuestras almas.

«¡Noche del 10 de Mayo!---decía el padre Castro en su inolvidable
Atalaya.---¡Ah, tú serás contada entre los días más solemnes que vio el
mundo!\ldots{} Españoles, alabemos y ensalcemos al Señor: que nuestra
lengua no cese de cantar sus misericordias.

»Sí, españoles: \emph{Confitemini Domino quoniam bonus, quoniam in
sæculum misericordia ejus}. Los principales cabezas de esta rebelión
están ya presos en la capital y en las provincias. La sabiduría de
nuestro idolatrado {\textsc{Fernando}}». ha sabido combinar de tal modo
los caminos de nuestra futura dicha, que es menester confesar que el
Señor está en él. En un mismo día y en una misma hora han sido
sorprehendidos todos estos verdugos de nuestra patria, y su exemplar
castigo será la garantía más segura de nuestra perpetua felicidad.
\emph{Confitemini Domino, quoniam bonus, quoniam in sæculum misericordia
ejus}. Españoles, alabad y bendecid al Señor. Nuestra patria es ya
feliz: ya reina {\textsc{Fernando}}».

¡Sí, ya reinan Dios y Fernando!

\hypertarget{iii}{%
\chapter{III}\label{iii}}

¡Alabado sea el Santísimo Sacramento del Altar!\ldots{} Señor, ¿con qué
lengua cantaré tus alabanzas? ¿Qué palabras hay que no sean pálidas y
frías para expresar mi gratitud? En la humildad nací, y del muladar de
mi oscura condición sacome tu mano poderosa para llevarme a los dorados
alcázares, donde las grandezas humanas dan idea de las grandezas
divinas. Mi corazón se estremece de gozo al recordar mi primer paso por
la dorada senda.

Era un domingo; habían pasado algunos días después de la entrada del
Rey; funcionaba ya el nuevo ministerio; habían levantado su majestuosa
cabeza, coronada con los laureles de cien siglos, el Real Consejo y
Cámara de Castilla y la Sala de Alcaldes, cuando D. Buenaventura (algún
nombre he de dar a mi protector para que se le distinga entre los
individuos de que haré mención), me llamó a su despacho, y melifluamente
me habló así:

---Dime, Braguitas, en cuál oficina quieres colocarte, pues ya he dado
tu nombre al ministro, y no falta más que saber tu deseo para
satisfacerle al punto.

---Señor---repuse,---como vayan por delante los veinte mil reales que
Vuecencia me ha prometido, lo demás es cuestión secundaria. Sin embargo,
mis aficiones\ldots{}

---Ya sé que tú te inclinas a la Real Hacienda. Vas a lo positivo. ¿Te
convendría la Caja de Amortización, los Pósitos, la Revisión de
juros?\ldots{}

---Iré, si Vuecencia no lo toma a mal, a Paja y Utensilios.

---Corriente\ldots{} Mañana mismo tendrás tu nombramiento\ldots{} Dime,
¿has llevado la carta a las monjas Bernardas?

---Desde esta mañana.

---¿Me has limpiado las botas?

---Están como espejos.

---Bueno: antes de marcharte, pídele a doña Nicanora los calzones y la
casaca que te prometí ayer. Con un poco de obra quedarán ambas prendas
como nuevas\ldots{} Ahora necesitas cierta ostentación, Juan: es preciso
que te presentes como corresponde a un señor oficial segundo de Paja y
Utensilios, y lo primero que has de hacer es dar las gracias al señor
Ministro\ldots{}

---¿Las gracias?

---Seguramente. Ganabas 5.000 rs. en las covachuelas de la secretaría de
Gracia y Justicia, y de golpe y porrazo pasas con 20.000 a Paja y
Utensilios\ldots{}

Mortificado por mi dignidad, un poco ofendida, permanecí en silencio;
pero el insigne repúblico debió de adivinar mis pensamientos con su
seguro tino, y me dijo:

---¿Qué, no estás contento todavía? No sé en qué piensan los muchachos
del día\ldots{} Ya se ve\ldots{} los tiempos que corren y los escándalos
de estos últimos años han despertado las ambiciones de tal modo\ldots{}
En mis tiempos, lo que hoy se te da equivalía a un arzobispado de los de
mejor renta.

---No me quejaré---repuse humildemente,---porque es propio de mi
condición no pedir nada y aceptar lo que me dan; pero\ldots{} si han de
acomodarse las recompensas a los merecimientos\ldots{}

---¡Tus merecimientos!---exclamó su señoría con desdén.---¿Cuáles son?
¿Qué letras has cursado, perillán? ¿Qué tratados de materia jurídica o
teológica has escrito? ¿Qué servicios has prestado a la administración,
bergante? ¿Qué ejércitos acaudillaste, zopenco, ni qué Rey te debió la
corona?

---Sobre eso hay mucho que hablar, señor D. Buenaventura de mi
alma---respondí con brío.---Si a todos se repartiera por igual no me
quejaría; pero se están viendo improvisaciones escandalosas. Ahí tiene
Vd. a Antonio Moreno. ¿Qué era hace un mes?, ayuda de peluquero, pues ni
siquiera podía llamarse maestro peluquero. ¿Qué es hoy?\ldots{}
consejero de Hacienda.

D. Buenaventura calló. Le dejé suspenso y absorto.

---Es verdad---dijo al fin.---Ya lo sabía\ldots{} pero eso no tiene nada
de particular. Antonio Moreno era\ldots{} un excelente profesor de
cabezas\ldots{} No debe olvidarse que en Valencia sirvió de amanuense
cuando se redactó el célebre decreto del 4.

---¡Consejero de Hacienda!---exclamé yo alzando los brazos.---¡Consejero
de Hacienda un vil peluquero!

---Pero a nosotros ¿qué nos importa? Allá se las compongan\ldots{} Dime
tú, ¿qué pedazo de pan nos quitan de la boca haciendo a Moreno
consejero? Además, el honor de haber redactado tan sublime documento,
merece perpetuarse con una posición decente\ldots{} ¿Qué piensas? ¿Qué
opinas? ¿Por qué has hecho ese gesto de monja escandalizada cuando he
nombrado el decreto del 4 de Mayo? ¿No te gusta? ¿No te parece
categórico? ¿No lo crees una obra admirable y que nada deja que desear?

Yo callaba, porque mil dudas y desconfianzas ocupaban mi espíritu.

---No puede escribirse nada más contundente---continuó D. Buenaventura
leyendo un papel,---que el párrafo en el cual se declara «aquella
Constitución y decretos nulos y de ningún valor ni efecto, ahora ni en
tiempo alguno, como si no hubiesen pasado jamás tales actos, y se
\emph{quitaran de en medio del tiempo}\ldots» Está dicho todo, y con
tales palabras bastaba.

---Esa es mi opinión. Con eso bastaba. Pero más arriba, el Rey,
obedeciendo a pérfidas inspiraciones, ha dicho que aborrece el
despotismo, que convocará Cortes, que establecerá la seguridad
individual, con otras zarandajas que o mucho me engaño, o son el primer
paso para volver a las andadas, mi Sr.~D. Buenaventura.

---Pero ven acá, majadero impenitente, ¿cuándo has visto que tales
fórmulas sean otra cosa que una satisfacción dada a esas entrometidas
naciones de Europa que quieren ver las cosas de España marchando al
compás y medida de lo que pasa más allá de los Pirineos? Ríete de
fórmulas. No se pueden hacer, ni menos decir las cosas tan en crudo que
los afeminados cortesanos de Francia, Inglaterra y Prusia se
escandalicen. ¡Reunir Cortes! Primero se hundirá el cielo que verse tal
plaga en España, mientras alumbre el sol\ldots{} ¡Seguridad individual!
¡Bonito andaría el reino, si se diesen leyes para que los vasallos
obraran libremente dentro de ellas, y se dictaran reglas para enjuiciar,
y se concedieran garantías a la acción de gente tan ingobernable,
díscola y revoltosa! El Rey, sus ministros y esos sapientísimos y útiles
Consejos y Salas, sin cuyo dictamen no saben los españoles dónde tienen
el brazo derecho, bastan para consolidar el más admirable gobierno que
han visto humanos ojos. Así es y así seguirá por los siglos de los
siglos\ldots{} ¿Eres tan tonto, que crees en manifiestos de reyes? Como
los de los revolucionarios, dicen lo que no se ha de cumplir y lo que
exigen las circunstancias. Bajo las fugaces palabras están las inmóviles
ideas, como bajo las vagas nubes las montañas ingentes, que no dan un
paso adelante ni atrás. Las nubes pasan y los montes se quedan como
estaban. Así es el absolutismo, hijo mío; sus palabras podrán ser
bonitas, rosadas, luminosas y movibles; pero sus ideas son fijas,
inmutables, pesadas. No mires lo de fuera sino lo de dentro. Estudia el
corazón de los hombres y no atiendas a lo que articulan los labios, que
siempre han de pagar tributo a las conveniencias, a la moda, a las
preocupaciones\ldots{}

D. Buenaventura se expresaba con calor. No me atreví a contestarle, y
mis pensamientos se acomodaron a los suyos, como sucedía casi siempre
que hablábamos de política.

---¡Ah!, se me olvidaba una cosa---exclamó después de breve pausa:---ya
he dicho al Ministro que te exima durante algunos días de ir a la
oficina. Es preciso que me ayudes en este delicado negocio que tengo
entre manos\ldots{} Ya sabes que Su Majestad me ha nombrado fiscal de la
comisión de Estado que ha de sentenciar a los presos de la noche del 10.

---Tarea fácil, a mi modo de ver, mientras no desaparezcan del mapa
Melilla, Ceuta y el Peñón.

---Eres excesivamente ejecutivo. No puede hacerse la distribución, sin
fundar en algo los castigos. Es preciso buscarle el pelo al huevo, como
suele decirse, registrar papeles, sacar de ellos la quinta esencia de la
maldad, llegar testigos aunque sea en las entrañas de la tierra,
estrujar los autos hasta que destilen la amarga hiel de la evidencia,
cumplir en todas sus partes la larga serie de procedimientos que son
gloria de nuestra jurisprudencia, y en fin, \emph{hacer} los procesos de
tal modo que no les falte ni una tilde y aparezcan en toda su horrible
desnudez las necesarias maldades de esos hombres.

---Con el plan de república (algo más verosímil que el de la Iberiana),
revelado por el padre Castro en su \emph{Atalaya}---repuse,---bastará
para \emph{hacer} las más lindas causas que se han visto en tribunales
españoles.

---A eso vamos. La \emph{Confederación} descubierta por el Atalayero es
ingeniosa. Además, algunos testigos han hecho declaraciones de perlas.

---El conde del Montijo\ldots{}

---Asegura que los liberales formaron causa al Rey en un café de Cádiz y
le condenaron a muerte.

---Ostolaza\ldots{}

---Ha delatado los \emph{pensamientos} de sus compañeros de Cortes,
asegurando que querían deshonrar al Rey, con otras preciosísimas
afirmaciones que constituyen un verdadero tesoro.

---La persecución del Obispo de Orense y del marqués del Palacio, así
como el destierro del Nuncio Sr.~Gravina, son materia abundante.

---Abundantísima.

---Bien sabemos todos que Mejía dijo en las Cortes \emph{que no existe
Dios}; Argüelles, \emph{que no debían obedecerse los preceptos de la
Iglesia}.

---Feliú dijo, \emph{que la religión era una farsa.}..

---Y Arispe afirmó, que la grandeza española \emph{tenía sangre de
perro}. Bien mirado, el testigo más explícito, más claro, es el archivo
y las actas de las Cortes.

---Sin duda. ¿No está allí escrito que el danzante de Martínez de la
Rosa propuso fuera condenado a muerte el que propusiese adición o
reforma en la Constitución de Cádiz?

---Recuerdo perfectamente su pedantesco discurso del 21 de Abril, en que
decía \emph{que los pueblos deben darse ellos mismos las leyes
fundamentales}.

---También yo tengo buena memoria---añadió D. Buenaventura.---Habló
mucho de \emph{derechos imprescriptibles}, y concluyó así: \emph{Se
acabaron nuestras desgracias}. \emph{Ya reinan las leyes.}..

---Que es como decir \emph{que no reinará el Rey},---afirmé, tomando un
polvo que D. Buenaventura me ofreció.

---¡Y qué más, mi querido Bragas! ¿No consta en el libro de las sesiones
la abominable expresión de Canga Argüelles?

---\emph{Que estaba pronto a derramar la última gota de su sangre en
defensa de la Constitución}.

---Así mismo lo dijo.

---No recuerdo bien cuál de ellos aseguró que \emph{destruidos los
conventos, se cortan las fuentes que mantienen las preocupaciones y
cuentos de viejas}.

---Page, el mismo que expresó la opinión de que es \emph{delito de lesa
majestad llamar} {\textsc{soberano}} \emph{al Rey}\ldots{} ¿No fue
Istúriz quien dijo aquellas palabrotas?\ldots{}

---Sí, ya recuerdo. \emph{Hoy somos ciudadanos de una gran república,
aunque bajo las formas características de la monarquía; el Rey no es
nuestro señor, es nuestro jefe, porque queremos y de la manera que
queremos que lo sea, y nada más}.

---Admirable memoria tienes---dijo D. Buenaventura, tomando la
pluma.---Voy a apuntar eso. Se confrontarán las \emph{Sesiones}.

---No olvidará Vd. los méritos y servicios de Gallardo. Fue el que
estampó en letras de molde, \emph{que los obispos debían echar
bendiciones con los pies, colgados de una cuerda}. Ahora recuerdo
también que Ramajo, redactor de \emph{El Conciso}, amenazó al Rey con la
venida de Carlos IV, si no juraba la Constitución.

---Deliciosísimo, amigo Bragas. Tras los diccionaristas y gaceteros,
viene la pestilente chusma de poetas, a quienes es preciso también poner
como nuevos. Ahí tienes por ejemplo, a Sánchez Barbero\ldots{}

---El autor de aquellos versitos:

\small
\newlength\mlena
\settowidth\mlena{De independencia y libertad gozamos,}
\begin{center}
\parbox{\mlena}{Aquí nosotros los sagrados dones                     \\
                De independencia y libertad gozamos,                 \\
                Y monarca, no déspota, juramos.}                     \\
\end{center}
\normalsize

---Yo también me acuerdo, yo también---exclamó con júbilo mi amigo.---El
infame bibliotecario de San Isidro se despachó a su gusto en estas
endechas:

\small
\newlength\mlenb
\settowidth\mlenb{Doquiera ya: *Constitución inflama*...}
\begin{center}
\parbox{\mlenb}{El fanático error vencido cede,                      \\
                Y la sin par \textit{Constitución} sucede;           \\
                \textit{Constitución} resuena                        \\
                Doquiera ya: \textit{Constitución} inflama...}       \\
\end{center}
\normalsize

¡Ya te inflamarán a ti!\ldots{} ¡Miserables poetas, se os ha acabado el
\emph{doquiera!} Encerraditos en Melilla, podréis cantar la
\emph{soberana}.

---Muñoz Torrero---añadí, gozoso de poner mi retentiva al servicio del
Estado,---fue el que dijo que la \emph{soberanía de la nación estaba en
las Cortes}, lo cual es como poner a la burra las arracadas.

---Justamente. \emph{Y que las personas de los diputados eran
inviolables}. ¡Inviolables el veneno de la serpiente y la lengua del
escorpión!

---Pues ¿y García Herreros? Fue el que tuvo el atrevimiento de asentar
que los \emph{Reyes están sujetos a las leyes que les dicta la nación}.

---Y \emph{que la ley es superior al Rey,} lo cual es como decir que la
espuela gobierna al jinete.

---Casi todos ellos firmaron el decreto de 2 de Febrero, en el cual se
dijo que \emph{no se conocería por libre al Rey, ni menos se le
prestaría obediencia, hasta que él prestase juramento a la
Constitución.}

---Gutiérrez de Terán firmó como secretario el manifiesto de 19 de
Febrero, que era la segunda parte del tal decreto.

---Y Martínez de la Rosa, o sea el \emph{Sr.~Bello Rosal}, como le llama
\emph{La Abeja}, lo escribió.

---Y Feliú lo leía a voz en cuello en los cafés.

---Adonde iban a emborracharse.

\noindent{\dotfill}

D. Buenaventura tomaba apuntes, demostrando a cada nueva adquisición
cierta alegría pueril. Como hombre que en el cumplimiento de sus deberes
y en el servicio del Rey y del Estado ponía su alma toda entera, sin
proceder jamás de ligero en ningún asunto grave, allegaba cuantos datos
pudieran ilustrar su entendimiento en materia tan ardua, y con ansiedad
de avariento los iba guardando. El buen señor se veía precisado a
sentenciar a muerte o a presidio a unos cuantos malvados, y no pudiendo
hacerse esto rectamente sin pruebas, las buscaba para que aquellos
infelices no fueran al patíbulo sin saber por qué. ¡Tunantes! ¡Cuándo
merecieron ellos tropezar con varón tan justo, tan humanitario y
compasivo como aquel! ¡Ni cómo habían ellos de soñar que, merced a los
cristianos sentimientos de tan ejemplar magistrado, enemigo del
derramamiento de sangre, se verían galardonados, como quien dice, con
unos cuantos años de presidio, en vez de la horca que merecían!

Más adelante se sabrá su destino; que ahora no puedo levantar mano del
trabajo de mi propia historia, en la cual ocupan lugar muy preferente
los sucesos que se verán a continuación.

\hypertarget{iv}{%
\chapter{IV}\label{iv}}

Siempre fui hombre que lo mismo servía para un fregado que para un
barrido, y de tanta actividad, que solapadamente me multiplicaba,
esclavo de diversas y contrapuestas obligaciones, atento siempre al
servicio del Estado y a mi propio interés, como Dios manda, vigilante y
despierto en todos los momentos de la vida para que ninguna ocasión de
ganancia se me escapase, y con cien ojos puestos en el panorama de los
acontecimientos para sacar de ellos provecho. Así es que ayudaba a D.
Buenaventura en sus quebraderos de cabeza dentro de la comisión de
Estado, y servía mi plaza en Paja y Utensilios, mereciendo plácemes
sinceros del jefe, y no poca envidia de mis compañeros. En poco tiempo
supe conquistar la amistad de muchos personajes eminentes de aquella era
feliz, tal como D. Blas Ostolaza, espejo de los predicadores, confesor
del infante D. Carlos y hombre de muchísimo influjo, don Pedro Ceballos,
D. Juan Lozano de Torres, D. Juan Pérez Villamil, célebre por lo de
Móstoles, D. Pedro Labrador, el incomparable diplomático que en el
Consejo de Viena dejó pasmados a todos los embajadores de las grandes
potencias, D. Miguel de Lardizábal, ministro de Indias, el gran
magistrado D. Ignacio Villela, el Sr.~Vadillo, alcalde de Casa y Corte,
y otros muchos individuos tan insignes, tan eminentes, que bien podía
decirse de ellos que tenían las cabezas podridas de talento.

Como yo era tan entrometido, fácilmente ensanchaba el círculo de mis
amistades, unas veces solicitando favores con tal empeño, que me los
concedían porque me quitase de encima, otras prestando los pequeños
servicios que de mi reducido poder dependían\ldots{} Pues digo\ldots{}
cuando alguno de aquellos señorones venía a mi oficina, a la inmediata
de Rentas decimales (donde yo tenía tantos amigos) o a otra cualquiera
de las del ramo, a solicitar reservadamente que se hiciera perdidizo un
miserable expedientillo de Propios o de Arrendamiento de oficios\ldots{}
vamos\ldots{} aquello era una bendición. Viendo que yo abría la mano y
no me hacía de rogar, siempre que se trataba de poner mi firma en un
\emph{Cargo y Data}, enviado por el alcalde, por el contratista o por el
recaudador, me traían en volandas. ¿Qué le importaba a la nación que se
escurrieran entre los papeles algunos disimulados sapos y culebras, o
que se variara con caligráfica ingeniosidad un par de números, siempre
que quedase contento aquel o el otro empingorotado repúblico, cuyo
bienestar importaba tanto al Estado? ¡Pues no faltaba más, sino que por
no hacer el gusto a un regidor amigo o a un alcabalero pariente, se
sofocara uno de aquellos esclarecidos varones, y revolviéndosele los
humores, perdiera la salud, tan necesaria al buen servicio y esplendor
de la monarquía!

Unas veces era preciso conseguir una moratoria de diez años para que tal
o cual duque no se viese importunado por los estúpidos de sus
acreedores\ldots{} Otras veces había que beber los vientos para
conseguir que el fuero del Honrado Concejo amparase a Fulanito, en cuyo
caso, y mientras aquel decidiera, este no tenía que apurarse por la
fruslería del pago de sus arrendamientos\ldots{} Pues ¿y cuando había
que conseguir de la sala de Alcaldes una provisioncita para que en tal o
cual pueblo se repartieran los oficios dos o tres individuos de una
familia, de modo que por ser hermanos el alcalde, el secretario, el
escribano y el procurador síndico, no había la más mínima disputa en el
arreglo del común?---Existiendo estos asuntillos, era necesario entonces
tener en Madrid un amigo listo y de mucha mano en las oficinas, para que
volviese lo blanco negro y lo verde encarnado en las cuentas, para que
visitase a algún señor del Consejo y con él se entendiese; que si no,
capaz era el tal Consejo de darse de calabazadas por averiguar dónde se
había escurrido algún terreno baldío rematado en tiempo de los
franceses\ldots{}

También solían ocuparme los señores de Madrid y muchos de provincias en
diversos negocios referentes a Tercias Reales, a ciertos atrasillos de
Alcabalas, a compaginar las cuentas del receptor de bulas de tal pueblo
para que no apareciesen distintas de las del alcalde, a resucitar cual
expediente de Manda Pía forzosa, añadiéndole un par de planas a la
antigua, tan diestramente imitadas que ni aun les faltaba la
polilla\ldots{} ¿y para qué cansar más?\ldots{} ocupábanme en todo lo
que fuese del mangoneo subterráneo de las oficinas, pues yo, por mi
índole rebuscona, mi carácter dulce y la prodigiosa facultad de
insinuación que me otorgó Natura, había establecido una red oculta, una
multitud de hilos de connivencia tendidos de covachuela en covachuela y
de despacho en despacho, con tal arte que nada me era difícil.

Verdad es que algunos envidiosos dieron en decir que se deshonraban
teniéndome a su lado, y hasta se susurró que Su Excelencia quería
echarme a la calle\ldots{} (ya se hubiera tentado la ropa antes de
hacerlo); pero yo tenía muy buenos asideros en la administración y de
todo me burlaba. Antes hubieran movido de sus graníticos cimientos el
Escorial que moverme a mí de mi silla en Paja y Utensilios. Como que mis
calumniadores eran unos pobres papanatas que a penas sabían hacer otra
cosa que el trabajo material de su oficina, y así era de ver el mal
trato de sus casas, pues muchos de ellos no tenían camisa que poner a
sus chiquillos. En cuanto al aspecto de sus rostros y personas, daba
grima verles, según estaban de rotos, descomidos y trasijados, y no
podía uno menos de avergonzarse al pensar qué idea formarían de la
administración española los extranjeros que acertaran a conocerles.

Mi casa, por el contrario, era una tierra de promisión. ¡Bendito sea
Dios que a nadie desampara! Tan pronto venía la caja de dulce como la
tarea de chocolate macho, ora las sartas de chorizos, ora un par de
jamones: el plato de leche no faltaba nunca en las solemnidades, ni el
par de capones en 24 de Julio\ldots{} en fin, aquello parecía una
colmena. Tanto iba creciendo mi clientela y buena suerte, que me ocurrió
poner una agencia de negocios. Había que ver cómo me solicitaban damas,
oficiales, canónigos, marquesitos, ¿qué digo?\ldots{} ¡hasta un señor
obispo me honró con su confianza! Mi nombre fue bien pronto conocido en
todo Madrid, quizás en todo el reino y sus Indias; transformose mi
persona; me sentí crecer, ¡oh!, crecer hasta sobresalir por encima de
las eminencias cortesanas; vi bajo mis pies a muchos de carroza y
venera, miré cara a cara el sol de la grandeza y del poder, y la
ambición empezó a morderme las entrañas, ¡pero qué ambición y qué
entrañas las mías!

Entre tanto, mi D. Buenaventura seguía enredado con los procesos, sin
acertar a despacharlos. Las causas eran un embrollo estúpido, y en ellas
no constaba nada positivo ni terminante, por lo cual los tontainas de la
comisión de Estado no acertaban a condenar a muerte a ningún
diputadillo. Lleno de ansiedad el Rey porque se hiciera pronta justicia,
nombró una segunda comisión de Estado, y como esta se atascara también,
fue preciso designar la tercera, hasta que el gobierno se cansó de
comisiones que nada hacían, y supo dictar por sí aquella saludable
medida que cortó de plano la cuestión. Hízolo, si se quiere, por
humanidad, pues a los infelices diputados que se estaban pudriendo en
las fétidas mazmorras de Madrid, les venía bien tomar los salutíferos
aires de Melilla y el Peñón por ocho o diez años.

Y no se crea que un Rey tan recto y tan celoso por el buen gobierno, se
dormía en las pajas. Él mismo extendió de su real puño una orden,
disponiendo que el Sr.~Argüelles no se moviese de Ceuta, durante ocho
años, sin duda porque así convenía a la quebrantada salud del Divino
asturiano.

Este decreto contra los diputados y el que en 30 de Mayo de 1814 se dio
contra los afrancesados que estaban en la emigración, además de sus
ventajas como contra-veneno del constitucionalismo, ofreció el
inestimable beneficio de librarnos de toda la plaga de literatos, poetas
y prosadores, que desde años atrás habían empezado a infestar al
país.---Pues no sé\ldots{} ¡si no andan listos nuestros gobernantes,
buenas se hubieran puesto las cosas! De seguro que Moratín nos habría
aturdido con sus comedias y Meléndez con su pastoril caramillo, y
Gallego con su retumbante trompa. De fijo que Quintana y Sánchez Barbero
y Burgos y Lista y Tapia y Martínez de la Rosa habrían lanzado sobre la
afligida nación un diluvio de obras poéticas de diversos géneros,
teniendo después el descaro de pretender que el público se las pagara en
época de tan poco dinero. También Conde y Toreno nos hubieran mareado
con sus historietas, y Antillón y Ciscar con sus obras científicas,
soliviantando a la nación y metiendo ruido, para que los españoles
despertaran del plácido letargo sabroso en que por fortuna vivían
entonces.

A fin de establecer en todo el país aquella calma perfecta y absoluta,
que es condición precisa para que puedan lucirse los buenos gobernantes,
fue preciso encausar a muchos que no habían sido diputados, ni
literatos, ni siquiera poetas, sino simples particulares oscuros, aunque
cargados de crímenes nefandos. ¡Si era cosa que daba horror oír contar
las maldades de aquella gente!\ldots{} Hubo quien conversando en los
cafés, en círculo de amigos, habló mal del despotismo. Me acuerdo de la
causa formada al brigadier \emph{Moscoso por no haber desplegado los
labios}, mientras otros oficiales elogiaban la Constitución\ldots{}
Vamos, si no se puede uno contener tratando de esto. Bien hizo el fiscal
en pedir para Moscoso la pena de muerte, porque el deber de este era
reprender a los desvergonzados oficiales\ldots{} ¿Pues y los muchos a
quienes se formó sumaria y fueron a Ceuta por haber escrito en los
papeles públicos en tiempo de la Constitución, o por haber sido
partidarios de ella, a pesar de que nunca dijeron «esta boca es
mía»?\ldots{} Nada, nada se les escapaba a aquellos benditos señores de
la comisión de Estado, y de ellos puede decirse que se excedían a sí
mismos y hacían los imposibles por la rápida y eficaz administración de
justicia.

Verdad es que tenían en su auxilio a multitud de patricios vehementes
que delataban sin cesar a los pícaros, refiriendo lo que oyeron tres
años antes y descifrando minuciosa y hábilmente el pensamiento de tal o
cual persona. La delación ¡ay!, no era cosa fácil, sino muy trabajosa y
comprometida, porque había de meterse en las casas fingiéndose amigo,
interceptar cartas en el correo, seducir a los criados, engañar a los
tontos y llevarles a los cafés, excitándoles a hablar; en fin, era obra
difícil, a la cual sólo podían hacer frente la mucha fe y el desmedido
amor al Monarca.

No se crea que este dejó sin premio tan grandes virtudes y la abnegación
de aquellos leales sujetos que olvidaban los menesteres de sus casas
para meterse en las ajenas, no; aquel sabio gobierno premió largamente a
los delatores, dando a unos el privilegio de abastos de tal villa; a
otros una plaza de fiel de matanza; a Fulano una procuraduría; a Zutano
un oficio enajenable, etc., etc.

Lo más notable es que no se vio en aquellos días ninguna ejecución de
pena capital, pues ni el mismo \emph{Cojo de Málaga} llegó a bailar en
la cuerda, como lo tenía dispuesto el gobierno en castigo de haber
alborotado y aplaudido en las tribunas públicas de las Cortes. Delito
tan feo, tan contrario a los fueros de la nación, a la dignidad del Rey
y a la fe católica exigía expiación durísima, y un castigo ejemplar que
sonase en todos los ámbitos de la tierra española. El pueblo estaba
furioso contra el \emph{cojo}, el clero escandalizado, los patricios
muertos de impaciencia porque de una vez y sin pérdida de tiempo
desapareciese de entre los vivos el inmundo reo; pero ved aquí que el
embajador de Inglaterra (son los extranjeros muy amigos de farandulear)
se interpuso, rogó, suspiró, aun dicen que amenazó, hasta que nuestro
Rey, no queriendo malquistarse con la Gran Bretaña por un cojo de más o
de menos, le conmutó la pena capital por la de presidio indefinido. La
suerte fue que cuando llegó la orden, ya estaba Pablo Rodríguez con un
pie en el cadalso y había tragado lo más amargo de la alcuza. Quien más
perdió fue el pueblo, que ya contaba por segura la ejecución y se quedó
a media miel.

Tampoco subió al cadalso doña María Villalba, señora de mucha bondad y
hermosura, según decían. Sí, ¡buena sería ella!\ldots{} ¿Qué puede
pensarse de una dama que cometió la felonía de escribir en confianza a
cierta amiga, contándole algunos lances amorosos del Rey?\ldots{}
Afortunadamente el gobierno de entonces tenía la gracia de que no se
escapaba en correos una pícara carta que contuviese algo
importante\ldots{} ¡Y la doña María se quedaría tan fresca, creyendo que
su gran crimen no iba a ser descubierto! ¡Véase si vale de mucho el ojo
diligente de la administración; véanse las ventajas de una estafeta
celosa del bien público! Los buenos gobiernos han de estar en todo, y
meter la cabeza hasta dentro de las faltriqueras de los gobernados,
porque si no\ldots{} ¡No faltaba más sino que cada uno pudiera escribir
lo que le diese la gana, y después encargar al gobierno la comisión de
llevarlo!\ldots{} En fin, doña María Villalba fue puesta a la sombra, y
si conservó la vida, fue porque se movieron en pro muchas personas de
influencia y todo Madrid se puso sobre un pie.

Pero todo no había de ser blanduras, porque en aquellos días
restablecimos la Inquisición.

\hypertarget{v}{%
\chapter{V}\label{v}}

\emph{Restablecimos}: permitidme que hable en plural. Yo tenía derecho a
ello desde que logré meter mi cucharada en la tertulia del infante D.
Antonio. ¡Quién me había de decir que me vería en tales excelsitudes,
mano a mano con gente nacida de vientre de reinas! Parecíame mentira, y
me causaban admiración mi propia persona, mis propias palabras. Sin
quererlo me hacía cortesías a mí mismo. Aprendí a vestirme con
elegancia, y los que me habían conocido meses antes, se asombraban de mi
transformación.

Antes de dar a conocer la tertulia del infante, enumeraré la serie de
relaciones que me condujeron a palacio.

Desde que comencé a hacerme hombre de pro, solía visitar a las señoras
de Porreño, una de ellas hermana del señor marqués de Porreño, que había
muerto poco antes, hija del mismo la otra, y sobrina la tercera. Aquella
casa, que ya venía muy agrietada desde el siglo anterior, estaba a punto
de hundirse completamente, por cuya razón las tres excelentes señoras
necesitaban buenos amigos que les ayudaran con amena tertulia y delicado
trato a conllevar las pesadumbres de su lamentable decadencia.

En casa de estas señoras conocí a D. Blas Ostolaza, confesor del infante
D. Carlos y predicador de palacio, hombre de los más eminentes que han
vivido en España. Eclesiásticos como aquel debieran nacer aquí todos los
días, y aunque saliera uno detrás de cada piedra, no estaría de más. Él
fue quien felicitó a Fernando desde el púlpito por el restablecimiento
de la Inquisición, diciéndole: «Apenas ha vuelto V. M. de su cautiverio,
y ya se han borrado todos los infortunios de su pueblo. La sabiduría y
el talento han salido a la pública luz del día, y se ven recompensados
con los grandes honores; y la religión sobre todo protegida por V. M.,
ha disipado las tinieblas, como el astro luminoso del día».

Él fue quien escandalizó en las Cortes de Cádiz por su frescura
olímpica, que hacía reír a la gente de las tribunas; y como mi hombre
tanto a los \emph{galerios} como a los diputados les aporreaba a
verdades, cada vez que hablaba todo Cádiz se ponía en movimiento. La
fama de estas hazañas, así como la de sus mortíferos discursos, corrió
por toda España, de tal suerte que cuando Su Majestad volvió de
Valencey, estuvo en un tris que me lo hiciera obispo.

Él fue quien durante las causas de que antes hablé, reveló los
\emph{pensamientos} de sus compañeros de Congreso en las sesiones
secretas. Eso sí, tenía mi D. Blas una memoria asombrosa, y no dijeron
los charlatanes palabrilla pecaminosa ni herética argucia que él no
recordase, por lo cual su boca fue una mina de oro en aquellos benditos
autos.

Era tan celoso por la causa del Rey y del buen régimen de la monarquía,
que si le dejaran ¡Dios poderoso!, habría suprimido por innecesaria la
mitad de los españoles, para que pudiera vivir en paz y disfrutar
mansamente de los bienes del reino la otra mitad. Fue de ver cómo se
puso aquel hombre cuando se restableció la Inquisición. Parecía no caber
en su pellejo de puro gozo. Una sola pena entristecía su alma cristiana,
y era que no le hubieran nombrado Inquisidor general. ¡Oh!, entonces no
se habría dado el escándalo de que se pasearan tranquilamente por Madrid
muchos tunantes que tenían casas atestadas de libros y que recibían
gacetas extranjeras sin que nadie se metiese con ellos.

No sólo era predicador insigne, sino que como escritor religioso bien
puede decirse que Melchor Cano, Sánchez y el padre Rivadeneyra,
comparados con él, ignoraban dónde tenían las narices. ¿A qué rincón de
la Europa culta no llegaron sus célebres novenas, impresas con las armas
reales, amén del retrato del monarca, y en las cuales, ora en prosa ora
en verso, aparecían charlando barba con barba Dios y Fernando VII?
¡Válganme los cielos! Aquello era escribir, y quien no ha visto tales
cosas no sabe lo que es literatura.

En tratándose de púlpito no había otro. Era cosa de estar oyéndole con
la boca abierta, sin perder ni una sílaba de su pasmosa elocuencia. No
le habían de pedir que hablase de los santos ni de religión, que eso era
para predicadorcillos de tumba y hachero. Él, desde que ponía el pie en
la grada, la emprendía con las Cortes, con los diputados, con las ideas
liberales, y mientras más hablaba, aún parecía que se le quedaban dentro
más vituperios que decir. En tocando este punto llevaba hilo de no
acabar en tres días. La gente se aporreaba en las puertas de los templos
para entrar a oírle, y\ldots{} no hay que darle vueltas\ldots{} ¡ni don
Ramón de la Cruz con sus sainetes populares atrajo más gente! ¡Y cómo
entusiasmaba a la multitud! Oíanse gritos dentro de la iglesia, y si al
salir de ella hubieran topado los fieles con algún liberal, ya habría
podido este encomendarse al diablo.

Fue, en verdad, grandísimo error que no le dieran la mitra que pretendió
y por la cual bebió vientos y tempestades en las antecámaras de palacio.
El Sr. Creux, a quien prefirieron, no había revelado tan fielmente como
Ostolaza los pensamientos de sus compañeros los diputados. Pero no era
hombre D. Blas a propósito para quedarse callado ante el desaire, y
volviendo por los fueros de su dignidad ofendida, habló más que siete
procuradores, aderezando su charla con cierta intriga un poco subida de
punto. Pero ni por esas: en vez de hacerle caso, le mortificaron más. No
puede darse mayor injusticia. Llegó la crueldad hasta el extremo de
alejarle de la corte, nombrándole director de la casa de niñas huérfanas
de Murcia. Y lo peor es que no paró aquí la persecución del inimitable
D. Blas, pues ¡mentira parece!, se dijo que su conducta en el referido
colegio no era un modelo de honestidad; y lo aseguraba todo el mundo,
siendo tales y tan feos los casos que se contaban, que parecían pura
verdad. Lo que más me confirmaba a mí, conocedor de nuestra justicia, en
que D. Blas era inocente, fue el ver que le formaron causa. ¡Desgraciado
sujeto! Preso estuvo en la Cartuja de Sevilla, y después confinado a las
Batuecas, consumiéndose de tristeza. ¡Quién se lo había de decir a él y
a todos sus amigos! ¡Triste era, en verdad, considerar incapacitados
aquellos grandes bríos que tenía para todo, oscurecida aquella luminosa
facundia para el púlpito, imposibilitadas aquellas manos de ángel para
enredar los hilos de la conspiración menuda!

De su piedad y devoción, ¿qué puedo decir sino que edificaba a todos, y
especialmente al infante, de quien era director espiritual? Pues ¿a
quién sino a mi amigo debió D. Carlos el haber salido tan temeroso de
Dios, tan fiel esclavo de los preceptos religiosos, que más que príncipe
y futuro candidato al trono parecía un santo, según era de compungido
dentro de la iglesia y ejemplar fuera de ella en todos sus actos y
palabras? Amaba tan entrañablemente D. Carlos a su confesor, que no se
podía pasar sin él. Rezaban juntos por las noches, y cuando el príncipe
se acostaba, Ostolaza, después de decir las últimas oraciones
fervorosamente prosternado ante la imagen de Nuestra Señora, rociaba el
lecho de S. A. con agua bendita para alejar los sueños pecaminosos.

No se crea por esto que mi amigo era gazmoño ni melindroso, que esto
habría sido grave falta en un hombre llamado a las luchas del mundo.
Sabía perfectamente dar a cada hora su propio afán, concediendo parte
del tiempo a las buenas relaciones sociales, porque igualmente se ha de
cumplir con Dios y con los hombres. Por tal ley, Ostolaza, luego que
dejaba a su hijo espiritual dentro de las purificadas sábanas, bien
santiguado y bien rociado por banda y banda, de tal modo que en la
alcoba regia podrían pasear los serafines; luego que D. Blas, repito,
desempeñaba así su difícil cargo, se embozaba en su capa, ya avanzada la
noche, y corría a la calle, apretado por el deseo de compensar los
muchos afanes con un poco de libre holganza. Yo no sé adónde iba, porque
se recataba mucho de los amigos, pero es indudable que no pasaba la
noche al raso, ni buscando yerbas a lo anacoreta, ni mirando al cielo
como astrólogo. Lo de no querer que sus amigos le vieran a tales horas y
el esconderse de ellos, se explica en varón tan meticuloso por su deseo
de apartarse de los peligros que siempre traen consigo las malas
compañías.

Cara redonda y arrebolada, gestos muy vivos y un modo de mirar que daba
a conocer a tiro de ballesta su superioridad; cuerpo sólido; una voz
campanuda y gruesa, como toda voz creada para decir grandes cosas,
formaban el físico de aquel mi nuevo amigo, a quien tanto debí, y a
quien hoy pago un piquillo nada más de la inmensa deuda de gratitud que
con él tengo, sacándole a relucir en estas mis \emph{Memorias}, aunque
su fama no necesita tardías trompetas para sonar por todo el orbe.

¡Ay!, ya no nacen hombres como aquel. No sé qué se ha hecho del jugo
poderoso de esta tierra fecunda. Generación de enanos, mira aquí los
gigantes de que has nacido.

\hypertarget{vi}{%
\chapter{VI}\label{vi}}

Nos tratamos, como he dicho, en casa de las señoras de Porreño. Él había
oído hablar de mí y deseaba conocerme. Pidiome el primer día de nuestro
trato algunos favores y se los hice con el mayor gozo. No era más que
emparedar ciertos expedientes de un hermano suyo, teniente de resguardo,
a quien la Real Hacienda se había empeñado en mortificar impíamente por
unas cuentas\ldots{} ¿Pues no se le había antojado al badulaque del
ministro oprimir y vejar instituciones tan honradas como las tenencias
de resguardo? En fin, todo se arregló a maravilla y se acabaron los
disgustos. Por mi parte nada pedí a D. Blas sino que me tuviera presente
en sus oraciones; pero un día sin previa solicitud, ni esperanza, ni aun
sospecha, encontreme ascendido a una plaza de cuarenta mil reales en
Tercias Reales.

Es que el gobierno buscaba empleados celosos, y cuando alguno llegaba a
hacerse nombre en la administración, no necesitaba empeños. Llegó a mis
oídos que el ministro, al ver mi nombramiento, se puso furioso, diciendo
de mí cuanto la envidia y mala voluntad pueden inspirar a un ministro
regañón, el cual no sólo me puso cual no digan dueñas, sino que se negó
a darme posesión del nuevo destino; pero la orden venía de arriba, es
decir, venía de la cámara real, en forma de minuta extendida por el
ayuda de cámara y firmada por ÉL\ldots{} Don Cristóbal Góngora, ministro
de Hacienda, bajó la cabeza y yo alcé la mía. No está demás decir que un
ministro era entonces un cero a la izquierda, un secretarillo del
despacho, que a veces daba compasión. No servían para maldita la cosa, y
fuera del \emph{coram vobis}, allá se iban con cualquier escribiente.
Todos saben que a un célebre ministro y hombre de Estado y gran
repúblico, le destituyó el Rey entonces \emph{por su cortedad de vista}.

Llevome Ostolaza, como he dicho, a la tertulia del infante D. Antonio,
hijo de Carlos III y famoso por su despedida al Sr.~Gil en 2 de Mayo de
1808.

Aquella epopeya tuvo también su bufonada. El Infante era viejo y no
tenía pretensiones de buen decir, siendo su lenguaje, así como sus
ideas, de hombre campechano y rudo. Hacía gala de ignorancia. Carlos
III, ante quien los ayos de D. Antonio se alzaron en queja, lamentando
la desaplicación del niño, dijo: «\emph{Si el infante no quiere
estudiar, que no estudie}», y el chico lo hizo al pie de la letra.
Cuando fue grande se dedicó a los libros\ldots{} quiero decir que era
encuadernador.

Sí; encuadernaba primorosamente, hacía jaulas y tocaba la zampoña, artes
de gran utilidad y nobleza en un hijo de reyes. Su fisonomía era
inocentona, y cuantos le veían juzgábanle bueno. En su edad madura
aprendió a conspirar. Conspiró en Aranjuez para echar a Godoy y
destronar a su hermano. Conspiró en Valencia y en todo el camino de
Valencey a Madrid para dar el golpe a la Constitución. Últimamente había
descuidado la zampoña y las jaulas y metídose a repúblico, mostrándose
tan entusiasta que su cuarto era, como si dijéramos, el gabinete de las
piadosas relaciones o la primera instancia de las comisiones del Estado.
La Inquisición restablecida, el decreto contra los afrancesados, el que
dispuso la devolución a los frailes de los bienes vendidos, fueron
primero ¡oh Providencia!, huevecillos que al calor de aquella reunión y
bajo las alas del infante, se abrieron para echar al mundo arrogantes
polluelos. ¡Cuántas medidas benéficas salieron de allí! ¡Cuántos hombres
modestos y oscuros se dieron a conocer por tal medio! ¡Cuántas grandezas
dio a luz la famosa tertulia, en que resplandecían astros tan brillantes
como D. Pedro Gravina, el célebre nuncio a quien dio los pasaportes la
Regencia de Cádiz, el duque del Infantado, general que tenía la mejor
mano del mundo para perder todas las batallas en que se encontraba, el
famoso canónigo Escóiquiz, a quien Napoleón tiraba de las orejas, y mi
buen Ostolaza, del cual ya he dicho todo cuanto hay que decir!

¡Qué hombres tan eminentes! ¡Cuán agradable era su conversación, cuán
ameno su trato, sin dejar de ser provechoso, por las muchas enseñanzas
útiles que a cada instante caían como celestial maná de aquellas
insignes bocas! No se crea que el Nuncio D. Pedro Gravina nos aburría
con teologías ni palabrotas de moral cristiana: por el contrario, era el
hombre más salado del mundo para idear persecuciones, y su agudo ingenio
nos tenía siempre con la felicitación en los labios.

El duque del I\ldots{} era otro que tal. ¡Cuántas grandezas podrían
contarse de aquel insigne prócer y guerrero! Acaudillando nuestras
tropas en la guerra de la Independencia, tuvo la amargura de verlas
derrotadas. Como político, aunque en Cádiz le calumniaron, suponiéndole
algo liberal, bien puede asegurarse que era más realista que el Rey. En
1815 ocupaba uno de los primeros puestos de la nación, la presidencia
del Real Consejo de Castilla. Había que ver su llaneza en todo lo que no
fuera del oficio. ¡Excelente señor! ¡Cuántas veces le vi en un palco del
teatro del Príncipe, acompañado de \emph{Pepa la Malagueña!}

En la tertulia del infante era el noticiero mayor, por lo cual siempre
que entraba, decíamos: «Ahí viene la \emph{Gaceta de Holanda}». No
faltaban nunca nuevas de importancia que nos sirvieran de placentera
distracción, tales como un nuevo cargamento de presos para Filipinas o
el buen éxito de las comisiones militares en provincias, y el inimitable
celo con que Negrete sentaba la mano a los liberales de Andalucía.

Escóiquiz criticaba mucho al gobierno porque no era bastante enérgico y
consentía que un Macanaz soñase con resucitar las Cortes, aunque
vestidas a la antigua. Ostolaza y yo hacíamos un espurgo de todos,
absolutamente de todos los individuos que figuraban por aquellos días.
Señalábamos los que nos parecían buenos a carta cabal, los tibios o
fililíes y los sospechosos a quienes precisaba quitar de en medio lo más
pronto posible. Aquí era donde yo me lucía, porque se me ocurrían
invenciones tan peregrinas para echar por tierra a cualquier señorón de
los más trompeteados, sin hacer ruido ni ofenderle descubiertamente, que
se embobaban oyéndome. Bien pronto llegué a hacerme tan importante en la
pequeña corte del infante, que este mismo, siempre que se hablaba de
algo referente a zancadillas en proyecto o quiebros por realizar, me
miraba atentamente para conocer mi opinión antes de emitir la suya.

¡Y cuidado si era sabio el príncipe! Como que la Universidad de Alcalá
le hizo doctor de golpe y porrazo, dándole patente de Aristóteles.
Nombrole el Rey poco después gran almirante de sus escuadras, por cuyo
motivo, aunque nunca había visto el mar, diose al estudio de la náutica,
y en la conversación corriente encajaba términos de marina, diciendo con
mucho énfasis: «\emph{Las cosas van viento en popa}», o bien
«\emph{echaremos a pique a los liberales}».

Yo crecía en favor, en importancia, en poder de día en día. Eran tantos
los asuntos delicados, espinosos y resbaladizos que se me confiaban, que
me vi obligado a valerme de agentes. ¡Y cómo me festejaban y mimaban los
grandes señores, sin dejarme nunca de la mano! Todo era «Pipaón acá,
Pipaón allá», y a cualquier hora Pipaón para todo.

Pues ¿y las peticiones de destinos? Como las minutas que yo extendía en
la tertulia del infante, pasaban muy bien recomendadas a manos de quien
sabía despacharlas con gran primor, no había candidato que no cuajase,
ni ahijado mío que no se viese en camino de papa o senescal desde que yo
le tomaba por mi cuenta. Así es que llovían las peticiones. Las cartas
entraban en mi casa por almudes, no siempre solas, en verdad, sino a
menudo acompañadas del bocadito, de la caja de cigarros, del tarro de
dulce. Siempre que iba a mi vivienda encontrábala tan atestada de
hambrones menudos, como portería de convento en tiempos de miseria.

Yo procuraba quitarme de encima tanto gorrón holgazán que, cual enjambre
de langosta, caía o anhelaba caer sobre la Real Hacienda; pero son los
pretendientes como las moscas, que cuanto más las sacuden más se pegan.
A muchos coloqué; pero como el frecuente ir y venir de oficina en
oficina me obligaba a gastar mucho tiempo y no pocos zapatos, discurrí
que era preciso hacer que los interesados me indemnizaran módicamente de
aquellas pérdidas.

Cuando se me presentaba alguno en cuya facha conocía yo que era hombre
de posibles, mayormente si venía de provincias con cierto cascarón de
inocencia, le recibía cordialmente, conferenciábamos a solas, le
persuadía de la necesidad de tapar la boca a la gente menuda de las
oficinas, conveníamos en la cantidad que me había de dar, y si se
brindaba rumbosamente a ello, cogía su destino. Siempre era una
friolera, obra de diez, doce o veinte mil reales lo que cerraba el
contrato, menos cuando se trataba de una canonjía, pensión sobre
encomienda u otro terrón apetitoso, en cuyo caso había que remontarse a
cifras más excelsas. Si nos arreglábamos, se depositaba la cantidad en
casa de un comerciante que estaba en el ajo, y después yo me entendía
con los superiores, si no me era posible despachar el negocio por mi
propia cuenta.

Asunto era este delicadísimo y que exigía grandes precauciones. Por no
tomarlas y fiarse de personas indiscretas, no dotadas de aquella fina
agudeza a pocos concedida, cayó desde la altura de su poltrona a la
ignominia de un calabozo un célebre ministro de Gracia y
Justicia\footnote{Macanaz.}.

\hypertarget{vii}{%
\chapter{VII}\label{vii}}

Con estas y otras artimañas iba yo \emph{viento en popa} como diría el
infante. Era tan considerable el número de mis amigos, que no acertaba a
contarlos.

Seguía en buenas relaciones con mi antiguo protector D. Buenaventura,
pero ni este se atrevía a ocuparme en viles menesteres, ni yo lo habría
consentido. Despachábamos juntos y mano a mano algunos asuntos
delicados, tocantes al Real Consejo, porque ha de saberse que el D.
Ventura, desde que cuajara el despotismo y se restableciera el régimen
antiguo, alcanzó la plaza de camarista, por la cual tenía antojos el
pobrecito señor desde su mocedad, o casi desde el vientre materno. ¡Oh!
¡Ningún arrimo se puede comparar al arrimo del Real Consejo y Cámara!
Daba gana de dormir en aquellos sillones, bajo aquellos techos
eminentes, en medio de aquella paz, de aquel reposo, de aquella
estabilidad inalterable, de aquella majestuosa petrificación de los
siglos, de aquel silencio, sólo turbado por los estornudos de algún
camarista y el ruido de los viejos, polvorosos y amarillos folios cuando
la flaca, la rapante mano del escribano los volvía. Era una tumba para
el mundo y un paraíso para los que estaban dentro\ldots{} Para el reino
la muerte, para los privilegiados dulce y reposada vida.

---«No hay institución más sabia que esta del Consejo---me decía D.
Buenaventura, con aquel entusiasmo que ponía siempre en sus palabras, al
hablar de las cosas venerandas, sublimadas por los siglos.---Eso de que
no pueda moverse un dedo en todo el reino sin que nosotros entendamos de
ello, es admirable para el buen concierto de las Españas y sus Indias.
Nuestra sala de Alcaldes es un primor. Con ser tan pequeña todo lo
abraza. Sin que ella lo autorice no puede el español sacar un pececillo
de las aguas de un río, ni vender una libra de uvas, ni echar la sal al
puchero. Todo lo pequeño está en nuestras manos, lo mismo que lo grande.
Sin nuestro permiso el reino no puede sublevarse ni tampoco rascarse. No
puede hacer revoluciones, ni cambiar de dinastía, ni reunir cortes, ni
establecer formas de gobierno, ni tampoco ir a los toros, ni cazar con
hurón, ni tener un desahoguillo mujeril, ni escupir, ni toser.

»Somos una máquina admirable que con sus grandes palancas aporrea al
mundo y con sus dientecillos roe lo que encuentra. Aquí todo se
convierte en polilla. Nada se nos escapa, y el vasallo de Fernando VII
tiene que venir aquí para que le digamos dónde tiene las manos.---¡Ay de
aquel que se atreva a alterar la dulce armonía en que vive la nación,
regocijándose en sí misma y mirándose en el espejo de su estabilidad
secular, como Narciso en la fuente! Si alguna cabeza hueca concibe
proyecto de aparente utilidad para desviar el suave curso de la española
vida, bien alterando las leyes del comercio, bien las de la fabricación,
ora los impuestos, ora la agricultura, nosotros acudimos solícitos allí
donde prendió el incendio de la reforma y procuramos apagarlo,
apoderándonos del proyecto o solicitud o requisitoria o informe o
memorándum para ponerle encima una losa de papel, bajo la cual se queda
criando musgo, si no gusanos, por los siglos de los siglos.

»En suma, es nuestra misión sostener en las esferas todas del país el
estado de sabrosísimo sueño que constituye su felicidad desde que
renunció a las conquistas. Nosotros arrullamos esta inmensa cuna
cantando el \emph{ro-ro}; y si por acaso en la agitación de su
placentero dormir saca una mano, se la metemos entre las sábanas; si
pronuncia alguna palabra, le tapamos la boca; si suspira, le rociamos
con agua bendita; si se mueve ¡ay!, si se mueve, nos asustamos mucho
porque creemos que se va a despertar\ldots{} Pero ahora tenemos
tranquilidad para un rato, amigo mío: el turbulento niño duerme; todo es
calma, todo es silencio, todo es paz, y apenas oímos el rugido del
descontento en el fondo de este gran pecho, que suavemente se alza y se
deprime con el reposado aliento de la satisfacción».

Así dijo. Concluía de comer, y levantándose, añadió:

---Adiós, Pipaón, me voy al Consejo a dormir la siesta.

La pintura de aquella alta institución narcótico-nacional despertaba más
en mí el deseo de afincarme en ella, como quien dice, proporcionándome
una plaza de camarista, que era la mejor almohada del mundo para reposar
una cabeza cargada de años y de trabajos. Contrariábame mi juventud y la
poca duración de mis servicios, si bien es verdad que para cubrir una
vacante en aquellos tiempos no había los ridículos escrúpulos y reparos
de antaño. Ya no se buscaba con candil, como en los días de Jovellanos y
Campomanes, un vejete sabihondo para endilgarle la cédula de
nombramiento, sin más méritos que haber escrito veinte mil informes
indigestos. Godoy echó por tierra estos abusos, llevando a la Cámara a
quien le dio la gana, sin distinción de talentos reales o postizos; y en
mi época esta tolerancia había llegado a su colmo, siendo evidente que
desde la entrada de D. Antonio Moreno en el Consejo de Hacienda, todos
los peluqueros de Madrid se vieron ya con un pie dentro de la Sala.

Esto me daba aliento, y no me acostaba ninguna noche sin pensar, al
persignarme, en las dulzuras de la anhelada canonjía del Consejo. Crecía
mi favor como la espuma, y a los comienzos de 1815 pude pasar del cuarto
del príncipe al del Rey, que era el Olimpo de la cortesanía, y trabar
comercio más íntimo con personajes del mayor prestigio y que, al decir
de las gentes, traían en los cinco dedos de su mano toda la grandeza del
reino, del cual eran árbitros, sin dar de ello cuenta al Dios ni al
diablo.

Impulsome por estos excelsos caminos la amistad que en Octubre de 1814
contraje con un hombre que en aquella época comenzaba a ser poderoso, y
después lo fue en tan alto grado, que siendo su nombre D. Antonio
Ugarte, el vulgo le llamaba \emph{Antonio I}, para significar un poder,
grandeza y predominio que al del mismo monarca se igualaba.

¿Y quién era Ugarte, quién era ese hombre poderoso, que por algún tiempo
dispuso del Tesoro de la nación, y tuvo a sus pies a todas las
eminencias civiles y militares, y dio que hablar dentro y fuera de
España casi tanto como Godoy en el reinado de Carlos IV?---Pues era
simplemente un maestro de baile.

Hombre tan insigne merece capítulo aparte.

\hypertarget{viii}{%
\chapter{VIII}\label{viii}}

En los últimos años del siglo anterior, Ugarte había venido de Vizcaya a
los 15 años de su edad. Menos afortunado que yo y con menos recursos,
tuvo que ponerse a servir de mozo de esportilla en casa del señor
Consejero de Hacienda, D. Juan José Eulate y Santa, donde se dio tan
buena maña y mostró tanto ingenio, que bien pronto, ayudado de su buena
letra y singular destreza en la aritmética, hiciéronle amanuense de la
casa. Habiendo nacido Antoñuelo para grandes empresas, no quiso su
destino que se prolongase por mucho tiempo la oscuridad de aquella vida,
y ved aquí que una aventurilla doméstica, en la cual apareció demasiado
listo, le obligó a separarse del Sr.~Eulate. El mancebo vizcaíno,
viéndose sin arrimo, pasó revista a todas las artes y ciencias, y
discurriendo cuál de ellas tomaría por instrumento de la gran ambición
que en su noble pecho abrigaba, adoptó la coreografía. Ya le tenemos de
maestro de baile, o como si dijéramos, con ambos pies dentro de la
esfera de la fortuna, que en aquellos tiempos solía favorecer a la gente
danzante.

Era Ugarte de hermosa presencia, agraciado, vivaracho, ingeniosísimo en
las frases, saludos y cumplidos, y extremadamente listo, con el más
claro ojo del mundo para conocer a las personas y captarse su simpatía y
buena voluntad. Vestía con toda la elegancia que sus mermados
emolumentos le permitían; conocía a fondo el \emph{ars umbelaria}, que
era el modo de ponerse el sombrero, y el \emph{ars incedaria}, que era
lo que modernamente y con más llaneza llamamos el \emph{modo de andar}.
No sólo daba lecciones de baile, sino que las daba también de
\emph{zorongo}, es decir, enseñaba a los jóvenes a hacer con la mayor
elegancia posible el gesto de afectadísima urbanidad conocido con este
nombre.

A pesar de tan supinos talentos, Ugarte no salía de su pobreza, que
entonces acompañaba, como el lazarillo al ciego, a las más nobles artes
de la cabeza o de los pies. Pero quiso el cielo que se prendase del
bailante vizcaíno una dama burgalesa (cuyo nombre no hace al caso), la
cual vivía en la Costanilla de Capuchinos de la Paciencia. Desde
entonces todo cambió. Baste decir que Godoy gobernaba a España y sus
Indias. Para medrar, Antoñuelo que tanto había movido los pies, no
necesitó más que el apoyo de una blanca mano. Sintiéndose con un gran
caudal de iniciativa y de recursos de ingenio, resolvió no meterse en
las telarañas de las covachuelas, y se hizo agente de negocios de
Indias, de los Cinco Gremios y de la dirección de Rentas. ¡Colosal mina!
Antoñuelo tenía talento en la cabeza, y dedos en las manos.

Por lo que yo hice con mediano ingenio en tiempos posteriores, y ya muy
explotados, júzguese lo que haría Ugarte con más genio para los negocios
que Nelson para la Marina, y en tiempos tan primitivos y virginales, que
bastaba alargar la mano para coger el sustento de hoy\ldots{} y el de
mañana. La Providencia divina, que en lo de mimar a Ugarte era una madre
débil y complaciente, le puso entonces en relaciones con el barón
Strogonoff, embajador de Rusia, el cual encargó a nuestro ex-bailarín el
desempeño de diversos asuntillos. Hízolo a pedir de boca, quedando el
moscovita tan complacido, que se fue para las Rusias en 1808, y dejó a
cargo de Ugarte todos sus intereses.

Durante la guerra, D. Antonio no se movió de Madrid. Firme en su
agencia, servía a españoles y franceses, sin malquistarse jamás con unos
ni con otros, que este es privilegio de ciertos hombres sutilísimos. Ni
los franceses le molestaron en 1812, aunque encubiertamente favorecía a
los nacionales, ni en 1814 le persiguieron por afrancesado los españoles
de la restauración. Con todo el mundo tenía buenas relaciones; para todo
se echaba mano de Ugarte. Murat y José, lo mismo que los regentes de
Cádiz, el cardenal de la Scala lo mismo que Fernando, el
\emph{botellesco} Cabarrús igualmente que el leal Eguía, le consideraban
y atendían. Hízose superior a los partidos, y a todos servía. Había
tenido hasta entonces el singular talento de no funcionar dentro de la
jurisdicción de las pasiones políticas, reservándose la esfera interior
de los negocios. Mientras arriba los bobos andaban al pelo por la
soberanía del pueblo y los derechos del trono, él resbalaba abajo
injiriéndose en los intereses públicos y particulares\ldots{} No era
nada; no era más que agente.

Aquí hemos visto muchos hombres, de esta clase; pero el maestro, el
patriarca, el Adán de estos bien aventurados camaleones, fue, sin duda
alguna, Antonio I, agente de todo lo agenciable.

Por entonces empezó la gran influencia de los rusos en la corte de
España, aunque todavía no habían aparecido por las ventas de Alcorcón.
Concluida la guerra vino acá el célebre Tattischief (a quien daré a
conocer más adelante), el cual por su antecesor tenía ya noticias de las
sutilezas de nuestro agente. Se hicieron tan amigos, que ambos salían de
paseo, dándose el brazo, confundiéndose los bailarinescos antecedentes
del uno con la noble prosapia del otro, para regocijo de la democracia
que ya empezaba a invadirlo todo. El ruso, que era emprendedorcillo,
como se verá en lo sucesivo y no había venido a Madrid a coger moscas,
encontró su mano derecha en Ugarte, y este halló en el ruso un admirable
espantajo que le sirviese de pantalla en la corte. Llevó Tattischief a
Antonio I a la tertulia de Fernando, hízole conocer a este las altas
dotes del antiguo maestro de \emph{zorongo}, y no fue preciso más. La
agencia de Ugarte se extendió; puso una mano en el corazón de la
monarquía, y extendió la otra a los últimos confines de ella en Europa y
en América. Un solo mundo no le bastaba.

Por aquella época (repito que al concluir 1814) nos hicimos amigos.
Habíame ocupado D. Antonio en diversos menesteres de mi incumbencia, los
cuales desempeñé tan bien, que se me confirieron secretos importantes y
fui asociado a empresas de mayor cuantía. Nos comprendimos, encajamos el
uno en el otro como el pie en el zapato; él conociéndome y yo
conociéndole, habíamos hecho la principal conquista de nuestra vida.

Y aquí levanto la mano del bosquejo de este hombre, porque sus
principales hechos no han ocurrido aún en los días a que me refiero.
Ellos irán saliendo poco a poco, y le pintarán por completo en todas sus
fases, siendo tan sólo mi propósito ahora trazar una leve figura lineal,
que por sí irá vistiéndose de colorido con la misma luz de los próximos
sucesos. Cuando yo conocí a D. Antonio, empezaba el gran poder de aquel
hombre, arbitrista, asentista, \emph{factotum}; de aquel agente
universal, que resolvió, en connivencia secreta con el Rey, graves
negocios de Estado; que tramó revoluciones y mudanzas, celebró tratados
y manejó la Hacienda pública sin responsabilidad; organizó ejércitos y
compró buques; todo esto sin intervención ninguna de los vanos
ministros, y obrando casi siempre a espaldas del llamado gobierno.

La figura de mi D. Antonio no revelaba entonces su antiguo oficio de
maestro danzante, ni tenía la ligereza que arte de tantos vuelos exigía:
era bastante obeso y de procerosa estatura, rostro de satisfacción,
doble barba con mucha enjundia, ojos muy móviles y una sonrisa más bien
esculpida que pintada en su rostro, por la fijeza de ella y por lo que
acompañaba a todas sus palabras. Ponía semblante afectuoso a chicos y
grandes, y con todos aparecía obsequioso y servicial, aunque después no
lo fuese. Tenía suma destreza para resolver en todo; respondía siempre a
medida, sin decir ni más ni menos de lo necesario; disimulaba sus
proyectos con discreción excelsa, a prueba de ajena perspicacia; jamás
emitía ideas exageradas, sino, por el contrario, era juicioso, y en sus
conversaciones sobre fútil política, siempre daba la razón a su
interlocutor; hablaba con veneración del Rey, guardando prudente
silencio sobre la dominación francesa, y no insultaba jamás a los
vencidos, sin duda por la consideración de que podían ser vencedores.
Cuando nombraba a alguno de los personajes desterrados o presos, decía
\emph{mi desgraciado amigo Fulano de Tal}, y a todos los hombres de viso
que entonces privaron les sahumaba con muchos elogios en presencia y
ausencia.

Delante de los tontos decía afectadamente tonterías, y delante de los
sabios sabidurías, y jamás hablaba mal de nadie, aunque estuviese en
Melilla o Ceuta. Era religioso y cuchicheaba con frailes y monjas; pero
nunca le vi abogar celosamente por la Inquisición, ni dio al fuego sus
libros filosóficos y enciclopedistas, pues los tenía buenos. Se
lamentaba de que los revolucionarios fueran tan malos; pero en más de
una ocasión le sorprendí en secreto con ciertos pajarracos que a cien
leguas me olían al musguillo húmedo de las logias y a sociedad secreta;
en fin, era hombre tan completo, que difícilmente se encontraría otro
ejemplar, ni quien, como él, estuviese siempre en la justa medida,
atento a su beneficio y realizando las supremas leyes de la vida con tal
arte, que el Criador del mundo debía de estar muy satisfecho por haber
criado a Ugarte. Sin duda después que lo echó al mundo, vio que era
bueno.

Este y Ostolaza, fueron los dos arcángeles que tiraron (permítaseme la
figura) del carro celestial de mi encumbramiento. Si uno me introdujo en
el cuarto del infante, llevome el otro al del Rey. Muchas y no
despreciables cosas tengo que contar de mis conexiones con los primeros
cortesanos de la época; pero antes de llegar al lugar sagrado, se me
permitirá que me ocupe de otras menudencias, que no por serlo, dejan de
ser indispensables para el conocimiento de lo que vendrá después, y de
cierto asunto que por mi propia cuenta emprendí. Como aquí entran
personas de menos copete y algunas madamitas, también abro capítulo
aparte.

\hypertarget{ix}{%
\chapter{IX}\label{ix}}

A casa de las de Porreño iba yo a menudo, y constantemente desde que se
apareció en aquellos tristes salones cierta condesa de Rumblar,
acompañada de un lindo femenil pimpollo, nombrado Presentacioncita, la
cual era un conjunto de gracias, seducciones y monerías de imposible
descripción. Tenía tal garabato para burlarse de Ostolaza y de mí,
elogiándonos en apariencia, que ni él ni yo sabíamos enfadarnos para
salvar la dignidad. Nos zahería muy sandungueramente, y por mi parte me
moría de gusto. La luz chispeante de sus ojitos negros como la noche,
deslumbraba los míos, y se me entraba y esparcía por todo el cuerpo,
escarbándome el corazón. Cuando reía, figurábasele a uno tener delante
un coro de angelitos insolentes jugueteando de nube en nube; cuando se
ponía seria, era preciso estar en guardia, porque de fijo estaba
tramando alguna ingeniosa picardía. Su gravedad era una máscara detrás
de la cual se fraguaban hipócritamente todas las aleves conspiraciones
contra nuestras casacas, contra nuestras chupas y también contra
nuestras pobres carnes.

Temblábamos ante ella y por mi parte me derretía de gozo cuando mi cara
se bañaba en su aliento durante una partida de mediator. Moralmente
hablando, nos pellizcaba sin cesar, pues no podían ser otra cosa sus
punzantes burlas. Digo punzantes, porque en cierta ocasión clavó en los
sillones donde Ostolaza y yo nos sentábamos, algunos alfileres tan
soberanamente dispuestos, que mi buen amigo y yo vimos sin ser
astrólogos, todo el sistema planetario. Otra vez cosió mis faldones a un
infame aparato, que moviéndose echó por tierra la cesta de costura donde
doña Paz tenía mil distintas suertes de labores, ovillos, canutillos,
lienzos, de tal modo, que levantarme yo y venir el mujeril aparato al
suelo, fue todo uno. A veces inventaba un juego de acertijo, en el cual
había un plato artificiosamente ahumado, que nos aplicábamos a la cara
para saber el secreto, y puesta la sala a oscuras, resultaba después que
aparecíamos Ostolaza y yo con la cara tiznada, de lo cual se holgaban y
reían mucho los concurrentes. A menudo recibía yo cartitas y recados de
monjas mandándome llamar, y luego salíamos con que era mentira. Y no
digo nada de aquella graciosísima invención que consistía en darme un
dulce, y cuando yo todo almibarado de gozo me lo metía en la boca,
resultaba más amargo que la misma hiel.

¡Ay!, en aquellas tertulias había verdadero entretenimiento; se divertía
uno con la más rigurosa honestidad, sin propasarse jamás a cosas
mayores, y aunque se padecía un poco del mal de Tántalo, como teníamos
el juego de la gallina ciega, siempre había algún yo y tú casual entre
tapices, y se podía coger al vuelo un par de blancas manos, algún
torneado brazo, u otra cualquier obra admirable del Criador. Daba la
maldita casualidad de que siempre que se estaba rezando el rosario,
sonaba adentro descomunal y pavoroso ruido, y a oscuras o con un
candilejo era preciso ir a ver lo que era, no faltando damas valerosas
que le acompañasen a uno por los solitarios corredores. Por supuesto, al
fin venía a resultar que aquellos espantables ruidos eran obra del gato,
haciendo de las suyas en la cocina.

Con estos y otros inocentes placeres, se pasaban dos o tres horas de la
noche sin sentirlo.

Una noche noté que Presentacioncita no nos dio bromas ni a Ostolaza ni a
mí. No di importancia a aquel suceso. A la noche siguiente no fue a la
tertulia, y se dijo que estaba enferma: pero apareció tres noches
después bastante desmejorada y muy triste, lo cual me sorprendió mucho,
y observé. Observé su semblante, su mirar, qué conversaciones prefería,
a cuáles palabras prestaba más atención. Observé sus suspiros y la
distracción honda en que comúnmente estaba, deduciendo de todo que
Presentacioncita tenía un gran pesar sobre su alma.

Pero lo más extraño fue que la graciosa niña no sólo se abstenía por
completo de toda burla mordaz conmigo, sino que me trataba con
inusitadas consideraciones, fijando en mí los ojos, cual si quisiese
leer mis pensamientos y por ellos adivinar mis deseos, para
satisfacerlos.

Atendía al juego, alegrándose mucho cuando yo ganaba, y demostrándome en
sus ojos profunda pena si la suerte no me era propicia. Al retirarme me
miró mucho, preguntándome con vivísimo interés si faltaría a la tertulia
de la noche siguiente.

Acosteme y no dormí. Los dos ojos de Presentación fulguraban en la
oscuridad de mi alcoba como estrellas en el negro cielo. Pero yo no soy
hombre que pierde el tino por afán de ideales amores, ni en mi vida he
experimentado el embrutecimiento de que hablan los poetas, dolencia
común a cabezas hueras y a gente vagabunda. Reíme, pues, de aquello, y
vino el día y tras él la noche. Pareciome al entrar en la tertulia que
con mi visita se disipaba la tristeza de Presentacioncita, como con la
presencia del sol huyen las nieblas que oscurecen y enfrían la tierra.
¿A qué negarlo?, yo estaba inflado de orgullo.

Conocí que deseaba hablarme, y por mi parte sentía ardiente anhelo de
decirle un par de palabritas al oído, sin que lo viera mi señora la
condesa. Ofreciósenos a entrambos ocasión propicia cuando los demás
hablaban ardientemente de la caída de Macanaz. Presentacioncita me dijo
con la mayor zozobra:

---Sr.~de Pipaón, tengo que hablar con usted.

---Y yo también, señora doña Presentacioncita, tengo que\ldots---repuse
sin poder encontrar una fórmula de madrigal.

---Pero mucho, mucho---añadió ella, poniéndose más encarnada que un
cardenal.

---¿Mucho?

---Tengo\ldots{} tengo que confiar a Vd\ldots{}

---Sí, yo también\ldots{}

---Un gran pesar.

---¿Pesar?

---Sí, una gran pesadumbre, y espero\ldots{}

---Yo también espero\ldots{}

---Espero que Vd. me hará el favor que he de pedirle\ldots{} Vd., sí, me
han dicho que sólo usted\ldots{}

Yo estaba confundido y nada contesté.

---Mañana, Sr.~de Pipaón\ldots---dijo disimulando todo lo posible su
inquietud;---mañana\ldots{}

---Mañana, o cuando Vd. quiera\ldots{}

---Venga Vd. aquí. Estaremos solas doña Salomé y yo. Mi madre, doña Paz
y doña Paulita van a visitar a las monjas de Chamartín. Yo he dicho que
vendré a ayudar a doña Salomé en una labor que trae entre manos.

Al siguiente día a la hora marcada acudí presuroso a la cita, poniéndome
de veinticinco alfileres. Retirose la de Porreño cuando yo entré, y
Presentacioncita no esperó a que me sentara para decir:

---Sr.~de Pipaón, en Vd. confío, en su mucha bondad y cortesanía. Se
trata de una obra de caridad.

---¡Una obra de caridad!\ldots{} ¡Y para eso\ldots!---exclamé
desconcertado.

---Se lo agradeceré a Vd. toda mi vida, toda mi vida---dijo ella
cruzando las manos y clavando en mí hechiceras miradas.

Empecé a sospechar si sería aquella una refinada burla, con gran arte
preparada.

---Veamos: ¿qué obra de caridad es esa?---pregunté tan inquieto y
sobrecogido, cual si sintiera en el asiento de la silla los alfileres de
marras.

Presentacioncita fijó los ojos en el suelo, y doblando y desdoblando la
punta del pañuelo, dijo:

---Yo tengo\ldots{}

---Vamos, acabe Vd.

---Me cuesta mucho trabajo, Sr.~de Pipaón; pero no tengo otro remedio
que decírselo a Vd.

---Pues oigo. ¿Tiene Vd.?\ldots{}

---Vergüenza.

---¿Es algún pecado?

---Pecado no.

---Entonces es amor.

Presentación respiró cual si la quitaran de encima un gran peso.

---Eso es. Cuesta mucho decirlo\ldots{} Gracias, Sr.~D. Juan. Me ha
adivinado usted. Bien dicen que otro de más ingenio no lo hay bajo el
sol.

---¿Y quién es ese dichoso joven?---pregunté de muy mal talante,
esforzándome en poner cara indiferente.

---Ese joven\ldots{} es\ldots{} vamos, un joven\ldots{} muy desgraciado
por cierto, si Vd. no lo remedia.

---¿Yo?\ldots{} ¿Y en qué puedo servirle?

---¡Ay!, para un hombre como Vd. no hay nada imposible. Por su mucho
talento ha logrado ganarse una buena posición; es amigo de Antonio I,
del infante, y tiene gran poder en la corte\ldots---añadió con mucha
zalamería.

---¡Yo!

---O en el gobierno. ¡Qué gusto para la madre que tal hijo crió! Verle
encumbrado por sus méritos nada más y gran entendimiento; verle
solicitado de los grandes señores y hasta de los obispos\ldots{} No
sabemos a dónde va a llegar Vd., Sr.~de Pipaón, y si no para de subir,
le veremos ministro o gobernador del Consejo o embajador el día menos
pensado.

---Gracias, señora doña Presentacioncita. Pero\ldots{}

---Pero\ldots{} déjeme Vd. seguir---repuso impaciente, porque la
revelación del principal secreto le había devuelto su normal viveza y
desenvoltura.

---Ya oigo.

---Decía que si Vd. me libra de la grande aflicción que tengo, rezaré
todas las noches un padre nuestro para que Dios le haga a usted
embajador o ministro.

---Hecho el trato---respondí riendo.---Su novio de Vd\ldots{}

---¡Por Dios y por todos los santos, sea Vd. reservado! Hago a Vd. esta
confianza porque conozco su prudencia, su bondad, su discreción. Antes
moriría que fiarme de Ostalaza.

---Lo creo.

---Y si usted dice alguna palabra por la cual mi señora madre pueda
sospechar\ldots{}

---¡Oh!\ldots{} lo que es eso\ldots{}

---Entonces tomaré venganza tan horrenda, tan espantosa\ldots{}

---Lo creo, sí, lo creo sin juramento.

---Tan espantosa, que\ldots{} vamos: ya estoy teniendo compasión de Vd.
¡Oh!, de veras\ldots{} será Vd. el más desgraciado de los hombres.

---El más feliz seré si consigo sacar a Vd. de ese mal paso\ldots{}

---A mí no, a él---exclamó con viveza.

---¿Quién es? ¿No se puede saber?

---Vd. le conoce---dijo, fiando a mi penetración lo que sólo
correspondía a su franqueza.

Avergonzábase de pronunciar el nombre de su adorado, y todo era medias
palabritas, reticencias, adivinanzas, mucho de \emph{que se quema
usted}, hasta que al fin, con más trabajo que para sacar alma del
Purgatorio, la saqué del cuerpo el dichoso vocablo, resultando que
aquella Tisbe tenía por Píramo a un mozalbete de buena familia, llamado
Gasparito Grijalva, hijo de don Alfonso de Grijalva, propietario muy
adinerado.

---¿Y en qué apreturas se encuentra ese joven, que tanto necesita de mí?

Presentacioncita se sintió conmovida, y llevándose el pañuelo a los
ojos, dijo:

---Está preso.

---Vamos, madamita, no llorar. Eso no conduce a nada---repuse, dándole
algunas palmadas en el hombro.---¿Y qué diabluras ha hecho?\ldots{}
Alguna pendencia, alguna disputa quizás por esos lindos ojos?\ldots{}

---No es nada de eso---añadió sollozando.---Le prendieron porque en el
café dijo que Su Majestad era narigudo.

No pude contener la risa.

---¿Por eso, nada más que por eso?

---Y por haber dicho que Su Majestad escribía cartas a Napoleón desde
Valencey, felicitándole y pidiéndole una princesa para casarse.

---¡Oh!, grave desacato es ese\ldots{}

---¡Ay! Sr.~D. Juan---exclamó, cubriéndose el rostro y llorando sin
freno,---yo me muero de aflicción, yo no puedo vivir\ldots{}

---Calma, mucha calma, señora mía, y discurramos lo que se ha de hacer.

---¡Y dicen que le van a ahorcar, Sr.~de Pipaón!---añadió, volviendo a
mostrar sus ojos, más bellos entre la humedad del llanto, como es más
bello el sol después de la lluvia.---Eso sería una iniquidad, un
crimen\ldots{} ¡Ahorcarle por decir una tontería!\ldots{}

---Por eso se ahorca hoy\ldots{} Discurramos. El delito es
horrendo\ldots{}

---¿Horrendo?

---Sí; ¡calumniar a Su Majestad, diciendo que anduvo en tratos con el
infame monstruo!\ldots{}

---¡Cosas de muchachos! Como su padre es algo liberal, según dicen, y
parece que no quiere toda la Inquisición, sino una parte de ella, desean
castigarle en la persona del pobre, del inocente Gaspar\ldots{} ¡Ah! ¡Si
viera Vd. qué carta me escribió ayer!\ldots{} Yo no sé cómo se las
compuso para escribirla en la cárcel y enviármela, pero ello es que la
recibí. Me suplica que le mande secretamente un cordel o un puñal para
darse la muerte, antes que el verdugo ponga las manos sobre él. ¡Esto
parte el corazón! Parece que siento el puñal clavado en mi pecho y la
cuerda alrededor de mi cuello\ldots{} Y gracias a que Dios me ha
deparado un amigo tan bueno y generoso como Vd., pues ¿quién duda que
beberá los vientos para que pongan a Gasparito en libertad?

---Falta que lo consiga, porque la justicia de estos tiempos no se anda
con tiquis miquis, y si bien es posible que el niño no lleve corbata de
cáñamo por ahora, casi casi se le puede dar una carta de recomendación
para los que están en Ceuta o en Melilla.

---¡En África, en presidio!\ldots{} Para Vd., según dicen, no hay nada
difícil, todo lo consigue y es el más activo correveidile, el más
bullidorcito y hormiguilla de los empleados públicos de hoy.

---Gracias.

---De modo que si Vd. no quiere verme morir de pena, si Vd. no quiere
que le maldiga en mi última hora y que desde este momento le aborrezca
como a mi más cruel enemigo, prométame que dentro de unos pocos días
estará Gaspar en libertad.

---Mucho pedir es, señora doña Presentacioncita. Yo no tengo poder en la
corte, ni en la camarilla, que es donde se prende y se suelta a todo el
mundo. ¿Por qué no se franquea Vd. con Ostolaza?

---¡Jesús, ni pensarlo!---exclamó con terror.---Se lo contaría todo a
mamá.

---En fin, yo haré lo que pueda---dije, prometiéndome interiormente no
volver a ocuparme de tal asunto.

---¡Lo que pueda!\ldots{} Eso es bien poco. Ha de hacer Vd. lo que no
pueda, lo imposible, señor de Pipaón. Por ahí le llaman a Vd. Santa
Rita.

---Mucho se me pide---indiqué dulcemente, discurriendo que bien podían
darse algunos pasos, con tal que fueran remunerados de alguna manera---y
nada se me ofrece.

---¿Y mi agradecimiento eterno, mi amistad, lo mucho que rezaré por Vd.
para que siempre goce de buena salud y llegue a ser, cuando menos,
ministro, y pueda repartir beneficios a los necesitados?---repuso con
hechicera sonrisa, que valía más que todas las razones, y podía más que
todos los ruegos.

---Presentacioncita---dije, acercándome más a ella.---Nunca creí que una
niña tan linda, tan discreta, tan bondadosa, de tantísimo mérito como
Vd., fuese a caer en las redes de un\ldots{}

---Menos incienso, Sr.~D. Juan---replicó con malicia,---hoy no estoy
para zalamerías.

---Pues qué, ¿esos ojos celestiales, esos\ldots?

Alargué una mano para tocar la suya, cuando rechinaron los goznes de la
puerta y yo salté en mi silla. La puerta se abrió, dando entrada a una
figura pomposa, que desde su primer paso y desde su primera mirada
empezó a irradiar magnificencia dentro de la habitación. Era doña María
de la Paz Jesús, hermana del señor marqués de Porreño, y desde la muerte
de este, jefe de la ilustre cuanto desgraciada familia\footnote{Véase
  \emph{La Fontana de Oro}.}. Venía de la calle, y como era mujer de
corpulencia, con el cansancio y la pesadez de sus carnes traía muy
sofocado el rostro y fatigosa la respiración. Sentose al punto, sin
despojarse del mantón ni soltar el ridículo, abanico, sombrilla y manojo
de papeles que en la mano traía como Minerva sus atributos, y lejos de
enojarse por verme allí a hora tan impropia, pareció alegrarse mucho de
mi presencia.

Aquella señora tan grave, tan rigurosa, tan ceñuda, tan implacable con
toda clase de libertades, sonreía ante mí, dignándose echar el velo de
su delicadísimo disimulo sobre aquel coloquio a solas, que en época
posterior habría sido inocente, pero que en tiempos tan honestos era
poco menos que escandaloso, casi nefando. Yo esperaba una tempestad, y
me encontré con un arco iris.

Oigámosla ahora.

\hypertarget{x}{%
\chapter{X}\label{x}}

Antes de responder a mi saludo, me dijo:

---Espero que Vd., Sr.~de Pipaón, como hombre de gran influencia, amigo
de Ugarte Alagón y Pedro Collado, nos apoyará en nuestra justa
pretensión, haciendo cuanto esté de su mano para que salgamos adelante.

---¿Y cuál es el asunto?\ldots---pregunté confundido.

---¿Pues no lo sabe Vd.? ¿No estuvimos hablando de eso más de dos horas
anteanoche?

---¡Oh!, sí, señora mía, ya recuerdo, es\ldots{}

---La moratoria que pretendemos\ldots{} Ya hemos hecho la solicitud a Su
Majestad, y se nos ha prometido que pronto se dará cuenta de ella a la
regia Cámara, y que la apoyarán los más cariñosos amigos del soberano.

---¿Una moratoria? ¿Conque una moratoria?\ldots{}

---Nada más justo---dijo doña María de la Paz, con acento de convicción
profundísima.---Ni se me alcanza por qué han de ser tan lentas y
fastidiosas las formalidades para concederla; debiera ser cuestión de un
par de días y de una esquelita de Su Majestad al Real Consejo.

---Señora, una moratoria siempre es asunto de gravedad.

---Pero no en el caso presente, Sr.~de Pipaón---exclamó con viveza
arrojando de sí una llamarada de orgullo que se extinguió bien pronto,
como las chispas brotadas del pedernal.---Nosotras reclamamos una cosa
muy justa. Mi padre y mi hermano contrajeron algunas deudas\ldots{} la
cantidad no hace al caso. Hiciéronlo así, porque el lustre de nuestra
casa lo exigía, pues sólo en una comida de caza y pesca que se dio al
Rey, al pasar por Montoro, cuando la batalla de las Naranjas, se
gastaron treinta mil ducados. Ahora los acreedores, de los cuales el
principal es D. Alonso de Grijalva, han dado en reclamar su dinero y
quieren apropiarse las fincas libres que nos quedan, pues bien sabe Vd.
que el mayorazgo, conforme a la ley de su principal instituto, se ha
extinguido en nuestra línea por falta de varón.

---Ya, ya sé. ¿Vds., por falta de varón?\ldots{} Comprendido.

---¿Cómo es posible, pues, que un Rey justiciero, que ha venido a
establecer en España las buenas doctrinas y a limpiar el reino de toda
impiedad y bajeza, consienta en este despojo, en este embargo inicuo,
insólito, irrespetuoso con que se nos amenaza?

---Señora, los acreedores\ldots{} Ellos dieron, mejor dicho, colocaron
su dinero\ldots{} ---indiqué respetuosamente.

---Sí, señor---añadió, despidiendo otro chispazo de soberbia que iluminó
velozmente su rostro.---¿Pero qué vale su dinero?\ldots{} ¡Miserable
metal! Como si no hubiera en el mundo más que dinero\ldots{} ¿Pues y las
virtudes, pues y las glorias y grandezas del reino, pues y el lustre,
fíjese Vd. bien, el lustre de las familias?

---El lustre. Sí, convengo en que el lustre\ldots{}

---No, no es posible que un gobierno justo nos quite la hacienda que
honrosamente poseyeron nuestros antepasados. ¡A dónde vamos a parar!
Estaría bueno que un D. Alonso de Grijalva, un hombre que ha salido de
la nada, pues público es y notorio que vino a Madrid de la Maragatería,
conduciendo un par de mulas; estaría bueno, repito, que un D. Alonso de
Grijalva, fíjese Vd. bien, un D. Alonso de Grijalva, se calzase nuestros
estados de Galicia y Aragón. ¡Oh! Es zapato muy grande para tal pie.
Esos hombrecillos, nacidos de los tomillos y mastranzos, tienen una
osadía que espanta. Tanto alzaron el vuelo en tiempos de la
Constitución, que se creían dueños del mundo, y por lo que veo, aun
después de vueltas las cosas a su ser y estado primero, continúan
alzando la cabeza y amenazando con sus viles usurpaciones.

---En suma, Vds. solicitan que se ponga coto al inconcebible
atrevimiento de los que han dado en la flor de llamarse acreedores.

---¡Oh!, nosotras no negamos la deuda, ni tampoco el proposito firmísimo
de pagar algún día---repuso con voz firme.---Pero deseamos que esos
señores confíen en nuestra probidad y esperen tranquilos la hora
oportuna de recoger lo suyo. ¿Pues quién duda que es suyo? Nuestra
pretensión no puede ser más natural. Sólo pedimos a Su Majestad que nos
conceda una moratoria nada más que de diez años, fíjese Vd. bien, de
diez años\ldots{}

---Ya estoy fijo, sí. Me parece muy justo. Dentro de diez años\ldots{}

---No creo que Su Majestad, tan piadoso, tan buen cristiano, tan
justiciero, tan cariñoso para todos los que no nos hemos contaminado de
la constitucional pestilencia, niegue una pretensión tan razonable,
mayormente si considera que el fiero enemigo, de cuyas garras queremos
librarnos, es un hombre a quien suponen un poco desafecto al régimen
actual.

---El Sr.~de Grijalva no se mezcla en política. Es hombre modestísimo,
que sólo se ocupa de gobernar su casa y sus intereses.

---¡Oh!, qué mal lo conoce Vd.---repuso con súbito arranque.---Si yo
dijera que no hay lengua más cortante contra el gobierno ni tijera más
diestra que la suya para cortar vestidos a los amigos de Su
Majestad\ldots{} En fin, ¿qué tal hombre será y qué tal educación dará a
sus hijos, cuando ha sido preso Gasparito por desacatos al Rey y no sé
qué abominables dichos y hechos?

---Parece que el niño dijo en un café que Su Majestad era narigudo.

---Algo más sería---afirmó doña María de la Paz, con verdadera
saña.---Descubriose que andaba en logias, escribiendo papeles y
reclutando gente de mal vivir.

Presentación parecía de cera.

---¡Oh!, si es cierto---afirmé,---el hijo y el padre lo pasarán mal.

Presentación parecía de mármol.

---No, tales infamias no pueden quedar sin castigo. Veo que Su Majestad,
llevado de su buen corazón, está por las blanduras y perdona a todo el
mundo. ¡Escarmiento!\ldots{} duro con ellos, Sr.~de Pipaón. ¡Si no se
castiga a nadie!

Presentación había enrojecido y parecía de fuego.

---Pero cualquiera que sea el fin de estas abominables
conspiraciones---continuó la dama,---Vd. tomará a pechos nuestro
negocio, usted nos prestará su poderoso apoyo, Vd. arrimará su hombro al
sagrado muro, fíjese Vd. bien, al sagrado muro de nuestra moratoria. ¿No
es verdad amigo mío?---dijo doña María de la Paz, levantándose para
retirarse.

---Yo\ldots{}

No pude decir más, porque en aquel instante concebí una idea grandiosa,
colosal, una de esas ideas que de tarde en tarde fulguran en el cerebro
del hombre, abriendo ante sus ojos inmenso horizonte en los espacios de
la vida, una idea que absorbió mis potencias todas por breve rato, no
permitiéndome ver cosa alguna, ni pensar en nada que estuviese fuera de
la esfera de mí mismo. Tras de la idea vino un propósito firme,
poderoso, y después un plan, cuyo sencillo organismo se me representó
clarísimo en todas sus partes.

---Señora, no necesito decir que haré los imposibles porque se consiga
esa moratoria---manifesté con artificioso interés a la dama, cuando se
retiraba.

Después volví al lado de Presentacioncita. Su cólera, mal contenida, se
desahogaba en amargo llanto.

---Adorada y adorable niña---le dije con acento de profundísima
verdad.---No llore usted: todo se arreglará.

---Vd. es muy bueno, ¿Vd. será capaz\ldots?---dijo levantándose y
poniéndose ante mí con las manos cruzadas, como se pone la gente piadosa
y afligida delante de una imagen.

---Tranquilícese Vd.; Gasparito será puesto en libertad---afirmé con el
mayor aplomo.

---¿Cuándo?

---Cuando se pueda. No hay que impacientarse. El muchacho no irá a
presidio.

---¡Oh! ¡Qué hermosas palabras!---dijo saltando de alegría y secando sus
lágrimas.---De modo que no\ldots{}

---No le condenarán.

---¿Vd. lo promete?

---Solemnemente.

---¡Qué bueno es Vd\ldots{} pero qué bueno! ¡Ay qué guapo es Vd.! Sí,
¡qué guapo y buen mozo me parece! ¿Por qué no lo he de decir? ¿Conque
Vd. promete que no le harán daño?

---Lo juro. Óigalo Vd. bien. Lo juro.

---¡Oh!, gracias, gracias, Sr.~de Pipaón. Que Dios le dé a usted la
gloria eterna, y en este mundo mucha salud, toda la felicidad, todos los
destinos de la nación, todos los sueldos, todas las encomiendas, todas
las grandes cruces del mundo, y aún me parece poco para lo mucho que Vd.
se merece.

Diciéndolo así y desahogando en tiernos votos la loca alegría de su
corazón, alargaba hacia mí sus cruzadas manos con ademán patético.

Salí de la casa. ¿Cuál era mi idea, mi propósito, mi plan? Se verá más
adelante.

\hypertarget{xi}{%
\chapter{XI}\label{xi}}

Era Ugarte muy amigo del duque de Alagón, capitán de Guardias de la Real
persona, inseparable acompañante del monarca dentro y fuera de Palacio.
Yo también tuve relaciones estrechas con el duque, a quien visitaba
frecuentemente por encargo de D. Antonio, para tratar de asuntos
reservados, en los cuales no era posible otra tercería que la del nieto
de mi abuela.

Por cuenta, pues, de Ugarte y por la mía propia (llevado del luminoso
plan que mencioné más arriba), fui a ver cierto día al señor duque de
Alagón, que vivía en palacio. Cuando entré en su despacho, Su Excelencia
no estaba solo. Acompañábale un hombre de mediana edad, de aspecto no
desagradable, aunque tenía muy poco de fino, de semblante fresco, rudo,
como de quien en su crianza vivió más bien al desamparo de los montes
que en la regalada comodidad de los regios salones; vestido lujosamente,
aunque sin ninguna elegancia, con librea de flamantes galones; un
personaje, en fin, del cual se podía decir que era un cortesano que
parecía lacayo, y un lacayo que parecía cortesano. Recostado en muelle
sillón, fumaba un habano, y su coloquio con el duque era tan corriente y
por igual, que dos duques no se hubieran hablado de otro modo\ldots{} ni
tampoco dos lacayos.

Cuando entré, el duque dijo:

---Podemos seguir hablando, Sr.~Collado. Pipaón es de confianza y no
importa que nos oiga.

---Es que Su Majestad se despertará pronto; llamará y tengo que llevar
el agua ---repuso Collado mirando el reló.

---Aún es tiempo---dijo el duque vivamente.---Para concluir,
Sr.~Collado\ldots{}

---Para concluir, señor duque\ldots{}

---Concedo las dos bandoleras a cambio de la canonjía.

---Que no puede ser, que no puede ser\ldots{}

---Pues vaya\ldots{} tres bandoleras.

---¡Qué pesadez de hombre!---exclamó el de la librea, que no era otro
que el eminente Chamorro, ayuda de cámara de un alto personaje.---He
dicho a Su Excelencia que me pida el arzobispado de Toledo o media
docena de mitras sufragáneas, pero que me deje en paz esa canonjía de
Murcia, que es plaza de gran empeño para mí, porque la tengo prometida
al sobrino de mi cuñada.

---Pues precisamente esa canonjía de Murcia y no otra es la que yo
quiero con preferencia al arzobispado metropolitano---afirmó el duque
agitando los brazos.---Se la prometí a la condesa, se la prometí, le di
mi palabra de honor\ldots{} Sr.~Collado, por amor de Dios\ldots{}
Disponga usted de dos plazas de guardia\ldots{} vamos, de tres.

---Ni de cuatro. ¿Para qué quiero yo eso?---repuso Collado con desdén,
contemplando el humo que desde su boca subía hasta el techo en blancas
espirales.---Traigo entre manos la comandancia general de la plaza de
Santoña\ldots{}

---Ya sé para quién es eso---dijo el duque con presteza.---Ya se convino
en darla al marido de la Pepita.

---De doña Rafaela, dirá Vd., de doña Rafaela.

---¡Doña Rafaela! Esa mujer es insaciable. Se ha llevado ya todas las
plazas fuertes, y quiere también echar mano al Consejo Supremo de la
guerra. No he visto mujer que tenga más parientes. Es prima, hermana y
sobrina de medio ejército\ldots{} ¡Y la pobre Pepita a quien yo
prometí!\ldots{}

---No faltará para ella---repuso Collado.---En esa lista de vacantes que
tiene Su Excelencia, ¿no se le había señalado a Pepita (para su tío el
clérigo, se entiende) la Colecturía general de Expolios y Vacantes,
Medias Annatas y Fondo Pío beneficial?

---Si no hay tales vacantes---repuso el duque de mal humor;---las he
provisto todas. Veamos otra cosa: ¿quién cae?

---Ya recordará Vuecencia los que perecieron anoche---manifestó Collado,
sonriendo con malicia.---Está abierto el hoyo para dos consejeros de
Órdenes, por \emph{tibios} y amigos de Macanaz.

---Y para el director de Tercias Reales, si no recuerdo mal.

---Y para dos beneficiados del \emph{Venerable e inmemorial cabildo de
Guadalajara}.

---También tiene la marca en la frente---añadió el duque, con
satisfacción parecida a la de los labradores cuando hablan de buena
cosecha,---el superintendente de Correos, por haberse negado a dar
cuenta de aquellas cartas sobre el baile de máscaras.

---Muchos puestos hay---afirmó Chamorro con enfáticas pretensiones de
gracejo,---pero hoy han venido tres obispos con trescientas solicitudes
de guerra o marina. Esto es mezclar berzas con capachos.

---¡Qué demonio!\ldots{} ¿Y destierros, hay algunos?

---Tal cual\ldots{} así andamos. Pero ¿no se le concedieron a Vuecencia
unos trece o catorce la semana pasada?

---Es verdad; pero los he gastado todos. Quisiera más---dijo Alagón con
disgusto.---¿No ve Vd. que necesito muchos puestos vacíos? ¡La condesa,
Juanita, doña Romualda! Si no me dejan respirar\ldots{} Esa gente con
nada se satisface. Creen que la nación se ha hecho para ellas. Ya se ve:
como ellas parecen hechas para la nación\ldots{}

---Pues Su Majestad hace días que anda muy reacio, señor duque---afirmó
Pedro con burda socarronería.---Dice que abusamos.

---¡Que abusamos!

---Y que es preciso en la provisión de destinos dejar algo a los
ministros, porque estos se quejan de la nulidad a que están reducidos y
del tristísimo papel que hacen.

---Aquí hay alguna mano oculta, Sr.~Collado---exclamó con rabia el
duque.---Aquí hay alguna intriga. A Vd. y a mí nos están engañando, y
con vivir tan cerca de Su Majestad, no sabemos lo que pasa.

Chamorro se encogió de hombros. El duque mirome con atención, y sus ojos
parecían decirme: ¿Qué piensa Vd.?

---Todo depende---dije yo, rompiendo el silencio que, por darme mayor
importancia, había guardado hasta entonces;---todo depende de los humos
que han echado algunos ministros, como el fatuo, el insolente D. Pedro
Ceballos; como D. Juan Pérez Villamil y otros.

---Bien, muy bien dicho---exclamó el antiguo aguador de la fuente del
Berro, dándome una palmada en la rodilla para demostrarme su conformidad
absoluta con mi parecer.

---Observen Vds. bien, cuál es el plan de los ministros---proseguí
enfáticamente.---El plan de los ministros bien claro se ve\ldots{} es
apoderarse del ánimo de Su Majestad, inclinarle a aceptar todas las
medidas que ellos proponen, ordenar las cosas de modo que todos los
asuntos públicos sean resueltos por ellos, y todos los destinos dados y
quitados por ellos.

---Justo, eso, eso es---exclamó el duque,---Pipaón ha puesto el dedo en
la llaga.

---Bien claro lo demuestran las providencias que se están tomando---dijo
Chamorro con ademán meditabundo.---Para imponer su voluntad, han
empezado por aconsejar al Rey que vaya dejando a un lado las medidas de
rigor. ¡Oh!, aquí hay algo. En la aldehuela, más mal hay del que se
suena.

---Como que ya han acordado suprimir las comisiones de Estado, y se han
prohibido las denominaciones de \emph{serviles y liberales}---indiqué
yo.---En suma, señores, hay en el ministerio algunos individuos que se
manifiestan deferentes ante el monarca; pero ¿qué pensaremos de un
Ceballos, de un Villamil? ¿Qué pensaremos, repito, al verles empeñados
en llevar el gobierno por los torcidos caminos de una tibieza hipócrita?

---Una tibieza que no es más que constitucionalismo disfrazado---dijo
Alagón, dándoselas de muy perspicuo.

---¡Constitucionalismo!---repitió Collado.---Así se lo he dicho esta
mañana. Debajo del sayal hay al.

---¿Y qué dijo? ¿No hizo alguna observación chusca?---preguntó con
interés vivísimo el duque.

---Siempre que le hablo de esto, calla como un cartujo---repuso con
descorazonamiento Collado. Al buen callar llaman Fernando.

Los dos palaciegos permanecieron meditabundos por breve rato.

---Yo no sé qué raíces echa el tal D. Pedro donde quiera que pone los
pies---dije yo;---pero es lo cierto, que cuando se instala, no se deja
echar a dos tirones.

---Es hombre listo y que sabe manejarse---añadió el duque.---Cuando ha
sabido hacer olvidar sus servicios a Bonaparte en Bayona y a las Cortes
en Cádiz\ldots{}

---Pues si he de ser franco, señores---afirmé yo con mucha hinchazón y
petulancia,---manifestaré a Vds. una cosa, y es que\ldots{} Vamos, lo
diré en dos palabras. Si yo viviera en esta casa, D. Pedro Ceballos no
duraría una semana en el ministerio.

---¡Ay, amigo!---me dijo el duque, poniéndome familiarmente su noble
mano en el hombro.---¡Vd. no sabe qué clase de casa es esta!

---Se intentará, señores, se intentará---dijo Collado, rascándose la
frente.---Otras cosas ha habido más difíciles.

---Mucho más fácil sería dar en tierra con Villamil; ¿no es verdad,
Sr.~Pedro?

---Ese tiene su pasaporte colgado de un pelo, como la espada de
Demóstenes ---afirmó socarronamente el aguador.

---De Damocles, querrá Vd. decir---indicó Alagón.---Pues es preciso
romper ese cabello; ¿me entiende Vd., Sr.~Collado?

---Ya, ya, se hará---murmuró el ex-aguador, dándose importancia.---Yo
creo que Su Majestad tiene razón, señor duque. Estamos abusando, estamos
abusando de su mucha bondad. Verdad es que si algo hacemos, muévenos el
gran cariño que le tenemos todos.

---¡Abusar!---exclamó el duque con desabrimiento.---Por mi parte hace
tiempo que estoy casi en desgracia. Recibo muy pocos favores.

---¡Hombre de Dios, y todavía se queja!---gruñó Collado, con cierto
enojo.---¡Después que a cambio de las condenadas bandoleras, se ha
llevado la mitad de los beneficios, de las prebendas, de las raciones,
de las abadías, de las capellanías, de las colecturías, de las
examinadurías sinodales, de las definidurías de la Santa Iglesia! Y
todavía pide más. ¿Qué es lo que pide la mona? piñones mondados.

---Ya ve Vd\ldots---repuso el prócer con mal humor.---No he podido
conseguir la canonjía de Murcia, que es para mí de gran empeño\ldots{}
Pero no cedo; esta noche misma hablaré de ello a Su Majestad\ldots{}
Veremos si cuento con Artieda, hombre de gran poder en la provisión de
piezas eclesiásticas.

---Artieda---repuso Chamorro,---trae entre manos una moratoria que
solicitan las señoras de Porreño.

---¿Y se la concederán?---pregunté sin mostrar interés.

---Creo que sí. Viene recomendada por una cáfila de reverendos.

---Si es cosa de Artieda---añadió el duque,---la doy por ganada. Ese
endiablado guarda-ropas, con su aire mortecino y su cabeza caída como
higo maduro, vale más que pesa.

---Fue criado de la casa de Porreño---dijo Collado con distracción,
arrojando la cola del cigarro.

---¡Pobre Sr.~de Grijalva!---exclamó Alagón.---Buen chasco se lleva, si
las de Porreño consiguen la moratoria.

---Por cierto que soy amigo de Grijalva---manifestó Chamorro,---y ha
venido esta mañana a solicitar mi favor para que pongan en libertad a su
hijo.

---Un mal criado niño, que en los cafés ha calumniado al mejor de los
Reyes y al más generoso de los hombres---dije.

---¡Calaveradas!---balbució el duque.---Y usted, Sr.~Collado, ¿aboga por
Gasparito?

---Sí señor---repuso el ayuda de cámara.---Tengo empeño en ello, y creo
que no me será difícil\ldots{}

---Si es Vd. omnipotente\ldots{}

Collado se levantó.

---Repito mi proposición---le dijo el duque, agarrándole por la solapa
de la librea.---Doy dos bandoleras.

---No.

---Tres.

---No\ldots{} he dicho que no.

---¿Pero se va Vd.?

De repente callaron ambos, porque se abrió la puerta, y apareciendo en
ella un lacayo, gritó:

---¡Sr.~Collado, la campanilla!

Chamorro corrió fuera de la habitación con la rapidez de un gato.

---Ha llamado---dijo el duque sentándose.---Sr.~de Pipaón hablemos.

\hypertarget{xii}{%
\chapter{XII}\label{xii}}

¡El duque!\ldots{} ¡Oh!, no puedo escribir una palabra más sin hablar
del duque largamente, para que se conozca a uno de los personajes más
extraordinarios de aquella eminente y nunca bien ponderada corte.

¿Quién no hablaba entonces del duque aunque sólo fuera para referir sus
antecedentes y contarle los pasos todos de su rápido encumbramiento,
pues fue hombre que en cuatro años pasó de la nada de \emph{Paquito
Córdoba} al Ducado de Alagón con grandeza de España, toisón de oro,
grandes cruces, y el mando de la guardia de la Real persona? Era espejo
de los libertinos de buena cepa, cabeza de los cortesanos y hombre de
sutiles trazas para zurcir y descoser voluntades palaciegas.

Gozaba el privilegio de una buena presencia, aunque se le iba gastando,
porque nada es menos duradero que la hermosura, y el duque con sus
cuarenta y cinco años a la espalda principiaba a ser una muestra
gloriosa, una sombra de grandezas pasadas. Su trato y sus modales eran
finos; su conversación poco agradable en lo que no fuese del dominio de
la intriga, porque no eran muchas sus humanidades. Verdad es que maldita
la falta que esto hacía a un señorón de sus condiciones, y que no había
de ponerse a maestro de escuela. Bastábale y aun le sobraba para realzar
su nobleza nativa y la posición conquistada un conocimiento profundo de
todas las suertes del toreo, desde las más antiguas hasta las más
modernas, picando en esto casi tan alto como Pedro Romero, a quien por
entonces le empezaba a despuntar sobre el coleto la borla de doctor y el
birrete de maestro de las aulas de Sevilla. Paquito Córdoba era además
en cuestión de caballos un centauro, es decir, tan buen caballero que
con el caballo se confundía. ¡Qué ojo el suyo para adivinar las buenas y
malas prendas de sangre sin más que ver el pelaje de aquellos nobles
brutos! ¡Qué mano la suya para entrar en razón al más díscolo, para
quitar resabios y dar aplomo al ligero, gracia y desenvoltura al pesado,
formalidad al querencioso!

No se crea por esto que el duque era aficionado a la guerra. El ruido le
daba dolor de cabeza, y además ¿para qué se había de molestar, cuando
había tantos que por un sueldo mezquino peleaban y morían por la patria?
Militar era el personaje que describo, y bien lo probaba su noble pecho
lleno de cuanto Dios crió en materia de cruces, cintas y galones\ldots{}
Y no se hable de improvisaciones y ascensos de golpe y porrazo; que
hasta los nueve años no tuvo mi niño su real despacho, merced a los
\emph{méritos contraídos por su madre como dama de honor}. A los once ya
le lucían sobre los hombros dos charreteras como dos soles, sin omitir
el sueldo que no era mucho para el trabajo ímprobo de ir todos los meses
a presentarse a la revista. A los veinte pescó la encomienda de
Santiago, y luego fueron cayéndole los grados, no atropelladamente y sin
motivo como los cazan estos que se elevan por el favor y la torpe
intriga, sino despacito y en solemnidades nacionales como un besamanos,
el parto de una reina, los días del Rey y otras fiestas de gran regocijo
público y privado. Bien ganados se los tenía, pues reinando Godoy, no
costaba pocas cortesías, mimos, genuflexiones y artimañas el coger un
grado en aquella inmensa Babel de los salones de la casa de Ministerios,
donde se chocaban unas contra otras, produciendo mareo y rumor
indefinible, grandes oleadas de pretendientes de ambos sexos.

Nombrole Fernando capitán de su guardia en 1814, cargo que desempeñaba a
pedir de boca. Daba gusto ver aquella guardia. Paquito la puso en tan
buen pie, que no parecía sino cosa de teatro. Verdad es que se gastaban
en el equipo de aquellos hombres sumas colosales, de las cuales nunca se
dio al Tesoro, ni había para qué, la correspondiente cuenta y razón.
Carecían de límite los dineros asignados a tan importante fin, y en ley
de tal, el duque iba pidiendo, pidiendo, y el Tesoro dando, dando; pero
como era para mayor esplendor de la corona, los ministros no decían
nada. Acontecía que muchas veces los oficiales del ejército de línea no
veían una paga en diez meses; pero ¡qué demonio!, no se podía atender a
todo, y eso de que cualquier bicho nacido, hasta los oficiales en activo
servicio, dé en la manía de estar siempre piando piando por dinero, es
cosa que aburre y mortifica a los más sabios gobernantes.

No sé cómo les aguantaban. Especialmente los marinos a quienes se debía
la bicoca de \emph{setenta} pagas, no dejaban pasar un año sin
importunar al Gobierno con ridículos memoriales que destilaban lágrimas.
Harto hizo Su Majestad, permitiéndoles consagrarse a la pesca, oficio
denigrante para tan noble instituto, y no lo tolerara ciertamente el
sabio poder absoluto, si no aconteciera que un oficial que había estado
en Trafalgar se murió de hambre en el Ferrol, y que otros cometieran la
villanía de ponerse a servir de criados para poder subsistir.

De seguro que los guardias de la real persona y su capitán el duque de
Alagón no se quejaban de falta de pagas, pues este las recibía
puntualmente, con la añadidura de mil valiosos regalillos que el Rey por
cualquier motivo le hacía. Los hombres que se hallan en posición tan
elevada no deben sufrir denigrantes escaseces; que eso sería deslustrar
el brillo del absolutismo, y rebajar la dignidad de todo el reino; y
como Paquito Córdoba no había heredado de sus padres cosa mayor, Su
Majestad le hizo cesión, a él y a otros individuos, de una parte del
territorio de las Floridas, que no era ningún barbecho. No bastando
esto, concediósele también el privilegio de introducir harinas en la
isla de Cuba con bandera extranjera, el cual derecho era una minita de
oro. Para explotarla, Alagón tenía por socio a un barón de Colly, de
quien no se sabía si era irlandés o francés; aventurero, arbitrista,
proyectista, hombre incalificable que años atrás había intentado sacar
de Valencey al príncipe cautivo y traerle a España.

Murmuraban muchos del privilegio de las harinas\ldots{} que es muy común
eso de no ver con buenos ojos al prójimo que saca el pie de la miseria.
¡Válgame Dios! ¿Por qué no se había de permitir al duque que se
redondeara? Pues qué, ¿no es muy conveniente para la república que
abunden en ella los hombres ricos? ¿Y por qué no había de serlo el
duque, cuando con ello no perjudicaba más que a los tunantes labradores
de toda Castilla, hombres ambiciosos, tan comidos de envidia como de
miseria, y que todo lo querían para sí?

La amistad del duque y el soberano era íntima. Algunos decían que Alagón
era un \emph{hombre asiático}. ¡Qué vil calumnia! ¡Llamarle así porque
gustaba de servir dignamente a su amigo! Buen tonto habría sido el duque
si hubiera permitido que otro se encargara de las comisiones que él
sabía desempeñar a maravilla. Sobre que el resultado habría sido el
mismo, llevábase el provecho cualquier hidalguete de gotera o capigorrón
entrometido.

Público es y notorio que ni uno ni otro gustaban de escándalos; nada de
eso. En las recepciones públicas y audiencias privadas, amo y siervo
tenían un sistema de señales mímicas, por las cuales se telegrafiaban
cuanto había que comunicar respecto a las damas postulantes. Como
aficionado a estudiar por si las costumbres del pueblo para aliviar sus
necesidades y ver prácticamente los resultados de su gobierno
absolutísimo, Fernando salía por las noches del regio alcázar, para lo
cual, puesto de acuerdo el duque con el oficial de la guardia, eran
alejados del paso todos los soldados. ¡Qué llaneza y familiaridad en un
príncipe autócrata! ¡Qué elevación en su humildad, y cuánto se sublimaba
abatiéndose hasta tocar con sus augustos codos los harapos del
pueblo!\ldots{} Porque Rey y favorito no salían para visitar los
palacios de los grandes, ni darse tono en las principales calles y
sitios públicos, entre galas y boato, sino que callandito y sin pompa se
iban muy a menudo en la oscuridad de la noche a visitar a los pobres.

Y daban muy buenas limosnas; vaya\ldots{} Me lo contó Juana la
Naranjera.

\hypertarget{xiii}{%
\chapter{XIII}\label{xiii}}

---¿Con que le conviene a Vd.---me dijo el Duque afectuosamente,---la
Real Caja de Amortización?

---Si el mejor servicio del Rey me lleva a esa
dirección---repuse,---¿por qué no?

---Ya convine con D. Agapito Ugarte, que es Vd. el único hombre a
propósito para tal puesto.

---Gracias, muchísimas gracias, señor duque. Es Usted tan
bondadoso\ldots{} Sí, D. Antonio tiene mucho empeño en que yo dirija la
Caja de Amortización. Esa serie de juros de 1803, que andan por ahí, sin
que nadie los quiera, necesitan una mano cariñosa que les dé colocación
con preferencia a los que ahora tienen el turno.

---Perfectamente---dijo satisfecho de mi perspicacia.---Esos pobres
juros no valen dos reales hoy; pero para todo hay remedio\ldots{}

---Para todo, señor duque.

---Los únicos poseedores de ese papel somos Ugarte, yo\ldots{} y otra
persona.

---Comprendido.

---Hicimos la tontería de adquirirlos al dos\ldots{}

---¡Oh!, no me cuente Vuecencia la historia. Si fui yo el encargado de
comprarlos. Se compraron con intención de asimilarlos a los demás juros.
D. Antonio y yo hemos hablado largamente del asunto, y es cosa
arreglada, habiendo una mano enérgica en la administración.

---Muy bien---dijo Su Excelencia regocijado de mis procedimientos
ejecutivos.---Pero harto sabe Vd., Pipaón, que esa mano enérgica (ya
hemos convenido que será la de Vd.), que esa mano enérgica, repito, no
podrá extender sus dedos de hierro, mientras sea ministro de Hacienda el
Sr.~D. Juan Pérez Villamil.

---Por de contado. Mas en Madrid todos dan por muerto a Villamil.

---De eso se trata---afirmó preocupado.---Pero no es tan fácil como
parece, por más que diga el Sr.~Collado\ldots{} ya Vd. le oyó\ldots{}
Villamil está apoyado por Ceballos, el cual tiene muy buenos asideros.

---Mas es tan deplorable la política de este señor, que no sería difícil
dar con él en tierra\ldots{} digo, me parece a mí.

---Vaya si es deplorable. Todo el reino está alarmado ante las amenazas
de los liberales---dijo el duque mostrando mucho su celo por el bien
público.---Las conspiraciones crecen.

---Y cómo no han de crecer, si ha desaparecido el coco de las comisiones
de Estado, si hasta se han prohibido las denominaciones de
\emph{liberales y serviles}; si se ha mandado que en el término de seis
meses queden falladas todas las causas por opiniones políticas.

---Así no hay gobierno posible; es lo que yo digo. Así volvemos a los
tumultos de la Constitución, al democratismo, al desorden de los papeles
periódicos, de los clubs y de los cafés discursantes.

---Y se conspira, se conspira. Ya se lo demostraremos a Su Majestad.

---Si es inconcebible que no lo comprenda. ¡Qué falta nos hace ahora el
bailío Tattischief! Ya podía haber dejado su viaje a París para mejor
ocasión. ¿Y el Sr.~de Ugarte cuándo viene de Guadalajara?

---De mañana a pasado. Por no poder hacerlo hoy me escribió para que, de
acuerdo con Vuecencia, estuviese a la mira del sucesor de Villamil en
caso de que éste caiga.

---¡Oh!, no hay duda en eso---afirmó el duque con resolución.---El nuevo
ministro de Hacienda será D. Felipe González Vallejo.

---Así lo espera D. Antonio.

---Y así será. Si es el candidato del infante D. Antonio, que hace
tiempo bebe los vientos por darle la cartera\ldots{}

---Y en verdad, no hay hombre más a propósito---indiqué yo.---Vallejo no
será tan reglamentario como ese testarudo alcalde de Móstoles, que no
perdona un número ni una letra, y abruma a todos los empleados con su
nimiedad escrupulosa. De todo quiere enterarse, y ha de meter su hocico
en los asuntos más insignificantes.

---¡Una calamidad!---exclamó Alagón con cierta somnolencia,
arrellanándose en su sillón.---Dicen por ahí que Vallejo no sirve para
el ministerio de Hacienda, porque ha derrochado su fortuna y la de su
mujer.

---Y que administró detestablemente la fábrica de paños de Guadalajara.

---Y que es un ignorante aturdido. Digan lo que quieran, para ser
ministro de Hacienda no se necesita ser una lumbrera, ¿no es verdad,
Pipaón? Cobrar lo que le dan, entregar lo que le piden\ldots{} Cuando no
lo hay, ellos no lo han de sacar de las piedras\ldots{}

---Y para echar contribuciones no se necesita ser un Séneca; ¿no es
verdad, señor duque?\ldots{}

---Si al menos lograran satisfacer las atenciones más sagradas\ldots{}
pero es calamitoso lo que pasa. El Tesoro privativo del Rey, aquel del
que libremente y a su antojo dispone Su Majestad, no toma del Tesoro
público todo lo que debiera tomar, porque las arcas están casi siempre
vacías. Verdad es que los directores de loterías y otros empleados de
Hacienda regalan a Su Majestad, bajo el pretexto de ahorros, grandes
sumas, que si no\ldots{}

---Aun así, este año van depositados en el Banco de Londres algunos
milloncejos ---dije con malicia.

---Poca cosa\ldots---repuso con desdén el duque.---Gracias a que Su
Majestad vive hoy con mucha economía\ldots{} Ya sabe Vd. que ha
dispuesto suprimir el regalo que antes se hacía a la servidumbre a fin
de año.

---Sí, toda la ropa blanca usada por las reales personas.

---Además ha suprimido mil inútiles despilfarros, porque el reino está
agobiado de contribuciones, el Tesoro público vacío\ldots{} Yo calculo
que Su Majestad, arreglándose a la mayor sobriedad posible, no habrá
gastado en el año que acaba de transcurrir, arriba de ciento veinte
millones.

---El año que viene será más. ¿No ha oído Vuecencia hablar de boda?

---No conozco más que los proyectos de Ugarte y de Tattischief\ldots{}
¡Una princesa rusa!\ldots---indicó meditabundo.---Dudo mucho que eso se
realice\ldots{} ¿Ha dicho Vd. que D. Antonio viene?\ldots{}

---Mañana o pasado.

---Si lográsemos despachar el asunto de Villamil, ya podría pensarse
después en lo de la princesa rusa.

---El asunto de Villamil---dije yo en el tono más lisonjero que me fue
posible,---me parece resuelto, desde que hombres tan poderosos han
puesto su mano en él. Por mi parte, en la Real Caja de Amortización
estaré a las órdenes de Vuecencia.

---Gracias, Pipaón---me dijo con benevolencia suma.---Ya sabe Vd. que si
el asunto fuera de interés mío exclusivamente, no lo tomaría tan a
pechos; pero alguna persona muy superior a nosotros desea que esto se
arregle.

---Comprendo\ldots{} La monarquía absoluta tiene gastos inmensos\ldots{}
Todo es poco para ella.

---También necesita atender a todo, señor mío---afirmó sentenciosamente.

---Por eso me congratulo en extremo---añadí humillando la frente,---de
contribuir con mis cortas fuerzas a este concierto admirable, sin que en
la humilde sumisión mía haya el menor asomo de interés\ldots{} pero ni
el menor asomo de interés. Nada pido, señor duque.

Diciendo esto, me levanté para marcharme.

---Usted no necesita pedir para obtener---replicó.---Tan grande es su
mérito y la solicitud que manifiesta en el buen servicio del Rey, y del
reino\ldots{} ¿No se le antoja a Vd. nada en estos días?\ldots{}

---No, nada\ldots{} Lo que es por ahora\ldots---dije vagamente, como
quien recuerda.

---¿Nada en que yo pueda servirle?---repitió levantándose también.

---Ahora recuerdo, señor duque\ldots{} una bicoca\ldots{} Tenía empeño
en\ldots{} Puesto que Vuecencia se empeña, voy a pedir dos favores, dos
favorcillos nada más.

---¿Dos nada más?

---Dos. He oído hablar hace poco de una moratoria\ldots{}

---Solicitada por la hermana del difunto marqués de Porreño. ¿Desea Vd.
que se conceda?

---Al contrario, deseo, mejor dicho, tengo mucho interés en que no se
conceda.

---Ese asunto lo trae en su cartera Artieda, guardarropa de Su Majestad.
Es muchacho hipócrita, pedigüeño, y que, como tal, sabe sacar mendrugo.
Es muy posible, muy posible, señor de Pipaón, que consiga la moratoria.
En fin, yo veré.

---Haga Vuecencia lo que pueda, que yo por mi parte, si voy estas noches
a la tertulia, veré cómo me las compongo con el Sr.~Artieda.

---¿Y el otro favor?

---Es relativo al hijo de D. Alonso de Grijalva.

---Ya\ldots{} es Vd. su amigo. ¡Hombre generoso! ¿Quiere Vd. que se deje
en paz al muchacho y se le ponga en libertad?

---Al contrario; deseo que siga en la prisión.

---¡Hola, hola!\ldots{} Por lo visto, Vd. protege el bolsillo de
Grijalva, pero no apadrina las calaveradas de Gasparito\ldots{} Buen
propósito; me parece un excelente sistema. Aquí vislumbro todo un plan
de moralidad perfecta.

---Me desvivo por arreglar a una familia perturbada. ¿Seré ayudado en mi
noble tarea por Vuecencia?

---Eso es más fácil. Un preso más, un viajero más a tomar los aires de
Ceuta.

---No, es que no quiero enviarle tan lejos. ¿A qué esa crueldad?
Tengámosle en la cárcel de la Corona hasta que madure.

---¿Hasta que el joven madure?\ldots{} Bien: por mi parte, haré lo que
pueda.

---Señor duque, las promesas vagas de Vuecencia son para mí concesiones,
y sus esperanzas realidades. Cuento con Vuecencia. Adiós.

---Adiós, Pipaón, que no deje Vd. de venir una de estas noches\ldots{}
Agrada Vd., agrada usted mucho\ldots{} Se celebran sus chascarrillos y
su gracejo para contar las cosas.

---Vendré, vendré. Hasta luego, señor duque.

---Abur.

\hypertarget{xiv}{%
\chapter{XIV}\label{xiv}}

Dirigime a casa de las señoras de Porreño, y hallé a doña María de la
Paz muy gozosa por el buen giro y excelente aspecto que iba tomando su
asunto. Acababa de salir de la casa el Sr.~de Artieda, quien dio tales
esperanzas y presentó la cuestión en tan buen pie para marchar a un
feliz éxito, que ya se consideraba ganada la partida. Artieda, y dos o
tres señores de la clerecía con el gobernador del Consejo, habían tomado
a su cargo el negocio, siendo evidente que con tales pilotos (frase de
doña María), el barco de la moratoria, combatido por los aquilones de la
envidia, no podía menos de llegar a puerto seguro.

Yo dije a la señora que acababa de hablar en pro de su pretensión a
varias personas de mucha raíz en la corte, lo cual me agradeció mucho.
Añadí que estuviera tranquila, pues yo tomaba el negocio como mío, y no
pararía hasta conseguirlo, empresa no difícil para un hombre que, a más
de tener tantas relaciones, escupía en corro con los señores del
Consejo. Después hícele una explicación detallada de lo que eran las
moratorias, enumerando las cuatro clases de ellas, a saber: \emph{cesión
de bienes, pleito u ocurrencia, espera o moratoria, y quita de
acreedores}, asentando que la que nos ocupaba pertenecía a la tercera
categoría, por ser concesión graciosa del príncipe; y aunque el
Consejo---dije con escrupulosidad curialesca,---rinda tributo a la
majestad de las leyes, dictando el auto de \emph{traslado al acreedor},
y luego el de \emph{pase a justicia}, todo será cuestión de fórmula,
resultando al cabo que el Sr.~de Grijalva no tendrá más remedio que
conformarse y tragar el auto final de \emph{no se moleste a la parte por
tantos o cuantos años}.

Esta explicación y los pomposos encarecimientos de mi poderío, fueron
causa de que las tres damas me obsequiaran con inusitado esplendor,
brindándome dulces de los mejores y vino de las tierras de Porreño.
Gustome el licor, y tomando pie de él y de su aromática finura,
conferenciamos acerca de aquellas tierras, yo pidiéndoles informes y
dándomelos las señoras con tanta ufanía como verbosidad.

A este punto entró la señora condesa de Rumblar con su linda hija, y
retirándose adentro después las señoras mayores y doña Paulita, que iba
a la tarea de sus devociones, nos quedamos solos Presentacioncita, doña
Salomé y yo.

---¿No repara Vd. que estoy muy alegre, Pipaón?---dijo la graciosa
muchacha.

---Sí, señora, lo había notado---respondí dando el último adiós al vino
y dulces con que acababan de obsequiarme.---Eso prueba que el tiempo es
la gran medicina de las enfermedades del corazón y del espíritu. Dígolo
porque hace ya algunos días que mi Sr.~D. Gasparito está a la sombra
(sin que hayan valido mis generosos esfuerzos por sacarle), y el
sustillo ha ido pasando, y con el sustillo la congojilla, y con la
congojilla ansiosa, las lágrimas dulces\ldots{} ¡Oh! ¡Dichoso el
prisionero cuyas rejas son regadas con el divino licor de esos ojos!

---D. Juan, D. Juan\ldots{} que se pone Vd. feo diciendo esas
cosas\ldots{} Si no lloro, si no estoy triste, si no hay ya nada de
congojas, ni suspirillos---exclamó con tan franco y seductor arranque de
alegría, que me desconcerté completamente.

---¿Pues qué, señora doña Presentacioncita?\ldots{}

---Si se ha escapado.

---¡Se ha escapado!---exclamé con súbita ira, dando un salto en la
silla.---¡Se ha escapado ese tunante! ¿Cuándo? ¿Cómo? ¡Qué carceleros,
santo Dios, qué carceleros!\ldots{} ¡Luego quieren que haya justicia en
España!

---¿Pero lo siente Vd.?

---¡Escaparse! Después de haber hablado en público de las cartas de Su
Majestad a Napoleón\ldots{}

---Más vale así. Se ahorra Vd. el trabajo\ldots{}

---No, no señora---dije procurando dominarme.---No, yo quería que fuese
puesto en libertad en toda regla, después de un \emph{sobreséase} como
un templo. De este modo estaría más seguro, y podría vivir
tranquilamente donde mejor le conviniera, mientras que habiéndose fugado
de la cárcel, le perseguirán, le cogerán de nuevo, y entonces sí que
será ahorcado.

---¡Ahorcado!---gritó con ira.---¡Ay! Me asusta Vd. Yo estaba contenta y
Vd. ha venido a afligirme otra vez.

---¿Sabe Vd. dónde está?

---Lo sé, sí señor. De eso iba a tratar cuando Vd. me ha puesto en
ascuas.

---¿Dónde, dónde?

---Despacio. No está en casa de su padre, al cual ha desagradado con su
escapatoria, por el temor de que se le persiga más.

---Es claro.

---Gasparito se ha refugiado en una casa humilde, muy humilde, desde la
cual me ha escrito, contándome todo. ¡Ay!, qué dolor tan grande---añadió
dando un suspiro.---Está muerto de hambre y lleno de inquietudes, por
miedo a que le denuncien los amos de la casa.

---Y harán perfectamente. Bien merecido le estará a ese jovenzuelo
imprudente su última calaverada y el no haberse estado quietecito en la
cárcel, esperando a que yo le sacara.

---Sea lo que quiera---dijo la niña en tono de mujer seria,---es preciso
sacarle de la terrible situación en que está.

---¡Sacarle!, y ¿cómo?

---Yo tenía un proyecto---indicó sonriendo con toda su gracia
exquisita,---un proyectillo\ldots{} y contaba con Vd., sí, señor, con
Vd., para que me ayudara.

---¡Conmigo!

---Con el hombre generoso y bueno, con el corazón de oro, con la
inteligencia sublime, con la voluntad firme, con Pipaón en fin.

---Eso es, Pipaón sirve para los apuros, para los peligros; pero en
tiempo de bonanza, Pipaón es un pobre hombre que no sirve sino para
burlas.

---Si vamos ahora a disputar sobre esto, no tendremos tiempo de
ocuparnos de lo otro---dijo con impaciencia.

---Veamos lo otro: siempre será otra\ldots{} bromita.

---Pipaón---añadió con voz meliflua, y poniendo en sus ojos un abreviado
paraíso de dulzura, de hechizo y de seducción.---Yo tengo un proyecto,
en el cual me ha de ayudar Vd\ldots{} Yo quiero ir esta noche a llevar
algún socorro a Gaspar, y cuento con que me acompañe, con que me lleve
Vd.

---¡Esta noche!\ldots{} ¡Los dos!---exclamé absorto, sin saber si
negarme o aceptar.

---¡Esta noche!\ldots{} ¡Solitos!\ldots{} mejor dicho, con doña Salomé,
que también quiere ir porque también quiere dar ella algún auxilio al
pobre muchacho.

La ilustre y ya marchita dama, que hasta entonces no había desplegado
sus labios, me miró con cierto vislumbrillo de enojo, y dijo:

---Si el Sr.~D. Juan no quiere ir con nosotras, no faltará un galán
cortés y fino que nos acompañe.

---¿Acaso he dicho yo algo, señoras?---repuse humildemente, considerando
que la expedición era muy conveniente para mí por todos los
conceptos.---Vamos a donde Vds. quieran, aunque sea al fin del mundo.

---No es tan lejos---dijo Presentación,---aunque por ahora no se le
revelará a Vd. la calle ni la casa.

---Yendo conmigo, la condesa dejará salir a Presentación. Salimos al
oscurecer ---afirmó doña Salomé, revelando en su rostro de tafetán el
deleite que aquellos livianos pensamientos de escapatoria le
causaban.---Decimos que vamos a la novena del Ángel de la Guarda, y que
a la vuelta subimos un ratito a casa de la marquesa, que ha dado a luz
dos niñas de un parto.

---Y luego que veamos al pobre Gasparito y le consolemos y le demos
algún socorro ---añadió la muchacha,---le sacaremos de allí, y como no
hay lugar más seguro que la vivienda de un cortesano del despotismo, D.
Juan se lo llevará a su casa.

---¡A mi casa! ¡Llevar a mi casa a un prófugo, a un reo de lesa
majestad!\ldots{}

---Vamos, amigo---dijo la niña con donaire, plantándome su divina
manecita en el hombro,---no nos venga Vd. aquí con palabrotas. Aquí no
hay delito ni majestades. Si Vd. no le lleva a su casa, si Vd. no le
esconde, reñiremos para siempre. No me mire Vd., no me hable, no se
ponga donde yo le vea.

Como prometer no era cumplir, ni la aquiescencia verbal equivalía a
positivas concesiones de mi parte, prometí cuanto me pidieron y convine
en todo lo que tuvieron a bien proponerme, con reserva de hacer después
lo que me pareciera más conforme a la justicia, al bien del Estado y a
mi propio sagrado interés.

~

Y para no cansar, aquí me tienen Vds. embozado en mi pañosa, con el
sombrero hasta las cejas (si bien la oscuridad de la noche y el
macilento alumbrado de la villa ahorraban precauciones), llevando una
madama pendiente de cada brazo, como en los buenos tiempos de
cuchilladas y amoríos, pasando de calle a callejón y de callejón a
plazuela, ora de prisa para huir de un grupo de curiosos, ora despacio
para recrearnos con el majo cantar que por las rejas de una casa humilde
salía a veces callados los tres, a ratos hablando y riendo, regocijadas
ellas de la libertad que gozaban, mientras las severas matronas nos
suponían carcomidos de devoción en la novena del bendito Arcángel.

A mí me gustaba también el paseo, porque eso de llevar dos damas, una a
cada costado, en la oscuridad de la noche y en un pueblo como Madrid,
donde se abren tantas puertas al aventurero amor y a los locos deseos,
no es cosa de despreciar. Yo oprimía con el vivo apetito del contacto el
brazo de la de Rumblar, dejando el de la otra en libertad de que juntara
o no su flaqueza con la del mío.

---¿Pero llegamos o no?---pregunté a la muchacha.

---Ya pronto. ¿Es esta la calle del Águila?

---La del Águila es.

---Bueno\ldots{} ahora a la del Rosario.

---Pues a la del Rosario. Supongo que no será para rezarlo. Parece
mentira que en una casa que lleva ese nombre tan devoto se esconda un
reo de lesa majestad.

Presentacioncita me clavó sus dedos en el brazo con tanta fuerza, que
lancé un grito.

---Por infame y deslenguado---dijo ella.

Al entrar en la mencionada calle, doña Salomé preguntó, señalando una
casa:

---¿No es por aquí?

---Aquí---dijo Presentación, señalando la inmediata y acompañando su
ademán de amoroso suspiro.---Creo que es el número 4\ldots{}

---El 4 es. ¿Llamamos?

Llamé a la puerta, no sin cierta zozobra de que algún bárbaro malsín
apareciera y me solfease de lo lindo. Según habíamos convenido, pregunté
a la mujer que franqueó la puerta si vivía en aquellos aposentos un
joven llamado D. Federico, el cual había venido poco ha de Toledo.
Díjonos la mujer con muy malos modos que el joven se había marchado de
aquella honrada casa para ir a otra de la calle del Bastero, número 6,
donde de seguro le encontraríamos, porque andaba muy tapujado y no salía
a la calle.

Fuimos a la del Bastero, y en su número 6 nos detuvimos para decidir qué
resolución se tomaría, porque no era prudente arriesgarse en aventuras
por tales sitios. Yo estaba ya arrepentido de haber metido mis manos en
aquel peligroso fregado, mayormente cuando oí rumor de pendencias en la
inmediata calle del Carnero.

---¿Qué hacemos?---pregunté a la decidida Presentacioncita.

---Llamar.

Doña Salomé, que participaba de mis temores, dijo:

---Es demasiado tarde y esto está muy lejos. Me arrepiento de haber
venido aquí. Soy de opinión que nos retiremos.

---Llame Vd., Pipaón, y pregunte---ordenó la joven.

En el piso bajo había una taberna, lo que me pareció de malísimo
augurio, y las voces y juramentos que de ella como de un antro infernal
brotaban, ponían miedo en el más esforzado corazón. Pero no hubo más
remedio; llamé y hecha mi pregunta salió un portero rufián, el cual con
muchísima sandunga nos dijo que entrásemos y que si no el doncel buscado
(de quien no podía asegurar estuviese en la casa), había otros muchos,
que recibirían bien a las madamas.

A regañadientes entré yo, empujado más que conducido por la amante
doncella, y bien pronto nos hallamos en un patio de esos que sirven de
centro a una casa de Tócame-Roque.

---¿En dónde nos hemos metido?---preguntó con zozobra doña Salomé.

---Eso digo yo. ¿En dónde nos hemos metido?

---¿Con que por quién preguntaban Vds.?---dijo el vejete portero, con
una sonrisa truhanesca, que me heló la sangre en las venas.---¿Por el
oficialito, por el abate, por\ldots?

---Por ninguno de esos, camarada---repuse,---porque ahora mismo nos
volvemos a la calle.

---No hagamos caso de este buen hombre---dijo con afán la
muchacha.---Subamos e iremos preguntando de puerta en puerta.

---¡Está Vd. loca! ¿Sabe Vd. qué clase de gente es la que vive en estas
casas?

---Gente muy honrada y cabal---afirmó el portero.---Una señora que fue
doncella de S. A. la infanta doña María Josefa\ldots{} un autor de
diccionarios, siete poetas, dos grabadores de retratos, un torero, uno
que fue magistrado del Crimen\ldots{}

Oíase un rumor de disputas en los pisos altos de aquella colmena, el
cual convidaba a salir cuanto antes en busca del silencio de la calle.
Cerrábanse y se abrían con estrépito las puertas, dando paso a la
claridad de las luces y al rumor de las voces, y un enjambre de
chicuelos corría por los pasillos jugando a la caballería ligera y
pesada. Dos traperos amontonaban no sé qué inmundos despojos en medio
del patio, y tres mujeres se ponían como ropa de pascuas por la
precedencia en sacar agua del pozo.

---Ábranos Vd. la puerta---dije resueltamente al Cancerbero, sacando una
moneda, con la cual pensaba ponerle de parte nuestra, si ocurría
cualquier accidente desgraciado.

Diciendo y haciendo, di algunos pasos hacia la puerta, cuando en esta
sonaron fuertes y repetidos golpes, acompañados de gran gritería y
algazara de fuera, a la que respondió al punto otra no menos discorde en
los corredores.

---¿Qué es esto, portero?

---Nada, señor---respondió con sandunga,---es la policía que viene en
busca de un señoritico lameplatos, mamón y liberal, que se nos refugió
aquí esta mañana\ldots{} Yo di parte\ldots{}

---¡Él! ¡Dios mío! ¿Dónde está?---gritó Presentación con angustia.

---Se descubrió que se había escapado de la cárcel, donde estaba por
injurias a nuestro querido Rey---añadió el portero, corriendo a abrir.

---Escondámonos\ldots{} salgamos de aquí---exclamó doña Salomé,
agarrándome el brazo y tirando de mí.

---¿Pero por dónde? Vamos a tropezar con la policía.

---Escondámonos.

---Adelante.

---Subamos.

---Bajemos.

---Busquemos otra salida. Si nos ven\ldots{}

---Señoras, no somos criminales---dije procurando sosegarlas.---Si la
policía nos ve, nos verá. ¿Qué importa?

Diciéndolo, vi que entraban hasta media docena de alguaciles, asistidos
de otros tantos soldados, y tras ellos una multitud de personas del bajo
pueblo, todos los que a la sazón bullían en la taberna, muchas mujeres
de la vecindad y el contingente completo de la chiquillería de la calle.
Vociferaban, gruñían, chillaban y reían en bestial coro.

Una aprehensión en aquellos tiempos no era gran novedad, pero por viejo
y gastado que el asunto fuese, siempre tenía irresistibles encantos para
el pueblo, que estaba muy soliviantado entonces y enfurecido contra todo
lo que a liberal o afrancesado trascendiera.

---¡Le van a matar!---murmuró entre sollozos Presentación, llorando sin
consuelo.

---Veamos si podemos escabullirnos---dije yo.

---No\ldots{} no---gritó la afligida muchacha.---Veamos si le podemos
salvar. Pipaón, diga usted que es un consejero de Castilla, un ministro;
que es amigo de los señores obispos, del Nuncio, del Rey.

---Chitón\ldots{} No se gastan bromas con esta gente.

---Yo quiero subir, yo quiero hablar a la policía---exclamó, alzando la
voz con desesperación.---Vds. no tienen alma\ldots{} yo estoy loca.
¡Socorro!

Maldita la gracia que me hacía aquella situación, que empezó a ser
apuradísima desde que la dolorida muchacha puso el grito en el cielo,
atenta sólo a su amorosa aflicción, y sin hacer caso de lo demás. No sé
en qué hubiera parado trance tan amargo, si el agudísimo y tunante
portero, conociendo al vuelo el apuro en que yo estaba, no viniera en
nuestro auxilio, cuando ya la gente de la vecindad nos rodeaba, nos
observaba, señalándonos como a tres entes extrañísimos en aquel sitio.

---Vengan usías por aquí---dijo el vejete, llevándonos al fondo del
patio.---Pues no se puede salir, entren en mi cuarto y aguarden a que
pase esta batahola.

Mucho trabajo costó llevar a Presentacioncita al oscuro albergue del
señor portero, mas a fuerza de ruegos y prometiéndole yo que al día
siguiente haría poner al preso en libertad, se aplacó un tanto. El
portero, luego que nos puso en seguridad dentro de su aposento, nos
dijo:

---Aquí no les molestará nadie. Cerraré la puerta. Cuando la policía se
lleve al barbilindo y se despeje el patio, y se tranquilice la vecindad,
saldrán Vds. Esto no es un palacio; pero aquí estarán las señoras como
en su casa\ldots{} Pueden sentarse\ldots{} hay silla y media\ldots{} Mi
cama es blanda y sobre este trombón (porque yo soy músico)\ldots{} sobre
este trombón, digo, puede sentarse una de las madamas.

---Gracias, gracias.

El miserable hablaba con diabólica truhanería. Después de ponderar las
comodidades de su alojamiento, salió, y cerrando por fuera la puerta,
nos dejó dentro de aquel sepulcro.

\hypertarget{xv}{%
\chapter{XV}\label{xv}}

Situación era aquella más crítica que la primera. Encerrados allí,
estábamos a merced de un tunante, que a juzgar por su facha y lenguaje,
no debía de ser modelo de virtudes porteriles. Los tres estábamos con
mucha congoja, y ya nos creíamos cercados de ladrones y asesinos,
aumentándose nuestro pavor con el cercano rugido del pueblo que llenaba
el patio y corredores. Presentacioncita era la menos afectada de nuestra
desdicha, porque tenía alma y corazón y sentidos fijos en los pasos de
la policía y en el subir y bajar de la inquieta gente.

Transcurrió bastante tiempo sin que cesase nuestro apuro. Yo me
desesperaba, y maldecía el instante en que neciamente consentí en la
descabellada expedición; doña Salomé rezaba para que algún santo del
cielo viniese en amparo nuestro, y Presentacioncita gemía sin hallar en
nada consuelo. Lo peor de todo era que iba siendo ya muy tarde; había
pasado la hora de la novena del Santo Ángel, habían dado las ocho, las
nueve, iban a dar las diez\ldots{} ¡horrible trance!, darían las también
las once, las doce sin poder salir de allí.

Por fin, Dios quiso que los alguaciles encontraran al prófugo y lo
sacasen fuera y se lo llevasen con dos mil demonios. Iba desocupándose
el patio, se extinguían las voces poco a poco, y al fin, ¡San Antonio
bendito!, el endiablado portero nos sacó de nuestro calabozo.

---¡Vámonos a la calle pronto!---exclamó doña Salomé, ardiendo en
impaciencia.

---¡A la calle, a la calle! ¿Por dónde se sale, buen hombre?---dije,
sosteniendo a Presentacioncita, que por su mucha aflicción apenas podía
con su lindo cuerpo.

---Si no quieren Vds. salir por la calle del Bastero, donde hay muchos
tunantes y borrachos---repuso el portero,---por este pasillo que hay a
la derecha saldrán a la casa inmediata y a la calle de Mira el Río.

Yo temblaba de susto: por todas partes, en todos los rincones veía
ladrones y asesinos, alzando horrorosos puñales sobre mi pecho. El
viejecillo nos llevó del patio grande a otro más pequeño, y de este a un
largo y húmedo zaguán, en cuyo extremo se veía la claridad de la calle.
Cuando le di la propina, me pareció sentir ruido de pasos detrás de
nosotros; pero aunque atentamente miré, nada vi.

---Por aquí derechos a la calle---dijo nuestro amparador, retirándose
repentinamente.

Dejonos solos, y a la verdad fue como si nos dejara de su santa mano el
ángel de nuestra guarda; porque no habíamos dado cuatro pasos hacia la
claridad que al extremo del zaguán se veía, cuando una voz bronca y
temerosa, que en su clueco graznido indicaba ser producto del hombre y
del aguardiente, resonó como un trueno en aquellos ámbitos oscuros,
diciendo:

---¡Alto allá\ldots{} alto! señoritos zampatortas, ¡alto, alto!\ldots{}

El reventar de un cráter no me hubiera causado más espanto. Quedeme
frío, y sobre frío absorto y petrificado, cual si en estatua de hielo me
convirtiese. Y al mismo tiempo se sentían unos pasos, unos saltos como
de gigante borracho que venía dando traspiés por la cercana escalera.

Lanzaron agudísimos gritos las damas, colgándose de mis brazos para que
yo las amparase; pero más que nadie necesitaba yo amparo y protección,
porque me quedé sin habla, sin fuerzas para correr, sin ojos para mirar,
ni orejas más que para oír la voz, ¿qué digo?, las voces de los que se
acercaban, pues, quitando lo que multiplicase mi espantada imaginación,
bien podía asegurarse que eran media docena.

No se me oculta que mi deber en tan crítico momento era tirar de la
espada o sacar las pistolas para esperar a pie firme a los ladrones y
acabar con ellos o morir antes que mis dos compañeras fueran
atropelladas; pero yo no tenía espada, y ni remotamente me acordé de que
llevaba una pistola en el cinto. Temblando como alma que llevan los
demonios, recordé aquello de que una retirada a tiempo es una gran
victoria, y apreté a correr hacia la calle. Las dos damas eran dos alas
que me impulsaban con rapidez suma. ¡Ah!, cómo corrimos, cómo corrimos
gritando, «¡favor, socorro, ladrones!»

Tras nosotros corría alguien. No le mirábamos. Sentimos carcajadas,
blasfemias, un juramento horrible, qué sé yo\ldots{} Corríamos siempre;
las dos damas se separaron de mí y se quedaron detrás. ¡Ay!, yo era el
viento mismo.

Vi dos hombres que andaban en dirección contraria a la mía y su
presencia me dio aliento\ldots{} ¡dos hombres que no eran, o al menos no
parecían ladrones ni asesinos!---¡Socorro, favor!---repetí con ahogado
aliento.

Detuviéronse ellos. Me pareció ver una cara conocida; pero en mi
azoramiento no llegué a formar juicio alguno\ldots{} Detúveme yo
también. En el mismo momento sentí un ¡ay! agudísimo. Era
Presentacioncita que había caído al suelo. Doña Salomé se había parado
en el mismo sitio. Retrocedí, porque la presencia de los dos
desconocidos me infundió algún valor y porque mirando hacia atrás
observé que nuestros perseguidores se habían quedado muy lejos.

Uno de los dos desconocidos se adelantó corriendo a levantar del suelo a
Presentacioncita, mientras el otro soltó la risa diciendo:

---Si es Pipaón.

---¡Ah! ¿Es Vd. señor duque? Hemos sido atacados por unos
tunantes\ldots{} Vamos a ver si se ha hecho daño esa niña.

El hombre que estaba junto a mí era el duque de Alagón; el otro\ldots{}

\hypertarget{xvi}{%
\chapter{XVI}\label{xvi}}

Detente pluma\ldots{} El otro alzaba del suelo a la pobre
Presentacioncita, que al perder el equilibrio, y dar con su cuerpo en
tierra, perdió también el conocimiento. Nos acercamos y el duque me miró
con fijeza y malicia poniendo sobre los labios su dedo índice.

---¡Jesús\ldots{} se ha desmayado!---balbució doña Salomé, examinando a
su amiga que aún estaba en brazos del otro.

---Esto no será nada, señora\ldots---exclamó el
desconocido.---Señorita\ldots{}

---El susto ha sido tan grande\ldots---dije yo,---y gracias a que no se
atrevieron a seguirnos. ¡Pobres señoras, si hubieran venido solas!

---¿A dónde llevamos esto?---preguntó el compañero del duque, dando
algunos pasos con la desmayada en brazos, tan sin trabajo cual si fuese
una pluma.

Pareció perplejo el duque, y como no acertara a indicar una resolución
conveniente, el compañero dijo:

---Vamos allá. Adelántate y llama.

Hízolo así Alagón, y no habíamos andado veinte pasos siguiendo todos al
generoso caballero, cuando se abrió una puerta, y Alagón primero,
después su compañero con la niña en brazos y detrás doña Salomé y yo,
penetramos en una hermosa pieza iluminada por dos luces. Un hombre y una
mujer encontrábanse allí, ambos en pie y tan respetuosos que por lo
callados y circunspectos parecían estatuas. Veíase en el fondo una
puerta entreabierta, por la cual apareció el rostro de una mujer de tan
acabada hermosura que a pesar de lo apurado del lance, no pude menos de
fijar en ella mis ojos. De la pared pendía una guitarra.

El compañero del duque depositó su preciosa carga en una silla. Callaban
todos: el desconocido pidió un vaso de agua, mientras doña Salomé,
observando que la muchacha empezaba a dar señales de vida, hacía
esfuerzos por reanimarla, diciéndole:

---Presentación, vuelve en ti. Eso no es nada\ldots{} ¿A ver? ¿Te has
hecho daño?\ldots{}

---Vamos, beba Vd. un poco de agua---dijo el desconocido, acercando el
vaso a los labios de la joven, que recobraban poco a poco su vivo
carmín, así como las descoloridas mejillas.

Cuando la muchacha bebía, observé al generoso galán, que solícitamente
sostenía con su mano izquierda la cabeza de la joven, mientras le daba
de beber con la otra. Era un hombre admirablemente formado, de cuerpo
estatuario y arrogante. Su edad no pasaría de los treinta y dos años,
hallándose, según la apariencia, en aquella plenitud de la fuerza, del
vigor y del desarrollo físico que marcan el apogeo de la vida. Vestía
sencillo y elegante traje negro por entero y ancha capa, que
habiéndosele caído en los primeros momentos del lance, fue recogida por
el duque. Sus ojos eran negros, grandes y hermosos, llenos de fuego, de
no sé qué intención terrible, flechadores y relampagueantes. Bajo sus
cejas, semejantes a pequeñas alas de cuervo, centelleaba deshecho en
ascuas mil por las movibles pupilas, el fuego de todas las pasiones
violentas. Su nariz era desenfrenadamente grande, corva y caída; una
especie de voluptuosidad, una crápula de nariz. La carne, superabundante
había crecido, representando con fértil desarrollo su preponderancia en
aquella naturaleza. El labio inferior que avanzaba hacia fuera, parecía
indicar no sé qué insaciabilidad mortificante. La personificación de la
sed habría tenido una boca así. Una línea más de desarrollo, y aquel
belfo hubiera tocado en la caricatura. Observándole bien, se veía en la
tal fisonomía, peregrina mezcla de majestad y de innobleza, de hermosura
y de ridiculez. Tenía de todo, y era difícil deslindar en aquel rostro
híbrido las líneas pertenecientes a las grandes razas de las que
pertenecían a la degeneración propia de todo lo humano. Por su mandíbula
inferior se filiaba remotamente con Carlos V, mas por sus ojos
truhanescos y las patillas cortas, se iba derecho a la majería. El
cráneo era bien conformado, el pelo negro y corto, con mechoncillos
vagabundos sobre la frente y sienes. En suma, el perfil de aquel hombre
solía verse en las onzas de oro.

Presentacioncita, abriendo los ojos, demostró tal asombro al verse en
aquel desconocido sitio y ante personas extrañas, que creímos se iba a
desmayar de nuevo.

---Ánimo---le dijo el belfo,---ánimo, señora mía, eso no es nada.

---¡Ah!\ldots{} ¿quién es Vd.? Gracias, caballero\ldots{} ¿En dónde
estoy?---balbució la muchacha.---¡Ah!, doña Salomé\ldots{} Sr.~de
Pipaón\ldots{} Están aquí\ldots{} creí que me habían abandonado.

---Aquí estamos, sí, niña querida\ldots{}

---Pero al instante nos vamos a marchar---afirmó con febril impaciencia
la de Porreño.---Presentación, prueba a levantarte.

---Señora doña Presentacioncita---dijo el belfo sonriendo,---no hay
prisa. Descanse Vd. un poco.

---Vámonos, vámonos---añadió doña Salomé.---Hija, haz un esfuerzo y
levántate. ¿Puedes andar?

Presentación dio algunos pasos: cojeaba un poco, a causa de una leve
torcedura en el pie derecho al caer; pero andaba. Volviose para dar las
gracias al incógnito caballero; yo también quise decirle algo por pura
fórmula, pero nos miramos unos a otros con sorpresa. El caballero,
volviéndonos la espalda, desapareció por la puerta que había en el
fondo.

---Gracias, muchas gracias, señores---dijo Presentación, dirigiéndose al
duque.

---Por aquí---indicó este, que sin duda deseaba que nos
marcháramos.---Yo acompañaré a Vds. hasta la calle de Toledo.

---Por aquí\ldots{} a la calle\ldots{} gracias, mil gracias señor duque.

El duque, mientras las dos mujeres salían, se me puso delante y abriendo
mucho los ojos, aplicó de nuevo el índice a los labios.

Salimos y los minutos nos parecían siglos, porque Presentacioncita
andaba muy despacio. Era ya tarde, por cuya razón a las contrariedades
expuestas se unía la pavorosa contrariedad del sermón que nos esperaba
cuando nuestras pecadoras frentes se pusieran al alcance de los ojos de
la señora condesa y nuestros oídos al blanco de la grave voz de doña
María de la Paz. Al pensar en esto, los tres no teníamos más que un
deseo: que la tierra se abriese haciéndonos el favor de tragarnos.

Pero la Providencia que nunca abandona a los débiles, nos sugirió
ingeniosísimas trazas para salir del paso, y fue que discurrimos sacar
del propio mal el remedio, achacando la tardanza a la misma torcedura
del pie de Presentacioncita, cuya invención, llevada a feliz término por
mi elocuencia ante las dos irritadas matronas, tuvo el éxito más
completo que pueda imaginarse.

---Es claro\ldots{} ¡cómo habíamos de venir a tiempo!\ldots{} Bajamos la
escalera\ldots{} Presentacioncita dio un paso en falso. Subimos otra
vez\ldots{} La Marquesa no quería dejarla salir\ldots{} Se buscó un
simón; el simón no parecía\ldots{} Se sacó la litera de mano; estaba
rota\ldots{} Discurre por aquí, discurre por allá\ldots{} Yo estaba en
ascuas y quise venir a avisar para que no se asustaran Vds\ldots{} En
fin, demos gracias a Dios de que no se rompiera un pie.

---¿No puedes andar?---preguntó la condesa a su hija con
desabrimiento.---Esta sí que es fiesta. Estamos convidadas para la
función de mañana en la Trinidad.

---Con manifiesto y asistencia de Su Majestad---repitió doña María de la
Paz.---Y es preciso ir sin remedio. Yo al menos no puedo faltar, porque
el prior nos ha prometido que podremos hablar a Su Majestad y entregarle
nuestros memoriales.

---Mañana---repetí.---También yo he recibido invitación de los padres.
¿Con que van ustedes a la Trinidad?

---¿Puedes andar, Presentación? ¿Puedes andar, sí o no?---preguntó con
afán indescriptible doña Paulita.

La niña se levantó resueltamente y dio algunos pasos por la habitación
con pie seguro.

\hypertarget{xvii}{%
\chapter{XVII}\label{xvii}}

¿Cómo había yo de faltar a la función de los Trinitarios, si era hombre
que a ninguno cedía en religiosidad ni perdonaba medio de que se me
tuviese por escrupuloso guardador de los preceptos y prácticas de la
Iglesia? Además, poco antes había sido nombrado prioste de la
archicofradía de Luz y Vela, y como tal me correspondía asistir a la
función y acudir al pórtico de la iglesia, donde habíamos puesto el
mostradorcito con varios objetos devotos y otros profanos, que al son de
trompeta y tamboril se vendían o rifaban para atender a los gastos de la
corporación.

Desde muy temprano estaba yo con mi cinta al cuello, espetado en el
pórtico, en compaña de mis colegas el señor licenciado Moñino, de la
suprema Inquisición, D. Felipe Rojo, racionero medio de Toledo y el
sub-colector de espolios, D. Vicente Barbajosa. El gentío era inmenso, y
se agolpaba en las distintas puertas del edificio, estorbando el paso de
los fieles, lo que perjudicaba mucho la venta.

En el atrio del convento estaba el zaguanete de la Guardia de la Real
persona. No tardó en aparecer Su Majestad, desplegando en su persona y
comitiva tanta pompa y aparato, que se sentía uno orgulloso de ser
español y llamarse vasallo de quien por tal modo y con tal grandeza
representaba en la tierra la autoridad emanada de Dios. Daba gusto ver
aquella fila de coches, tirados por sendos pares de caballos a tres
pares cada uno. Cada individuo de la Familia Real iba en el suyo,
resultando una procesión que cogía medio Madrid, con la multitud de
batidores, correos, lacayos, escoltas, carruajes de respeto,
palafreneros, caballerizos y demás figuras admirables que recreaban la
vista y el alma. ¡Qué profusión de uniformes, cuánto plumacho y galón,
qué diferentes clases de sombreros, de uniformes, de caras, de arreos!
Parecía que le trasportaban a uno al Oriente, o a las pomposas fiestas
de la India. ¡Feliz nación la nuestra, que tal magnificencia podía
ofrecer a los aburridos ojos de los súbditos, para que se alegraran y
diesen gracias a la Divina Providencia por haber hecho de nuestros reyes
los más rumbosos y magníficos de la tierra! Allí se veía la grandeza de
nuestra nación, allí sus inmensos tesoros, allí su dignidad excelsa,
allí la representación más admirable de su gran poderío. ¡Viva España!

Formaron los guardias (a quien entonces llamaba el vulgo los
\emph{chocolateros}, no sé por qué), y el estrépito de tambores y
clarines llenaba los aires. Tales sones y el limpio sol que inundara
aquel día las calles, daban a la regia comitiva esplendor y armonía
celestes. Los gritos de ¡viva el Rey absoluto! resonaban por doquiera.
¡Oh, feliz consorcio de la monarquía absoluta y la religión santísima!
¡Quiera el Cielo que existan luengos siglos y que estas dos
instituciones, hijas de Dios, vayan siempre de la mano y partiendo un
piñón, para que los fieles cristianos y súbditos del encantador Fernando
vivamos pacíficamente en la tierra, libres de revoluciones impías y de
locas mudanzas!

Salió la comunidad con palio a recibir al monarca, y llevándole en
procesión a un camarín riquísimo que le habían preparado en el Claustro,
rogáronle que se adornase el pecho con media docena de escapularios y
alguna reliquia milagrosa de huesecillos o retazo de santo, lo cual como
hombre piadosísimo, hizo de buena gana. El infante D. Carlos y D.
Antonio Pascual imitáronle, dirigiéndose después todos, cirio en mano, a
la vecina iglesia, donde ocuparon sus asientos en medio del respeto y la
admiración de los fieles.

Todavía me parece que le estoy mirando. No puedo olvidar aquella
majestuosa figura arrodillada, con los ojos fijos en el Santísimo
Sacramento en actitud tan edificante, que la misma impiedad se habría
ablandado y convertido contemplándole. ¡Con cuánta devoción atendía a
las sonoras preces, y con cuánta fe al sermón que predicó el padre
Vargas, y en el cual no faltó aquello de llamarle Trajano y Constantino,
y de elogiar \emph{sus sabios dictamentos para dirigir sabiamente la
nave del Estado}! ¡Con cuánta unción y evangélica mansedumbre besó las
reliquias que el padre Ximénez de Azofra le presentara, y dijo después
las oraciones finales para implorar de Su Divina Majestad la gracia y el
buen consejo! Todos los presentes estábamos conmovidos, y parecía que se
nos comunicaba algo de la celestial pureza de aquel varón insigne, ante
cuya preciosa cabeza se postraba mudo y sumiso el pueblo escogido de
Dios. ¡Oh qué gusto ser español!

Concluida la ceremonia, pasó Su Majestad al camarín, donde ya se había
dispuesto una lujosísima mesa, como destinada a boca y paladar de tal
príncipe, y en la cual las viandas más apetitosas reclamaban la vista y
olfato, recreando y extasiando el alma. No sé qué angelicales reposteros
pusieron sus manos en aquello; pero lo cierto es que la tal mesa parecía
destinada a servirse en los altos comedores del Paraíso, para regalo de
las más excelsas potestades. Aunque allí como en los claustros no tenían
entrada sino las personas convidadas, muchas damas de lo más granado de
Madrid, consejeros, generales, oficiales, marinos, presidentes y
priostes de las cofradías, capellanes de palacio, alguaciles y
familiares de la Inquisición, canónigos de San Isidro y demás sujetos de
viso, el gentío era grande, porque los trinitarios, deseosos de dar
lucimiento a la fiesta, habían abierto mucho la mano en las
invitaciones. No nos podíamos rebullir; todos querían ver los augustos
semblantes de Su Majestad y Altezas. Los frailes no cabían en su pellejo
de puro satisfechos, y trataban de atender a todo.

Su Majestad no hizo más que probar algunos platos; obsequió con dulces a
las damas, dando muestras, allí como en todas partes, de su exquisita
galantería, y se retiró a la sala capitular para despedirse de los
bondadosos y humildes padres. Pugnaban los convidados por penetrar en la
sala, llevados unos del deseo de saciar sus ojos en la contemplación del
rostro de nuestro soberano, otros aguijoneados por el afán de
presentarle memoriales. Gracias al padre Salmón, que se me apareció como
emisario del cielo, pude penetrar en la sala, llevando conmigo a la
señora condesa de Rumblar con su hija y a las señoras de Porreño. Las
cinco damas estuvieron a punto de quedarse fuera. Sensible sobre toda
ponderación hubiera sido este accidente, porque la condesa iba a
presentar al Rey un memorial pidiendo una bandolera para su hijo, y doña
María otro en pro de la tan deseada moratoria.

¡Oh!, espectáculo sublime, y qué hermoso es ver a un Rey, atendiendo con
paternal solicitud al socorro de sus hijos, recibiendo las peticiones de
estos y prometiendo satisfacerlas con generosidad, con esa generosidad
regia, que es un reflejo de la misericordia divina. Puesto Su Majestad
en un estrado que a propósito se había construido, el prior Ximénez de
Azofra le presentó un memorial, solicitando no sé qué mercedes para dos
sobrinos suyos y dos cuñaditos de su hermana; y después que el bendito
trinitario cumplió los deberes domésticos, mirando por el bien de su
venerable parentela, fue presentando al Rey uno por uno a todos los
demás postulantes, que ya habían convenido con él en los pormenores de
esta ceremonia. Recogió Fernando las peticiones con tanta bondad, que
era imposible contener las lágrimas viéndole. A todos prometía villas y
castillos, dirigía algunas preguntitas, hacía el obsequio de una
sonrisa, cuando no de palabras, y daba a besar su real mano con una
llaneza que no desmentía la dignidad. ¡Oh qué inefable delicia ser
español y súbdito de tal monarca!

Cuando Ximénez de Azofra indicó a la señora de Rumblar que se acercase,
y vio Su Majestad a la grave madre y al lindo retoño, se rió de una
manera tan franca que todos nos quedamos pasmados; y al recibir el
memorial fijó los negros ojos de fuego en Presentacioncita, la cual,
turbada, azorada, trémula, vaciló y hubiera caído en tierra si no la
sostuviéramos. Estaba la muchacha más roja que una cereza. Dirigiole el
paternal y bondadoso monarca la palabra, preguntándole si tenía padre, a
lo cual doña María, hecha un mar de lágrimas, contestó que no.

Todos nos asombramos de la inmensa bondad del Rey, que en aquella
pregunta como que quería constituirse en padre de todos los huérfanos
del reino.

Cuando nos retirábamos, Presentacioncita estaba pálida como el mármol.

---¿Le vio Vd. bien?---me dijo en voz baja.---¡Ay! Sr.~de Pipaón, estoy
asombrada, aterrada.

No pude oírla más, porque sentí que entre el gentío me ponían una mano
en la espalda.

Era el duque de Alagón, que quería hablarme a solas\ldots{} pues no
podía pasar mucho tiempo sin que él y yo tratásemos algo importante para
el bien del estado.

\hypertarget{xviii}{%
\chapter{XVIII}\label{xviii}}

A las dos del siguiente día estaba yo en Palacio. Enviome D. Antonio
Ugarte, recién llegado a Madrid, para que diestramente y con amañados
pretextos observase lo que allí pasaba. Después de hablar con varios
gentiles hombres y mayordomos, llevome uno de estos al salón que precede
a las regias estancias, y en el cual suele verse en días de audiencia
gran marejada de pretendientes que entran o salen. Presentóseme allí el
duque de Alagón, que llevándome a parte, me señaló un anciano que en el
mismo instante salía de la Cámara Real.

---¿Conoce Vd. a ese?---me dijo.

---Es D. Alonso de Grijalva---contesté sin disimular mi
disgusto.---¡Maldito vejete! No puede dudarse que ha venido a implorar
el perdón de su hijo.

---Y lo ha conseguido; yo puedo asegurarlo, porque estaba presente
durante la audiencia. ¿Creerá Vd. que el buen señor se ha echado a
llorar delante del Rey?

---¡Qué falta de cortesía!

---Su Majestad le ha recibido bien. Grijalva goza de muy buena opinión:
es realista vehemente.

---Vamos, que se ha salido con la suya.

---De una manera absoluta. Por esta vez, amigo Pipaón\ldots{} Además
vino presentado por dos personas de la primera nobleza y por el
Patriarca, y precedido por una carta del Nuncio.

---¿De modo que se nos escapó Gasparito?---dije yo, tomándolo a broma.

---Sin remedio ninguno. Su Majestad se ha mostrado tan decidido, tan
categórico\ldots{} Al despedirse, le dijo: «Puedes marcharte tranquilo a
tu casa, que mañana sin falta estará tu hijo en libertad y se sobreseerá
esa causa. Te lo prometo, te lo prometo, te lo prometo». Lo repitió tres
veces.

---¡Cómo ha de ser!\ldots{} A lo hecho pecho\ldots---dije, discurriendo
en aquel mismo instante qué nuevos medios emplearía para llevar adelante
mi plan.

Pero sacome de mis meditaciones el duque mismo llevándome de sala en
sala, hasta una en que acostumbraban a reunirse los cortesanos para
arreglar sus cuentas de favoritismo unos con otros, sopesar su
respectiva influencia y regodearse en común de ver la buena marcha de
los asuntos del gobierno.

Cuando entramos el duque y yo, había en el salón cuatro personas;
paseábanse juntos de un ángulo a otro en la diagonal de la estancia,
Pedro Collado y D. Francisco Eguía, teniente general, ministro de la
Guerra, anciano casi decrépito, aunque no privado aún de cierta
agilidad, y con una singular comezón de hablar y moverse, que era el
rasgo distintivo de su espíritu, así como la coleta y corcovilla lo eran
de su cuerpo. Formando grupo aparte, hablaban por lo bajo sentados en un
diván, D. Pedro Ceballos, ministro de Estado, y D. Baltasar Hidalgo de
Cisneros, ministro de Marina.

Detuviéronse Eguía y Collado al vernos, y el primero, que no por ser de
carácter inflexible y duro en los negocios públicos dejaba de mostrar
mucha llaneza en la conversación familiar, me dijo:

---¡Cuánto bueno por aquí! Me han dicho que va Vd. a la Caja de
Amortización\ldots{} Sea enhorabuena.

---Gracias, muchas gracias---repuse con modestia.---Bien saben todos que
no lo he solicitado.

---Bien hayan los hombres de mérito---dijo Collado.---Ellos no necesitan
de recomendaciones para subir como la espuma.

---Nos hemos propuesto darle su merecido a este tunante de
Pipaón---declaró el duque con cortesanía,---y poco a poco lo vamos
consiguiendo. Este va para ministro, Sr.~D. Francisco.

---Lo creo, lo creo---repuso el anciano alzando la abatida cabeza y
guiñando el ojo para mirarme.---Pero no le arriendo la ganancia\ldots{}
¡Santo Dios, qué laberinto, qué torre de Babel es un ministerio!

---Lo creo, Sr.~D. Francisco---dije con oficiosidad.---Pero sin su
poquito de abnegación, no se concibe al buen súbdito de Su Majestad.

---¡Oh!, es claro; nos debemos a Su Majestad\ldots{} Pero a mis años, la
enorme carga de un ministerio es insoportable\ldots{} Precisamente en
estos días la balumba de asuntos puestos al despacho me ha rendido más
que una batalla.

---Pues es preciso cuidarse, Sr.~D. Francisco.

---¿Querrá Vd. creer, Sr.~Collado---dijo el guerrero gesticulando con
desenvoltura,---que ya están despachados todos los nombramientos que Vd.
me recomendó en aquella minuta?\ldots{}

---¿Las doce comandancias de provincias, seis plazas fuertes y no sé
cuántas tenencias de resguardos?\ldots{} Pues la mitad de esas limosnas
son para el señor duque que nos está oyendo.

---Vamos---continuó D. Francisco con socarronería,---que por falta de
pedir no se les pondrá mohosa la lengua. Yo, que soy ministro, no he
podido satisfacer el deseo que ha tiempo tengo de regalar un
arciprestazgo al sobrino de mi cuñada. ¿Y por qué? Porque no me ocupo de
pedir, ni gusto de importunar por un miserable destino.

---Se tendrá en cuenta---afirmó gravemente Collado.

---Hace pocos días---continuó el general,---hablé de esto a Moyano, y me
dijo que Su Majestad se había reservado la provisión de todas las
plazas.

---No es cierto, ¡qué enredo!---expresó el ayuda de
Cámara.---¡Reservarse Su Majestad todas las plazas!

---Quien se las ha reservado---afirmó el duque, con enojo,---es el mismo
ministro, el insaciable D. Tomás Moyano, que tiene media nación por
parentela.

---¡Es gracioso!---dijo Eguía riendo.---Cuentan que ha despoblado a
Castilla; que ya no hay en Valladolid quien tome el arado, porque los
labradores todos han pasado a la secretaría de Gracia y Justicia.

¡Cuánto nos reímos a costa del ministro ausente! Yo, que no quería
perder la coyuntura de demostrar a D. Francisco Eguía la admiración que
me causaba su desmedida aptitud para los asuntos militares, dije con
gravedad:

---No me nombren a mí esos ministros que no se ocupan más que de la
provisión de destinos, de colocar parientes y despoblar aldeas para
rellenar secretarías. Tales hombres no hacen la felicidad del
reino\ldots{} Señores, no todos los ministros cumplen con su deber. Casi
puede decirse que la mayor parte van por mal camino; casi, casi, se
puede afirmar que uno solo\ldots{} y no lo digo porque esté delante don
Francisco Eguía\ldots{} Cuantos me conocen estarán hartos de oírme
asegurar que de todos los secretarios del despacho, el que con más celo
se consagra a asuntos beneficiosos y de interés general, es el que nos
está oyendo.

---Gracias, gracias---exclamó el guerrero, poniendo su guerrera mano en
mi hombro.---He hecho lo que me ordenaban mis antecedentes militares.

---La verdad es que sólo el trabajo de las nuevas ordenanzas basta a
asegurar la reputación de un ministro.

---¡Y cuánto me han dado que hacer las tales ordenanzas!---dijo D.
Francisco, con voz hueca y ponderativos ademanes.---Como que abrazaban
multitud de puntos delicados y que no era posible resolver a dos
tirones. Ha sido preciso dictar disposiciones nuevas, que no figuraban
en nuestros antiguos códigos militares. ¿Creen Vds. que es un grano de
anís? Fácil era prohibir a los soldados que cantasen las estrofas que
les guiaron al combate durante la guerra; pero ¿y la orden de rezar el
rosario en cuerpo todos los días?\ldots{} ¿y la serie de minuciosas
instrucciones sobre el modo de tomar agua bendita al entrar formados en
la iglesia? Luchábamos con el vacío que la legislación militar ofrece
hasta hoy en este punto, y hemos tenido que hacerlo todo nuevo.

---¡Es admirable!---exclamé.---Pero sírvale a Vd. de consuelo por su
trabajo, la gratitud del ejército.

---¿Qué deseo yo sino su bien?---prosiguió el venerable militar.---Sabe
Dios que me contrista en extremo el que se deban tantas pagas; pero eso
no está en mi mano remediarlo.

---Ni en la de nadie---afirmó el duque.

---Pero váyase lo uno por lo otro---dije yo.---Si no cobran, en cambio
el Sr.~D. Francisco ha decretado la construcción de un hospital de
inválidos.

---Es verdad, también tengo esa gloria. Yo he dado ese decreto, y si el
hospital no se construye, no es culpa mía.

---Ni mía---repitió maquinalmente Collado.

---A falta de pagas---añadió Eguía con juvenil complacencia,---preparo
una disposición, en virtud de la cual, cada año de campaña se cuenta
como dos de servicio, lo cual tiene la ventaja de que muchos militares
noveles y que ahora empiezan su carrera, pueden retirarse a sus casas
con una pingüe cesantía\ldots{} Vamos, no se quejarán.

---Sobre eso écheles Vd. las cruces recientemente creadas.

---Justamente---dijo D. Francisco.---Miren Vds.: no paré hasta no
conseguir el establecimiento de la \emph{Cruz de Lealtad de Valencey},
con la cual se ha premiado a los que acompañaron a Su Majestad, mientras
aquí ardía la más feroz de las guerras\ldots{} En fin, en mi ministerio
se ha trabajado. Sólo siento que mis años y achaques no me permitan
desplegar mayor actividad, y me alegraré de tener un sucesor que no
levante mano hasta poner a nuestro ejército en el pie de magnificencia
que le corresponde.

A este punto llegaba, cuando se acercaron a nosotros el ministro de
Marina y D. Pedro Ceballos.

---¿Quién va al cuarto del infante D. Antonio?---preguntó D. Baltasar
Hidalgo de Cisneros, disponiéndose a salir.

---Corra Vd., corra Vd\ldots---repuso el duque con sandunga.---Su Alteza
está muy impaciente por saber el estado de la mar.

---Barcos no tenemos---indicó maliciosamente Ceballos,---pero
almirante\ldots{}

---El Almirantazgo ha quedado constituido al fin---dijo
Cisneros,---gracias a mis esfuerzos. Por algo se empieza. Hay que tener
paciencia.

---Es claro; los barcos se harán después---apunté yo.

---Gracias a Dios---indicó Cisneros,---ya tenemos Almirantazgo.
Precisamente acaba este de tomar una determinación importante.

---¿Cuál?

---Ceder al infante los derechos que la corporación percibe. Es una
bonita renta.

---Lo que dice Pipaón---manifestó Ceballos.---Tiempo hay de hacer los
barcos. La cosa no urge.

Cisneros no habló más y se retiró. Era un viejo caduco y tristón que no
infundía ya sentimientos de afecto ni de antipatía. Había estado en el
combate de Trafalgar, mandando en la \emph{Trinidad}, como Mayor General
de Uriarte. En 1810, hallándose de virrey en Buenos-Aires fue débil, tan
débil que permitió a los rebeldes formar una junta de gobierno, con tal
que le diesen un puesto en ella. Pero los insurgentes americanos,
después que se apoderaron del gobierno y de las fuerzas navales,
despidieron ignominiosamente a Cisneros. Vuelto a España, no encontró un
patíbulo, sino la capitanía general del departamento de Cádiz, que era
un buen momio, y después el ministerio de Marina. Cisneros tenía pocos
amigos. Apenas le traté, porque su lúgubre tristeza me aburría en
extremo.

---Si Cisneros y yo seguimos en Marina y Guerra---afirmó Eguía con
petulancia,---hemos de poner a marineros y soldados, como antes dije, en
el pie de magnificencia que les corresponde.

---Mientras no se encargue de calzar ese pie de magnificencia el señor
duque que está presente\ldots---dijo Ceballos mirando con maliciosa
intención a Paquito Córdoba.---Mientras todo el ejército de mar y tierra
no vista y coma al compás de los rollizos galanes de la guardia\ldots{}
El señor duque puede comunicar al señor ministro de la Guerra su receta
para engordar soldados.

Con estas frases malignas, zahería el astuto ministro de Estado al señor
duque de Alagón. Hacía tiempo que no se miraban con buenos ojos.

---La guardia de la Real persona---dijo Paquito Córdoba,---come lo que
Su Majestad se digna darle. En ella no hay un solo individuo que haya
metido su mano en la olla del Rey José, ni en el puchero de las Cortes
de Cádiz.

Esta saeta era muy punzante para Ceballos, que desde 1808 se había
sentado a todas las mesas. No contestó el ladino cortesano a la
insinuación del duque y varió de conversación. Era Ceballos hombre
instruidísimo en diplomacia máxima y mínima; muy conocedor de las
grandes vías, así como de los callejones de la política. Reservándome
para más adelante el trazar su historia, diré aquí tan sólo, que era el
más instruido de los que allí estábamos presentes, sumamente listo, de
semblante simpático y modales muy finos, como de quien había cursado en
diferentes cortes europeas, distinguiéndose además por su aparente
dignidad y cordura al tratar las cuestiones de Estado. Detestaba
cordialmente la camarilla, a la cual llamaba \emph{vil chusma}, aunque
nunca se atrevió a combatirla abiertamente, ni tampoco renunció a su
apoyo cuando lo necesitaba. Más que odio inspirábale envidia la
camarilla, porque podía más que él. En cuanto a mi persona, en aquella
sazón Ceballos me consideraba mucho, por el afán de congraciarse con
Ugarte, a quien envidiaba y temía. Así es que no bien disparole el duque
la alusioncilla picante de su afrancesamiento, entabló coloquio conmigo,
mientras los demás, se ocupaban de otro negocio.

---¿Con que va Vd. a la Caja de Amortización?---me dijo.

---Por mi parte nada sé---repuse con modestia.---Algunos me lo han
dicho; pero puedo asegurar que no lo solicité, ni hasta ahora me lo han
propuesto.

---Dígolo, Sr.~Pipaón---añadió disimulando con una sonrisita forzada y
modales respetuosos el desprecio que aquel fatuo sentía hacia
mí,---dígolo, porque me parece una de las mercedes más justas que se han
dado en estos tiempos\ldots{} Vamos a ver, ¿por qué no se viene Vd. con
nosotros?

---¿Al ministerio de Estado?

---Justo. Hombre, se lo he de decir a Ugarte, a mi querido amigo el
Sr.~D. Antonio\ldots{} Allí necesitamos hombres de actividad, hombres de
ingenio despierto\ldots{}

---Gracias, Sr.~D. Pedro. Yo no sirvo para la diplomacia.

Firme en mi propósito de no desperdiciar ripio para ganar la estimación
de cuantos hombres figuraban, hubiesen figurado o estuviesen en vías de
figurar por aquellos días, dije al don Pedro:

---En el ministerio de Estado no pueden servir hombres legos y sin
ninguna ciencia diplomática. Desgraciadamente en España tenemos tan
pocas personas idóneas para este ramo\ldots{}

---Es verdad.

---Tan pocas, que se pueden contar---repetí,---y si nos concretamos al
desempeño de la primera Secretaría, no sé, no sé que haya más de
uno\ldots{} No lo digo porque me esté Vd. oyendo. Cuantas veces he
hablado de esto con mis amigos les he dicho: «Cítenme Vds. un hombre,
uno solo que pueda reemplazar a D. Pedro Ceballos, si por desgracia
dejara la cartera de Estado».

---¡Oh!, es Vd. muy benévolo, Pipaón---dijo, no muy sensible a mis
lisonjas.

---Es la verdad---proseguí con calor.---Yo me asombro de la delicadeza y
dificultad de los negocios diplomáticos en que hay que tratar con
naciones extrañas, y procurar engañarlas a todas si es posible\ldots{}
Cualquier ministerio puede desempeñarse fácilmente; pero el de
Vd\ldots{} Bien lo conoce Su Majestad, que al tolerar en las demás
secretarías a personajes tan nulos como D. Francisco Eguía ---bajé la
voz, aunque estaba lejos,---pone en las de Estado, al único hombre de
talento y saber que frecuenta estas salas\ldots{}

---¡Qué lisonjero!

---¡La verdad! Vamos a ver. ¿No da risa ver al frente del ramo de Guerra
a ese grotesco señor de la coleta, que poco ha ponderaba las ridículas
ordenanzas que ha dado al ejército?

D. Pedro Ceballos no pudo contener la risa.

---Calle Vd., calle Vd.---me dijo, haciendo alarde de prudencia y
compañerismo.

Luego bajando la voz, y tomándome el brazo para alejarnos más de los
demás palaciegos, me dijo:

---Sea Vd. franco. Esa \emph{vil chusma}, con la cual Vd. anda a brazo
partido, ¿ha dicho hoy algo de la caída de Villamil?

---No ha dicho una sola palabra, Sr.~D. Pedro: ellos no se franquean
conmigo ---respondí.---Saben que les desprecio altamente\ldots{}

---Se murmura que Villamil no durará dos días. ¡Qué desventurado reino!
Aquí no hay nada seguro; vivimos a merced de esa gentuza\ldots{}

---Si yo no sé cómo Su Majestad tolera que ese vil criado, ese libertino
duque\ldots{}

---Más bajo\ldots{}

---Y no dudo que lo consigan---añadí con magistral oficiosidad.---Será
lástima que un ministro tan probo, tan entendido, tan decente como el
Sr.~D. Juan Pérez\ldots{}

---¡Oh! Yo pienso hablar al Rey hoy mismo con energía---dijo aquel
hombre que no había sido nunca enérgico más que para pasarse de un
partido a otro.---Esta detestable servidumbre, que es autora de la
bárbara política que se hace hoy, así como de las crueldades de los
comisarios enviados a provincias por privada disposición del Rey sin
contar con nosotros; esa vil servidumbre, esa desastrosa política,
repito\ldots{}

No dijo más, porque se acercó a nosotros un nuevo personaje. Era el
obispo de Almería, Inquisidor general.

---Bien venido sea el señor obispo---dijo don Pedro ceremoniosamente.

---Felices, hijo mío---repuso el prelado sonriendo;---¿esa salud cómo
va? ¿Pero no anda por aquí el Sr.~Collado?\ldots{} ¡Sr.~Collado!

Y dirigió sus miradas a un lado y otro sin dejar la sonrisita.

El lacayo acudió presuroso mientras los presentes besábamos el anillo a
Su Ilustrísima. Tenía el de Almería un semblante de angelical bondad,
que al punto le ganaba las simpatías de cuantos tenían la inefable dicha
de tratarle. Hombre menudillo y achacoso, no dejaba por eso de ofrecer
un aspecto verdaderamente patriarcal. ¡Bondadísimo varón! Viéndole, se
sentía uno inclinado a las buenas acciones, a la mansedumbre evangélica,
a la exaltación mística y a la piedad. No salía de su boca palabra
alguna que no fuese la misma devoción y un compendio del Evangelio.

---No he querido retirarme sin hablar con usted---dijo a
Chamorro.---Vengo de ver a Su Majestad, y le he recomendado el asunto de
las señoras de Porreño. Se presenta muy favorable; pero es preciso que
me lo apoye Vd., pero que me lo apoye en forma, ¿estamos?

---Descuide Su Ilustrísima---repuso el ex-aguador.---Se atenderá con
mucho gusto.

---También el Sr.~Artieda lo toma con gran calor---prosiguió el príncipe
de la Iglesia, con benévola sonrisa;---pero no me fío de Artieda, que es
un poco falso. Vd. es más formal, Sr.~Collado\ldots{} ¡Ay!, como Vd. me
descuide este asunto\ldots{} Son infinitas las personas de viso que se
interesan por esas pobres señoras. Aquí precisamente tenemos una.

El obispo me señaló. Inclineme respetuosamente.

---En efecto---dije.---Conozco mucho a esas señoras y ya he dado algunos
pasos\ldots{} Es indudable que alcanzarán lo que solicitan\ldots{} O
hemos de poder poco, Ilustrísimo Señor, o lo hemos de conseguir.

---Es preciso hacer algo por los desgraciados---afirmó el Inquisidor,
dando un suspiro, y poniendo los ojos en blanco.---Esto es más que un
favor, Sr.~Collado; es una obra de caridad\ldots{} No me descuide Vd.
tampoco aquel asuntillo de mis primas, ¿eh?

---Puede Su Ilustrísima ir sin cuidado---replicó el ex-aguador.---Todo
se hará.

---Si no se tratara de obras de caridad, no molestaría\ldots---dijo el
prelado en tono de protesta.---Pero, amados hijos míos, no se ven más
que lástimas por todos lados\ldots{} Yo quisiera atender a todo; pero
soy un pobre pastor viejo que apenas puede ya con el cayado\ldots{} Con
que ¿quedamos en ello?---añadió con apresuramiento y afán de marcharse,
porque había llegado la hora de la comida.---No necesitaré dar a usted
nota escrita, ¿verdad?

---Tengo buena memoria---repuso el criado, besando de nuevo el anillo al
noble prelado.---Téngala Usía Ilustrísima también para mí en sus
oraciones.

Nos disponíamos a acompañarle hasta la sala inmediata, donde le
aguardaban sus familiares, cuando a él y a nosotros nos detuvo otro
sujeto, también anciano simpático y venerable, que de improviso entró.
Era don Tomás Moyano, ministro de Gracia y Justicia, célebre por sus
muchos parientes, que iban viniendo en tribus invasoras de los pueblos
de Rueda, Medina y La Seca, para acomodarse en la Administración. Había
sustituido a Macanaz. Si he de decir verdad, era hombre altamente
insignificante, que por nada se distinguía, como no fuera por su
obesidad. Al entrar hizo algunos gestos, como mandando a todos que nos
detuviéramos para comunicarnos algo de mucha importancia, y antes que le
preguntáramos, dijo a voces:

---Aquí llevo el decreto para que lo firme Su Majestad.

---¿Qué decreto?---preguntaron varios con curiosidad suma.

---Señores---exclamó declamatoriamente,---felicitemos todos al señor
Inquisidor general por la merecida distinción con que acaba de
agraciarle Su Majestad.

---Nada más justo---dijo Ceballos, descifrando el enigma y haciendo una
cortesía al digno prelado.---Su Majestad ha concedido a Su Ilustrísima
la Gran Cruz de Carlos III.

---¿Y eso era?\ldots---balbució el pastor.---Pero ¿en qué están Vds.
pensando?\ldots{} ¡Darme a mí la gran cruz, a mí, que estoy muy lejos de
merecerla, cuando hay tantos otros!\ldots{}

---Fue idea mía, señores---dijo Moyano con vanidad
indescriptible.---Anoche lo propuse a Su Majestad, y al punto\ldots{}
Hoy he extendido el decreto---añadió pasando la vista por un papel
escrito,---y no falta más que la firma\ldots{} «En atención a los
méritos del muy reverendo, etc\ldots{} y en \emph{premio de su humildad
apostólica}\ldots»

---\emph{En premio de su humildad apostólica}---repitió Ceballos.---Me
parece admirable. Señor obispo, felicito a Usía Ilustrísima.

---¡Todo sea por amor de Dios!---exclamó el obispo juntando las manos.

Todos nos inclinamos, y aquello fue un coro de felicitaciones y
plácemes. Al santo y humilde pastor casi se le saltaban las lágrimas de
puro enternecimiento. Yo estaba también muy conmovido.

---En vez de ocuparse de dar cruces a los pobres viejos achacosos---dijo
el Inquisidor, con ese tono de represión benévola y delicada que se
emplea para condenar aparentemente las cosas que más nos
agradan,---debiera Vd. ocuparse, Sr.~Moyano, de expedir de una vez ese
decreto en que Su Majestad nos concede el uso diario y constante de
nuestra venera.

---Es verdad---repuso Ceballos,---pero ya hemos tratado en Consejo este
asunto. No se puede hacer todo de una vez.

---Se ha despachado primero la creación de la \emph{Cruz de
Valencey}---dijo Eguía.

---La \emph{Cruz de los Persas} nos ha dado también mucho que
hacer---añadió Moyano.

---Y la \emph{Cruz del Escorial}.

---Pero la de los señores inquisidores quedará despachada bien pronto, y
podrán usar su distintivo diariamente, como los caballeros de Calatrava
y Santiago, a fin de que sean conocidos del pueblo y respetados y
considerados como merece ese alto instituto.

---La visita que Su Majestad nos hizo el otro día---dijo con dulzura el
prelado,---dignándose ver y fallar varias causas, sentado al lado
nuestro y compartiendo nuestras fatigas, debía señalarse con una
distinción solemne hecha al Supremo Consejo. Así entiendo yo la cruz que
se me ha dado, señores: se ha querido honrar a toda la corporación,
honrando a este indigno soldado de la fe. Doy las gracias a los
generosos ministros que se han acordado de este humilde siervo de Dios;
y pues nobleza obliga, suplico a los señores ministros presentes que me
acompañen hoy a la mesa.

---Yo acepto---dijo D. Pedro Ceballos, con cortesana
desenvoltura.---Desde el banquete que Su Ilustrísima dio al Rey el día
de la célebre visita, corre por estos barrios la noticia de que el
cocinero del Inquisidor general es uno de los mejores de Madrid.

---Un pasar decoroso y nada más---repuso el prelado.---Con que señores,
¿no hay otro de ustedes que quiera hacer penitencia?

---Harela yo también, señor obispo---dijo don Francisco Eguía,
estrechando fervorosamente la mano que el reverendo le alargaba.

---Por mi parte, no desairaré a Su Ilustrísima---manifestó Moyano, lleno
de piedad cristiana.---El despacho con Su Majestad será breve.

---Señor duque---dijo Su Ilustrísima, despidiéndose.---Sr.~Collado,
Sr.~Pipaón, mil bendiciones para todos y mil millones de gracias por sus
bondades.

Salieron.

---¡Id con Dios!\ldots{} ¡Fuera, fuera, \emph{vil chusma}!---exclamó el
duque, moviendo los brazos como cuando se espanta una turba de insectos
importunos.---Esta sí que es \emph{vil chusma}.

---Los pobrecitos se contentan con lo que les dan---indicó Chamorro,
sonriendo.---La verdad es que no son muy molestos.

---Ya Ceballos da por muerto a su compañero y amigo Villamil---dije
yo.---Ese fatuo insoportable me ha pedido noticias, y dice que esta
noche piensa echar a Su Majestad un discursito acerca de la \emph{vil
chusma}.

---Ya veremos---afirmó Alagón, haciendo ademán de pegar.

---Después lo veremos---repitió el ex-aguador.

---Y qué tal, Sr.~Collado---preguntó Paquito,---¿ha podido Vd. conseguir
algo esta mañana?

---Así, así---repuso el lacayo, rascándose la sien.---Todavía no se
acaba de convencer.

---Se le ha puesto entre ceja y ceja que Villamil es un hombre
necesario, y apéele Vd. de esa burra---dijo el duque.

---Creo que esta noche le convenceremos---indicó el aguador.---Ya esta
tarde, cuando le vestimos, parecía más inclinado\ldots{}

---¿Ha habido piano esta tarde?---preguntó con afán el capitán de la
guardia.

---Un poquitín de \emph{forte piano}.---replicó maliciosamente el
lacayo.

---¿Y esta mañana?

---Rasca y más rasca\ldots{} No se le podía meter el diente. Artieda,
por importuno, se llevó una rociada de vocablos, que si fuera de palos
no le quedara hueso en su lugar.

Esto necesita una explicación. Los favoritos habían observado que cuando
Su Majestad, al sentarse junto a la mesa de su despacho, movía
volublemente los dedos sobre ella, como quien toca el piano, modulando
al par entre dientes un sordo musiqueo, estaba en excelente disposición
para conceder lo que se le pedía. Por el contrario, cuando se rascaba la
oreja o se pasaba la palma de la mano por la frente, era casi seguro que
negaría la petición. Ajustaban todos hábilmente su conducta a estos
externos signos del humor del príncipe, y por tal ley se regían los
sucesos. Un gran movimiento en palacio, excesivo flujo y reflujo de
intrigas, febril actividad en los excelsos camarilleros, indicaban que
era día de piano.

---Esta tarde vamos a paseo---dijo el duque,---y daré otro ataque. ¿Qué
órdenes hay para esta noche?

---Come solo.

---Mejor. Ya me ha dicho que no irá al teatro en toda la semana. Habrá
tertulia ---murmuró el duque reflexionando.---No falte usted a la
tertulia, Pipaón.

---Ni tampoco el Sr.~Ugarte---dijo Chamorro levantándose.

---No faltará---aseguré yo.

---Voy adentro antes que me llame---añadió el aguador.---Hasta la noche,
señores.

---Hasta la noche.

Luego que nos quedamos solos, el duque me dijo:

---Que no deje de venir esta noche D. Antonio. Es hombre a quien cada
vez estima más Su Majestad. Personas de tales prendas debieran poseer
por entero la confianza de los Reyes; no ese estúpido Chamorro\ldots{}

---¡Ah! Vd. piensa como yo\ldots---dije adaptándome rapidísimamente,
según mi costumbre, a las ideas de mi interlocutor.

---¿Qué?

---Que ese Chamorro es un bestia.

---Un dromedario, en cuya joroba no vendrían mal todos los palos que él
daba a su pollino cuando traía agua de la fuente del Berro.

---Quién sabe\ldots{} puede que el palo esté ya cortado de la rama y
alguien esté afilándole los nudos\ldots{}

El duque se echó a reír, marchando ya hacia la puerta, para ir a la
Cámara regia.

---Si de mí dependiera\ldots{} Cuidado, amiguito Pipaón---añadió
cautelosamente,---con dejar entrever a ese avestruz el asuntillo de que
hablamos ayer en la Trinidad.

---¡Oh, el asuntillo! ¡Y qué asuntillo, señor duque!---exclamé
restregándome ambas palmas de las manos una con otra, y alzando los
hombros.

El duque se puso el índice en la boca, y cordialmente se separó de mí.
Poco después estaba yo en casa de D. Antonio Ugarte, contándole todo lo
que había visto y oído.

\hypertarget{xix}{%
\chapter{XIX}\label{xix}}

A las nueve de la noche pisaba yo la Cámara real, aquella deslumbradora
cuadra, colgada y ornada de amarillo, en cuyas paredes los más hermosos
productos del arte (todavía no se había formado el Museo del Prado)
recibían diariamente, como gentil holocausto, el humo de los mejores
cigarros del mundo. Diversos bustos de príncipes de ambos sexos puestos
sobre las mesas, alegraban la estancia con sus caras satisfechas. Las
miradas de sus ojos de mármol parece que confluían al centro, y se
contemplaban unos a otros, a veces risueños, ceñudos a veces, según
estaba festiva o lúgubre la tertulia. Casi en el centro de uno de los
testeros, media docena de hombres desvergonzados, sucios, casi desnudos
unos y haraposos otros, con semblante estúpido y ademanes incultos
todos, se reían de la tertulia constantemente, embrutecidos por el vino.
Eran \emph{Los Borrachos} de Velázquez. A veces aquellos hombres puestos
en alto, entre los cuales el del centro escrutaba con su mirar insolente
toda la sala, parecían una especie de tribunal de locos. En un rincón,
junto al hueco de la ventana, refugiado en la sombra y casi invisible
estaba un hombre lívido, exangüe, cuya mirada oblicua lo abarcaba todo
desde el ángulo oscuro. Vestía de negro y en una de sus manos llevaba un
rosario. Era \emph{Felipe II}, pintado por Pantoja. Ante aquel retrato
se detuvo en pie Napoleón, contemplándolo con atención profunda un día
de Diciembre de 1808.

Cuando yo entré en la Cámara Real, Su Majestad estaba sentado en un
sillón a poca distancia de la chimenea encendida; tenía la cabeza echada
hacia atrás, de modo que miraba al techo, dirigiendo hacia él el humo de
su cigarro. A espaldas de su señor estaba Pedro Collado, y no lejos
Artieda, que era menudillo y algo compungido, de semblante un poco
aclerigado, ya viejo, tardo en hablar y en moverse, pero de ojos muy
observadores. El duque había entrado conmigo. Saludamos al Rey,
distinguiéndome yo por mis exageradas muestras de veneración y amor, a
estilo Lozano de Torres (aún no es ocasión de hablar de este personaje).
Fernando me recibió con aquella placentera bondad que le reconocen
amigos y enemigos, y luego en el tono más campechano del mundo nos dijo:

---Duque, siéntate\ldots{} Siéntate, Pipaón.

Volviendo la cabeza a un lado y otro, añadió:

---Collado y Artieda, sentaos.

Los dos venerables criados, el prócer ilustre y yo, humilde hijo de
labradores, nos sentamos frente al poderoso en los divanes que había a
un lado y otro de la chimenea.

Puso Fernando una pierna sobre la otra (¡cuán presentes tengo estos
detalles!) y retorciendo el cigarro en la boca, dejó caer de sus
augustos labios estas palabras:

---¿Qué se dice por ahí?

---Esta tarde---replicó Collado,---han ido a comer con el Inquisidor
general, D. Pedro Ceballos, Eguía y el Sr.~Majaderano.

---¿Quién es Majaderano?---preguntó con indiferencia Fernando.

---El ministro de Gracia y Justicia---repuso Alagón.---Así le llamaba
\emph{Gallardo} en su graciosa \emph{Abeja}.

No nos reímos, porque el monarca permaneció impasible. Al fin,
sonriendo, dijo:

---¡Ceballos sentado a la mesa con el Inquisidor!

La señal fue dada. Todos soltamos la risa.

---¿Si querrá D. Pedro participar al prelado cómo va la secta masónica
de que es jefe?---dijo el duque.

---Yo había oído que era masón---afirmé con malicia,---pero hasta ahora
no sabía que era el Papa de los Hermanos.

---Tan cierto como es noche---dijo Alagón, observando el semblante de Su
Majestad, que impasible hasta entonces demostraba poco interés en la
conversación.

---Lo que más asombrará al mundo---indicó Collado,---es saber que los
masones tienen su logia en la casa misma de la Inquisición.

---Hombre, tanto como eso\ldots{} murmuró el Rey con indolencia.

Todos fijamos en él la vista.

---Quizás se trate hoy de eso en la comida del Inquisidor---añadió
Paquito.

---Artieda---ordenó Fernando bruscamente.---Trae cigarros.

El lacayo dio al Rey lo que este pedía, y habiéndonos ofrecido a todos
los presentes, fumamos. El humo de los cuatro cortesanos juntábase con
el del Rey en los oscuros ámbitos del techo, donde hacían cabriolas
media docena de dioses y ninfas pintadas por Bayeu.

---¿Qué habláis ahí de franc-masonería?---preguntó Fernando después de
una larga pausa en que no se oía más ruido que el del enorme reló cuya
ancha esfera y pagana figura de bronce ornaban la chimenea.

---El señor ministro de Estado de Vuestra Majestad lo podrá
decir---repuso Collado.

---¿Qué hablas ahí, estúpido?---dijo Fernando, sacudiendo un poco su
somnolencia.

---Señor---repuso el criado, apoyando los codos en las rodillas y
observando el cigarro mientras lo volteaba entre los dedos, liando y
desliando la ensalivada capa.---Los tontos y estúpidos son los que dicen
las verdades. Vaya por las que he dicho a V. M. en ocho años.

---¿Hablabas de Ceballos?

---Sí señor.

---Decías que era franc-masón. ¿Acaso hay ahora
franc-masones?---preguntó el hijo de Carlos IV con viveza.

---Los hay, los hay---exclamó Collado.---Esta mañana hablábamos el
Sr.~Pipaón y yo de la taifa de masones que va saliendo por todos lados,
como mosquitos en verano y\ldots{} que cuente el Sr.~Pipaón lo que sabe.

---Pipaón---dijo el Rey con evidente deseo de variar la conversación y
sonriendo picarescamente,---no entiende más que de cortejar muchachas
bonitas.

Hice una reverencia a la bondadosa Majestad, única contestación que me
era permitido dar a broma tan impropia de la gravedad de mi carácter.

---Sí---añadió el señor de dos mundos, juntando la nariz con la
barba,---con esa cara de Pascua florida y esa hinchazón de consejero de
Castilla, es el mayor amparador de doncellas que hay en Madrid. Se mete
en las casas más honestas, saca los tiernos pimpollos, los conduce
socolor de música y fiestas a los barrios bajos, los lleva también a las
procesiones, a las fiestas de los conventos\ldots{}

---Señor, señor\ldots{}

Yo no podía decir otra cosa, humillando mi frente de vasallo, ante la
sonrisa de quien me honraba dejando caer sobre mí las relucientes ascuas
de sus burlas reales. De repente aquellos cortesanos tan diestros, tan
hábiles en el conocimiento de las conveniencias de la cámara, así como
de la caprichosa voluntad de su señor en la marcha de los diálogos que
allí se sostenían, dejáronme solo en presencia de Su Majestad. El duque
llevó a los dos criados al otro lado de la estancia.

Hubo una pausa. Fernando contemplaba el techo, y al fin, como quien sale
de honda distracción, mirome fijamente y preguntó:

---¿Qué decías?

---Señor, Collado ha apelado a mi testimonio en apoyo de sus opiniones
sobre la franc-masonería, y yo debo decir\ldots{}

---Que todos son masones, y yo el jefe de ellos\ldots{} ¿Te ríes? Pues
no falta quien lo asegura así.

---¡Oh!, señor, antes que pronunciar tal desacato, mis labios callarían
para siempre.

---La verdad es que hay un Oriente en Granada, del cual es presidente el
conde del Montijo\ldots---continuó el Rey.

---Justamente, señor y\ldots{}

---Y en el cual parece andan también muchos hombres graves que no
debieran ponerse en ridículo\ldots{} pues tengo para mí que eso de la
masonería es una farsa grotesca, que no conduce a nada bueno ni a nada
malo. Muchos son masones para ocultar sus amores nocturnos---añadió con
viveza;---por ejemplo tú\ldots{} Dime, ¿a qué logia ibas anoche con
aquellas dos damas?

---Señor\ldots---repetí confundido.

Indudablemente me puse como una cereza. Él dijo con mucha gracia:

---La desmayada se me presentó otra vez al día siguiente en la Trinidad.
Cojeaba un poco y estuvo a punto de caer segunda vez. Muchos tropiezos
son en tan poco tiempo.

---¡Oh!, sí, muchos tropiezos. Vuestra Majestad sabe ya quién es la
madre, la hija, el hermano, etc. En cuanto a la niña, no hay otra en
Madrid ni más linda ni más graciosa.

---En verdad---indicó el Rey, dando a aquel asunto un interés
inmenso,---sus facciones no son perfectas; pero la expresión de su cara
es encantadora y el conjunto de sus facciones\ldots{}

---¡Oh, seductor! ¿Pues y aquellos torneados brazos y aquel cuello de
alabastro?\ldots{}

---¡Y qué pie tan bonito! ¿No es verdad?---dijo Fernando con sencillez
suma, no menos engolfado que un mozalbete en la contemplación imaginaria
de la beldad soñada.---Paco no ha podido decirme los motivos de aquel
brusco encuentro; ¿a dónde ibais?, ¿de dónde veníais?

Comprendiendo que marchaba por buen camino, expuse a mi interlocutor los
verídicos hechos de mi paseo nocturno, sin omitir nada, ni alterarlos,
ni olvidar antecedentes ni móvil alguno, y en el momento en que
pronuncié el nombre de Gasparito Grijalva, sorprendiose mucho y alzando
la voz, me dijo:

---Hoy ha estado aquí su padre a pedirme que ponga en libertad a ese
niño. Es una buena obra\ldots{} lo he concedido al momento. ¿No crees tú
que es una buena acción? La pobre muchacha merece esta recompensa por su
puro y noble amor.

Yo callé.

---¿No crees tú que es una buena obra ponerle en libertad?\ldots{} ¿No
crees que mañana mismo?\ldots{}

Seguí callando y moví la cabeza en ademán dubitativo.

---¡Cuán dulce prerrogativa es la del perdón en los
reyes!---exclamé.---Dios se la ha concedido para que sean superiores a
las mismas leyes, que no tienen más que la de la justicia.

Fernando pareció fastidiado de mi pedantería, y bruscamente me dijo:

---¿Qué crees tú? Dilo con franqueza.

---Mi opinión, señor---repuse con humildad,---no debe ser de ningún peso
en las resoluciones de Vuestra Majestad, pero si me viera precisado a
darla\ldots{}

---Ya la espero---afirmó con impaciencia aquel hombre prudentísimo que
no quería nunca proceder de ligero en sus resoluciones.

---¿No hay tiempo de poner en libertad a ese loco?---dije con la mayor
osadía.---¿Por fuerza ha de ser mañana, señor?

---Verdaderamente es así. Pero yo prometí a ese anciano la libertad de
su hijo\ldots{}

---¡Qué dulce prerrogativa es la del perdón!---repetí
compungidamente.---¡Y qué placer tan grande debe de experimentar el
corazón de un monarca al conceder mercedes a sus súbditos sin omitir a
los más grandes criminales! Las alegrías que con una sola palabra
produce, ¡cuán benditas son! ¡Cuántas lágrimas se enjugan! ¡Cuántos
corazones palpitan gozosos! El de Presentacioncita, en este caso,
saltará dentro del blanco seno, más por ver logrado su empeño que por
amor al mancebo.

---Pues qué, ¿no está enamorada de ese calaverón?\ldots---preguntó con
mucha viveza, hondamente interesado en todo aquello que pudiera
contribuir al bien de sus súbditos.

---No lo creo\ldots{} Le tiene afecto, un afecto caprichoso y nada más.
Es niña de mucha ambición\ldots{} Ha de saber Vuestra Majestad que tiene
aspiraciones locas, insensatas\ldots{}

---Aspiraciones locas---repitió.---¡Vaya con la niña!

---Si Vuestra Majestad la tratase, si pudiera apreciar por sí mismo los
vuelos de aquella imaginación ardiente\ldots{}

---La cojita no puede ser más mona---dijo, dando a sus ojos expresión
semejante a la que en los suyos tenía alguno de los individuos del
lienzo de Velázquez.---¡Y qué cuerpo tan bien formado!\ldots{} Es una
preciosidad\ldots{} una joyita de carne y hueso.

Hablome en este tono largo rato, demostrándome su mucha afición a las
artes, y principalmente a la escultura, de la que era especial devoto.

---¡Y pensar que tales tesoros van a ser para ese tronera de Gasparito
Grijalva! ---exclamé yo.---Vamos, ¡quién le había de decir a ese
calumniador de Vuestra Majestad, a ese charlatán irreverente y
desvergonzado que mañana mismo va a recibir de Vuestra Majestad
generosísima el perdón de sus culpas, y que con el perdón va a entrar en
el pleno goce de sus derechos amatorios!\ldots{}

---¡Es su novio, su pretendiente!\ldots{} ¡Cómo se divierten esos
chicos\ldots{} que no son reyes!

---Y no la deja ni a sol ni a sombra. ¡Qué pesado es! Como la condesa le
permite entrar en la casa, allí está a todas horas el barbilindo cosido
a las faldas de su Filis. No puede la niña pestañear sin que el moscón
se entere\ldots{}

---¡Hombre!---exclamó el Rey, dándose una palmada en la rodilla,---me
carga ese niño.

---¡Y qué lengua!\ldots{} ¡Qué lengua! Es capaz de revolver a todo
Madrid.

---En verdad, Pipaón, que si no fuese porque prometí a Grijalva ponerle
en libertad\ldots{}

---¿Pero por fuerza ha de ser mañana?---me atreví a decir.---¡Ah!
Vuestra Majestad no sabe ser generoso a medias, y por hacer bien, no
repara que favorece a sus enemigos.

---No estaría demás que ese D. Gasparito, o D. Moscón, durmiese unas
noches más en la cárcel, ¿qué te parece, Pipaón?

---Admirable: unos días más de cárcel, y después se le pone en la
calle\ldots{} ¡Generosidad y previsión! ¡Ejemplares virtudes que no
deben separarse jamás!

---Dices bien; pero yo\ldots---objetó Su Majestad sacudiendo el cigarro
y pidiéndome fuego para encenderlo,---pero yo quisiera servir al pobre y
leal D. Alonso\ldots{} Cuando yo estaba en Francia, me prestó varias
cantidades sin interés ninguno.

---Si Vuestra Majestad aprecia en algo mi parecer me tomaré la libertad
de decirle que Grijalva tiene asuntos de más interés que el de su hijo,
y en los cuales puede recibir inmensos favores de su Soberano.

---¿Cuáles?, dímelo pronto.

---El de la moratoria que solicitan las señoras de Porreño\ldots{}
Conceder esa merced y dar golpe terrible a Grijalva es todo uno.

---¿Grijalva es el acreedor?---preguntó con anhelo.

---El mismo. Suponga Vuestra Majestad qué gracia le hará esperar diez o
doce años para poder embargar los bienes de esas señoras\ldots{}

---Porreño se comió su fortuna y la ajena, diose buena vida, y ahora sus
herederos no quieren pagar\ldots{} ¡Qué excelente sistema! Veo que esas
señoras tienen talento, Pipaón---dijo Su Majestad con expresión festiva.

---¡Excelente sistema!---repetí yo.

---¡Y sobre todo muy español!---añadió el Rey de las Españas, con un
aplomo humorístico que a pesar mío me hizo reír.---Gastar lo propio y lo
ajeno, vivir a lo príncipe, y después encastillarse en la grandeza y
dignidad de los títulos nobiliarios para rechazar el pago de las deudas
como una ignominia\ldots{}

¡Oh, qué delicioso país y qué incomparable gente!

---Sin embargo, se dice que Grijalva no cobrará\ldots{}

---Que sí cobrará\ldots{} pues no faltaba otra cosa---exclamó Fernando
con firmeza.---Se me presenta la ocasión más bonita que pudiera apetecer
para contentar al buen D. Alonso sin ponerle en libertad al niño.

---Con lo cual se le hacen dos favores.

---¡Collado!---gritó el Rey volviendo el rostro.

Acudió el cortesano, y Su Majestad sin mirarle, le dijo:

---¿Apuntaste para mañana el \emph{sobreséase} del hijo de Grijalva?

---Sí señor, aquí está---repuso Chamorro sacando un papel.---Esta noche
pienso que pase al señor Echevarri.

---No, no hay nada de lo dicho\ldots{} ¡Artieda!

El ayuda de cámara se acercó.

---¿No fuiste tú quien tomó nota de la moratoria?\ldots{}

---Para pasarla al Consejo Real\ldots{} Ya le he dicho al señor obispo
de Menorca y al señor Escóiquiz, que estaba concedida.

---Estúpido ¿quién te mandó prometer?\ldots{}

---El señor Inquisidor general---dijo Collado,---me la recomendó también
con vivo interés\ldots{}

---Perdone Vuestra Majestad---repuso Artieda humildemente.---Sin duda yo
entendí mal, cuando Vuestra Majestad se dignó acceder a la petición que
le hicieron el reverendísimo señor obispo de Menorca, el reverendísimo
señor obispo de Astorga, y el reverendísimo Inquisidor general.

---¡Vete al diablo tú y tus reverendísimos!\ldots---exclamó Fernando,
con el rostro encendido por la ira, lo cual le acontecía a la menor
incomodidad.

---Entonces\ldots---balbució el ayuda de cámara.

---Entonces\ldots---repitió el Rey, remedando, no sin gracejo, el aire
contrito y el sonsonete quejumbrón de Artieda,---entonces quiero decir
que no concedo la moratoria\ldots{} ¿Lo entiendes? ¿Todavía quieren más
los reverendos? Ya no les queda nada que pedir para sí, y piden
moratorias para sus tramposos amigos, tenencias de resguardo para los
cortejos de sus sobrinas y beneficios simples para los niños de teta de
sus señoras amas\ldots{}

---El señor obispo de Almería---dijo Collado con timidez,---me dijo que
tenía tanto, tantísimo interés en que esas señoras\ldots{} Y Su
Ilustrísima\ldots{}

---Basta de Ilustrísimas y de sobrinos de Ilustrísimas---dijo Fernando
con hastío.---Collado, quedamos en que no hay \emph{sobreséase} para el
hijo de Grijalva. Artieda, quedamos en que no hay moratoria para las
señoras de Porreño\ldots{} Ambas cosas negadas.

Hubo una pausa. Los criados se retiraron taciturnos. Observé que desde
el rincón de Felipe II, cuatro ojos me miraban con enojo.

Un instante después entró en la tertulia mi maestro y señor D. Antonio
Ugarte.

\hypertarget{xx}{%
\chapter{XX}\label{xx}}

Entró risueño, rebosando alegría, repartiendo sonrisas, cautivando con
su amabilidad de tal suerte, que la tertulia sólo con su presencia
adquirió la animación de que antes carecía. Recibiole Fernando con mucho
gozo, y después que cambiaron varias palabras, mitad en broma, mitad en
veras, diole el Rey las quejas por su ausencia, a lo cual contestó
Ugarte:

---Pues qué, ¿este tunante de Pipaón no dijo a Vuestra Majestad que salí
de Madrid a desempeñar un encargo del señor ministro de Rusia?\ldots{} Y
a propósito, señor, ¿con que ya no tenemos ministro de Hacienda?

---¡Ya no tenemos ministro de Hacienda!---replicó Fernando con
afectación de pesadumbre festiva.---Estamos sin ministro de Hacienda.
¡Qué desventura! Di Ugarte, ¿tenemos aire que respirar y sol que nos
alumbre?

Todos prorrumpieron en sonoras carcajadas, fórmula entonces la más
gráfica de la adulación.

---¡Oh!, señor---dijo Ugarte con irónico acento dramático,---estamos muy
mal. ¡El mundo se desquicia!\ldots{} ¿Qué va a ser del reino sin
ministro de Hacienda?

---Como que no sabemos que dos y dos son cuatro si el ministro de
Hacienda no nos lo dice\ldots---añadió el Rey, produciendo nueva
explosión de risas.---Pero recobra el aliento, querido Ugarte, que hay
ministro.

---¿Quién, señor? ¿Se puede saber?

---El mismo, el \emph{señor alcalde de Móstoles}.

---¡Oh!---exclamó Ugarte con cierta confusión.---Me habían dicho que el
Sr.~D. Juan Pérez se había ido esta tarde a tocar el órgano del pueblo a
que debe la celebridad.

---No hagas caso---indicó el Rey,---no tengo motivos para despedir a
Villamil. Sólo que esta \emph{vil chusma}, como dice Ceballos, es capaz
con sus chismes y enredos de trastornarme los ministerios todos los
días.

---Pues por Madrid ha corrido la noticia---añadió Antonio I.---Por
cierto que se daba a D. Felipe González Vallejo como sucesor de D. Juan
Pérez.

---Eso quieren estos---dijo Fernando, señalando con desdén a Alagón y a
los dos criados.---En caso de vacante, tal vez\ldots{}

---Pues el consejo del duque me parece acertado---dijo Ugarte.---Vallejo
es hombre que lo entiende, aunque no lo parece. Es de esos cuya
apariencia engaña.

---¡Y tanto que engaña!---repitió Fernando con malicia.---Cualquiera
creería, oyendo a Vallejo, que es tonto solemne de siete capas. Se lleva
uno cada chasco\ldots{}

---Casi siempre engaña la apariencia en los hombres de Estado---repuso
Ugarte.

---Vamos, ya cogió D. Antonio su tema favorito---dijo el duque
riendo.---Va a hablar pestes de Ceballos.

---No, nada de eso\ldots{} Acabo de separarme de él en casa de unos
amigos,---replicó D. Antonio.---Tan guapote como siempre\ldots{}

---Aquí---dijo el Rey sonriendo,---se ha dicho esta noche que es el jefe
de los masones.

---Como D. Pedro ha de estar en todo---repuso Ugarte con mucho
gracejo,---nada tiene de particular que esté también en la masonería.
¿No le llaman por ahí \emph{el indispensable}?

---Y el \emph{cambia-colores}.

---¿No ha figurado en todos los partidos desde 1808?

---Vamos, no murmurar---dijo Fernando.---Se miente mucho y se dicen
muchas falsedades.

---Ciertamente---añadió Alagón con punzante ironía.---Que D. Pedro
Ceballos, después de ser ministro de Carlos IV y del Sr.~D. Fernando
VII, fue a Bayona y se vendió a Bonaparte\ldots{} ¡falsedad! Que el
Sr.~D. Pedro Ceballos, acompañado del masón Urquijo y del inquisidor
Llorente, redactó la Constitución de Bayona\ldots{} ¡falsedad! Que el
mismo señor firmó la circular del 8 de Julio a los agentes diplomáticos,
mandándoles reconocer al rey Botellas\ldots{} ¡falsedad! Que el
susodicho, volviéndose del revés, publicó un célebre manifiesto en que
ponía como ropa de pascuas a Napoleón, a José y a Godoy\ldots{}
¡falsedad! Que después ofreció sus servicios a las Cortes de Cádiz, las
cuales le hicieron consejero de Estado\ldots{} también falsedad y
calumnia\ldots{} En fin, que mi hombre cansado de tantos naufragios,
arribó al puerto del gobierno absoluto, donde echó el ancla e hizo
bandera de\ldots{}

---¡Alto, alto!\ldots---exclamó con mucha zunga Fernando VII;---alto,
querido Alagón, que te metes en terreno de mi tío el almirante.

Todos prorrumpimos en alegres risotadas.

Un lacayo anunció la visita de dos personajes, diciendo:

---D. Pedro Ceballos, D. Juan Pérez Villamil.

~

Pocos minutos después, en la tertulia y placentero corrillo junto a la
chimenea y alrededor de nuestro Rey, éramos siete; ocho, contando con el
astro hispano de que éramos satélites.

Villamil hablaba poco y era hombre muy serio. Ceballos, por el
contrario, gustaba de recrearse en sus propias palabras y era festivo,
grave, frívolo o sesudo, según el humor de sus interlocutores. El
primero que rompió la palabra, sin embargo, fue el ministro de Hacienda,
sin duda porque traía dentro del cuerpo algo que anhelaba echar fuera.

---Señor---dijo respetuosamente.---Por ahí se dice que he dejado de ser
ministro de Hacienda. Como Vuestra Majestad no se dignó decirme nada
esta mañana, vengo a saber si es cierto, para retirarme al sosiego de mi
casa, de donde no me gusta salir sino para el servicio de Vuestra
Majestad.

---¿Qué estás hablando? ¡Que dejas de ser ministro!---exclamó Fernando
con afectado asombro.

---Así se dice, señor.

---¿Habéis oído algo?---preguntó Su Majestad, recorriendo con sus ojos
el círculo de semblantes que ante sí tenía.

---Yo no he oído nada\ldots{}

---Ni yo.

Todos dijimos que no, haciéndonos los pasmados.

---Ya estoy cansado de recomendar que no se haga caso de
paparruchas---dijo gravemente y con mucha energía nuestro
soberano.---Pues qué, ¿dejarías tú de saberlo, si no estuviese contento
de tu ministerio? ¿Por qué había de ocultarlo hasta el momento de
sustituirte?

---Eso mismo digo yo. Si Vuestra Majestad\ldots{}

---¿Y qué tenemos de negocios?---dijo bruscamente Fernando,
interrumpiendo a su ministro.

---Los decretos que pasaron a informe del Consejo, están ya
despachados---repuso Ceballos.

---¿Cuándo quiere Vuestra Majestad que se publiquen?---preguntó
Villamil.

---Cuanto antes, hombre. Ya deberían estar publicados.

---No se dirá que no se trabaja en los ministerios---manifestó Ugarte,
dirigiendo principalmente sus miradas al secretario de Estado.---Ahí es
nada la balumba de disposiciones que van a promulgarse estos días.

---Decreto prohibiendo las máscaras---dijo Ceballos;---decreto
prohibiendo los periódicos; decreto encargando la educación de los niños
y las niñas a los frailes y las monjas; decreto recomendando que se
respete y venere a los ministros del altar; circular mandando a los
españoles que guarden la mayor compostura dentro de la iglesia; circular
disponiendo que las señoras se vistan con modestia para asistir a las
funciones religiosas\ldots{} en fin, la perturbación en que el reino
quedó después de las Cortes, exige que se trate de poner algún arreglo
en esta sociedad\ldots{} He enumerado las disposiciones que Vuestra
Majestad se ha dignado proponer y que se me entregaron en minuta escrita
de su puño y letra\ldots{} La previsión y tino de Vuestra Majestad son
dignos del mayor elogio. Los citados decretos son convenientísimos y de
grande aplicación en el estado del reino\ldots{} Queda, sin embargo,
mucho por hacer todavía. Nosotros, como más en contacto que Vuestra
Majestad con los negocios públicos y las necesidades del reino, hemos
observado irregularidades y asperezas y situaciones anómalas y tirantes
que deben desaparecer.

Fernando oía con profunda atención a su ministro de Estado, y los demás
también.

---Explícate mejor---dijo el Rey.---Ya sabes que siempre te oigo con
gusto.

Inclinándose agradecido Ceballos, prosiguió así:

---Aquello en que principalmente hay que poner mano es la irregularidad
del gobierno de las provincias de Andalucía. Hay en Sevilla un hombre
llamado Negrete, a quien todos conocemos, el cual domina allí como
dictador, sin documento alguno que acredite su autoridad, diciéndose
emisario del gobierno y atropellando a todo el mundo del modo más
inicuo. La exageración y la saña son tan perjudiciales al Estado, como
la tibieza y blandura excesivas. Las provincias de Andalucía están
aterradas, señor, con la presencia de tal monstruo. No sabemos qué magia
terrible lleva ese hombre en sus palabras; pero es lo cierto que los
propios jueces tiemblan ante él. Llena ese vil los calabozos sin más ley
que su capricho, y socolor de perseguir y exterminar a los liberales,
comete los más infames atropellos. Él mismo forma brevemente las causas,
asistido de viles sicarios, y las falla en el tribunal de la
Inquisición, donde se ha constituido en juez supremo\ldots{} Ahora digo
yo, señor, ¿puede esto tolerarse?\ldots{} ¿es posible gobernar a una
nación de esta manera? Vuestra Majestad no ha dado poderes a ese
hombre\ldots{}

---¡Oh, no; seguramente que no!---dijo Fernando con aplomo
imperturbable.

---Nosotros los ministros tampoco; el Consejo tampoco: luego ese hombre
es un falsario; ese hombre es instrumento de algunos pérfidos que
subterráneamente, o quizás de un modo hipócrita, fingiendo interés por
Vuestra Majestad, se complacen en sostener esta sangrienta intriga, que
perturba el reino todo y hace odioso el paternal gobierno establecido a
costa de tantos sacrificios.

Hubo una pausa. El soberano meditaba.

---Cosas de la masonería---indicó Ugarte.

Y repitieron todos.

---Cosas de la masonería.

En aquel tiempo, la culpa de todo se echaba al gato, es decir, a los
masones.

---Yo encargaré a Echevarri---dijo al fin Fernando muy seriamente,---que
se ocupe con empeño de descubrir los autores de tales atentados y en
ponerles remedio.

Echevarri era el ministro de Seguridad pública.

Todos fijamos la vista en Su Majestad, que contemplando el fuego, movía
dulcemente los labios, tarareando y sonriendo.

---Ceballos, ¿has visto hoy a Pepita?---dijo de súbito.

---¡Oh, sí!---repuso el cortesano, cambiando repentinamente de semblante
y tono y poniendo en olvido como por encanto a Negrete y sus
tropelías.---La he visto. Está muy incomodada con el duque por cierta
canonjía.

---¿De veras?---preguntó Su Majestad riendo.

---Traslado la incomodidad al Sr.~Collado---dijo el duque,---que en su
afán ambicioso ha dejado a esa señora sin la prebenda que le prometí.

---¡Qué demonio!---exclamó perezosamente Fernando.---Dádsela, dadle
cualquier cosa\ldots{} Por no oírla se le podrían regalar dos mitras.

---¡Dos mitras!---dije yo.---Las tiene todas la negra del Sr.~Villela.

Más adelante hablaré del Sr.~Villela, de su negra y de las mitras de la
negra del Sr.~Villela.

---Como esa canonjía estaba ya dada---manifestó Collado,---pensé que le
vendría bien a doña Pepita una superintendencia de Arbitrios, y esta
mañana le di la nota al Sr.~Villamil.

---Se hará inmediatamente---repuso el hacendista.

---O se le dará la bandolera vacante---propuso Alagón.

---¿Pero hay todavía superintendencias de Arbitrios?---preguntó
humorísticamente el Monarca,---mejor dicho, ¿hay arbitrios todavía? Yo
pensé que todo eso pertenecía a la historia, según están las cajas del
Tesoro de lisas y mondas.

---Señor---dijo Villamil,---el estado del Erario no se oculta a Vuestra
Majestad. El escaso producto de los impuestos no basta ni con mucho a
cubrir los enormes gastos, aumentados cada día con la creación de nuevos
destinos. El reino no tiene recursos para costearse su ejército, ni su
marina, ni para dotar dignamente la Casa Real ni su regia guardia;
España es pobre, pobrísima; necesita los caudales de América para vivir
con algún decoro entre las naciones de Europa.

---Y esos caudales de América, ¿dónde están?

---¡Ay, eso es lo que a todos nos contrista! Fácil sería gobernar la
Hacienda, si América nos enviase los tesoros que aquí nos hacen falta.
Esa gran canonjía de nuestra nación no ha durado todo lo que debiera.
Reflexione Vuestra Majestad, como Rey previsor, sobre la gravedad de
esta situación. La América está toda sublevada, y las juntas rebeldes
funcionan en Buenos-Aires, en Caracas, en Valparaíso, en Bogotá, en
Montevideo. Si Méjico está aún libre del contagio, los americanos de
Washington se encargan de trastornar también aquel país, del mismo modo
que el Brasil nos trastorna el Uruguay, e Inglaterra nos revuelve a
Chile. La insurrección americana exige un gran esfuerzo, un colosal
esfuerzo. Es preciso mandar allá un ejército; pero para esto, señor, se
necesitan tres cosas: hombres, dinero y barcos.

---¡Hombres, dinero, barcos!

---Lo primero no falta; pero ¿cómo los equiparemos, y sobre todo, en qué
buques les lanzaremos al mar? Vuestra Majestad no tiene en su marina un
solo navío que valga dos cuartos, y los arsenales carecen de elementos
para la construcción.

---¡Risueño cuadro acabas de trazar!---dijo Fernando, hundiendo la barba
en el pecho.

---Risueño no pero sí verdadero---afirmó D. Juan Pérez.---Si ocultase a
mi Rey la verdad, sería indigno del afecto que Vuestra Majestad me
profesa.

---Y que te profesaré siempre. Has hablado como un buen ministro. Nada
de fantasías ni palabras bonitas. Así me gusta a mí\ldots{} Pues es
preciso buscar dinero y buscar hombres y buscar barcos.

---Señor, no olvide Vuestra Majestad---dijo Ceballos,---que si se lleva
adelante la negociación con Inglaterra sobre la abolición de la trata de
negros, o hemos de poder poco o nos han de dar una indemnización de
muchos miles de libras.

---Es verdad: para resarcir los perjuicios de los tratantes de
esclavos\ldots{} A ver, Ceballos, Villamil---añadió Fernando con
dulzura,---estudiad un plan, un plan cualquiera que mejore la situación
en que nos hallamos. A uno y a otro os sobra talento para eso y para
mucho más\ldots{} ¿Me entendéis? Discurrid un plan vasto, que nos
proporcione los recursos necesarios para sofocar la insurrección
americana, bien sea creando impuestos, bien pidiendo dinero a los
holandeses o a los judíos de Francfort, bien logrando los buenos oficios
de alguna nación poderosa\ldots{} en fin, ya me entendéis.

---Ya manifestaré más adelante a Vuestra Majestad algo de lo mucho que
he meditado sobre el particular---dijo Ceballos.

---Y tú, Villamil, discurre, trabaja, proponme algo---prosiguió
Fernando.---Por supuesto, no puedes figurarte lo que me mortifica que
hayas creído en esas ridículas hablillas acerca de tu destitución.

---Señor\ldots{}

---Hablaremos más despacio mañana\ldots{} Puedes irte tranquilo y seguro
de que sé apreciar tu lealtad\ldots{} ¡Oh, Villamil!\ldots{} No abundan
los hombres como tú\ldots{} Vamos, otro cigarrito.

Diciendo esto Su Majestad, con aquella bondad peculiar, que indicaba
tanta honradez y nobleza en su carácter, ofreció un cigarro a D. Juan
Pérez Villamil.

---Gracias, señor, acabo de fumar.

---Enciéndelo para salir. Como este habrás fumado pocos\ldots{} Mira,
puedes llevarte todo el mazo---añadió ofreciéndoselo galantemente.

---Señor\ldots{}

---Nada, que te lo lleves. Tengo gusto en ello.

Cuando D. Juan Pérez, apremiado por la bondadosísima y gallarda fineza
del Príncipe, tomaba los cigarros, yo sentía que un cuerpo duro tocaba
mi codo. Era el codo del señor duque de Alagón.

Villamil y Ceballos se levantaron para marcharse.

---Que vengas mañana temprano---repitió el Rey.---A ver si discurres
algo. Y tú Ceballos, si ves a Pepita\ldots{} en fin, ya sabes: una
superintendencia de provincia o la bandolera vacante\ldots{} lo que ella
prefiera.

---En el despacho de mañana---dijo Ceballos, que se había quedado muy
taciturno,---tendré el honor de leer a Vuestra Majestad la contestación
que he dado a la nota de D. Pedro Gómez Labrador.

---Sí, bueno, todo lo que quieras\ldots{} mañana\ldots{} adiós, ¡pero
qué tarde es!\ldots{} Podéis retiraros\ldots{} yo también me voy a
recoger---dijo Fernando con impaciencia.

Los ministros salieron y quedamos solos los camarilleros.

\hypertarget{xxi}{%
\chapter{XXI}\label{xxi}}

Apenas se cerró la puerta tras los dos repúblicos, Fernando se levantó,
y con las manos en los bolsillos, dio algunos pasos por la habitación.
Ugarte le miraba sonriendo. Ninguno de los demás nos atrevíamos a
desplegar los labios, y el silencio se prolongó hasta que el mismo
soberano se dignara romperlo, preguntando:

---¿Qué dices a esto, Ugarte?

---Que admiro la paciencia de Vuestra Majestad---repuso el
ex-bailarín.---Según el señor Juan Pérez, ya no hay colonias, ya no hay
soldados, ya no hay barcos, ya los españoles no tienen alma para vencer
las dificultades. Sostendrá también el abuelillo que ya no hay aire que
respirar, ni sol en el cielo.

---La verdad es---dijo Fernando deteniéndose meditabundo ante la
chimenea,---que no estamos en Jauja.

Y luego dando un suspiro, añadió:

---Hay que despedirse de las Américas.

---¿Por qué, señor?---dijo bruscamente Ugarte.---Se exagera mucho.
Persona venida hace poco de allá, me ha dicho que toda la insurrección
americana se reduce a cuatro perdidos que gritan en las plazuelas.

---Lo mismo me ha escrito a mí un amigo---añadí yo, forzando los
argumentos de mi patrono.---Unos cuantos presidiarios, con algunos
ingleses y norte-americanos, echados por tramposos de sus respectivos
países, sostienen la alarma en aquellos lejanos reinos de Vuestra
Majestad.

---Pues id vosotros a reducir a la obediencia a esas dos docenas de
facciosos ---dijo el Rey.

---Señor, en resumen---manifestó Ugarte,---mande Vuestra Majestad a
América, un ejército, un verdadero ejército, con una escuadra, en vez de
medias compañías dentro de una goleta como se ha hecho hasta aquí, y a
los cuatro meses se verán los resultados.

---¿Y ese ejército, dónde está?---preguntó fríamente.

---¿Dónde están los vencedores de Napoleón? Parece mentira que Vuestra
Majestad haga tales preguntas.

---Hombres valerosos no faltan; pero ¿cómo se les organiza, cómo se les
viste, cómo se les mantiene?

---Muy sencillamente---repuso Ugarte, alzando los
hombros:---organizándolos, vistiéndolos, manteniéndolos.

---Tú tendrás alguna mina. ¿Quieres decirme dónde está?

---Dos palabras, señor---dijo Ugarte, echando el cuerpo hacia adelante
en su sillón y apoyando el codo en la rodilla, mientras el Rey se
sentaba junto a él.---He dicho a Vuestra Majestad la otra noche que me
atrevía a organizar un ejército expedicionario, siempre que tuviera para
ello la competente autorización.

---Yo te la doy---replicó Fernando.---A ver de dónde vas a sacar ese
ejército, y cómo lo vas a sostener.

---Vuestra Majestad me dijo también la otra noche que consagraría a tal
objeto y pondría a mi disposición una parte mínima de las rentas reales.

---Es verdad.

---Pues el alistamiento se hará, señor---afirmó D. Antonio con
resolución admirable.---No tiene que pensar más en ello Vuestra
Majestad.

---Bueno, ya está el alistamiento. Ahora hazme el favor de decirme si
vas a mandar a América esos soldados en cáscaras de nuez.

---No señor, que los mandaré en magníficos navíos y barcos de
trasporte---repuso el arbitrista con una placentera y llana confianza
que a todos nos dejó pasmados.

---Pero ya sabes que no los tenemos.

---Se compran.

---¡Se compran!\ldots{} Y dice «se compran» como si costaran dos
pesetas.

La naturalidad admirable con que Ugarte hacía frente a los mayores
obstáculos, la frescura, digámoslo así, con que todo lo resolvía y
allanaba, no podían menos de cautivar el ánimo del Soberano, agobiado
por el continuo clamoreo de sus ministros. Todos los demás contertulios
observábamos con verdadero asombro la prodigiosa iniciativa de Ugarte, y
ante tanto ingenio, ante tan firme voluntad, callábamos, confundidos.

---Pues es claro que se compran---añadió el proyectista.---Apostaría a
que Vuestra Majestad va a preguntarme que con qué dinero.

---Justo.

---Pues yo respondo que, si poseo la confianza de mi Soberano, me
sobrarán fondos en que elegir.

---Quizás cuentas con la indemnización que nos va a dar Inglaterra.

---¿Por qué no?

---Pero es para resarcir a los negreros.

---Eso es, pagar a los negreros y que se pierdan las Américas. ¿No vale
más dejarles sin indemnización, y conservarles los esclavos y las
tierras?

---Está dicho todo---exclamó resueltamente Fernando, cediendo por
completo a la seductora sugestión de aquel brujo que prometía los
imposibles y teñía con frescos y brillantes colores el entenebrecido
horizonte de la política.---Está dicho todo. Tienes mi autorización para
hacer el alistamiento, para tomar de la real Hacienda los fondos
necesarios, para tratar de la compra de buques, vestuario y demás.

De aquella conversación, brotó el poder oculto que D. Antonio Ugarte
tuvo durante algún tiempo, y en virtud del cual, hasta llegó a celebrar
tratados con potencias extranjeras en calidad de secretario íntimo de Su
Majestad. Más adelante veremos cómo alistaba tropas y qué tal mano para
comprar buques tenía D. Antonio. Sus proyectos forman una página curiosa
en la historia del absolutismo.

---Ya se ve---dijo después de una pausa, durante la cual observaba los
dibujos de la alfombra.---Con hombres como Villamil las dificultades se
multiplican. Al buen alcalde se le antojan sus dedos huéspedes, y como
en todas las ocasiones difíciles se asesora de Ceballos\ldots{}

---El pobre Ceballos---dijo Fernando,---ha trabajado como un negro en
ese fastidioso asunto del Congreso de Viena. No se le debe criticar, y
si no se ha conseguido más, no ha sido por culpa suya.

---Entre Labrador y Ceballos, como si dijéramos, entre Herodes y
Pilatos, España está haciendo un papel ridículo en Viena.

---¿Pero qué puede esperarse de un plenipotenciario que ya ha mostrado
no tener ni dignidad ni carácter?---dijo el duque de Alagón.---¿No fue
Labrador ministro de Estado en las Cortes de Cádiz, y después realista
furibundo?

---Y al presentarse en Cádiz felicitó a las Cortes por el \emph{sabio
Código} que habían hecho---añadí yo.

---En manos de estos hombres que ayer eran liberales locos y hoy
absolutistas rabiosos---dijo Ugarte,---nuestra política exterior no
puede menos de ser desastrosa. ¡Rutina incurable! Nuestra nación, señor,
ha de vivir siempre bajo la vigilancia interesada, mejor dicho, bajo la
tutela de Inglaterra o de Francia. La primera trabaja porque perdamos
las Américas y porque se arruine nuestro comercio; la segunda no nos
perdonará nunca el haber vencido a sus soldados, aunque fueran mandados
por el general Bonaparte.

---En eso creo que tienes razón---dijo fríamente Fernando.

---Pues si tengo razón, ¿por qué no intenta Vuestra Majestad estrechar
sus relaciones con un poderoso imperio, bastante fuerte para ser buen
aliado, bastante remoto para no disputarnos nuestro territorio?

---Soy muy amigo de Alejandro---repuso el autócrata secamente.

---Pero esa amistad sería unión indestructible, si Vuestra Majestad, que
seguramente no puede permanecer soltero más tiempo, se enlazara con una
princesa rusa.

Al decir esto, Ugarte había pronunciado la última palabra del
atrevimiento. Hubo una larga pausa. Observamos todos el semblante del
Rey, que con las piernas estiradas, las manos en los bolsillos del
pantalón y la barba sobre el pecho, indolentemente tendido más bien que
sentado en el sillón, no se dignaba contestar ni con palabras, ni gesto,
ni mirada ni sonrisa a las palabras de Ugarte. Por último, le vimos
mover los brazos, luego alzar la cabeza, y aguardamos con ansiedad
vivísima el sonido de su voz.

---¿Te parece---dijo,---que debo refrenar un poco a Negrete?

---Las atrocidades del comisario secreto son tan grandes---repuso
Ugarte,---que convendría ponerle a un lado y prescindir de sus
servicios. Ceballos tiene razón. Están tan irritados los andaluces, que
son capaces de volverse todos liberales, si ese verdugo sigue haciendo
de las suyas.

---La cuestión es delicada. Negrete tiene órdenes mías, y si intentamos
sujetarle por la vía de las autoridades legítimas, no es fácil que ceda.

---Para eso se manda un nuevo comisionado a Andalucía, un hombre hábil,
enérgico, ingenioso y muy discreto, Pipaón, por ejemplo---dijo D.
Antonio mirándome.

---No---replicó vivamente Fernando, mirándome también.---Yo no quiero
que Pipaón salga de Madrid por ahora. Ya se buscará otro comisionado.
Después de todo, nada se pierde con que Negrete continúe sentando la
mano algunos días más. Andalucía está infestada de jacobismo.

---Y Madrid también---afirmó el duque.

---Las sociedades secretas rebullen por todos lados.

---No será porque dejamos de tener ministerio de Seguridad
pública---dijo con ironía el Rey.

---Echevarri encarcela a los mentecatos y deja en libertad a los pillos.
Los calabozos están repletos de tontos. Pero ¿qué ha de suceder si los
principales personajes del gobierno están inficionados de liberalismo?
Ceballos es masón, Villamil y Moyano no ocultan sus ideas favorables a
un sistema templado como el de Macanaz; Escóiquiz augura desastres;
Ballesteros quiere que se dé una especie de amnistía; en toda España se
conspira. Ábrase un poco la mano y las revoluciones brotarán por todas
partes como pinos en almáciga.

---Pues se cerrará la mano, se cerrará la mano---dijo Fernando
incorporándose en su asiento.---Duque, pon algunas líneas mandando a
Negrete que siga aplastando el jacobinismo; pero con la condición de que
no sea bárbaro\ldots{} No se puede confiar a nadie una comisión
delicada\ldots{}

Artieda acercó un velador con recado de escribir, y bien pronto la
tertulia se trocó en oficina. El duque tomó una pluma.

---Ugarte---añadió el Rey,---puedes redactar las bases de la
autorización que te doy para alistar el ejército expedicionario y demás.
Me quedaré con tu borrador para meditarlo, y después te daré la copia
firmada.

D. Antonio tomó otra pluma. Acariciándose la boca con las barbas de
esta, miró al Rey.

---Permítame Vuestra Majestad---dijo,---que decline el grande, el
insigne honor que quiere hacerme, depositando en mí toda su confianza.

Fernando le miró con asombro, y los demás también.

---De nada serviría mi abnegación, mi trabajo, mis grandes cavilaciones
y proyectos---continuó el arbitrista,---si desde el principio tropezara
con obstáculos insuperables. Yo he prometido a Vuestra Majestad reunir
tropas y equiparlas, y comprar los buques necesarios para que vayan a
América\ldots{}

---Pero una cosa es prometer, y otra\ldots{}

---Es que no puedo pensar en el desarrollo de mis proyectos, mientras
sea ministro de Hacienda el Sr.~Villamil.

---¡Bah, bah!---exclamó Fernando con tono de indolencia y fastidio.

Hubo una pausa. Todos contemplábamos al Rey, el cual, arqueando las
cejas se pasaba la mano por la cabeza, cual si se cepillara el pelo
hacia adelante.

---Pipaón---dijo al fin,---extiende la destitución de Villamil\ldots{}
Que se le lleve esta misma noche.

Yo tomé otra pluma.

Así cayó D. Juan Pérez Villamil; así cayeron también Echevarri,
Ballesteros, Macanaz, Escóiquiz, el mismo Vallejo, nombrado aquella
noche, Moyano, León Pizarro, Lozano de Torres, y otros muchos.

---Ahora extiende el nombramiento de don Felipe González Vallejo,
ministro de Hacienda.

Así subió Vallejo.

---¿Qué más hay?---preguntó Fernando con cierta somnolencia.

---Vuestra Majestad me concedió una bandolera---dijo tímidamente
Artieda,---para el sobrino del señor Arcipreste de Alcaraz\ldots{}

---Es que hay una sola vacante---añadió Collado avariciosamente,---y Su
Majestad me la tiene prometida.

---Es verdad---dijo el Rey.

Artieda miró a Chamorro con enojo.

---Esa vacante me la había reservado yo para mí---objetó con sequedad
Paquito Córdoba.---Es mucha la ambición del Sr.~Collado. Después que me
ha disputado esa miserable canonjía de Murcia como si fuese un
imperio\ldots{}

---Tienes razón---murmuró Fernando.

El aguador clavó sus ojos en el duque con expresión de envidia.

---Señor---dijo con suavidad sonriente don Antonio Ugarte.---Pocas veces
pido mercedes de esta clase a Vuestra Majestad. Ya dije el otro día que
deseaba una bandolera para un joven pariente mío.

---Nada más justo---repuso el Rey cerrando los ojos
perezosamente.---Ugarte, todo lo que quieras.

El duque dirigió a Antonio I una mirada rencorosa.

---Señor---dije yo, sin encomendarme a Dios ni al diablo,---no olvide
Vuestra Majestad que prometió una bandolera al señor conde de Rumblar,
mi querido amigo.

El Rey abrió los ojos, sacudiendo la pereza, y exclamó enérgicamente,
con aquella resolución a que ningún cortesano podía oponerse:

---La bandolera, para el señor conde de Rumblar\ldots{} lo mando\ldots{}
Alagón, extiende el nombramiento ahora mismo.

Ugarte me miró, frunciendo el ceño.

Y se levantó la sesión, como dicen los liberales.

~

Como se ha visto, en las tertulias de Su Majestad nadie podía
vanagloriarse de tener ascendiente absoluto y constante. Unos días
privaba este, otros aquel, según las voluntades recónditas y jamás
adivinadas de un monarca que debiera haberse llamado Disimulo I. Además
aquel discreto príncipe, que así delegaba su autoridad y
democráticamente compartía el manto regio con sus buenos amigos, como
compartió San Martín su capa con el pobre, no tuvo realmente favorito,
no dio su confianza a uno solo, elevándole sobre los demás; jugaba con
todos, suscitando entre ellos hábilmente rivalidades y salutífera
emulación, con lo cual estaba mejor servido y los destinos y prebendas
más equitativamente repartidos.

De lo que anteriormente he contado puede dar fe un ministro de Su
Majestad por aquellos años\footnote{Lardizábal, Ministro de Indias
  (absolutista).}, el cual, en papel impreso muy conocido, dice,
echándosela de rigorista y de censor: «\ldots pero lo peor es que por la
noche da entrada y escucha a las gentes de peor nota y más malignas, que
desacreditan y ponen más negros que la pez, en concepto de S. M. a los
que le han sido y le son más leales\ldots{} y de aquí resulta que, dando
crédito a tales sujetos, S. M. sin más consejo, pone de su propio puño
decretos y toma providencias, no sólo sin consultar con los ministros,
sino contra lo que ellos le informan\ldots{} Esto me sucedió a mí muchas
veces y a los demás ministros de mi tiempo\ldots{} Ministros hubo de
veinte días o poco más, y dos hubo de 48 horas; ¡pero qué ministros!»

Por las declamaciones de este escrupuloso descontentadizo no se vaya a
creer que la camarilla era cosa mala. Era, por el contrario, lo mejor de
mundo, sobre todo para nosotros, que traíamos los negocios del reino de
mano en mano y de boca en boca, despachándolos tan a gusto del país, que
aquello era una bendición de Dios. Ninguno, sin embargo, pudo jactarse
de ser el primero en la voluntad y paternal cariño de aquel bondadoso
soberano absoluto; y en prueba de ello referiré lo que sucedió al día
siguiente de la reunión que con todos sus puntos y señales he descrito,
no apartándome en todo el discurso de ella ni un ápice de la verdad.

Al día siguiente, como dije, volví a palacio y encontré al Sr.~Collado,
al Sr. Artieda y al señor duque muy alarmados. ¿Por qué? Porque el Rey
estaba conferenciando a solas con un sujeto que hasta entonces no había
sido recomendado ni introducido por ninguno de los sobredichos
palaciegos. Creyose que sería algún emisario de Ugarte, pero entró
enseguida don Antonio y negó el caso.

Reunímonos todos en la antesala y a poco vimos salir a un fraile
francisco, joven, bien parecido, excelente mozo, que más parecía
guerrero que fraile; de aspecto y ademanes resueltos, mirada viva y
revelando en todo su continente y facciones una disposición no común
para cualquier difícil cosa que se le encomendara.

---¿Quién es este pájaro?---preguntó Ugarte demostrando en su tono que
estaba completamente desconcertado.

---Se llama fray Cirilo de Alameda y Brea---dijo Artieda, que estaba
fuerte en todo lo referente al personal eclesiástico de la monarquía.

---Y ¿qué es este hombre?

---Fue maestro de escuela en Pinto.

---Y después marchó a Montevideo, donde se ocupaba\ldots{} No sería en
cosa buena.

---En redactar \emph{Gacetas}.

---Es hombre que pone bien la pluma, según parece.

---Vino por vez primera con el general Vigodet---añadió Paquito
Córdoba.---Su Majestad le ha recibido después en varias ocasiones, y
nunca he podido averiguar\ldots{}

---¿No ha dejado traslucir nada?

---Absolutamente nada.

---Hoy ha durado la conferencia dos horas.

---¿Y ninguno de Vds. sabe nada?---repitió Ugarte, interrogando todos
los semblantes.---Yo estoy confundido.

---No sabemos una palabra.

---Pues estamos bien\ldots{} ¿Apostamos a que este tunante de Pipaón lo
sabe todo?

---Ni una palabra---respondí tan confuso como los demás.

Y era la verdad que nada sabía. Más adelante todos desciframos el
enigma, que me hizo decir \emph{no hay función sin fraile}; pero no ha
llegado aún la ocasión de revelarlo.

\hypertarget{xxii}{%
\chapter{XXII}\label{xxii}}

Antes de seguir, quiero indicar las observaciones que sugirió el
manuscrito de estas Memorias a una persona de aquellos tiempos y de
estos. D. Gabriel Araceli\footnote{Protagonista de la \emph{Primera
  serie}.}, a quien lo mostré (no es preciso decir cuándo ni cómo), me
dijo que los lectores de él, si por acaso lograba tener algunos, no
podrían menos de ver en mí un personaje de las mismas mañas y estofa que
Guzmán de Alfarache, D. Gregorio de Guadaña o el Pobrecito Holgazán; a
lo cual le contesté que sí, y que de ello me holgaba, por ser aquellos
célebres pícaros de distintas edades los más eminentes hombres de su
tiempo, y caballeros de una caballería que yo quería resucitar para que
se perpetuase en la edad moderna. Dijo también el sobredicho señor, que
nada de lo que apunté o describí con burdo o sutil estilo, se
diferenciaba un punto de la verdad.

---La comparsa en que Vd. figuró, señor D. Juan---dijo al fin,
echándoselas de dómine sermonista,---fue de las más abominables y al
mismo tiempo de las más grotescas que han gastado tacones en nuestro
escenario político. Cuanto puede denigrar a los hombres, la bajeza, la
adulación, la falsedad, la doblez, la vil codicia, la envidia, la
crueldad, todo lo acumuló aquel sexenio en su nefanda empolladura, que
ni siquiera supo hacer el mal con talento. El alma se abate, el corazón
se oprime al considerar aquel vacío inmenso, aquella ruin y enfermiza
vida, que no tiene más síntomas visibles en la exterioridad de la
nación, que los execrables vicios y las mezquinas pasiones de una corte
corrompida. No hay ejemplo de una esterilidad más espantosa, ni jamás ha
sido el genio español tan eunuco.

«Los junteros de 1808, los regentes de 1810, los constitucionalistas de
1812, cometieron grandes errores. Iban de equivocación en equivocación,
cayendo y levantándose, acometiendo lo imposible, deslumbrados por un
ideal, ciegos, sí, pero ciegos de tanto mirar al sol. Cometieron
errores, fueron apasionados, intemperantes, imprudentes, desatentados;
pero les movía una idea; llevaban en su bandera la creación; fueron
valientes al afrontar la empresa de reconstruir una desmoronada sociedad
entre el fragor de cien batallas; y rodeados de escombros, soñaron la
grandeza y hermosura del más acabado edificio. Hasta se puede asegurar
que se equivocaron en todo lo que era procedimiento, porque los que
discurrían como sabios lo hacían como niños. La especie de tutela a que
quisieron sujetar en 1814 al Rey, viajero desde Valencey a Madrid, y el
pueril formulismo ideado para hacerle jurar a él, vástago postrero del
absolutismo, la precoz Constitución de Cádiz, fueron yerros que debían
producir el golpe de Estado del 10 de Mayo. Hasta se puede sostener que
Fernando estaba en su derecho al hacer lo que hizo; pero nada de esto
atenúa las grandes, las inmensas faltas de la monarquía del 14. Fue la
ceguera de las cegueras. La crueldad, la gárrula ignorancia de aquella
política no tiene ejemplo en Europa. Para buscarle pareja hay que acudir
a las atrocidades grotescas del Paraguay, allí donde las dictaduras han
sido sainetes sangrientos, y han aparecido en una misma pieza el tirano
y el payaso.

»No existe nada más fuera de razón, más inútil, más absurdo, que la
reacción de 1814; no sucedió a ningún desenfreno demagógico; no sucedió
a la guillotina, porque los doceañistas no la establecieron, ni a la
irreligión, porque los doceañistas proclamaron la unidad católica; ni a
la persecución de la nobleza, porque los nobles no fueron perseguidos:
fue, pues, una brutalidad semejante a los golpes del hado antiguo, sin
lógica, sin sentido común. Nada de aquello venía al caso. Si Fernando
hubiera cumplido la promesa hecha en el manifiesto del 4 de Mayo, si
hubiera imitado la sabia conducta de Luis XVIII, que desde la altura de
su derecho saludaba el derecho de las naciones; ¡cuán distinta sería hoy
nuestra suerte! Sin necesidad de aceptar la Constitución de Cádiz, que
era un traje demasiado ancho para nuestra flaqueza, Fernando hubiera
podido admitir el principio liberal, inaugurando un gobierno templado y
pacífico para la nación y por la nación. Pero nada de esto hizo, sino lo
que usted ha descrito, y aquellos seis años fueron nido de revoluciones.
El desorden germinó en ellos, como los gusanos en el cuerpo insepulto.
Desde 1814 a 1820 hubo en España trece conspiraciones, todas para
derrocar el gobierno absoluto, una para esto y para asesinar al Rey.
Abortaron las trece, pero la décima cuarta parió\ldots{} Los liberales
se presentaron con la rabia del vencedor y la hiel criada en el
destierro. ¿Qué les impulsaba en 1812? La ley. ¿Y en 1820? La venganza.
Continuaba el vicio, la corrupción, la crueldad; pero el absolutismo de
Vds. había sido tan rematadamente malo, que en los liberales del trienio
famoso podía haber crueldad, ambición, rapacidad, venganza, imprudencia
y aun dosis no pequeña de tontería\ldots{} podían aquellos benditos
avanzar hasta un grado extremo en la escala de estos defectos, sin temor
de llegar nunca, no digo a superar, pero ni siquiera a igualar a sus
antecesores».

Así mismo me lo dijo, y se quedó tan fresco.

\hypertarget{xxiii}{%
\chapter{XXIII}\label{xxiii}}

Pero vamos adelante con mi cuento.

¿Se ha comprendido ya cuál era mi plan en el asunto, o si se quiere, en
la hábil intriga cuyo hilo se extendía desde los intereses de la familia
de Porreño hasta la paternidad de D. Alonso de Grijalva? Creo que no
serán necesarias explicaciones prolijas de aquella \emph{operación},
como hoy se dice, hecha sin dificultades mayores y con éxito mejor del
que podía esperarse, considerada su delicadeza. Aburrido Grijalva de ver
que a pesar de la palabra real, no echaban de las cárceles al tuno de su
hijo, admitió las propuestas que mañosamente y por conducto de varones
esclarecidísimos y muy discretos le hice, resultando de ellas que me
vendió los créditos contra las señoras de Porreño por la mitad de su
valor. Anduvo en aquestos tratos el licenciado Lobo, con tan buen pie y
mano, que D. Alonso, muy rebelde al principio, llenose de miedo y a todo
lo que quisimos asintió al fin.

Después me quedaba lo peor y más amargo del caso, cual fue apretar a las
señoras de Porreño, para que pagasen, y, quitándoles toda esperanza de
moratoria (por la rotunda negativa del sabio y justiciero Consejo),
proceder al embargo de bienes. Aquí sí que no fue posible disimular,
porque D. Gil Carrascosa vendió a las venerandas señoras mi secreto, y
un día en que tuve el mal acuerdo de presentarme en la casa recibiéronme
como es de suponer. Desde entonces, quitado el último puntal de aquella
histórica casa, todo vino con estrépito al suelo, entre alaridos de
rabia y sollozos de aflicción. Las señoras de Porreño pasaron a la
religión de las sombras. Su última época, solitaria y lúgubre está
escrita en otro libro\footnote{En \emph{La Fontana de Oro}.}.

Renuncié, como es consiguiente, a su amistad, y me ocupé de aquellas
excelentes tierras de Hiendelaencina, de Porreño y Torre Don Jimeno, tan
diestramente ganadas con mi talento, con mis ahorros y con el dinero que
don Antonio Ugarte me prestara para reunir la cantidad necesaria. Mucho
tardé en adjudicármelas, a causa de las dilaciones de la curia; pero al
fin constituime en propietario, soñando con establecer un mayorazgo.

Pero retrocedamos a los días de mi anterior relación, que eran los
últimos de Febrero y primeros de Marzo de 1815. La Real Caja de
Administración tuvo el honor, nunca por ella soñado, de caer en mis
manos. ¡Bendito sea Dios Todopoderoso y Misericordioso, que arregla las
cosas de modo que ningún desvalido quede sin amparo! Dígolo por aquellos
miserables y huérfanos juros que hasta mi elevación no tuvieron arte ni
parte en ninguna operación rentística. Los pobrecitos no soñaban sin
duda que toparían conmigo ni con la destreza de estas limpias manos, y a
poco de mi entrada en la Caja engordaron hasta el punto de que no los
conocía el pícaro secretario de Hacienda que los inventó.

¡Qué satisfechos quedaron de mis servicios el noble duque, y D. Antonio
Ugarte! ¡Qué elogios hacían de mi impetuosa voluntad, la cual
derechamente se iba al asunto sin reparar en pelillos! Yo también estaba
envanecido de mí mismo, y entonces empecé a conocer lo mucho que para
tales asuntos valía. Yo era una firme columna del Estado; yo desplegaba
en servicio de mi Soberano absoluto y del sumiso reino, tendido a sus
pies como un perro enfermo y calenturiento que no puede moverse de pura
miseria, las más altas calidades intelectuales. Indudablemente Dios
debía de estar satisfecho de haberme criado, viéndome tan hormiguilla,
tan allegador, tan mete-y-saca, tan buen amparador de los poderosos para
que los poderosos me amparasen a mí. ¡Qué minita era aquella sacrosanta
Administración! ¡Qué terrenos inexplorados! En tal materia yo, era más
que Colón, porque este descubrió sólo un mundo y yo descubría todos los
días uno nuevo.

No hay que decir que yo navegaba a toda vela, como diría mi amigo el
Infante, hacia el Real Consejo. Todo marchaba a pedir de boca en
derredor mío. ¿Y qué diré de aquel seráfico ministro de Hacienda, D.
Felipe González Vallejo? Hombre de mejor pasta no se ha sentado en
poltrona. El pobrecito era tan buenazo, tan sano de corazón, tan amable
y complaciente, que todos los negocios pequeños, como nombramientos y
demás menudencias, estaban en manos de Artieda y del Sr. Chamorro. De
los grandes se encargaba D. Antonio Ugarte. Dios se lo pague a aquel
bendito ministro, que no tenía gota de hiel en su corazón, ni humos de
vanidad en su cabeza. Parecía que no había tal ministro. Si todos los
que han ocupado el sillón hubieran sido como él, otra sería la suerte de
este desamparado y caído reino.

En asuntos que no eran administrativos, iban mis cosas medianamente.
Antes de lo referido últimamente, yo veía a Presentacioncita todos los
días en casa de las señoras de Porreño; pero cuando estas descubrieron
la sutil urdimbre que mi travesura les preparara, concluyeron para mí
las entradas en la casa de la calle del Sacramento. Asistió
Presentacioncita a la ruidosa escena en que doña Paz y doña Salomé me
notificaron con encrespadas razones, no menos sonantes que las olas del
mar, su soberano desprecio, lo cual me causó pena, porque no era muy de
mi gusto pasar por un intrigante de mal género a los ojos de la dulce
niña de la condesa. Pocos días habían pasado después de la escena en la
Cámara regia que antes describí. Robáronme algún tiempo los amigos que
de Vitoria y la Puebla de Arganzón vinieron a solicitar mi ayuda para
distintas pretensiones, entre ellos el venerable patriarca D. Miguel de
Baraona, con su encantadora nieta (próxima a ser esposa de un joven
guerrillero), D. Blas Arriaga, capellán de las monjas de Santa Brígida
de Vitoria, y otros que más adelante serán conocidos; pero luego que me
dieron algún respiro, consagreme en cuerpo y alma a la adorable
Presentacioncita, en virtud de proyectos más o menos dulces,
recientemente concebidos; que en materia de proyectos, mi cabeza no
conocía el descanso, ni mi impetuosa voluntad el hastío.

Contra lo que yo esperaba, la señora condesa de Rumblar no me cerró las
puertas de su casa, ni aun decoró su estatuario semblante, cual solía,
con el grandioso ceño, y los agridulces mohínes propios de tan alta
señora. Verdad es que yo, además de entregarle la bandolera para su
hijo, haciéndole comprender que sin mí nada le habría valido la
recomendación de Ximénez de Azofra, le había prometido mi eficaz amparo
en el pleito que desde 1811 sostenía contra los Leivas. Tampoco
Presentacioncita se mostró ceñuda, a pesar de su adhesión a la familia
de Porreño; pero no lo extrañé, porque siendo yo el libertador de
Gasparito, bien merecía perdón; y el novio suelto no debía valer menos
que las amigas arruinadas.

Todo mi afán consistía en disponer de lugar y hora a propósito para
hablarle largamente a solas, apretándome a ello el deseo de comunicarle
cosas de la mayor importancia. Sin esperanza de que me concediera tal
gracia, pero decidido a todo, propúsele la conferencia, y ¿cuál sería mi
sorpresa al ver que aceptaba y que bondadosamente prometía señalar sitio
y momento oportuno, de tal suerte que la vigilancia materna no nos
estorbase? Yo estaba absorto: indudablemente habíase verificado en su
carácter cierta mudanza radical, porque la dichosa niña ponía en todos
sus actos y palabras mucha seriedad, cesando de mortificarme con las
burlas y epigramas de antaño.

Discurrió ella el modo de que a solas la hablase, y fue por un arte
ingenioso, tomando el traje de cierta muchacha que entonces la servía, y
poniéndose de noche a una reja, donde la doncella acostumbraba
conferenciar con cierto dragón de Farnesio.

No se me olvidará jamás aquella noche en que tuve la dicha de respirar
el dulce aliento de la adorable niña, tan de cerca, que el calor de su
rostro aumentaba el del mío, mareándome. ¡Y cómo brillaban sus negras
pupilas en la oscuridad! Cada vez que aquel vivo rayo diminuto surcaba
el espacio comprendido entre nuestros semblantes, yo me ponía trémulo.
¡Qué linda, qué seductora estaba aquella noche! Su agraciado rostro se
magnificaba con la melancólica seriedad en que le envolvía como en un
velo misterioso. Estaba descolorida, desvelada, y así como no había
frescos colores en su rostro, tampoco había en su alma aquella plácida
felicidad risueña que en época anterior irradiaba de ella, como del
astro la luz, haciendo felices también a cuantos la rodeaban. Pálida y
meditabunda ahora, parecía ocupada de pensamientos extraños.

Yo también lo estaba\ldots{} ¡ay!, yo estaba intranquilo, demente; yo no
dormía, yo no tenía paz en el corazón, porque me agitaba un ansioso
afán, un proyecto de inmensa gravedad que absorbía las potencias todas
de mi alma incansable e insaciable.

\hypertarget{xxiv}{%
\chapter{XXIV}\label{xxiv}}

Llegó al fin la hora de la cita.

---¡Qué miedo tengo Sr.~de Pipaón!---dijo cuando cambiamos los primeros
saludos,---¡qué miedo tengo, a pesar de las precauciones tomadas! No es
fácil que mamá ni mi hermano me descubran; pero sí Gaspar, que por las
noches ronda la casa, no contento con vigilarme de día, imponiéndome su
voluntad hasta en los actos más insignificantes\ldots{}

Después de tranquilizarla sobre este particular, le dije:

---Encantadora niña, ¡cuán mal sienta a esa incomparable persona, digna
de un emperador, afanarse por un mozalbete sin fundamento, como
Gasparito Grijalva! Mal empleados ojos puestos en él, mal empleada boca
hablándole, y mal empleado corazón amándole. Presentacioncita, Vd. no se
ha mirado al espejo, Vd. no conoce su mérito, Vd. no ha sabido apreciar
el inmenso valor de su propia persona, la cual es de tanta valía, que
casi casi no conozco ningún hombre digno de poseerla.

---¡Qué adulador es Vd.!---replicó sonriendo vagamente.---¿Es eso lo que
tenía que decirme?

---Por ahí empiezo, niña mía; empiezo por pasmarme de que quiera Vd. al
hijo de don Alonso, habiendo en el mundo tanto bueno\ldots{}

---Puesto que he venido aquí a hablar a usted con franqueza---dijo
interrumpiéndome,---no le ocultaré que Gasparito no me interesa ya gran
cosa.

---¡Oh, confesión admirable!---exclamé con gozo.---Mire Vd\ldots{} me lo
figuraba. Si no podía ser de otra manera. Si esos ojos fueran nacidos
para mirar a Gasparito, merecerían cegar. Digan lo que quieran, no se
hizo el sol para los insectos.

---Yo no sé lo que ha pasado en mí---prosiguió,---pero de la mañana a la
noche se me ha concluido la afición que a Gasparito tenía. Esto parece
raro, pero no lo es, porque a muchas ha ocurrido lo mismo.

---Es que algunas chiquillas toman por amor lo que no lo es; y cuando
viene la pasión verdadera, se asombran de haber derramado aquellas
primeras frías lagrimitas por un objeto indigno.

---Yo creí estar apasionada de Gaspar ¡cosas de chiquillas! Cuando una
juega con sus muñecas cree amarlas mucho, y después se ríe de ellas.

---¡Admirable idea!\ldots{} Gasparito es una muñeca, y para Vd. acabó de
repente la época de los juegos.

---Confieso que en un tiempo le quise\ldots{}

---¡Ah, en un tiempo!\ldots{} Luego\ldots{}

---Gaspar es un muchachuelo vulgar, un joven adocenado---dijo
expresándose con cierto desdén.---¡Parece mentira que yo le
amara!\ldots{} ¡Qué grande error!

---¡Enorme error!\ldots{} pero en fin, nada se ha perdido. Ahora bien:
¿puedo saber desde cuándo?\ldots{}

---¿Desde cuándo?---repitió en un tono que revelaba sin género de duda
cortedad de genio.

---Pero no me lo confiese Vd., niña---dije con viveza.---A ver si lo
adivino yo. ¿Apostamos a que lo adivino?

---¿Apostamos a que no?

---¡Ay! Presentacioncita, yo no carezco de perspicacia. Desde aquella
noche en que salimos de casa y tuvimos la malhadada aventura de la calle
del Bastero, y aquel descomunal susto, cuando me vi precisado a hacer
uso de las armas\ldots{}

---Que se quema, que se quema Vd.

---Sí, desde aquella noche, desde aquel encuentro con dos caballeros
desconocidos, cuando Vd. perdió el sentido y\ldots{} ¿acierto, mi señora
doña Presentacioncita? ¿Sí o no?

---Sí---repuso con voz que apenas se oía, más semejante a un suspiro que
a una voz.

Alzando los ojos contemplaba el cielo con tristeza.

---Pues bien---añadí lleno de entusiasmo,---los pensamientos de Vd. se
avienen perfectamente con lo que yo tenía que decirle. Nos entendemos.
¡Benditos corazones los nuestros que así concuerdan, respondiendo el uno
a los afanes del otro!

---Yo soy muy desgraciada, D. Juan---me dijo.---¿No conviene Vd. en que
soy muy desgraciada?

---Según y cómo---respondí,---según y cómo. Puede Vd. ser muy
desgraciada, pero muy desgraciada, y puede ser feliz, muy feliz,
felicísima.

---Lo primero es lo cierto.

---¡Ah, si Vd. supiera, si yo dijera aquí todo lo que sé!, ¡oh, arcángel
enviado por Dios a la tierra para consuelo de los tristes
mortales!\ldots{} Pero vamos por partes. ¿Se acuerda Vd. de la función
de los Trinitarios y de la recepción de Su Majestad en la sala capitular
del convento?

---¡Que si me acuerdo!---exclamó, cubriendo el rostro con sus manos y
descubriéndolo después más pálido, más bello, más interesante.---Ya que
se ha establecido entre nosotros cierta confianza, ya que he hecho
ciertas revelaciones que me han costado mucho, no ocultaré nada,
respetable amigo mío\ldots{} Aquel día la presencia de Su Majestad y el
reconocer en sus nobles facciones las mismas del generoso caballero que
me había amparado la noche anterior, produjeron general trastorno en mi
alma. Sentí primero una especie de terror. Yo no había visto nunca a Su
Majestad. La idea de haber estado tan cerca, de haber estado en los
mismos augustos brazos del Rey, de aquel gloriosísimo monarca, de aquel
hombre que casi no lo es, por su superioridad sobre los demás, me
conturbaba y confundía de tal manera, que no era dueña de mí misma.
Durante todo el día estuve atónita, paralizada, estupefacta. Parecíame
que resonaba su voz en mis oídos constantemente, y que no se apartaban
de mí aquellos negros ojos majestuosos, a los de ningún hombre
parecidos.

---¡Admirable concordia de sentimientos!---exclamé
interrumpiéndola.---¿Pero es Vd. una mujer o un serafín?

---Aquella noche no pude dormir. Estaba fascinada y no sabía apartarme
del retrato del Rey que mamá tiene en su cuarto haciendo juego con la
estampa del señor San José. En los siguientes días traté de vencer la
irresistible atracción que me llevaba violentísimamente a recrear mi
espíritu con los recuerdos de aquella noche y aquel día. Pero ¡ay!, mi
señor D. Juan. La noble, la gallarda, la incomparable imagen no se podía
apartar de mi imaginación. Cuando oía leer la Gaceta y pronunciaban
delante de mí el nombre del Rey; cuando Ostolaza le nombraba en la
tertulia para encomiarle hasta las nubes por sus buenas acciones, mi
rostro se encendía, parecía que iban a estallar mis venas todas y a
romperse en mil pedazos mi corazón.

---¡Oh!, lo creo, lo creo---dije con calor.---Su Majestad cautiva de ese
modo el ánimo de cuantos le miran. ¡Qué gallardía en su persona!, ¡qué
nobleza y grave hermosura en su semblante!, ¡qué caballerosidad e
hidalguía en sus modales!, ¡qué dulce música en su voz! No existe otro
más seductor en el conjunto de los hombres\ldots{} ¿Pues qué diré de sus
elevados pensamientos, de aquella bondad de corazón, de aquella
inteligencia suprema, para la cual no hay en el arte del gobierno
oscuridades ni enigmas? ¿Qué diré de su espíritu de justicia, del gran
amor que profesa a sus vasallos, de su religiosidad supina, de todas las
admirables prendas de su alma, las cuales son tantas, que parece mentira
haya puesto Dios en una sola pieza tal número de perfecciones? Vd. le
tratará más de cerca, Vd. le oirá, Vd. podrá conocer por sí misma que
las cualidades de ese angélico ser a quien Dios ha puesto al frente de
la infeliz España exceden con mucho a sus altas perfecciones físicas.

---La nariz es un poco grande---dijo Presentacioncita con una salida de
tono que me hizo estremecer,---pero no por eso deja de ser admirable el
conjunto del rostro.

---¡La nariz grande! Así la tuvieron Trajano, Federico el Grande, así
eran también la de Cicerón, la de Ovidio y tantos otros hombres
eminentes\ldots{} Pero esto no hace al caso. Lo que importa es que sepa
Vd. los sentimientos que ha despertado en aquel noble y generoso
corazón, no ocupado enteramente del amor a la patria y al sabio gobierno
absoluto. ¡Oh, mujer feliz entre las mujeres felices!---añadí con mucho
calor.---¡Oh, flor escogida entre las flores escogidas! ¡Oh, virgen
superior a todas las vírgenes!, puede Vd. vanagloriarse de ser la
primera que ha encendido una llama ardiente, pura, una llama\ldots{}

Presentacioncita se cubrió de nuevo el rostro con las manos. Entonces
pasó por mi mente las sospechas de que fuese yo en aquel momento víctima
de un bromazo tremendo. ¿Pero cómo era posible que el fingimiento de la
muchacha fuese tan magistral? No, ninguna actriz de la tierra, aunque se
llamase María Ladvenant o Rita Luna, era capaz de simular los
sentimientos con tal perfección, desfigurando el rostro, estudiando las
palabras, midiendo las actitudes, sin que ni un solo momento se
descuidase y revelara el pérfido artificio.

Observé a Presentacioncita con atención profunda, y cuanto más la
miraba, más me confirmaba en mi creencia de que cuanto veía y oía era la
realidad incontrovertible de una pasión verdadera. Mis últimas zozobras
se disiparon, cuando la vi alzar la frente y me mostró su rostro bañado
en lágrimas, de verdaderas lágrimas de ternura y dolor. ¡Oh, estaba
preciosa! Entre ahogados sollozos exclamó:

---Sr.~D. Juan, ¡por amor de Dios!, no me diga Vd. eso, no me lo diga
Vd. Es una falta de caridad jugar así con el corazón de esta
desgraciada.

Sus dulces lágrimas humedecieron mi mano. ¡Qué lástima que aquel rocío
celeste no fuera para mí! Me avergoncé de haber dudado un solo instante.

---¿No me cree Vd.?---dije.---Pues muy fácilmente puede convencerse de
mi veracidad. Yo le proporcionaré ocasión de que oiga Vd. misma de los
labios\ldots{}

---¡Oh!, eso no puede ser\ldots---afirmó con dignidad.

---No propongo nada contrario al honor---añadí.---Su Majestad creo que
daría la mitad de su corona por poder manifestar a Vd. los sentimientos
que le ha inspirado. Yo tengo el honor de ser amigo de Su Majestad, y me
ha confiado este deseo de su corazón\ldots{} ¿A qué conduce el negarle
tan dulce y legítimo consuelo, cuando él, por la misma sublimidad de su
amor, no aspira a nada que arroje sombra de mancilla sobre la adorada
persona de usted?

---¡Oh, qué disparates!---dijo con miedo.---No, esto no puede pasar de
aquí. Ni mi humilde condición con respecto a la suya me permite
acercarme a él con legítimo fin, ni mi honra me lo consiente de otro
modo. Es este un problema que no puede resolverse. No lo resolverá Su
Majestad con todo su poder, ni me deslumbrará el esplendor de su corona
hasta cegarme los ojos con que miro mi deber, la reputación de mi nombre
y mi casa. ¡Jamás! Oiga Vd. bien lo que digo. Jamás consentiré en ver ni
hablar a esa alta persona. Si he confesado lo que Vd. acaba de oír, lo
he hecho porque mi corazón necesitaba esta noble, esta leal expansión
con un cariñoso amigo que no puede venderme.

---Pero él\ldots{}

---Ni una sola palabra más sobre este asunto. ¡Qué necia he sido! ¿Por
qué no se me abrasó la lengua? Antes moriré cien veces que consentir en
ser recibida por su amigo de Vd. o en aceptar su visita. ¡Miserable de
mí! Me daría yo misma con mis propias manos la muerte, si me viese
cogida en una inicua celada por los cortesanos y aduladores de Su
Majestad.

---¿Usted ha podido creer que yo?\ldots---dije muy confundido.

---¿Por qué lo he de negar? Creo que a pesar de su honradez, el deseo de
servir a su señor le impulsa a abusar de mi confianza, de mi debilidad,
de esta franqueza quizás culpable con que le he hablado\ldots{} ¡Oh Dios
mío!, ¡cuán desgraciada soy!, ¡cuán desgraciada!

---Señora, yo juro que nada he pensado contrario al honor de Vd. y de su
hidalga familia. Pero no negaré que he creído posible y hasta
conveniente para la tranquilidad del mejor de los hombres y del más
virtuoso de los reyes, el preparar una entrevista amistosa\ldots{}

---¡Por Dios!, ¡por todos los santos!---exclamó con acento
dolorido.---Vd. ha tramado perderme; Vd. no es ni puede ser un hombre
leal. Pipaón, se acabó, ni una palabra más; retírese Vd. ¡Al momento, al
momento!

---Calma, calma. Lo decidiremos despacio y sin reñir, ni llamarme
desleal.

---¿Qué quiere Vd. decir con entrevistas amistosas?

---Una conferencia de amigos, una explicación\ldots{}

Quedose meditabunda largo rato, y yo pendiente de su contestación, con
el alma en los oídos.

---Bien, lo pensaré. Deme Vd. esta noche para pensarlo.

---¿Y mañana recibiré la contestación?

---Sí, mañana en este mismo sitio y a la misma hora.

Cuando esto decía, sentí un rumor extraño en el interior de la casa.

---Mi hermano viene---dijo con zozobra.---Retírese Vd. al momento, al
momento, y apriete Vd. el paso. ¡Oh! Ha sido una suerte que Gasparito
esté malo y no pueda salir de noche.

---Dios le conserve el mal\ldots{} Conque hasta mañana, ¿eh? Adiós, niña
mía.

Cerró la reja y me retiré a mi casa. Yo también necesitaba meditar.

\hypertarget{xxv}{%
\chapter{XXV}\label{xxv}}

Al día siguiente oí a doña María quejarse de la profunda distracción de
Presentacioncita, de sus nerviosidades y palideces, del trastorno muy
visible que en sus maneras y lenguaje se había verificado, lo que acabó
de confirmar mi creencia respecto a la veracidad de la niña en las
confianzas que me hiciera. Llegada la noche, acudí a la segunda cita y
pareciome que se habían agravado en la hermosa muchacha los síntomas de
exaltada y febril pasión.

---¡Cuánto ha tardado Vd., D. Juan!---me dijo reconviniéndome.

---He venido a la hora marcada, incomparable niña---repuse.---Si Vd. se
ha anticipado, no me acuse de tardío. Y ¿qué tal? ¿Se ha meditado mucho?
¿Cómo está esa preciosa cabeza? ¿Se ha serenado, se ha aclarado ese
entendimiento?

---He pensado mucho en ello, Sr.~D. Juan---exclamó con abatimiento,---y
mi mal no tiene remedio.

---¡Que no tiene remedio! Eso lo veremos más adelante. Pero por de
pronto, dígame Vd. su parecer acerca de la entrevista amistosa.

Contestome con hondo suspiro.

---La entrevista amistosa serviría tan sólo para aumentar mi desgracia.
Déjeme Vd., Pipaón, déjeme Vd. Ni su amistad me sirve de nada ni quizás
la merezco tampoco\ldots{} me moriré sola.

---Seamos razonables, adorada niña---dije alargando una mano por entre
los hierros de la reja.---Aquella persona a quien he dado esperanzas de
obtener algunos castos favores, está loca de alegría. Hoy no ha habido
despacho, y España y sus Indias andarán desgobernadas, mientras aquel
desatentado corazón no se tranquilice.

---¿Y si yo consintiera en la entrevista?---preguntó con afán.

---Entonces pronto se conocería en el risueño aspecto del reino y en la
marcha rapidísima de los expedientes, que el trono había recobrado su
asiento.

---¿Pues qué---preguntó con incertidumbre,---el trono es capaz de
desquiciarse por mí?

---Presentacioncita, es máxima de la antigüedad, que los reyes
contrariados en sus amores no gobiernan bien a los pueblos.

---¡Ay! Pipaón, cada vez me inspira usted menos confianza---dijo
ella.---Se me figura que mientras yo manifiesto mis sentimientos más
escondidos con tanta sinceridad y tanta nobleza, Vd. fingiendo interés
por mí, trata de engañarme, de perderme alevosamente, por servir a un
caprichoso amigo.

---¡Yo falso, yo alevoso, yo traidor!---exclamé con mucho brío.---Dar
tales nombres a quien es la lealtad en persona\ldots{} a quien daría
gustoso su vida por el prójimo, por Vd., Presentacioncita de mi alma.
Por Dios, no me estime Vd. en menos de lo que valgo.

---No; Vd. no es sincero; Vd. oculta mucho sus pensamientos---dijo en
tonillo quejumbroso.---Lo que ha hecho Vd. con las señoras de Porreño,
mis queridas amigas, prueba su mucho arte para el disimulo.

---¿Pues qué he hecho yo con esas dignas señoras?---interrogué,
maldiciendo interiormente aquel pícaro sesgo que había tomado nuestro
coloquio.

---¡Y lo pregunta!\ldots{} Vd. las entretuvo con promesas, mientras
consumaba su ruina; usted compró los créditos de D. Alonso de Grijalva
con la libertad de Gasparito, y después\ldots{}

---Basta, basta---exclamé con indignación.---Esos hechos no pueden
juzgarse en dos palabras. Si yo diera a Vd. explicaciones, ¡cuán
distinta sería su opinión acerca de esas supuestas maldades!

---No, si no digo yo que sean maldades. El hombre debe mirar por sí
antes que por los demás. Nada malo hay en procurar uno su propio bien,
aunque sea a costa ajena. Lo que digo es que Vd. sabe fingir muy bien;
lo que digo es que Vd. me está engañando.

---¡Oh! Santa Virgen de los Dolores, Señora y patrona mía. ¿Cómo
convenceré a esta pícara de mi sinceridad, de mi buena fe?---dije con
vehemencia.---Yo juro que nada he pensado que pueda ser contrario a la
perfecta felicidad de usted, a su virtud esclarecida, al interés de su
noble familia.

Y era verdad lo que pensaba. ¿Qué hacía yo sino proporcionar a la
abatida familia de Rumblar fabulosos adelantamientos y repentina
prosperidad? Interesado vivamente por el bien del reino en general y de
cada español en particular, yo me constituía en protector de una
familia, harto necesitada de una buena mano que la ayudase a salir del
atolladero de sus deudas y del pantano de sus inacabables pleitos.

---Y si no cree Vd. mis palabras---exclamé resueltamente,---a los hechos
me atengo. Ya he ofrecido a Vd. el medio de cerciorarse por sí misma, y
no digo más.

---Acepto---dijo con viva energía, golpeando con el puño el antepecho de
la ventanilla.---Acepto la entrevista amistosa. ¡Que Dios tenga piedad
de mí!

---¡Oh, mujer feliz entre todas las mujeres felices de la tierra! En
vuestra grandeza, señora mía, no olvidéis de hacer algo por este humilde
servidor de Vuestra Majestad.

Al decir esto, me descubrí respetuosamente ante ella. Presentacioncita
rompió a reír con vanidosa expresión.

---¡Yo Majestad!---exclamó.---Vamos, que pierdo el tino; ¡que lo pierdo
sin remedio!

---Otras cosas hay más imposibles.

---No desvariemos, Pipaón. Sería locura pensar que he de salir de mi
estado y condición actual. ¡Jesús!\ldots{}

---Monaguillo te vean mis ojos, que obispo\ldots{}

---No, no hay que pensar en tales imposibilidades\ldots{} posibles, pero
que yo rechazo desde ahora. Lo que digo es que si por acaso me levantase
yo dos dedos más arriba de donde estoy ahora, emplearía mi valimiento en
hacer todo el bien posible.

---¡Admirable corazón!\ldots---dije con fingido entusiasmo.---Permítame
Vd. señora, que salude en Vd. al iris de paz de la hispana monarquía.
¡Oh, señora!, ¡oh, excelsa joven!, ¡cuánto siento no estar en sitio
donde pueda prosternarme!\ldots{}

---¡Se va Vd. a poner de rodillas!---dijo riendo.---No tanto, Sr.~D.
Juan. Sólo decía que en caso de tener algún poder\ldots{}

---¡Algún poder!\ldots{} Inmenso poderío tendrá usted\ldots{} ¡Oh,
señora, no se olvide Vd. de los desgraciados, de los menesterosos, de
los pobrecitos!, ¡ay!, de los pobrecitos huérfanos sobre todo.

---Sobre todo de los infelices que gimen en las cárceles y en los
presidios por opiniones políticas.

---También, también, ¿por qué no? Apiádese usted de todo bicho viviente.

---Nada me contrista tanto---añadió con gravedad,---como oír hablar de
esas crueles comisiones militares, de esas persecuciones horrendas. ¡Oh!
¡Qué dulce será conseguir el perdón de los desgraciados para quienes se
ha levantado la horca! ¡Qué inefable dicha correr en busca de la
afligida madre, de la esposa, de la inocente hija, para decirles: «por
intercesión mía tenéis padre, tenéis marido, tenéis hijo»! ¡Abrir las
puertas de la patria a los proscriptos, arrancar la vil soga de manos
del verdugo, aplacar la ira de los furibundos jueces, derramar el
bálsamo de la caridad en el irritado y endurecido corazón del mejor de
los reyes!\ldots{} ¡Oh, qué hermoso papel! ¡Dios mío, mátame, o déjame
hacer ese papel!

A esta exaltación sublime siguió en la sensible muchacha un abatimiento
profundo. Yo la contemplaba, diciendo para mí:

---Tan atroz es su pasión, que poco le falta para estar rematadamente
loca.

---¡Qué sueños!---murmuró de un modo patético pasando la mano por su
abrasada frente.---¡Qué disparates he dicho, Pipaón!\ldots{} Pero mi
desvarío es disculpable, ¿no es verdad? ¿Quién no pierde la vista
hallándose tan cerca del sol?, ¿quién al sentir en su rostro el calor
que irradia aquel centro de luz y de poder, de grandeza y munificencia,
no se trastorna y marea?\ldots{} Yo no sé lo que pienso, yo estoy
absorta. Me parece que estoy amando a una sombra regia, a una figura
magnífica y arrebatadora que para seducirme ha brotado de las estampas
de un libro de historia. ¡Son tan altos los reyes! Feliz el gusano
miserable que cae bajo su augusto pie. Honran hasta aquello que
aplastan\ldots{} Mi destino está ya decidido. No puedo
contenerme---añadió con brío.---Adelante; Dios estará conmigo, puesto
que está con él, como decía La Atalaya. ¿No es el hijo predilecto de
Dios? ¿No le ha puesto Dios en el trono? ¿No emanan sus acciones todas
de inspiración divina? ¿No están de antemano aprobados todos sus actos
por el Eterno Padre? Adelante. Cúmplase mi destino y la voluntad de
Dios.

No era ocasión de perder el tiempo en vanas retóricas. Deseando
concluir, le dije:

---Su Majestad va casi todas las tardes a la Casa de Campo.

---¿Al otro lado del Manzanares?\ldots{} No he estado nunca
allí---repuso en tono pueril.---Dicen que es muy bonito. Hay jardines
preciosos y un lago\ldots{} todo de agua.

---Todo de agua, exactamente. Es un lugar delicioso. Iremos allá los
dos.

---Bueno. Pasearemos primero por entre los árboles.

---Y nos embarcaremos en los botes del lago.

---¡Oh! ¡En los botes del lago! ¡Qué delicia! Pero ¡ay!---exclamó con
pena,---ocurre una dificultad grande.

---¿Cuál?

---Gasparito\ldots{}

---Al diantre con Gasparito.

---No es esa la principal dificultad. Por la mañana le encargaré una
comisión cualquiera, y cuando venga a darme la respuesta, ya habré
salido yo.

---¡Admirable idea!

---Pero mamá no me dejará salir sola de casa. Forzosamente me ha de
acompañar mi hermano.

---¡El Sr.~D. Diego!---exclamé meditabundo, considerando que el heredero
de aquella noble casa no pecaba de sabio.

---No puede ser de otra manera. Mi hermano ha de ir conmigo, pero bien
sabe Vd. que aunque se ha corregido mucho, es bastante aturdido---dijo
con malicia.

---Me ocurre una idea---repuse, encontrando solución a aquella
contrariedad.---No importa que el Sr.~D. Diego nos acompañe hasta la
posesión regia. Entraremos los tres: nos pasearemos por espacio de una
hora u hora y media; luego se le hace salir con cualquier pretexto.

---Y volverá a entrar.

---No; de que no vuelva a entrar me encargo yo.

---¡Cómo resuelve Vd. todas las dificultades!\ldots{} Por mi parte yo
procuraré catequizar desde esta noche a mi señor hermano, que ahora está
muy fino y complaciente conmigo. Le diré que Vd. nos ha convidado para
pasear por la Casa de Campo sin que lo sepa mamá; que Vd. conoce al
administrador, el cual nos permitirá divertirnos mucho, correr por todos
lados, hacer lo que queramos, como si la posesión fuese nuestra.

---Y cazar y pescar. Prométale Vd. lo que quiera. Haremos locuras para
que nadie sospeche. Cuando llegue la ocasión en que su presencia nos
estorbe, Vd. dirá que se le ha olvidado cualquier cosa, que desea una
fruslería, por ejemplo\ldots{}

---Caramelos.

---No hay tal cosa por aquellos alrededores; pero se pueden
pedir\ldots{}

---Anises.

---En los puestos del río los hay. Vd. manda a su hermano que le traiga
anises, ¿eh? Él sale\ldots{}

---Y no vuelve a entrar\ldots{}

---Es Vd. el mismo demonio. En fin, estoy decidida. Que no me abandone
Dios es lo que deseo.

Después estremeciéndose de súbito, lanzó un suspiro y con voz conmovida
me dijo:

---¡Qué paso tan arriesgado voy a dar, y qué falta tan enorme voy a
cometer!\ldots{} Aunque ningún pensamiento impuro me arrastra, yo sé que
esto es una falta, una culpa que Dios no me perdonará\ldots{} ¡no,
Pipaón, no me la perdonará Dios!

---¡Oh!, siempre fue escrupulosa la inocencia---exclamé con
zalamería.---¡Angelical criatura! Si a mí me fuera concedido una mínima
parte de la celestial gracia de Vd\ldots{} ¡Pecado, culpabilidad,
impureza! ¿A qué pronunciar estas palabras quien por su condición
seráfica está libre del contacto del mal? Écheme usted la bendición y me
creeré bueno.

Lejos de calmarse con mis afectadas razones, afligiose más. Vi que
rodaban por sus mejillas abundantes lágrimas y que cruzando las manos,
alzaba al cielo los ojos.

---¡Dios mío, perdóname!\ldots{} ¡Madre mía, familia mía, abuelos y
ascendientes míos, perdonadme!---murmuró sordamente.

Satisfecho yo también de la madurez de su pasión, le dije mil cosillas
consoladoras, estrechando sus manos entre las mías. Ella inclinó la
frente, y sentí el vivo calor de ella, así como la humedad de su llanto
en mi mano.

---Pipaón---dijo con ansiedad,---júreme usted que no dirá esto a nadie;
que todo quedará en profundo misterio; júreme Vd. que no me despreciará
si por acaso\ldots{} júreme Vd. que sus propósitos son buenos, sus
intenciones leales\ldots{}

Yo juré cuanto ella quiso que jurase.

---Es tarde---dije al fin.---Retirémonos. Júreme Vd. que no faltará
mañana a la cita.

---¿Lo duda Vd.? A las dos, ¿no es eso?

---A las dos. ¡Ay!, ¡qué doloroso, qué horrible es desear y temer al
mismo tiempo!

---Esperaré en la Cuesta de la Vega con un coche simón, téngalo Vd.
presente, con un coche simón.

---Iré con mi hermano.

---Sólo con su hermano.

---No hay que hablar más. Adiós. Hasta mañana.

\hypertarget{xxvi}{%
\chapter{XXVI}\label{xxvi}}

En la mañana del siguiente día no dejé de visitar a D. S\ldots{}
S\ldots, uno de los funcionarios más respetables, más insignes de
aquella preclara monarquía. Desempeñaba el cargo dificilísimo de
administrador de la Casa de Campo tan a gusto de Su Majestad, que no le
cambiara éste por uno de sus mejores ministros. No le nombraré más que
por sus iniciales, con cuya delicada reserva evitaré que salgan ahora a
reclamar la gloria de su descendencia algunos de esos holgazanes que
faltos de virtudes propias, se gallardean y ufanan con las de sus
mayores. D. S\ldots{} S\ldots{} no había salido de ninguna Universidad,
sino de las cocinas de palacio, en cuyas humildes aulas consiguió
prestar al entonces Príncipe de Asturias repetidos servicios,
denunciándole supuestos envenenamientos en algunos platos. Por estos
escalones llegó D. S\ldots{} S\ldots{} a subir tan alto, que después de
1814 era hombre que no se cambiaría por Pedro Collado ni por el duque de
Alagón.

Desempeñaba sus funciones este sujeto con solicitud admirable. Se le
veía en todos los sitios públicos, y con frecuencia en el interior de
los teatros, donde nunca faltaba alguna cómica o bailarina a quien
tuviese que dar un recadillo. Había que verle en la Casa de Campo a
ciertas horas y en ciertos días, dando pruebas de tan consumada
prudencia y discreción y talento que no se podía pedir más. Yo me
honraba con su amistad, y cuando le anuncié mi visita a la Real posesión
acompañado de una madamita, alegrose en extremo, y se extendió en largas
disertaciones acerca de las dificultades de su cargo, prometiéndome al
fin que nos recibiría espléndidamente. Eso sí: a obsequioso y amable le
ganaban pocos.

\noindent{\dotfill}

A las dos de la tarde estaba ya en la Cuesta de la Vega, muy acicalado y
vestido con las finísimas ropas que por aquellos días me había hecho y a
poco se me apareció Presentacioncita. ¡Válgame Dios, qué linda estaba! A
sus encantos naturales, duplicados por la dulce emoción que teñía de
suave rosicler su rostro, unía el más elegante y gracioso atavío que la
fecunda inventiva de una mujer enamorada puede idear. ¡Cómo lucían
aquellos incendiarios ojos, que a cada movimiento de sus pupilas dejaban
entrever llamaradas del cielo! ¡Qué sonrisa tan deliciosa la de sus
rojos labios!, ¡qué gracia en el abanico!, ¡qué caídas las de la
mantilla!, ¡qué deslumbradora claridad, qué irradiación de hermosura
desde la peineta hasta las puntas de los diminutos pies! Yo estaba
trastornado de admiración.

Acompañábala D. Diego, no tan risueño y aturdido como de costumbre, sino
por el contrario con ciertas pretensiones de gravedad que no me hicieron
gracia\ldots{} ¿Sospecharía? Yo le hablé de la gira campestre que íbamos
a emprender, de lo mucho que nos divertiríamos en la regia posesión, y
añadí que lo mejor hubiera sido decir claramente a la señora condesa el
empleo higiénico que íbamos a dar al día.

---Entonces no nos hubiera dejado venir---repuso, entrando en el
simón.---Más vale así.

---Aprisa, aprisa---dijo Presentación con impaciencia.---A ese cochero
que eche a andar y que no pare hasta la Casa de Campo. Temo que
Gasparito descubra a dónde vamos. Desde esta mañana anda rondando la
casa.

El coche partió. D. Diego recobraba poco a poco su habitual volubilidad
y me hacía mil preguntas diversas relativas a la pesca del lago, a la
caza de Cantarranas, a las embarcaciones de los infantes y otras
menudencias. Doña Presentacioncita no hablaba nada. Yo no cesaba de
contemplarla. ¡Qué expresión tan extraña tenían su rostro y sus ojos no
menos picarescos que apasionados! Sin duda había en toda ella la
expresión, el aire, el indefinible aspecto del justo que se dispone a
ser pecador.

En medio de la confianza que me inspiraba la niña, tenía yo cierta
sospecha vaga, que aun después de verme en el camino del triunfo, se
removía vagamente en el fondo de mi espíritu. A cada instante creía que
la encantadora muchacha iba a escaparse de mis manos, dejándome
burlado\ldots{} Pero cuando entramos en los jardines disipáronse mis
últimas inquietudes.

---Aquí dentro---dije para mí, inundado de secreto gozo,---no te me
escapas. ¡Victoria completa! Ahora, ángel celeste, aunque te
arrepintieras no tendrías salvación.

Yo estaba como el general que acaba de ganar una batalla.

Abandonando el coche, avanzamos por las hermosas alamedas de aquel ameno
sitio. Don Diego, despabilándose con la hermosura de lo que veía,
charlaba por los tres. No había acabado de entrar y ya quería cazar
todas las aves, pescar todos los peces y modificar a su antojo la
posesión. Tal alameda no debía estar como la plantaron sus fundadores,
sino de otra manera: tales árboles debían ser arrancados y sustituidos
por otros: en determinado sitio debía construirse un edificio, un
pabellón\ldots{} en fin, para aquel impetuoso joven nada debía ser como
era.

Presentacioncita se extasiaba en la contemplación del hermoso lago, que
es principal adorno y riqueza de la hermosa finca. Después de observar
largo rato el risueño espectáculo que ofrece la enorme masa de agua
rodeada de amena verdura y corpulentos árboles, me dijo:

---Paseemos un poquito por el charco.

---Voy un instante a ver al administrador---le dije en voz baja,
mientras D. Diego se dirigía a los botes.---Pronto vuelvo: no se olvide
Vd. de los anises.

---¿Nos dejarán embarcar, Pipaón?---me preguntó el conde.

---Voy a pedir licencia.

En cuatro palabras me puse de acuerdo con el respetable D. S\ldots{}
S\ldots{} acerca de los medios de plantar en la calle el estorbo que por
necesidad habíamos traído. El conde saldría; pero antes que a entrar
volviera se convertirían en anises todas las piedras del cercano río.

Un momento después era desamarrado uno de los botes, y ocupándole D.
Diego que empuñaba resueltamente los remos, después de describir varias
curvas se acercó mansamente a la orilla.

---Entren Vds\ldots{} Presentación, adentro. Señor D. Juan, salte Vd.

Saltamos adentro y tomamos asiento en los bancos del bote. Era la
primera vez en mi vida que yo me embarcaba.

---¿Saben Vds.---dije a los dos jóvenes cuando habíamos avanzado como
cinco varas por el agua,---que este suave movimiento no me agrada? Se me
va la cabeza.

---¡Se le va la cabeza!---dijo Presentación.---¡Qué será de la
monarquía, si se le va una de sus principales cabezas!\ldots{}

La miré por ver si reía; pero estaba seria.

---¡Una de sus principales cabezas!---repitió D. Diego remando cada vez
con más fuerza.---Ahora me acuerdo de que no he dado a Vd. las
gracias\ldots{} ¡qué distraído soy!\ldots{} por la bandolera que me ha
conseguido.

---Eso no vale nada, amiguito. Vd. se merece más---dije con mucha
inquietud.---Hágame Vd. el favor de poner la proa a tierra\ldots{} Por
mi amigo el infante D. Antonio juro que el navegar es cosa imponente.

---¿Pero se marea Vd. aquí?\ldots{} ¡hombre de Dios! ¿Y no se avergüenza
Vd.?

---Un hombre de Estado, una eminencia---dijo Presentación,---una
lumbrera de España y del siglo, ¿perder su aplomo tan fácilmente?

---No me mareo, pero la verdad, esto no me gusta\ldots{} A la otra
orilla, que es tarde y tenemos que ver la pajarera.

---Otro poquito más---dijo la niña.---Me encanta este suave movimiento.
¡Qué hermosa es el agua!\ldots{} Mire Vd., mire Vd. los pescaditos.
¿Pues y esas yerbas verdes y negras que se ven debajo?\ldots{} Aquí
tienen ellos sus nidos, sus casas, sus alcobas, sus camas, sus
despensas\ldots{} Mire Vd. cómo van en bandadas por el agua, cómo se
juntan y se separan. Parece que se dicen un secreto, que se hacen
preguntas, que disputan y se reconcilian después. Y ¡cómo se ve el cielo
en el fondo!, parece otro cielo, ¿no es verdad, Pipaón? ¡Qué bien se ven
desde aquí los árboles de la orilla; se ven dos veces, unos vueltos
hacia arriba y otros hacia abajo! ¡Oh!, por allí vienen los cisnes. De
lejos parecen una escuadra navegando a toda vela. ¡Ay! Pipaón ¡qué
hermoso es esto!\ldots{} A ver si sé yo remar.

---¡Tonta! Tú no tienes fuerza---dijo D. Diego, defendiendo los remos.

---Señor conde, diríjase Vd. a la otra orilla---exclamé yo, empuñando el
timón, con no menos brío que un Sebastián Elcano.---La verdad es que
estas cáscaras de nuez no me inspiran gran confianza. Puede romperse una
tabla con la mayor facilidad, y aquí se ahoga uno sin remedio.

---Yo no, porque nado como un pez---dijo D. Diego.

---A tierra, a tierra.

---¿Que se ahoga uno? ¡Dios mío!---exclamó con espanto
Presentacioncita.---¿Si uno se cae aquí, se ahoga?

---Sin remedio.

Por más que ordenábamos al remero que nos llevara a tierra, se empeñaba
el tunante en dar vueltas y más vueltas alrededor del lago. Corría
velozmente la frágil embarcación, y la niña de la condesa parecía muy
complacida de aquel extraño modo de pasear, porque aspiraba con delicia
el aire que en nuestra carrera nos azotaba el rostro, y con sus
manecitas agitaba el agua, salpicándola, cual si también remase.

---Basta, basta ya. ¡A tierra!

---Está Vd. pálido, Pipaón---me dijo la niña, acercándose a mí con mucho
interés.

---Pálido no---repuse,---pero nos hemos paseado ya bastante por los
mares.

---¿Quiere Vd. un caramelo?---añadió registrándose los bolsillos.---¡Qué
diablura! Se me han olvidado.

---Habrá Vd. traído anises.

---Tampoco---añadió con mucho desconsuelo.---Mira, Diego, en cuanto
volvamos a la orilla, saldrás a comprarme unos anises. Verdaderamente,
no me puedo pasar sin anises.

---En los puestos del río los hay---indiqué yo.

Daba el bote una vuelta, cuando vi que un guarda con descompuestos
ademanes de ira nos hacía señas para que fuésemos a la orilla. Era un
ardid convenido con D. S\ldots{} S\ldots{} para poner término a la
excursión naval, si se prolongaba demasiado.

---¿Ven Vds.? El guarda nos hace señas de que salgamos del bote---grité,
fingiendo el mayor enfado.---¡Qué desacato hemos cometido! Nos van a
echar de la posesión.

---Vamos, vamos---dijo la niña.---Aquel buen hombre está muy enfadado.

Pero el conde seguía remando, y la nave su suave curso alrededor del
vasto charco. Disponíame yo a arrancar los remos de las manos del joven,
cuando divisé en la orilla de enfrente muchedumbre de hombres y
caballos.

Presentación se puso pálida.

---Buena la hemos hecho---exclamé, reconociendo los coches de la Casa
Real.---Ahí está Su Majestad\ldots{} Cuando menos nos mandan a la
cárcel.

---¡Jesús, qué miedo!---dijo la muchacha.---¿Dónde nos esconderemos?
Diego, tú tienes la culpa. Vamos a tierra pronto, hijito, o échanos a
pique, para que ocultemos nuestra vergüenza.

El muchacho reía con un desparpajo que me arrebató de cólera.

El guarda seguía haciendo señas. Tras el coche del Rey entraron otros, y
bien pronto vimos paseando por la orilla a Su Majestad en persona,
acompañado del duque y seguido de distintos individuos de su alta
servidumbre. Poco después aparecieron algunas damas. Don Dieguito remaba
suavemente hacia tierra.

De pronto observamos que el Rey y todos los que le acompañaban se
detenían a mirarnos. Estábamos sirviendo de espectáculo a la corte.

---¡Qué vergüenza!---dijo Presentacioncita.---¡Cómo nos miran!\ldots{}
Su Majestad se ha fijado en Vd., Pipaón. Parece que se sonríe.

En efecto, sonreía mirando el bote.

---Salude Vd. a Su Majestad, Pipaón, salude Vd., hombre---exclamó con
afán la niña.---¡Por Dios, no sea Vd. grosero!\ldots{} ¡Qué
poste!\ldots{} Pero hombre, levántese Vd.

Púseme en pie, sombrero en mano\ldots{} y en el mismo instante ¡Dios
Todopoderoso y Misericordioso!\ldots{} sentí unas pequeñas pero
enérgicas manos que se apoyaron en mi espalda\ldots{} recibí un impulso
terrible, del cual no pude defenderme, por estar desprevenido, y caí con
estrépito y como una piedra en el agua\ldots{} ¡Horror incomparable!

Cuando mi cuerpo chocó con la superficie del agua y esta salpicó con
estruendo y chasquido horrible y sumergime repentinamente, sentí un
rumor espantoso de carcajadas, y sobre mí la voz de Presentacioncita,
que con el ardor de la venganza, exclamaba:

---¡Por tunante!, ¡por cobarde!, ¡por pillo!, ¡por traidor!, ¡por
al\ldots!

La última palabra no la copio por respeto a mí mismo.

\noindent{\dotfill}

Yo nadaba como una peña. Fui derecho al fondo. Agua por todas partes,
agua en mis ojos, en mi boca, dentro de mi cuerpo, agua en mi aliento,
que ya no era aliento, sino el angustioso hálito de la asfixia. Tragaba
la muerte\ldots{} me moría por dentro y por fuera\ldots{} ¡me
ahogaba!\ldots{}

¡Ay! Cuando me sacaron, no sin trabajo, los guardas, ayudándose de
ganchos, mi persona inspiraba horror, según me han dicho. Yo era una
masa de fango pestilente. Los cortesanos huyeron de mí con asco,
mientras los guardas me envolvían en mantas, haciéndome los tratamientos
necesarios para volverme a la vida. Dentro de mi estómago tenía todo el
estanque, todo el Océano y hasta el bote.

Cuando adquirí la certeza de que aún vivía para bien de la humanidad y
amparo de los desvalidos, era ya de noche. Todo era silencio. Estaba en
una sala, y a mi lado no vi ni Rey ni cortesanos. Los guardas me miraban
y recordando el chasco, se reían.

Entonces, trayendo a la torpe memoria accidentes y pormenores, empecé a
caer en la cuenta de que Presentacioncita se había burlado de mí,
haciéndome una obra maestra de estudiada farsa, de disimulo, de pérfido
engaño. ¡Maldita sea mil veces! Recordando su comedia, su bien fingido
enamoramiento, sus coloquios conmigo, la habilidad suprema con que me
fue conduciendo poco a poco a la nefanda catástrofe, de acuerdo con su
hermano, con su novio y sus criados, me parecía mentira que todo fuese
una burla. Después he sabido que mi conducta con las señoras de Porreño
y el señor de Grijalva le inspiraron aquel plan de venganza, que llevó
adelante con su incontrastable voluntad y su agudísimo entendimiento. Me
aborrecía apasionadamente, me odiaba con exaltación; soñaba con la
venganza, y ningún ideal amoroso, ninguna fantasía de mujer hubiera
enloquecido su mente, como aquella ansia de burlarme de un modo cruel,
inaudito, no contentándose con el martirio de la ridiculez, sino
aspirando a daños mayores, a la muerte quizás\ldots{} Confesó la pícara
que nada se le importaba que me ahogase, pues un ser tan vil y
despreciable como Pipaón (así mismo lo afirmó) debía morir donde vivía,
es decir, en el lodo.

¡Hórrida, bella! Desde entonces, Presentación me causó espanto. Yo no me
parecía a Marat; pero ella tenía no poco de Carlota Corday.

---Pero después de tal infamia, ¿les dejaron marchar
tranquilos?---pregunté a D. S\ldots{} S\ldots{} que se me acercó para
informarse de mi estado.

---La muchacha reía---me dijo;---el joven remaba con mucha fuerza para
llegar a la otra orilla; pero por mucha prisa que se dio, ya les
aguardaban allá los guardas, dispuestos a hacer presa en ellos\ldots{}
Fueron, pues, cogidos ambos hermanos, porque son hermanos, ¿no es
verdad? La muchacha estaba serena, tan serena que parecía un ángel; y
cuando le afeamos su conducta, respondió que Vd. por trapisondista y
farsante\ldots{} (no sé cuántas insolencias salieron de aquella linda
boca), bien merecía el remojón delante de la corte, y aun la muerte.

---¿Y Su Majestad no dispuso\ldots?

---Su Majestad, cuando vio que mi señor D. Juan salía lleno de fango,
dijo sonriendo: «¿está vivo ese tunante?»

---¿Ese tunante?

---Así mismo. Luego añadió: «yerba ruin nunca muere», y fue hacia donde
estaban los dos criminales detenidos por los guardas.

---Sin duda iba a disponer un castigo tremendo\ldots{}

---Su Majestad reía de tan buena gana, que daba gusto verle. Todos nos
reíamos. De repente algunos señores de la corte que acababan de entrar
en la posesión se encontraron con Su Majestad en la senda que da vuelta
al lago. Detuviéronse todos: aquellos señores traían una grave noticia,
venida hoy por el correo de Francia, una noticia estupenda, horrible,
que dejó absorto y frío y pálido a Su Majestad, y mudos de espanto a
todos los que le rodeamos.

---¿Y esos dos muñecos?\ldots{}

---Su Majestad permaneció un rato mudo y quieto, como si se convirtiera
en estatua. Después dijo: «Vamos al instante a palacio»; y pusiéronse
todos en marcha.

---¿Y esos dos muñecos?\ldots{}

---Yo interrogué al Rey para saber lo que hacíamos con ellos y entonces
volvió a reír\ldots{}

---¡A reír!

---Y con mucha complacencia nos dijo: «que se les deje en libertad, y no
se les moleste por su travesura».

---¡Travesura! ¡Se escaparon! ¡La impunidad!\ldots{} ¿Y qué noticia es
esa\ldots?

---Que Napoleón ha vuelto de la isla de Elba.

\flushright{Madrid.—Octubre de 1875.}

~

\bigskip
\bigskip
\begin{center}
\textsc{Fin de las memorias de un cortesano de 1815}
\end{center}

\end{document}
